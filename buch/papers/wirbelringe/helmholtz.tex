%
% helmholtz.tex -- Geht auf die Helmholzschen Wirbelsätze ein
%
% !TEX root = ../../buch.tex
% !TEX encoding = UTF-8
%
\section{Helmholzsche Wirbelsätze}

\subsection{Historisches}

Um das Verhalten von Wirbelringen besser zu verstehen, sind die sogenannten helmholzschen Wirbelsätze sehr nützlich. 
Mitte 19. Jahrhundert formulierte der Deutsche Physiker Hermann von Helmholtz 3 Wirbelsätze und veröffentlichte diese im Journal für die reine und angewandte Mathematik\cite{Wirbelringe:JournalHelmholz}.
In dieser definiert er auch gleich auch Linien, um die Wirbelbewegungen, besser beschreiben zu können:

\subsubsection*{Wirbellinien}
\label{paper:Wirbelringe:Wirbellinien}

Wirbellinien sind die „Mittelachse“ eines Wirbels. 
Um diese Achse rotieren die Teile, welche teil eines Wirbels sind. 
Diese hat an sich kein Volumen, allerdings kann es sein das Teilchen auf dieser Achse zu liegen kommen. 
Diese Linie kann an einem echten Wirbel beobachtet werden. 
In der Praxis ist eine Wirbellinie nicht gerade, sondern gekrümmt oder sogar spiralenähnlich. 
Ein idealer Wirbelring besitzt eine Wirbellinie in der Form eines Kreises.
Des Weiteren ist eine sehr wichtige Eigenschaft, dass Wirbellinien nur auf einer Grenzfläche enden können. 
Siehe Kapitel \ref{paper:Wirbelringe:Grenzflaechen}

\subsubsection*{Wirbelfäden}
\label{paper:Wirbelringe:Wirbelfaden}

Ein Wirbelfaden ist ein Zylinder, welcher eine Wirbellinie als Zentrum des Zylinders hat. 
Schneidet man nun diesen Zylinder senkrecht zu der Wirbellinie, ergibt sich ein einzelner Wirbel. 
Wirbelfäden werden auch Wirbelröhren genannt.

\subsection{Erster Helmholzscher Wirbelsatz}

\begin{displayquote}
    In Abwesenheit von wirbel anfachenden äusseren Kräften bleiben wirbelfreie Strömungsgebiete wirbelfrei.
\end{displayquote}

Kurz gesagt, Teilchen die ruhen, bleiben in Ruhe. 
Siehe Abbildung \ref{buch:papers:Wirbelringe:fig:Helmholtz_1}. 
Alle Teilchen (in Blau) bewegen sich nicht da der Betrachtungsraum abgeschlossen ist und diese \textbf{nicht} teil eines Wirbelrings sind.

\subsection{Zweiter Helmholzscher Wirbelsatz}

\begin{displayquote}
    Fluidelemente, die auf einer Wirbellinie liegen, verbleiben auf dieser Wirbellinie.
\end{displayquote}

Dies gilt auch, wenn sich diese Wirbellinie fortbewegt. 
Allerdings heisst das nicht das Teilchen die sich nicht von Anfang an auf einer Wirbellinie befinden, dort nicht mehr hingelangen können  
Siehe Abbildung \ref{buch:papers:Wirbelringe:fig:Helmholtz_2}. 
Alle Teilchen (in Blau) bewegen sich nicht, da sich die Wirbellinie (relativ zum Betrachtungsraum) nicht bewegt.

\subsection{Dritter Helmholzscher Wirbelsatz}

\begin{displayquote}
    Die Zirkulation entlang einer Wirbelröhre ist konstant. 
\end{displayquote}

Dieser Wirbelsatz kann mit dem Integral 
\[
\Gamma
= 
\oint_{c} \vec{v} \cdot d \vec{l}
=
\text{const}
\]
zusammengefasst werden. 
Wobei \(\Gamma\) die Zirkulation des jeweiligen Flächenstücks der Gesamtwirbelröhre ist, c der Umfang des jeweiligen Flächenstücks und \(\vec{v}\) die Geschwindigkeit der rotierenden Partikel. 
Siehe Abbildung \ref{buch:papers:Wirbelringe:fig:Helmholtz_3}. 
\input{papers/wirbelringe/fig/Helmholz_wirbelsätze.tex}

\subsection{Stokes \label{paper:Wirbelringe:Stokes}}

\(\Gamma\) kann auch anders beschrieben werden. 
Ein Wirbelring hat eine Wirbelstärke $\omega$, welche die Divergenz der Lokalen Geschwindigkeit eines Fluidteilchens ist
\[
\omega
=
\nabla \times \vec{u}.
\]
Integriert man jetzt $\omega$ über eine Fläche $A$ so erhält man die gesamte Zirkulation $\Gamma$ des entstandenen Wirbelrings
\[
\Gamma
=
\iint_{A} \omega\cdot d \vec{A}.
\]
Im Gegensatz zu der Definition zuvor ist die Definition ein Integral über die Fläche und nicht über die Kontur des Wirbelrings. 
Todo : stokes \ref{buch:green:green:satz:stokes}

\subsection{Verhalten an Grenzflächen \label{paper:Wirbelringe:Grenzflaechen}}

Mit den vorhergegangenen Regel kann das Verhalten von Wirbelringen nun einfacher nachvollzogen werden. 
Allerdings können noch nicht alle verhalten begründet werden. 
Bisher wurde nur das Verhalten im Freien Raum betrachtet und Grenzflächen ignoriert. 
Allerdings sind diese wichtig, da in der Realität, immer solche Grenzflächen vorhanden sind. 
Grenzflächen entstehen immer dort wo es Übergänge von Materialien gibt oder am Rand dem Betrachtungsraum.
Bereits in der Einleitung ist ersichtlich, dass die Rotation eines Wirbelfelds erhalten bleibt. 
Daher können Teile in einem Wirbelring diese Grenzflächen nicht durchqueren. 
Allerdings können Wirbellinien darauf Enden da die Bedingung der Divergenzfreiheit implizit erfüllt wird. 
