\section{Einleitung} \label{poinbendix:section:einleitung}

Der Satz von Poincaré-Bendixson beschreibt mögliche Bahnkurven von zweidimensionalen dynamischen Systemen.
\index{Poincaré-Bendixson, Satz von}%
\index{dynamisches System}%
Unter einem dynamischen System versteht man die Bewegung durch ein Vektorfeld entlang der Vektoren.

Beschrieben wird dies durch ein Differentialgleichungssystem in den Koordinaten $x$ und $y$ der Form
\begin{equation*}
\frac{d}{dt}
\begin{pmatrix}x\\y\end{pmatrix}
=
f(x,y),
\end{equation*}
wobei $f(x,y)$ eine vektorwertige Funktion darstellt.
Im ganzen Kapitel wird immer implizit von Differentialgleichungssystemen dieser Art gesprochen.
Geschrieben werden sie aber in den Beispielen immer als zwei Differentialgleichungen
\begin{align*}
    \dot{x} &= f_x(x, y) \\
    \dot{y} &= f_y(x, y).
\end{align*}

Unabhängig von der Natur des zweidimensionalen dynamischen Systems schränkt der Satz von Poincaré-Bendixson die möglichen Bahnkurven auf drei Fälle ein.
Somit können Bahnkurven in zwei Dimensionen kein chaotisches oder unberechenbares Verhalten zeigen.
\index{chaotisch}%

Im ersten Abschnitt \ref{poinbendix:section:nullklinen} wird ein intuitives Verständnis über das Verhalten solcher Bahnkurven mithilfe der sogenannten Nullklinen aufgebaut.
Dies hilft uns den eigentlichen Satz in Abschnitt~\ref{poinbendix:section:poinbendix} besser zu verstehen.
Neben dem Satz von Poincaré-Bendixson finden sich in diesem Abschnitt auch einige Beispiele.
