%
% teil1.tex -- Beispiel-File für das Paper
%
% (c) 2020 Prof Dr Andreas Müller, Hochschule Rapperswil
%
% !TEX root = ../../buch.tex
% !TEX encoding = UTF-8
%
\section{Lokalität
\label{parallelisierung:sec:Lokalitaet}}
\kopfrechts{Lokalität}
Lokalität beschreibt den Umstand, dass ein physikalischer oder mathematischer Vorgang nur von seiner unmittelbaren Umgebung beeinflusst wird. 
\index{Lokalität}%
Dieses Konzept ist zentral für das Verständnis vieler physikalischer Prozesse und bildet zugleich die Grundlage für numerische Verfahren, mit denen wir Feldgleichungen berechnen können. 
Ohne die Annahme der Lokalität wäre es nicht möglich, komplexe Systeme in Teilprobleme zu zerlegen und effizient auf Rechnerarchitekturen umzusetzen.

\subsection{Lokalität im physikalischen Sinn
	\label{parallelisierung:sub:LokalitaetPhysik}}
Auf einfache Weise lässt sich Lokalität am Beispiel der Wärmeleitung veranschaulichen. 
\index{Wärmeleitung}%
Die Temperatur an einem bestimmten Ort hängt nicht unmittelbar von allen Punkten im Material ab, sondern nur von den direkt benachbarten Zellen. 
Der Wärmestrom ergibt sich lokal aus den Temperaturdifferenzen zwischen einem Punkt und seinen Nachbarn.
Eine heisse Stelle in einem Metallblock beeinflusst also nicht direkt eine weit entfernte Stelle, sondern nur die benachbarten Bereiche. 
Diese Nachbarstellen wiederum beeinflussen die nächsten, und so breitet sich Wärme schrittweise aus. 
Es gibt Beispiele von Fernwirkung in der Physik wie zum Beispiel das newtonsche Gravitationsgesetz oder Coulombs Gesetz für elektrische Ladung.
Diese sind für viele Anwendungen sehr nützlich, es sind aber nur vereinfachte Gesetze.
In der modernen Physik können jedoch alle physikalischen Prozesse durch Felder mit streng lokalen Wechselwirkungen erklärt werden.
Es gibt ausserhalb der Quantenmechanik keine echte Fernwirkung in der Physik.

\subsection{Lokalität im mathematischen Sinn
\label{parallelisierung:sub:LokalitaetMathematik}}
In der Mathematik tritt Lokalität vor allem bei Differentialoperatoren auf. 
Eine Ableitung an einer Stelle beschreibt das Verhalten einer Funktion in einer sehr kleinen Umgebung dieses Punktes. 
In der numerischen Mathematik, wo kontinuierliche Operatoren durch diskrete Approximationen ersetzt werden, wird dieses Prinzip besonders deutlich. 
So wird etwa eine erste Ableitung durch Differenzen zwischen benachbarten Gitterpunkten dargestellt, und eine zweite Ableitung durch Kombination dieser Differenzen. 
Damit greift die Berechnung stets nur auf Werte aus der unmittelbaren Nachbarschaft zu. 
Diese Eigenschaft macht Differentialoperatoren lokal und ermöglicht die Umsetzung auf Gittern.

\subsection{Lokalität und partielle Differentialgleichungen
\label{parallelisierung:sub:LokalitaetPDE}}
Für partielle Differentialgleichungen (PDEs) ist die Lokalität von zentraler Bedeutung. 
\index{partielle Differentialgleichung}
Die Veränderung einer Grösse in Raum und Zeit hängt immer nur von den lokalen Gradienten oder Krümmungen ab. 
Würde ein Operator auf das gesamte Feld wirken, könnte man die Gleichung nicht sinnvoll numerisch lösen. 
Zudem ist Lokalität entscheidend für die Parallelisierung, denn nur da jedes Teilgebiet seine Lösung ausschliesslich aus lokalen Informationen berechnen kann, lassen sich deren Berechnungen auf mehrere Prozessoren verteilen. 
Lediglich an den Grenzen der Teilgebiete muss kommuniziert werden, um den Austausch der Randzellen sicherzustellen. 
Somit bildet Lokalität nicht nur die theoretische Grundlage für PDEs, sondern auch die praktische Voraussetzung für deren effiziente Berechnung auf modernen Rechnerarchitekturen.

