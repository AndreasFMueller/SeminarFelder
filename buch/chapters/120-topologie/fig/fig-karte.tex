%
% fig-karte.tex
%
% (c) 2025 Prof Dr Andreas Müller
%
\begin{figure}
\centering
\includegraphics[width=\textwidth]{chapters/120-topologie/images/karte.png}
\caption{Die Höhenlinien sind die Niveaulinien der auf der Erdoberfläche
\index{Höhenlinien}%
\index{Niveaulinien}%
definierten Funktion, die die Höhe eines Punktes angibt.
Die Niveaulinien zerlegen die Oberfläche der Erde in Streifen oder Ringe.
Die Idee der Morse-Theorie ist, dass sich die Topologie der Erdoberfläche
aus diesen Elementen rekonstruieren lässt.
\label{buch:topologie:morse:fig:karte}}
\end{figure}
