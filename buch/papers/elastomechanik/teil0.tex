%
% einleitung.tex -- Beispiel-File für die Einleitung
%
% (c) 2020 Prof Dr Andreas Müller, Hochschule Rapperswil
%
% !TEX root = ../../buch.tex
% !TEX encoding = UTF-8
%
\section{Einleitung}
\label{elastomechanik:section:Einleitung}
Wer schon einmal ein Stück Knetmasse in der Hand gedrückt hat, hat dabei ganz praktisch erlebt, wie sich Materialien unterschiedlich verhalten können. 
Während die Knete seitlich ausweicht wenn man sie zusammendrückt gibt es andere Materialien die sich nur schwer verformen lassen und dabei ihr Volumen nahezu beibehalten. 
Willkommen in der faszinierenden Welt der Elastizitätstheorie!
In diesem Kapitel der Seminararbeit über Felder beschäftigen wir uns mit der mathematischen Beschreibung elastischer Materialien.
Ziel ist es Deformationen also Formänderungen infolge äusserer Belastungen zu berechnen und zu verstehen wie sich Körper unter Krafteinwirkung verhalten.
Ein zentrales Modell zur Beschreibung dieses Verhaltens ist das sogenannte Hookesche Gesetz.
Es stellt in seiner einfachsten Form einen linearen Zusammenhang zwischen mechanischer Spannung $\sigma$ Kraft pro Fläche und Dehnung $\varepsilon$ relative Längenänderung her:
	\begin{equation}
		\sigma = 
		E \cdot \varepsilon
	\end{equation}
Der Proportionalitätsfaktor $E$ ist der Elastizitätsmodul eines Materials.
Er beschreibt wie steif ein Material ist je grösser $E$ desto weniger lässt es sich dehnen oder stauchen.
Für eine vollständige Beschreibung des Materialverhaltens insbesondere im dreidimensionalen Raum genügt diese einfache Beziehung jedoch nicht.
In der Regel benötigt man zwei Materialkonstanten um das mechanische Verhalten realistisch zu modellieren.
Hier lohnt sich ein Blick auf zwei extreme Beispiele:
\begin{description}
\item[\textbf{Inkompressible Materialien:}] Wie Gummi behalten ihr Volumen unter Belastung nahezu gleich.
Wird Druck ausgeübt weichen sie zur Seite aus.
Dies lässt sich physikalisch durch eine hohe Poissonzahl beschreiben.
\item[\textbf{Kompressible Materialien:}]  Wie Schaumstoff reagieren hingegen stark auf Volumenänderung.
Sie lassen sich leicht zusammendrücken wobei sich ihr Volumen deutlich reduziert.
Dies lässt sich physikalisch durch das sogenannte Kompressionsmodul beschreiben.
\end{description}

Solche Beobachtungen aus dem Alltag lassen sich mithilfe der Mathematik analysieren.
Im Verlauf dieser Arbeit werden wir dazu Felder einführen die die räumliche Verteilung von Kräften Spannungen und Verformungen beschreiben.
Auf dieser Grundlage entwickeln wir mathematische Modelle insbesondere in Form partieller Differentialgleichungen die das elastische Verhalten von Körpern präzise abbilden.
Mathematik trifft hier auf physikalische Intuition ein spannendes Zusammenspiel das nicht nur theoretisch herausfordernd sondern auch praktisch äusserst relevant ist.