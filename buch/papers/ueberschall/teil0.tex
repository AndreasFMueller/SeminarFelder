%
% einleitung.tex -- Beispiel-File für die Einleitung
%
% (c) 2020 Prof Dr Andreas Müller, Hochschule Rapperswil
%
% !TEX root = ../../buch.tex
% !TEX encoding = UTF-8
%
\section{Einleitung\label{ueberschall:Einleitung}}
\kopfrechts{Einleitung}
\begin{align}
    v_x = \frac{\partial \phi}{\partial x}\\
    \intertext{}
    v_y = \frac{\partial \phi}{\partial y}\\
    \intertext{}
    v_i = \frac{\partial\phi}{\partial x_i}\\
    \intertext{}
    \frac{\partial (\rho v_x)}{\partial x} + \frac{\partial (\rho v_y)}{\partial y} = \mathrm{div}\,\rho \vec{v} = 0\\
    \intertext{}
    v_x \frac{\partial v_x}{\partial x} + v_y \frac{\partial v_x}{\partial y} = -\frac{1}{\rho} \frac{\partial p}{\partial x}\\
    v_x \frac{\partial v_y}{\partial x} + v_y \frac{\partial v_y}{\partial y} = -\frac{1}{\rho} \frac{\partial p}{\partial y}\\
    \intertext{}
    \intertext{}
    \frac{\partial^2 \phi}{\partial x^2} + \frac{\partial^2 \phi}{\partial y^2} = 0\\
    \intertext{}
    \xi = x\\
    \eta = y \sqrt{\beta}\\
    \frac{\partial^2 \phi}{\partial \xi^2} + \frac{\partial^2 \phi}{\partial \eta^2} = 0\\
    \intertext{}
    \xi = x\\
    \eta = y \sqrt{-\beta}\\
    \frac{\partial^2 \phi}{\partial \xi^2} - \frac{\partial^2 \phi}{\partial \eta^2} = 0\\
    F_D = \frac{1}{2} \rho A v^2 C
\end{align}
