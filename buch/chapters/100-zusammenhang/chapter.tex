%
% chapter.tex -- Zusammenhang und kovariante Ableitung
%
% (c) 2025 Prof Dr Andreas Müller
%
\chapter{Zusammenhang und kovariante Ableitung
\label{chapter:zusammenhang}}
\kopflinks{Zusammenhang und kovariante Ableitung}

\noindent
Die Diskussion der Lie-Ableitung in
Abschnitt~\ref{buch:koordinaten:section:differentialoperatoren}
hat gezeigt, dass die zweite Ableitung eines Kurve nicht auf
allgemein kovariante Art definiert werden kann.
Der Grund dafür ist, dass die Tangentialvektoren in den Punkten
$\gamma(t)$ und $\gamma(t+\Delta t)$ in den disjunkten Vektorräumen
$T_{\gamma(t)}M$ und $T_{\gamma(t+\Delta t)}M$ liegen und sich nicht
einfach so miteinander verrechnen lassen.
Dazu wird eine Methode benötigt, die Tangentialvektoren vom einen
Punkt in einen nahegelegenen Punkt zu transportieren.
Die einfachste Methode, dies zu erreichen, ist eine Metrik $g$ auf der
Mannigfaltigkeit zu verwenden, und Vektoren ``parallel'' zu transportieren,
wie dies in Abschnitt~\ref{buch:zusammenhang:section:paralleltransport}
unternommen wird.
{\em Geodäten} sind kürzeste Verbindungen auf einer Mannigfaltigkeit, sie
lassen sich aber auch als Kurven charakterisieren, die ihren Tangentialvektor
parallel transportieren.
\index{Geodäte}%
Dies definiert eine Differentialgleichung zweiter Ordnung für die
Geodäten (Abschnitt~\ref{buch:zusammenhang:section:geodaeten}).
Die kovariante Ableitung von
Abschnitt~\ref{buch:zusammenhang:section:kovarianteableitung}
verallgemeinert den Differentialoperator, der in der Differentialgleichung
der Geodäten vorkommt, für beliebige Vektoren.
Die kovariante Ableitung ist festgelegt, wenn man auf der Mannigfaltigkeit
eine sogenannten {\em Zusammenhang} definiert hat
(Abschnitt~\ref{buch:zusammenhang:section:zusammenhang}). 
\index{Zusammenhang}%
In Abschnitt~\ref{buch:zusammenhang:section:paralleltransport} wurden
Vektoren mit Hilfe des Paralleltransports verglichen.
Ein Zusammenhang verallgemeinert diese Idee und ermöglicht damit auch
Vektoren zu vergleichen, die nicht mit der Metrik verrechnet werden
können.

\section{Paralleltransport
\label{buch:zusammenhang:section:paralleltransport}}

\section{Geodäten
\label{buch:zusammenhang:section:geodaeten}}

\section{Kovariante Ableitung
\label{buch:zusammenhang:section:kovarianteableitung}}

\section{Zusammenhang
\label{buch:zusammenhang:section:zusammenhang}}

\section{Strömungsgleichungen
\label{buch:zusammenhang:section:stroemungsgleichungen}}

