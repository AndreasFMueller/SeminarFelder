%
% teil2.tex -- Beispiel-File für teil2
%
% (c) 2020 Prof Dr Andreas Müller, Hochschule Rapperswil
%
% !TEX root = ../../buch.tex
% !TEX encoding = UTF-8
%
\section{Kantorowitch-Formulierung%
\label{mongekant:section:teil2}}
\kopfrechts{Kantorowitch-Formulierung}

Im vorangegangenen Abschnitt haben wir das klassische Monge‑Problem beschrieben.
Dieses Modell ist intuitiv,
aber die Forderung $T_{\#}\mu=\nu$ ist sehr streng.
Wie wir bereits in Abschnitt~\ref{mongekant:subsection:monge_inexistence} gesehen haben,
kann das Monge‑Problem bereits für sehr einfache Masse keine Lösung besitzen.

Um dieses Existenzproblem zu umgehen,
hat Leonid Witaljewitch~Kantorowitch das Modell relaxiert.
Anstatt nach einer Abbildung $T\colon X\to Y$ zu suchen,
wird ein \emph{Transportplan} $\gamma(x,y) \in \mathcal{P}(X \times Y)$ eingeführt.
Ein Transportplan ist ein Wahrscheinlichkeitsmass,
dass beschreibt,
wie viel Masse von einem Punkt $x\in X$ zu jedem Punkt in $Y$ transportiert wird.
Zudem muss ein Transportplan $\gamma$ die folgenden Randbedingungen erfüllen:
\begin{align}
\begin{aligned}
\gamma(A \times Y)
&=
\mu(A)
,\quad \forall A \subset X
\\
\gamma(X \times B)
&=
\nu(B)
,\quad \forall B \subset Y
.
\end{aligned}
\label{mongekant:eq:kantorowitch_marginals}
\end{align}
Somit ändert sich das Optimierungsproblem zu
\begin{align}
C_K(\mu, \nu)
&:=
\inf_{\gamma}
\int_{X \times Y} c(x,y)\, d\gamma(x,y)
,\quad
\forall \gamma \in \Gamma(\mu, \nu)
,
\label{mongekant:eq:kantorowitch_problem}
\end{align}
wobei $\Gamma(\mu, \nu)$ die Menge aller Transportpläne ist,
die die Randbedingungen \eqref{mongekant:eq:kantorowitch_marginals} erfüllen.

Die Kantorowitch‑Formulierung bringt zwei zentrale Vorteile mit sich,
die wir im Folgenden ausnutzen werden:
\begin{enumerate}
\item \textbf{Existenz.}
Durch die Relaxation zu einer linearen Optimierung
über die konvexe Menge $\Gamma(\mu,\nu)$ kann man,
gemäss \cite{mongekant:ethlecture},
bereits unter sehr milden Annahmen
einen optimalen Plan $\gamma^{\ast}$ garantieren.
\item \textbf{Dualität.}
Das lineare Programm lässt sich in ein \emph{duales} Problem umformen,
was zu weiteren Einsichten in das Problem führt.
In Abschnitt~\ref{mongekant:subsection:kantorowitch_duality} werden
wir das duale Problem herleiten.
\end{enumerate}

\subsection{Zusammenhang zu Monge-Formulierung%
\label{mongekant:subsection:monge_kantorowitch_connection}}
Nun stellt sich die Frage,
wie die Monge- und Kantorowitch-Formulierung zusammenhängen und
ob sie ähnliche Ergebnisse liefern.
Dafür nehmen wir an,
dass eine optimale Abbildung $T\colon X \to Y$ für das Monge-Problem existiert,
für die \eqref{mongekant:eq:monge_problem} minimal wird.
Wählen wir jetzt
\begin{align}
d\gamma(x,y)
&=
d\mu(x) \delta_{y=T(x)}
\label{mongekant:eq:plan_from_map}
\end{align}
und setzen dies in die Randbedingungen \eqref{mongekant:eq:kantorowitch_marginals} ein,
\begin{align*}
\gamma(A \times Y)
&=
\int_A \delta_{T(x) \in Y}\, d\mu(x)
\\
&=
\mu(A)
\\
\gamma(X \times B)
&=
\int_X \delta_{T(x) \in B}\, d\mu(x)
\\
&=
\mu\left(T^{-1}[B]\right)
\\
&=
\nu(B)
,
\end{align*}
sehen wir,
dass $\gamma$ diese erfüllt.
Jetzt können wir also \eqref{mongekant:eq:plan_from_map} in \eqref{mongekant:eq:kantorowitch_problem} einsetzen:
\begin{align*}
\int_{X \times Y} c(x,y)\, d\gamma(x,y)
&=
\int_X \int_Y c(x,y) \delta_{y=T(x)}\, dy\, d\mu(x)
\\
&=
\int_X c(x, T(x))\, d\mu(x)
.
\end{align*}
Das entpricht genau den Kosten in \eqref{mongekant:eq:monge_transport_cost}.
Daraus folgt,
\begin{align*}
C_K(\mu, \nu)
&\leq
C_M(\mu, \nu)
.
\end{align*}
Somit liefert die Kantorowitch-Formulierung immer mindestens so gute Ergebnisse
wie die Monge-Formulierung.
Berücksichtigt man zudem,
dass die Kantorowitch-Formulierung weniger strenge Anforderungen an das Problem stellt,
ergibt sich,
ein klarer Vorteil für die Kantorowitch-Formulierung.

\subsection{Dualität im Kantorowitch-Problem%
\label{mongekant:subsection:kantorowitch_duality}}

Um das duale Problem herzuleiten,
starten wir mit dem Kantorowitch-Problem \eqref{mongekant:eq:kantorowitch_problem}
und versuchen die Randbedingungen \eqref{mongekant:eq:kantorowitch_marginals}
mit Lagrangemultiplikatoren $\xi \colon X \to \mathbb{R}$ und
$\eta \colon Y \to \mathbb{R}$ implizit zu berücksichtigen,
so dass $\gamma$ nur noch durch die Nichtnegativitätsbedingung im Gebiet $X\times Y$
eingeschränkt ist:
\begin{align*}
\inf_{\gamma \in \Gamma}
\int_{X \times Y} c(x,y)\, d\gamma
=&
\inf_{\gamma \geq 0}
\int_{X \times Y} c(x,y)\, d\gamma
\\
 & + \sup_{\xi, \eta} \Biggl(
\underbrace{
\int_X \xi(x)\, d\mu(x) - \int_{X \times Y} \xi(x)\, d\gamma
}_{\gamma(A \times Y) = \mu(A)}
+
\underbrace{
\int_Y \eta(x)\, d\nu(x) - \int_{X \times Y} \eta(x)\, d\gamma
}_{\gamma(X \times B) = \nu(B)}
\Biggr)
.
\end{align*}
Somit gilt für
\begin{align*}
\sup_{\xi, \eta} (\ldots)
&=
\begin{cases}
0,
& \text{falls } \gamma \in \Gamma(\mu, \nu) \\
+\infty,
& \text{sonst}
\end{cases}
.
\end{align*}
Verschieben wir jetzt das Supremum nach aussen,
was erlaubt ist,
da das Supremum über $\xi$ und $\eta$ nicht von $\gamma$ abhängt,
und fassen wir die Integrale zusammen,
ergibt sich
\begin{align*}
\inf_{\gamma \in \Gamma}
\int_{X \times Y} c(x,y)\, d\gamma
&=
\inf_{\gamma \geq 0}
\sup_{\xi, \eta}
\int_X \xi(x)\, d\mu(x)
+ \int_Y \eta(y)\, d\nu(y)
+ \int_{X \times Y} \bigl(c(x,y) - \xi(x) - \eta(y)\bigr)\, d\gamma
\\
&=
\inf_{\gamma \geq 0}
\sup_{\xi, \eta}
\int_X \xi(x)\, d\mu(x)
+ \int_Y \eta(y)\, d\nu(y)
+ \int_{X \times Y} \bigl(c(x,y) - \xi(x) - \eta(y)\bigr)\, d\gamma
\\
&=
\sup_{\xi, \eta}
\int_X \xi(x)\, d\mu(x)
\int_Y \eta(y)\, d\nu(y)
+ \underbrace{
\inf_{\gamma \geq 0} \int_{X \times Y} \bigl(c(x,y) - \xi(x) - \eta(y)\bigr)\, d\gamma
}_{\Xi}
.
\end{align*}
Dabei gilt
\begin{align*}
\Xi
&=
\begin{cases}
0,
& \text{falls } c(x,y) - \xi(x) - \eta(y) \geq 0
,\quad\forall (x,y) \in X \times Y
\\
-\infty,
& \text{sonst}
\end{cases}
.
\end{align*}
Somit ergibt sich die duale Beziehung
\begin{align}
\inf_{\gamma \in \Gamma(\mu, \nu)}
\int_{X \times Y} c(x,y)\, d\gamma
&=
\sup_{\xi, \eta}
\int_X \xi(x)\, d\mu(x)
+ \int_Y \eta(y)\, d\nu(y)
,
\label{mongekant:eq:kantorowitch_dual}
\end{align}
wobei $\xi + \eta \leq c$ gilt.

Allerdings ist jetzt fraglich,
wie die Funktionen $\xi(x)$ und $\eta(y)$ zu interpretieren sind.
Dafür möchten wir ein
auf die heutige Zeit angepasstes Beispiel aus \cite{mongekant:villani} verwenden.

Jeden Morgen geniessen die Menschen zu Hause ihre Kaffeepause.
Insgesamt kostet es Nespresso $c(x,y)$ Franken,
um eine Schachtel der nötigen Espresso‑Kapseln von Ort $x$ zu Ort $y$ zu transportieren,
also von den Lagern zu den Haushalten.
Nespresso möchten diese teure Gewohnheit optimieren und
dafür das passende Kantorovich‑Problem formulieren.
Mathematiker kommen zu Nespresso und schlagen ein neues Zahlungsmodell vor.
Für jede Schachtel,
die an Ort $x$ liegt,
verlangen sie $\xi(x)$ Franken,
und für die Lieferung an Ort $y$ verlangen sie $\eta(y)$ Franken.
Allerdings werden die Mathematiker die konkreten
Versandrouten nicht preisgeben.
Damit Nespresso dieses Angebot akzeptiert,
muss selbstverständlich gelten $\xi(x)+\eta(y) \leq c(x,y)$.


Die Idee ist:
Wenn die Mathematiker klug genug sind,
können sie die Sendung günstiger machen.
Das wird durch den \emph{Kantorovich‑Dualitätssatz} garantiert.

\noindent
\emph{Hinweis.}
In manchen Fällen können die Mathematiker sogar \emph{negative} Preise ansetzen.
Das bedeutet,
sie erhalten für bestimmte Abhol‑ oder Lieferorte einen Zuschuss,
um insgesamt die Gesamtkosten zu senken.
