%
% table-polyeder.tex
%
% (c) 2025 Prof Dr Andreas Müller
%
\begin{table}
\centering
\begin{tabular}{l>{$}r<{$}>{$}r<{$}>{$}r<{$}>{$}r<{$}}
\hline
Polyeder          &\text{Ecken}&\text{Kanten}&\text{Flächen}&\chi(M)\\
\hline
Tetraeder         &           4&            6&             4&      2\\
Oktaeder          &           6&           12&             8&      2\\
Hexader (Würfel)  &           8&           12&             6&      2\\
Dodekaeder        &          20&           30&            12&      2\\
Ikosaeder         &          12&           30&            20&      2\\
\hline
Kugel             &         266&          792&           528&      2\\
Torus             &         576&         1728&          1152&      0\\
\hline
\end{tabular}
\caption{Die eulerschen Polyeder haben alle die Euler-Charakteristik $2$.
Während die Triangulation einer Kugel aus
\index{Euler-Charakteristik}%
\index{Tetraeder}%
\index{Oktaeder}%
\index{Hexaeder}%
\index{Wurfel@Würfel}%
\index{Dodekaeder}%
\index{Ikosaeder}%
Abbildung~\ref{buch:topologie:intro:fig:sphere} wie die Polyeder
Euler-Charakteristik $2$ hat, ergibt die Triangulation des Torus von
\index{Triangulation}%
Abbildung~\ref{buch:topologie:intro:fig:torus} jedoch
der Wert 0, der zum Ausdruck bringt, dass der Torus ein ``Loch''
\index{Torus}%
\index{Kugel}%
hat.
\label{buch:topologie:intro:table:eulercharakteristik}}
\end{table}
