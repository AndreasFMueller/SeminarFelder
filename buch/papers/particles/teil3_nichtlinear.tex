%
% teil2.tex -- Beispiel-File für teil2 
%
% (c) 2020 Prof Dr Andreas Müller, Hochschule Rapperswil
%
% !TEX root = ../../buch.tex
% !TEX encoding = UTF-8
%
\section{Nichtlineares Medium
\label{particles:section:nichtlinear}}
\kopfrechts{Nichtlineares Medium}
Entgegen des linearen Mediums wobei der Proportionalitätsfaktor nicht von der wirkenden Kraft abhängig war, 
so wird dies bei der Nichtlinearität nun zum Problem.
Dies bedeutet also, dass sich das Hookesche Gesetz von 
\[
    \Delta l
    = 
    \frac{F}{D}
\]
zu
\[
    \Delta l
    = 
    \frac{F}{D(F)}
\]
abändert. 
Wie dieser Proportionalitätsfaktor genau von der Kraft abhängt, 
ist je nach Medium unterschiedlich.

% TODO: Grafik mit gleichem Aufbau wie in "lineares Medium", aber mit nichtlinearem medium (Zwei Lichtwellen)

\subsection{Schwinger-Limit in der Praxis}
Wie bereits in Abschnitt~\ref{particles:section:lin-medium:schwinger} erwähnt, 
ist das praktische Schwinger-Limit noch nicht genug erforscht um eine eindeutige Aussage darüber zu machen.
Es gibt jedoch reichlich andere Medien---beispielsweise Glasfasern---welche bereits bei geringen Feldstärken nichtlineare Eigenschaften aufweisen.


% TODO: Folgenden Titel anpassen, da noch unsicher, wie man das auf deutsch genau nennen sollte
\subsection{Selbstfokussierung -- Confined Energy} % NOTE: "Confined Energy"

% TODO: Was für Bedingungen müssen erfüllt sein, damit dies geschieht? 
%       Kann man das überhaupt so einfach klassifizieren?
%       Allenfalls Analogie zu Schwingkreisen herstellen?

% TODO: Grafik von "Partikel" einfügen

