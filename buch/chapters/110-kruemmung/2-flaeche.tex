%
% Krümmungstensor einer Fläche
%
\section{Krümmungstensor einer Fläche
\label{buch:kruemmung:section:flaeche}}
Im Falle einer zweidimensionalen Mannigfaltigkeit enthält der
riemannsche Krümmungstensor nur eine einzige unabhängige Komponente,
wie in diesem Abschnitt gezeigt werden soll.
Dies eröffnet die Möglichkeit, die Krümmung eines Raumes dadurch zu
studieren, dass die Krümmung zweidimensionaler Untermannigfaltigkeiten
untersucht wird.
Dies wird in Abschnitt~\ref{buch:kruemmung:section:schnittkruemmung}
mit der Schnittkrümmung durchgeführt.

%
% Symmetrien des Krümmungstensors
%
\subsection{Symmetrien des Krümmungstensors}
Wir betrachten die kovarianten Komponenten des riemannschen
Krümmungstensors $R_{hlik}$, wobei die Indizes nur die Werte 1 oder 2
annehmen können.
Er ist antisymmetrisch im ersten und letzten Indexpaar.
Alle Komponenten mit $h=l$ oder $i=k$ verschwinden, es bleiben daher
nur noch die Komponenten
\[
R_{1212}
=
-R_{2112}
=
R_{2121}
=
-R_{1221},
\]
die möglicherweise nicht verschwinden.

\subsubsection{Bianchi-Identität}
Für eine zweidimensionale Mannigfaltigkeit ist der Krümmungstensor also
durch genau eine Funktion $R_{1212}(x)$ vollständig bestimmt.
Der Krümmungstensor erfüllt aber zusätzlich noch die Bianchi-Identitäten,
die in diesem Fall eine Aussage über die Grösse
\[
B
=
R_{1{\color{darkred}212}}
+
R_{1{\color{darkred}122}}
+
R_{1{\color{darkred}221}}
\]
ist.
Die roten Indizes sind zyklisch permutiert worden.
Der mittlere Term verschwindet wegen der Antisymmetrie in den ersten
zwei und den letzten zwei Indizes.
Der letzte Term ist aber $R_{1212}=-R_{1221}$, so dass die Summe zu
\[
B
=
R_{1212}+R_{1221}
=
R_{1212}-R_{1212}
=
0,
\]
die Bianchi-Identität ist daher immer erfüllt.

\subsubsection{Die differentielle Bianchi-Identität}
Ausserdem erfüllt der riemannsche Krümmungstensor die differentielle
Bianchi-Identität, die in Komponenten
\[
R_{abik;l}
+
R_{abkl;i}
+
R_{abli;k}
=
0
\]
ist.
Für eine zweidimensionale Mannigfaltigkeit können wir $a=1$ und $b=2$
setzen.
Ausserdem ist die Identität trivialerweise erfüllt, wenn die hinteren
drei Indizes $i=k=l$ gleich sind, denn dann verschwindet jeder Term.
Es müssen daher nur noch zwei Fälle untersucht werden, nämlich
\begin{itemize}
\item Fall $i=k=1$ und $l=2$:
\[
R_{1211;2} + R_{1212;1} + R_{1221;1} 
=
R_{1212;1}-R_{1212;1} = 0.
\]
\item Fall $i=k=2$ und $l=1$:
\[
R_{1222;1} + R_{1221;2} + R_{1212;2}  
=
R_{1221;2} - R_{1221;2}
=
0.
\]
\end{itemize}
Auch die differentiellen Bianchi-Identitäten sind also automatisch
erfüllt und liefern keine zusätzliche Information über den Krümmungstensor.

%
% Die Gauss-Krümmung
%
\subsection{Die Gauss-Krümmung}
Für im dreidimensionalen Raum eingebettete Flächen wurde die Krümmung
besonders von Gauss ausführlich untersucht.
Die Fläche erbt von der Längenmessung in $\mathbb{R}^3$ eine
Metrik, der riemannsche Krümmungstensor lässt sich daraus berechnen
und führt auf die Komponente $R_{1212}$, die den Krümmungstensor vollständig
beschreibt.
Gauss konstruierte einen Krümmungsbegriff aus der Untersuchung der
Krümmung von Kurven in der Fläche.
Gauss konnte zeigen, dass einer der von ihm gefundenen Krümmungsbegriffe,
der heute die Gauss-Krümmung $K$ genannt wird, nur von der Metrik selbst
abhängt, nicht von weiteren Eigenschaften der Einbettung.
Es stellt sich heraus, dass  $K$ im wesentlichen die Komponenten $R_{1212}$
des riemannschen Krümmungstensors ist.

\subsubsection{Untermannigfaltigkeiten von $\mathbb{R}^3$}
Eine 2-dimensionale Untermannigfaltigkeit von $\mathbb{R}^3$ kann
mindestens lokal durch eine Parametrisierung
\[
f\colon U\to \mathbb{R}^3
:
(u,v)\mapsto f(u,v)
=
\begin{pmatrix}
x(u,v)\\
y(u,v)\\
z(u,v)
\end{pmatrix}
\]
dargestellt werden, wobei die Funktionen $x(u,v)$, $y(u,v)$ und $z(u,v)$
mindestens zweimal stetig differenzierbar sein sollen.

Nicht jede Abbildung ist als Parametrisierung einer Fläche geeignet.
Damit sich eine differenzierbare Mannigfaltigkeit ergibt, muss die die
$f$ lokal ein Diffeomorphismus zwischen der $U$ und der Bildmenge $f(U)$
sein.
Dies ist genau dann möglich, wenn die Tangentialvektoren
\[
f_u
=
\frac{\partial f}{\partial u}
\qquad\text{und}\qquad
f_v
=
\frac{\partial f}{\partial v}
\]
in jedem Punkt linear unabhängig sind.
Dies kann auch dadurch ausgedrückt werden, dass der Vektor
\[
\vec{n}
=
f_u\times f_v
=
\frac{\partial f}{\partial u}
\times
\frac{\partial f}{\partial v}
\]
nicht verschwindet.
Der Vektor steht senkrecht auf der Mannigfaltigkeit.
Für eine solche eingebettete Mannigfaltigkeit lässt sich also sofort
ein wohldefinierter Normalenvektor angeben.

\begin{beispiel}
Ist $z(x,y)$ eine zweimal stetig differenzierbare Funktion der Variablen
$x$ und $y$, dann ist
\[
f\colon
U\to \mathbb{R}^3
:
(x,y) \mapsto \begin{pmatrix}x\\y\\z(x,y)\end{pmatrix}
\]
eine Abbildung dieser Art.
Damit die Abbildung nicht entartet ist, müssen die Vektoren der partiellen
Ableitungen linear unabhängig sein, diese sind
\begin{equation}
\frac{\partial f}{\partial x}
=
f_x
=
\begin{pmatrix}
1\\0\\ \frac{\partial z}{\partial x}
\end{pmatrix}
\qquad\text{und}\qquad
\frac{\partial f}{\partial y}
=
f_y
=
\begin{pmatrix}
0\\1\\ \frac{\partial z}{\partial y}.
\end{pmatrix}
\label{buch:kruemmung:gausskruemmung:eqn:xytangential}
\end{equation}
Die Projektion dieser Vektoren in die $x$-$y$-Ebene ergibt die 
Standarbasisvektoren in zwei Dimensionen, die natürlich linear
unabhängig sind.
Also sind auch die Vektoren $f_x$ und $f_y$ linear unabhängig.
\end{beispiel}

%
% Metrik
%
\subsubsection{Metrik}
Das Standardskalarprodukt auf $\mathbb{R}^3$ definiert auch ein
Skalarprodukt auf der durch $f$ eingebetteten Mannigfaltigkeit $M$.
Ein Tangentialvektor $X$ an die Mannigfaltigkeit $M$ ist eine
Linearkombination
\[
X
=
\xi
\frac{\partial f}{\partial u}
+
\eta
\frac{\partial f}{\partial v}
\]
der Tangentialvektoren der Koordinatenableitungen.
Die Länge des Tangentialvektors kann daher mit dem Standardskalarprodukt
in $\mathbb{R}^3$ ausgerechnet werden und ist
\begin{align*}
g(X,X)
&=
\langle
\xi f_u + \eta f_v,
\xi f_u + \eta f_v
\rangle
\\
&=
\xi^2 \langle f_u,f_u\rangle
+2
\xi\eta \langle f_u,f_v\rangle
+
\eta^2 \langle f_v,f_v\rangle.
\intertext{Der letzte Ausdruck lässt sich auch in Matrixform als}
&=
\begin{pmatrix}\xi\\\eta\end{pmatrix}^t
\begin{pmatrix}
\langle f_u,f_u\rangle & \langle f_u,f_v\rangle \\
\langle f_u,f_v\rangle & \langle f_v,f_v\rangle
\end{pmatrix}
\begin{pmatrix}\xi\\\eta\end{pmatrix}
=
\begin{pmatrix}\xi\\\eta\end{pmatrix}^t
G
\begin{pmatrix}\xi\\\eta\end{pmatrix}
\end{align*}
mit der Gram-Matrix
\begin{equation}
G
=
\begin{pmatrix}
\langle f_u,f_u\rangle & \langle f_u,f_v\rangle \\
\langle f_u,f_v\rangle & \langle f_v,f_v\rangle
\end{pmatrix}
\label{buch:kruemmung:gausskrümmung:eqn:gram}
\end{equation}
schreiben.
Die Matrix $G$ beschreibt also die Metrik in den $u$-$v$-Koordinaten
auf der Fläche.

\begin{beispiel}
Für den Fall des Graphen einer Funktion $z(x,y)$ ist die Gram-Matrix
leicht auszurechnen.
Aus \eqref{buch:kruemmung:gausskrümmung:eqn:gram} ergibt sich
\begin{align*}
G
&=
\begin{pmatrix}
\langle f_x,f_x\rangle & \langle f_x,f_y\rangle \\
\langle f_x,f_y\rangle & \langle f_y,f_y\rangle
\end{pmatrix}
\\
&=
\begin{pmatrix}
1+z_x(x,y)^2 & z_x(x,y)z_y(x,y) \\
z_x(x,y) z_y(x,y) & 1 + z_y(x,y)^2
\end{pmatrix},
\end{align*}
wobei die Darstellung
\eqref{buch:kruemmung:gausskruemmung:eqn:xytangential}
der Tangentialvektoren verwendet wurde.
\end{beispiel}

Die durch die Matrix $G$ beschriebene Bilinearform wird auch die
erste Fundamentalform der Fläche genannt.

%
% Krümmungsradius von Kurven auf der Fläche
%
\subsubsection{Krümmungsradius von Kurven auf der Fläche}
Wir betrachten jetzt Kurven auf der eingebetteten Mannigfaltigkeit.
Da die Mannigfaltigkeit lokal wie $\mathbb{R}^2$ aussieht, lassen sich
darin Kurven mit beliebig kleinen Krümmungsradien finden.
Andererseits gibt es zum Beispiel auf einer Kugeloberfläche keine 
ungekrümmten Kurven.
Die Krümmung der Fläche wird also dadurch charakterisiert,
wie gross der Krümmungsradius maximal sein kann.
Intuitiv erwartet man, dass von allen Kurven, die mit einem vorgegebenen
Tangentialvektor durch den Punkt $P$ gehen, diejenige Kurve den grössen
Krümmungsradius hat, die in der Ebene aufgespannt vom Tangentialvektor
und der Normalen liegt.

Sie die Fläche wieder durch die Abbildung $f\colon U\to\mathbb{R}^3$
gegeben und die Kurve durch 
$\gamma\colon I\to U : t\mapsto(u(t),v(t))$ im
Parametergebiet $U$ derart, dass $f\circ\gamma(0)=P$.
Wir können annehmen, dass $t$ ein Bogenlängenparameter ist, dass
also der Tangentialvektor 
\[
\vec{v}
=
\frac{d}{dt} f\circ\gamma(t) 
\bigg|_{t=0}
\]
die Länge 1 hat.

Die Krümmung misst, wie schnell der normierte Normalenvektor seine
Richtung ändert.
Dabei spielt nur Komponente parallel zu $\vec{v}$ eine Rolle.
Die Krümmung ist das Skalarprodukt
\[
\frac{1}{R}
=
\vec{v} \cdot \frac{d\vec{n}\circ\gamma}{dt}(0),
\]
wobei $R$ der Krümmungsradius ist.
Die Krümmung ist also eine Funktion, die von der Richtung $\vec{v}$
der Kurventangente abhängt.

Die Normale steht natürlich senkrecht auf dem Tangentialvektor, 
es ist also $\vec{v}\cdot\vec{n}=0$.
Dies gilt auch entlang der Kurve, 
\[
\vec{n}\circ\gamma(t) \cdot \frac{d}{dt}f\circ\gamma(t)
=
0
\]
für alle $t\in I$.
Die Ableitung nach $t$ ist
\[
\frac{d}{dt}
\vec{n}\circ\gamma(t)
\cdot
\frac{d}{dt}f\circ\gamma(t)
+
\vec{n}\circ\gamma(t)
\cdot
\frac{d^2}{dt^2}f\circ\gamma(t)
=
0
\]
was
\begin{equation}
\frac{d}{dt}\vec{n}\circ\gamma(t)\cdot \frac{d}{dt}f\circ\gamma(t)
=
-
\vec{n}\circ\gamma(t)
\cdot
\frac{d^2}{dt^2}f\circ\gamma(t)
\label{buch:kruemmung:flaeche:eqn:sk}
\end{equation}
zur Folge hat.
Die Linke Seite von \eqref{buch:kruemmung:flaeche:eqn:sk}
ist die Krümmung.

%
% Die zweite Fundamentalform
%
\subsubsection{Die zweite Fundamentalform}
Für die rechte Seite von \eqref{buch:kruemmung:flaeche:eqn:sk}
muss die zweite Ableitung bestimmt werden, sie ist
\begin{align*}
\frac{d^2}{dt^2}
f\circ\gamma(t)
&=
\frac{d}{dt}
\biggl(
\frac{\partial f}{\partial u}(u(t),v(t)) \dot{u}(t)
+
\frac{\partial f}{\partial v}(u(t),v(t)) \dot{v}(t)
\biggr)
\\
&=
\frac{\partial^2 f}{\partial u^2}(u(t),v(t)) \, \dot{u}(t)^2
+
2\frac{\partial^2 f}{\partial u\,\partial v}(u(t),v(t)) \, \dot{u}(t) \dot{v}(t)
+
\frac{\partial^2 f}{\partial v^2}(u(t),v(t)) \, \dot{v}(t)^2
\\
&\quad
+
\frac{\partial f}{\partial u}(u(t),v(t)) \, \ddot{u}(t)
+
\frac{\partial f}{\partial v}(u(t),v(t)) \, \ddot{v}(t).
\end{align*}
Die letzten zwei Term auf der rechten Seite bilden einen Vektor, der von $f_u$
und $f_v$ aufgespannt wird, also einen Tangentialvektor.
Dieser Vektor steht auf der Normalen $\vec{n}$ senkrecht, in der
Formel \eqref{buch:kruemmung:flaeche:eqn:sk} kann er daher weggelassen
werden.
Es bleibt dann die Krümmung
\begin{align*}
\frac{1}{R}
&=
-\vec{n}\circ\gamma(t)
\cdot 
\biggl(
\frac{\partial^2 f}{\partial u^2}(u(t),v(t)) \, \dot{u}(t)^2
+
2\frac{\partial^2 f}{\partial u\,\partial v}(u(t),v(t)) \, \dot{u}(t) \dot{v}(t)
+
\frac{\partial^2 f}{\partial v^2}(u(t),v(t)) \, \dot{v}(t)^2
\biggr)
\\
&=
\vec{n}\cdot\frac{\partial^2 f}{\partial u^2} \,
\dot{u}(t)^2
+
\vec{n}\cdot\frac{\partial^2 f}{\partial u\,\partial v} \,
\dot{u}(t)\,\dot{v}(t)
+
\vec{n}\cdot\frac{\partial^2 f}{\partial v^2} \,
\dot{v}(t)^2
\end{align*}
Dies ist eine Bilinearform mit der Koeffizientenmatrix
\[
H
=
\begin{pmatrix}
h_{11} & h_{12} \\
h_{21} & h_{22}
\end{pmatrix}
=
\begin{pmatrix}
\vec{n}\cdot f_{uu} & \vec{n}\cdot f_{uv} \\
\vec{n}\cdot f_{vu} & \vec{n}\cdot f_{vv}
\end{pmatrix}.
\]
Sie heisst die zweite Fundamentalform und wird auch
\[
II(\vec{w},\vec{q})
=
\vec{w}^t H \vec{q}
\]
geschrieben.
Nach dem Satz von Schwarz kommt es auf die Reihenfolge der partiellen
Ableitungen nicht an, also ist $f_{uv}=f_{vu}$, die Matrix $H$ ist daher
symmetrisch.

Die Polarisierungsformel zeigt, dass die bilineare Funktion
$(\vec{w},\vec{q}) \mapsto II(\vec{w},\vec{q})$ vollständig
bestimmt ist durch die Werte $II(\vec{w},\vec{w})$ bestimmt,
denn aus der Symmetrie der zweiten Fundamentalform folgt
\begin{align*}
II(\vec{w}+\vec{q},\vec{w}+\vec{q})
&=
II(\vec{w},\vec{w}) + 2\,II(\vec{w},\vec{q}) + II(\vec{q},\vec{q})
\\
II(\vec{w}-\vec{q},\vec{w}-\vec{q})
&=
II(\vec{w},\vec{w}) - 2\,II(\vec{w},\vec{q}) + II(\vec{q},\vec{q})
\intertext{mit der Differenz}
II(\vec{w}+\vec{q},\vec{w}+\vec{q})
-
II(\vec{w}-\vec{q},\vec{w}-\vec{q})
&=
4\,II(\vec{w},\vec{q}),
\intertext{die man nach nach}
II(\vec{w},\vec{q})
&=
{\textstyle\frac14}\bigl(
II(\vec{w}+\vec{q},\vec{w}+\vec{q})
-
II(\vec{w}-\vec{q},\vec{w}-\vec{q})
\bigr).
\end{align*}
auflösen, dies ist die Polarisierungsformel.
Die rechte Seite enthält wie versprochen nur Ausdrücke der Form
$II(\vec{a},\vec{a})$.

%
% Hauptkrümmungen
%
\subsubsection{Hauptkrümmungen}
Für jeden tangentialen Richtungsvektor $\vec{w}\in T_pM$ im Pumkt $p$
liefert die zweite Fundamentalform die Krümmung $\vec{w}^t H\vec{v}$ 
der Schnittkurve.
Die tangentialen Vektoren $\vec{w}$ im Punkt $p$ mit Länge $1$ bilden
eine kompakte Menge. 
Da die Funktion $\vec{w}\mapsto II(\vec{w},\vec{w})$ ist eine stetige
Funktion und nimmt daher auf dem Kreis ein Maximum und ein Minimum an.
Nach der allgemeinen Eigenwerttheorie der symmetrischen Matrizen
geschieht dies für Richtungen, die Eigenvektoren der Matrix $H$
sind.
Diese Richtungen heissen die \emph{Hauptkrümmungsrichtungen}.
Die zugehörigen Eigenwerte heissen die \emph{Hauptkrümmungen}.

%
% Die Mittlere Krümmung
%
\subsubsection{Mittlere Krümmung}
Die lineare Algebra liefert skalare Invarianten, die von der zweiten
Fundamentalform abgeleitet werden können.
Die Spur der Matrix ist
\[
\operatorname{Spur} H
=
h_{11}
+
h_{22}
=
\vec{n}\cdot f_{uu}
+
\vec{n}\cdot f_{vv}
=
\vec{n}\cdot(f_{uu}+f_{vv})
=
\vec{n}\cdot\Delta f.
\]
Sie heisst die \emph{mittlere Krümmung} der Fläche $M$.
Man kann zeigen, dass Minimalflächen dadurch charakterisiert sind
durch die Eigenschaft, dass die mittlere Krümmung verschwindet.

%
% Die Gauss-Krümmung
%
\subsubsection{Gauss-Krümmung}
Eine weitere Invariante ist die Determinante der Matrix $H$, sie ist
\[
K
=
\det H
=
h_{11}h_{22}-h_{12}^2
=
\left|\,\begin{matrix}
\vec{n}\cdot f_{uu} & \vec{n}\cdot f_{uv} \\
\vec{n}\cdot f_{vu} & \vec{n}\cdot f_{vv}
\end{matrix}\,\right|.
\]
Sie heisst die \emph{gausssche Krümmung}.
Gauss hat gezeigt, dass die gaussche Krümmung eine intrinsische Eigenschaft
der Fläche ist.
Er nannte diese überraschende Eigenschaft das \emph{Theorema egregium}, 
Die Gauss-Krümmung muss sich also auch allein aus den metrischen
Koeffizienten $g_{ik}$ berechnen lassen.

Da die vollständige intrinsische Krümmungsinformation bereits durch die
Komponente $R_{1212}$ gegeben ist, muss es einen Zusammenhang zwischen
$R_{1212}$ und der Gausskrümmung geben.
Für ein Koordinatensystem, für weldhes im Punkt $p$  die Vektoren
$\partial/\partial u$ und $\partial/\partial v$ orthonormiert sind,
ist $R_{1212}=K$.
Für ein beliebiges Koordinatensystem gilt
\[
R_{1212}
=
K(g_{11}g_{22}-g_{12}^2),
\]
wobei der Klammerfaktor die Koordinatentransformation auf dem 
eindimensionalen Raum $\bigwedge^2 T_pM$ ausdrückt.
