
\section{Wedgeprodukt
\label{geoalgebra:section:wedgeprodukt}}
\kopfrechts{Wedgeprodukt}

Im Rahmen der geometrischen Algebra möchten wir eine neue Operation
definieren, das \emph{Wedgeprodukt} $\wedge$.
Es ist die grundlegende Operation der geometrischen
Algebra.
Angenommen, wir haben die
zwei Vektoren $\a, \b$ und bilden das Wedgeprodukt
\begin{equation}
  B = \a \wedge \b.
\end{equation}
erhalten wir ein neues mathematisches Konzept, den \emph{Bivektor} $B$.
Ein Bivektor unterscheidet sich grundsätzlich von einem Vektor (siehe \autoref{geoalgebra:section:bivektoren}).

\subsection{Definition und grundlegende Axiome}
\label{geoalgebra:section:axiome}
Um das Wedgeprodukt definieren zu können, benötigen wir einige wenige Axiome, d.h. grundlegende Annahmen, die wir treffen,
um uns eine Algebra darauf aufzubauen. Somit definieren wir

\begin{axiom}
  Das Wedgeprodukt ist \em{antikommutativ}

  \begin{equation}
  \a \wedge \b = -\b \wedge \a.
  \end{equation}
  \label{geoalgebra:eq:antikommutativ}
\end{axiom}
Daraus folgt:
\begin{lemma}
  \label{geoalgebra:lemma:null}
  Das Wedgeprodukt eines Vektors mit sich selbst ist $0$
  \begin{equation*}
  \a \wedge \a = 0.
  \end{equation*}
\end{lemma}

\begin{axiom}
  Es gilt das \em{Distributivgesetz}
  \begin{equation*}
  \a \wedge (\b + \c) = \a \wedge \b + \a \wedge \c.
  \end{equation*}
\end{axiom}

\begin{axiom}
  Das Wedgeprodukt ist bilinear in beiden Faktoren
  \begin{equation}
    \a \wedge (\lambda \b + \mu \c) = \lambda (\a \wedge \b) + \mu (\a \wedge \c).
    \label{geoalgebra:eq:bilinear}
  \end{equation}
\end{axiom}

Für die Definition von \emph{Trivektoren} benötigen wir desweiteren noch das Axiom
\begin{axiom}
  Das Wedgeprodukt ist \em{assoziativ}
  \begin{equation*}
    (\mathbf{a} \wedge \mathbf{b}) \wedge \mathbf{c} = \mathbf{a} \wedge (\mathbf{b} \wedge \mathbf{c})
  \end{equation*}
\end{axiom}

\subsection{Das Wedgeprodukt in zwei Dimensionen}
\renewcommand{\subsectionautorefname}{Abschnitt}
Mit den Axiomen aus \autoref{geoalgebra:section:axiome} können wir bereits das Wedgeprodukt zwischen zwei
Vektoren in der zwei Dimensionen herleiten.
Dazu machen wir uns zu nutze, dass wir jeden Vektor als
Linearkombination von den Standardbasisvektoren beschreiben können.
Es gilt
\begin{align}
  \a &= \begin{pmatrix} a_1 \\ a_2 \end{pmatrix} = a_1 \mathbf{e}_1 + a_2 \mathbf{e}_2.
\end{align}

\begin{definition}
  Wir betrachten die beiden zweidimensionalen Vektoren $\a,
  \b$.
  Das Wedgeprodukt von $\a$ und $\b$ ist

  \begin{equation}
    \begin{aligned}
    \a \wedge \b &= (a_1 \eone + a_2 \etwo) \wedge
    (b_1 \eone + b_2 \etwo) \\
    &= a_1 \eone \wedge (b_1 \eone + b_2 \etwo) + a_2 \etwo \wedge (b_1 \eone + b_2 \etwo) \\
    &= a_1 \eone \wedge b_1 \eone + a_1 \eone \wedge b_2 \etwo + a_2 \etwo \wedge b_1 \eone + a_2 \etwo \wedge b_2 \etwo \\
    &= a_1 b_1 (\underbrace{\eone \wedge \eone}_{0}) + a_1 b_2 (\eone \wedge \etwo) + a_2 b_1 (\etwo \wedge \eone) + a_2 b_2 (\underbrace{\etwo \wedge \etwo}_{0}) \\
    &= (a_1 b_2 - a_2 b_1) \eone \wedge \etwo.
    \end{aligned}
    \label{geoalgebra:eq:2d-wedgeproduct}
  \end{equation}
\end{definition}
\noindent Der Skalarfaktor im letzten Ausdruck ist eine Determinante und kann daher als
\begin{equation}
  \begin{vmatrix}
    a_1 & b_1 \\
    a_2 & b_2
  \end{vmatrix}
  \eone \wedge \etwo
\end{equation}
beschrieben werden.

\subsection{Das Wedgeprodukt in drei Dimensionen}
Zunächst wollen wir uns nochmals unser bereits bekanntes
Vektorprodukt ansehen.
\begin{definition}
  Wir betrachten die beiden dreidimensionalen Vektoren $\mathbf{a}, \mathbf{b}$
  und bilden das Kreuzprodukt
  \begin{equation}
    \mathbf{a} \times \mathbf{b} = \begin{pmatrix} a_2 b_3 - a_3 b_2 \\ a_3 b_1 - a_1 b_3 \\ a_1 b_2 - a_2 b_1 \end{pmatrix}.
    \label{geoalgebra:eq:vektorprodukt}
  \end{equation}
  
\end{definition}

Das Wedgeprodukt funktioniert nicht nur in zwei, sondern auch
in beliebigen Dimensionen. Im folgenden möchten wir kurz
das Wedgeprodukt zwischen zwei dreidimensionalen Vektoren
herleiten.

\begin{definition}
Wir betrachten die \emph{dreidimensionalen} Vektoren $\mathbf{a}, \mathbf{b}$.
Das Wedgeprodukt von $\mathbf{a}$ und $\mathbf{b}$ ist
\begin{equation}
  \begin{aligned}
      \mathbf{a} \wedge \mathbf{b} &= (a_1 \eone + a_2 \etwo + a_3 \ethree) \wedge (b_1 \eone + b_2 \etwo + b_3 \ethree) \\
      &= a_1 \eone \wedge (b_1 \eone + b_2 \etwo + b_3 \ethree) \\
      &\quad+ a_2 \etwo \wedge (b_1 \eone + b_2 \etwo + b_3 \ethree) \\
      &\quad+ a_3 \ethree \wedge (b_1 \eone + b_2 \etwo + b_3 \ethree) \\
      &= (a_1 b_2 - a_2 b_1) \eone \wedge \etwo \\
      &\quad+ (a_2 b_3 - a_3 b_2) \etwo \wedge \ethree \\
      &\quad+ (a_3 b_1 - a_1 b_3) \ethree \wedge \eone.
  \end{aligned}
\end{equation}
\label{geoalgebra:eq:wedgeprodukt-dreidimensional}
\end{definition}
\noindent Wir stellen fest, dass die Koeffizienten hinter den drei entstandenen Wedgeprodukten
gleich sind, wie die Komponenten des Vektorprodukts in \eqref{geoalgebra:eq:vektorprodukt}.

\subsection{Rückblick}
Hier wollen wir einen kurzen Rückblick vornehmen. Wir haben zunächst
die komplexen Zahlen nochmals kurz betrachtet und haben gesehen,
wie sie eine natürliche Erweiterung der reellen Zahlen bieten.
Analog dazu bietet die geometrische Algebra eine natürliche
Erweiterung zur bereits bekannten Vektorgeometrie. Wir wissen,
dass das Vektorprodukt $\a_1 \times \a_2$ nur
in drei Dimensionen funktioniert und haben erkannt, dass
die neue Operation des Wedgeprodukts $\a_1 \wedge \a_2$
im Gegensatz dazu auch in zwei Dimensionen funktioniert.
Später wollen wir das Ganze noch auf beliebige Dimensionen erweitern.
