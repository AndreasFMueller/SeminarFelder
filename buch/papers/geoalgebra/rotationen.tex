\section{Rotationen}
\renewcommand{\u}{\hat{\mathbf{u}}}
\subsection{Rotationen sind zwei aufeinanderfolgende Spiegelungen}
Eine Rotation kann durch zwei aufeinanderfolgende Spiegelungen ausgedrückt werden (siehe
\autoref{geoalgebra:fig:rotation-as-two-reflections}).
\begin{figure}
  \begin{center}
\begin{tikzpicture}[>=latex, scale=2]
\fill[green!30] (-0.5, 0.5) -- (-0.5, 2) -- (-0.1, 0.9) -- cycle;
\draw[thick, ->] (0, 0) -- (0.5, 2) node[left]{$\v$};
\draw[thick, ->] (0, 0) -- (2, 0.5) node[above]{$\u$};
\fill[green!30] (0.68, 0.21) -- (1.38, 1.53) -- (0.51, 0.75) -- cycle;
\fill[green!30] (0.69, 0.13) -- (1.94, -0.70) -- (0.80, -0.42) -- cycle;
\draw[thick, blue] (-0.5, 0.5) -- (0.68, 0.21);
\draw[thick, blue] (-0.5, 2) -- (1.38, 1.53);
\draw[thick, blue] (-0.1, 0.9) -- (0.51, 0.75);

\draw[thick, red] (0.68, 0.21) -- (0.69, 0.13);
\draw[thick, red] (1.38, 1.53) -- (1.94, -0.70);
\draw[thick, red] (0.51, 0.75) -- (0.80, -0.42);
\end{tikzpicture}

  \end{center}
  \caption{Eine Rotation kann durch zwei aufeinanderfolgende Spiegelungen abgebildet werden}
\label{geoalgebra:fig:rotation-as-two-reflections}
\end{figure}

\subsection{Rotation eines Vektors}
Man betrachte den Vektor $\mathbf{x}$, der zunächst an $\v$ und danach an $\u$ gespiegelt werden
sollte (siehe \autoref{geoalgebra:fig:rotation}). $\mathbf{x}'$ wird analog zu berechnet durch
\begin{equation}
\mathbf{x}' = \v \mathbf{x} \v.
\end{equation}
Analog dazu kann der neu entstandene Vektor $\mathbf{x}'$ an $\u$ gespiegelt werden
\begin{equation}
\mathbf{x}'' = \u \mathbf{x}' \u.
\end{equation}
Die ganze Rotation wird somit durch
\begin{equation}
\mathbf{x}'' = \u \v \mathbf{x} \v \u
\end{equation}
berechnet.
\begin{figure}
  \begin{center}
\begin{tikzpicture}[>=latex, scale=2]
\draw[thick, darkgreen, ->] (0, 0) -- (-0.5, 2) node[left]{$\mathbf{x}$};
\draw[thick, ->] (0, 0) -- (0.5, 2) node[left]{$\v$};
\draw[thick, ->] (0, 0) -- (2, 0.5) node[above]{$\u$};;
\draw[thick, darkgreen!30, ->] (0, 0) -- (1.38, 1.53) node[left]{$\mathbf{x}'$};
\draw[thick, darkgreen!30, ->] (0, 0) -- (1.94, -0.70) node[below]{$\mathbf{x}''$};
\draw[thick, blue] (-0.5, 2) -- (1.38, 1.53);
\draw[thick, darkred] (1.38, 1.53) -- (1.94, -0.70);
\end{tikzpicture}


  \end{center}
  \caption{Rotation eines Vektors}
\label{geoalgebra:fig:rotation}
\end{figure}


