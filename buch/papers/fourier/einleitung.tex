%
% einleitung.tex -- Einleitung und Motivation
%
% (c) 2020 Prof Dr Andreas Müller, Hochschule Rapperswil
%
% !TEX root = ../../buch.tex
% !TEX encoding = UTF-8
%

\section{Einleitung\label{fourier:section:einleitung}}
%\kopfrechts{Grundlagen der Fourier-Analyse}


Felder kann man sich als eine unendliche Ansammlung von Feldlinien vorstellen, die sich kreuz und quer durch Raum-Zeit schlängeln. Die Fourier-Analyse hilft dabei, dieses scheinbare Chaos in einfache Sinus- und Cosinus-Schwingungen zu zerlegen. 
In der Quantenfeldtheorie werden diese Schwingungen anders interpretiert, was letztlich dazu führt, dass das Feld als Summe einer beschränkten Anzahl von Teilchen verstanden wird.









