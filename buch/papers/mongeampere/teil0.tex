%
% einleitung.tex -- Beispiel-File für die Einleitung
%
% (c) 2020 Prof Dr Andreas Müller, Hochschule Rapperswil
%
% !TEX root = ../../buch.tex
% !TEX encoding = UTF-8
%
\section{Die Differentialgleichung\label{mongeampere:section:teil0}}
\kopfrechts{Teil 0}
Die monge-ampèresche Gleichung ist
\begin{equation}
  \det D^2 u = f(x, u, \nabla u),
  \label{mongeampere:eq:mongeampere}
\end{equation}
wobei $D$ die hessesche Matrix
\begin{equation*}
  D^2 u =
\renewcommand{\arraystretch}{1.7}
  \begin{pmatrix}
    \displaystyle \frac{\partial^2 u}{\partial x_{1\mathstrut}^2\mathstrut}
	& \dots
		&\displaystyle \frac{\partial^2 u}{\partial x_{1\mathstrut}^{\phantom{2}} \,\partial x_{n\mathstrut}^{\phantom{2}}\mathstrut} \\[-2pt]
  \vdots & \ddots & \vdots \\
  \displaystyle \frac{\partial^2 u}{\partial x_{n\mathstrut}^{\phantom{2}} \,\partial x_{1\mathstrut}^{\phantom{2}}\mathstrut}
	& \dots
		&\displaystyle \frac{\partial^2 u}{\partial x_{n\mathstrut}^2\mathstrut} \\
  \end{pmatrix}
  %\label{mongeampere:eq:hess}
\end{equation*}
ist.
Wenn man die Gleichung ansieht, erkennt man, dass es sich bei ihr um eine partielle
nicht-lineare Differentialgleichung handelt.
Das Lösen einer solchen Differentialgleichung ist im Allgemeinen ein anspruchvolles Problem.
Anhand von Bedingungen für $u$ und $f$ kann man \eqref{mongeampere:eq:mongeampere} 
in einen bestimmten Typ Differentialgleichung bringen, was unterschiedliche Lösungsansätze ermöglicht.
Setzt man voraus, dass $u$ eine glatte konvexe Function ist, und $f$ positiv, dann ist \eqref{mongeampere:eq:mongeampere}
eine elliptische partielle Differentialgleichung.
\index{elliptische partielle Differentialgleichung}%
Gemäss \cite{mongeampere:figalli2018mongeampereequation} gibt eine solche Form Hoffnung auf die Existenz einer regulären Lösung.
Ein anderes Beispiel ist in \cite{mongeampere:figalli2022prescribednegativegausscurvature} zu finden, wo die 
Gleichung linearisiert wird um lokale Lösungen zu finden.

\subsection{Minkowski-Problem}
\index{Minkowski-Problem}%
Eine Anwendung, welche vom deutschen Mathematiker Hermann Minkwoski entdekt wurde, ist das Lösen des
\index{Minkowski, Hermann}%
\emph{Prescribed Gaussian Curvature Problem} mit der Monge-Ampère-Gleichung.
\index{Prescribed Gaussian Curvature Problem}%
Diese Anwendung befasst damit eine konvexe kompakte Fläche zu finden, deren gausssche Krümmung 
an jedem Punkt gegeben ist.

