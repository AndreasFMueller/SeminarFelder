%
% Die Friedmann-Gleichung und die Geschichte des Universums
%
\section{Die Friedmann-Gleichungen und die Geschichte des Universums
\label{buch:kruemmung:section:friedmann}}
\kopfrechts{Die Friedmann-Gleichungen}%
Vor dem frühen 20.~Jahrhundert gab es weder theoretische noch experimentelle
Hinweise darauf, dass das Universum sich über die Zeit verändern könnte.
Es wurde daher allgemein angenommen, dass das Universum immer gleich
ausgesehen hat.
Es gab daher auch keine Möglichkeit und keinen Bedarf, eine Theorie der
Geschichte des Universums zu entwickeln.
Die allgemeine Relativitätstheorie ermöglichte erstmals, das ganze
Universum mathematisch zu erfassen und damit seine Geschichte aus
den aktuellen Beobachtungen zu extrapolieren.

1923 konnte Edwin Hubble nachweisen, dass der Andromeda-Nebel, wie
\index{Hubble, Edwin}%
er damals genannt wurde, eine eigene Galaxie weit ausserhalb unserer
eigenen ist.
Dazu verwendete er den Zusammenhang zwischen absoluter Helligkeit
und Periode gewisser veränderlicher Sterne, die er auch in der
Andromeda-Galaxie fand.
\index{Andromeda-Galaxie}%
Die Methode ermöglicht erstmals, die Distanzmessung bis zu nahen Galaxien
zu erstrecken.
Messungen von Galaxienspektren durch Milton Humason 
\index{Humason, Milton}%
zeigte eine Rotverschiebung, entfernte Galaxien scheinen sich zu
\index{Rotverschiebung}%
entfernen.
1929 postulierte Hubble aufgrund weiterer Messungen einen linearen
Zusammenhang zwischen Rotverschiebung und Entfernung, die heute
das hubblesche Gesetz genannt wird.
\index{hubblesches Gesetz}%
\index{Gesetz!von Hubble}%
Sie liess den Schluss zu, dass sich das ganze Universum ausdehnt.
Aus der Ausdehnungsgeschwindigkeit, der sogenannten 
\emph{Hubble-Konstante} $H_0$
\index{Hubble-Konstante}%
lässt sich auch erstmals eine Schätzung für das Alter des Universums
abgeben.

%
% Das kosmologische Prinzip
%
\subsection{Das kosmologische Prinzip}
Die einsteinschen Feldgleichungen beschreiben die Metrik in Abhängigkeit
von der Verteilung von Energie und Impuls im Universum.
Doch wie soll man diese messen?
Über kleine Distanzen zum Beispiel innerhalb unserer Galaxie ist die
Massedichte extrem unterschiedlich:
Neutronensterne haben extrem hohe Dichte, aber der grösste Teil des
Universums ist ein sehr gutes Vakuum.
Aber auch noch über die Distanzen zwischen Galaxien gibt es grosse
Unterschiede: Materie im Universum scheint vor allem in Galaxien
zusammengeklumpt zu sein, dazwischen findet man fast nichts.
Über noch grössere Distanzen scheinen auch die Galaxien in Clustern
mit grossen leeren Räumen genannt ``voids'' dazwischen organisiert
zu sein.

Andererseits gibt es keinen Grund, dass die Erde mit ihrer Sonne, ihrer
eigenen Milchstrasse und den wenigen Galaxien der lokalen Gruppe in
irgend einer Form speziell sein sollte.
Daraus lässt sich das \emph{kosmologische Prinzip} ableiten.
\index{kosmologisches Prinzip}%
Es besagt, dass das Universum homogen und isotrop ist.
\emph{Homogen} bedeutet, dass an jedem Punkt die gleichen Verhältnisse
anzutreffen sind.
\index{homogen}%
\emph{Isotrop} bedeutet, dass keine Richtung im Universum speziell ist.
Im Grossen ist daher das Universum gefüllt mit einem Gas konstanter
Dichte.
\index{isotrop}%
Die rechte Seite der einsteinschen Feldgleichungen ist für die 
Zwecke der relativistischen Kosmologie ein räumlich konstanter Tensor.

%
% Die Robertson-Walker-Metrik
%
\subsection{Die Friedmann-Lema\^\i tre-Robertson-Walker-Metrik}
Auch die Lösung der einsteinschen Feldgleichungen muss die vom
kosmologischen Prinzip geforderten Eigenschaften erfüllen.
Die Metrik und die zugehörige Krümmung darf nicht vom Ort
abhängen.
Nur eine Zeitabhängigkeit darf übrig bleiben, die mit hubbleschen
Gesetz der Ausdehnung des Universums kompatibel sein muss.

Die homogenen und isotropen Räume können mit den Methoden der
riemannschen Geometrie untersucht und klassifiziert werden.
Man kann zeigen, dass dreidimensionale Räume konstanter Krümmung 
die Metrik
\[
ds^2
=
\frac{dr^2}{1-kr^2} + r^2\,(d\vartheta^2 + \sin\vartheta\,d\varphi^2)
\]
haben, wobei $k$ der sogenannte \emph{Krümmungsparameter} ist.
Für $k=0$ entsteht die bekannte euklidische Metrik.

Die Beobachtungen von Hubble zeigen, dass Distanzen zwischen
den Galaxien mit der Zeit grösser werden.
Diese Zeitabhängigkeit kann mit einer Funktion $a(t)$ beschrieben
werden, die zum heutigen Zeitpunkt $t_0$ den Wert $a(t_0)=1$ hat.
Die Metrik
\begin{equation}
ds^2
=
a(t)^2
\biggl(
\frac{dr^2}{1-kr^2} + r^2\,(d\vartheta^2 + \sin\vartheta\,d\varphi^2)
\biggr)
\label{buch:kruemmung:kosmologie:eqn:ads}
\end{equation}
beschreibt Abstände, die um den Faktor $a(t)$ grösser werden.

Die Zeitabhängigkeit muss zu jedem Zeitpunkt eine Minkowski-Metrik
ergeben, damit im Grenzfall ohne Gravitation die empirisch bestätigte
spezielle Raum-Zeit-Struktur des Universums entsteht.
Es resultiert die Metrik der folgenden Definition.

\begin{definition}[Friedmann-Lema\^\i tre-Robertson-Walker-Metrik]
Die \emph{Friedmann-Le\-ma\^\i tre-Robertson-Walker-Metrik} ist die Metrik
\index{Friedmann-Lemaitre-Robertson-Walker-Metrik@Friedmann-Lema\^\i tre-Robertson-Walker- Metrik}%
mit dem Linienelement
\begin{equation}
ds^2
=
c^2\,dt^2
-
a(t)^2\biggl(
\frac{dr^2}{1-kr^2} + r^2\,(d\vartheta^2 + \sin\vartheta\,d\varphi^2)
\biggr)
\end{equation}
Die Funktion $a(t)$ heisst der Skalenfaktor und hat den Wert $a(t_0)=1$
zum aktuellen Zeitpunkt $t_0$.
\index{Skalenfaktor}%
\end{definition}

Um die Geschichte des Universums zu ergründen müssen also sowohl der
Krümmungsparameter $k$ wie auch die Funktion $a(t)$ bestimmt werden.
Aus dem hubbleschen Gesetz ist ausserdem bekannt, wie schnell $a(t)$
ändert, es ist $\dot{a}(t)=H_0$.
Es würde also genügen, eine Differentialgleichung zweiter Ordnung
für $a(t)$ zu finden.
Eine solche müsste sich aus den einsteinschen Feldgleichungen 
gewinnen lassen.
Dazu sind die Ricci-Krümmung und der Krümmungsskalar zu bestimmen,
was eine ziemlich aufwendige Rechnung ist.
Wir notieren daher hier nur die Resultate.
Die Komponenten des Ricci-Tensors sind
\begin{align*}
R_{00}
&=
-3 \frac{\ddot{a}(t)}{a(t)}
\\
R_{ik}
&=
-
\biggl(
\frac{\ddot{a}(t)}{a(t)} + 2\frac{\dot{a}(t)^2}{a(t)^2}+\frac{2k}{a(t)^2}
\biggr)
g_{ik},
\intertext{der Krümmungsskalar wird}
R
&=
-6
\biggl(
\frac{\ddot{a}(t)}{a(t)} + \frac{\dot{a}(t)^2}{a(t)^2}+\frac{k}{a(t)^2}
\biggr).
\end{align*}
Die Komponenten des Einstein-Tensors können daraus wie folgt
ermittelt werden:
\begin{equation}
G_{\mu\nu}
=
R_{\mu\nu} - \frac12 g_{\mu\nu} R
\qquad\Rightarrow\qquad
\left\{
\begin{aligned}
G_{00}
&=
\frac{3}{2}
\frac{\dot{a}(t)^2}{a(t)^2} 
+\frac{k}{a(t)^2}
\\
G_{ii}
&=
2
\frac{\ddot{a}(t)}{a(t)}
+
\frac{\dot{a}(t)^2}{a(t)}
+
\frac{k}{a(t)^2}.
\end{aligned}
\right.
\label{buch:kruemmung:kosmologie:eqn:G}
\end{equation}

%
% Die Friedmann-Gleichung
%
\subsection{Die Friedmann-Gleichung}
Mit Hilfe des Einstein-Tensors, der in \eqref{buch:kruemmung:kosmologie:eqn:G}
berechnet wurde, kann man jetzt auch die Einstein-Gleichungen
für das Universum aufstellen.
Dazu braucht man aber auf der rechten Seite der Gleichung einen 
Ausdruck für den Energie-Impuls-Tensor.
Seine $00$-Komponente enthält die Massedichte, die anderen diagonalen
Komponenten geben die Energie wieder, die von der Bewegung der
Teilchen herrührt.
Sie äussert sich im Druck $p$ des Gases.
Die Komponenten von $T_{ik}$ sind daher
\begin{equation}
\begin{aligned}
T_{00} &= c^2 \varrho 
\\
T_{ii} &= -p g_{ii}&&\text{für $i>0$}
\end{aligned}
\end{equation}
Einsetzen in die einsteinsche Feldgleichung
\eqref{buch:kruemmung:feldgleichung:eqn:feldgleichung}
folgen die Gleichungen
\begin{align*}
6
\frac{\dot{a}(t)^2}{a(t)^2} 
+\frac{k}{a(t)^2}
&=
\frac{8\pi G}{c^4}c^2\varrho
&&\Rightarrow&
\biggl(\frac{\dot{a}(t)}{a(t)}\biggr)^2
+
\frac{k}{a(t)^2}
&=\frac{4\pi G\varrho}{c^2}
\\
2
\frac{\ddot{a}(t)}{a(t)}
+
\frac{\dot{a}(t)^2}{a(t)}
+
\frac{k}{a(t)^2}
&=
-\frac{8\pi G}{c^4}p.
\end{align*}
Diese beiden Gleichungen heissen auch die 
\emph{Friedmann-Gleichungen}.
\index{Friedmann-Gleichungen}
Sie wurden von Aleksander Friedmann 1922 hergeleitet.
Sie beschreiben, wie der Skalenfaktor $a(t)$ des Universums sich
mit der Zeit verändern kann.

Zur vollständigen Lösung der Friedmann-Gleichungen braucht man 
zusätzliche Information über die Abhängigkeit von $\varrho$
und $p$ vom Skalenfaktor.
\begin{itemize}
\item
Ein gewöhnliches Gas wird die Dichte mit zunehmendem Skalenfaktor
wie $a(t)^{-3}$ abnehmen.
\item
Für die im Universum vorhandene Strahlung gilt dies jedoch nicht.
Bei der Ausdehnung wird die Wellenlänge der Strahlung ebenfalls um
den Faktor $a(t)$ gestreckt, die Energie eines Photons wird daher
um den Faktor $a(t)^{-1}$ geringer.
Zusammen mit der Abnahme der Dichte der Photonen wird der
Strahlungsdruck daher wie $a(t)^{-4}$ abnehmen.
\item
In den letzten Jahrzehnten hat sich gezeigt, dass die beobachtbare
Materie (inklusive der dunklen Materie) und Strahlung nicht ausreicht,
um das verhalten des Universums zu erklären.
Die sogenannte \emph{dunkle Energie} füllt das Vakuum gleichmässig
aus.
Insbesondere hängt sie nicht von $a(t)$ ab.
\index{dunkle Energie}%
\end{itemize}
Die Entwicklung des Radius des Universums hängt also auch vom Energiemix
des Universums ab.

%
% Ds Alter des Universums
%
\subsection{Das Alter des Universums}
Kennt man die Anteile der verschiedenen Komponenten im
Energie-Impuls-Tensor, kann man aus den Friedmann-Gleichungen eine
Formel für das Alter des Universums herleiten.
Für
\begin{itemize}
\item
$\Omega_r$: Anteil der Strahlung an der Energiedichte im heutigen
Universum.
Das heutige Universum ist sehr dunkel, der Strahlungsanteil ist
nur etwa $5.5\cdot10^{-5}$.
\item
$\Omega_m$: Anteil der Materie (inklusive dunkler Materie) an der
Energiedichte im heutigen Universum.
\item
$\Omega_k$: Beitrag des Krümmungsparameters.
Beobachtungen der Inhomogeneitäten des kosmischen Mikrowellenhintergrundes
haben gezeigt, dass dieser Beitrag sehr klein ist.
\item
$\Omega_\Lambda$: Anteil der dunklen Energie an der Energiedichte im 
heutigen Universum.
Nach aktuellem Wissen macht die dunkle Energie mehr als zwei Drittel
der Energiedichte aus.
\end{itemize}
lautet die Formel
\begin{equation}
t_0
=
\frac{1}{H_0}
\int_0^{a_0}
\frac{da}{\!\sqrt{\Omega/a^2 + \Omega_m/a + \Omega_k + \Omega_\Lambda a^2}}.
\label{buch:kruemmung:kosmologie:eqn:alter}
\end{equation}
Die Exponenten des Skalenfaktors reflektieren die Abhängigkeit 
der Energiedichte der entsprechenden Komponente vom den Skalenfaktor.

Mit der Formel
\eqref{buch:kruemmung:kosmologie:eqn:alter}
kann man für die experimentell von der Planck-Mission bestimmten Werte
\begin{align*}
H_0 &= 67.15\,\textrm{km/s/Mpc} &
\Omega_m &= 0.315,&
\Omega_r &= 5.5\cdot 10^{-5},&
\Omega_\Lambda &= 0.685,&
\Omega_k &< 0.005
\end{align*}
das Alter von $13.844\,\textrm{Gyr}$ finden.

