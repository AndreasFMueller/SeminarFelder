%
% teil4.tex -- Beispiel-File für Teil 4
%
% (c) 2020 Prof Dr Andreas Müller, Hochschule Rapperswil
%
% !TEX root = ../../buch.tex
% !TEX encoding = UTF-8
%
%\usepackage{siunitx}
%\usepackage{amsmath}
%\usepackage[ngerman]{babel}
\section{Beispiele aus der Bauingenieurspraxis}
\label{elastomechanik:section:teil4}
In den vorangegangenen Abschnitten (15.1–15.4) haben wir die allgemeinen Grundlagen der Elastizitätstheorie entwickelt: 
von der Definition der Verzerrungen über das Hookesche Gesetz bis hin zu den Gleichgewichtsbedingungen und der Formulierung der Navier--Cauchy-Gleichungen. 
Diese theoretischen Resultate bilden die Grundlage, um reale Ingenieurprobleme zu beschreiben. 
Im Folgenden wollen wir an zwei typischen Beispielen, einem Balken unter Mittellast und einem Oedometerversuch, aufzeigen, wie sich aus den allgemeinen Gleichungen praktische Berechnungen ableiten lassen. 
Zur Erinnerung fassen wir die wesentlichen Zusammenhänge zusammen, die wir in \S15.1–15.4 hergeleitet haben:
die Kinematik
\[
  \varepsilon_{ij} = \tfrac12(\partial_i u_j + \partial_j u_i),
\]
das konstitutive Gesetz
\[
  \sigma_{ij} = \lambda\, \delta_{ij}\,\varepsilon_{kk} + 2\mu\, \varepsilon_{ij},
\]
und das Gleichgewicht
\[
  \partial_j \sigma_{ij} + f_i = 0.
\]
die zusammen zu den Navier--Cauchy-Gleichungen führen.
Für schlanke Balken ergeben die Euler--Bernoulli-Annahmen die Balkenbeziehung
\[
  EI\, w^{(4)}(x) = q(x),
\]
aus der die Formeln für Biegespannung und Durchbiegung folgen.
Für den Oedometerversuch gilt ein eindimensionaler Verzerrungszustand
\(\varepsilon_x=\varepsilon_y=0\); daraus folgt
\[
  \sigma_z = M \,\varepsilon_z, \qquad
  M=\frac{E(1-\nu)}{(1+\nu)(1-2\nu)},
\]
wobei \(M\) als eingeschränkter (oedometrischer) Modul bezeichnet wird.

\subsection{Balken unter Mittellast}
Wir betrachten einen Einfeldträger der Länge \(l = 5\,\mathrm{m}\), der in der Mitte mit einer Einzellast \(F = 100\,\mathrm{kN}\) belastet wird. 
Das System ist statisch bestimmt, sodass sich die Auflagerreaktionen unmittelbar aus den Gleichgewichtsbedingungen ergeben:
\[
  R_A = R_B = \frac{F}{2}
\]
Das Moment an einer Stelle \(x\) zwischen dem linken Auflager und der Last ergibt sich zu
\[
  M(x) = R_A \, x = \frac{F}{2}\,x
\]
Für \(x=l/2\), also an der Stelle der Belastung, nimmt das Moment seinen 
maximalen Wert an:
\begin{equation}
  M_{\max} = \frac{F\,l}{4} = 125\,\mathrm{kN\,m}
  \label{eq:Mmax}
\end{equation}
Für den Rechteckquerschnitt mit Breite \(b=200\,\mathrm{mm}\) 
und Höhe \(h=300\,\mathrm{mm}\) ist das Flächenträgheitsmoment
\[
  I = \frac{b h^3}{12} = 4.5 \times 10^8\,\mathrm{mm^4}
  \qquad 
\]
Der Abstand \(y\) gibt den maximalen Abstand der äussersten Faser von der neutralen Achse an. 
Für einen rechteckigen Querschnitt liegt die neutrale Achse 
in der Mitte, daher gilt:
\[
  y = \frac{h}{2}= 150\mathrm{mm}
\]
Damit ergibt sich die maximale Biegespannung zu
\begin{equation}
  \sigma_{\max} = \frac{M_{\max}\,y}{I} = 41.7\,\mathrm{MPa}
  \label{eq:sigma_max}
\end{equation}
Die Durchbiegung in der Feldmitte erhält man schließlich aus der Balkendifferentialgleichung \(EI\,w''(x)=M(x)\). 
Einmaliges Integrieren liefert die Querkraftlinie, zweimaliges Integrieren die Biegelinie. 
Mit den Randbedingungen \(w(0)=w(l)=0\) ergibt sich für die 
maximale Durchbiegung:
\begin{equation}
  w_{\max} = \frac{F\,l^3}{48\,E\,I}
  \label{eq:wmax}
\end{equation}
Eingesetzt ergeben sich 
\(w_{\max} \approx 48.2\,\mathrm{mm}\) für 
\(\,E_{\emph{Holz}}=12\,\mathrm{GPa}\)
und
\(w_{\max} \approx 2.76\,\mathrm{mm}\) für 
\(\,E_{\emph{Stahl}}=210\,\mathrm{GPa}\).
\subsection{Oedometerversuch}
Im Oedometerversuch wird eine Bodenprobe so eingespannt, dass seitliche Dehnungen nicht auftreten können. 
Belastet man die Probe mit einer vertikalen Spannung \(\sigma_z\), so kann sie sich ausschliesslich in vertikaler Richtung verkürzen.  
Der Zusammenhang zwischen Spannung und Dehnung lässt sich aus dem isotropen Hooke’schen Gesetz ableiten (vgl. Kapitel 15.2). 
Da die seitlichen Dehnungen unter den Oedometerversuchsbedingungen verschwinden, setzen wir
\[
  \varepsilon_x = \varepsilon_y = 0.
\]
Wenn wir diese Bedingung in das Hooke’sche Gesetz einsetzen, zeigt sich, dass in horizontaler Richtung eine Spannung \(\sigma_h\) auftritt, die proportional zur vertikalen Spannung \(\sigma_z\) ist. 
Diese Spannung lässt sich schreiben als:
\[
  \sigma_h = \frac{\nu}{1-\nu}\,\sigma_z,
\]
wobei \(\nu\) die Poissonzahl ist.  
Sobald die seitliche Spannung berücksichtigt ist, können wir die vertikale Dehnung \(\varepsilon_z\) bestimmen. 
Durch Einsetzen von \(\sigma_h\) in das Hooke’sche Gesetz ergibt sich:
\[
  \varepsilon_z = \frac{\sigma_z}{E}\,\frac{1-\nu-2\nu^2}{1-\nu}.
\]
Aus der Definition des Oedometer-Moduls als Verhältnis von Spannung zu vertikaler Dehnung folgt dann direkt:
\[
  E_o = \frac{\sigma_z}{\varepsilon_z} = \frac{E(1-\nu)}{(1+\nu)(1-2\nu)}.
\]
\paragraph{Beispiel}   
Betrachten wir eine Probe mit einer Höhe von \(h_0 = 20\,\mathrm{mm}\), die unter einer Belastung von \(\sigma_z = 100\,\mathrm{kPa}\) um \(\Delta h = 0.5\,\mathrm{mm}\) verkürzt wird.  
Zuerst berechnen wir die vertikale Dehnung:
\[
  \varepsilon_z = \frac{\Delta h}{h_0} = \frac{0.5}{20} = 0.025.
\]
Anschliessend ergibt sich der Oedometer-Modul direkt aus der Spannung geteilt durch die Dehnung:
\[
  E_o = \frac{\sigma_z}{\varepsilon_z} = \frac{100}{0.025} = 4000\,\mathrm{kPa}.
\]
\paragraph{Bemerkung}  
Dieser Versuch verdeutlicht, dass der Oedometer-Modul \(E_o\) nicht mit dem dreidimensionalen Elastizitätsmodul \(E\) identisch ist. 
Er hängt sowohl von \(E\) als auch von der Poissonzahl \(\nu\) ab und ergibt sich unmittelbar aus dem Hooke’schen Gesetz in Verbindung mit den speziellen Deformationsbedingungen des Oedometerversuchs.
