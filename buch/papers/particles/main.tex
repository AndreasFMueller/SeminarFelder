%
% main.tex -- Paper zum Thema <particles>
%
% (c) 2020 Autor, OST Ostschweizer Fachhochschule
%
% !TEX root = ../../buch.tex
% !TEX encoding = UTF-8
%
\newcommand{\mytodo}[1]{\textbf{\textcolor{red}{TODO: #1}}}
\newcommand{\mynote}[1]{\textbf{\textcolor{orange}{NOTE: #1}}}
\chapter{Turning Waves into Particles\label{chapter:particles}}
\kopflinks{Turning Waves into Particles}
\begin{refsection}
\chapterauthor{Flurin Brechbühler, Laurin Heitzer}

Wellen treten überall auf, egal ob in Wasser, als Schallwelle, Licht oder in der Elektrotechnik.
Bei diesen Medien geht man oft davon aus, dass sie sich linear verhalten.
Viel spannender wird es aber, wenn sich Nichtlinearitäten einschleichen, 
welche einige interessante Nebeneffekte mit sich bringen.
Solche Nebeneffekte wurden im Video \emph{Turning Waves into Particles} \mytodo{Refernz einfügen} von Huygens Optics (YouTube) gezeigt,
wobei dort spezifisch die Idee war, Energie räumlich zu fangen mithilfe der Nichtlinearität des Mediums.

In der folgenden Arbeit wird dieses Phänomen nochmals mathematisch aufgearbeitet. % TODO: Satz passt noch nicht
Ziel ist es, den theoretischen Hintergrund zu linearen und nichtlinearen Medien zu erklären 
und anhand numerischer Simulation deren Auswirkungen auf die Wellenausbreitung zu veranschaulichen.


% TODO: Unser Paper verwendet numerische Simulation um "einen blick über den Horizont der linearen Felder zu werfen"

% NOTE: Aufbau abändern in: Teil 1: lineares Medium; Teil 2: nichtlineares Medium; Teil 3: Simulation; Teil 4: Fazit
% TODO: Dateinamen anpassen!
%
% teil1.tex -- Beispiel-File für das Paper
%
% (c) 2020 Prof Dr Andreas Müller, Hochschule Rapperswil
%
% !TEX root = ../../buch.tex
% !TEX encoding = UTF-8
%
\section{Lineares Medium\label{particles:section:linear}}
\kopfrechts{Lineares Medium}
% TODO: Quellenangaben
% [ ]: https://en.wikipedia.org/wiki/Linearity
% [ ]: Lineare Algebra: Eine anwendungsorientierte Einführung, Seite: 27, ISBN: 978-3-662-67865-7, Published: 01 September 2023, DOI: https://doi.org/10.1007/978-3-662-67866-4

\subsection{Was ist Linearität?}
Für den einfachsten und üblichsten Fall, nimmt man oft ein \emph{lineares Medium} an.
Solch ein Medium nennt man \emph{linear}, wenn dessen Definition sowohl \emph{additiv}
\[
    f(x_{1} + y_{1}, \ldots, x_{n} + y_{n}) 
    = 
    f(x_{1}, \ldots, x_{n}) 
    + 
    f(y_{1}, \ldots, y_{n}),
\]
als auch \emph{homogen}
\[
    f(\lambda x_{1}, \ldots, \lambda x_{n}) 
    = 
    \lambda f(x_{1}, \ldots, x_{n})
\]
ist.
Hierbei ist angemerkt, dass $x_{k}$, $y_{k}$ und $\lambda$ nicht rein reell sein müssen, 
sondern einem beliebigen Vektorraum angehören können, 
was für den zweidimensionalen Fall wichtig ist.

In der Wellentheorie bedeutet dies, 
dass die Materialeigenschaften---beispielsweise die Elastizität---nicht von der Amplitude der Welle abhängen.
Diese Elastizität eines Mediums lässt sich durch das \emph{Hookesche Gesetz} \mytodo{Citation}
\[
    \Delta l
    = 
    \frac{F}{D}
    \quad
    (D = \text{const.} 
    \Rightarrow 
    \frac{\partial D}{\partial F} 
    \overset{!}{=} 
    0 
    \quad 
    \forall F)
    \label{particles:eq:hookesches-gesetz}
\]
beschreiben.
Dabei beschreibt $\Delta l$ die ändernde Distanz zweier Punkte,
$F$ die dazwischen wirkende Kraft und $D$ einen Proportionalitätsfaktor.
Mittels Substitution kann man nun darauf schliessen, 
dass es sich hierbei tatsächlich um eine lineare Funktion handeln muss.


\subsection{Superpositionsprinzip}
Das Superpositionsprinzip fasst die Bedingungen zur Linearität nochmal etwas schöner in eine Formel zusammen, 
nämlich
\[
    T(\lambda x + \mu y)
    = 
    \lambda T(x) 
    + 
    \mu T(y).
\]
Blickt man wieder auf die Wellentheorie, so bedeutet dies, 
dass sich Wellen in linearen Medien überlagern, 
sich aber nicht gegenseitig beeinflussen.
Dies kann man schön anhand zweier sich kreuzende Laserstrahlen im Vakuum veranschaulichen, 
wie sie in Abbildung \mytodo{Abbildung zweier Laserstrahlen, die sich kreuzen, lokal interferieren, jedoch den weiteren Verlauf nicht verändern.} gezeigt werden.
Lokal interferieren diese Laserstrahlen zwar, stören jedoch nicht ihren weiteren Verlauf. 

\mytodo{Irgendwo sollte man noch ein Beispiel mit $E$- oder $B$-Feld einbringen}


\subsection{Schwinger-Limit}\label{particles:section:lin-medium:schwinger}
Bei extrem hohen Feldstärken tritt ein neues Phänomen auf, 
wobei hier das Vakuum selbst, nichtlinear wird.
Dieser Übergang zur Nichtlinearität des Vakuums nennt man das \emph{Schwinger-Limit}\mytodo{Quellenangabe}, 
benannt nach Julian Schwinger, welcher 1951 dieses Phänomen erstmals theoretisierte.
Da es sich hierbei um elektrische und magnetische Feldstärken im Bereich von $10^{18}\,\frac{\text{V}}{\text{m}}$ und $10^9\,\text{T}$ handelt,
konnte dieses Limit bisher noch nicht konkret nachgewiesen werden und ist noch immer Bestandteil aktueller Forschungen. \mytodo{Quelle hierfür angeben}

Beschränkt man sich jedoch nicht nur auf das Vakuum, 
so gibt es noch andere Medien, welche bereits bei kleineren Feldstärken nichtlineare Eigenschaften aufweisen.
% Im nächsten Abschnitt soll diese theoretische Barriere jedoch überschritten werden und 
% die Auswirkung eines solchen nichtlinearen Mediums etwas genauer unter die Lupe genommen werden.

% Drückt man die Deformation $D$ aus Formel~\ref{particles:eq:lin-elastizitaet} als eine Funktion von $\varepsilon$ aus
% und stellt man die Formel ein wenig um zu 
% \[
%     D(\varepsilon)
%     = 
%     \frac{1}{E} 
%     \cdot 
%     \varepsilon,
% \]
% so kann man daraus folgern, dass es sich hierbei ebenfalls um eine lineare Funktion handelt, 
% sofern $E = \text{const.}$ eingehalten wird. % NOTE: Könnte man auch als $\frac{\partial E}{\partial \varepsilon} \overset{!}{=} 0 \quad \forall \varepsilon$ schreiben
% Folglich ist das angenommene Medium also linear und erfüllt somit auch die Superpositionsbedingungen.
% Dies vereinfacht die Berechnung des Feldes enorm, 
% da nicht für jedes einzelne $\varepsilon$ ein separates $E$ berechnet werden muss. 


% TODO: Folgendes irgendwo besser einbringen:
% \subsection{Feldgleichung}
% Das Hookesche Gesetz, so wie es einleitend erwähnt ist, 
% beschreibt im Grunde genommen nur den eindimensionalen Fall. 
% Jedoch kann man aufgrund der Superposition eine vereinfachte,
% zweidimensionale Feldgleichung herleiten, 
% indem man die Komponenten der Gleichung mit Vektoren substituiert % NOTE: Darf man das?
% \[
%     \mathbf{D} 
%     =   
%     \frac{1}{E}
%     \cdot
%     \bm{\varepsilon}
%     =   
%     \begin{bmatrix}
%         D_1\\
%         D_2
%     \end{bmatrix}
%     =   
%     \frac{1}{E}
%     \cdot 
%     \begin{bmatrix}
%         \varepsilon_1\\
%         \varepsilon_2
%     \end{bmatrix}.
% \]

% \mytodo{Allenfalls noch verallgemeinertes Hookesches Gesetz erwähnen? Ist jedoch fast etwas zu komplex in diesem Kontext\ldots}

% \mynote{Dürfen wir hier einfach substituieren oder braucht es allenfalls eine bessere Herleitung?}
% TODO: Herleitung Feldgleichung aus Elastizitätsgleichung etc.
% In der Elastizitätsgleichung ist die Feldgleichung zwar gewissermassen enthalten,
% jedoch ist sie nicht direkt ersichtlich. 
% TODO: Statt Herleitung allenfalls einfach das verallgemeinerte Hookesche Gesetz erwähnen: https://de.wikipedia.org/wiki/Hookesches_Gesetz#Verallgemeinertes_hookesches_Gesetz

% NOTE: Idee: "Wir reden zwar von dem hier, aber können dies auch auf z.B. das elektrische- / magnetische-Medium anwenden."

% TODO: Wellengleichung, Referenz auf 9.2 Wellengleichung
% \subsection{Wellengleichung}
% \mytodo{Auf 9.2 Wellengleichung referenzieren und darauf aufbauen. Zweidimensionalen Fall erläutern (herleiten?)}

% \mytodo{Verbindung von Hookeschem Gesetz zur Wellengleichung machen. }
% \mytodo{Wellengleichung}


% TODO: Bezug zur Simulation
% \mynote{Diskretisierte Wellengleichung als Bezug zur Simulation(?)}

% TODO: Grafik aus Simulation -> Besonders zeigen, dass z.B. Wellen mehrerer Quellen nur lokal interferieren, den weiteren Verlauf aber nicht stören (Zwei Lichtwellen)
% TODO Teil 1:
% - Was ist Linearität?
% - Superposition
% - Grenze der Linearität: Schwinger-Limit
%
% teil2.tex -- Beispiel-File für teil2 
%
% (c) 2020 Prof Dr Andreas Müller, Hochschule Rapperswil
%
% !TEX root = ../../buch.tex
% !TEX encoding = UTF-8
%
\section{Nichtlineares Medium
\label{particles:section:nichtlinear}}
\kopfrechts{Nichtlineares Medium}
Entgegen des linearen Mediums wobei der Proportionalitätsfaktor nicht von der wirkenden Kraft abhängig war, 
so wird dies bei der Nichtlinearität nun zum Problem.
Dies bedeutet also, dass sich das Hookesche Gesetz von 
\[
    \Delta l
    = 
    \frac{F}{D}
\]
zu
\[
    \Delta l
    = 
    \frac{F}{D(F)}
\]
abändert. 
Wie dieser Proportionalitätsfaktor genau von der Kraft abhängt, 
ist je nach Medium unterschiedlich.
Ein Beispiel davon, wird im Video \mytodo{Referenz Einfügen} von Huygens Optics gezeigt,
wobei für $D$ die Formel 
\[
    D(F)
    =
    D_0
    \cdot
    (1 + \alpha \abs{F}^n)
\]
eingesetzt wird.
Wiederholen wir nun wieder die Simulation aus Abschnitt~\ref{particles:section:lin-medium:superposition}, 
so sieht diese in aus wie in Abbildung~\mytodo{Abbildung mit nichtlinearem Medium und Referenz dazu einfügen}.
% TODO: Grafik mit gleichem Aufbau wie in "lineares Medium", aber mit nichtlinearem medium (Zwei Lichtwellen)


\subsection{Schwinger-Limit in der Praxis}
Wie bereits in Abschnitt~\ref{particles:section:lin-medium:schwinger} erwähnt, 
ist das praktische Schwinger-Limit noch nicht genug erforscht um eine eindeutige Aussage darüber zu machen.
Es gibt jedoch reichlich andere Medien---beispielsweise Glasfasern---welche bereits bei geringen Feldstärken nichtlineare Eigenschaften aufweisen.
\mynote{Hier fehlt noch ein wenig Inhalt. Bin mir aber nicht ganz sicher, was genau\ldots}


\subsection{Selbstfokussierung -- Confined Energy}
% TODO: Was für Bedingungen müssen erfüllt sein, damit dies geschieht? 
%       Kann man das überhaupt so einfach klassifizieren?
%       Allenfalls Analogie zu Schwingkreisen herstellen?
Die Interaktion von den beiden Lichtstrahlen in Abbildung~\mytodo{Referenz auf Abbildung nichtlinearer Lichtstrahlen}
ist bereits interessant, jedoch wird es noch viel spannender, 
wenn statt zwei Strahlenquellen, eine punktuelle Quelle eingefügt wird.
Passt man nun die Parameter des nichtlinearen Teils, sowie die der Quelle an, 
so verschwinden die emittierten Wellen nicht einfach, 
sondern sie interagieren so, dass sie sich selbst fokussieren.
Betrachtet man nun die Kräfte, die im Feld wirken, 
so scheint es, als ob sich ein Teil vom Rest abgrenzt.
\mytodo{Grafik von "Partikel" einfügen.}
% TODO: Grafik von "Partikel" einfügen

Doch was geschieht hier genau? 
\mynote{Analogie zu Schwingkreis herstellen.}
\mynote{Hat das Medium eine Eigenfrequenz? Falls ja, ist diese ausschlaggebend, ob sich die Energie fängt?}

 % TODO: Was passiert bei Nichtlinearitäten
% TODO Teil 2:
% - Definition, Beispiele
% - Schwinger-Limit in der Praxis (?)
% - Selbstfokussierung / Confined Energy
% - Soliton
%
% einleitung.tex -- Beispiel-File für die Einleitung
%
% (c) 2020 Prof Dr Andreas Müller, Hochschule Rapperswil
%
% !TEX root = ../../buch.tex
% !TEX encoding = UTF-8
%
\section{Simulationsprogramm\label{particles:section:simulation}}
\kopfrechts{Simulationsprogramm}

Für die Simulation wurde eine von \textbf{insert name} angepasste Version des Open-Source Wellensimulators WaveSimulator2D~\cite{repo:wavesim2d} verwendet. % TODO: Insert name and cite
Dieser simuliert eine Szene, in dem er die Wellengleichung
\[
    u_{tt} = c^2 \Delta u \label{particles:eq:wellen}
\] 
für einen diskreten Zeitschritt simuliert und das resultierende Feld $u$ anzeigt.

\subsection{Szenendefinition}
Der Simulator verwendet als Eingang ein RGB-Bild.
Den drei Farbkanälen wird jeweils eine Funktion zugeteilt.
Der rote Kanal bestimmt den Brechungsindex des jeweiligen pixels, während über den blauen Kanal die Dämpfung bestimmt werden kann.
Über den grünen Kanal stellt im unabgeänderten Simulator Quellen dar, wobei der Farbwert die Frequenz der Quelle darstellt.
Damit die in der Simulation enthaltene Energie nicht fortlaufend ansteigt, wurde die Funktion des grünen Kanals durch \textbf{insert name} angepasst. % TODO: Echter Name bon Huygens einfügen
Er diktiert nun das Spannungsfeld am Anfang der Simulation. % CHECK: ist es die Spannung oder die Deformation?
% TODO: Beispielszene

\subsection{Nichtlinearität}
Die nichtlinearität des Mediums wird durch die Formel
\[
    \text{stress} = a \cdot \text{strain} \cdot e^{b \cdot \text{strain}} % TODO: Tatsächliche Formel
\]
gegeben, wobei das nichtlineare Verhalten durch die Parameter $a$ und $b$ angepasst werden kann.
Einige Beispiele der daraus resultierenden Spannungs-Deformations-Kurven sind in Abb.~\ref{particles:abb:nonlin} gezeigt. % TODO: Abbildung

\subsection{Simulation}
Die Simulation ist trotz relativ tiefer Auflösung nicht trivial. 
Der WaveSimulator2D zieht dabei nutzen aus \textbf{OpenGL} um die Simulation auf der Grafikkarte laufen zu lassen. % CHECK: Ist es OpenGL

Zum Lösen der Wellengleichung~\ref{particles:eq:wellen} wird... % TODO: Funktion der Simulation beschreiben? % TODO: Wie funktioniert die Simulation?
% TODO Teil 3:
% - Wellengleichung
% - Diskretisierung
% - Verbindung zur Simulation (?)
%
% teil3.tex -- Beispiel-File für Teil 3
%
% (c) 2020 Prof Dr Andreas Müller, Hochschule Rapperswil
%
% !TEX root = ../../buch.tex
% !TEX encoding = UTF-8
%
\section{Fazit\label{particles:section:fazit}}
\kopfrechts{Fazit}
Sed ut perspiciatis unde omnis iste natus error sit voluptatem
accusantium doloremque laudantium, totam rem aperiam, eaque ipsa
quae ab illo inventore veritatis et quasi architecto beatae vitae
dicta sunt explicabo. Nemo enim ipsam voluptatem quia voluptas sit
aspernatur aut odit aut fugit, sed quia consequuntur magni dolores
eos qui ratione voluptatem sequi nesciunt. Neque porro quisquam
est, qui dolorem ipsum quia dolor sit amet, consectetur, adipisci
velit, sed quia non numquam eius modi tempora incidunt ut labore
et dolore magnam aliquam quaerat voluptatem. Ut enim ad minima
veniam, quis nostrum exercitationem ullam corporis suscipit laboriosam,
nisi ut aliquid ex ea commodi consequatur? Quis autem vel eum iure
reprehenderit qui in ea voluptate velit esse quam nihil molestiae
consequatur, vel illum qui dolorem eum fugiat quo voluptas nulla
pariatur?

\subsection{De finibus bonorum et malorum
\label{particles:subsection:malorum}}
At vero eos et accusamus et iusto odio dignissimos ducimus qui
blanditiis praesentium voluptatum deleniti atque corrupti quos
dolores et quas molestias excepturi sint occaecati cupiditate non
provident, similique sunt in culpa qui officia deserunt mollitia
animi, id est laborum et dolorum fuga. Et harum quidem rerum facilis
est et expedita distinctio. Nam libero tempore, cum soluta nobis
est eligendi optio cumque nihil impedit quo minus id quod maxime
placeat facere possimus, omnis voluptas assumenda est, omnis dolor
repellendus. Temporibus autem quibusdam et aut officiis debitis aut
rerum necessitatibus saepe eveniet ut et voluptates repudiandae
sint et molestiae non recusandae. Itaque earum rerum hic tenetur a
sapiente delectus, ut aut reiciendis voluptatibus maiores alias
consequatur aut perferendis doloribus asperiores repellat.


 % TODO: Fazit: Was könnten wir hier beobachtet haben?
\printbibliography[heading=subbibliography]
\end{refsection}
