%
% teil1.tex -- Beispiel-File für das Paper
%
% (c) 2020 Prof Dr Andreas Müller, Hochschule Rapperswil
%
% !TEX root = ../../buch.tex
% !TEX encoding = UTF-8
%
\section{Operatoren erster Ordnung
\label{diffortho:section:ersteordnung}}
\kopfrechts{Operatoren erster Ordnung}
In diesem Abschnitt gehen wir von einer Koordinatentransformation
\[
y^k
=
y^k(x^1,\dots,x^n)
\]
aus, die orthogonale Koordinaten $(x^1,\dots,x^n)$ in die kartesischen
Koordinaten $(y^1,\dots,y^n)$ transformiert.

Die partiellen Ableitungsoperatoren
\begin{align*}
\frac{\partial f}{\partial x^i}
&=
\sum_{k=1}^n
\frac{\partial y^k}{\partial x^i}\frac{\partial f}{\partial y^k}
\\
\Rightarrow
\qquad
\frac{\partial}{\partial x^i}
&=
\sum_{k=1}^n
\frac{\partial y^k}{\partial x^i}\frac{\partial}{\partial y^k}
=
\sum_{k=1}^n
J^k\mathstrut_i \frac{\partial}{\partial y^k}
\end{align*}
werden mit der Jacobi-Matrix transformiert.
Im nächsten Abschnitt möchten wir den Laplace-Operator in den
$x^i$-Koordinaten ausdrücken.
Dazu müssen wir die Ableitungen $\partial/\partial y^k$ durch die
$\partial/\partial x^i$ ausdrücken.
Die Transformation muss daher in der umgekehrten Richtung als
\begin{align}
\frac{\partial}{\partial y^k}
&=
\sum_{i=1}^n
{\tilde J}^i\mathstrut_k
\frac{\partial}{\partial x^i},
\label{diffortho:eqn:1ordnung}
\end{align}
wobei $\tilde{J}^i\mathstrut_k$ die Einträge der Inversen der
Jacobi-Matrix sind.

In der klassischen Vektorgeometrie werden die Operatoren der
Divergenz, des Gradienten und der Rotation definiert.
Die Formel \eqref{diffortho:eqn:1ordnung} ermöglicht, diese
Operatoren in beliebigen Koordinaten zu schreiben.
Allerdings hat das Kapitel~\ref{chapter:pformen} gezeigt, dass
diese Definitionen vom Skalarprodukt abhängig sind und
man daher besser mit $p$-Formen arbeiten sollte.
Formal lassen sich die Formeln für diese Operatoren durch
Einsetzen der Operatoren \eqref{diffortho:eqn:1ordnung} in die
definierenden Formeln in kartesischen Koordinaten einsetzen.
Da dies nur von begrenztem Nutzen ist, wird statt der ausführlichen
Rechnung auf die Resulate in \cite{diffortho:ref:wikipedia} verwiesen.

Für später berechnen wir das Produkt
\begin{align*}
\sum_{k=1}^n \tilde{J}^i\mathstrut_k \tilde{J}^j\mathstrut_k
&=
(\tilde{J}\tilde{J}^t)_{ij}.
\end{align*}
Da $\tilde{J}$ die Inverse von $J$ ist, kann man 
\[
\tilde{J}\tilde{J}^t
=
J^{-1}(J^{-1})^t
=
J^{-1}(J^t)^{-1}
=
(J^tJ)^{-1}
\]
rechnen.
Auch diese Matrix ist eine Diagonalmatrix und
wir können schliessen, dass
\begin{equation}
\sum_{k=1}^n \tilde{J}^i\mathstrut_k \tilde{J}^j\mathstrut_k
=
\frac{\delta_{ij}}{h_i(x)^2}
\label{diffortho:eqn:Jij}
\end{equation}
ist.

