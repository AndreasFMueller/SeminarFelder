%
% zerlegung.tex -- Zerlegung der Einheit
%
% (c) 2021 Prof Dr Andreas Müller, OST Ostschweizer Fachhochschule
%
\documentclass[tikz]{standalone}
\usepackage{amsmath}
\usepackage{times}
\usepackage{txfonts}
\usepackage{pgfplots}
\usepackage{csvsimple}
\usetikzlibrary{arrows,intersections,math}
\begin{document}
\def\skala{1}
\definecolor{darkred}{rgb}{0.8,0,0}
\definecolor{farbe1}{rgb}{0.4,0.8,0.0}
\definecolor{farbe2}{rgb}{0.0,0.8,0.4}
\definecolor{farbe3}{rgb}{0.0,0.4,0.8}
\definecolor{farbe4}{rgb}{0.4,0.0,0.8}
\definecolor{farbe5}{rgb}{0.8,0.0,0.4}
\definecolor{farbe6}{rgb}{0.8,0.4,0.0}
\begin{tikzpicture}[>=latex,thick,scale=\skala]
\def\dy{1.2}
\def\dx{1}
\def\aone{-3}
\def\bone{3}
\def\atwo{9}
\def\btwo{15}
\def\athree{2}
\def\bthree{6}
\def\afour{3}
\def\bfour{9}
\def\afive{5}
\def\bfive{10}
\def\asix{8}
\def\bsix{11}
\def\psione{
	({0.0000*\dx},{1.0000*\dy})
	-- ({0.0100*\dx},{1.0000*\dy})
	-- ({0.0200*\dx},{1.0000*\dy})
	-- ({0.0300*\dx},{1.0000*\dy})
	-- ({0.0400*\dx},{1.0000*\dy})
	-- ({0.0500*\dx},{1.0000*\dy})
	-- ({0.0601*\dx},{1.0000*\dy})
	-- ({0.0701*\dx},{1.0000*\dy})
	-- ({0.0801*\dx},{1.0000*\dy})
	-- ({0.0901*\dx},{1.0000*\dy})
	-- ({0.1001*\dx},{1.0000*\dy})
	-- ({0.1101*\dx},{1.0000*\dy})
	-- ({0.1201*\dx},{1.0000*\dy})
	-- ({0.1301*\dx},{1.0000*\dy})
	-- ({0.1401*\dx},{1.0000*\dy})
	-- ({0.1501*\dx},{1.0000*\dy})
	-- ({0.1601*\dx},{1.0000*\dy})
	-- ({0.1701*\dx},{1.0000*\dy})
	-- ({0.1802*\dx},{1.0000*\dy})
	-- ({0.1902*\dx},{1.0000*\dy})
	-- ({0.2002*\dx},{1.0000*\dy})
	-- ({0.2102*\dx},{1.0000*\dy})
	-- ({0.2202*\dx},{1.0000*\dy})
	-- ({0.2302*\dx},{1.0000*\dy})
	-- ({0.2402*\dx},{1.0000*\dy})
	-- ({0.2502*\dx},{1.0000*\dy})
	-- ({0.2602*\dx},{1.0000*\dy})
	-- ({0.2702*\dx},{1.0000*\dy})
	-- ({0.2802*\dx},{1.0000*\dy})
	-- ({0.2902*\dx},{1.0000*\dy})
	-- ({0.3003*\dx},{1.0000*\dy})
	-- ({0.3103*\dx},{1.0000*\dy})
	-- ({0.3203*\dx},{1.0000*\dy})
	-- ({0.3303*\dx},{1.0000*\dy})
	-- ({0.3403*\dx},{1.0000*\dy})
	-- ({0.3503*\dx},{1.0000*\dy})
	-- ({0.3603*\dx},{1.0000*\dy})
	-- ({0.3703*\dx},{1.0000*\dy})
	-- ({0.3803*\dx},{1.0000*\dy})
	-- ({0.3903*\dx},{1.0000*\dy})
	-- ({0.4003*\dx},{1.0000*\dy})
	-- ({0.4103*\dx},{1.0000*\dy})
	-- ({0.4204*\dx},{1.0000*\dy})
	-- ({0.4304*\dx},{1.0000*\dy})
	-- ({0.4404*\dx},{1.0000*\dy})
	-- ({0.4504*\dx},{1.0000*\dy})
	-- ({0.4604*\dx},{1.0000*\dy})
	-- ({0.4704*\dx},{1.0000*\dy})
	-- ({0.4804*\dx},{1.0000*\dy})
	-- ({0.4904*\dx},{1.0000*\dy})
	-- ({0.5004*\dx},{1.0000*\dy})
	-- ({0.5104*\dx},{1.0000*\dy})
	-- ({0.5204*\dx},{1.0000*\dy})
	-- ({0.5304*\dx},{1.0000*\dy})
	-- ({0.5405*\dx},{1.0000*\dy})
	-- ({0.5505*\dx},{1.0000*\dy})
	-- ({0.5605*\dx},{1.0000*\dy})
	-- ({0.5705*\dx},{1.0000*\dy})
	-- ({0.5805*\dx},{1.0000*\dy})
	-- ({0.5905*\dx},{1.0000*\dy})
	-- ({0.6005*\dx},{1.0000*\dy})
	-- ({0.6105*\dx},{1.0000*\dy})
	-- ({0.6205*\dx},{1.0000*\dy})
	-- ({0.6305*\dx},{1.0000*\dy})
	-- ({0.6405*\dx},{1.0000*\dy})
	-- ({0.6505*\dx},{1.0000*\dy})
	-- ({0.6606*\dx},{1.0000*\dy})
	-- ({0.6706*\dx},{1.0000*\dy})
	-- ({0.6806*\dx},{1.0000*\dy})
	-- ({0.6906*\dx},{1.0000*\dy})
	-- ({0.7006*\dx},{1.0000*\dy})
	-- ({0.7106*\dx},{1.0000*\dy})
	-- ({0.7206*\dx},{1.0000*\dy})
	-- ({0.7306*\dx},{1.0000*\dy})
	-- ({0.7406*\dx},{1.0000*\dy})
	-- ({0.7506*\dx},{1.0000*\dy})
	-- ({0.7606*\dx},{1.0000*\dy})
	-- ({0.7706*\dx},{1.0000*\dy})
	-- ({0.7807*\dx},{1.0000*\dy})
	-- ({0.7907*\dx},{1.0000*\dy})
	-- ({0.8007*\dx},{1.0000*\dy})
	-- ({0.8107*\dx},{1.0000*\dy})
	-- ({0.8207*\dx},{1.0000*\dy})
	-- ({0.8307*\dx},{1.0000*\dy})
	-- ({0.8407*\dx},{1.0000*\dy})
	-- ({0.8507*\dx},{1.0000*\dy})
	-- ({0.8607*\dx},{1.0000*\dy})
	-- ({0.8707*\dx},{1.0000*\dy})
	-- ({0.8807*\dx},{1.0000*\dy})
	-- ({0.8907*\dx},{1.0000*\dy})
	-- ({0.9008*\dx},{1.0000*\dy})
	-- ({0.9108*\dx},{1.0000*\dy})
	-- ({0.9208*\dx},{1.0000*\dy})
	-- ({0.9308*\dx},{1.0000*\dy})
	-- ({0.9408*\dx},{1.0000*\dy})
	-- ({0.9508*\dx},{1.0000*\dy})
	-- ({0.9608*\dx},{1.0000*\dy})
	-- ({0.9708*\dx},{1.0000*\dy})
	-- ({0.9808*\dx},{1.0000*\dy})
	-- ({0.9908*\dx},{1.0000*\dy})
	-- ({1.0008*\dx},{1.0000*\dy})
	-- ({1.0108*\dx},{1.0000*\dy})
	-- ({1.0209*\dx},{1.0000*\dy})
	-- ({1.0309*\dx},{1.0000*\dy})
	-- ({1.0409*\dx},{1.0000*\dy})
	-- ({1.0509*\dx},{1.0000*\dy})
	-- ({1.0609*\dx},{1.0000*\dy})
	-- ({1.0709*\dx},{1.0000*\dy})
	-- ({1.0809*\dx},{1.0000*\dy})
	-- ({1.0909*\dx},{1.0000*\dy})
	-- ({1.1009*\dx},{1.0000*\dy})
	-- ({1.1109*\dx},{1.0000*\dy})
	-- ({1.1209*\dx},{1.0000*\dy})
	-- ({1.1309*\dx},{1.0000*\dy})
	-- ({1.1410*\dx},{1.0000*\dy})
	-- ({1.1510*\dx},{1.0000*\dy})
	-- ({1.1610*\dx},{1.0000*\dy})
	-- ({1.1710*\dx},{1.0000*\dy})
	-- ({1.1810*\dx},{1.0000*\dy})
	-- ({1.1910*\dx},{1.0000*\dy})
	-- ({1.2010*\dx},{1.0000*\dy})
	-- ({1.2110*\dx},{1.0000*\dy})
	-- ({1.2210*\dx},{1.0000*\dy})
	-- ({1.2310*\dx},{1.0000*\dy})
	-- ({1.2410*\dx},{1.0000*\dy})
	-- ({1.2510*\dx},{1.0000*\dy})
	-- ({1.2611*\dx},{1.0000*\dy})
	-- ({1.2711*\dx},{1.0000*\dy})
	-- ({1.2811*\dx},{1.0000*\dy})
	-- ({1.2911*\dx},{1.0000*\dy})
	-- ({1.3011*\dx},{1.0000*\dy})
	-- ({1.3111*\dx},{1.0000*\dy})
	-- ({1.3211*\dx},{1.0000*\dy})
	-- ({1.3311*\dx},{1.0000*\dy})
	-- ({1.3411*\dx},{1.0000*\dy})
	-- ({1.3511*\dx},{1.0000*\dy})
	-- ({1.3611*\dx},{1.0000*\dy})
	-- ({1.3711*\dx},{1.0000*\dy})
	-- ({1.3812*\dx},{1.0000*\dy})
	-- ({1.3912*\dx},{1.0000*\dy})
	-- ({1.4012*\dx},{1.0000*\dy})
	-- ({1.4112*\dx},{1.0000*\dy})
	-- ({1.4212*\dx},{1.0000*\dy})
	-- ({1.4312*\dx},{1.0000*\dy})
	-- ({1.4412*\dx},{1.0000*\dy})
	-- ({1.4512*\dx},{1.0000*\dy})
	-- ({1.4612*\dx},{1.0000*\dy})
	-- ({1.4712*\dx},{1.0000*\dy})
	-- ({1.4812*\dx},{1.0000*\dy})
	-- ({1.4912*\dx},{1.0000*\dy})
	-- ({1.5013*\dx},{1.0000*\dy})
	-- ({1.5113*\dx},{1.0000*\dy})
	-- ({1.5213*\dx},{1.0000*\dy})
	-- ({1.5313*\dx},{1.0000*\dy})
	-- ({1.5413*\dx},{1.0000*\dy})
	-- ({1.5513*\dx},{1.0000*\dy})
	-- ({1.5613*\dx},{1.0000*\dy})
	-- ({1.5713*\dx},{1.0000*\dy})
	-- ({1.5813*\dx},{1.0000*\dy})
	-- ({1.5913*\dx},{1.0000*\dy})
	-- ({1.6013*\dx},{1.0000*\dy})
	-- ({1.6113*\dx},{1.0000*\dy})
	-- ({1.6214*\dx},{1.0000*\dy})
	-- ({1.6314*\dx},{1.0000*\dy})
	-- ({1.6414*\dx},{1.0000*\dy})
	-- ({1.6514*\dx},{1.0000*\dy})
	-- ({1.6614*\dx},{1.0000*\dy})
	-- ({1.6714*\dx},{1.0000*\dy})
	-- ({1.6814*\dx},{1.0000*\dy})
	-- ({1.6914*\dx},{1.0000*\dy})
	-- ({1.7014*\dx},{1.0000*\dy})
	-- ({1.7114*\dx},{1.0000*\dy})
	-- ({1.7214*\dx},{1.0000*\dy})
	-- ({1.7314*\dx},{1.0000*\dy})
	-- ({1.7415*\dx},{1.0000*\dy})
	-- ({1.7515*\dx},{1.0000*\dy})
	-- ({1.7615*\dx},{1.0000*\dy})
	-- ({1.7715*\dx},{1.0000*\dy})
	-- ({1.7815*\dx},{1.0000*\dy})
	-- ({1.7915*\dx},{1.0000*\dy})
	-- ({1.8015*\dx},{1.0000*\dy})
	-- ({1.8115*\dx},{1.0000*\dy})
	-- ({1.8215*\dx},{1.0000*\dy})
	-- ({1.8315*\dx},{1.0000*\dy})
	-- ({1.8415*\dx},{1.0000*\dy})
	-- ({1.8515*\dx},{1.0000*\dy})
	-- ({1.8616*\dx},{1.0000*\dy})
	-- ({1.8716*\dx},{1.0000*\dy})
	-- ({1.8816*\dx},{1.0000*\dy})
	-- ({1.8916*\dx},{1.0000*\dy})
	-- ({1.9016*\dx},{1.0000*\dy})
	-- ({1.9116*\dx},{1.0000*\dy})
	-- ({1.9216*\dx},{1.0000*\dy})
	-- ({1.9316*\dx},{1.0000*\dy})
	-- ({1.9416*\dx},{1.0000*\dy})
	-- ({1.9516*\dx},{1.0000*\dy})
	-- ({1.9616*\dx},{1.0000*\dy})
	-- ({1.9716*\dx},{1.0000*\dy})
	-- ({1.9817*\dx},{1.0000*\dy})
	-- ({1.9917*\dx},{1.0000*\dy})
	-- ({2.0017*\dx},{1.0000*\dy})
	-- ({2.0117*\dx},{0.9997*\dy})
	-- ({2.0217*\dx},{0.9988*\dy})
	-- ({2.0317*\dx},{0.9974*\dy})
	-- ({2.0417*\dx},{0.9954*\dy})
	-- ({2.0517*\dx},{0.9928*\dy})
	-- ({2.0617*\dx},{0.9896*\dy})
	-- ({2.0717*\dx},{0.9857*\dy})
	-- ({2.0817*\dx},{0.9811*\dy})
	-- ({2.0917*\dx},{0.9759*\dy})
	-- ({2.1018*\dx},{0.9699*\dy})
	-- ({2.1118*\dx},{0.9632*\dy})
	-- ({2.1218*\dx},{0.9558*\dy})
	-- ({2.1318*\dx},{0.9475*\dy})
	-- ({2.1418*\dx},{0.9386*\dy})
	-- ({2.1518*\dx},{0.9288*\dy})
	-- ({2.1618*\dx},{0.9183*\dy})
	-- ({2.1718*\dx},{0.9070*\dy})
	-- ({2.1818*\dx},{0.8950*\dy})
	-- ({2.1918*\dx},{0.8822*\dy})
	-- ({2.2018*\dx},{0.8686*\dy})
	-- ({2.2118*\dx},{0.8543*\dy})
	-- ({2.2219*\dx},{0.8393*\dy})
	-- ({2.2319*\dx},{0.8237*\dy})
	-- ({2.2419*\dx},{0.8074*\dy})
	-- ({2.2519*\dx},{0.7905*\dy})
	-- ({2.2619*\dx},{0.7730*\dy})
	-- ({2.2719*\dx},{0.7550*\dy})
	-- ({2.2819*\dx},{0.7366*\dy})
	-- ({2.2919*\dx},{0.7177*\dy})
	-- ({2.3019*\dx},{0.6984*\dy})
	-- ({2.3119*\dx},{0.6787*\dy})
	-- ({2.3219*\dx},{0.6588*\dy})
	-- ({2.3319*\dx},{0.6387*\dy})
	-- ({2.3420*\dx},{0.6184*\dy})
	-- ({2.3520*\dx},{0.5980*\dy})
	-- ({2.3620*\dx},{0.5775*\dy})
	-- ({2.3720*\dx},{0.5570*\dy})
	-- ({2.3820*\dx},{0.5365*\dy})
	-- ({2.3920*\dx},{0.5162*\dy})
	-- ({2.4020*\dx},{0.4960*\dy})
	-- ({2.4120*\dx},{0.4759*\dy})
	-- ({2.4220*\dx},{0.4562*\dy})
	-- ({2.4320*\dx},{0.4366*\dy})
	-- ({2.4420*\dx},{0.4174*\dy})
	-- ({2.4520*\dx},{0.3986*\dy})
	-- ({2.4621*\dx},{0.3801*\dy})
	-- ({2.4721*\dx},{0.3620*\dy})
	-- ({2.4821*\dx},{0.3443*\dy})
	-- ({2.4921*\dx},{0.3271*\dy})
	-- ({2.5021*\dx},{0.3104*\dy})
	-- ({2.5121*\dx},{0.2942*\dy})
	-- ({2.5221*\dx},{0.2784*\dy})
	-- ({2.5321*\dx},{0.2632*\dy})
	-- ({2.5421*\dx},{0.2484*\dy})
	-- ({2.5521*\dx},{0.2342*\dy})
	-- ({2.5621*\dx},{0.2205*\dy})
	-- ({2.5721*\dx},{0.2073*\dy})
	-- ({2.5822*\dx},{0.1947*\dy})
	-- ({2.5922*\dx},{0.1825*\dy})
	-- ({2.6022*\dx},{0.1709*\dy})
	-- ({2.6122*\dx},{0.1598*\dy})
	-- ({2.6222*\dx},{0.1491*\dy})
	-- ({2.6322*\dx},{0.1390*\dy})
	-- ({2.6422*\dx},{0.1293*\dy})
	-- ({2.6522*\dx},{0.1201*\dy})
	-- ({2.6622*\dx},{0.1114*\dy})
	-- ({2.6722*\dx},{0.1031*\dy})
	-- ({2.6822*\dx},{0.0952*\dy})
	-- ({2.6922*\dx},{0.0878*\dy})
	-- ({2.7023*\dx},{0.0807*\dy})
	-- ({2.7123*\dx},{0.0741*\dy})
	-- ({2.7223*\dx},{0.0678*\dy})
	-- ({2.7323*\dx},{0.0619*\dy})
	-- ({2.7423*\dx},{0.0564*\dy})
	-- ({2.7523*\dx},{0.0511*\dy})
	-- ({2.7623*\dx},{0.0463*\dy})
	-- ({2.7723*\dx},{0.0417*\dy})
	-- ({2.7823*\dx},{0.0375*\dy})
	-- ({2.7923*\dx},{0.0335*\dy})
	-- ({2.8023*\dx},{0.0298*\dy})
	-- ({2.8123*\dx},{0.0264*\dy})
	-- ({2.8224*\dx},{0.0233*\dy})
	-- ({2.8324*\dx},{0.0204*\dy})
	-- ({2.8424*\dx},{0.0177*\dy})
	-- ({2.8524*\dx},{0.0152*\dy})
	-- ({2.8624*\dx},{0.0130*\dy})
	-- ({2.8724*\dx},{0.0110*\dy})
	-- ({2.8824*\dx},{0.0092*\dy})
	-- ({2.8924*\dx},{0.0076*\dy})
	-- ({2.9024*\dx},{0.0061*\dy})
	-- ({2.9124*\dx},{0.0048*\dy})
	-- ({2.9224*\dx},{0.0037*\dy})
	-- ({2.9324*\dx},{0.0028*\dy})
	-- ({2.9425*\dx},{0.0020*\dy})
	-- ({2.9525*\dx},{0.0013*\dy})
	-- ({2.9625*\dx},{0.0008*\dy})
	-- ({2.9725*\dx},{0.0004*\dy})
	-- ({2.9825*\dx},{0.0002*\dy})
	-- ({2.9925*\dx},{0.0000*\dy})
	-- ({3.0025*\dx},{0.0000*\dy})
	-- ({3.0125*\dx},{0.0000*\dy})
	-- ({3.0225*\dx},{0.0000*\dy})
	-- ({3.0325*\dx},{0.0000*\dy})
	-- ({3.0425*\dx},{0.0000*\dy})
	-- ({3.0525*\dx},{0.0000*\dy})
	-- ({3.0626*\dx},{0.0000*\dy})
	-- ({3.0726*\dx},{0.0000*\dy})
	-- ({3.0826*\dx},{0.0000*\dy})
	-- ({3.0926*\dx},{0.0000*\dy})
	-- ({3.1026*\dx},{0.0000*\dy})
	-- ({3.1126*\dx},{0.0000*\dy})
	-- ({3.1226*\dx},{0.0000*\dy})
	-- ({3.1326*\dx},{0.0000*\dy})
	-- ({3.1426*\dx},{0.0000*\dy})
	-- ({3.1526*\dx},{0.0000*\dy})
	-- ({3.1626*\dx},{0.0000*\dy})
	-- ({3.1726*\dx},{0.0000*\dy})
	-- ({3.1827*\dx},{0.0000*\dy})
	-- ({3.1927*\dx},{0.0000*\dy})
	-- ({3.2027*\dx},{0.0000*\dy})
	-- ({3.2127*\dx},{0.0000*\dy})
	-- ({3.2227*\dx},{0.0000*\dy})
	-- ({3.2327*\dx},{0.0000*\dy})
	-- ({3.2427*\dx},{0.0000*\dy})
	-- ({3.2527*\dx},{0.0000*\dy})
	-- ({3.2627*\dx},{0.0000*\dy})
	-- ({3.2727*\dx},{0.0000*\dy})
	-- ({3.2827*\dx},{0.0000*\dy})
	-- ({3.2927*\dx},{0.0000*\dy})
	-- ({3.3028*\dx},{0.0000*\dy})
	-- ({3.3128*\dx},{0.0000*\dy})
	-- ({3.3228*\dx},{0.0000*\dy})
	-- ({3.3328*\dx},{0.0000*\dy})
	-- ({3.3428*\dx},{0.0000*\dy})
	-- ({3.3528*\dx},{0.0000*\dy})
	-- ({3.3628*\dx},{0.0000*\dy})
	-- ({3.3728*\dx},{0.0000*\dy})
	-- ({3.3828*\dx},{0.0000*\dy})
	-- ({3.3928*\dx},{0.0000*\dy})
	-- ({3.4028*\dx},{0.0000*\dy})
	-- ({3.4128*\dx},{0.0000*\dy})
	-- ({3.4229*\dx},{0.0000*\dy})
	-- ({3.4329*\dx},{0.0000*\dy})
	-- ({3.4429*\dx},{0.0000*\dy})
	-- ({3.4529*\dx},{0.0000*\dy})
	-- ({3.4629*\dx},{0.0000*\dy})
	-- ({3.4729*\dx},{0.0000*\dy})
	-- ({3.4829*\dx},{0.0000*\dy})
	-- ({3.4929*\dx},{0.0000*\dy})
	-- ({3.5029*\dx},{0.0000*\dy})
	-- ({3.5129*\dx},{0.0000*\dy})
	-- ({3.5229*\dx},{0.0000*\dy})
	-- ({3.5329*\dx},{0.0000*\dy})
	-- ({3.5430*\dx},{0.0000*\dy})
	-- ({3.5530*\dx},{0.0000*\dy})
	-- ({3.5630*\dx},{0.0000*\dy})
	-- ({3.5730*\dx},{0.0000*\dy})
	-- ({3.5830*\dx},{0.0000*\dy})
	-- ({3.5930*\dx},{0.0000*\dy})
	-- ({3.6030*\dx},{0.0000*\dy})
	-- ({3.6130*\dx},{0.0000*\dy})
	-- ({3.6230*\dx},{0.0000*\dy})
	-- ({3.6330*\dx},{0.0000*\dy})
	-- ({3.6430*\dx},{0.0000*\dy})
	-- ({3.6530*\dx},{0.0000*\dy})
	-- ({3.6631*\dx},{0.0000*\dy})
	-- ({3.6731*\dx},{0.0000*\dy})
	-- ({3.6831*\dx},{0.0000*\dy})
	-- ({3.6931*\dx},{0.0000*\dy})
	-- ({3.7031*\dx},{0.0000*\dy})
	-- ({3.7131*\dx},{0.0000*\dy})
	-- ({3.7231*\dx},{0.0000*\dy})
	-- ({3.7331*\dx},{0.0000*\dy})
	-- ({3.7431*\dx},{0.0000*\dy})
	-- ({3.7531*\dx},{0.0000*\dy})
	-- ({3.7631*\dx},{0.0000*\dy})
	-- ({3.7731*\dx},{0.0000*\dy})
	-- ({3.7832*\dx},{0.0000*\dy})
	-- ({3.7932*\dx},{0.0000*\dy})
	-- ({3.8032*\dx},{0.0000*\dy})
	-- ({3.8132*\dx},{0.0000*\dy})
	-- ({3.8232*\dx},{0.0000*\dy})
	-- ({3.8332*\dx},{0.0000*\dy})
	-- ({3.8432*\dx},{0.0000*\dy})
	-- ({3.8532*\dx},{0.0000*\dy})
	-- ({3.8632*\dx},{0.0000*\dy})
	-- ({3.8732*\dx},{0.0000*\dy})
	-- ({3.8832*\dx},{0.0000*\dy})
	-- ({3.8932*\dx},{0.0000*\dy})
	-- ({3.9033*\dx},{0.0000*\dy})
	-- ({3.9133*\dx},{0.0000*\dy})
	-- ({3.9233*\dx},{0.0000*\dy})
	-- ({3.9333*\dx},{0.0000*\dy})
	-- ({3.9433*\dx},{0.0000*\dy})
	-- ({3.9533*\dx},{0.0000*\dy})
	-- ({3.9633*\dx},{0.0000*\dy})
	-- ({3.9733*\dx},{0.0000*\dy})
	-- ({3.9833*\dx},{0.0000*\dy})
	-- ({3.9933*\dx},{0.0000*\dy})
	-- ({4.0033*\dx},{0.0000*\dy})
	-- ({4.0133*\dx},{0.0000*\dy})
	-- ({4.0234*\dx},{0.0000*\dy})
	-- ({4.0334*\dx},{0.0000*\dy})
	-- ({4.0434*\dx},{0.0000*\dy})
	-- ({4.0534*\dx},{0.0000*\dy})
	-- ({4.0634*\dx},{0.0000*\dy})
	-- ({4.0734*\dx},{0.0000*\dy})
	-- ({4.0834*\dx},{0.0000*\dy})
	-- ({4.0934*\dx},{0.0000*\dy})
	-- ({4.1034*\dx},{0.0000*\dy})
	-- ({4.1134*\dx},{0.0000*\dy})
	-- ({4.1234*\dx},{0.0000*\dy})
	-- ({4.1334*\dx},{0.0000*\dy})
	-- ({4.1435*\dx},{0.0000*\dy})
	-- ({4.1535*\dx},{0.0000*\dy})
	-- ({4.1635*\dx},{0.0000*\dy})
	-- ({4.1735*\dx},{0.0000*\dy})
	-- ({4.1835*\dx},{0.0000*\dy})
	-- ({4.1935*\dx},{0.0000*\dy})
	-- ({4.2035*\dx},{0.0000*\dy})
	-- ({4.2135*\dx},{0.0000*\dy})
	-- ({4.2235*\dx},{0.0000*\dy})
	-- ({4.2335*\dx},{0.0000*\dy})
	-- ({4.2435*\dx},{0.0000*\dy})
	-- ({4.2535*\dx},{0.0000*\dy})
	-- ({4.2636*\dx},{0.0000*\dy})
	-- ({4.2736*\dx},{0.0000*\dy})
	-- ({4.2836*\dx},{0.0000*\dy})
	-- ({4.2936*\dx},{0.0000*\dy})
	-- ({4.3036*\dx},{0.0000*\dy})
	-- ({4.3136*\dx},{0.0000*\dy})
	-- ({4.3236*\dx},{0.0000*\dy})
	-- ({4.3336*\dx},{0.0000*\dy})
	-- ({4.3436*\dx},{0.0000*\dy})
	-- ({4.3536*\dx},{0.0000*\dy})
	-- ({4.3636*\dx},{0.0000*\dy})
	-- ({4.3736*\dx},{0.0000*\dy})
	-- ({4.3837*\dx},{0.0000*\dy})
	-- ({4.3937*\dx},{0.0000*\dy})
	-- ({4.4037*\dx},{0.0000*\dy})
	-- ({4.4137*\dx},{0.0000*\dy})
	-- ({4.4237*\dx},{0.0000*\dy})
	-- ({4.4337*\dx},{0.0000*\dy})
	-- ({4.4437*\dx},{0.0000*\dy})
	-- ({4.4537*\dx},{0.0000*\dy})
	-- ({4.4637*\dx},{0.0000*\dy})
	-- ({4.4737*\dx},{0.0000*\dy})
	-- ({4.4837*\dx},{0.0000*\dy})
	-- ({4.4937*\dx},{0.0000*\dy})
	-- ({4.5038*\dx},{0.0000*\dy})
	-- ({4.5138*\dx},{0.0000*\dy})
	-- ({4.5238*\dx},{0.0000*\dy})
	-- ({4.5338*\dx},{0.0000*\dy})
	-- ({4.5438*\dx},{0.0000*\dy})
	-- ({4.5538*\dx},{0.0000*\dy})
	-- ({4.5638*\dx},{0.0000*\dy})
	-- ({4.5738*\dx},{0.0000*\dy})
	-- ({4.5838*\dx},{0.0000*\dy})
	-- ({4.5938*\dx},{0.0000*\dy})
	-- ({4.6038*\dx},{0.0000*\dy})
	-- ({4.6138*\dx},{0.0000*\dy})
	-- ({4.6239*\dx},{0.0000*\dy})
	-- ({4.6339*\dx},{0.0000*\dy})
	-- ({4.6439*\dx},{0.0000*\dy})
	-- ({4.6539*\dx},{0.0000*\dy})
	-- ({4.6639*\dx},{0.0000*\dy})
	-- ({4.6739*\dx},{0.0000*\dy})
	-- ({4.6839*\dx},{0.0000*\dy})
	-- ({4.6939*\dx},{0.0000*\dy})
	-- ({4.7039*\dx},{0.0000*\dy})
	-- ({4.7139*\dx},{0.0000*\dy})
	-- ({4.7239*\dx},{0.0000*\dy})
	-- ({4.7339*\dx},{0.0000*\dy})
	-- ({4.7440*\dx},{0.0000*\dy})
	-- ({4.7540*\dx},{0.0000*\dy})
	-- ({4.7640*\dx},{0.0000*\dy})
	-- ({4.7740*\dx},{0.0000*\dy})
	-- ({4.7840*\dx},{0.0000*\dy})
	-- ({4.7940*\dx},{0.0000*\dy})
	-- ({4.8040*\dx},{0.0000*\dy})
	-- ({4.8140*\dx},{0.0000*\dy})
	-- ({4.8240*\dx},{0.0000*\dy})
	-- ({4.8340*\dx},{0.0000*\dy})
	-- ({4.8440*\dx},{0.0000*\dy})
	-- ({4.8540*\dx},{0.0000*\dy})
	-- ({4.8641*\dx},{0.0000*\dy})
	-- ({4.8741*\dx},{0.0000*\dy})
	-- ({4.8841*\dx},{0.0000*\dy})
	-- ({4.8941*\dx},{0.0000*\dy})
	-- ({4.9041*\dx},{0.0000*\dy})
	-- ({4.9141*\dx},{0.0000*\dy})
	-- ({4.9241*\dx},{0.0000*\dy})
	-- ({4.9341*\dx},{0.0000*\dy})
	-- ({4.9441*\dx},{0.0000*\dy})
	-- ({4.9541*\dx},{0.0000*\dy})
	-- ({4.9641*\dx},{0.0000*\dy})
	-- ({4.9741*\dx},{0.0000*\dy})
	-- ({4.9842*\dx},{0.0000*\dy})
	-- ({4.9942*\dx},{0.0000*\dy})
	-- ({5.0042*\dx},{0.0000*\dy})
	-- ({5.0142*\dx},{0.0000*\dy})
	-- ({5.0242*\dx},{0.0000*\dy})
	-- ({5.0342*\dx},{0.0000*\dy})
	-- ({5.0442*\dx},{0.0000*\dy})
	-- ({5.0542*\dx},{0.0000*\dy})
	-- ({5.0642*\dx},{0.0000*\dy})
	-- ({5.0742*\dx},{0.0000*\dy})
	-- ({5.0842*\dx},{0.0000*\dy})
	-- ({5.0942*\dx},{0.0000*\dy})
	-- ({5.1043*\dx},{0.0000*\dy})
	-- ({5.1143*\dx},{0.0000*\dy})
	-- ({5.1243*\dx},{0.0000*\dy})
	-- ({5.1343*\dx},{0.0000*\dy})
	-- ({5.1443*\dx},{0.0000*\dy})
	-- ({5.1543*\dx},{0.0000*\dy})
	-- ({5.1643*\dx},{0.0000*\dy})
	-- ({5.1743*\dx},{0.0000*\dy})
	-- ({5.1843*\dx},{0.0000*\dy})
	-- ({5.1943*\dx},{0.0000*\dy})
	-- ({5.2043*\dx},{0.0000*\dy})
	-- ({5.2143*\dx},{0.0000*\dy})
	-- ({5.2244*\dx},{0.0000*\dy})
	-- ({5.2344*\dx},{0.0000*\dy})
	-- ({5.2444*\dx},{0.0000*\dy})
	-- ({5.2544*\dx},{0.0000*\dy})
	-- ({5.2644*\dx},{0.0000*\dy})
	-- ({5.2744*\dx},{0.0000*\dy})
	-- ({5.2844*\dx},{0.0000*\dy})
	-- ({5.2944*\dx},{0.0000*\dy})
	-- ({5.3044*\dx},{0.0000*\dy})
	-- ({5.3144*\dx},{0.0000*\dy})
	-- ({5.3244*\dx},{0.0000*\dy})
	-- ({5.3344*\dx},{0.0000*\dy})
	-- ({5.3445*\dx},{0.0000*\dy})
	-- ({5.3545*\dx},{0.0000*\dy})
	-- ({5.3645*\dx},{0.0000*\dy})
	-- ({5.3745*\dx},{0.0000*\dy})
	-- ({5.3845*\dx},{0.0000*\dy})
	-- ({5.3945*\dx},{0.0000*\dy})
	-- ({5.4045*\dx},{0.0000*\dy})
	-- ({5.4145*\dx},{0.0000*\dy})
	-- ({5.4245*\dx},{0.0000*\dy})
	-- ({5.4345*\dx},{0.0000*\dy})
	-- ({5.4445*\dx},{0.0000*\dy})
	-- ({5.4545*\dx},{0.0000*\dy})
	-- ({5.4646*\dx},{0.0000*\dy})
	-- ({5.4746*\dx},{0.0000*\dy})
	-- ({5.4846*\dx},{0.0000*\dy})
	-- ({5.4946*\dx},{0.0000*\dy})
	-- ({5.5046*\dx},{0.0000*\dy})
	-- ({5.5146*\dx},{0.0000*\dy})
	-- ({5.5246*\dx},{0.0000*\dy})
	-- ({5.5346*\dx},{0.0000*\dy})
	-- ({5.5446*\dx},{0.0000*\dy})
	-- ({5.5546*\dx},{0.0000*\dy})
	-- ({5.5646*\dx},{0.0000*\dy})
	-- ({5.5746*\dx},{0.0000*\dy})
	-- ({5.5847*\dx},{0.0000*\dy})
	-- ({5.5947*\dx},{0.0000*\dy})
	-- ({5.6047*\dx},{0.0000*\dy})
	-- ({5.6147*\dx},{0.0000*\dy})
	-- ({5.6247*\dx},{0.0000*\dy})
	-- ({5.6347*\dx},{0.0000*\dy})
	-- ({5.6447*\dx},{0.0000*\dy})
	-- ({5.6547*\dx},{0.0000*\dy})
	-- ({5.6647*\dx},{0.0000*\dy})
	-- ({5.6747*\dx},{0.0000*\dy})
	-- ({5.6847*\dx},{0.0000*\dy})
	-- ({5.6947*\dx},{0.0000*\dy})
	-- ({5.7048*\dx},{0.0000*\dy})
	-- ({5.7148*\dx},{0.0000*\dy})
	-- ({5.7248*\dx},{0.0000*\dy})
	-- ({5.7348*\dx},{0.0000*\dy})
	-- ({5.7448*\dx},{0.0000*\dy})
	-- ({5.7548*\dx},{0.0000*\dy})
	-- ({5.7648*\dx},{0.0000*\dy})
	-- ({5.7748*\dx},{0.0000*\dy})
	-- ({5.7848*\dx},{0.0000*\dy})
	-- ({5.7948*\dx},{0.0000*\dy})
	-- ({5.8048*\dx},{0.0000*\dy})
	-- ({5.8148*\dx},{0.0000*\dy})
	-- ({5.8249*\dx},{0.0000*\dy})
	-- ({5.8349*\dx},{0.0000*\dy})
	-- ({5.8449*\dx},{0.0000*\dy})
	-- ({5.8549*\dx},{0.0000*\dy})
	-- ({5.8649*\dx},{0.0000*\dy})
	-- ({5.8749*\dx},{0.0000*\dy})
	-- ({5.8849*\dx},{0.0000*\dy})
	-- ({5.8949*\dx},{0.0000*\dy})
	-- ({5.9049*\dx},{0.0000*\dy})
	-- ({5.9149*\dx},{0.0000*\dy})
	-- ({5.9249*\dx},{0.0000*\dy})
	-- ({5.9349*\dx},{0.0000*\dy})
	-- ({5.9450*\dx},{0.0000*\dy})
	-- ({5.9550*\dx},{0.0000*\dy})
	-- ({5.9650*\dx},{0.0000*\dy})
	-- ({5.9750*\dx},{0.0000*\dy})
	-- ({5.9850*\dx},{0.0000*\dy})
	-- ({5.9950*\dx},{0.0000*\dy})
	-- ({6.0050*\dx},{0.0000*\dy})
	-- ({6.0150*\dx},{0.0000*\dy})
	-- ({6.0250*\dx},{0.0000*\dy})
	-- ({6.0350*\dx},{0.0000*\dy})
	-- ({6.0450*\dx},{0.0000*\dy})
	-- ({6.0550*\dx},{0.0000*\dy})
	-- ({6.0651*\dx},{0.0000*\dy})
	-- ({6.0751*\dx},{0.0000*\dy})
	-- ({6.0851*\dx},{0.0000*\dy})
	-- ({6.0951*\dx},{0.0000*\dy})
	-- ({6.1051*\dx},{0.0000*\dy})
	-- ({6.1151*\dx},{0.0000*\dy})
	-- ({6.1251*\dx},{0.0000*\dy})
	-- ({6.1351*\dx},{0.0000*\dy})
	-- ({6.1451*\dx},{0.0000*\dy})
	-- ({6.1551*\dx},{0.0000*\dy})
	-- ({6.1651*\dx},{0.0000*\dy})
	-- ({6.1751*\dx},{0.0000*\dy})
	-- ({6.1852*\dx},{0.0000*\dy})
	-- ({6.1952*\dx},{0.0000*\dy})
	-- ({6.2052*\dx},{0.0000*\dy})
	-- ({6.2152*\dx},{0.0000*\dy})
	-- ({6.2252*\dx},{0.0000*\dy})
	-- ({6.2352*\dx},{0.0000*\dy})
	-- ({6.2452*\dx},{0.0000*\dy})
	-- ({6.2552*\dx},{0.0000*\dy})
	-- ({6.2652*\dx},{0.0000*\dy})
	-- ({6.2752*\dx},{0.0000*\dy})
	-- ({6.2852*\dx},{0.0000*\dy})
	-- ({6.2952*\dx},{0.0000*\dy})
	-- ({6.3053*\dx},{0.0000*\dy})
	-- ({6.3153*\dx},{0.0000*\dy})
	-- ({6.3253*\dx},{0.0000*\dy})
	-- ({6.3353*\dx},{0.0000*\dy})
	-- ({6.3453*\dx},{0.0000*\dy})
	-- ({6.3553*\dx},{0.0000*\dy})
	-- ({6.3653*\dx},{0.0000*\dy})
	-- ({6.3753*\dx},{0.0000*\dy})
	-- ({6.3853*\dx},{0.0000*\dy})
	-- ({6.3953*\dx},{0.0000*\dy})
	-- ({6.4053*\dx},{0.0000*\dy})
	-- ({6.4153*\dx},{0.0000*\dy})
	-- ({6.4254*\dx},{0.0000*\dy})
	-- ({6.4354*\dx},{0.0000*\dy})
	-- ({6.4454*\dx},{0.0000*\dy})
	-- ({6.4554*\dx},{0.0000*\dy})
	-- ({6.4654*\dx},{0.0000*\dy})
	-- ({6.4754*\dx},{0.0000*\dy})
	-- ({6.4854*\dx},{0.0000*\dy})
	-- ({6.4954*\dx},{0.0000*\dy})
	-- ({6.5054*\dx},{0.0000*\dy})
	-- ({6.5154*\dx},{0.0000*\dy})
	-- ({6.5254*\dx},{0.0000*\dy})
	-- ({6.5354*\dx},{0.0000*\dy})
	-- ({6.5455*\dx},{0.0000*\dy})
	-- ({6.5555*\dx},{0.0000*\dy})
	-- ({6.5655*\dx},{0.0000*\dy})
	-- ({6.5755*\dx},{0.0000*\dy})
	-- ({6.5855*\dx},{0.0000*\dy})
	-- ({6.5955*\dx},{0.0000*\dy})
	-- ({6.6055*\dx},{0.0000*\dy})
	-- ({6.6155*\dx},{0.0000*\dy})
	-- ({6.6255*\dx},{0.0000*\dy})
	-- ({6.6355*\dx},{0.0000*\dy})
	-- ({6.6455*\dx},{0.0000*\dy})
	-- ({6.6555*\dx},{0.0000*\dy})
	-- ({6.6656*\dx},{0.0000*\dy})
	-- ({6.6756*\dx},{0.0000*\dy})
	-- ({6.6856*\dx},{0.0000*\dy})
	-- ({6.6956*\dx},{0.0000*\dy})
	-- ({6.7056*\dx},{0.0000*\dy})
	-- ({6.7156*\dx},{0.0000*\dy})
	-- ({6.7256*\dx},{0.0000*\dy})
	-- ({6.7356*\dx},{0.0000*\dy})
	-- ({6.7456*\dx},{0.0000*\dy})
	-- ({6.7556*\dx},{0.0000*\dy})
	-- ({6.7656*\dx},{0.0000*\dy})
	-- ({6.7756*\dx},{0.0000*\dy})
	-- ({6.7857*\dx},{0.0000*\dy})
	-- ({6.7957*\dx},{0.0000*\dy})
	-- ({6.8057*\dx},{0.0000*\dy})
	-- ({6.8157*\dx},{0.0000*\dy})
	-- ({6.8257*\dx},{0.0000*\dy})
	-- ({6.8357*\dx},{0.0000*\dy})
	-- ({6.8457*\dx},{0.0000*\dy})
	-- ({6.8557*\dx},{0.0000*\dy})
	-- ({6.8657*\dx},{0.0000*\dy})
	-- ({6.8757*\dx},{0.0000*\dy})
	-- ({6.8857*\dx},{0.0000*\dy})
	-- ({6.8957*\dx},{0.0000*\dy})
	-- ({6.9058*\dx},{0.0000*\dy})
	-- ({6.9158*\dx},{0.0000*\dy})
	-- ({6.9258*\dx},{0.0000*\dy})
	-- ({6.9358*\dx},{0.0000*\dy})
	-- ({6.9458*\dx},{0.0000*\dy})
	-- ({6.9558*\dx},{0.0000*\dy})
	-- ({6.9658*\dx},{0.0000*\dy})
	-- ({6.9758*\dx},{0.0000*\dy})
	-- ({6.9858*\dx},{0.0000*\dy})
	-- ({6.9958*\dx},{0.0000*\dy})
	-- ({7.0058*\dx},{0.0000*\dy})
	-- ({7.0158*\dx},{0.0000*\dy})
	-- ({7.0259*\dx},{0.0000*\dy})
	-- ({7.0359*\dx},{0.0000*\dy})
	-- ({7.0459*\dx},{0.0000*\dy})
	-- ({7.0559*\dx},{0.0000*\dy})
	-- ({7.0659*\dx},{0.0000*\dy})
	-- ({7.0759*\dx},{0.0000*\dy})
	-- ({7.0859*\dx},{0.0000*\dy})
	-- ({7.0959*\dx},{0.0000*\dy})
	-- ({7.1059*\dx},{0.0000*\dy})
	-- ({7.1159*\dx},{0.0000*\dy})
	-- ({7.1259*\dx},{0.0000*\dy})
	-- ({7.1359*\dx},{0.0000*\dy})
	-- ({7.1460*\dx},{0.0000*\dy})
	-- ({7.1560*\dx},{0.0000*\dy})
	-- ({7.1660*\dx},{0.0000*\dy})
	-- ({7.1760*\dx},{0.0000*\dy})
	-- ({7.1860*\dx},{0.0000*\dy})
	-- ({7.1960*\dx},{0.0000*\dy})
	-- ({7.2060*\dx},{0.0000*\dy})
	-- ({7.2160*\dx},{0.0000*\dy})
	-- ({7.2260*\dx},{0.0000*\dy})
	-- ({7.2360*\dx},{0.0000*\dy})
	-- ({7.2460*\dx},{0.0000*\dy})
	-- ({7.2560*\dx},{0.0000*\dy})
	-- ({7.2661*\dx},{0.0000*\dy})
	-- ({7.2761*\dx},{0.0000*\dy})
	-- ({7.2861*\dx},{0.0000*\dy})
	-- ({7.2961*\dx},{0.0000*\dy})
	-- ({7.3061*\dx},{0.0000*\dy})
	-- ({7.3161*\dx},{0.0000*\dy})
	-- ({7.3261*\dx},{0.0000*\dy})
	-- ({7.3361*\dx},{0.0000*\dy})
	-- ({7.3461*\dx},{0.0000*\dy})
	-- ({7.3561*\dx},{0.0000*\dy})
	-- ({7.3661*\dx},{0.0000*\dy})
	-- ({7.3761*\dx},{0.0000*\dy})
	-- ({7.3862*\dx},{0.0000*\dy})
	-- ({7.3962*\dx},{0.0000*\dy})
	-- ({7.4062*\dx},{0.0000*\dy})
	-- ({7.4162*\dx},{0.0000*\dy})
	-- ({7.4262*\dx},{0.0000*\dy})
	-- ({7.4362*\dx},{0.0000*\dy})
	-- ({7.4462*\dx},{0.0000*\dy})
	-- ({7.4562*\dx},{0.0000*\dy})
	-- ({7.4662*\dx},{0.0000*\dy})
	-- ({7.4762*\dx},{0.0000*\dy})
	-- ({7.4862*\dx},{0.0000*\dy})
	-- ({7.4962*\dx},{0.0000*\dy})
	-- ({7.5063*\dx},{0.0000*\dy})
	-- ({7.5163*\dx},{0.0000*\dy})
	-- ({7.5263*\dx},{0.0000*\dy})
	-- ({7.5363*\dx},{0.0000*\dy})
	-- ({7.5463*\dx},{0.0000*\dy})
	-- ({7.5563*\dx},{0.0000*\dy})
	-- ({7.5663*\dx},{0.0000*\dy})
	-- ({7.5763*\dx},{0.0000*\dy})
	-- ({7.5863*\dx},{0.0000*\dy})
	-- ({7.5963*\dx},{0.0000*\dy})
	-- ({7.6063*\dx},{0.0000*\dy})
	-- ({7.6163*\dx},{0.0000*\dy})
	-- ({7.6264*\dx},{0.0000*\dy})
	-- ({7.6364*\dx},{0.0000*\dy})
	-- ({7.6464*\dx},{0.0000*\dy})
	-- ({7.6564*\dx},{0.0000*\dy})
	-- ({7.6664*\dx},{0.0000*\dy})
	-- ({7.6764*\dx},{0.0000*\dy})
	-- ({7.6864*\dx},{0.0000*\dy})
	-- ({7.6964*\dx},{0.0000*\dy})
	-- ({7.7064*\dx},{0.0000*\dy})
	-- ({7.7164*\dx},{0.0000*\dy})
	-- ({7.7264*\dx},{0.0000*\dy})
	-- ({7.7364*\dx},{0.0000*\dy})
	-- ({7.7465*\dx},{0.0000*\dy})
	-- ({7.7565*\dx},{0.0000*\dy})
	-- ({7.7665*\dx},{0.0000*\dy})
	-- ({7.7765*\dx},{0.0000*\dy})
	-- ({7.7865*\dx},{0.0000*\dy})
	-- ({7.7965*\dx},{0.0000*\dy})
	-- ({7.8065*\dx},{0.0000*\dy})
	-- ({7.8165*\dx},{0.0000*\dy})
	-- ({7.8265*\dx},{0.0000*\dy})
	-- ({7.8365*\dx},{0.0000*\dy})
	-- ({7.8465*\dx},{0.0000*\dy})
	-- ({7.8565*\dx},{0.0000*\dy})
	-- ({7.8666*\dx},{0.0000*\dy})
	-- ({7.8766*\dx},{0.0000*\dy})
	-- ({7.8866*\dx},{0.0000*\dy})
	-- ({7.8966*\dx},{0.0000*\dy})
	-- ({7.9066*\dx},{0.0000*\dy})
	-- ({7.9166*\dx},{0.0000*\dy})
	-- ({7.9266*\dx},{0.0000*\dy})
	-- ({7.9366*\dx},{0.0000*\dy})
	-- ({7.9466*\dx},{0.0000*\dy})
	-- ({7.9566*\dx},{0.0000*\dy})
	-- ({7.9666*\dx},{0.0000*\dy})
	-- ({7.9766*\dx},{0.0000*\dy})
	-- ({7.9867*\dx},{0.0000*\dy})
	-- ({7.9967*\dx},{0.0000*\dy})
	-- ({8.0067*\dx},{0.0000*\dy})
	-- ({8.0167*\dx},{0.0000*\dy})
	-- ({8.0267*\dx},{0.0000*\dy})
	-- ({8.0367*\dx},{0.0000*\dy})
	-- ({8.0467*\dx},{0.0000*\dy})
	-- ({8.0567*\dx},{0.0000*\dy})
	-- ({8.0667*\dx},{0.0000*\dy})
	-- ({8.0767*\dx},{0.0000*\dy})
	-- ({8.0867*\dx},{0.0000*\dy})
	-- ({8.0967*\dx},{0.0000*\dy})
	-- ({8.1068*\dx},{0.0000*\dy})
	-- ({8.1168*\dx},{0.0000*\dy})
	-- ({8.1268*\dx},{0.0000*\dy})
	-- ({8.1368*\dx},{0.0000*\dy})
	-- ({8.1468*\dx},{0.0000*\dy})
	-- ({8.1568*\dx},{0.0000*\dy})
	-- ({8.1668*\dx},{0.0000*\dy})
	-- ({8.1768*\dx},{0.0000*\dy})
	-- ({8.1868*\dx},{0.0000*\dy})
	-- ({8.1968*\dx},{0.0000*\dy})
	-- ({8.2068*\dx},{0.0000*\dy})
	-- ({8.2168*\dx},{0.0000*\dy})
	-- ({8.2269*\dx},{0.0000*\dy})
	-- ({8.2369*\dx},{0.0000*\dy})
	-- ({8.2469*\dx},{0.0000*\dy})
	-- ({8.2569*\dx},{0.0000*\dy})
	-- ({8.2669*\dx},{0.0000*\dy})
	-- ({8.2769*\dx},{0.0000*\dy})
	-- ({8.2869*\dx},{0.0000*\dy})
	-- ({8.2969*\dx},{0.0000*\dy})
	-- ({8.3069*\dx},{0.0000*\dy})
	-- ({8.3169*\dx},{0.0000*\dy})
	-- ({8.3269*\dx},{0.0000*\dy})
	-- ({8.3369*\dx},{0.0000*\dy})
	-- ({8.3470*\dx},{0.0000*\dy})
	-- ({8.3570*\dx},{0.0000*\dy})
	-- ({8.3670*\dx},{0.0000*\dy})
	-- ({8.3770*\dx},{0.0000*\dy})
	-- ({8.3870*\dx},{0.0000*\dy})
	-- ({8.3970*\dx},{0.0000*\dy})
	-- ({8.4070*\dx},{0.0000*\dy})
	-- ({8.4170*\dx},{0.0000*\dy})
	-- ({8.4270*\dx},{0.0000*\dy})
	-- ({8.4370*\dx},{0.0000*\dy})
	-- ({8.4470*\dx},{0.0000*\dy})
	-- ({8.4570*\dx},{0.0000*\dy})
	-- ({8.4671*\dx},{0.0000*\dy})
	-- ({8.4771*\dx},{0.0000*\dy})
	-- ({8.4871*\dx},{0.0000*\dy})
	-- ({8.4971*\dx},{0.0000*\dy})
	-- ({8.5071*\dx},{0.0000*\dy})
	-- ({8.5171*\dx},{0.0000*\dy})
	-- ({8.5271*\dx},{0.0000*\dy})
	-- ({8.5371*\dx},{0.0000*\dy})
	-- ({8.5471*\dx},{0.0000*\dy})
	-- ({8.5571*\dx},{0.0000*\dy})
	-- ({8.5671*\dx},{0.0000*\dy})
	-- ({8.5771*\dx},{0.0000*\dy})
	-- ({8.5872*\dx},{0.0000*\dy})
	-- ({8.5972*\dx},{0.0000*\dy})
	-- ({8.6072*\dx},{0.0000*\dy})
	-- ({8.6172*\dx},{0.0000*\dy})
	-- ({8.6272*\dx},{0.0000*\dy})
	-- ({8.6372*\dx},{0.0000*\dy})
	-- ({8.6472*\dx},{0.0000*\dy})
	-- ({8.6572*\dx},{0.0000*\dy})
	-- ({8.6672*\dx},{0.0000*\dy})
	-- ({8.6772*\dx},{0.0000*\dy})
	-- ({8.6872*\dx},{0.0000*\dy})
	-- ({8.6972*\dx},{0.0000*\dy})
	-- ({8.7073*\dx},{0.0000*\dy})
	-- ({8.7173*\dx},{0.0000*\dy})
	-- ({8.7273*\dx},{0.0000*\dy})
	-- ({8.7373*\dx},{0.0000*\dy})
	-- ({8.7473*\dx},{0.0000*\dy})
	-- ({8.7573*\dx},{0.0000*\dy})
	-- ({8.7673*\dx},{0.0000*\dy})
	-- ({8.7773*\dx},{0.0000*\dy})
	-- ({8.7873*\dx},{0.0000*\dy})
	-- ({8.7973*\dx},{0.0000*\dy})
	-- ({8.8073*\dx},{0.0000*\dy})
	-- ({8.8173*\dx},{0.0000*\dy})
	-- ({8.8274*\dx},{0.0000*\dy})
	-- ({8.8374*\dx},{0.0000*\dy})
	-- ({8.8474*\dx},{0.0000*\dy})
	-- ({8.8574*\dx},{0.0000*\dy})
	-- ({8.8674*\dx},{0.0000*\dy})
	-- ({8.8774*\dx},{0.0000*\dy})
	-- ({8.8874*\dx},{0.0000*\dy})
	-- ({8.8974*\dx},{0.0000*\dy})
	-- ({8.9074*\dx},{0.0000*\dy})
	-- ({8.9174*\dx},{0.0000*\dy})
	-- ({8.9274*\dx},{0.0000*\dy})
	-- ({8.9374*\dx},{0.0000*\dy})
	-- ({8.9475*\dx},{0.0000*\dy})
	-- ({8.9575*\dx},{0.0000*\dy})
	-- ({8.9675*\dx},{0.0000*\dy})
	-- ({8.9775*\dx},{0.0000*\dy})
	-- ({8.9875*\dx},{0.0000*\dy})
	-- ({8.9975*\dx},{0.0000*\dy})
	-- ({9.0075*\dx},{0.0000*\dy})
	-- ({9.0175*\dx},{0.0000*\dy})
	-- ({9.0275*\dx},{0.0000*\dy})
	-- ({9.0375*\dx},{0.0000*\dy})
	-- ({9.0475*\dx},{0.0000*\dy})
	-- ({9.0575*\dx},{0.0000*\dy})
	-- ({9.0676*\dx},{0.0000*\dy})
	-- ({9.0776*\dx},{0.0000*\dy})
	-- ({9.0876*\dx},{0.0000*\dy})
	-- ({9.0976*\dx},{0.0000*\dy})
	-- ({9.1076*\dx},{0.0000*\dy})
	-- ({9.1176*\dx},{0.0000*\dy})
	-- ({9.1276*\dx},{0.0000*\dy})
	-- ({9.1376*\dx},{0.0000*\dy})
	-- ({9.1476*\dx},{0.0000*\dy})
	-- ({9.1576*\dx},{0.0000*\dy})
	-- ({9.1676*\dx},{0.0000*\dy})
	-- ({9.1776*\dx},{0.0000*\dy})
	-- ({9.1877*\dx},{0.0000*\dy})
	-- ({9.1977*\dx},{0.0000*\dy})
	-- ({9.2077*\dx},{0.0000*\dy})
	-- ({9.2177*\dx},{0.0000*\dy})
	-- ({9.2277*\dx},{0.0000*\dy})
	-- ({9.2377*\dx},{0.0000*\dy})
	-- ({9.2477*\dx},{0.0000*\dy})
	-- ({9.2577*\dx},{0.0000*\dy})
	-- ({9.2677*\dx},{0.0000*\dy})
	-- ({9.2777*\dx},{0.0000*\dy})
	-- ({9.2877*\dx},{0.0000*\dy})
	-- ({9.2977*\dx},{0.0000*\dy})
	-- ({9.3078*\dx},{0.0000*\dy})
	-- ({9.3178*\dx},{0.0000*\dy})
	-- ({9.3278*\dx},{0.0000*\dy})
	-- ({9.3378*\dx},{0.0000*\dy})
	-- ({9.3478*\dx},{0.0000*\dy})
	-- ({9.3578*\dx},{0.0000*\dy})
	-- ({9.3678*\dx},{0.0000*\dy})
	-- ({9.3778*\dx},{0.0000*\dy})
	-- ({9.3878*\dx},{0.0000*\dy})
	-- ({9.3978*\dx},{0.0000*\dy})
	-- ({9.4078*\dx},{0.0000*\dy})
	-- ({9.4178*\dx},{0.0000*\dy})
	-- ({9.4279*\dx},{0.0000*\dy})
	-- ({9.4379*\dx},{0.0000*\dy})
	-- ({9.4479*\dx},{0.0000*\dy})
	-- ({9.4579*\dx},{0.0000*\dy})
	-- ({9.4679*\dx},{0.0000*\dy})
	-- ({9.4779*\dx},{0.0000*\dy})
	-- ({9.4879*\dx},{0.0000*\dy})
	-- ({9.4979*\dx},{0.0000*\dy})
	-- ({9.5079*\dx},{0.0000*\dy})
	-- ({9.5179*\dx},{0.0000*\dy})
	-- ({9.5279*\dx},{0.0000*\dy})
	-- ({9.5379*\dx},{0.0000*\dy})
	-- ({9.5480*\dx},{0.0000*\dy})
	-- ({9.5580*\dx},{0.0000*\dy})
	-- ({9.5680*\dx},{0.0000*\dy})
	-- ({9.5780*\dx},{0.0000*\dy})
	-- ({9.5880*\dx},{0.0000*\dy})
	-- ({9.5980*\dx},{0.0000*\dy})
	-- ({9.6080*\dx},{0.0000*\dy})
	-- ({9.6180*\dx},{0.0000*\dy})
	-- ({9.6280*\dx},{0.0000*\dy})
	-- ({9.6380*\dx},{0.0000*\dy})
	-- ({9.6480*\dx},{0.0000*\dy})
	-- ({9.6580*\dx},{0.0000*\dy})
	-- ({9.6681*\dx},{0.0000*\dy})
	-- ({9.6781*\dx},{0.0000*\dy})
	-- ({9.6881*\dx},{0.0000*\dy})
	-- ({9.6981*\dx},{0.0000*\dy})
	-- ({9.7081*\dx},{0.0000*\dy})
	-- ({9.7181*\dx},{0.0000*\dy})
	-- ({9.7281*\dx},{0.0000*\dy})
	-- ({9.7381*\dx},{0.0000*\dy})
	-- ({9.7481*\dx},{0.0000*\dy})
	-- ({9.7581*\dx},{0.0000*\dy})
	-- ({9.7681*\dx},{0.0000*\dy})
	-- ({9.7781*\dx},{0.0000*\dy})
	-- ({9.7882*\dx},{0.0000*\dy})
	-- ({9.7982*\dx},{0.0000*\dy})
	-- ({9.8082*\dx},{0.0000*\dy})
	-- ({9.8182*\dx},{0.0000*\dy})
	-- ({9.8282*\dx},{0.0000*\dy})
	-- ({9.8382*\dx},{0.0000*\dy})
	-- ({9.8482*\dx},{0.0000*\dy})
	-- ({9.8582*\dx},{0.0000*\dy})
	-- ({9.8682*\dx},{0.0000*\dy})
	-- ({9.8782*\dx},{0.0000*\dy})
	-- ({9.8882*\dx},{0.0000*\dy})
	-- ({9.8982*\dx},{0.0000*\dy})
	-- ({9.9083*\dx},{0.0000*\dy})
	-- ({9.9183*\dx},{0.0000*\dy})
	-- ({9.9283*\dx},{0.0000*\dy})
	-- ({9.9383*\dx},{0.0000*\dy})
	-- ({9.9483*\dx},{0.0000*\dy})
	-- ({9.9583*\dx},{0.0000*\dy})
	-- ({9.9683*\dx},{0.0000*\dy})
	-- ({9.9783*\dx},{0.0000*\dy})
	-- ({9.9883*\dx},{0.0000*\dy})
	-- ({9.9983*\dx},{0.0000*\dy})
	-- ({10.0083*\dx},{0.0000*\dy})
	-- ({10.0183*\dx},{0.0000*\dy})
	-- ({10.0284*\dx},{0.0000*\dy})
	-- ({10.0384*\dx},{0.0000*\dy})
	-- ({10.0484*\dx},{0.0000*\dy})
	-- ({10.0584*\dx},{0.0000*\dy})
	-- ({10.0684*\dx},{0.0000*\dy})
	-- ({10.0784*\dx},{0.0000*\dy})
	-- ({10.0884*\dx},{0.0000*\dy})
	-- ({10.0984*\dx},{0.0000*\dy})
	-- ({10.1084*\dx},{0.0000*\dy})
	-- ({10.1184*\dx},{0.0000*\dy})
	-- ({10.1284*\dx},{0.0000*\dy})
	-- ({10.1384*\dx},{0.0000*\dy})
	-- ({10.1485*\dx},{0.0000*\dy})
	-- ({10.1585*\dx},{0.0000*\dy})
	-- ({10.1685*\dx},{0.0000*\dy})
	-- ({10.1785*\dx},{0.0000*\dy})
	-- ({10.1885*\dx},{0.0000*\dy})
	-- ({10.1985*\dx},{0.0000*\dy})
	-- ({10.2085*\dx},{0.0000*\dy})
	-- ({10.2185*\dx},{0.0000*\dy})
	-- ({10.2285*\dx},{0.0000*\dy})
	-- ({10.2385*\dx},{0.0000*\dy})
	-- ({10.2485*\dx},{0.0000*\dy})
	-- ({10.2585*\dx},{0.0000*\dy})
	-- ({10.2686*\dx},{0.0000*\dy})
	-- ({10.2786*\dx},{0.0000*\dy})
	-- ({10.2886*\dx},{0.0000*\dy})
	-- ({10.2986*\dx},{0.0000*\dy})
	-- ({10.3086*\dx},{0.0000*\dy})
	-- ({10.3186*\dx},{0.0000*\dy})
	-- ({10.3286*\dx},{0.0000*\dy})
	-- ({10.3386*\dx},{0.0000*\dy})
	-- ({10.3486*\dx},{0.0000*\dy})
	-- ({10.3586*\dx},{0.0000*\dy})
	-- ({10.3686*\dx},{0.0000*\dy})
	-- ({10.3786*\dx},{0.0000*\dy})
	-- ({10.3887*\dx},{0.0000*\dy})
	-- ({10.3987*\dx},{0.0000*\dy})
	-- ({10.4087*\dx},{0.0000*\dy})
	-- ({10.4187*\dx},{0.0000*\dy})
	-- ({10.4287*\dx},{0.0000*\dy})
	-- ({10.4387*\dx},{0.0000*\dy})
	-- ({10.4487*\dx},{0.0000*\dy})
	-- ({10.4587*\dx},{0.0000*\dy})
	-- ({10.4687*\dx},{0.0000*\dy})
	-- ({10.4787*\dx},{0.0000*\dy})
	-- ({10.4887*\dx},{0.0000*\dy})
	-- ({10.4987*\dx},{0.0000*\dy})
	-- ({10.5088*\dx},{0.0000*\dy})
	-- ({10.5188*\dx},{0.0000*\dy})
	-- ({10.5288*\dx},{0.0000*\dy})
	-- ({10.5388*\dx},{0.0000*\dy})
	-- ({10.5488*\dx},{0.0000*\dy})
	-- ({10.5588*\dx},{0.0000*\dy})
	-- ({10.5688*\dx},{0.0000*\dy})
	-- ({10.5788*\dx},{0.0000*\dy})
	-- ({10.5888*\dx},{0.0000*\dy})
	-- ({10.5988*\dx},{0.0000*\dy})
	-- ({10.6088*\dx},{0.0000*\dy})
	-- ({10.6188*\dx},{0.0000*\dy})
	-- ({10.6289*\dx},{0.0000*\dy})
	-- ({10.6389*\dx},{0.0000*\dy})
	-- ({10.6489*\dx},{0.0000*\dy})
	-- ({10.6589*\dx},{0.0000*\dy})
	-- ({10.6689*\dx},{0.0000*\dy})
	-- ({10.6789*\dx},{0.0000*\dy})
	-- ({10.6889*\dx},{0.0000*\dy})
	-- ({10.6989*\dx},{0.0000*\dy})
	-- ({10.7089*\dx},{0.0000*\dy})
	-- ({10.7189*\dx},{0.0000*\dy})
	-- ({10.7289*\dx},{0.0000*\dy})
	-- ({10.7389*\dx},{0.0000*\dy})
	-- ({10.7490*\dx},{0.0000*\dy})
	-- ({10.7590*\dx},{0.0000*\dy})
	-- ({10.7690*\dx},{0.0000*\dy})
	-- ({10.7790*\dx},{0.0000*\dy})
	-- ({10.7890*\dx},{0.0000*\dy})
	-- ({10.7990*\dx},{0.0000*\dy})
	-- ({10.8090*\dx},{0.0000*\dy})
	-- ({10.8190*\dx},{0.0000*\dy})
	-- ({10.8290*\dx},{0.0000*\dy})
	-- ({10.8390*\dx},{0.0000*\dy})
	-- ({10.8490*\dx},{0.0000*\dy})
	-- ({10.8590*\dx},{0.0000*\dy})
	-- ({10.8691*\dx},{0.0000*\dy})
	-- ({10.8791*\dx},{0.0000*\dy})
	-- ({10.8891*\dx},{0.0000*\dy})
	-- ({10.8991*\dx},{0.0000*\dy})
	-- ({10.9091*\dx},{0.0000*\dy})
	-- ({10.9191*\dx},{0.0000*\dy})
	-- ({10.9291*\dx},{0.0000*\dy})
	-- ({10.9391*\dx},{0.0000*\dy})
	-- ({10.9491*\dx},{0.0000*\dy})
	-- ({10.9591*\dx},{0.0000*\dy})
	-- ({10.9691*\dx},{0.0000*\dy})
	-- ({10.9791*\dx},{0.0000*\dy})
	-- ({10.9892*\dx},{0.0000*\dy})
	-- ({10.9992*\dx},{0.0000*\dy})
	-- ({11.0092*\dx},{0.0000*\dy})
	-- ({11.0192*\dx},{0.0000*\dy})
	-- ({11.0292*\dx},{0.0000*\dy})
	-- ({11.0392*\dx},{0.0000*\dy})
	-- ({11.0492*\dx},{0.0000*\dy})
	-- ({11.0592*\dx},{0.0000*\dy})
	-- ({11.0692*\dx},{0.0000*\dy})
	-- ({11.0792*\dx},{0.0000*\dy})
	-- ({11.0892*\dx},{0.0000*\dy})
	-- ({11.0992*\dx},{0.0000*\dy})
	-- ({11.1093*\dx},{0.0000*\dy})
	-- ({11.1193*\dx},{0.0000*\dy})
	-- ({11.1293*\dx},{0.0000*\dy})
	-- ({11.1393*\dx},{0.0000*\dy})
	-- ({11.1493*\dx},{0.0000*\dy})
	-- ({11.1593*\dx},{0.0000*\dy})
	-- ({11.1693*\dx},{0.0000*\dy})
	-- ({11.1793*\dx},{0.0000*\dy})
	-- ({11.1893*\dx},{0.0000*\dy})
	-- ({11.1993*\dx},{0.0000*\dy})
	-- ({11.2093*\dx},{0.0000*\dy})
	-- ({11.2193*\dx},{0.0000*\dy})
	-- ({11.2294*\dx},{0.0000*\dy})
	-- ({11.2394*\dx},{0.0000*\dy})
	-- ({11.2494*\dx},{0.0000*\dy})
	-- ({11.2594*\dx},{0.0000*\dy})
	-- ({11.2694*\dx},{0.0000*\dy})
	-- ({11.2794*\dx},{0.0000*\dy})
	-- ({11.2894*\dx},{0.0000*\dy})
	-- ({11.2994*\dx},{0.0000*\dy})
	-- ({11.3094*\dx},{0.0000*\dy})
	-- ({11.3194*\dx},{0.0000*\dy})
	-- ({11.3294*\dx},{0.0000*\dy})
	-- ({11.3394*\dx},{0.0000*\dy})
	-- ({11.3495*\dx},{0.0000*\dy})
	-- ({11.3595*\dx},{0.0000*\dy})
	-- ({11.3695*\dx},{0.0000*\dy})
	-- ({11.3795*\dx},{0.0000*\dy})
	-- ({11.3895*\dx},{0.0000*\dy})
	-- ({11.3995*\dx},{0.0000*\dy})
	-- ({11.4095*\dx},{0.0000*\dy})
	-- ({11.4195*\dx},{0.0000*\dy})
	-- ({11.4295*\dx},{0.0000*\dy})
	-- ({11.4395*\dx},{0.0000*\dy})
	-- ({11.4495*\dx},{0.0000*\dy})
	-- ({11.4595*\dx},{0.0000*\dy})
	-- ({11.4696*\dx},{0.0000*\dy})
	-- ({11.4796*\dx},{0.0000*\dy})
	-- ({11.4896*\dx},{0.0000*\dy})
	-- ({11.4996*\dx},{0.0000*\dy})
	-- ({11.5096*\dx},{0.0000*\dy})
	-- ({11.5196*\dx},{0.0000*\dy})
	-- ({11.5296*\dx},{0.0000*\dy})
	-- ({11.5396*\dx},{0.0000*\dy})
	-- ({11.5496*\dx},{0.0000*\dy})
	-- ({11.5596*\dx},{0.0000*\dy})
	-- ({11.5696*\dx},{0.0000*\dy})
	-- ({11.5796*\dx},{0.0000*\dy})
	-- ({11.5897*\dx},{0.0000*\dy})
	-- ({11.5997*\dx},{0.0000*\dy})
	-- ({11.6097*\dx},{0.0000*\dy})
	-- ({11.6197*\dx},{0.0000*\dy})
	-- ({11.6297*\dx},{0.0000*\dy})
	-- ({11.6397*\dx},{0.0000*\dy})
	-- ({11.6497*\dx},{0.0000*\dy})
	-- ({11.6597*\dx},{0.0000*\dy})
	-- ({11.6697*\dx},{0.0000*\dy})
	-- ({11.6797*\dx},{0.0000*\dy})
	-- ({11.6897*\dx},{0.0000*\dy})
	-- ({11.6997*\dx},{0.0000*\dy})
	-- ({11.7098*\dx},{0.0000*\dy})
	-- ({11.7198*\dx},{0.0000*\dy})
	-- ({11.7298*\dx},{0.0000*\dy})
	-- ({11.7398*\dx},{0.0000*\dy})
	-- ({11.7498*\dx},{0.0000*\dy})
	-- ({11.7598*\dx},{0.0000*\dy})
	-- ({11.7698*\dx},{0.0000*\dy})
	-- ({11.7798*\dx},{0.0000*\dy})
	-- ({11.7898*\dx},{0.0000*\dy})
	-- ({11.7998*\dx},{0.0000*\dy})
	-- ({11.8098*\dx},{0.0000*\dy})
	-- ({11.8198*\dx},{0.0000*\dy})
	-- ({11.8299*\dx},{0.0000*\dy})
	-- ({11.8399*\dx},{0.0000*\dy})
	-- ({11.8499*\dx},{0.0000*\dy})
	-- ({11.8599*\dx},{0.0000*\dy})
	-- ({11.8699*\dx},{0.0000*\dy})
	-- ({11.8799*\dx},{0.0000*\dy})
	-- ({11.8899*\dx},{0.0000*\dy})
	-- ({11.8999*\dx},{0.0000*\dy})
	-- ({11.9099*\dx},{0.0000*\dy})
	-- ({11.9199*\dx},{0.0000*\dy})
	-- ({11.9299*\dx},{0.0000*\dy})
	-- ({11.9399*\dx},{0.0000*\dy})
	-- ({11.9500*\dx},{0.0000*\dy})
	-- ({11.9600*\dx},{0.0000*\dy})
	-- ({11.9700*\dx},{0.0000*\dy})
	-- ({11.9800*\dx},{0.0000*\dy})
	-- ({11.9900*\dx},{0.0000*\dy})
	-- ({12.0000*\dx},{0.0000*\dy})
}
\def\psitwo{
	({0.0000*\dx},{0.0000*\dy})
	-- ({0.0100*\dx},{0.0000*\dy})
	-- ({0.0200*\dx},{0.0000*\dy})
	-- ({0.0300*\dx},{0.0000*\dy})
	-- ({0.0400*\dx},{0.0000*\dy})
	-- ({0.0500*\dx},{0.0000*\dy})
	-- ({0.0601*\dx},{0.0000*\dy})
	-- ({0.0701*\dx},{0.0000*\dy})
	-- ({0.0801*\dx},{0.0000*\dy})
	-- ({0.0901*\dx},{0.0000*\dy})
	-- ({0.1001*\dx},{0.0000*\dy})
	-- ({0.1101*\dx},{0.0000*\dy})
	-- ({0.1201*\dx},{0.0000*\dy})
	-- ({0.1301*\dx},{0.0000*\dy})
	-- ({0.1401*\dx},{0.0000*\dy})
	-- ({0.1501*\dx},{0.0000*\dy})
	-- ({0.1601*\dx},{0.0000*\dy})
	-- ({0.1701*\dx},{0.0000*\dy})
	-- ({0.1802*\dx},{0.0000*\dy})
	-- ({0.1902*\dx},{0.0000*\dy})
	-- ({0.2002*\dx},{0.0000*\dy})
	-- ({0.2102*\dx},{0.0000*\dy})
	-- ({0.2202*\dx},{0.0000*\dy})
	-- ({0.2302*\dx},{0.0000*\dy})
	-- ({0.2402*\dx},{0.0000*\dy})
	-- ({0.2502*\dx},{0.0000*\dy})
	-- ({0.2602*\dx},{0.0000*\dy})
	-- ({0.2702*\dx},{0.0000*\dy})
	-- ({0.2802*\dx},{0.0000*\dy})
	-- ({0.2902*\dx},{0.0000*\dy})
	-- ({0.3003*\dx},{0.0000*\dy})
	-- ({0.3103*\dx},{0.0000*\dy})
	-- ({0.3203*\dx},{0.0000*\dy})
	-- ({0.3303*\dx},{0.0000*\dy})
	-- ({0.3403*\dx},{0.0000*\dy})
	-- ({0.3503*\dx},{0.0000*\dy})
	-- ({0.3603*\dx},{0.0000*\dy})
	-- ({0.3703*\dx},{0.0000*\dy})
	-- ({0.3803*\dx},{0.0000*\dy})
	-- ({0.3903*\dx},{0.0000*\dy})
	-- ({0.4003*\dx},{0.0000*\dy})
	-- ({0.4103*\dx},{0.0000*\dy})
	-- ({0.4204*\dx},{0.0000*\dy})
	-- ({0.4304*\dx},{0.0000*\dy})
	-- ({0.4404*\dx},{0.0000*\dy})
	-- ({0.4504*\dx},{0.0000*\dy})
	-- ({0.4604*\dx},{0.0000*\dy})
	-- ({0.4704*\dx},{0.0000*\dy})
	-- ({0.4804*\dx},{0.0000*\dy})
	-- ({0.4904*\dx},{0.0000*\dy})
	-- ({0.5004*\dx},{0.0000*\dy})
	-- ({0.5104*\dx},{0.0000*\dy})
	-- ({0.5204*\dx},{0.0000*\dy})
	-- ({0.5304*\dx},{0.0000*\dy})
	-- ({0.5405*\dx},{0.0000*\dy})
	-- ({0.5505*\dx},{0.0000*\dy})
	-- ({0.5605*\dx},{0.0000*\dy})
	-- ({0.5705*\dx},{0.0000*\dy})
	-- ({0.5805*\dx},{0.0000*\dy})
	-- ({0.5905*\dx},{0.0000*\dy})
	-- ({0.6005*\dx},{0.0000*\dy})
	-- ({0.6105*\dx},{0.0000*\dy})
	-- ({0.6205*\dx},{0.0000*\dy})
	-- ({0.6305*\dx},{0.0000*\dy})
	-- ({0.6405*\dx},{0.0000*\dy})
	-- ({0.6505*\dx},{0.0000*\dy})
	-- ({0.6606*\dx},{0.0000*\dy})
	-- ({0.6706*\dx},{0.0000*\dy})
	-- ({0.6806*\dx},{0.0000*\dy})
	-- ({0.6906*\dx},{0.0000*\dy})
	-- ({0.7006*\dx},{0.0000*\dy})
	-- ({0.7106*\dx},{0.0000*\dy})
	-- ({0.7206*\dx},{0.0000*\dy})
	-- ({0.7306*\dx},{0.0000*\dy})
	-- ({0.7406*\dx},{0.0000*\dy})
	-- ({0.7506*\dx},{0.0000*\dy})
	-- ({0.7606*\dx},{0.0000*\dy})
	-- ({0.7706*\dx},{0.0000*\dy})
	-- ({0.7807*\dx},{0.0000*\dy})
	-- ({0.7907*\dx},{0.0000*\dy})
	-- ({0.8007*\dx},{0.0000*\dy})
	-- ({0.8107*\dx},{0.0000*\dy})
	-- ({0.8207*\dx},{0.0000*\dy})
	-- ({0.8307*\dx},{0.0000*\dy})
	-- ({0.8407*\dx},{0.0000*\dy})
	-- ({0.8507*\dx},{0.0000*\dy})
	-- ({0.8607*\dx},{0.0000*\dy})
	-- ({0.8707*\dx},{0.0000*\dy})
	-- ({0.8807*\dx},{0.0000*\dy})
	-- ({0.8907*\dx},{0.0000*\dy})
	-- ({0.9008*\dx},{0.0000*\dy})
	-- ({0.9108*\dx},{0.0000*\dy})
	-- ({0.9208*\dx},{0.0000*\dy})
	-- ({0.9308*\dx},{0.0000*\dy})
	-- ({0.9408*\dx},{0.0000*\dy})
	-- ({0.9508*\dx},{0.0000*\dy})
	-- ({0.9608*\dx},{0.0000*\dy})
	-- ({0.9708*\dx},{0.0000*\dy})
	-- ({0.9808*\dx},{0.0000*\dy})
	-- ({0.9908*\dx},{0.0000*\dy})
	-- ({1.0008*\dx},{0.0000*\dy})
	-- ({1.0108*\dx},{0.0000*\dy})
	-- ({1.0209*\dx},{0.0000*\dy})
	-- ({1.0309*\dx},{0.0000*\dy})
	-- ({1.0409*\dx},{0.0000*\dy})
	-- ({1.0509*\dx},{0.0000*\dy})
	-- ({1.0609*\dx},{0.0000*\dy})
	-- ({1.0709*\dx},{0.0000*\dy})
	-- ({1.0809*\dx},{0.0000*\dy})
	-- ({1.0909*\dx},{0.0000*\dy})
	-- ({1.1009*\dx},{0.0000*\dy})
	-- ({1.1109*\dx},{0.0000*\dy})
	-- ({1.1209*\dx},{0.0000*\dy})
	-- ({1.1309*\dx},{0.0000*\dy})
	-- ({1.1410*\dx},{0.0000*\dy})
	-- ({1.1510*\dx},{0.0000*\dy})
	-- ({1.1610*\dx},{0.0000*\dy})
	-- ({1.1710*\dx},{0.0000*\dy})
	-- ({1.1810*\dx},{0.0000*\dy})
	-- ({1.1910*\dx},{0.0000*\dy})
	-- ({1.2010*\dx},{0.0000*\dy})
	-- ({1.2110*\dx},{0.0000*\dy})
	-- ({1.2210*\dx},{0.0000*\dy})
	-- ({1.2310*\dx},{0.0000*\dy})
	-- ({1.2410*\dx},{0.0000*\dy})
	-- ({1.2510*\dx},{0.0000*\dy})
	-- ({1.2611*\dx},{0.0000*\dy})
	-- ({1.2711*\dx},{0.0000*\dy})
	-- ({1.2811*\dx},{0.0000*\dy})
	-- ({1.2911*\dx},{0.0000*\dy})
	-- ({1.3011*\dx},{0.0000*\dy})
	-- ({1.3111*\dx},{0.0000*\dy})
	-- ({1.3211*\dx},{0.0000*\dy})
	-- ({1.3311*\dx},{0.0000*\dy})
	-- ({1.3411*\dx},{0.0000*\dy})
	-- ({1.3511*\dx},{0.0000*\dy})
	-- ({1.3611*\dx},{0.0000*\dy})
	-- ({1.3711*\dx},{0.0000*\dy})
	-- ({1.3812*\dx},{0.0000*\dy})
	-- ({1.3912*\dx},{0.0000*\dy})
	-- ({1.4012*\dx},{0.0000*\dy})
	-- ({1.4112*\dx},{0.0000*\dy})
	-- ({1.4212*\dx},{0.0000*\dy})
	-- ({1.4312*\dx},{0.0000*\dy})
	-- ({1.4412*\dx},{0.0000*\dy})
	-- ({1.4512*\dx},{0.0000*\dy})
	-- ({1.4612*\dx},{0.0000*\dy})
	-- ({1.4712*\dx},{0.0000*\dy})
	-- ({1.4812*\dx},{0.0000*\dy})
	-- ({1.4912*\dx},{0.0000*\dy})
	-- ({1.5013*\dx},{0.0000*\dy})
	-- ({1.5113*\dx},{0.0000*\dy})
	-- ({1.5213*\dx},{0.0000*\dy})
	-- ({1.5313*\dx},{0.0000*\dy})
	-- ({1.5413*\dx},{0.0000*\dy})
	-- ({1.5513*\dx},{0.0000*\dy})
	-- ({1.5613*\dx},{0.0000*\dy})
	-- ({1.5713*\dx},{0.0000*\dy})
	-- ({1.5813*\dx},{0.0000*\dy})
	-- ({1.5913*\dx},{0.0000*\dy})
	-- ({1.6013*\dx},{0.0000*\dy})
	-- ({1.6113*\dx},{0.0000*\dy})
	-- ({1.6214*\dx},{0.0000*\dy})
	-- ({1.6314*\dx},{0.0000*\dy})
	-- ({1.6414*\dx},{0.0000*\dy})
	-- ({1.6514*\dx},{0.0000*\dy})
	-- ({1.6614*\dx},{0.0000*\dy})
	-- ({1.6714*\dx},{0.0000*\dy})
	-- ({1.6814*\dx},{0.0000*\dy})
	-- ({1.6914*\dx},{0.0000*\dy})
	-- ({1.7014*\dx},{0.0000*\dy})
	-- ({1.7114*\dx},{0.0000*\dy})
	-- ({1.7214*\dx},{0.0000*\dy})
	-- ({1.7314*\dx},{0.0000*\dy})
	-- ({1.7415*\dx},{0.0000*\dy})
	-- ({1.7515*\dx},{0.0000*\dy})
	-- ({1.7615*\dx},{0.0000*\dy})
	-- ({1.7715*\dx},{0.0000*\dy})
	-- ({1.7815*\dx},{0.0000*\dy})
	-- ({1.7915*\dx},{0.0000*\dy})
	-- ({1.8015*\dx},{0.0000*\dy})
	-- ({1.8115*\dx},{0.0000*\dy})
	-- ({1.8215*\dx},{0.0000*\dy})
	-- ({1.8315*\dx},{0.0000*\dy})
	-- ({1.8415*\dx},{0.0000*\dy})
	-- ({1.8515*\dx},{0.0000*\dy})
	-- ({1.8616*\dx},{0.0000*\dy})
	-- ({1.8716*\dx},{0.0000*\dy})
	-- ({1.8816*\dx},{0.0000*\dy})
	-- ({1.8916*\dx},{0.0000*\dy})
	-- ({1.9016*\dx},{0.0000*\dy})
	-- ({1.9116*\dx},{0.0000*\dy})
	-- ({1.9216*\dx},{0.0000*\dy})
	-- ({1.9316*\dx},{0.0000*\dy})
	-- ({1.9416*\dx},{0.0000*\dy})
	-- ({1.9516*\dx},{0.0000*\dy})
	-- ({1.9616*\dx},{0.0000*\dy})
	-- ({1.9716*\dx},{0.0000*\dy})
	-- ({1.9817*\dx},{0.0000*\dy})
	-- ({1.9917*\dx},{0.0000*\dy})
	-- ({2.0017*\dx},{0.0000*\dy})
	-- ({2.0117*\dx},{0.0000*\dy})
	-- ({2.0217*\dx},{0.0000*\dy})
	-- ({2.0317*\dx},{0.0000*\dy})
	-- ({2.0417*\dx},{0.0000*\dy})
	-- ({2.0517*\dx},{0.0000*\dy})
	-- ({2.0617*\dx},{0.0000*\dy})
	-- ({2.0717*\dx},{0.0000*\dy})
	-- ({2.0817*\dx},{0.0000*\dy})
	-- ({2.0917*\dx},{0.0000*\dy})
	-- ({2.1018*\dx},{0.0000*\dy})
	-- ({2.1118*\dx},{0.0000*\dy})
	-- ({2.1218*\dx},{0.0000*\dy})
	-- ({2.1318*\dx},{0.0000*\dy})
	-- ({2.1418*\dx},{0.0000*\dy})
	-- ({2.1518*\dx},{0.0000*\dy})
	-- ({2.1618*\dx},{0.0000*\dy})
	-- ({2.1718*\dx},{0.0000*\dy})
	-- ({2.1818*\dx},{0.0000*\dy})
	-- ({2.1918*\dx},{0.0000*\dy})
	-- ({2.2018*\dx},{0.0000*\dy})
	-- ({2.2118*\dx},{0.0000*\dy})
	-- ({2.2219*\dx},{0.0000*\dy})
	-- ({2.2319*\dx},{0.0000*\dy})
	-- ({2.2419*\dx},{0.0000*\dy})
	-- ({2.2519*\dx},{0.0000*\dy})
	-- ({2.2619*\dx},{0.0000*\dy})
	-- ({2.2719*\dx},{0.0000*\dy})
	-- ({2.2819*\dx},{0.0000*\dy})
	-- ({2.2919*\dx},{0.0000*\dy})
	-- ({2.3019*\dx},{0.0000*\dy})
	-- ({2.3119*\dx},{0.0000*\dy})
	-- ({2.3219*\dx},{0.0000*\dy})
	-- ({2.3319*\dx},{0.0000*\dy})
	-- ({2.3420*\dx},{0.0000*\dy})
	-- ({2.3520*\dx},{0.0000*\dy})
	-- ({2.3620*\dx},{0.0000*\dy})
	-- ({2.3720*\dx},{0.0000*\dy})
	-- ({2.3820*\dx},{0.0000*\dy})
	-- ({2.3920*\dx},{0.0000*\dy})
	-- ({2.4020*\dx},{0.0000*\dy})
	-- ({2.4120*\dx},{0.0000*\dy})
	-- ({2.4220*\dx},{0.0000*\dy})
	-- ({2.4320*\dx},{0.0000*\dy})
	-- ({2.4420*\dx},{0.0000*\dy})
	-- ({2.4520*\dx},{0.0000*\dy})
	-- ({2.4621*\dx},{0.0000*\dy})
	-- ({2.4721*\dx},{0.0000*\dy})
	-- ({2.4821*\dx},{0.0000*\dy})
	-- ({2.4921*\dx},{0.0000*\dy})
	-- ({2.5021*\dx},{0.0000*\dy})
	-- ({2.5121*\dx},{0.0000*\dy})
	-- ({2.5221*\dx},{0.0000*\dy})
	-- ({2.5321*\dx},{0.0000*\dy})
	-- ({2.5421*\dx},{0.0000*\dy})
	-- ({2.5521*\dx},{0.0000*\dy})
	-- ({2.5621*\dx},{0.0000*\dy})
	-- ({2.5721*\dx},{0.0000*\dy})
	-- ({2.5822*\dx},{0.0000*\dy})
	-- ({2.5922*\dx},{0.0000*\dy})
	-- ({2.6022*\dx},{0.0000*\dy})
	-- ({2.6122*\dx},{0.0000*\dy})
	-- ({2.6222*\dx},{0.0000*\dy})
	-- ({2.6322*\dx},{0.0000*\dy})
	-- ({2.6422*\dx},{0.0000*\dy})
	-- ({2.6522*\dx},{0.0000*\dy})
	-- ({2.6622*\dx},{0.0000*\dy})
	-- ({2.6722*\dx},{0.0000*\dy})
	-- ({2.6822*\dx},{0.0000*\dy})
	-- ({2.6922*\dx},{0.0000*\dy})
	-- ({2.7023*\dx},{0.0000*\dy})
	-- ({2.7123*\dx},{0.0000*\dy})
	-- ({2.7223*\dx},{0.0000*\dy})
	-- ({2.7323*\dx},{0.0000*\dy})
	-- ({2.7423*\dx},{0.0000*\dy})
	-- ({2.7523*\dx},{0.0000*\dy})
	-- ({2.7623*\dx},{0.0000*\dy})
	-- ({2.7723*\dx},{0.0000*\dy})
	-- ({2.7823*\dx},{0.0000*\dy})
	-- ({2.7923*\dx},{0.0000*\dy})
	-- ({2.8023*\dx},{0.0000*\dy})
	-- ({2.8123*\dx},{0.0000*\dy})
	-- ({2.8224*\dx},{0.0000*\dy})
	-- ({2.8324*\dx},{0.0000*\dy})
	-- ({2.8424*\dx},{0.0000*\dy})
	-- ({2.8524*\dx},{0.0000*\dy})
	-- ({2.8624*\dx},{0.0000*\dy})
	-- ({2.8724*\dx},{0.0000*\dy})
	-- ({2.8824*\dx},{0.0000*\dy})
	-- ({2.8924*\dx},{0.0000*\dy})
	-- ({2.9024*\dx},{0.0000*\dy})
	-- ({2.9124*\dx},{0.0000*\dy})
	-- ({2.9224*\dx},{0.0000*\dy})
	-- ({2.9324*\dx},{0.0000*\dy})
	-- ({2.9425*\dx},{0.0000*\dy})
	-- ({2.9525*\dx},{0.0000*\dy})
	-- ({2.9625*\dx},{0.0000*\dy})
	-- ({2.9725*\dx},{0.0000*\dy})
	-- ({2.9825*\dx},{0.0000*\dy})
	-- ({2.9925*\dx},{0.0000*\dy})
	-- ({3.0025*\dx},{0.0000*\dy})
	-- ({3.0125*\dx},{0.0000*\dy})
	-- ({3.0225*\dx},{0.0000*\dy})
	-- ({3.0325*\dx},{0.0000*\dy})
	-- ({3.0425*\dx},{0.0000*\dy})
	-- ({3.0525*\dx},{0.0000*\dy})
	-- ({3.0626*\dx},{0.0000*\dy})
	-- ({3.0726*\dx},{0.0000*\dy})
	-- ({3.0826*\dx},{0.0000*\dy})
	-- ({3.0926*\dx},{0.0000*\dy})
	-- ({3.1026*\dx},{0.0000*\dy})
	-- ({3.1126*\dx},{0.0000*\dy})
	-- ({3.1226*\dx},{0.0000*\dy})
	-- ({3.1326*\dx},{0.0000*\dy})
	-- ({3.1426*\dx},{0.0000*\dy})
	-- ({3.1526*\dx},{0.0000*\dy})
	-- ({3.1626*\dx},{0.0000*\dy})
	-- ({3.1726*\dx},{0.0000*\dy})
	-- ({3.1827*\dx},{0.0000*\dy})
	-- ({3.1927*\dx},{0.0000*\dy})
	-- ({3.2027*\dx},{0.0000*\dy})
	-- ({3.2127*\dx},{0.0000*\dy})
	-- ({3.2227*\dx},{0.0000*\dy})
	-- ({3.2327*\dx},{0.0000*\dy})
	-- ({3.2427*\dx},{0.0000*\dy})
	-- ({3.2527*\dx},{0.0000*\dy})
	-- ({3.2627*\dx},{0.0000*\dy})
	-- ({3.2727*\dx},{0.0000*\dy})
	-- ({3.2827*\dx},{0.0000*\dy})
	-- ({3.2927*\dx},{0.0000*\dy})
	-- ({3.3028*\dx},{0.0000*\dy})
	-- ({3.3128*\dx},{0.0000*\dy})
	-- ({3.3228*\dx},{0.0000*\dy})
	-- ({3.3328*\dx},{0.0000*\dy})
	-- ({3.3428*\dx},{0.0000*\dy})
	-- ({3.3528*\dx},{0.0000*\dy})
	-- ({3.3628*\dx},{0.0000*\dy})
	-- ({3.3728*\dx},{0.0000*\dy})
	-- ({3.3828*\dx},{0.0000*\dy})
	-- ({3.3928*\dx},{0.0000*\dy})
	-- ({3.4028*\dx},{0.0000*\dy})
	-- ({3.4128*\dx},{0.0000*\dy})
	-- ({3.4229*\dx},{0.0000*\dy})
	-- ({3.4329*\dx},{0.0000*\dy})
	-- ({3.4429*\dx},{0.0000*\dy})
	-- ({3.4529*\dx},{0.0000*\dy})
	-- ({3.4629*\dx},{0.0000*\dy})
	-- ({3.4729*\dx},{0.0000*\dy})
	-- ({3.4829*\dx},{0.0000*\dy})
	-- ({3.4929*\dx},{0.0000*\dy})
	-- ({3.5029*\dx},{0.0000*\dy})
	-- ({3.5129*\dx},{0.0000*\dy})
	-- ({3.5229*\dx},{0.0000*\dy})
	-- ({3.5329*\dx},{0.0000*\dy})
	-- ({3.5430*\dx},{0.0000*\dy})
	-- ({3.5530*\dx},{0.0000*\dy})
	-- ({3.5630*\dx},{0.0000*\dy})
	-- ({3.5730*\dx},{0.0000*\dy})
	-- ({3.5830*\dx},{0.0000*\dy})
	-- ({3.5930*\dx},{0.0000*\dy})
	-- ({3.6030*\dx},{0.0000*\dy})
	-- ({3.6130*\dx},{0.0000*\dy})
	-- ({3.6230*\dx},{0.0000*\dy})
	-- ({3.6330*\dx},{0.0000*\dy})
	-- ({3.6430*\dx},{0.0000*\dy})
	-- ({3.6530*\dx},{0.0000*\dy})
	-- ({3.6631*\dx},{0.0000*\dy})
	-- ({3.6731*\dx},{0.0000*\dy})
	-- ({3.6831*\dx},{0.0000*\dy})
	-- ({3.6931*\dx},{0.0000*\dy})
	-- ({3.7031*\dx},{0.0000*\dy})
	-- ({3.7131*\dx},{0.0000*\dy})
	-- ({3.7231*\dx},{0.0000*\dy})
	-- ({3.7331*\dx},{0.0000*\dy})
	-- ({3.7431*\dx},{0.0000*\dy})
	-- ({3.7531*\dx},{0.0000*\dy})
	-- ({3.7631*\dx},{0.0000*\dy})
	-- ({3.7731*\dx},{0.0000*\dy})
	-- ({3.7832*\dx},{0.0000*\dy})
	-- ({3.7932*\dx},{0.0000*\dy})
	-- ({3.8032*\dx},{0.0000*\dy})
	-- ({3.8132*\dx},{0.0000*\dy})
	-- ({3.8232*\dx},{0.0000*\dy})
	-- ({3.8332*\dx},{0.0000*\dy})
	-- ({3.8432*\dx},{0.0000*\dy})
	-- ({3.8532*\dx},{0.0000*\dy})
	-- ({3.8632*\dx},{0.0000*\dy})
	-- ({3.8732*\dx},{0.0000*\dy})
	-- ({3.8832*\dx},{0.0000*\dy})
	-- ({3.8932*\dx},{0.0000*\dy})
	-- ({3.9033*\dx},{0.0000*\dy})
	-- ({3.9133*\dx},{0.0000*\dy})
	-- ({3.9233*\dx},{0.0000*\dy})
	-- ({3.9333*\dx},{0.0000*\dy})
	-- ({3.9433*\dx},{0.0000*\dy})
	-- ({3.9533*\dx},{0.0000*\dy})
	-- ({3.9633*\dx},{0.0000*\dy})
	-- ({3.9733*\dx},{0.0000*\dy})
	-- ({3.9833*\dx},{0.0000*\dy})
	-- ({3.9933*\dx},{0.0000*\dy})
	-- ({4.0033*\dx},{0.0000*\dy})
	-- ({4.0133*\dx},{0.0000*\dy})
	-- ({4.0234*\dx},{0.0000*\dy})
	-- ({4.0334*\dx},{0.0000*\dy})
	-- ({4.0434*\dx},{0.0000*\dy})
	-- ({4.0534*\dx},{0.0000*\dy})
	-- ({4.0634*\dx},{0.0000*\dy})
	-- ({4.0734*\dx},{0.0000*\dy})
	-- ({4.0834*\dx},{0.0000*\dy})
	-- ({4.0934*\dx},{0.0000*\dy})
	-- ({4.1034*\dx},{0.0000*\dy})
	-- ({4.1134*\dx},{0.0000*\dy})
	-- ({4.1234*\dx},{0.0000*\dy})
	-- ({4.1334*\dx},{0.0000*\dy})
	-- ({4.1435*\dx},{0.0000*\dy})
	-- ({4.1535*\dx},{0.0000*\dy})
	-- ({4.1635*\dx},{0.0000*\dy})
	-- ({4.1735*\dx},{0.0000*\dy})
	-- ({4.1835*\dx},{0.0000*\dy})
	-- ({4.1935*\dx},{0.0000*\dy})
	-- ({4.2035*\dx},{0.0000*\dy})
	-- ({4.2135*\dx},{0.0000*\dy})
	-- ({4.2235*\dx},{0.0000*\dy})
	-- ({4.2335*\dx},{0.0000*\dy})
	-- ({4.2435*\dx},{0.0000*\dy})
	-- ({4.2535*\dx},{0.0000*\dy})
	-- ({4.2636*\dx},{0.0000*\dy})
	-- ({4.2736*\dx},{0.0000*\dy})
	-- ({4.2836*\dx},{0.0000*\dy})
	-- ({4.2936*\dx},{0.0000*\dy})
	-- ({4.3036*\dx},{0.0000*\dy})
	-- ({4.3136*\dx},{0.0000*\dy})
	-- ({4.3236*\dx},{0.0000*\dy})
	-- ({4.3336*\dx},{0.0000*\dy})
	-- ({4.3436*\dx},{0.0000*\dy})
	-- ({4.3536*\dx},{0.0000*\dy})
	-- ({4.3636*\dx},{0.0000*\dy})
	-- ({4.3736*\dx},{0.0000*\dy})
	-- ({4.3837*\dx},{0.0000*\dy})
	-- ({4.3937*\dx},{0.0000*\dy})
	-- ({4.4037*\dx},{0.0000*\dy})
	-- ({4.4137*\dx},{0.0000*\dy})
	-- ({4.4237*\dx},{0.0000*\dy})
	-- ({4.4337*\dx},{0.0000*\dy})
	-- ({4.4437*\dx},{0.0000*\dy})
	-- ({4.4537*\dx},{0.0000*\dy})
	-- ({4.4637*\dx},{0.0000*\dy})
	-- ({4.4737*\dx},{0.0000*\dy})
	-- ({4.4837*\dx},{0.0000*\dy})
	-- ({4.4937*\dx},{0.0000*\dy})
	-- ({4.5038*\dx},{0.0000*\dy})
	-- ({4.5138*\dx},{0.0000*\dy})
	-- ({4.5238*\dx},{0.0000*\dy})
	-- ({4.5338*\dx},{0.0000*\dy})
	-- ({4.5438*\dx},{0.0000*\dy})
	-- ({4.5538*\dx},{0.0000*\dy})
	-- ({4.5638*\dx},{0.0000*\dy})
	-- ({4.5738*\dx},{0.0000*\dy})
	-- ({4.5838*\dx},{0.0000*\dy})
	-- ({4.5938*\dx},{0.0000*\dy})
	-- ({4.6038*\dx},{0.0000*\dy})
	-- ({4.6138*\dx},{0.0000*\dy})
	-- ({4.6239*\dx},{0.0000*\dy})
	-- ({4.6339*\dx},{0.0000*\dy})
	-- ({4.6439*\dx},{0.0000*\dy})
	-- ({4.6539*\dx},{0.0000*\dy})
	-- ({4.6639*\dx},{0.0000*\dy})
	-- ({4.6739*\dx},{0.0000*\dy})
	-- ({4.6839*\dx},{0.0000*\dy})
	-- ({4.6939*\dx},{0.0000*\dy})
	-- ({4.7039*\dx},{0.0000*\dy})
	-- ({4.7139*\dx},{0.0000*\dy})
	-- ({4.7239*\dx},{0.0000*\dy})
	-- ({4.7339*\dx},{0.0000*\dy})
	-- ({4.7440*\dx},{0.0000*\dy})
	-- ({4.7540*\dx},{0.0000*\dy})
	-- ({4.7640*\dx},{0.0000*\dy})
	-- ({4.7740*\dx},{0.0000*\dy})
	-- ({4.7840*\dx},{0.0000*\dy})
	-- ({4.7940*\dx},{0.0000*\dy})
	-- ({4.8040*\dx},{0.0000*\dy})
	-- ({4.8140*\dx},{0.0000*\dy})
	-- ({4.8240*\dx},{0.0000*\dy})
	-- ({4.8340*\dx},{0.0000*\dy})
	-- ({4.8440*\dx},{0.0000*\dy})
	-- ({4.8540*\dx},{0.0000*\dy})
	-- ({4.8641*\dx},{0.0000*\dy})
	-- ({4.8741*\dx},{0.0000*\dy})
	-- ({4.8841*\dx},{0.0000*\dy})
	-- ({4.8941*\dx},{0.0000*\dy})
	-- ({4.9041*\dx},{0.0000*\dy})
	-- ({4.9141*\dx},{0.0000*\dy})
	-- ({4.9241*\dx},{0.0000*\dy})
	-- ({4.9341*\dx},{0.0000*\dy})
	-- ({4.9441*\dx},{0.0000*\dy})
	-- ({4.9541*\dx},{0.0000*\dy})
	-- ({4.9641*\dx},{0.0000*\dy})
	-- ({4.9741*\dx},{0.0000*\dy})
	-- ({4.9842*\dx},{0.0000*\dy})
	-- ({4.9942*\dx},{0.0000*\dy})
	-- ({5.0042*\dx},{0.0000*\dy})
	-- ({5.0142*\dx},{0.0000*\dy})
	-- ({5.0242*\dx},{0.0000*\dy})
	-- ({5.0342*\dx},{0.0000*\dy})
	-- ({5.0442*\dx},{0.0000*\dy})
	-- ({5.0542*\dx},{0.0000*\dy})
	-- ({5.0642*\dx},{0.0000*\dy})
	-- ({5.0742*\dx},{0.0000*\dy})
	-- ({5.0842*\dx},{0.0000*\dy})
	-- ({5.0942*\dx},{0.0000*\dy})
	-- ({5.1043*\dx},{0.0000*\dy})
	-- ({5.1143*\dx},{0.0000*\dy})
	-- ({5.1243*\dx},{0.0000*\dy})
	-- ({5.1343*\dx},{0.0000*\dy})
	-- ({5.1443*\dx},{0.0000*\dy})
	-- ({5.1543*\dx},{0.0000*\dy})
	-- ({5.1643*\dx},{0.0000*\dy})
	-- ({5.1743*\dx},{0.0000*\dy})
	-- ({5.1843*\dx},{0.0000*\dy})
	-- ({5.1943*\dx},{0.0000*\dy})
	-- ({5.2043*\dx},{0.0000*\dy})
	-- ({5.2143*\dx},{0.0000*\dy})
	-- ({5.2244*\dx},{0.0000*\dy})
	-- ({5.2344*\dx},{0.0000*\dy})
	-- ({5.2444*\dx},{0.0000*\dy})
	-- ({5.2544*\dx},{0.0000*\dy})
	-- ({5.2644*\dx},{0.0000*\dy})
	-- ({5.2744*\dx},{0.0000*\dy})
	-- ({5.2844*\dx},{0.0000*\dy})
	-- ({5.2944*\dx},{0.0000*\dy})
	-- ({5.3044*\dx},{0.0000*\dy})
	-- ({5.3144*\dx},{0.0000*\dy})
	-- ({5.3244*\dx},{0.0000*\dy})
	-- ({5.3344*\dx},{0.0000*\dy})
	-- ({5.3445*\dx},{0.0000*\dy})
	-- ({5.3545*\dx},{0.0000*\dy})
	-- ({5.3645*\dx},{0.0000*\dy})
	-- ({5.3745*\dx},{0.0000*\dy})
	-- ({5.3845*\dx},{0.0000*\dy})
	-- ({5.3945*\dx},{0.0000*\dy})
	-- ({5.4045*\dx},{0.0000*\dy})
	-- ({5.4145*\dx},{0.0000*\dy})
	-- ({5.4245*\dx},{0.0000*\dy})
	-- ({5.4345*\dx},{0.0000*\dy})
	-- ({5.4445*\dx},{0.0000*\dy})
	-- ({5.4545*\dx},{0.0000*\dy})
	-- ({5.4646*\dx},{0.0000*\dy})
	-- ({5.4746*\dx},{0.0000*\dy})
	-- ({5.4846*\dx},{0.0000*\dy})
	-- ({5.4946*\dx},{0.0000*\dy})
	-- ({5.5046*\dx},{0.0000*\dy})
	-- ({5.5146*\dx},{0.0000*\dy})
	-- ({5.5246*\dx},{0.0000*\dy})
	-- ({5.5346*\dx},{0.0000*\dy})
	-- ({5.5446*\dx},{0.0000*\dy})
	-- ({5.5546*\dx},{0.0000*\dy})
	-- ({5.5646*\dx},{0.0000*\dy})
	-- ({5.5746*\dx},{0.0000*\dy})
	-- ({5.5847*\dx},{0.0000*\dy})
	-- ({5.5947*\dx},{0.0000*\dy})
	-- ({5.6047*\dx},{0.0000*\dy})
	-- ({5.6147*\dx},{0.0000*\dy})
	-- ({5.6247*\dx},{0.0000*\dy})
	-- ({5.6347*\dx},{0.0000*\dy})
	-- ({5.6447*\dx},{0.0000*\dy})
	-- ({5.6547*\dx},{0.0000*\dy})
	-- ({5.6647*\dx},{0.0000*\dy})
	-- ({5.6747*\dx},{0.0000*\dy})
	-- ({5.6847*\dx},{0.0000*\dy})
	-- ({5.6947*\dx},{0.0000*\dy})
	-- ({5.7048*\dx},{0.0000*\dy})
	-- ({5.7148*\dx},{0.0000*\dy})
	-- ({5.7248*\dx},{0.0000*\dy})
	-- ({5.7348*\dx},{0.0000*\dy})
	-- ({5.7448*\dx},{0.0000*\dy})
	-- ({5.7548*\dx},{0.0000*\dy})
	-- ({5.7648*\dx},{0.0000*\dy})
	-- ({5.7748*\dx},{0.0000*\dy})
	-- ({5.7848*\dx},{0.0000*\dy})
	-- ({5.7948*\dx},{0.0000*\dy})
	-- ({5.8048*\dx},{0.0000*\dy})
	-- ({5.8148*\dx},{0.0000*\dy})
	-- ({5.8249*\dx},{0.0000*\dy})
	-- ({5.8349*\dx},{0.0000*\dy})
	-- ({5.8449*\dx},{0.0000*\dy})
	-- ({5.8549*\dx},{0.0000*\dy})
	-- ({5.8649*\dx},{0.0000*\dy})
	-- ({5.8749*\dx},{0.0000*\dy})
	-- ({5.8849*\dx},{0.0000*\dy})
	-- ({5.8949*\dx},{0.0000*\dy})
	-- ({5.9049*\dx},{0.0000*\dy})
	-- ({5.9149*\dx},{0.0000*\dy})
	-- ({5.9249*\dx},{0.0000*\dy})
	-- ({5.9349*\dx},{0.0000*\dy})
	-- ({5.9450*\dx},{0.0000*\dy})
	-- ({5.9550*\dx},{0.0000*\dy})
	-- ({5.9650*\dx},{0.0000*\dy})
	-- ({5.9750*\dx},{0.0000*\dy})
	-- ({5.9850*\dx},{0.0000*\dy})
	-- ({5.9950*\dx},{0.0000*\dy})
	-- ({6.0050*\dx},{0.0000*\dy})
	-- ({6.0150*\dx},{0.0000*\dy})
	-- ({6.0250*\dx},{0.0000*\dy})
	-- ({6.0350*\dx},{0.0000*\dy})
	-- ({6.0450*\dx},{0.0000*\dy})
	-- ({6.0550*\dx},{0.0000*\dy})
	-- ({6.0651*\dx},{0.0000*\dy})
	-- ({6.0751*\dx},{0.0000*\dy})
	-- ({6.0851*\dx},{0.0000*\dy})
	-- ({6.0951*\dx},{0.0000*\dy})
	-- ({6.1051*\dx},{0.0000*\dy})
	-- ({6.1151*\dx},{0.0000*\dy})
	-- ({6.1251*\dx},{0.0000*\dy})
	-- ({6.1351*\dx},{0.0000*\dy})
	-- ({6.1451*\dx},{0.0000*\dy})
	-- ({6.1551*\dx},{0.0000*\dy})
	-- ({6.1651*\dx},{0.0000*\dy})
	-- ({6.1751*\dx},{0.0000*\dy})
	-- ({6.1852*\dx},{0.0000*\dy})
	-- ({6.1952*\dx},{0.0000*\dy})
	-- ({6.2052*\dx},{0.0000*\dy})
	-- ({6.2152*\dx},{0.0000*\dy})
	-- ({6.2252*\dx},{0.0000*\dy})
	-- ({6.2352*\dx},{0.0000*\dy})
	-- ({6.2452*\dx},{0.0000*\dy})
	-- ({6.2552*\dx},{0.0000*\dy})
	-- ({6.2652*\dx},{0.0000*\dy})
	-- ({6.2752*\dx},{0.0000*\dy})
	-- ({6.2852*\dx},{0.0000*\dy})
	-- ({6.2952*\dx},{0.0000*\dy})
	-- ({6.3053*\dx},{0.0000*\dy})
	-- ({6.3153*\dx},{0.0000*\dy})
	-- ({6.3253*\dx},{0.0000*\dy})
	-- ({6.3353*\dx},{0.0000*\dy})
	-- ({6.3453*\dx},{0.0000*\dy})
	-- ({6.3553*\dx},{0.0000*\dy})
	-- ({6.3653*\dx},{0.0000*\dy})
	-- ({6.3753*\dx},{0.0000*\dy})
	-- ({6.3853*\dx},{0.0000*\dy})
	-- ({6.3953*\dx},{0.0000*\dy})
	-- ({6.4053*\dx},{0.0000*\dy})
	-- ({6.4153*\dx},{0.0000*\dy})
	-- ({6.4254*\dx},{0.0000*\dy})
	-- ({6.4354*\dx},{0.0000*\dy})
	-- ({6.4454*\dx},{0.0000*\dy})
	-- ({6.4554*\dx},{0.0000*\dy})
	-- ({6.4654*\dx},{0.0000*\dy})
	-- ({6.4754*\dx},{0.0000*\dy})
	-- ({6.4854*\dx},{0.0000*\dy})
	-- ({6.4954*\dx},{0.0000*\dy})
	-- ({6.5054*\dx},{0.0000*\dy})
	-- ({6.5154*\dx},{0.0000*\dy})
	-- ({6.5254*\dx},{0.0000*\dy})
	-- ({6.5354*\dx},{0.0000*\dy})
	-- ({6.5455*\dx},{0.0000*\dy})
	-- ({6.5555*\dx},{0.0000*\dy})
	-- ({6.5655*\dx},{0.0000*\dy})
	-- ({6.5755*\dx},{0.0000*\dy})
	-- ({6.5855*\dx},{0.0000*\dy})
	-- ({6.5955*\dx},{0.0000*\dy})
	-- ({6.6055*\dx},{0.0000*\dy})
	-- ({6.6155*\dx},{0.0000*\dy})
	-- ({6.6255*\dx},{0.0000*\dy})
	-- ({6.6355*\dx},{0.0000*\dy})
	-- ({6.6455*\dx},{0.0000*\dy})
	-- ({6.6555*\dx},{0.0000*\dy})
	-- ({6.6656*\dx},{0.0000*\dy})
	-- ({6.6756*\dx},{0.0000*\dy})
	-- ({6.6856*\dx},{0.0000*\dy})
	-- ({6.6956*\dx},{0.0000*\dy})
	-- ({6.7056*\dx},{0.0000*\dy})
	-- ({6.7156*\dx},{0.0000*\dy})
	-- ({6.7256*\dx},{0.0000*\dy})
	-- ({6.7356*\dx},{0.0000*\dy})
	-- ({6.7456*\dx},{0.0000*\dy})
	-- ({6.7556*\dx},{0.0000*\dy})
	-- ({6.7656*\dx},{0.0000*\dy})
	-- ({6.7756*\dx},{0.0000*\dy})
	-- ({6.7857*\dx},{0.0000*\dy})
	-- ({6.7957*\dx},{0.0000*\dy})
	-- ({6.8057*\dx},{0.0000*\dy})
	-- ({6.8157*\dx},{0.0000*\dy})
	-- ({6.8257*\dx},{0.0000*\dy})
	-- ({6.8357*\dx},{0.0000*\dy})
	-- ({6.8457*\dx},{0.0000*\dy})
	-- ({6.8557*\dx},{0.0000*\dy})
	-- ({6.8657*\dx},{0.0000*\dy})
	-- ({6.8757*\dx},{0.0000*\dy})
	-- ({6.8857*\dx},{0.0000*\dy})
	-- ({6.8957*\dx},{0.0000*\dy})
	-- ({6.9058*\dx},{0.0000*\dy})
	-- ({6.9158*\dx},{0.0000*\dy})
	-- ({6.9258*\dx},{0.0000*\dy})
	-- ({6.9358*\dx},{0.0000*\dy})
	-- ({6.9458*\dx},{0.0000*\dy})
	-- ({6.9558*\dx},{0.0000*\dy})
	-- ({6.9658*\dx},{0.0000*\dy})
	-- ({6.9758*\dx},{0.0000*\dy})
	-- ({6.9858*\dx},{0.0000*\dy})
	-- ({6.9958*\dx},{0.0000*\dy})
	-- ({7.0058*\dx},{0.0000*\dy})
	-- ({7.0158*\dx},{0.0000*\dy})
	-- ({7.0259*\dx},{0.0000*\dy})
	-- ({7.0359*\dx},{0.0000*\dy})
	-- ({7.0459*\dx},{0.0000*\dy})
	-- ({7.0559*\dx},{0.0000*\dy})
	-- ({7.0659*\dx},{0.0000*\dy})
	-- ({7.0759*\dx},{0.0000*\dy})
	-- ({7.0859*\dx},{0.0000*\dy})
	-- ({7.0959*\dx},{0.0000*\dy})
	-- ({7.1059*\dx},{0.0000*\dy})
	-- ({7.1159*\dx},{0.0000*\dy})
	-- ({7.1259*\dx},{0.0000*\dy})
	-- ({7.1359*\dx},{0.0000*\dy})
	-- ({7.1460*\dx},{0.0000*\dy})
	-- ({7.1560*\dx},{0.0000*\dy})
	-- ({7.1660*\dx},{0.0000*\dy})
	-- ({7.1760*\dx},{0.0000*\dy})
	-- ({7.1860*\dx},{0.0000*\dy})
	-- ({7.1960*\dx},{0.0000*\dy})
	-- ({7.2060*\dx},{0.0000*\dy})
	-- ({7.2160*\dx},{0.0000*\dy})
	-- ({7.2260*\dx},{0.0000*\dy})
	-- ({7.2360*\dx},{0.0000*\dy})
	-- ({7.2460*\dx},{0.0000*\dy})
	-- ({7.2560*\dx},{0.0000*\dy})
	-- ({7.2661*\dx},{0.0000*\dy})
	-- ({7.2761*\dx},{0.0000*\dy})
	-- ({7.2861*\dx},{0.0000*\dy})
	-- ({7.2961*\dx},{0.0000*\dy})
	-- ({7.3061*\dx},{0.0000*\dy})
	-- ({7.3161*\dx},{0.0000*\dy})
	-- ({7.3261*\dx},{0.0000*\dy})
	-- ({7.3361*\dx},{0.0000*\dy})
	-- ({7.3461*\dx},{0.0000*\dy})
	-- ({7.3561*\dx},{0.0000*\dy})
	-- ({7.3661*\dx},{0.0000*\dy})
	-- ({7.3761*\dx},{0.0000*\dy})
	-- ({7.3862*\dx},{0.0000*\dy})
	-- ({7.3962*\dx},{0.0000*\dy})
	-- ({7.4062*\dx},{0.0000*\dy})
	-- ({7.4162*\dx},{0.0000*\dy})
	-- ({7.4262*\dx},{0.0000*\dy})
	-- ({7.4362*\dx},{0.0000*\dy})
	-- ({7.4462*\dx},{0.0000*\dy})
	-- ({7.4562*\dx},{0.0000*\dy})
	-- ({7.4662*\dx},{0.0000*\dy})
	-- ({7.4762*\dx},{0.0000*\dy})
	-- ({7.4862*\dx},{0.0000*\dy})
	-- ({7.4962*\dx},{0.0000*\dy})
	-- ({7.5063*\dx},{0.0000*\dy})
	-- ({7.5163*\dx},{0.0000*\dy})
	-- ({7.5263*\dx},{0.0000*\dy})
	-- ({7.5363*\dx},{0.0000*\dy})
	-- ({7.5463*\dx},{0.0000*\dy})
	-- ({7.5563*\dx},{0.0000*\dy})
	-- ({7.5663*\dx},{0.0000*\dy})
	-- ({7.5763*\dx},{0.0000*\dy})
	-- ({7.5863*\dx},{0.0000*\dy})
	-- ({7.5963*\dx},{0.0000*\dy})
	-- ({7.6063*\dx},{0.0000*\dy})
	-- ({7.6163*\dx},{0.0000*\dy})
	-- ({7.6264*\dx},{0.0000*\dy})
	-- ({7.6364*\dx},{0.0000*\dy})
	-- ({7.6464*\dx},{0.0000*\dy})
	-- ({7.6564*\dx},{0.0000*\dy})
	-- ({7.6664*\dx},{0.0000*\dy})
	-- ({7.6764*\dx},{0.0000*\dy})
	-- ({7.6864*\dx},{0.0000*\dy})
	-- ({7.6964*\dx},{0.0000*\dy})
	-- ({7.7064*\dx},{0.0000*\dy})
	-- ({7.7164*\dx},{0.0000*\dy})
	-- ({7.7264*\dx},{0.0000*\dy})
	-- ({7.7364*\dx},{0.0000*\dy})
	-- ({7.7465*\dx},{0.0000*\dy})
	-- ({7.7565*\dx},{0.0000*\dy})
	-- ({7.7665*\dx},{0.0000*\dy})
	-- ({7.7765*\dx},{0.0000*\dy})
	-- ({7.7865*\dx},{0.0000*\dy})
	-- ({7.7965*\dx},{0.0000*\dy})
	-- ({7.8065*\dx},{0.0000*\dy})
	-- ({7.8165*\dx},{0.0000*\dy})
	-- ({7.8265*\dx},{0.0000*\dy})
	-- ({7.8365*\dx},{0.0000*\dy})
	-- ({7.8465*\dx},{0.0000*\dy})
	-- ({7.8565*\dx},{0.0000*\dy})
	-- ({7.8666*\dx},{0.0000*\dy})
	-- ({7.8766*\dx},{0.0000*\dy})
	-- ({7.8866*\dx},{0.0000*\dy})
	-- ({7.8966*\dx},{0.0000*\dy})
	-- ({7.9066*\dx},{0.0000*\dy})
	-- ({7.9166*\dx},{0.0000*\dy})
	-- ({7.9266*\dx},{0.0000*\dy})
	-- ({7.9366*\dx},{0.0000*\dy})
	-- ({7.9466*\dx},{0.0000*\dy})
	-- ({7.9566*\dx},{0.0000*\dy})
	-- ({7.9666*\dx},{0.0000*\dy})
	-- ({7.9766*\dx},{0.0000*\dy})
	-- ({7.9867*\dx},{0.0000*\dy})
	-- ({7.9967*\dx},{0.0000*\dy})
	-- ({8.0067*\dx},{0.0000*\dy})
	-- ({8.0167*\dx},{0.0000*\dy})
	-- ({8.0267*\dx},{0.0000*\dy})
	-- ({8.0367*\dx},{0.0000*\dy})
	-- ({8.0467*\dx},{0.0000*\dy})
	-- ({8.0567*\dx},{0.0000*\dy})
	-- ({8.0667*\dx},{0.0000*\dy})
	-- ({8.0767*\dx},{0.0000*\dy})
	-- ({8.0867*\dx},{0.0000*\dy})
	-- ({8.0967*\dx},{0.0000*\dy})
	-- ({8.1068*\dx},{0.0000*\dy})
	-- ({8.1168*\dx},{0.0000*\dy})
	-- ({8.1268*\dx},{0.0000*\dy})
	-- ({8.1368*\dx},{0.0000*\dy})
	-- ({8.1468*\dx},{0.0000*\dy})
	-- ({8.1568*\dx},{0.0000*\dy})
	-- ({8.1668*\dx},{0.0000*\dy})
	-- ({8.1768*\dx},{0.0000*\dy})
	-- ({8.1868*\dx},{0.0000*\dy})
	-- ({8.1968*\dx},{0.0000*\dy})
	-- ({8.2068*\dx},{0.0000*\dy})
	-- ({8.2168*\dx},{0.0000*\dy})
	-- ({8.2269*\dx},{0.0000*\dy})
	-- ({8.2369*\dx},{0.0000*\dy})
	-- ({8.2469*\dx},{0.0000*\dy})
	-- ({8.2569*\dx},{0.0000*\dy})
	-- ({8.2669*\dx},{0.0000*\dy})
	-- ({8.2769*\dx},{0.0000*\dy})
	-- ({8.2869*\dx},{0.0000*\dy})
	-- ({8.2969*\dx},{0.0000*\dy})
	-- ({8.3069*\dx},{0.0000*\dy})
	-- ({8.3169*\dx},{0.0000*\dy})
	-- ({8.3269*\dx},{0.0000*\dy})
	-- ({8.3369*\dx},{0.0000*\dy})
	-- ({8.3470*\dx},{0.0000*\dy})
	-- ({8.3570*\dx},{0.0000*\dy})
	-- ({8.3670*\dx},{0.0000*\dy})
	-- ({8.3770*\dx},{0.0000*\dy})
	-- ({8.3870*\dx},{0.0000*\dy})
	-- ({8.3970*\dx},{0.0000*\dy})
	-- ({8.4070*\dx},{0.0000*\dy})
	-- ({8.4170*\dx},{0.0000*\dy})
	-- ({8.4270*\dx},{0.0000*\dy})
	-- ({8.4370*\dx},{0.0000*\dy})
	-- ({8.4470*\dx},{0.0000*\dy})
	-- ({8.4570*\dx},{0.0000*\dy})
	-- ({8.4671*\dx},{0.0000*\dy})
	-- ({8.4771*\dx},{0.0000*\dy})
	-- ({8.4871*\dx},{0.0000*\dy})
	-- ({8.4971*\dx},{0.0000*\dy})
	-- ({8.5071*\dx},{0.0000*\dy})
	-- ({8.5171*\dx},{0.0000*\dy})
	-- ({8.5271*\dx},{0.0000*\dy})
	-- ({8.5371*\dx},{0.0000*\dy})
	-- ({8.5471*\dx},{0.0000*\dy})
	-- ({8.5571*\dx},{0.0000*\dy})
	-- ({8.5671*\dx},{0.0000*\dy})
	-- ({8.5771*\dx},{0.0000*\dy})
	-- ({8.5872*\dx},{0.0000*\dy})
	-- ({8.5972*\dx},{0.0000*\dy})
	-- ({8.6072*\dx},{0.0000*\dy})
	-- ({8.6172*\dx},{0.0000*\dy})
	-- ({8.6272*\dx},{0.0000*\dy})
	-- ({8.6372*\dx},{0.0000*\dy})
	-- ({8.6472*\dx},{0.0000*\dy})
	-- ({8.6572*\dx},{0.0000*\dy})
	-- ({8.6672*\dx},{0.0000*\dy})
	-- ({8.6772*\dx},{0.0000*\dy})
	-- ({8.6872*\dx},{0.0000*\dy})
	-- ({8.6972*\dx},{0.0000*\dy})
	-- ({8.7073*\dx},{0.0000*\dy})
	-- ({8.7173*\dx},{0.0000*\dy})
	-- ({8.7273*\dx},{0.0000*\dy})
	-- ({8.7373*\dx},{0.0000*\dy})
	-- ({8.7473*\dx},{0.0000*\dy})
	-- ({8.7573*\dx},{0.0000*\dy})
	-- ({8.7673*\dx},{0.0000*\dy})
	-- ({8.7773*\dx},{0.0000*\dy})
	-- ({8.7873*\dx},{0.0000*\dy})
	-- ({8.7973*\dx},{0.0000*\dy})
	-- ({8.8073*\dx},{0.0000*\dy})
	-- ({8.8173*\dx},{0.0000*\dy})
	-- ({8.8274*\dx},{0.0000*\dy})
	-- ({8.8374*\dx},{0.0000*\dy})
	-- ({8.8474*\dx},{0.0000*\dy})
	-- ({8.8574*\dx},{0.0000*\dy})
	-- ({8.8674*\dx},{0.0000*\dy})
	-- ({8.8774*\dx},{0.0000*\dy})
	-- ({8.8874*\dx},{0.0000*\dy})
	-- ({8.8974*\dx},{0.0000*\dy})
	-- ({8.9074*\dx},{0.0000*\dy})
	-- ({8.9174*\dx},{0.0000*\dy})
	-- ({8.9274*\dx},{0.0000*\dy})
	-- ({8.9374*\dx},{0.0000*\dy})
	-- ({8.9475*\dx},{0.0000*\dy})
	-- ({8.9575*\dx},{0.0000*\dy})
	-- ({8.9675*\dx},{0.0000*\dy})
	-- ({8.9775*\dx},{0.0000*\dy})
	-- ({8.9875*\dx},{0.0000*\dy})
	-- ({8.9975*\dx},{0.0000*\dy})
	-- ({9.0075*\dx},{0.0000*\dy})
	-- ({9.0175*\dx},{0.0001*\dy})
	-- ({9.0275*\dx},{0.0002*\dy})
	-- ({9.0375*\dx},{0.0003*\dy})
	-- ({9.0475*\dx},{0.0006*\dy})
	-- ({9.0575*\dx},{0.0008*\dy})
	-- ({9.0676*\dx},{0.0011*\dy})
	-- ({9.0776*\dx},{0.0015*\dy})
	-- ({9.0876*\dx},{0.0019*\dy})
	-- ({9.0976*\dx},{0.0023*\dy})
	-- ({9.1076*\dx},{0.0028*\dy})
	-- ({9.1176*\dx},{0.0034*\dy})
	-- ({9.1276*\dx},{0.0039*\dy})
	-- ({9.1376*\dx},{0.0046*\dy})
	-- ({9.1476*\dx},{0.0052*\dy})
	-- ({9.1576*\dx},{0.0060*\dy})
	-- ({9.1676*\dx},{0.0067*\dy})
	-- ({9.1776*\dx},{0.0075*\dy})
	-- ({9.1877*\dx},{0.0084*\dy})
	-- ({9.1977*\dx},{0.0093*\dy})
	-- ({9.2077*\dx},{0.0103*\dy})
	-- ({9.2177*\dx},{0.0112*\dy})
	-- ({9.2277*\dx},{0.0123*\dy})
	-- ({9.2377*\dx},{0.0134*\dy})
	-- ({9.2477*\dx},{0.0145*\dy})
	-- ({9.2577*\dx},{0.0156*\dy})
	-- ({9.2677*\dx},{0.0168*\dy})
	-- ({9.2777*\dx},{0.0181*\dy})
	-- ({9.2877*\dx},{0.0194*\dy})
	-- ({9.2977*\dx},{0.0207*\dy})
	-- ({9.3078*\dx},{0.0221*\dy})
	-- ({9.3178*\dx},{0.0236*\dy})
	-- ({9.3278*\dx},{0.0250*\dy})
	-- ({9.3378*\dx},{0.0266*\dy})
	-- ({9.3478*\dx},{0.0281*\dy})
	-- ({9.3578*\dx},{0.0297*\dy})
	-- ({9.3678*\dx},{0.0314*\dy})
	-- ({9.3778*\dx},{0.0331*\dy})
	-- ({9.3878*\dx},{0.0348*\dy})
	-- ({9.3978*\dx},{0.0366*\dy})
	-- ({9.4078*\dx},{0.0385*\dy})
	-- ({9.4178*\dx},{0.0403*\dy})
	-- ({9.4279*\dx},{0.0423*\dy})
	-- ({9.4379*\dx},{0.0443*\dy})
	-- ({9.4479*\dx},{0.0463*\dy})
	-- ({9.4579*\dx},{0.0484*\dy})
	-- ({9.4679*\dx},{0.0505*\dy})
	-- ({9.4779*\dx},{0.0527*\dy})
	-- ({9.4879*\dx},{0.0549*\dy})
	-- ({9.4979*\dx},{0.0571*\dy})
	-- ({9.5079*\dx},{0.0595*\dy})
	-- ({9.5179*\dx},{0.0618*\dy})
	-- ({9.5279*\dx},{0.0643*\dy})
	-- ({9.5379*\dx},{0.0667*\dy})
	-- ({9.5480*\dx},{0.0693*\dy})
	-- ({9.5580*\dx},{0.0719*\dy})
	-- ({9.5680*\dx},{0.0745*\dy})
	-- ({9.5780*\dx},{0.0772*\dy})
	-- ({9.5880*\dx},{0.0799*\dy})
	-- ({9.5980*\dx},{0.0827*\dy})
	-- ({9.6080*\dx},{0.0856*\dy})
	-- ({9.6180*\dx},{0.0885*\dy})
	-- ({9.6280*\dx},{0.0915*\dy})
	-- ({9.6380*\dx},{0.0945*\dy})
	-- ({9.6480*\dx},{0.0976*\dy})
	-- ({9.6580*\dx},{0.1008*\dy})
	-- ({9.6681*\dx},{0.1040*\dy})
	-- ({9.6781*\dx},{0.1072*\dy})
	-- ({9.6881*\dx},{0.1106*\dy})
	-- ({9.6981*\dx},{0.1140*\dy})
	-- ({9.7081*\dx},{0.1174*\dy})
	-- ({9.7181*\dx},{0.1210*\dy})
	-- ({9.7281*\dx},{0.1246*\dy})
	-- ({9.7381*\dx},{0.1282*\dy})
	-- ({9.7481*\dx},{0.1320*\dy})
	-- ({9.7581*\dx},{0.1357*\dy})
	-- ({9.7681*\dx},{0.1396*\dy})
	-- ({9.7781*\dx},{0.1435*\dy})
	-- ({9.7882*\dx},{0.1476*\dy})
	-- ({9.7982*\dx},{0.1516*\dy})
	-- ({9.8082*\dx},{0.1558*\dy})
	-- ({9.8182*\dx},{0.1600*\dy})
	-- ({9.8282*\dx},{0.1643*\dy})
	-- ({9.8382*\dx},{0.1687*\dy})
	-- ({9.8482*\dx},{0.1731*\dy})
	-- ({9.8582*\dx},{0.1776*\dy})
	-- ({9.8682*\dx},{0.1822*\dy})
	-- ({9.8782*\dx},{0.1869*\dy})
	-- ({9.8882*\dx},{0.1916*\dy})
	-- ({9.8982*\dx},{0.1965*\dy})
	-- ({9.9083*\dx},{0.2014*\dy})
	-- ({9.9183*\dx},{0.2063*\dy})
	-- ({9.9283*\dx},{0.2114*\dy})
	-- ({9.9383*\dx},{0.2165*\dy})
	-- ({9.9483*\dx},{0.2218*\dy})
	-- ({9.9583*\dx},{0.2271*\dy})
	-- ({9.9683*\dx},{0.2324*\dy})
	-- ({9.9783*\dx},{0.2379*\dy})
	-- ({9.9883*\dx},{0.2434*\dy})
	-- ({9.9983*\dx},{0.2491*\dy})
	-- ({10.0083*\dx},{0.2548*\dy})
	-- ({10.0183*\dx},{0.2606*\dy})
	-- ({10.0284*\dx},{0.2665*\dy})
	-- ({10.0384*\dx},{0.2725*\dy})
	-- ({10.0484*\dx},{0.2786*\dy})
	-- ({10.0584*\dx},{0.2848*\dy})
	-- ({10.0684*\dx},{0.2912*\dy})
	-- ({10.0784*\dx},{0.2976*\dy})
	-- ({10.0884*\dx},{0.3041*\dy})
	-- ({10.0984*\dx},{0.3108*\dy})
	-- ({10.1084*\dx},{0.3176*\dy})
	-- ({10.1184*\dx},{0.3245*\dy})
	-- ({10.1284*\dx},{0.3315*\dy})
	-- ({10.1384*\dx},{0.3386*\dy})
	-- ({10.1485*\dx},{0.3458*\dy})
	-- ({10.1585*\dx},{0.3531*\dy})
	-- ({10.1685*\dx},{0.3606*\dy})
	-- ({10.1785*\dx},{0.3682*\dy})
	-- ({10.1885*\dx},{0.3759*\dy})
	-- ({10.1985*\dx},{0.3837*\dy})
	-- ({10.2085*\dx},{0.3916*\dy})
	-- ({10.2185*\dx},{0.3996*\dy})
	-- ({10.2285*\dx},{0.4078*\dy})
	-- ({10.2385*\dx},{0.4161*\dy})
	-- ({10.2485*\dx},{0.4244*\dy})
	-- ({10.2585*\dx},{0.4329*\dy})
	-- ({10.2686*\dx},{0.4416*\dy})
	-- ({10.2786*\dx},{0.4503*\dy})
	-- ({10.2886*\dx},{0.4591*\dy})
	-- ({10.2986*\dx},{0.4681*\dy})
	-- ({10.3086*\dx},{0.4772*\dy})
	-- ({10.3186*\dx},{0.4863*\dy})
	-- ({10.3286*\dx},{0.4956*\dy})
	-- ({10.3386*\dx},{0.5050*\dy})
	-- ({10.3486*\dx},{0.5144*\dy})
	-- ({10.3586*\dx},{0.5240*\dy})
	-- ({10.3686*\dx},{0.5336*\dy})
	-- ({10.3786*\dx},{0.5434*\dy})
	-- ({10.3887*\dx},{0.5532*\dy})
	-- ({10.3987*\dx},{0.5631*\dy})
	-- ({10.4087*\dx},{0.5731*\dy})
	-- ({10.4187*\dx},{0.5832*\dy})
	-- ({10.4287*\dx},{0.5933*\dy})
	-- ({10.4387*\dx},{0.6034*\dy})
	-- ({10.4487*\dx},{0.6137*\dy})
	-- ({10.4587*\dx},{0.6239*\dy})
	-- ({10.4687*\dx},{0.6343*\dy})
	-- ({10.4787*\dx},{0.6446*\dy})
	-- ({10.4887*\dx},{0.6550*\dy})
	-- ({10.4987*\dx},{0.6654*\dy})
	-- ({10.5088*\dx},{0.6758*\dy})
	-- ({10.5188*\dx},{0.6862*\dy})
	-- ({10.5288*\dx},{0.6966*\dy})
	-- ({10.5388*\dx},{0.7069*\dy})
	-- ({10.5488*\dx},{0.7173*\dy})
	-- ({10.5588*\dx},{0.7276*\dy})
	-- ({10.5688*\dx},{0.7379*\dy})
	-- ({10.5788*\dx},{0.7481*\dy})
	-- ({10.5888*\dx},{0.7582*\dy})
	-- ({10.5988*\dx},{0.7683*\dy})
	-- ({10.6088*\dx},{0.7783*\dy})
	-- ({10.6188*\dx},{0.7882*\dy})
	-- ({10.6289*\dx},{0.7980*\dy})
	-- ({10.6389*\dx},{0.8077*\dy})
	-- ({10.6489*\dx},{0.8172*\dy})
	-- ({10.6589*\dx},{0.8266*\dy})
	-- ({10.6689*\dx},{0.8358*\dy})
	-- ({10.6789*\dx},{0.8449*\dy})
	-- ({10.6889*\dx},{0.8538*\dy})
	-- ({10.6989*\dx},{0.8625*\dy})
	-- ({10.7089*\dx},{0.8711*\dy})
	-- ({10.7189*\dx},{0.8794*\dy})
	-- ({10.7289*\dx},{0.8875*\dy})
	-- ({10.7389*\dx},{0.8954*\dy})
	-- ({10.7490*\dx},{0.9030*\dy})
	-- ({10.7590*\dx},{0.9104*\dy})
	-- ({10.7690*\dx},{0.9176*\dy})
	-- ({10.7790*\dx},{0.9245*\dy})
	-- ({10.7890*\dx},{0.9311*\dy})
	-- ({10.7990*\dx},{0.9374*\dy})
	-- ({10.8090*\dx},{0.9435*\dy})
	-- ({10.8190*\dx},{0.9493*\dy})
	-- ({10.8290*\dx},{0.9547*\dy})
	-- ({10.8390*\dx},{0.9599*\dy})
	-- ({10.8490*\dx},{0.9648*\dy})
	-- ({10.8590*\dx},{0.9693*\dy})
	-- ({10.8691*\dx},{0.9736*\dy})
	-- ({10.8791*\dx},{0.9775*\dy})
	-- ({10.8891*\dx},{0.9811*\dy})
	-- ({10.8991*\dx},{0.9844*\dy})
	-- ({10.9091*\dx},{0.9874*\dy})
	-- ({10.9191*\dx},{0.9901*\dy})
	-- ({10.9291*\dx},{0.9924*\dy})
	-- ({10.9391*\dx},{0.9944*\dy})
	-- ({10.9491*\dx},{0.9961*\dy})
	-- ({10.9591*\dx},{0.9975*\dy})
	-- ({10.9691*\dx},{0.9986*\dy})
	-- ({10.9791*\dx},{0.9994*\dy})
	-- ({10.9892*\dx},{0.9998*\dy})
	-- ({10.9992*\dx},{1.0000*\dy})
	-- ({11.0092*\dx},{1.0000*\dy})
	-- ({11.0192*\dx},{1.0000*\dy})
	-- ({11.0292*\dx},{1.0000*\dy})
	-- ({11.0392*\dx},{1.0000*\dy})
	-- ({11.0492*\dx},{1.0000*\dy})
	-- ({11.0592*\dx},{1.0000*\dy})
	-- ({11.0692*\dx},{1.0000*\dy})
	-- ({11.0792*\dx},{1.0000*\dy})
	-- ({11.0892*\dx},{1.0000*\dy})
	-- ({11.0992*\dx},{1.0000*\dy})
	-- ({11.1093*\dx},{1.0000*\dy})
	-- ({11.1193*\dx},{1.0000*\dy})
	-- ({11.1293*\dx},{1.0000*\dy})
	-- ({11.1393*\dx},{1.0000*\dy})
	-- ({11.1493*\dx},{1.0000*\dy})
	-- ({11.1593*\dx},{1.0000*\dy})
	-- ({11.1693*\dx},{1.0000*\dy})
	-- ({11.1793*\dx},{1.0000*\dy})
	-- ({11.1893*\dx},{1.0000*\dy})
	-- ({11.1993*\dx},{1.0000*\dy})
	-- ({11.2093*\dx},{1.0000*\dy})
	-- ({11.2193*\dx},{1.0000*\dy})
	-- ({11.2294*\dx},{1.0000*\dy})
	-- ({11.2394*\dx},{1.0000*\dy})
	-- ({11.2494*\dx},{1.0000*\dy})
	-- ({11.2594*\dx},{1.0000*\dy})
	-- ({11.2694*\dx},{1.0000*\dy})
	-- ({11.2794*\dx},{1.0000*\dy})
	-- ({11.2894*\dx},{1.0000*\dy})
	-- ({11.2994*\dx},{1.0000*\dy})
	-- ({11.3094*\dx},{1.0000*\dy})
	-- ({11.3194*\dx},{1.0000*\dy})
	-- ({11.3294*\dx},{1.0000*\dy})
	-- ({11.3394*\dx},{1.0000*\dy})
	-- ({11.3495*\dx},{1.0000*\dy})
	-- ({11.3595*\dx},{1.0000*\dy})
	-- ({11.3695*\dx},{1.0000*\dy})
	-- ({11.3795*\dx},{1.0000*\dy})
	-- ({11.3895*\dx},{1.0000*\dy})
	-- ({11.3995*\dx},{1.0000*\dy})
	-- ({11.4095*\dx},{1.0000*\dy})
	-- ({11.4195*\dx},{1.0000*\dy})
	-- ({11.4295*\dx},{1.0000*\dy})
	-- ({11.4395*\dx},{1.0000*\dy})
	-- ({11.4495*\dx},{1.0000*\dy})
	-- ({11.4595*\dx},{1.0000*\dy})
	-- ({11.4696*\dx},{1.0000*\dy})
	-- ({11.4796*\dx},{1.0000*\dy})
	-- ({11.4896*\dx},{1.0000*\dy})
	-- ({11.4996*\dx},{1.0000*\dy})
	-- ({11.5096*\dx},{1.0000*\dy})
	-- ({11.5196*\dx},{1.0000*\dy})
	-- ({11.5296*\dx},{1.0000*\dy})
	-- ({11.5396*\dx},{1.0000*\dy})
	-- ({11.5496*\dx},{1.0000*\dy})
	-- ({11.5596*\dx},{1.0000*\dy})
	-- ({11.5696*\dx},{1.0000*\dy})
	-- ({11.5796*\dx},{1.0000*\dy})
	-- ({11.5897*\dx},{1.0000*\dy})
	-- ({11.5997*\dx},{1.0000*\dy})
	-- ({11.6097*\dx},{1.0000*\dy})
	-- ({11.6197*\dx},{1.0000*\dy})
	-- ({11.6297*\dx},{1.0000*\dy})
	-- ({11.6397*\dx},{1.0000*\dy})
	-- ({11.6497*\dx},{1.0000*\dy})
	-- ({11.6597*\dx},{1.0000*\dy})
	-- ({11.6697*\dx},{1.0000*\dy})
	-- ({11.6797*\dx},{1.0000*\dy})
	-- ({11.6897*\dx},{1.0000*\dy})
	-- ({11.6997*\dx},{1.0000*\dy})
	-- ({11.7098*\dx},{1.0000*\dy})
	-- ({11.7198*\dx},{1.0000*\dy})
	-- ({11.7298*\dx},{1.0000*\dy})
	-- ({11.7398*\dx},{1.0000*\dy})
	-- ({11.7498*\dx},{1.0000*\dy})
	-- ({11.7598*\dx},{1.0000*\dy})
	-- ({11.7698*\dx},{1.0000*\dy})
	-- ({11.7798*\dx},{1.0000*\dy})
	-- ({11.7898*\dx},{1.0000*\dy})
	-- ({11.7998*\dx},{1.0000*\dy})
	-- ({11.8098*\dx},{1.0000*\dy})
	-- ({11.8198*\dx},{1.0000*\dy})
	-- ({11.8299*\dx},{1.0000*\dy})
	-- ({11.8399*\dx},{1.0000*\dy})
	-- ({11.8499*\dx},{1.0000*\dy})
	-- ({11.8599*\dx},{1.0000*\dy})
	-- ({11.8699*\dx},{1.0000*\dy})
	-- ({11.8799*\dx},{1.0000*\dy})
	-- ({11.8899*\dx},{1.0000*\dy})
	-- ({11.8999*\dx},{1.0000*\dy})
	-- ({11.9099*\dx},{1.0000*\dy})
	-- ({11.9199*\dx},{1.0000*\dy})
	-- ({11.9299*\dx},{1.0000*\dy})
	-- ({11.9399*\dx},{1.0000*\dy})
	-- ({11.9500*\dx},{1.0000*\dy})
	-- ({11.9600*\dx},{1.0000*\dy})
	-- ({11.9700*\dx},{1.0000*\dy})
	-- ({11.9800*\dx},{1.0000*\dy})
	-- ({11.9900*\dx},{1.0000*\dy})
	-- ({12.0000*\dx},{1.0000*\dy})
}
\def\psithree{
	({0.0000*\dx},{0.0000*\dy})
	-- ({0.0100*\dx},{0.0000*\dy})
	-- ({0.0200*\dx},{0.0000*\dy})
	-- ({0.0300*\dx},{0.0000*\dy})
	-- ({0.0400*\dx},{0.0000*\dy})
	-- ({0.0500*\dx},{0.0000*\dy})
	-- ({0.0601*\dx},{0.0000*\dy})
	-- ({0.0701*\dx},{0.0000*\dy})
	-- ({0.0801*\dx},{0.0000*\dy})
	-- ({0.0901*\dx},{0.0000*\dy})
	-- ({0.1001*\dx},{0.0000*\dy})
	-- ({0.1101*\dx},{0.0000*\dy})
	-- ({0.1201*\dx},{0.0000*\dy})
	-- ({0.1301*\dx},{0.0000*\dy})
	-- ({0.1401*\dx},{0.0000*\dy})
	-- ({0.1501*\dx},{0.0000*\dy})
	-- ({0.1601*\dx},{0.0000*\dy})
	-- ({0.1701*\dx},{0.0000*\dy})
	-- ({0.1802*\dx},{0.0000*\dy})
	-- ({0.1902*\dx},{0.0000*\dy})
	-- ({0.2002*\dx},{0.0000*\dy})
	-- ({0.2102*\dx},{0.0000*\dy})
	-- ({0.2202*\dx},{0.0000*\dy})
	-- ({0.2302*\dx},{0.0000*\dy})
	-- ({0.2402*\dx},{0.0000*\dy})
	-- ({0.2502*\dx},{0.0000*\dy})
	-- ({0.2602*\dx},{0.0000*\dy})
	-- ({0.2702*\dx},{0.0000*\dy})
	-- ({0.2802*\dx},{0.0000*\dy})
	-- ({0.2902*\dx},{0.0000*\dy})
	-- ({0.3003*\dx},{0.0000*\dy})
	-- ({0.3103*\dx},{0.0000*\dy})
	-- ({0.3203*\dx},{0.0000*\dy})
	-- ({0.3303*\dx},{0.0000*\dy})
	-- ({0.3403*\dx},{0.0000*\dy})
	-- ({0.3503*\dx},{0.0000*\dy})
	-- ({0.3603*\dx},{0.0000*\dy})
	-- ({0.3703*\dx},{0.0000*\dy})
	-- ({0.3803*\dx},{0.0000*\dy})
	-- ({0.3903*\dx},{0.0000*\dy})
	-- ({0.4003*\dx},{0.0000*\dy})
	-- ({0.4103*\dx},{0.0000*\dy})
	-- ({0.4204*\dx},{0.0000*\dy})
	-- ({0.4304*\dx},{0.0000*\dy})
	-- ({0.4404*\dx},{0.0000*\dy})
	-- ({0.4504*\dx},{0.0000*\dy})
	-- ({0.4604*\dx},{0.0000*\dy})
	-- ({0.4704*\dx},{0.0000*\dy})
	-- ({0.4804*\dx},{0.0000*\dy})
	-- ({0.4904*\dx},{0.0000*\dy})
	-- ({0.5004*\dx},{0.0000*\dy})
	-- ({0.5104*\dx},{0.0000*\dy})
	-- ({0.5204*\dx},{0.0000*\dy})
	-- ({0.5304*\dx},{0.0000*\dy})
	-- ({0.5405*\dx},{0.0000*\dy})
	-- ({0.5505*\dx},{0.0000*\dy})
	-- ({0.5605*\dx},{0.0000*\dy})
	-- ({0.5705*\dx},{0.0000*\dy})
	-- ({0.5805*\dx},{0.0000*\dy})
	-- ({0.5905*\dx},{0.0000*\dy})
	-- ({0.6005*\dx},{0.0000*\dy})
	-- ({0.6105*\dx},{0.0000*\dy})
	-- ({0.6205*\dx},{0.0000*\dy})
	-- ({0.6305*\dx},{0.0000*\dy})
	-- ({0.6405*\dx},{0.0000*\dy})
	-- ({0.6505*\dx},{0.0000*\dy})
	-- ({0.6606*\dx},{0.0000*\dy})
	-- ({0.6706*\dx},{0.0000*\dy})
	-- ({0.6806*\dx},{0.0000*\dy})
	-- ({0.6906*\dx},{0.0000*\dy})
	-- ({0.7006*\dx},{0.0000*\dy})
	-- ({0.7106*\dx},{0.0000*\dy})
	-- ({0.7206*\dx},{0.0000*\dy})
	-- ({0.7306*\dx},{0.0000*\dy})
	-- ({0.7406*\dx},{0.0000*\dy})
	-- ({0.7506*\dx},{0.0000*\dy})
	-- ({0.7606*\dx},{0.0000*\dy})
	-- ({0.7706*\dx},{0.0000*\dy})
	-- ({0.7807*\dx},{0.0000*\dy})
	-- ({0.7907*\dx},{0.0000*\dy})
	-- ({0.8007*\dx},{0.0000*\dy})
	-- ({0.8107*\dx},{0.0000*\dy})
	-- ({0.8207*\dx},{0.0000*\dy})
	-- ({0.8307*\dx},{0.0000*\dy})
	-- ({0.8407*\dx},{0.0000*\dy})
	-- ({0.8507*\dx},{0.0000*\dy})
	-- ({0.8607*\dx},{0.0000*\dy})
	-- ({0.8707*\dx},{0.0000*\dy})
	-- ({0.8807*\dx},{0.0000*\dy})
	-- ({0.8907*\dx},{0.0000*\dy})
	-- ({0.9008*\dx},{0.0000*\dy})
	-- ({0.9108*\dx},{0.0000*\dy})
	-- ({0.9208*\dx},{0.0000*\dy})
	-- ({0.9308*\dx},{0.0000*\dy})
	-- ({0.9408*\dx},{0.0000*\dy})
	-- ({0.9508*\dx},{0.0000*\dy})
	-- ({0.9608*\dx},{0.0000*\dy})
	-- ({0.9708*\dx},{0.0000*\dy})
	-- ({0.9808*\dx},{0.0000*\dy})
	-- ({0.9908*\dx},{0.0000*\dy})
	-- ({1.0008*\dx},{0.0000*\dy})
	-- ({1.0108*\dx},{0.0000*\dy})
	-- ({1.0209*\dx},{0.0000*\dy})
	-- ({1.0309*\dx},{0.0000*\dy})
	-- ({1.0409*\dx},{0.0000*\dy})
	-- ({1.0509*\dx},{0.0000*\dy})
	-- ({1.0609*\dx},{0.0000*\dy})
	-- ({1.0709*\dx},{0.0000*\dy})
	-- ({1.0809*\dx},{0.0000*\dy})
	-- ({1.0909*\dx},{0.0000*\dy})
	-- ({1.1009*\dx},{0.0000*\dy})
	-- ({1.1109*\dx},{0.0000*\dy})
	-- ({1.1209*\dx},{0.0000*\dy})
	-- ({1.1309*\dx},{0.0000*\dy})
	-- ({1.1410*\dx},{0.0000*\dy})
	-- ({1.1510*\dx},{0.0000*\dy})
	-- ({1.1610*\dx},{0.0000*\dy})
	-- ({1.1710*\dx},{0.0000*\dy})
	-- ({1.1810*\dx},{0.0000*\dy})
	-- ({1.1910*\dx},{0.0000*\dy})
	-- ({1.2010*\dx},{0.0000*\dy})
	-- ({1.2110*\dx},{0.0000*\dy})
	-- ({1.2210*\dx},{0.0000*\dy})
	-- ({1.2310*\dx},{0.0000*\dy})
	-- ({1.2410*\dx},{0.0000*\dy})
	-- ({1.2510*\dx},{0.0000*\dy})
	-- ({1.2611*\dx},{0.0000*\dy})
	-- ({1.2711*\dx},{0.0000*\dy})
	-- ({1.2811*\dx},{0.0000*\dy})
	-- ({1.2911*\dx},{0.0000*\dy})
	-- ({1.3011*\dx},{0.0000*\dy})
	-- ({1.3111*\dx},{0.0000*\dy})
	-- ({1.3211*\dx},{0.0000*\dy})
	-- ({1.3311*\dx},{0.0000*\dy})
	-- ({1.3411*\dx},{0.0000*\dy})
	-- ({1.3511*\dx},{0.0000*\dy})
	-- ({1.3611*\dx},{0.0000*\dy})
	-- ({1.3711*\dx},{0.0000*\dy})
	-- ({1.3812*\dx},{0.0000*\dy})
	-- ({1.3912*\dx},{0.0000*\dy})
	-- ({1.4012*\dx},{0.0000*\dy})
	-- ({1.4112*\dx},{0.0000*\dy})
	-- ({1.4212*\dx},{0.0000*\dy})
	-- ({1.4312*\dx},{0.0000*\dy})
	-- ({1.4412*\dx},{0.0000*\dy})
	-- ({1.4512*\dx},{0.0000*\dy})
	-- ({1.4612*\dx},{0.0000*\dy})
	-- ({1.4712*\dx},{0.0000*\dy})
	-- ({1.4812*\dx},{0.0000*\dy})
	-- ({1.4912*\dx},{0.0000*\dy})
	-- ({1.5013*\dx},{0.0000*\dy})
	-- ({1.5113*\dx},{0.0000*\dy})
	-- ({1.5213*\dx},{0.0000*\dy})
	-- ({1.5313*\dx},{0.0000*\dy})
	-- ({1.5413*\dx},{0.0000*\dy})
	-- ({1.5513*\dx},{0.0000*\dy})
	-- ({1.5613*\dx},{0.0000*\dy})
	-- ({1.5713*\dx},{0.0000*\dy})
	-- ({1.5813*\dx},{0.0000*\dy})
	-- ({1.5913*\dx},{0.0000*\dy})
	-- ({1.6013*\dx},{0.0000*\dy})
	-- ({1.6113*\dx},{0.0000*\dy})
	-- ({1.6214*\dx},{0.0000*\dy})
	-- ({1.6314*\dx},{0.0000*\dy})
	-- ({1.6414*\dx},{0.0000*\dy})
	-- ({1.6514*\dx},{0.0000*\dy})
	-- ({1.6614*\dx},{0.0000*\dy})
	-- ({1.6714*\dx},{0.0000*\dy})
	-- ({1.6814*\dx},{0.0000*\dy})
	-- ({1.6914*\dx},{0.0000*\dy})
	-- ({1.7014*\dx},{0.0000*\dy})
	-- ({1.7114*\dx},{0.0000*\dy})
	-- ({1.7214*\dx},{0.0000*\dy})
	-- ({1.7314*\dx},{0.0000*\dy})
	-- ({1.7415*\dx},{0.0000*\dy})
	-- ({1.7515*\dx},{0.0000*\dy})
	-- ({1.7615*\dx},{0.0000*\dy})
	-- ({1.7715*\dx},{0.0000*\dy})
	-- ({1.7815*\dx},{0.0000*\dy})
	-- ({1.7915*\dx},{0.0000*\dy})
	-- ({1.8015*\dx},{0.0000*\dy})
	-- ({1.8115*\dx},{0.0000*\dy})
	-- ({1.8215*\dx},{0.0000*\dy})
	-- ({1.8315*\dx},{0.0000*\dy})
	-- ({1.8415*\dx},{0.0000*\dy})
	-- ({1.8515*\dx},{0.0000*\dy})
	-- ({1.8616*\dx},{0.0000*\dy})
	-- ({1.8716*\dx},{0.0000*\dy})
	-- ({1.8816*\dx},{0.0000*\dy})
	-- ({1.8916*\dx},{0.0000*\dy})
	-- ({1.9016*\dx},{0.0000*\dy})
	-- ({1.9116*\dx},{0.0000*\dy})
	-- ({1.9216*\dx},{0.0000*\dy})
	-- ({1.9316*\dx},{0.0000*\dy})
	-- ({1.9416*\dx},{0.0000*\dy})
	-- ({1.9516*\dx},{0.0000*\dy})
	-- ({1.9616*\dx},{0.0000*\dy})
	-- ({1.9716*\dx},{0.0000*\dy})
	-- ({1.9817*\dx},{0.0000*\dy})
	-- ({1.9917*\dx},{0.0000*\dy})
	-- ({2.0017*\dx},{0.0000*\dy})
	-- ({2.0117*\dx},{0.0003*\dy})
	-- ({2.0217*\dx},{0.0012*\dy})
	-- ({2.0317*\dx},{0.0026*\dy})
	-- ({2.0417*\dx},{0.0046*\dy})
	-- ({2.0517*\dx},{0.0072*\dy})
	-- ({2.0617*\dx},{0.0104*\dy})
	-- ({2.0717*\dx},{0.0143*\dy})
	-- ({2.0817*\dx},{0.0189*\dy})
	-- ({2.0917*\dx},{0.0241*\dy})
	-- ({2.1018*\dx},{0.0301*\dy})
	-- ({2.1118*\dx},{0.0368*\dy})
	-- ({2.1218*\dx},{0.0442*\dy})
	-- ({2.1318*\dx},{0.0525*\dy})
	-- ({2.1418*\dx},{0.0614*\dy})
	-- ({2.1518*\dx},{0.0712*\dy})
	-- ({2.1618*\dx},{0.0817*\dy})
	-- ({2.1718*\dx},{0.0930*\dy})
	-- ({2.1818*\dx},{0.1050*\dy})
	-- ({2.1918*\dx},{0.1178*\dy})
	-- ({2.2018*\dx},{0.1314*\dy})
	-- ({2.2118*\dx},{0.1457*\dy})
	-- ({2.2219*\dx},{0.1607*\dy})
	-- ({2.2319*\dx},{0.1763*\dy})
	-- ({2.2419*\dx},{0.1926*\dy})
	-- ({2.2519*\dx},{0.2095*\dy})
	-- ({2.2619*\dx},{0.2270*\dy})
	-- ({2.2719*\dx},{0.2450*\dy})
	-- ({2.2819*\dx},{0.2634*\dy})
	-- ({2.2919*\dx},{0.2823*\dy})
	-- ({2.3019*\dx},{0.3016*\dy})
	-- ({2.3119*\dx},{0.3213*\dy})
	-- ({2.3219*\dx},{0.3412*\dy})
	-- ({2.3319*\dx},{0.3613*\dy})
	-- ({2.3420*\dx},{0.3816*\dy})
	-- ({2.3520*\dx},{0.4020*\dy})
	-- ({2.3620*\dx},{0.4225*\dy})
	-- ({2.3720*\dx},{0.4430*\dy})
	-- ({2.3820*\dx},{0.4635*\dy})
	-- ({2.3920*\dx},{0.4838*\dy})
	-- ({2.4020*\dx},{0.5040*\dy})
	-- ({2.4120*\dx},{0.5241*\dy})
	-- ({2.4220*\dx},{0.5438*\dy})
	-- ({2.4320*\dx},{0.5634*\dy})
	-- ({2.4420*\dx},{0.5826*\dy})
	-- ({2.4520*\dx},{0.6014*\dy})
	-- ({2.4621*\dx},{0.6199*\dy})
	-- ({2.4721*\dx},{0.6380*\dy})
	-- ({2.4821*\dx},{0.6557*\dy})
	-- ({2.4921*\dx},{0.6729*\dy})
	-- ({2.5021*\dx},{0.6896*\dy})
	-- ({2.5121*\dx},{0.7058*\dy})
	-- ({2.5221*\dx},{0.7216*\dy})
	-- ({2.5321*\dx},{0.7368*\dy})
	-- ({2.5421*\dx},{0.7516*\dy})
	-- ({2.5521*\dx},{0.7658*\dy})
	-- ({2.5621*\dx},{0.7795*\dy})
	-- ({2.5721*\dx},{0.7927*\dy})
	-- ({2.5822*\dx},{0.8053*\dy})
	-- ({2.5922*\dx},{0.8175*\dy})
	-- ({2.6022*\dx},{0.8291*\dy})
	-- ({2.6122*\dx},{0.8402*\dy})
	-- ({2.6222*\dx},{0.8509*\dy})
	-- ({2.6322*\dx},{0.8610*\dy})
	-- ({2.6422*\dx},{0.8707*\dy})
	-- ({2.6522*\dx},{0.8799*\dy})
	-- ({2.6622*\dx},{0.8886*\dy})
	-- ({2.6722*\dx},{0.8969*\dy})
	-- ({2.6822*\dx},{0.9048*\dy})
	-- ({2.6922*\dx},{0.9122*\dy})
	-- ({2.7023*\dx},{0.9193*\dy})
	-- ({2.7123*\dx},{0.9259*\dy})
	-- ({2.7223*\dx},{0.9322*\dy})
	-- ({2.7323*\dx},{0.9381*\dy})
	-- ({2.7423*\dx},{0.9436*\dy})
	-- ({2.7523*\dx},{0.9489*\dy})
	-- ({2.7623*\dx},{0.9537*\dy})
	-- ({2.7723*\dx},{0.9583*\dy})
	-- ({2.7823*\dx},{0.9625*\dy})
	-- ({2.7923*\dx},{0.9665*\dy})
	-- ({2.8023*\dx},{0.9702*\dy})
	-- ({2.8123*\dx},{0.9736*\dy})
	-- ({2.8224*\dx},{0.9767*\dy})
	-- ({2.8324*\dx},{0.9796*\dy})
	-- ({2.8424*\dx},{0.9823*\dy})
	-- ({2.8524*\dx},{0.9848*\dy})
	-- ({2.8624*\dx},{0.9870*\dy})
	-- ({2.8724*\dx},{0.9890*\dy})
	-- ({2.8824*\dx},{0.9908*\dy})
	-- ({2.8924*\dx},{0.9924*\dy})
	-- ({2.9024*\dx},{0.9939*\dy})
	-- ({2.9124*\dx},{0.9952*\dy})
	-- ({2.9224*\dx},{0.9963*\dy})
	-- ({2.9324*\dx},{0.9972*\dy})
	-- ({2.9425*\dx},{0.9980*\dy})
	-- ({2.9525*\dx},{0.9987*\dy})
	-- ({2.9625*\dx},{0.9992*\dy})
	-- ({2.9725*\dx},{0.9996*\dy})
	-- ({2.9825*\dx},{0.9998*\dy})
	-- ({2.9925*\dx},{1.0000*\dy})
	-- ({3.0025*\dx},{1.0000*\dy})
	-- ({3.0125*\dx},{0.9999*\dy})
	-- ({3.0225*\dx},{0.9997*\dy})
	-- ({3.0325*\dx},{0.9994*\dy})
	-- ({3.0425*\dx},{0.9991*\dy})
	-- ({3.0525*\dx},{0.9986*\dy})
	-- ({3.0626*\dx},{0.9981*\dy})
	-- ({3.0726*\dx},{0.9974*\dy})
	-- ({3.0826*\dx},{0.9967*\dy})
	-- ({3.0926*\dx},{0.9959*\dy})
	-- ({3.1026*\dx},{0.9951*\dy})
	-- ({3.1126*\dx},{0.9941*\dy})
	-- ({3.1226*\dx},{0.9931*\dy})
	-- ({3.1326*\dx},{0.9921*\dy})
	-- ({3.1426*\dx},{0.9910*\dy})
	-- ({3.1526*\dx},{0.9898*\dy})
	-- ({3.1626*\dx},{0.9886*\dy})
	-- ({3.1726*\dx},{0.9873*\dy})
	-- ({3.1827*\dx},{0.9860*\dy})
	-- ({3.1927*\dx},{0.9846*\dy})
	-- ({3.2027*\dx},{0.9832*\dy})
	-- ({3.2127*\dx},{0.9817*\dy})
	-- ({3.2227*\dx},{0.9802*\dy})
	-- ({3.2327*\dx},{0.9787*\dy})
	-- ({3.2427*\dx},{0.9771*\dy})
	-- ({3.2527*\dx},{0.9755*\dy})
	-- ({3.2627*\dx},{0.9739*\dy})
	-- ({3.2727*\dx},{0.9722*\dy})
	-- ({3.2827*\dx},{0.9705*\dy})
	-- ({3.2927*\dx},{0.9687*\dy})
	-- ({3.3028*\dx},{0.9669*\dy})
	-- ({3.3128*\dx},{0.9651*\dy})
	-- ({3.3228*\dx},{0.9633*\dy})
	-- ({3.3328*\dx},{0.9615*\dy})
	-- ({3.3428*\dx},{0.9596*\dy})
	-- ({3.3528*\dx},{0.9577*\dy})
	-- ({3.3628*\dx},{0.9557*\dy})
	-- ({3.3728*\dx},{0.9538*\dy})
	-- ({3.3828*\dx},{0.9518*\dy})
	-- ({3.3928*\dx},{0.9498*\dy})
	-- ({3.4028*\dx},{0.9478*\dy})
	-- ({3.4128*\dx},{0.9457*\dy})
	-- ({3.4229*\dx},{0.9437*\dy})
	-- ({3.4329*\dx},{0.9416*\dy})
	-- ({3.4429*\dx},{0.9395*\dy})
	-- ({3.4529*\dx},{0.9374*\dy})
	-- ({3.4629*\dx},{0.9353*\dy})
	-- ({3.4729*\dx},{0.9331*\dy})
	-- ({3.4829*\dx},{0.9310*\dy})
	-- ({3.4929*\dx},{0.9288*\dy})
	-- ({3.5029*\dx},{0.9266*\dy})
	-- ({3.5129*\dx},{0.9244*\dy})
	-- ({3.5229*\dx},{0.9221*\dy})
	-- ({3.5329*\dx},{0.9199*\dy})
	-- ({3.5430*\dx},{0.9176*\dy})
	-- ({3.5530*\dx},{0.9154*\dy})
	-- ({3.5630*\dx},{0.9131*\dy})
	-- ({3.5730*\dx},{0.9108*\dy})
	-- ({3.5830*\dx},{0.9085*\dy})
	-- ({3.5930*\dx},{0.9061*\dy})
	-- ({3.6030*\dx},{0.9038*\dy})
	-- ({3.6130*\dx},{0.9014*\dy})
	-- ({3.6230*\dx},{0.8991*\dy})
	-- ({3.6330*\dx},{0.8967*\dy})
	-- ({3.6430*\dx},{0.8943*\dy})
	-- ({3.6530*\dx},{0.8919*\dy})
	-- ({3.6631*\dx},{0.8895*\dy})
	-- ({3.6731*\dx},{0.8870*\dy})
	-- ({3.6831*\dx},{0.8846*\dy})
	-- ({3.6931*\dx},{0.8821*\dy})
	-- ({3.7031*\dx},{0.8796*\dy})
	-- ({3.7131*\dx},{0.8772*\dy})
	-- ({3.7231*\dx},{0.8747*\dy})
	-- ({3.7331*\dx},{0.8721*\dy})
	-- ({3.7431*\dx},{0.8696*\dy})
	-- ({3.7531*\dx},{0.8671*\dy})
	-- ({3.7631*\dx},{0.8645*\dy})
	-- ({3.7731*\dx},{0.8620*\dy})
	-- ({3.7832*\dx},{0.8594*\dy})
	-- ({3.7932*\dx},{0.8568*\dy})
	-- ({3.8032*\dx},{0.8542*\dy})
	-- ({3.8132*\dx},{0.8516*\dy})
	-- ({3.8232*\dx},{0.8489*\dy})
	-- ({3.8332*\dx},{0.8463*\dy})
	-- ({3.8432*\dx},{0.8436*\dy})
	-- ({3.8532*\dx},{0.8409*\dy})
	-- ({3.8632*\dx},{0.8382*\dy})
	-- ({3.8732*\dx},{0.8355*\dy})
	-- ({3.8832*\dx},{0.8328*\dy})
	-- ({3.8932*\dx},{0.8301*\dy})
	-- ({3.9033*\dx},{0.8273*\dy})
	-- ({3.9133*\dx},{0.8246*\dy})
	-- ({3.9233*\dx},{0.8218*\dy})
	-- ({3.9333*\dx},{0.8190*\dy})
	-- ({3.9433*\dx},{0.8162*\dy})
	-- ({3.9533*\dx},{0.8134*\dy})
	-- ({3.9633*\dx},{0.8105*\dy})
	-- ({3.9733*\dx},{0.8077*\dy})
	-- ({3.9833*\dx},{0.8048*\dy})
	-- ({3.9933*\dx},{0.8019*\dy})
	-- ({4.0033*\dx},{0.7990*\dy})
	-- ({4.0133*\dx},{0.7961*\dy})
	-- ({4.0234*\dx},{0.7932*\dy})
	-- ({4.0334*\dx},{0.7902*\dy})
	-- ({4.0434*\dx},{0.7873*\dy})
	-- ({4.0534*\dx},{0.7843*\dy})
	-- ({4.0634*\dx},{0.7813*\dy})
	-- ({4.0734*\dx},{0.7783*\dy})
	-- ({4.0834*\dx},{0.7752*\dy})
	-- ({4.0934*\dx},{0.7722*\dy})
	-- ({4.1034*\dx},{0.7691*\dy})
	-- ({4.1134*\dx},{0.7660*\dy})
	-- ({4.1234*\dx},{0.7629*\dy})
	-- ({4.1334*\dx},{0.7598*\dy})
	-- ({4.1435*\dx},{0.7566*\dy})
	-- ({4.1535*\dx},{0.7534*\dy})
	-- ({4.1635*\dx},{0.7503*\dy})
	-- ({4.1735*\dx},{0.7471*\dy})
	-- ({4.1835*\dx},{0.7438*\dy})
	-- ({4.1935*\dx},{0.7406*\dy})
	-- ({4.2035*\dx},{0.7373*\dy})
	-- ({4.2135*\dx},{0.7340*\dy})
	-- ({4.2235*\dx},{0.7307*\dy})
	-- ({4.2335*\dx},{0.7274*\dy})
	-- ({4.2435*\dx},{0.7241*\dy})
	-- ({4.2535*\dx},{0.7207*\dy})
	-- ({4.2636*\dx},{0.7173*\dy})
	-- ({4.2736*\dx},{0.7139*\dy})
	-- ({4.2836*\dx},{0.7105*\dy})
	-- ({4.2936*\dx},{0.7070*\dy})
	-- ({4.3036*\dx},{0.7035*\dy})
	-- ({4.3136*\dx},{0.7000*\dy})
	-- ({4.3236*\dx},{0.6965*\dy})
	-- ({4.3336*\dx},{0.6930*\dy})
	-- ({4.3436*\dx},{0.6894*\dy})
	-- ({4.3536*\dx},{0.6858*\dy})
	-- ({4.3636*\dx},{0.6822*\dy})
	-- ({4.3736*\dx},{0.6786*\dy})
	-- ({4.3837*\dx},{0.6749*\dy})
	-- ({4.3937*\dx},{0.6712*\dy})
	-- ({4.4037*\dx},{0.6675*\dy})
	-- ({4.4137*\dx},{0.6638*\dy})
	-- ({4.4237*\dx},{0.6600*\dy})
	-- ({4.4337*\dx},{0.6563*\dy})
	-- ({4.4437*\dx},{0.6525*\dy})
	-- ({4.4537*\dx},{0.6486*\dy})
	-- ({4.4637*\dx},{0.6448*\dy})
	-- ({4.4737*\dx},{0.6409*\dy})
	-- ({4.4837*\dx},{0.6370*\dy})
	-- ({4.4937*\dx},{0.6331*\dy})
	-- ({4.5038*\dx},{0.6291*\dy})
	-- ({4.5138*\dx},{0.6251*\dy})
	-- ({4.5238*\dx},{0.6211*\dy})
	-- ({4.5338*\dx},{0.6171*\dy})
	-- ({4.5438*\dx},{0.6130*\dy})
	-- ({4.5538*\dx},{0.6090*\dy})
	-- ({4.5638*\dx},{0.6048*\dy})
	-- ({4.5738*\dx},{0.6007*\dy})
	-- ({4.5838*\dx},{0.5965*\dy})
	-- ({4.5938*\dx},{0.5923*\dy})
	-- ({4.6038*\dx},{0.5881*\dy})
	-- ({4.6138*\dx},{0.5839*\dy})
	-- ({4.6239*\dx},{0.5796*\dy})
	-- ({4.6339*\dx},{0.5753*\dy})
	-- ({4.6439*\dx},{0.5710*\dy})
	-- ({4.6539*\dx},{0.5666*\dy})
	-- ({4.6639*\dx},{0.5623*\dy})
	-- ({4.6739*\dx},{0.5578*\dy})
	-- ({4.6839*\dx},{0.5534*\dy})
	-- ({4.6939*\dx},{0.5489*\dy})
	-- ({4.7039*\dx},{0.5445*\dy})
	-- ({4.7139*\dx},{0.5399*\dy})
	-- ({4.7239*\dx},{0.5354*\dy})
	-- ({4.7339*\dx},{0.5308*\dy})
	-- ({4.7440*\dx},{0.5262*\dy})
	-- ({4.7540*\dx},{0.5216*\dy})
	-- ({4.7640*\dx},{0.5170*\dy})
	-- ({4.7740*\dx},{0.5123*\dy})
	-- ({4.7840*\dx},{0.5076*\dy})
	-- ({4.7940*\dx},{0.5029*\dy})
	-- ({4.8040*\dx},{0.4981*\dy})
	-- ({4.8140*\dx},{0.4933*\dy})
	-- ({4.8240*\dx},{0.4885*\dy})
	-- ({4.8340*\dx},{0.4837*\dy})
	-- ({4.8440*\dx},{0.4788*\dy})
	-- ({4.8540*\dx},{0.4739*\dy})
	-- ({4.8641*\dx},{0.4690*\dy})
	-- ({4.8741*\dx},{0.4641*\dy})
	-- ({4.8841*\dx},{0.4591*\dy})
	-- ({4.8941*\dx},{0.4541*\dy})
	-- ({4.9041*\dx},{0.4491*\dy})
	-- ({4.9141*\dx},{0.4441*\dy})
	-- ({4.9241*\dx},{0.4390*\dy})
	-- ({4.9341*\dx},{0.4340*\dy})
	-- ({4.9441*\dx},{0.4289*\dy})
	-- ({4.9541*\dx},{0.4237*\dy})
	-- ({4.9641*\dx},{0.4186*\dy})
	-- ({4.9741*\dx},{0.4134*\dy})
	-- ({4.9842*\dx},{0.4082*\dy})
	-- ({4.9942*\dx},{0.4030*\dy})
	-- ({5.0042*\dx},{0.3978*\dy})
	-- ({5.0142*\dx},{0.3926*\dy})
	-- ({5.0242*\dx},{0.3872*\dy})
	-- ({5.0342*\dx},{0.3819*\dy})
	-- ({5.0442*\dx},{0.3765*\dy})
	-- ({5.0542*\dx},{0.3711*\dy})
	-- ({5.0642*\dx},{0.3656*\dy})
	-- ({5.0742*\dx},{0.3601*\dy})
	-- ({5.0842*\dx},{0.3546*\dy})
	-- ({5.0942*\dx},{0.3490*\dy})
	-- ({5.1043*\dx},{0.3435*\dy})
	-- ({5.1143*\dx},{0.3378*\dy})
	-- ({5.1243*\dx},{0.3322*\dy})
	-- ({5.1343*\dx},{0.3265*\dy})
	-- ({5.1443*\dx},{0.3209*\dy})
	-- ({5.1543*\dx},{0.3152*\dy})
	-- ({5.1643*\dx},{0.3094*\dy})
	-- ({5.1743*\dx},{0.3037*\dy})
	-- ({5.1843*\dx},{0.2980*\dy})
	-- ({5.1943*\dx},{0.2922*\dy})
	-- ({5.2043*\dx},{0.2864*\dy})
	-- ({5.2143*\dx},{0.2807*\dy})
	-- ({5.2244*\dx},{0.2749*\dy})
	-- ({5.2344*\dx},{0.2691*\dy})
	-- ({5.2444*\dx},{0.2633*\dy})
	-- ({5.2544*\dx},{0.2576*\dy})
	-- ({5.2644*\dx},{0.2518*\dy})
	-- ({5.2744*\dx},{0.2461*\dy})
	-- ({5.2844*\dx},{0.2403*\dy})
	-- ({5.2944*\dx},{0.2346*\dy})
	-- ({5.3044*\dx},{0.2289*\dy})
	-- ({5.3144*\dx},{0.2232*\dy})
	-- ({5.3244*\dx},{0.2176*\dy})
	-- ({5.3344*\dx},{0.2119*\dy})
	-- ({5.3445*\dx},{0.2063*\dy})
	-- ({5.3545*\dx},{0.2007*\dy})
	-- ({5.3645*\dx},{0.1952*\dy})
	-- ({5.3745*\dx},{0.1897*\dy})
	-- ({5.3845*\dx},{0.1842*\dy})
	-- ({5.3945*\dx},{0.1788*\dy})
	-- ({5.4045*\dx},{0.1734*\dy})
	-- ({5.4145*\dx},{0.1680*\dy})
	-- ({5.4245*\dx},{0.1627*\dy})
	-- ({5.4345*\dx},{0.1575*\dy})
	-- ({5.4445*\dx},{0.1523*\dy})
	-- ({5.4545*\dx},{0.1472*\dy})
	-- ({5.4646*\dx},{0.1421*\dy})
	-- ({5.4746*\dx},{0.1371*\dy})
	-- ({5.4846*\dx},{0.1321*\dy})
	-- ({5.4946*\dx},{0.1273*\dy})
	-- ({5.5046*\dx},{0.1224*\dy})
	-- ({5.5146*\dx},{0.1177*\dy})
	-- ({5.5246*\dx},{0.1130*\dy})
	-- ({5.5346*\dx},{0.1084*\dy})
	-- ({5.5446*\dx},{0.1039*\dy})
	-- ({5.5546*\dx},{0.0995*\dy})
	-- ({5.5646*\dx},{0.0951*\dy})
	-- ({5.5746*\dx},{0.0908*\dy})
	-- ({5.5847*\dx},{0.0867*\dy})
	-- ({5.5947*\dx},{0.0825*\dy})
	-- ({5.6047*\dx},{0.0785*\dy})
	-- ({5.6147*\dx},{0.0746*\dy})
	-- ({5.6247*\dx},{0.0708*\dy})
	-- ({5.6347*\dx},{0.0670*\dy})
	-- ({5.6447*\dx},{0.0634*\dy})
	-- ({5.6547*\dx},{0.0598*\dy})
	-- ({5.6647*\dx},{0.0564*\dy})
	-- ({5.6747*\dx},{0.0530*\dy})
	-- ({5.6847*\dx},{0.0498*\dy})
	-- ({5.6947*\dx},{0.0466*\dy})
	-- ({5.7048*\dx},{0.0436*\dy})
	-- ({5.7148*\dx},{0.0406*\dy})
	-- ({5.7248*\dx},{0.0377*\dy})
	-- ({5.7348*\dx},{0.0350*\dy})
	-- ({5.7448*\dx},{0.0324*\dy})
	-- ({5.7548*\dx},{0.0298*\dy})
	-- ({5.7648*\dx},{0.0274*\dy})
	-- ({5.7748*\dx},{0.0250*\dy})
	-- ({5.7848*\dx},{0.0228*\dy})
	-- ({5.7948*\dx},{0.0207*\dy})
	-- ({5.8048*\dx},{0.0187*\dy})
	-- ({5.8148*\dx},{0.0168*\dy})
	-- ({5.8249*\dx},{0.0150*\dy})
	-- ({5.8349*\dx},{0.0133*\dy})
	-- ({5.8449*\dx},{0.0117*\dy})
	-- ({5.8549*\dx},{0.0102*\dy})
	-- ({5.8649*\dx},{0.0088*\dy})
	-- ({5.8749*\dx},{0.0075*\dy})
	-- ({5.8849*\dx},{0.0064*\dy})
	-- ({5.8949*\dx},{0.0053*\dy})
	-- ({5.9049*\dx},{0.0043*\dy})
	-- ({5.9149*\dx},{0.0034*\dy})
	-- ({5.9249*\dx},{0.0027*\dy})
	-- ({5.9349*\dx},{0.0020*\dy})
	-- ({5.9450*\dx},{0.0014*\dy})
	-- ({5.9550*\dx},{0.0009*\dy})
	-- ({5.9650*\dx},{0.0006*\dy})
	-- ({5.9750*\dx},{0.0003*\dy})
	-- ({5.9850*\dx},{0.0001*\dy})
	-- ({5.9950*\dx},{0.0000*\dy})
	-- ({6.0050*\dx},{0.0000*\dy})
	-- ({6.0150*\dx},{0.0000*\dy})
	-- ({6.0250*\dx},{0.0000*\dy})
	-- ({6.0350*\dx},{0.0000*\dy})
	-- ({6.0450*\dx},{0.0000*\dy})
	-- ({6.0550*\dx},{0.0000*\dy})
	-- ({6.0651*\dx},{0.0000*\dy})
	-- ({6.0751*\dx},{0.0000*\dy})
	-- ({6.0851*\dx},{0.0000*\dy})
	-- ({6.0951*\dx},{0.0000*\dy})
	-- ({6.1051*\dx},{0.0000*\dy})
	-- ({6.1151*\dx},{0.0000*\dy})
	-- ({6.1251*\dx},{0.0000*\dy})
	-- ({6.1351*\dx},{0.0000*\dy})
	-- ({6.1451*\dx},{0.0000*\dy})
	-- ({6.1551*\dx},{0.0000*\dy})
	-- ({6.1651*\dx},{0.0000*\dy})
	-- ({6.1751*\dx},{0.0000*\dy})
	-- ({6.1852*\dx},{0.0000*\dy})
	-- ({6.1952*\dx},{0.0000*\dy})
	-- ({6.2052*\dx},{0.0000*\dy})
	-- ({6.2152*\dx},{0.0000*\dy})
	-- ({6.2252*\dx},{0.0000*\dy})
	-- ({6.2352*\dx},{0.0000*\dy})
	-- ({6.2452*\dx},{0.0000*\dy})
	-- ({6.2552*\dx},{0.0000*\dy})
	-- ({6.2652*\dx},{0.0000*\dy})
	-- ({6.2752*\dx},{0.0000*\dy})
	-- ({6.2852*\dx},{0.0000*\dy})
	-- ({6.2952*\dx},{0.0000*\dy})
	-- ({6.3053*\dx},{0.0000*\dy})
	-- ({6.3153*\dx},{0.0000*\dy})
	-- ({6.3253*\dx},{0.0000*\dy})
	-- ({6.3353*\dx},{0.0000*\dy})
	-- ({6.3453*\dx},{0.0000*\dy})
	-- ({6.3553*\dx},{0.0000*\dy})
	-- ({6.3653*\dx},{0.0000*\dy})
	-- ({6.3753*\dx},{0.0000*\dy})
	-- ({6.3853*\dx},{0.0000*\dy})
	-- ({6.3953*\dx},{0.0000*\dy})
	-- ({6.4053*\dx},{0.0000*\dy})
	-- ({6.4153*\dx},{0.0000*\dy})
	-- ({6.4254*\dx},{0.0000*\dy})
	-- ({6.4354*\dx},{0.0000*\dy})
	-- ({6.4454*\dx},{0.0000*\dy})
	-- ({6.4554*\dx},{0.0000*\dy})
	-- ({6.4654*\dx},{0.0000*\dy})
	-- ({6.4754*\dx},{0.0000*\dy})
	-- ({6.4854*\dx},{0.0000*\dy})
	-- ({6.4954*\dx},{0.0000*\dy})
	-- ({6.5054*\dx},{0.0000*\dy})
	-- ({6.5154*\dx},{0.0000*\dy})
	-- ({6.5254*\dx},{0.0000*\dy})
	-- ({6.5354*\dx},{0.0000*\dy})
	-- ({6.5455*\dx},{0.0000*\dy})
	-- ({6.5555*\dx},{0.0000*\dy})
	-- ({6.5655*\dx},{0.0000*\dy})
	-- ({6.5755*\dx},{0.0000*\dy})
	-- ({6.5855*\dx},{0.0000*\dy})
	-- ({6.5955*\dx},{0.0000*\dy})
	-- ({6.6055*\dx},{0.0000*\dy})
	-- ({6.6155*\dx},{0.0000*\dy})
	-- ({6.6255*\dx},{0.0000*\dy})
	-- ({6.6355*\dx},{0.0000*\dy})
	-- ({6.6455*\dx},{0.0000*\dy})
	-- ({6.6555*\dx},{0.0000*\dy})
	-- ({6.6656*\dx},{0.0000*\dy})
	-- ({6.6756*\dx},{0.0000*\dy})
	-- ({6.6856*\dx},{0.0000*\dy})
	-- ({6.6956*\dx},{0.0000*\dy})
	-- ({6.7056*\dx},{0.0000*\dy})
	-- ({6.7156*\dx},{0.0000*\dy})
	-- ({6.7256*\dx},{0.0000*\dy})
	-- ({6.7356*\dx},{0.0000*\dy})
	-- ({6.7456*\dx},{0.0000*\dy})
	-- ({6.7556*\dx},{0.0000*\dy})
	-- ({6.7656*\dx},{0.0000*\dy})
	-- ({6.7756*\dx},{0.0000*\dy})
	-- ({6.7857*\dx},{0.0000*\dy})
	-- ({6.7957*\dx},{0.0000*\dy})
	-- ({6.8057*\dx},{0.0000*\dy})
	-- ({6.8157*\dx},{0.0000*\dy})
	-- ({6.8257*\dx},{0.0000*\dy})
	-- ({6.8357*\dx},{0.0000*\dy})
	-- ({6.8457*\dx},{0.0000*\dy})
	-- ({6.8557*\dx},{0.0000*\dy})
	-- ({6.8657*\dx},{0.0000*\dy})
	-- ({6.8757*\dx},{0.0000*\dy})
	-- ({6.8857*\dx},{0.0000*\dy})
	-- ({6.8957*\dx},{0.0000*\dy})
	-- ({6.9058*\dx},{0.0000*\dy})
	-- ({6.9158*\dx},{0.0000*\dy})
	-- ({6.9258*\dx},{0.0000*\dy})
	-- ({6.9358*\dx},{0.0000*\dy})
	-- ({6.9458*\dx},{0.0000*\dy})
	-- ({6.9558*\dx},{0.0000*\dy})
	-- ({6.9658*\dx},{0.0000*\dy})
	-- ({6.9758*\dx},{0.0000*\dy})
	-- ({6.9858*\dx},{0.0000*\dy})
	-- ({6.9958*\dx},{0.0000*\dy})
	-- ({7.0058*\dx},{0.0000*\dy})
	-- ({7.0158*\dx},{0.0000*\dy})
	-- ({7.0259*\dx},{0.0000*\dy})
	-- ({7.0359*\dx},{0.0000*\dy})
	-- ({7.0459*\dx},{0.0000*\dy})
	-- ({7.0559*\dx},{0.0000*\dy})
	-- ({7.0659*\dx},{0.0000*\dy})
	-- ({7.0759*\dx},{0.0000*\dy})
	-- ({7.0859*\dx},{0.0000*\dy})
	-- ({7.0959*\dx},{0.0000*\dy})
	-- ({7.1059*\dx},{0.0000*\dy})
	-- ({7.1159*\dx},{0.0000*\dy})
	-- ({7.1259*\dx},{0.0000*\dy})
	-- ({7.1359*\dx},{0.0000*\dy})
	-- ({7.1460*\dx},{0.0000*\dy})
	-- ({7.1560*\dx},{0.0000*\dy})
	-- ({7.1660*\dx},{0.0000*\dy})
	-- ({7.1760*\dx},{0.0000*\dy})
	-- ({7.1860*\dx},{0.0000*\dy})
	-- ({7.1960*\dx},{0.0000*\dy})
	-- ({7.2060*\dx},{0.0000*\dy})
	-- ({7.2160*\dx},{0.0000*\dy})
	-- ({7.2260*\dx},{0.0000*\dy})
	-- ({7.2360*\dx},{0.0000*\dy})
	-- ({7.2460*\dx},{0.0000*\dy})
	-- ({7.2560*\dx},{0.0000*\dy})
	-- ({7.2661*\dx},{0.0000*\dy})
	-- ({7.2761*\dx},{0.0000*\dy})
	-- ({7.2861*\dx},{0.0000*\dy})
	-- ({7.2961*\dx},{0.0000*\dy})
	-- ({7.3061*\dx},{0.0000*\dy})
	-- ({7.3161*\dx},{0.0000*\dy})
	-- ({7.3261*\dx},{0.0000*\dy})
	-- ({7.3361*\dx},{0.0000*\dy})
	-- ({7.3461*\dx},{0.0000*\dy})
	-- ({7.3561*\dx},{0.0000*\dy})
	-- ({7.3661*\dx},{0.0000*\dy})
	-- ({7.3761*\dx},{0.0000*\dy})
	-- ({7.3862*\dx},{0.0000*\dy})
	-- ({7.3962*\dx},{0.0000*\dy})
	-- ({7.4062*\dx},{0.0000*\dy})
	-- ({7.4162*\dx},{0.0000*\dy})
	-- ({7.4262*\dx},{0.0000*\dy})
	-- ({7.4362*\dx},{0.0000*\dy})
	-- ({7.4462*\dx},{0.0000*\dy})
	-- ({7.4562*\dx},{0.0000*\dy})
	-- ({7.4662*\dx},{0.0000*\dy})
	-- ({7.4762*\dx},{0.0000*\dy})
	-- ({7.4862*\dx},{0.0000*\dy})
	-- ({7.4962*\dx},{0.0000*\dy})
	-- ({7.5063*\dx},{0.0000*\dy})
	-- ({7.5163*\dx},{0.0000*\dy})
	-- ({7.5263*\dx},{0.0000*\dy})
	-- ({7.5363*\dx},{0.0000*\dy})
	-- ({7.5463*\dx},{0.0000*\dy})
	-- ({7.5563*\dx},{0.0000*\dy})
	-- ({7.5663*\dx},{0.0000*\dy})
	-- ({7.5763*\dx},{0.0000*\dy})
	-- ({7.5863*\dx},{0.0000*\dy})
	-- ({7.5963*\dx},{0.0000*\dy})
	-- ({7.6063*\dx},{0.0000*\dy})
	-- ({7.6163*\dx},{0.0000*\dy})
	-- ({7.6264*\dx},{0.0000*\dy})
	-- ({7.6364*\dx},{0.0000*\dy})
	-- ({7.6464*\dx},{0.0000*\dy})
	-- ({7.6564*\dx},{0.0000*\dy})
	-- ({7.6664*\dx},{0.0000*\dy})
	-- ({7.6764*\dx},{0.0000*\dy})
	-- ({7.6864*\dx},{0.0000*\dy})
	-- ({7.6964*\dx},{0.0000*\dy})
	-- ({7.7064*\dx},{0.0000*\dy})
	-- ({7.7164*\dx},{0.0000*\dy})
	-- ({7.7264*\dx},{0.0000*\dy})
	-- ({7.7364*\dx},{0.0000*\dy})
	-- ({7.7465*\dx},{0.0000*\dy})
	-- ({7.7565*\dx},{0.0000*\dy})
	-- ({7.7665*\dx},{0.0000*\dy})
	-- ({7.7765*\dx},{0.0000*\dy})
	-- ({7.7865*\dx},{0.0000*\dy})
	-- ({7.7965*\dx},{0.0000*\dy})
	-- ({7.8065*\dx},{0.0000*\dy})
	-- ({7.8165*\dx},{0.0000*\dy})
	-- ({7.8265*\dx},{0.0000*\dy})
	-- ({7.8365*\dx},{0.0000*\dy})
	-- ({7.8465*\dx},{0.0000*\dy})
	-- ({7.8565*\dx},{0.0000*\dy})
	-- ({7.8666*\dx},{0.0000*\dy})
	-- ({7.8766*\dx},{0.0000*\dy})
	-- ({7.8866*\dx},{0.0000*\dy})
	-- ({7.8966*\dx},{0.0000*\dy})
	-- ({7.9066*\dx},{0.0000*\dy})
	-- ({7.9166*\dx},{0.0000*\dy})
	-- ({7.9266*\dx},{0.0000*\dy})
	-- ({7.9366*\dx},{0.0000*\dy})
	-- ({7.9466*\dx},{0.0000*\dy})
	-- ({7.9566*\dx},{0.0000*\dy})
	-- ({7.9666*\dx},{0.0000*\dy})
	-- ({7.9766*\dx},{0.0000*\dy})
	-- ({7.9867*\dx},{0.0000*\dy})
	-- ({7.9967*\dx},{0.0000*\dy})
	-- ({8.0067*\dx},{0.0000*\dy})
	-- ({8.0167*\dx},{0.0000*\dy})
	-- ({8.0267*\dx},{0.0000*\dy})
	-- ({8.0367*\dx},{0.0000*\dy})
	-- ({8.0467*\dx},{0.0000*\dy})
	-- ({8.0567*\dx},{0.0000*\dy})
	-- ({8.0667*\dx},{0.0000*\dy})
	-- ({8.0767*\dx},{0.0000*\dy})
	-- ({8.0867*\dx},{0.0000*\dy})
	-- ({8.0967*\dx},{0.0000*\dy})
	-- ({8.1068*\dx},{0.0000*\dy})
	-- ({8.1168*\dx},{0.0000*\dy})
	-- ({8.1268*\dx},{0.0000*\dy})
	-- ({8.1368*\dx},{0.0000*\dy})
	-- ({8.1468*\dx},{0.0000*\dy})
	-- ({8.1568*\dx},{0.0000*\dy})
	-- ({8.1668*\dx},{0.0000*\dy})
	-- ({8.1768*\dx},{0.0000*\dy})
	-- ({8.1868*\dx},{0.0000*\dy})
	-- ({8.1968*\dx},{0.0000*\dy})
	-- ({8.2068*\dx},{0.0000*\dy})
	-- ({8.2168*\dx},{0.0000*\dy})
	-- ({8.2269*\dx},{0.0000*\dy})
	-- ({8.2369*\dx},{0.0000*\dy})
	-- ({8.2469*\dx},{0.0000*\dy})
	-- ({8.2569*\dx},{0.0000*\dy})
	-- ({8.2669*\dx},{0.0000*\dy})
	-- ({8.2769*\dx},{0.0000*\dy})
	-- ({8.2869*\dx},{0.0000*\dy})
	-- ({8.2969*\dx},{0.0000*\dy})
	-- ({8.3069*\dx},{0.0000*\dy})
	-- ({8.3169*\dx},{0.0000*\dy})
	-- ({8.3269*\dx},{0.0000*\dy})
	-- ({8.3369*\dx},{0.0000*\dy})
	-- ({8.3470*\dx},{0.0000*\dy})
	-- ({8.3570*\dx},{0.0000*\dy})
	-- ({8.3670*\dx},{0.0000*\dy})
	-- ({8.3770*\dx},{0.0000*\dy})
	-- ({8.3870*\dx},{0.0000*\dy})
	-- ({8.3970*\dx},{0.0000*\dy})
	-- ({8.4070*\dx},{0.0000*\dy})
	-- ({8.4170*\dx},{0.0000*\dy})
	-- ({8.4270*\dx},{0.0000*\dy})
	-- ({8.4370*\dx},{0.0000*\dy})
	-- ({8.4470*\dx},{0.0000*\dy})
	-- ({8.4570*\dx},{0.0000*\dy})
	-- ({8.4671*\dx},{0.0000*\dy})
	-- ({8.4771*\dx},{0.0000*\dy})
	-- ({8.4871*\dx},{0.0000*\dy})
	-- ({8.4971*\dx},{0.0000*\dy})
	-- ({8.5071*\dx},{0.0000*\dy})
	-- ({8.5171*\dx},{0.0000*\dy})
	-- ({8.5271*\dx},{0.0000*\dy})
	-- ({8.5371*\dx},{0.0000*\dy})
	-- ({8.5471*\dx},{0.0000*\dy})
	-- ({8.5571*\dx},{0.0000*\dy})
	-- ({8.5671*\dx},{0.0000*\dy})
	-- ({8.5771*\dx},{0.0000*\dy})
	-- ({8.5872*\dx},{0.0000*\dy})
	-- ({8.5972*\dx},{0.0000*\dy})
	-- ({8.6072*\dx},{0.0000*\dy})
	-- ({8.6172*\dx},{0.0000*\dy})
	-- ({8.6272*\dx},{0.0000*\dy})
	-- ({8.6372*\dx},{0.0000*\dy})
	-- ({8.6472*\dx},{0.0000*\dy})
	-- ({8.6572*\dx},{0.0000*\dy})
	-- ({8.6672*\dx},{0.0000*\dy})
	-- ({8.6772*\dx},{0.0000*\dy})
	-- ({8.6872*\dx},{0.0000*\dy})
	-- ({8.6972*\dx},{0.0000*\dy})
	-- ({8.7073*\dx},{0.0000*\dy})
	-- ({8.7173*\dx},{0.0000*\dy})
	-- ({8.7273*\dx},{0.0000*\dy})
	-- ({8.7373*\dx},{0.0000*\dy})
	-- ({8.7473*\dx},{0.0000*\dy})
	-- ({8.7573*\dx},{0.0000*\dy})
	-- ({8.7673*\dx},{0.0000*\dy})
	-- ({8.7773*\dx},{0.0000*\dy})
	-- ({8.7873*\dx},{0.0000*\dy})
	-- ({8.7973*\dx},{0.0000*\dy})
	-- ({8.8073*\dx},{0.0000*\dy})
	-- ({8.8173*\dx},{0.0000*\dy})
	-- ({8.8274*\dx},{0.0000*\dy})
	-- ({8.8374*\dx},{0.0000*\dy})
	-- ({8.8474*\dx},{0.0000*\dy})
	-- ({8.8574*\dx},{0.0000*\dy})
	-- ({8.8674*\dx},{0.0000*\dy})
	-- ({8.8774*\dx},{0.0000*\dy})
	-- ({8.8874*\dx},{0.0000*\dy})
	-- ({8.8974*\dx},{0.0000*\dy})
	-- ({8.9074*\dx},{0.0000*\dy})
	-- ({8.9174*\dx},{0.0000*\dy})
	-- ({8.9274*\dx},{0.0000*\dy})
	-- ({8.9374*\dx},{0.0000*\dy})
	-- ({8.9475*\dx},{0.0000*\dy})
	-- ({8.9575*\dx},{0.0000*\dy})
	-- ({8.9675*\dx},{0.0000*\dy})
	-- ({8.9775*\dx},{0.0000*\dy})
	-- ({8.9875*\dx},{0.0000*\dy})
	-- ({8.9975*\dx},{0.0000*\dy})
	-- ({9.0075*\dx},{0.0000*\dy})
	-- ({9.0175*\dx},{0.0000*\dy})
	-- ({9.0275*\dx},{0.0000*\dy})
	-- ({9.0375*\dx},{0.0000*\dy})
	-- ({9.0475*\dx},{0.0000*\dy})
	-- ({9.0575*\dx},{0.0000*\dy})
	-- ({9.0676*\dx},{0.0000*\dy})
	-- ({9.0776*\dx},{0.0000*\dy})
	-- ({9.0876*\dx},{0.0000*\dy})
	-- ({9.0976*\dx},{0.0000*\dy})
	-- ({9.1076*\dx},{0.0000*\dy})
	-- ({9.1176*\dx},{0.0000*\dy})
	-- ({9.1276*\dx},{0.0000*\dy})
	-- ({9.1376*\dx},{0.0000*\dy})
	-- ({9.1476*\dx},{0.0000*\dy})
	-- ({9.1576*\dx},{0.0000*\dy})
	-- ({9.1676*\dx},{0.0000*\dy})
	-- ({9.1776*\dx},{0.0000*\dy})
	-- ({9.1877*\dx},{0.0000*\dy})
	-- ({9.1977*\dx},{0.0000*\dy})
	-- ({9.2077*\dx},{0.0000*\dy})
	-- ({9.2177*\dx},{0.0000*\dy})
	-- ({9.2277*\dx},{0.0000*\dy})
	-- ({9.2377*\dx},{0.0000*\dy})
	-- ({9.2477*\dx},{0.0000*\dy})
	-- ({9.2577*\dx},{0.0000*\dy})
	-- ({9.2677*\dx},{0.0000*\dy})
	-- ({9.2777*\dx},{0.0000*\dy})
	-- ({9.2877*\dx},{0.0000*\dy})
	-- ({9.2977*\dx},{0.0000*\dy})
	-- ({9.3078*\dx},{0.0000*\dy})
	-- ({9.3178*\dx},{0.0000*\dy})
	-- ({9.3278*\dx},{0.0000*\dy})
	-- ({9.3378*\dx},{0.0000*\dy})
	-- ({9.3478*\dx},{0.0000*\dy})
	-- ({9.3578*\dx},{0.0000*\dy})
	-- ({9.3678*\dx},{0.0000*\dy})
	-- ({9.3778*\dx},{0.0000*\dy})
	-- ({9.3878*\dx},{0.0000*\dy})
	-- ({9.3978*\dx},{0.0000*\dy})
	-- ({9.4078*\dx},{0.0000*\dy})
	-- ({9.4178*\dx},{0.0000*\dy})
	-- ({9.4279*\dx},{0.0000*\dy})
	-- ({9.4379*\dx},{0.0000*\dy})
	-- ({9.4479*\dx},{0.0000*\dy})
	-- ({9.4579*\dx},{0.0000*\dy})
	-- ({9.4679*\dx},{0.0000*\dy})
	-- ({9.4779*\dx},{0.0000*\dy})
	-- ({9.4879*\dx},{0.0000*\dy})
	-- ({9.4979*\dx},{0.0000*\dy})
	-- ({9.5079*\dx},{0.0000*\dy})
	-- ({9.5179*\dx},{0.0000*\dy})
	-- ({9.5279*\dx},{0.0000*\dy})
	-- ({9.5379*\dx},{0.0000*\dy})
	-- ({9.5480*\dx},{0.0000*\dy})
	-- ({9.5580*\dx},{0.0000*\dy})
	-- ({9.5680*\dx},{0.0000*\dy})
	-- ({9.5780*\dx},{0.0000*\dy})
	-- ({9.5880*\dx},{0.0000*\dy})
	-- ({9.5980*\dx},{0.0000*\dy})
	-- ({9.6080*\dx},{0.0000*\dy})
	-- ({9.6180*\dx},{0.0000*\dy})
	-- ({9.6280*\dx},{0.0000*\dy})
	-- ({9.6380*\dx},{0.0000*\dy})
	-- ({9.6480*\dx},{0.0000*\dy})
	-- ({9.6580*\dx},{0.0000*\dy})
	-- ({9.6681*\dx},{0.0000*\dy})
	-- ({9.6781*\dx},{0.0000*\dy})
	-- ({9.6881*\dx},{0.0000*\dy})
	-- ({9.6981*\dx},{0.0000*\dy})
	-- ({9.7081*\dx},{0.0000*\dy})
	-- ({9.7181*\dx},{0.0000*\dy})
	-- ({9.7281*\dx},{0.0000*\dy})
	-- ({9.7381*\dx},{0.0000*\dy})
	-- ({9.7481*\dx},{0.0000*\dy})
	-- ({9.7581*\dx},{0.0000*\dy})
	-- ({9.7681*\dx},{0.0000*\dy})
	-- ({9.7781*\dx},{0.0000*\dy})
	-- ({9.7882*\dx},{0.0000*\dy})
	-- ({9.7982*\dx},{0.0000*\dy})
	-- ({9.8082*\dx},{0.0000*\dy})
	-- ({9.8182*\dx},{0.0000*\dy})
	-- ({9.8282*\dx},{0.0000*\dy})
	-- ({9.8382*\dx},{0.0000*\dy})
	-- ({9.8482*\dx},{0.0000*\dy})
	-- ({9.8582*\dx},{0.0000*\dy})
	-- ({9.8682*\dx},{0.0000*\dy})
	-- ({9.8782*\dx},{0.0000*\dy})
	-- ({9.8882*\dx},{0.0000*\dy})
	-- ({9.8982*\dx},{0.0000*\dy})
	-- ({9.9083*\dx},{0.0000*\dy})
	-- ({9.9183*\dx},{0.0000*\dy})
	-- ({9.9283*\dx},{0.0000*\dy})
	-- ({9.9383*\dx},{0.0000*\dy})
	-- ({9.9483*\dx},{0.0000*\dy})
	-- ({9.9583*\dx},{0.0000*\dy})
	-- ({9.9683*\dx},{0.0000*\dy})
	-- ({9.9783*\dx},{0.0000*\dy})
	-- ({9.9883*\dx},{0.0000*\dy})
	-- ({9.9983*\dx},{0.0000*\dy})
	-- ({10.0083*\dx},{0.0000*\dy})
	-- ({10.0183*\dx},{0.0000*\dy})
	-- ({10.0284*\dx},{0.0000*\dy})
	-- ({10.0384*\dx},{0.0000*\dy})
	-- ({10.0484*\dx},{0.0000*\dy})
	-- ({10.0584*\dx},{0.0000*\dy})
	-- ({10.0684*\dx},{0.0000*\dy})
	-- ({10.0784*\dx},{0.0000*\dy})
	-- ({10.0884*\dx},{0.0000*\dy})
	-- ({10.0984*\dx},{0.0000*\dy})
	-- ({10.1084*\dx},{0.0000*\dy})
	-- ({10.1184*\dx},{0.0000*\dy})
	-- ({10.1284*\dx},{0.0000*\dy})
	-- ({10.1384*\dx},{0.0000*\dy})
	-- ({10.1485*\dx},{0.0000*\dy})
	-- ({10.1585*\dx},{0.0000*\dy})
	-- ({10.1685*\dx},{0.0000*\dy})
	-- ({10.1785*\dx},{0.0000*\dy})
	-- ({10.1885*\dx},{0.0000*\dy})
	-- ({10.1985*\dx},{0.0000*\dy})
	-- ({10.2085*\dx},{0.0000*\dy})
	-- ({10.2185*\dx},{0.0000*\dy})
	-- ({10.2285*\dx},{0.0000*\dy})
	-- ({10.2385*\dx},{0.0000*\dy})
	-- ({10.2485*\dx},{0.0000*\dy})
	-- ({10.2585*\dx},{0.0000*\dy})
	-- ({10.2686*\dx},{0.0000*\dy})
	-- ({10.2786*\dx},{0.0000*\dy})
	-- ({10.2886*\dx},{0.0000*\dy})
	-- ({10.2986*\dx},{0.0000*\dy})
	-- ({10.3086*\dx},{0.0000*\dy})
	-- ({10.3186*\dx},{0.0000*\dy})
	-- ({10.3286*\dx},{0.0000*\dy})
	-- ({10.3386*\dx},{0.0000*\dy})
	-- ({10.3486*\dx},{0.0000*\dy})
	-- ({10.3586*\dx},{0.0000*\dy})
	-- ({10.3686*\dx},{0.0000*\dy})
	-- ({10.3786*\dx},{0.0000*\dy})
	-- ({10.3887*\dx},{0.0000*\dy})
	-- ({10.3987*\dx},{0.0000*\dy})
	-- ({10.4087*\dx},{0.0000*\dy})
	-- ({10.4187*\dx},{0.0000*\dy})
	-- ({10.4287*\dx},{0.0000*\dy})
	-- ({10.4387*\dx},{0.0000*\dy})
	-- ({10.4487*\dx},{0.0000*\dy})
	-- ({10.4587*\dx},{0.0000*\dy})
	-- ({10.4687*\dx},{0.0000*\dy})
	-- ({10.4787*\dx},{0.0000*\dy})
	-- ({10.4887*\dx},{0.0000*\dy})
	-- ({10.4987*\dx},{0.0000*\dy})
	-- ({10.5088*\dx},{0.0000*\dy})
	-- ({10.5188*\dx},{0.0000*\dy})
	-- ({10.5288*\dx},{0.0000*\dy})
	-- ({10.5388*\dx},{0.0000*\dy})
	-- ({10.5488*\dx},{0.0000*\dy})
	-- ({10.5588*\dx},{0.0000*\dy})
	-- ({10.5688*\dx},{0.0000*\dy})
	-- ({10.5788*\dx},{0.0000*\dy})
	-- ({10.5888*\dx},{0.0000*\dy})
	-- ({10.5988*\dx},{0.0000*\dy})
	-- ({10.6088*\dx},{0.0000*\dy})
	-- ({10.6188*\dx},{0.0000*\dy})
	-- ({10.6289*\dx},{0.0000*\dy})
	-- ({10.6389*\dx},{0.0000*\dy})
	-- ({10.6489*\dx},{0.0000*\dy})
	-- ({10.6589*\dx},{0.0000*\dy})
	-- ({10.6689*\dx},{0.0000*\dy})
	-- ({10.6789*\dx},{0.0000*\dy})
	-- ({10.6889*\dx},{0.0000*\dy})
	-- ({10.6989*\dx},{0.0000*\dy})
	-- ({10.7089*\dx},{0.0000*\dy})
	-- ({10.7189*\dx},{0.0000*\dy})
	-- ({10.7289*\dx},{0.0000*\dy})
	-- ({10.7389*\dx},{0.0000*\dy})
	-- ({10.7490*\dx},{0.0000*\dy})
	-- ({10.7590*\dx},{0.0000*\dy})
	-- ({10.7690*\dx},{0.0000*\dy})
	-- ({10.7790*\dx},{0.0000*\dy})
	-- ({10.7890*\dx},{0.0000*\dy})
	-- ({10.7990*\dx},{0.0000*\dy})
	-- ({10.8090*\dx},{0.0000*\dy})
	-- ({10.8190*\dx},{0.0000*\dy})
	-- ({10.8290*\dx},{0.0000*\dy})
	-- ({10.8390*\dx},{0.0000*\dy})
	-- ({10.8490*\dx},{0.0000*\dy})
	-- ({10.8590*\dx},{0.0000*\dy})
	-- ({10.8691*\dx},{0.0000*\dy})
	-- ({10.8791*\dx},{0.0000*\dy})
	-- ({10.8891*\dx},{0.0000*\dy})
	-- ({10.8991*\dx},{0.0000*\dy})
	-- ({10.9091*\dx},{0.0000*\dy})
	-- ({10.9191*\dx},{0.0000*\dy})
	-- ({10.9291*\dx},{0.0000*\dy})
	-- ({10.9391*\dx},{0.0000*\dy})
	-- ({10.9491*\dx},{0.0000*\dy})
	-- ({10.9591*\dx},{0.0000*\dy})
	-- ({10.9691*\dx},{0.0000*\dy})
	-- ({10.9791*\dx},{0.0000*\dy})
	-- ({10.9892*\dx},{0.0000*\dy})
	-- ({10.9992*\dx},{0.0000*\dy})
	-- ({11.0092*\dx},{0.0000*\dy})
	-- ({11.0192*\dx},{0.0000*\dy})
	-- ({11.0292*\dx},{0.0000*\dy})
	-- ({11.0392*\dx},{0.0000*\dy})
	-- ({11.0492*\dx},{0.0000*\dy})
	-- ({11.0592*\dx},{0.0000*\dy})
	-- ({11.0692*\dx},{0.0000*\dy})
	-- ({11.0792*\dx},{0.0000*\dy})
	-- ({11.0892*\dx},{0.0000*\dy})
	-- ({11.0992*\dx},{0.0000*\dy})
	-- ({11.1093*\dx},{0.0000*\dy})
	-- ({11.1193*\dx},{0.0000*\dy})
	-- ({11.1293*\dx},{0.0000*\dy})
	-- ({11.1393*\dx},{0.0000*\dy})
	-- ({11.1493*\dx},{0.0000*\dy})
	-- ({11.1593*\dx},{0.0000*\dy})
	-- ({11.1693*\dx},{0.0000*\dy})
	-- ({11.1793*\dx},{0.0000*\dy})
	-- ({11.1893*\dx},{0.0000*\dy})
	-- ({11.1993*\dx},{0.0000*\dy})
	-- ({11.2093*\dx},{0.0000*\dy})
	-- ({11.2193*\dx},{0.0000*\dy})
	-- ({11.2294*\dx},{0.0000*\dy})
	-- ({11.2394*\dx},{0.0000*\dy})
	-- ({11.2494*\dx},{0.0000*\dy})
	-- ({11.2594*\dx},{0.0000*\dy})
	-- ({11.2694*\dx},{0.0000*\dy})
	-- ({11.2794*\dx},{0.0000*\dy})
	-- ({11.2894*\dx},{0.0000*\dy})
	-- ({11.2994*\dx},{0.0000*\dy})
	-- ({11.3094*\dx},{0.0000*\dy})
	-- ({11.3194*\dx},{0.0000*\dy})
	-- ({11.3294*\dx},{0.0000*\dy})
	-- ({11.3394*\dx},{0.0000*\dy})
	-- ({11.3495*\dx},{0.0000*\dy})
	-- ({11.3595*\dx},{0.0000*\dy})
	-- ({11.3695*\dx},{0.0000*\dy})
	-- ({11.3795*\dx},{0.0000*\dy})
	-- ({11.3895*\dx},{0.0000*\dy})
	-- ({11.3995*\dx},{0.0000*\dy})
	-- ({11.4095*\dx},{0.0000*\dy})
	-- ({11.4195*\dx},{0.0000*\dy})
	-- ({11.4295*\dx},{0.0000*\dy})
	-- ({11.4395*\dx},{0.0000*\dy})
	-- ({11.4495*\dx},{0.0000*\dy})
	-- ({11.4595*\dx},{0.0000*\dy})
	-- ({11.4696*\dx},{0.0000*\dy})
	-- ({11.4796*\dx},{0.0000*\dy})
	-- ({11.4896*\dx},{0.0000*\dy})
	-- ({11.4996*\dx},{0.0000*\dy})
	-- ({11.5096*\dx},{0.0000*\dy})
	-- ({11.5196*\dx},{0.0000*\dy})
	-- ({11.5296*\dx},{0.0000*\dy})
	-- ({11.5396*\dx},{0.0000*\dy})
	-- ({11.5496*\dx},{0.0000*\dy})
	-- ({11.5596*\dx},{0.0000*\dy})
	-- ({11.5696*\dx},{0.0000*\dy})
	-- ({11.5796*\dx},{0.0000*\dy})
	-- ({11.5897*\dx},{0.0000*\dy})
	-- ({11.5997*\dx},{0.0000*\dy})
	-- ({11.6097*\dx},{0.0000*\dy})
	-- ({11.6197*\dx},{0.0000*\dy})
	-- ({11.6297*\dx},{0.0000*\dy})
	-- ({11.6397*\dx},{0.0000*\dy})
	-- ({11.6497*\dx},{0.0000*\dy})
	-- ({11.6597*\dx},{0.0000*\dy})
	-- ({11.6697*\dx},{0.0000*\dy})
	-- ({11.6797*\dx},{0.0000*\dy})
	-- ({11.6897*\dx},{0.0000*\dy})
	-- ({11.6997*\dx},{0.0000*\dy})
	-- ({11.7098*\dx},{0.0000*\dy})
	-- ({11.7198*\dx},{0.0000*\dy})
	-- ({11.7298*\dx},{0.0000*\dy})
	-- ({11.7398*\dx},{0.0000*\dy})
	-- ({11.7498*\dx},{0.0000*\dy})
	-- ({11.7598*\dx},{0.0000*\dy})
	-- ({11.7698*\dx},{0.0000*\dy})
	-- ({11.7798*\dx},{0.0000*\dy})
	-- ({11.7898*\dx},{0.0000*\dy})
	-- ({11.7998*\dx},{0.0000*\dy})
	-- ({11.8098*\dx},{0.0000*\dy})
	-- ({11.8198*\dx},{0.0000*\dy})
	-- ({11.8299*\dx},{0.0000*\dy})
	-- ({11.8399*\dx},{0.0000*\dy})
	-- ({11.8499*\dx},{0.0000*\dy})
	-- ({11.8599*\dx},{0.0000*\dy})
	-- ({11.8699*\dx},{0.0000*\dy})
	-- ({11.8799*\dx},{0.0000*\dy})
	-- ({11.8899*\dx},{0.0000*\dy})
	-- ({11.8999*\dx},{0.0000*\dy})
	-- ({11.9099*\dx},{0.0000*\dy})
	-- ({11.9199*\dx},{0.0000*\dy})
	-- ({11.9299*\dx},{0.0000*\dy})
	-- ({11.9399*\dx},{0.0000*\dy})
	-- ({11.9500*\dx},{0.0000*\dy})
	-- ({11.9600*\dx},{0.0000*\dy})
	-- ({11.9700*\dx},{0.0000*\dy})
	-- ({11.9800*\dx},{0.0000*\dy})
	-- ({11.9900*\dx},{0.0000*\dy})
	-- ({12.0000*\dx},{0.0000*\dy})
}
\def\psifour{
	({0.0000*\dx},{0.0000*\dy})
	-- ({0.0100*\dx},{0.0000*\dy})
	-- ({0.0200*\dx},{0.0000*\dy})
	-- ({0.0300*\dx},{0.0000*\dy})
	-- ({0.0400*\dx},{0.0000*\dy})
	-- ({0.0500*\dx},{0.0000*\dy})
	-- ({0.0601*\dx},{0.0000*\dy})
	-- ({0.0701*\dx},{0.0000*\dy})
	-- ({0.0801*\dx},{0.0000*\dy})
	-- ({0.0901*\dx},{0.0000*\dy})
	-- ({0.1001*\dx},{0.0000*\dy})
	-- ({0.1101*\dx},{0.0000*\dy})
	-- ({0.1201*\dx},{0.0000*\dy})
	-- ({0.1301*\dx},{0.0000*\dy})
	-- ({0.1401*\dx},{0.0000*\dy})
	-- ({0.1501*\dx},{0.0000*\dy})
	-- ({0.1601*\dx},{0.0000*\dy})
	-- ({0.1701*\dx},{0.0000*\dy})
	-- ({0.1802*\dx},{0.0000*\dy})
	-- ({0.1902*\dx},{0.0000*\dy})
	-- ({0.2002*\dx},{0.0000*\dy})
	-- ({0.2102*\dx},{0.0000*\dy})
	-- ({0.2202*\dx},{0.0000*\dy})
	-- ({0.2302*\dx},{0.0000*\dy})
	-- ({0.2402*\dx},{0.0000*\dy})
	-- ({0.2502*\dx},{0.0000*\dy})
	-- ({0.2602*\dx},{0.0000*\dy})
	-- ({0.2702*\dx},{0.0000*\dy})
	-- ({0.2802*\dx},{0.0000*\dy})
	-- ({0.2902*\dx},{0.0000*\dy})
	-- ({0.3003*\dx},{0.0000*\dy})
	-- ({0.3103*\dx},{0.0000*\dy})
	-- ({0.3203*\dx},{0.0000*\dy})
	-- ({0.3303*\dx},{0.0000*\dy})
	-- ({0.3403*\dx},{0.0000*\dy})
	-- ({0.3503*\dx},{0.0000*\dy})
	-- ({0.3603*\dx},{0.0000*\dy})
	-- ({0.3703*\dx},{0.0000*\dy})
	-- ({0.3803*\dx},{0.0000*\dy})
	-- ({0.3903*\dx},{0.0000*\dy})
	-- ({0.4003*\dx},{0.0000*\dy})
	-- ({0.4103*\dx},{0.0000*\dy})
	-- ({0.4204*\dx},{0.0000*\dy})
	-- ({0.4304*\dx},{0.0000*\dy})
	-- ({0.4404*\dx},{0.0000*\dy})
	-- ({0.4504*\dx},{0.0000*\dy})
	-- ({0.4604*\dx},{0.0000*\dy})
	-- ({0.4704*\dx},{0.0000*\dy})
	-- ({0.4804*\dx},{0.0000*\dy})
	-- ({0.4904*\dx},{0.0000*\dy})
	-- ({0.5004*\dx},{0.0000*\dy})
	-- ({0.5104*\dx},{0.0000*\dy})
	-- ({0.5204*\dx},{0.0000*\dy})
	-- ({0.5304*\dx},{0.0000*\dy})
	-- ({0.5405*\dx},{0.0000*\dy})
	-- ({0.5505*\dx},{0.0000*\dy})
	-- ({0.5605*\dx},{0.0000*\dy})
	-- ({0.5705*\dx},{0.0000*\dy})
	-- ({0.5805*\dx},{0.0000*\dy})
	-- ({0.5905*\dx},{0.0000*\dy})
	-- ({0.6005*\dx},{0.0000*\dy})
	-- ({0.6105*\dx},{0.0000*\dy})
	-- ({0.6205*\dx},{0.0000*\dy})
	-- ({0.6305*\dx},{0.0000*\dy})
	-- ({0.6405*\dx},{0.0000*\dy})
	-- ({0.6505*\dx},{0.0000*\dy})
	-- ({0.6606*\dx},{0.0000*\dy})
	-- ({0.6706*\dx},{0.0000*\dy})
	-- ({0.6806*\dx},{0.0000*\dy})
	-- ({0.6906*\dx},{0.0000*\dy})
	-- ({0.7006*\dx},{0.0000*\dy})
	-- ({0.7106*\dx},{0.0000*\dy})
	-- ({0.7206*\dx},{0.0000*\dy})
	-- ({0.7306*\dx},{0.0000*\dy})
	-- ({0.7406*\dx},{0.0000*\dy})
	-- ({0.7506*\dx},{0.0000*\dy})
	-- ({0.7606*\dx},{0.0000*\dy})
	-- ({0.7706*\dx},{0.0000*\dy})
	-- ({0.7807*\dx},{0.0000*\dy})
	-- ({0.7907*\dx},{0.0000*\dy})
	-- ({0.8007*\dx},{0.0000*\dy})
	-- ({0.8107*\dx},{0.0000*\dy})
	-- ({0.8207*\dx},{0.0000*\dy})
	-- ({0.8307*\dx},{0.0000*\dy})
	-- ({0.8407*\dx},{0.0000*\dy})
	-- ({0.8507*\dx},{0.0000*\dy})
	-- ({0.8607*\dx},{0.0000*\dy})
	-- ({0.8707*\dx},{0.0000*\dy})
	-- ({0.8807*\dx},{0.0000*\dy})
	-- ({0.8907*\dx},{0.0000*\dy})
	-- ({0.9008*\dx},{0.0000*\dy})
	-- ({0.9108*\dx},{0.0000*\dy})
	-- ({0.9208*\dx},{0.0000*\dy})
	-- ({0.9308*\dx},{0.0000*\dy})
	-- ({0.9408*\dx},{0.0000*\dy})
	-- ({0.9508*\dx},{0.0000*\dy})
	-- ({0.9608*\dx},{0.0000*\dy})
	-- ({0.9708*\dx},{0.0000*\dy})
	-- ({0.9808*\dx},{0.0000*\dy})
	-- ({0.9908*\dx},{0.0000*\dy})
	-- ({1.0008*\dx},{0.0000*\dy})
	-- ({1.0108*\dx},{0.0000*\dy})
	-- ({1.0209*\dx},{0.0000*\dy})
	-- ({1.0309*\dx},{0.0000*\dy})
	-- ({1.0409*\dx},{0.0000*\dy})
	-- ({1.0509*\dx},{0.0000*\dy})
	-- ({1.0609*\dx},{0.0000*\dy})
	-- ({1.0709*\dx},{0.0000*\dy})
	-- ({1.0809*\dx},{0.0000*\dy})
	-- ({1.0909*\dx},{0.0000*\dy})
	-- ({1.1009*\dx},{0.0000*\dy})
	-- ({1.1109*\dx},{0.0000*\dy})
	-- ({1.1209*\dx},{0.0000*\dy})
	-- ({1.1309*\dx},{0.0000*\dy})
	-- ({1.1410*\dx},{0.0000*\dy})
	-- ({1.1510*\dx},{0.0000*\dy})
	-- ({1.1610*\dx},{0.0000*\dy})
	-- ({1.1710*\dx},{0.0000*\dy})
	-- ({1.1810*\dx},{0.0000*\dy})
	-- ({1.1910*\dx},{0.0000*\dy})
	-- ({1.2010*\dx},{0.0000*\dy})
	-- ({1.2110*\dx},{0.0000*\dy})
	-- ({1.2210*\dx},{0.0000*\dy})
	-- ({1.2310*\dx},{0.0000*\dy})
	-- ({1.2410*\dx},{0.0000*\dy})
	-- ({1.2510*\dx},{0.0000*\dy})
	-- ({1.2611*\dx},{0.0000*\dy})
	-- ({1.2711*\dx},{0.0000*\dy})
	-- ({1.2811*\dx},{0.0000*\dy})
	-- ({1.2911*\dx},{0.0000*\dy})
	-- ({1.3011*\dx},{0.0000*\dy})
	-- ({1.3111*\dx},{0.0000*\dy})
	-- ({1.3211*\dx},{0.0000*\dy})
	-- ({1.3311*\dx},{0.0000*\dy})
	-- ({1.3411*\dx},{0.0000*\dy})
	-- ({1.3511*\dx},{0.0000*\dy})
	-- ({1.3611*\dx},{0.0000*\dy})
	-- ({1.3711*\dx},{0.0000*\dy})
	-- ({1.3812*\dx},{0.0000*\dy})
	-- ({1.3912*\dx},{0.0000*\dy})
	-- ({1.4012*\dx},{0.0000*\dy})
	-- ({1.4112*\dx},{0.0000*\dy})
	-- ({1.4212*\dx},{0.0000*\dy})
	-- ({1.4312*\dx},{0.0000*\dy})
	-- ({1.4412*\dx},{0.0000*\dy})
	-- ({1.4512*\dx},{0.0000*\dy})
	-- ({1.4612*\dx},{0.0000*\dy})
	-- ({1.4712*\dx},{0.0000*\dy})
	-- ({1.4812*\dx},{0.0000*\dy})
	-- ({1.4912*\dx},{0.0000*\dy})
	-- ({1.5013*\dx},{0.0000*\dy})
	-- ({1.5113*\dx},{0.0000*\dy})
	-- ({1.5213*\dx},{0.0000*\dy})
	-- ({1.5313*\dx},{0.0000*\dy})
	-- ({1.5413*\dx},{0.0000*\dy})
	-- ({1.5513*\dx},{0.0000*\dy})
	-- ({1.5613*\dx},{0.0000*\dy})
	-- ({1.5713*\dx},{0.0000*\dy})
	-- ({1.5813*\dx},{0.0000*\dy})
	-- ({1.5913*\dx},{0.0000*\dy})
	-- ({1.6013*\dx},{0.0000*\dy})
	-- ({1.6113*\dx},{0.0000*\dy})
	-- ({1.6214*\dx},{0.0000*\dy})
	-- ({1.6314*\dx},{0.0000*\dy})
	-- ({1.6414*\dx},{0.0000*\dy})
	-- ({1.6514*\dx},{0.0000*\dy})
	-- ({1.6614*\dx},{0.0000*\dy})
	-- ({1.6714*\dx},{0.0000*\dy})
	-- ({1.6814*\dx},{0.0000*\dy})
	-- ({1.6914*\dx},{0.0000*\dy})
	-- ({1.7014*\dx},{0.0000*\dy})
	-- ({1.7114*\dx},{0.0000*\dy})
	-- ({1.7214*\dx},{0.0000*\dy})
	-- ({1.7314*\dx},{0.0000*\dy})
	-- ({1.7415*\dx},{0.0000*\dy})
	-- ({1.7515*\dx},{0.0000*\dy})
	-- ({1.7615*\dx},{0.0000*\dy})
	-- ({1.7715*\dx},{0.0000*\dy})
	-- ({1.7815*\dx},{0.0000*\dy})
	-- ({1.7915*\dx},{0.0000*\dy})
	-- ({1.8015*\dx},{0.0000*\dy})
	-- ({1.8115*\dx},{0.0000*\dy})
	-- ({1.8215*\dx},{0.0000*\dy})
	-- ({1.8315*\dx},{0.0000*\dy})
	-- ({1.8415*\dx},{0.0000*\dy})
	-- ({1.8515*\dx},{0.0000*\dy})
	-- ({1.8616*\dx},{0.0000*\dy})
	-- ({1.8716*\dx},{0.0000*\dy})
	-- ({1.8816*\dx},{0.0000*\dy})
	-- ({1.8916*\dx},{0.0000*\dy})
	-- ({1.9016*\dx},{0.0000*\dy})
	-- ({1.9116*\dx},{0.0000*\dy})
	-- ({1.9216*\dx},{0.0000*\dy})
	-- ({1.9316*\dx},{0.0000*\dy})
	-- ({1.9416*\dx},{0.0000*\dy})
	-- ({1.9516*\dx},{0.0000*\dy})
	-- ({1.9616*\dx},{0.0000*\dy})
	-- ({1.9716*\dx},{0.0000*\dy})
	-- ({1.9817*\dx},{0.0000*\dy})
	-- ({1.9917*\dx},{0.0000*\dy})
	-- ({2.0017*\dx},{0.0000*\dy})
	-- ({2.0117*\dx},{0.0000*\dy})
	-- ({2.0217*\dx},{0.0000*\dy})
	-- ({2.0317*\dx},{0.0000*\dy})
	-- ({2.0417*\dx},{0.0000*\dy})
	-- ({2.0517*\dx},{0.0000*\dy})
	-- ({2.0617*\dx},{0.0000*\dy})
	-- ({2.0717*\dx},{0.0000*\dy})
	-- ({2.0817*\dx},{0.0000*\dy})
	-- ({2.0917*\dx},{0.0000*\dy})
	-- ({2.1018*\dx},{0.0000*\dy})
	-- ({2.1118*\dx},{0.0000*\dy})
	-- ({2.1218*\dx},{0.0000*\dy})
	-- ({2.1318*\dx},{0.0000*\dy})
	-- ({2.1418*\dx},{0.0000*\dy})
	-- ({2.1518*\dx},{0.0000*\dy})
	-- ({2.1618*\dx},{0.0000*\dy})
	-- ({2.1718*\dx},{0.0000*\dy})
	-- ({2.1818*\dx},{0.0000*\dy})
	-- ({2.1918*\dx},{0.0000*\dy})
	-- ({2.2018*\dx},{0.0000*\dy})
	-- ({2.2118*\dx},{0.0000*\dy})
	-- ({2.2219*\dx},{0.0000*\dy})
	-- ({2.2319*\dx},{0.0000*\dy})
	-- ({2.2419*\dx},{0.0000*\dy})
	-- ({2.2519*\dx},{0.0000*\dy})
	-- ({2.2619*\dx},{0.0000*\dy})
	-- ({2.2719*\dx},{0.0000*\dy})
	-- ({2.2819*\dx},{0.0000*\dy})
	-- ({2.2919*\dx},{0.0000*\dy})
	-- ({2.3019*\dx},{0.0000*\dy})
	-- ({2.3119*\dx},{0.0000*\dy})
	-- ({2.3219*\dx},{0.0000*\dy})
	-- ({2.3319*\dx},{0.0000*\dy})
	-- ({2.3420*\dx},{0.0000*\dy})
	-- ({2.3520*\dx},{0.0000*\dy})
	-- ({2.3620*\dx},{0.0000*\dy})
	-- ({2.3720*\dx},{0.0000*\dy})
	-- ({2.3820*\dx},{0.0000*\dy})
	-- ({2.3920*\dx},{0.0000*\dy})
	-- ({2.4020*\dx},{0.0000*\dy})
	-- ({2.4120*\dx},{0.0000*\dy})
	-- ({2.4220*\dx},{0.0000*\dy})
	-- ({2.4320*\dx},{0.0000*\dy})
	-- ({2.4420*\dx},{0.0000*\dy})
	-- ({2.4520*\dx},{0.0000*\dy})
	-- ({2.4621*\dx},{0.0000*\dy})
	-- ({2.4721*\dx},{0.0000*\dy})
	-- ({2.4821*\dx},{0.0000*\dy})
	-- ({2.4921*\dx},{0.0000*\dy})
	-- ({2.5021*\dx},{0.0000*\dy})
	-- ({2.5121*\dx},{0.0000*\dy})
	-- ({2.5221*\dx},{0.0000*\dy})
	-- ({2.5321*\dx},{0.0000*\dy})
	-- ({2.5421*\dx},{0.0000*\dy})
	-- ({2.5521*\dx},{0.0000*\dy})
	-- ({2.5621*\dx},{0.0000*\dy})
	-- ({2.5721*\dx},{0.0000*\dy})
	-- ({2.5822*\dx},{0.0000*\dy})
	-- ({2.5922*\dx},{0.0000*\dy})
	-- ({2.6022*\dx},{0.0000*\dy})
	-- ({2.6122*\dx},{0.0000*\dy})
	-- ({2.6222*\dx},{0.0000*\dy})
	-- ({2.6322*\dx},{0.0000*\dy})
	-- ({2.6422*\dx},{0.0000*\dy})
	-- ({2.6522*\dx},{0.0000*\dy})
	-- ({2.6622*\dx},{0.0000*\dy})
	-- ({2.6722*\dx},{0.0000*\dy})
	-- ({2.6822*\dx},{0.0000*\dy})
	-- ({2.6922*\dx},{0.0000*\dy})
	-- ({2.7023*\dx},{0.0000*\dy})
	-- ({2.7123*\dx},{0.0000*\dy})
	-- ({2.7223*\dx},{0.0000*\dy})
	-- ({2.7323*\dx},{0.0000*\dy})
	-- ({2.7423*\dx},{0.0000*\dy})
	-- ({2.7523*\dx},{0.0000*\dy})
	-- ({2.7623*\dx},{0.0000*\dy})
	-- ({2.7723*\dx},{0.0000*\dy})
	-- ({2.7823*\dx},{0.0000*\dy})
	-- ({2.7923*\dx},{0.0000*\dy})
	-- ({2.8023*\dx},{0.0000*\dy})
	-- ({2.8123*\dx},{0.0000*\dy})
	-- ({2.8224*\dx},{0.0000*\dy})
	-- ({2.8324*\dx},{0.0000*\dy})
	-- ({2.8424*\dx},{0.0000*\dy})
	-- ({2.8524*\dx},{0.0000*\dy})
	-- ({2.8624*\dx},{0.0000*\dy})
	-- ({2.8724*\dx},{0.0000*\dy})
	-- ({2.8824*\dx},{0.0000*\dy})
	-- ({2.8924*\dx},{0.0000*\dy})
	-- ({2.9024*\dx},{0.0000*\dy})
	-- ({2.9124*\dx},{0.0000*\dy})
	-- ({2.9224*\dx},{0.0000*\dy})
	-- ({2.9324*\dx},{0.0000*\dy})
	-- ({2.9425*\dx},{0.0000*\dy})
	-- ({2.9525*\dx},{0.0000*\dy})
	-- ({2.9625*\dx},{0.0000*\dy})
	-- ({2.9725*\dx},{0.0000*\dy})
	-- ({2.9825*\dx},{0.0000*\dy})
	-- ({2.9925*\dx},{0.0000*\dy})
	-- ({3.0025*\dx},{0.0000*\dy})
	-- ({3.0125*\dx},{0.0001*\dy})
	-- ({3.0225*\dx},{0.0003*\dy})
	-- ({3.0325*\dx},{0.0006*\dy})
	-- ({3.0425*\dx},{0.0009*\dy})
	-- ({3.0525*\dx},{0.0014*\dy})
	-- ({3.0626*\dx},{0.0019*\dy})
	-- ({3.0726*\dx},{0.0026*\dy})
	-- ({3.0826*\dx},{0.0033*\dy})
	-- ({3.0926*\dx},{0.0041*\dy})
	-- ({3.1026*\dx},{0.0049*\dy})
	-- ({3.1126*\dx},{0.0059*\dy})
	-- ({3.1226*\dx},{0.0069*\dy})
	-- ({3.1326*\dx},{0.0079*\dy})
	-- ({3.1426*\dx},{0.0090*\dy})
	-- ({3.1526*\dx},{0.0102*\dy})
	-- ({3.1626*\dx},{0.0114*\dy})
	-- ({3.1726*\dx},{0.0127*\dy})
	-- ({3.1827*\dx},{0.0140*\dy})
	-- ({3.1927*\dx},{0.0154*\dy})
	-- ({3.2027*\dx},{0.0168*\dy})
	-- ({3.2127*\dx},{0.0183*\dy})
	-- ({3.2227*\dx},{0.0198*\dy})
	-- ({3.2327*\dx},{0.0213*\dy})
	-- ({3.2427*\dx},{0.0229*\dy})
	-- ({3.2527*\dx},{0.0245*\dy})
	-- ({3.2627*\dx},{0.0261*\dy})
	-- ({3.2727*\dx},{0.0278*\dy})
	-- ({3.2827*\dx},{0.0295*\dy})
	-- ({3.2927*\dx},{0.0313*\dy})
	-- ({3.3028*\dx},{0.0331*\dy})
	-- ({3.3128*\dx},{0.0349*\dy})
	-- ({3.3228*\dx},{0.0367*\dy})
	-- ({3.3328*\dx},{0.0385*\dy})
	-- ({3.3428*\dx},{0.0404*\dy})
	-- ({3.3528*\dx},{0.0423*\dy})
	-- ({3.3628*\dx},{0.0443*\dy})
	-- ({3.3728*\dx},{0.0462*\dy})
	-- ({3.3828*\dx},{0.0482*\dy})
	-- ({3.3928*\dx},{0.0502*\dy})
	-- ({3.4028*\dx},{0.0522*\dy})
	-- ({3.4128*\dx},{0.0543*\dy})
	-- ({3.4229*\dx},{0.0563*\dy})
	-- ({3.4329*\dx},{0.0584*\dy})
	-- ({3.4429*\dx},{0.0605*\dy})
	-- ({3.4529*\dx},{0.0626*\dy})
	-- ({3.4629*\dx},{0.0647*\dy})
	-- ({3.4729*\dx},{0.0669*\dy})
	-- ({3.4829*\dx},{0.0690*\dy})
	-- ({3.4929*\dx},{0.0712*\dy})
	-- ({3.5029*\dx},{0.0734*\dy})
	-- ({3.5129*\dx},{0.0756*\dy})
	-- ({3.5229*\dx},{0.0779*\dy})
	-- ({3.5329*\dx},{0.0801*\dy})
	-- ({3.5430*\dx},{0.0824*\dy})
	-- ({3.5530*\dx},{0.0846*\dy})
	-- ({3.5630*\dx},{0.0869*\dy})
	-- ({3.5730*\dx},{0.0892*\dy})
	-- ({3.5830*\dx},{0.0915*\dy})
	-- ({3.5930*\dx},{0.0939*\dy})
	-- ({3.6030*\dx},{0.0962*\dy})
	-- ({3.6130*\dx},{0.0986*\dy})
	-- ({3.6230*\dx},{0.1009*\dy})
	-- ({3.6330*\dx},{0.1033*\dy})
	-- ({3.6430*\dx},{0.1057*\dy})
	-- ({3.6530*\dx},{0.1081*\dy})
	-- ({3.6631*\dx},{0.1105*\dy})
	-- ({3.6731*\dx},{0.1130*\dy})
	-- ({3.6831*\dx},{0.1154*\dy})
	-- ({3.6931*\dx},{0.1179*\dy})
	-- ({3.7031*\dx},{0.1204*\dy})
	-- ({3.7131*\dx},{0.1228*\dy})
	-- ({3.7231*\dx},{0.1253*\dy})
	-- ({3.7331*\dx},{0.1279*\dy})
	-- ({3.7431*\dx},{0.1304*\dy})
	-- ({3.7531*\dx},{0.1329*\dy})
	-- ({3.7631*\dx},{0.1355*\dy})
	-- ({3.7731*\dx},{0.1380*\dy})
	-- ({3.7832*\dx},{0.1406*\dy})
	-- ({3.7932*\dx},{0.1432*\dy})
	-- ({3.8032*\dx},{0.1458*\dy})
	-- ({3.8132*\dx},{0.1484*\dy})
	-- ({3.8232*\dx},{0.1511*\dy})
	-- ({3.8332*\dx},{0.1537*\dy})
	-- ({3.8432*\dx},{0.1564*\dy})
	-- ({3.8532*\dx},{0.1591*\dy})
	-- ({3.8632*\dx},{0.1618*\dy})
	-- ({3.8732*\dx},{0.1645*\dy})
	-- ({3.8832*\dx},{0.1672*\dy})
	-- ({3.8932*\dx},{0.1699*\dy})
	-- ({3.9033*\dx},{0.1727*\dy})
	-- ({3.9133*\dx},{0.1754*\dy})
	-- ({3.9233*\dx},{0.1782*\dy})
	-- ({3.9333*\dx},{0.1810*\dy})
	-- ({3.9433*\dx},{0.1838*\dy})
	-- ({3.9533*\dx},{0.1866*\dy})
	-- ({3.9633*\dx},{0.1895*\dy})
	-- ({3.9733*\dx},{0.1923*\dy})
	-- ({3.9833*\dx},{0.1952*\dy})
	-- ({3.9933*\dx},{0.1981*\dy})
	-- ({4.0033*\dx},{0.2010*\dy})
	-- ({4.0133*\dx},{0.2039*\dy})
	-- ({4.0234*\dx},{0.2068*\dy})
	-- ({4.0334*\dx},{0.2098*\dy})
	-- ({4.0434*\dx},{0.2127*\dy})
	-- ({4.0534*\dx},{0.2157*\dy})
	-- ({4.0634*\dx},{0.2187*\dy})
	-- ({4.0734*\dx},{0.2217*\dy})
	-- ({4.0834*\dx},{0.2248*\dy})
	-- ({4.0934*\dx},{0.2278*\dy})
	-- ({4.1034*\dx},{0.2309*\dy})
	-- ({4.1134*\dx},{0.2340*\dy})
	-- ({4.1234*\dx},{0.2371*\dy})
	-- ({4.1334*\dx},{0.2402*\dy})
	-- ({4.1435*\dx},{0.2434*\dy})
	-- ({4.1535*\dx},{0.2466*\dy})
	-- ({4.1635*\dx},{0.2497*\dy})
	-- ({4.1735*\dx},{0.2529*\dy})
	-- ({4.1835*\dx},{0.2562*\dy})
	-- ({4.1935*\dx},{0.2594*\dy})
	-- ({4.2035*\dx},{0.2627*\dy})
	-- ({4.2135*\dx},{0.2660*\dy})
	-- ({4.2235*\dx},{0.2693*\dy})
	-- ({4.2335*\dx},{0.2726*\dy})
	-- ({4.2435*\dx},{0.2759*\dy})
	-- ({4.2535*\dx},{0.2793*\dy})
	-- ({4.2636*\dx},{0.2827*\dy})
	-- ({4.2736*\dx},{0.2861*\dy})
	-- ({4.2836*\dx},{0.2895*\dy})
	-- ({4.2936*\dx},{0.2930*\dy})
	-- ({4.3036*\dx},{0.2965*\dy})
	-- ({4.3136*\dx},{0.3000*\dy})
	-- ({4.3236*\dx},{0.3035*\dy})
	-- ({4.3336*\dx},{0.3070*\dy})
	-- ({4.3436*\dx},{0.3106*\dy})
	-- ({4.3536*\dx},{0.3142*\dy})
	-- ({4.3636*\dx},{0.3178*\dy})
	-- ({4.3736*\dx},{0.3214*\dy})
	-- ({4.3837*\dx},{0.3251*\dy})
	-- ({4.3937*\dx},{0.3288*\dy})
	-- ({4.4037*\dx},{0.3325*\dy})
	-- ({4.4137*\dx},{0.3362*\dy})
	-- ({4.4237*\dx},{0.3400*\dy})
	-- ({4.4337*\dx},{0.3437*\dy})
	-- ({4.4437*\dx},{0.3475*\dy})
	-- ({4.4537*\dx},{0.3514*\dy})
	-- ({4.4637*\dx},{0.3552*\dy})
	-- ({4.4737*\dx},{0.3591*\dy})
	-- ({4.4837*\dx},{0.3630*\dy})
	-- ({4.4937*\dx},{0.3669*\dy})
	-- ({4.5038*\dx},{0.3709*\dy})
	-- ({4.5138*\dx},{0.3749*\dy})
	-- ({4.5238*\dx},{0.3789*\dy})
	-- ({4.5338*\dx},{0.3829*\dy})
	-- ({4.5438*\dx},{0.3870*\dy})
	-- ({4.5538*\dx},{0.3910*\dy})
	-- ({4.5638*\dx},{0.3952*\dy})
	-- ({4.5738*\dx},{0.3993*\dy})
	-- ({4.5838*\dx},{0.4035*\dy})
	-- ({4.5938*\dx},{0.4077*\dy})
	-- ({4.6038*\dx},{0.4119*\dy})
	-- ({4.6138*\dx},{0.4161*\dy})
	-- ({4.6239*\dx},{0.4204*\dy})
	-- ({4.6339*\dx},{0.4247*\dy})
	-- ({4.6439*\dx},{0.4290*\dy})
	-- ({4.6539*\dx},{0.4334*\dy})
	-- ({4.6639*\dx},{0.4377*\dy})
	-- ({4.6739*\dx},{0.4422*\dy})
	-- ({4.6839*\dx},{0.4466*\dy})
	-- ({4.6939*\dx},{0.4511*\dy})
	-- ({4.7039*\dx},{0.4555*\dy})
	-- ({4.7139*\dx},{0.4601*\dy})
	-- ({4.7239*\dx},{0.4646*\dy})
	-- ({4.7339*\dx},{0.4692*\dy})
	-- ({4.7440*\dx},{0.4738*\dy})
	-- ({4.7540*\dx},{0.4784*\dy})
	-- ({4.7640*\dx},{0.4830*\dy})
	-- ({4.7740*\dx},{0.4877*\dy})
	-- ({4.7840*\dx},{0.4924*\dy})
	-- ({4.7940*\dx},{0.4971*\dy})
	-- ({4.8040*\dx},{0.5019*\dy})
	-- ({4.8140*\dx},{0.5067*\dy})
	-- ({4.8240*\dx},{0.5115*\dy})
	-- ({4.8340*\dx},{0.5163*\dy})
	-- ({4.8440*\dx},{0.5212*\dy})
	-- ({4.8540*\dx},{0.5261*\dy})
	-- ({4.8641*\dx},{0.5310*\dy})
	-- ({4.8741*\dx},{0.5359*\dy})
	-- ({4.8841*\dx},{0.5409*\dy})
	-- ({4.8941*\dx},{0.5459*\dy})
	-- ({4.9041*\dx},{0.5509*\dy})
	-- ({4.9141*\dx},{0.5559*\dy})
	-- ({4.9241*\dx},{0.5610*\dy})
	-- ({4.9341*\dx},{0.5660*\dy})
	-- ({4.9441*\dx},{0.5711*\dy})
	-- ({4.9541*\dx},{0.5763*\dy})
	-- ({4.9641*\dx},{0.5814*\dy})
	-- ({4.9741*\dx},{0.5866*\dy})
	-- ({4.9842*\dx},{0.5918*\dy})
	-- ({4.9942*\dx},{0.5970*\dy})
	-- ({5.0042*\dx},{0.6022*\dy})
	-- ({5.0142*\dx},{0.6074*\dy})
	-- ({5.0242*\dx},{0.6126*\dy})
	-- ({5.0342*\dx},{0.6177*\dy})
	-- ({5.0442*\dx},{0.6229*\dy})
	-- ({5.0542*\dx},{0.6280*\dy})
	-- ({5.0642*\dx},{0.6331*\dy})
	-- ({5.0742*\dx},{0.6381*\dy})
	-- ({5.0842*\dx},{0.6431*\dy})
	-- ({5.0942*\dx},{0.6481*\dy})
	-- ({5.1043*\dx},{0.6530*\dy})
	-- ({5.1143*\dx},{0.6579*\dy})
	-- ({5.1243*\dx},{0.6628*\dy})
	-- ({5.1343*\dx},{0.6676*\dy})
	-- ({5.1443*\dx},{0.6724*\dy})
	-- ({5.1543*\dx},{0.6771*\dy})
	-- ({5.1643*\dx},{0.6817*\dy})
	-- ({5.1743*\dx},{0.6864*\dy})
	-- ({5.1843*\dx},{0.6909*\dy})
	-- ({5.1943*\dx},{0.6954*\dy})
	-- ({5.2043*\dx},{0.6998*\dy})
	-- ({5.2143*\dx},{0.7042*\dy})
	-- ({5.2244*\dx},{0.7085*\dy})
	-- ({5.2344*\dx},{0.7128*\dy})
	-- ({5.2444*\dx},{0.7170*\dy})
	-- ({5.2544*\dx},{0.7211*\dy})
	-- ({5.2644*\dx},{0.7251*\dy})
	-- ({5.2744*\dx},{0.7291*\dy})
	-- ({5.2844*\dx},{0.7329*\dy})
	-- ({5.2944*\dx},{0.7367*\dy})
	-- ({5.3044*\dx},{0.7404*\dy})
	-- ({5.3144*\dx},{0.7441*\dy})
	-- ({5.3244*\dx},{0.7476*\dy})
	-- ({5.3344*\dx},{0.7511*\dy})
	-- ({5.3445*\dx},{0.7545*\dy})
	-- ({5.3545*\dx},{0.7577*\dy})
	-- ({5.3645*\dx},{0.7609*\dy})
	-- ({5.3745*\dx},{0.7640*\dy})
	-- ({5.3845*\dx},{0.7670*\dy})
	-- ({5.3945*\dx},{0.7699*\dy})
	-- ({5.4045*\dx},{0.7727*\dy})
	-- ({5.4145*\dx},{0.7754*\dy})
	-- ({5.4245*\dx},{0.7780*\dy})
	-- ({5.4345*\dx},{0.7805*\dy})
	-- ({5.4445*\dx},{0.7829*\dy})
	-- ({5.4545*\dx},{0.7851*\dy})
	-- ({5.4646*\dx},{0.7873*\dy})
	-- ({5.4746*\dx},{0.7894*\dy})
	-- ({5.4846*\dx},{0.7913*\dy})
	-- ({5.4946*\dx},{0.7931*\dy})
	-- ({5.5046*\dx},{0.7949*\dy})
	-- ({5.5146*\dx},{0.7965*\dy})
	-- ({5.5246*\dx},{0.7980*\dy})
	-- ({5.5346*\dx},{0.7993*\dy})
	-- ({5.5446*\dx},{0.8006*\dy})
	-- ({5.5546*\dx},{0.8018*\dy})
	-- ({5.5646*\dx},{0.8028*\dy})
	-- ({5.5746*\dx},{0.8037*\dy})
	-- ({5.5847*\dx},{0.8045*\dy})
	-- ({5.5947*\dx},{0.8052*\dy})
	-- ({5.6047*\dx},{0.8058*\dy})
	-- ({5.6147*\dx},{0.8062*\dy})
	-- ({5.6247*\dx},{0.8065*\dy})
	-- ({5.6347*\dx},{0.8068*\dy})
	-- ({5.6447*\dx},{0.8069*\dy})
	-- ({5.6547*\dx},{0.8068*\dy})
	-- ({5.6647*\dx},{0.8067*\dy})
	-- ({5.6747*\dx},{0.8065*\dy})
	-- ({5.6847*\dx},{0.8061*\dy})
	-- ({5.6947*\dx},{0.8056*\dy})
	-- ({5.7048*\dx},{0.8051*\dy})
	-- ({5.7148*\dx},{0.8044*\dy})
	-- ({5.7248*\dx},{0.8035*\dy})
	-- ({5.7348*\dx},{0.8026*\dy})
	-- ({5.7448*\dx},{0.8016*\dy})
	-- ({5.7548*\dx},{0.8005*\dy})
	-- ({5.7648*\dx},{0.7992*\dy})
	-- ({5.7748*\dx},{0.7979*\dy})
	-- ({5.7848*\dx},{0.7964*\dy})
	-- ({5.7948*\dx},{0.7949*\dy})
	-- ({5.8048*\dx},{0.7932*\dy})
	-- ({5.8148*\dx},{0.7915*\dy})
	-- ({5.8249*\dx},{0.7896*\dy})
	-- ({5.8349*\dx},{0.7877*\dy})
	-- ({5.8449*\dx},{0.7856*\dy})
	-- ({5.8549*\dx},{0.7835*\dy})
	-- ({5.8649*\dx},{0.7813*\dy})
	-- ({5.8749*\dx},{0.7789*\dy})
	-- ({5.8849*\dx},{0.7765*\dy})
	-- ({5.8949*\dx},{0.7741*\dy})
	-- ({5.9049*\dx},{0.7715*\dy})
	-- ({5.9149*\dx},{0.7688*\dy})
	-- ({5.9249*\dx},{0.7661*\dy})
	-- ({5.9349*\dx},{0.7633*\dy})
	-- ({5.9450*\dx},{0.7604*\dy})
	-- ({5.9550*\dx},{0.7574*\dy})
	-- ({5.9650*\dx},{0.7544*\dy})
	-- ({5.9750*\dx},{0.7513*\dy})
	-- ({5.9850*\dx},{0.7481*\dy})
	-- ({5.9950*\dx},{0.7449*\dy})
	-- ({6.0050*\dx},{0.7416*\dy})
	-- ({6.0150*\dx},{0.7383*\dy})
	-- ({6.0250*\dx},{0.7350*\dy})
	-- ({6.0350*\dx},{0.7317*\dy})
	-- ({6.0450*\dx},{0.7284*\dy})
	-- ({6.0550*\dx},{0.7251*\dy})
	-- ({6.0651*\dx},{0.7219*\dy})
	-- ({6.0751*\dx},{0.7186*\dy})
	-- ({6.0851*\dx},{0.7154*\dy})
	-- ({6.0951*\dx},{0.7121*\dy})
	-- ({6.1051*\dx},{0.7089*\dy})
	-- ({6.1151*\dx},{0.7056*\dy})
	-- ({6.1251*\dx},{0.7024*\dy})
	-- ({6.1351*\dx},{0.6992*\dy})
	-- ({6.1451*\dx},{0.6960*\dy})
	-- ({6.1551*\dx},{0.6928*\dy})
	-- ({6.1651*\dx},{0.6896*\dy})
	-- ({6.1751*\dx},{0.6864*\dy})
	-- ({6.1852*\dx},{0.6832*\dy})
	-- ({6.1952*\dx},{0.6801*\dy})
	-- ({6.2052*\dx},{0.6769*\dy})
	-- ({6.2152*\dx},{0.6737*\dy})
	-- ({6.2252*\dx},{0.6706*\dy})
	-- ({6.2352*\dx},{0.6675*\dy})
	-- ({6.2452*\dx},{0.6643*\dy})
	-- ({6.2552*\dx},{0.6612*\dy})
	-- ({6.2652*\dx},{0.6581*\dy})
	-- ({6.2752*\dx},{0.6550*\dy})
	-- ({6.2852*\dx},{0.6519*\dy})
	-- ({6.2952*\dx},{0.6488*\dy})
	-- ({6.3053*\dx},{0.6458*\dy})
	-- ({6.3153*\dx},{0.6427*\dy})
	-- ({6.3253*\dx},{0.6396*\dy})
	-- ({6.3353*\dx},{0.6366*\dy})
	-- ({6.3453*\dx},{0.6335*\dy})
	-- ({6.3553*\dx},{0.6305*\dy})
	-- ({6.3653*\dx},{0.6275*\dy})
	-- ({6.3753*\dx},{0.6245*\dy})
	-- ({6.3853*\dx},{0.6215*\dy})
	-- ({6.3953*\dx},{0.6185*\dy})
	-- ({6.4053*\dx},{0.6155*\dy})
	-- ({6.4153*\dx},{0.6125*\dy})
	-- ({6.4254*\dx},{0.6096*\dy})
	-- ({6.4354*\dx},{0.6066*\dy})
	-- ({6.4454*\dx},{0.6037*\dy})
	-- ({6.4554*\dx},{0.6007*\dy})
	-- ({6.4654*\dx},{0.5978*\dy})
	-- ({6.4754*\dx},{0.5949*\dy})
	-- ({6.4854*\dx},{0.5919*\dy})
	-- ({6.4954*\dx},{0.5890*\dy})
	-- ({6.5054*\dx},{0.5862*\dy})
	-- ({6.5154*\dx},{0.5833*\dy})
	-- ({6.5254*\dx},{0.5804*\dy})
	-- ({6.5354*\dx},{0.5775*\dy})
	-- ({6.5455*\dx},{0.5747*\dy})
	-- ({6.5555*\dx},{0.5718*\dy})
	-- ({6.5655*\dx},{0.5690*\dy})
	-- ({6.5755*\dx},{0.5661*\dy})
	-- ({6.5855*\dx},{0.5633*\dy})
	-- ({6.5955*\dx},{0.5605*\dy})
	-- ({6.6055*\dx},{0.5577*\dy})
	-- ({6.6155*\dx},{0.5549*\dy})
	-- ({6.6255*\dx},{0.5521*\dy})
	-- ({6.6355*\dx},{0.5493*\dy})
	-- ({6.6455*\dx},{0.5466*\dy})
	-- ({6.6555*\dx},{0.5438*\dy})
	-- ({6.6656*\dx},{0.5410*\dy})
	-- ({6.6756*\dx},{0.5383*\dy})
	-- ({6.6856*\dx},{0.5355*\dy})
	-- ({6.6956*\dx},{0.5328*\dy})
	-- ({6.7056*\dx},{0.5301*\dy})
	-- ({6.7156*\dx},{0.5274*\dy})
	-- ({6.7256*\dx},{0.5247*\dy})
	-- ({6.7356*\dx},{0.5220*\dy})
	-- ({6.7456*\dx},{0.5193*\dy})
	-- ({6.7556*\dx},{0.5166*\dy})
	-- ({6.7656*\dx},{0.5139*\dy})
	-- ({6.7756*\dx},{0.5113*\dy})
	-- ({6.7857*\dx},{0.5086*\dy})
	-- ({6.7957*\dx},{0.5059*\dy})
	-- ({6.8057*\dx},{0.5033*\dy})
	-- ({6.8157*\dx},{0.5007*\dy})
	-- ({6.8257*\dx},{0.4980*\dy})
	-- ({6.8357*\dx},{0.4954*\dy})
	-- ({6.8457*\dx},{0.4928*\dy})
	-- ({6.8557*\dx},{0.4902*\dy})
	-- ({6.8657*\dx},{0.4876*\dy})
	-- ({6.8757*\dx},{0.4850*\dy})
	-- ({6.8857*\dx},{0.4824*\dy})
	-- ({6.8957*\dx},{0.4798*\dy})
	-- ({6.9058*\dx},{0.4772*\dy})
	-- ({6.9158*\dx},{0.4747*\dy})
	-- ({6.9258*\dx},{0.4721*\dy})
	-- ({6.9358*\dx},{0.4696*\dy})
	-- ({6.9458*\dx},{0.4670*\dy})
	-- ({6.9558*\dx},{0.4645*\dy})
	-- ({6.9658*\dx},{0.4619*\dy})
	-- ({6.9758*\dx},{0.4594*\dy})
	-- ({6.9858*\dx},{0.4569*\dy})
	-- ({6.9958*\dx},{0.4544*\dy})
	-- ({7.0058*\dx},{0.4518*\dy})
	-- ({7.0158*\dx},{0.4493*\dy})
	-- ({7.0259*\dx},{0.4468*\dy})
	-- ({7.0359*\dx},{0.4443*\dy})
	-- ({7.0459*\dx},{0.4419*\dy})
	-- ({7.0559*\dx},{0.4394*\dy})
	-- ({7.0659*\dx},{0.4369*\dy})
	-- ({7.0759*\dx},{0.4344*\dy})
	-- ({7.0859*\dx},{0.4320*\dy})
	-- ({7.0959*\dx},{0.4295*\dy})
	-- ({7.1059*\dx},{0.4270*\dy})
	-- ({7.1159*\dx},{0.4246*\dy})
	-- ({7.1259*\dx},{0.4221*\dy})
	-- ({7.1359*\dx},{0.4197*\dy})
	-- ({7.1460*\dx},{0.4173*\dy})
	-- ({7.1560*\dx},{0.4148*\dy})
	-- ({7.1660*\dx},{0.4124*\dy})
	-- ({7.1760*\dx},{0.4100*\dy})
	-- ({7.1860*\dx},{0.4075*\dy})
	-- ({7.1960*\dx},{0.4051*\dy})
	-- ({7.2060*\dx},{0.4027*\dy})
	-- ({7.2160*\dx},{0.4003*\dy})
	-- ({7.2260*\dx},{0.3979*\dy})
	-- ({7.2360*\dx},{0.3955*\dy})
	-- ({7.2460*\dx},{0.3931*\dy})
	-- ({7.2560*\dx},{0.3907*\dy})
	-- ({7.2661*\dx},{0.3883*\dy})
	-- ({7.2761*\dx},{0.3860*\dy})
	-- ({7.2861*\dx},{0.3836*\dy})
	-- ({7.2961*\dx},{0.3812*\dy})
	-- ({7.3061*\dx},{0.3788*\dy})
	-- ({7.3161*\dx},{0.3765*\dy})
	-- ({7.3261*\dx},{0.3741*\dy})
	-- ({7.3361*\dx},{0.3717*\dy})
	-- ({7.3461*\dx},{0.3694*\dy})
	-- ({7.3561*\dx},{0.3670*\dy})
	-- ({7.3661*\dx},{0.3646*\dy})
	-- ({7.3761*\dx},{0.3623*\dy})
	-- ({7.3862*\dx},{0.3599*\dy})
	-- ({7.3962*\dx},{0.3576*\dy})
	-- ({7.4062*\dx},{0.3552*\dy})
	-- ({7.4162*\dx},{0.3529*\dy})
	-- ({7.4262*\dx},{0.3506*\dy})
	-- ({7.4362*\dx},{0.3482*\dy})
	-- ({7.4462*\dx},{0.3459*\dy})
	-- ({7.4562*\dx},{0.3435*\dy})
	-- ({7.4662*\dx},{0.3412*\dy})
	-- ({7.4762*\dx},{0.3389*\dy})
	-- ({7.4862*\dx},{0.3365*\dy})
	-- ({7.4962*\dx},{0.3342*\dy})
	-- ({7.5063*\dx},{0.3319*\dy})
	-- ({7.5163*\dx},{0.3296*\dy})
	-- ({7.5263*\dx},{0.3272*\dy})
	-- ({7.5363*\dx},{0.3249*\dy})
	-- ({7.5463*\dx},{0.3226*\dy})
	-- ({7.5563*\dx},{0.3203*\dy})
	-- ({7.5663*\dx},{0.3179*\dy})
	-- ({7.5763*\dx},{0.3156*\dy})
	-- ({7.5863*\dx},{0.3133*\dy})
	-- ({7.5963*\dx},{0.3110*\dy})
	-- ({7.6063*\dx},{0.3086*\dy})
	-- ({7.6163*\dx},{0.3063*\dy})
	-- ({7.6264*\dx},{0.3040*\dy})
	-- ({7.6364*\dx},{0.3017*\dy})
	-- ({7.6464*\dx},{0.2994*\dy})
	-- ({7.6564*\dx},{0.2970*\dy})
	-- ({7.6664*\dx},{0.2947*\dy})
	-- ({7.6764*\dx},{0.2924*\dy})
	-- ({7.6864*\dx},{0.2901*\dy})
	-- ({7.6964*\dx},{0.2878*\dy})
	-- ({7.7064*\dx},{0.2854*\dy})
	-- ({7.7164*\dx},{0.2831*\dy})
	-- ({7.7264*\dx},{0.2808*\dy})
	-- ({7.7364*\dx},{0.2785*\dy})
	-- ({7.7465*\dx},{0.2761*\dy})
	-- ({7.7565*\dx},{0.2738*\dy})
	-- ({7.7665*\dx},{0.2715*\dy})
	-- ({7.7765*\dx},{0.2691*\dy})
	-- ({7.7865*\dx},{0.2668*\dy})
	-- ({7.7965*\dx},{0.2645*\dy})
	-- ({7.8065*\dx},{0.2621*\dy})
	-- ({7.8165*\dx},{0.2598*\dy})
	-- ({7.8265*\dx},{0.2575*\dy})
	-- ({7.8365*\dx},{0.2551*\dy})
	-- ({7.8465*\dx},{0.2528*\dy})
	-- ({7.8565*\dx},{0.2504*\dy})
	-- ({7.8666*\dx},{0.2481*\dy})
	-- ({7.8766*\dx},{0.2457*\dy})
	-- ({7.8866*\dx},{0.2434*\dy})
	-- ({7.8966*\dx},{0.2410*\dy})
	-- ({7.9066*\dx},{0.2387*\dy})
	-- ({7.9166*\dx},{0.2363*\dy})
	-- ({7.9266*\dx},{0.2340*\dy})
	-- ({7.9366*\dx},{0.2316*\dy})
	-- ({7.9466*\dx},{0.2292*\dy})
	-- ({7.9566*\dx},{0.2269*\dy})
	-- ({7.9666*\dx},{0.2245*\dy})
	-- ({7.9766*\dx},{0.2221*\dy})
	-- ({7.9867*\dx},{0.2197*\dy})
	-- ({7.9967*\dx},{0.2173*\dy})
	-- ({8.0067*\dx},{0.2149*\dy})
	-- ({8.0167*\dx},{0.2125*\dy})
	-- ({8.0267*\dx},{0.2100*\dy})
	-- ({8.0367*\dx},{0.2075*\dy})
	-- ({8.0467*\dx},{0.2049*\dy})
	-- ({8.0567*\dx},{0.2023*\dy})
	-- ({8.0667*\dx},{0.1997*\dy})
	-- ({8.0767*\dx},{0.1970*\dy})
	-- ({8.0867*\dx},{0.1942*\dy})
	-- ({8.0967*\dx},{0.1915*\dy})
	-- ({8.1068*\dx},{0.1887*\dy})
	-- ({8.1168*\dx},{0.1858*\dy})
	-- ({8.1268*\dx},{0.1830*\dy})
	-- ({8.1368*\dx},{0.1801*\dy})
	-- ({8.1468*\dx},{0.1771*\dy})
	-- ({8.1568*\dx},{0.1741*\dy})
	-- ({8.1668*\dx},{0.1711*\dy})
	-- ({8.1768*\dx},{0.1681*\dy})
	-- ({8.1868*\dx},{0.1651*\dy})
	-- ({8.1968*\dx},{0.1620*\dy})
	-- ({8.2068*\dx},{0.1589*\dy})
	-- ({8.2168*\dx},{0.1558*\dy})
	-- ({8.2269*\dx},{0.1527*\dy})
	-- ({8.2369*\dx},{0.1495*\dy})
	-- ({8.2469*\dx},{0.1464*\dy})
	-- ({8.2569*\dx},{0.1432*\dy})
	-- ({8.2669*\dx},{0.1400*\dy})
	-- ({8.2769*\dx},{0.1368*\dy})
	-- ({8.2869*\dx},{0.1337*\dy})
	-- ({8.2969*\dx},{0.1305*\dy})
	-- ({8.3069*\dx},{0.1273*\dy})
	-- ({8.3169*\dx},{0.1241*\dy})
	-- ({8.3269*\dx},{0.1209*\dy})
	-- ({8.3369*\dx},{0.1177*\dy})
	-- ({8.3470*\dx},{0.1146*\dy})
	-- ({8.3570*\dx},{0.1114*\dy})
	-- ({8.3670*\dx},{0.1083*\dy})
	-- ({8.3770*\dx},{0.1052*\dy})
	-- ({8.3870*\dx},{0.1021*\dy})
	-- ({8.3970*\dx},{0.0990*\dy})
	-- ({8.4070*\dx},{0.0959*\dy})
	-- ({8.4170*\dx},{0.0929*\dy})
	-- ({8.4270*\dx},{0.0899*\dy})
	-- ({8.4370*\dx},{0.0869*\dy})
	-- ({8.4470*\dx},{0.0840*\dy})
	-- ({8.4570*\dx},{0.0810*\dy})
	-- ({8.4671*\dx},{0.0782*\dy})
	-- ({8.4771*\dx},{0.0753*\dy})
	-- ({8.4871*\dx},{0.0725*\dy})
	-- ({8.4971*\dx},{0.0698*\dy})
	-- ({8.5071*\dx},{0.0670*\dy})
	-- ({8.5171*\dx},{0.0644*\dy})
	-- ({8.5271*\dx},{0.0617*\dy})
	-- ({8.5371*\dx},{0.0591*\dy})
	-- ({8.5471*\dx},{0.0566*\dy})
	-- ({8.5571*\dx},{0.0541*\dy})
	-- ({8.5671*\dx},{0.0517*\dy})
	-- ({8.5771*\dx},{0.0493*\dy})
	-- ({8.5872*\dx},{0.0469*\dy})
	-- ({8.5972*\dx},{0.0446*\dy})
	-- ({8.6072*\dx},{0.0424*\dy})
	-- ({8.6172*\dx},{0.0402*\dy})
	-- ({8.6272*\dx},{0.0381*\dy})
	-- ({8.6372*\dx},{0.0360*\dy})
	-- ({8.6472*\dx},{0.0340*\dy})
	-- ({8.6572*\dx},{0.0321*\dy})
	-- ({8.6672*\dx},{0.0302*\dy})
	-- ({8.6772*\dx},{0.0283*\dy})
	-- ({8.6872*\dx},{0.0266*\dy})
	-- ({8.6972*\dx},{0.0248*\dy})
	-- ({8.7073*\dx},{0.0232*\dy})
	-- ({8.7173*\dx},{0.0216*\dy})
	-- ({8.7273*\dx},{0.0200*\dy})
	-- ({8.7373*\dx},{0.0186*\dy})
	-- ({8.7473*\dx},{0.0171*\dy})
	-- ({8.7573*\dx},{0.0158*\dy})
	-- ({8.7673*\dx},{0.0145*\dy})
	-- ({8.7773*\dx},{0.0132*\dy})
	-- ({8.7873*\dx},{0.0120*\dy})
	-- ({8.7973*\dx},{0.0109*\dy})
	-- ({8.8073*\dx},{0.0098*\dy})
	-- ({8.8173*\dx},{0.0088*\dy})
	-- ({8.8274*\dx},{0.0078*\dy})
	-- ({8.8374*\dx},{0.0069*\dy})
	-- ({8.8474*\dx},{0.0061*\dy})
	-- ({8.8574*\dx},{0.0053*\dy})
	-- ({8.8674*\dx},{0.0046*\dy})
	-- ({8.8774*\dx},{0.0039*\dy})
	-- ({8.8874*\dx},{0.0033*\dy})
	-- ({8.8974*\dx},{0.0027*\dy})
	-- ({8.9074*\dx},{0.0022*\dy})
	-- ({8.9174*\dx},{0.0017*\dy})
	-- ({8.9274*\dx},{0.0013*\dy})
	-- ({8.9374*\dx},{0.0010*\dy})
	-- ({8.9475*\dx},{0.0007*\dy})
	-- ({8.9575*\dx},{0.0005*\dy})
	-- ({8.9675*\dx},{0.0003*\dy})
	-- ({8.9775*\dx},{0.0001*\dy})
	-- ({8.9875*\dx},{0.0000*\dy})
	-- ({8.9975*\dx},{0.0000*\dy})
	-- ({9.0075*\dx},{0.0000*\dy})
	-- ({9.0175*\dx},{0.0000*\dy})
	-- ({9.0275*\dx},{0.0000*\dy})
	-- ({9.0375*\dx},{0.0000*\dy})
	-- ({9.0475*\dx},{0.0000*\dy})
	-- ({9.0575*\dx},{0.0000*\dy})
	-- ({9.0676*\dx},{0.0000*\dy})
	-- ({9.0776*\dx},{0.0000*\dy})
	-- ({9.0876*\dx},{0.0000*\dy})
	-- ({9.0976*\dx},{0.0000*\dy})
	-- ({9.1076*\dx},{0.0000*\dy})
	-- ({9.1176*\dx},{0.0000*\dy})
	-- ({9.1276*\dx},{0.0000*\dy})
	-- ({9.1376*\dx},{0.0000*\dy})
	-- ({9.1476*\dx},{0.0000*\dy})
	-- ({9.1576*\dx},{0.0000*\dy})
	-- ({9.1676*\dx},{0.0000*\dy})
	-- ({9.1776*\dx},{0.0000*\dy})
	-- ({9.1877*\dx},{0.0000*\dy})
	-- ({9.1977*\dx},{0.0000*\dy})
	-- ({9.2077*\dx},{0.0000*\dy})
	-- ({9.2177*\dx},{0.0000*\dy})
	-- ({9.2277*\dx},{0.0000*\dy})
	-- ({9.2377*\dx},{0.0000*\dy})
	-- ({9.2477*\dx},{0.0000*\dy})
	-- ({9.2577*\dx},{0.0000*\dy})
	-- ({9.2677*\dx},{0.0000*\dy})
	-- ({9.2777*\dx},{0.0000*\dy})
	-- ({9.2877*\dx},{0.0000*\dy})
	-- ({9.2977*\dx},{0.0000*\dy})
	-- ({9.3078*\dx},{0.0000*\dy})
	-- ({9.3178*\dx},{0.0000*\dy})
	-- ({9.3278*\dx},{0.0000*\dy})
	-- ({9.3378*\dx},{0.0000*\dy})
	-- ({9.3478*\dx},{0.0000*\dy})
	-- ({9.3578*\dx},{0.0000*\dy})
	-- ({9.3678*\dx},{0.0000*\dy})
	-- ({9.3778*\dx},{0.0000*\dy})
	-- ({9.3878*\dx},{0.0000*\dy})
	-- ({9.3978*\dx},{0.0000*\dy})
	-- ({9.4078*\dx},{0.0000*\dy})
	-- ({9.4178*\dx},{0.0000*\dy})
	-- ({9.4279*\dx},{0.0000*\dy})
	-- ({9.4379*\dx},{0.0000*\dy})
	-- ({9.4479*\dx},{0.0000*\dy})
	-- ({9.4579*\dx},{0.0000*\dy})
	-- ({9.4679*\dx},{0.0000*\dy})
	-- ({9.4779*\dx},{0.0000*\dy})
	-- ({9.4879*\dx},{0.0000*\dy})
	-- ({9.4979*\dx},{0.0000*\dy})
	-- ({9.5079*\dx},{0.0000*\dy})
	-- ({9.5179*\dx},{0.0000*\dy})
	-- ({9.5279*\dx},{0.0000*\dy})
	-- ({9.5379*\dx},{0.0000*\dy})
	-- ({9.5480*\dx},{0.0000*\dy})
	-- ({9.5580*\dx},{0.0000*\dy})
	-- ({9.5680*\dx},{0.0000*\dy})
	-- ({9.5780*\dx},{0.0000*\dy})
	-- ({9.5880*\dx},{0.0000*\dy})
	-- ({9.5980*\dx},{0.0000*\dy})
	-- ({9.6080*\dx},{0.0000*\dy})
	-- ({9.6180*\dx},{0.0000*\dy})
	-- ({9.6280*\dx},{0.0000*\dy})
	-- ({9.6380*\dx},{0.0000*\dy})
	-- ({9.6480*\dx},{0.0000*\dy})
	-- ({9.6580*\dx},{0.0000*\dy})
	-- ({9.6681*\dx},{0.0000*\dy})
	-- ({9.6781*\dx},{0.0000*\dy})
	-- ({9.6881*\dx},{0.0000*\dy})
	-- ({9.6981*\dx},{0.0000*\dy})
	-- ({9.7081*\dx},{0.0000*\dy})
	-- ({9.7181*\dx},{0.0000*\dy})
	-- ({9.7281*\dx},{0.0000*\dy})
	-- ({9.7381*\dx},{0.0000*\dy})
	-- ({9.7481*\dx},{0.0000*\dy})
	-- ({9.7581*\dx},{0.0000*\dy})
	-- ({9.7681*\dx},{0.0000*\dy})
	-- ({9.7781*\dx},{0.0000*\dy})
	-- ({9.7882*\dx},{0.0000*\dy})
	-- ({9.7982*\dx},{0.0000*\dy})
	-- ({9.8082*\dx},{0.0000*\dy})
	-- ({9.8182*\dx},{0.0000*\dy})
	-- ({9.8282*\dx},{0.0000*\dy})
	-- ({9.8382*\dx},{0.0000*\dy})
	-- ({9.8482*\dx},{0.0000*\dy})
	-- ({9.8582*\dx},{0.0000*\dy})
	-- ({9.8682*\dx},{0.0000*\dy})
	-- ({9.8782*\dx},{0.0000*\dy})
	-- ({9.8882*\dx},{0.0000*\dy})
	-- ({9.8982*\dx},{0.0000*\dy})
	-- ({9.9083*\dx},{0.0000*\dy})
	-- ({9.9183*\dx},{0.0000*\dy})
	-- ({9.9283*\dx},{0.0000*\dy})
	-- ({9.9383*\dx},{0.0000*\dy})
	-- ({9.9483*\dx},{0.0000*\dy})
	-- ({9.9583*\dx},{0.0000*\dy})
	-- ({9.9683*\dx},{0.0000*\dy})
	-- ({9.9783*\dx},{0.0000*\dy})
	-- ({9.9883*\dx},{0.0000*\dy})
	-- ({9.9983*\dx},{0.0000*\dy})
	-- ({10.0083*\dx},{0.0000*\dy})
	-- ({10.0183*\dx},{0.0000*\dy})
	-- ({10.0284*\dx},{0.0000*\dy})
	-- ({10.0384*\dx},{0.0000*\dy})
	-- ({10.0484*\dx},{0.0000*\dy})
	-- ({10.0584*\dx},{0.0000*\dy})
	-- ({10.0684*\dx},{0.0000*\dy})
	-- ({10.0784*\dx},{0.0000*\dy})
	-- ({10.0884*\dx},{0.0000*\dy})
	-- ({10.0984*\dx},{0.0000*\dy})
	-- ({10.1084*\dx},{0.0000*\dy})
	-- ({10.1184*\dx},{0.0000*\dy})
	-- ({10.1284*\dx},{0.0000*\dy})
	-- ({10.1384*\dx},{0.0000*\dy})
	-- ({10.1485*\dx},{0.0000*\dy})
	-- ({10.1585*\dx},{0.0000*\dy})
	-- ({10.1685*\dx},{0.0000*\dy})
	-- ({10.1785*\dx},{0.0000*\dy})
	-- ({10.1885*\dx},{0.0000*\dy})
	-- ({10.1985*\dx},{0.0000*\dy})
	-- ({10.2085*\dx},{0.0000*\dy})
	-- ({10.2185*\dx},{0.0000*\dy})
	-- ({10.2285*\dx},{0.0000*\dy})
	-- ({10.2385*\dx},{0.0000*\dy})
	-- ({10.2485*\dx},{0.0000*\dy})
	-- ({10.2585*\dx},{0.0000*\dy})
	-- ({10.2686*\dx},{0.0000*\dy})
	-- ({10.2786*\dx},{0.0000*\dy})
	-- ({10.2886*\dx},{0.0000*\dy})
	-- ({10.2986*\dx},{0.0000*\dy})
	-- ({10.3086*\dx},{0.0000*\dy})
	-- ({10.3186*\dx},{0.0000*\dy})
	-- ({10.3286*\dx},{0.0000*\dy})
	-- ({10.3386*\dx},{0.0000*\dy})
	-- ({10.3486*\dx},{0.0000*\dy})
	-- ({10.3586*\dx},{0.0000*\dy})
	-- ({10.3686*\dx},{0.0000*\dy})
	-- ({10.3786*\dx},{0.0000*\dy})
	-- ({10.3887*\dx},{0.0000*\dy})
	-- ({10.3987*\dx},{0.0000*\dy})
	-- ({10.4087*\dx},{0.0000*\dy})
	-- ({10.4187*\dx},{0.0000*\dy})
	-- ({10.4287*\dx},{0.0000*\dy})
	-- ({10.4387*\dx},{0.0000*\dy})
	-- ({10.4487*\dx},{0.0000*\dy})
	-- ({10.4587*\dx},{0.0000*\dy})
	-- ({10.4687*\dx},{0.0000*\dy})
	-- ({10.4787*\dx},{0.0000*\dy})
	-- ({10.4887*\dx},{0.0000*\dy})
	-- ({10.4987*\dx},{0.0000*\dy})
	-- ({10.5088*\dx},{0.0000*\dy})
	-- ({10.5188*\dx},{0.0000*\dy})
	-- ({10.5288*\dx},{0.0000*\dy})
	-- ({10.5388*\dx},{0.0000*\dy})
	-- ({10.5488*\dx},{0.0000*\dy})
	-- ({10.5588*\dx},{0.0000*\dy})
	-- ({10.5688*\dx},{0.0000*\dy})
	-- ({10.5788*\dx},{0.0000*\dy})
	-- ({10.5888*\dx},{0.0000*\dy})
	-- ({10.5988*\dx},{0.0000*\dy})
	-- ({10.6088*\dx},{0.0000*\dy})
	-- ({10.6188*\dx},{0.0000*\dy})
	-- ({10.6289*\dx},{0.0000*\dy})
	-- ({10.6389*\dx},{0.0000*\dy})
	-- ({10.6489*\dx},{0.0000*\dy})
	-- ({10.6589*\dx},{0.0000*\dy})
	-- ({10.6689*\dx},{0.0000*\dy})
	-- ({10.6789*\dx},{0.0000*\dy})
	-- ({10.6889*\dx},{0.0000*\dy})
	-- ({10.6989*\dx},{0.0000*\dy})
	-- ({10.7089*\dx},{0.0000*\dy})
	-- ({10.7189*\dx},{0.0000*\dy})
	-- ({10.7289*\dx},{0.0000*\dy})
	-- ({10.7389*\dx},{0.0000*\dy})
	-- ({10.7490*\dx},{0.0000*\dy})
	-- ({10.7590*\dx},{0.0000*\dy})
	-- ({10.7690*\dx},{0.0000*\dy})
	-- ({10.7790*\dx},{0.0000*\dy})
	-- ({10.7890*\dx},{0.0000*\dy})
	-- ({10.7990*\dx},{0.0000*\dy})
	-- ({10.8090*\dx},{0.0000*\dy})
	-- ({10.8190*\dx},{0.0000*\dy})
	-- ({10.8290*\dx},{0.0000*\dy})
	-- ({10.8390*\dx},{0.0000*\dy})
	-- ({10.8490*\dx},{0.0000*\dy})
	-- ({10.8590*\dx},{0.0000*\dy})
	-- ({10.8691*\dx},{0.0000*\dy})
	-- ({10.8791*\dx},{0.0000*\dy})
	-- ({10.8891*\dx},{0.0000*\dy})
	-- ({10.8991*\dx},{0.0000*\dy})
	-- ({10.9091*\dx},{0.0000*\dy})
	-- ({10.9191*\dx},{0.0000*\dy})
	-- ({10.9291*\dx},{0.0000*\dy})
	-- ({10.9391*\dx},{0.0000*\dy})
	-- ({10.9491*\dx},{0.0000*\dy})
	-- ({10.9591*\dx},{0.0000*\dy})
	-- ({10.9691*\dx},{0.0000*\dy})
	-- ({10.9791*\dx},{0.0000*\dy})
	-- ({10.9892*\dx},{0.0000*\dy})
	-- ({10.9992*\dx},{0.0000*\dy})
	-- ({11.0092*\dx},{0.0000*\dy})
	-- ({11.0192*\dx},{0.0000*\dy})
	-- ({11.0292*\dx},{0.0000*\dy})
	-- ({11.0392*\dx},{0.0000*\dy})
	-- ({11.0492*\dx},{0.0000*\dy})
	-- ({11.0592*\dx},{0.0000*\dy})
	-- ({11.0692*\dx},{0.0000*\dy})
	-- ({11.0792*\dx},{0.0000*\dy})
	-- ({11.0892*\dx},{0.0000*\dy})
	-- ({11.0992*\dx},{0.0000*\dy})
	-- ({11.1093*\dx},{0.0000*\dy})
	-- ({11.1193*\dx},{0.0000*\dy})
	-- ({11.1293*\dx},{0.0000*\dy})
	-- ({11.1393*\dx},{0.0000*\dy})
	-- ({11.1493*\dx},{0.0000*\dy})
	-- ({11.1593*\dx},{0.0000*\dy})
	-- ({11.1693*\dx},{0.0000*\dy})
	-- ({11.1793*\dx},{0.0000*\dy})
	-- ({11.1893*\dx},{0.0000*\dy})
	-- ({11.1993*\dx},{0.0000*\dy})
	-- ({11.2093*\dx},{0.0000*\dy})
	-- ({11.2193*\dx},{0.0000*\dy})
	-- ({11.2294*\dx},{0.0000*\dy})
	-- ({11.2394*\dx},{0.0000*\dy})
	-- ({11.2494*\dx},{0.0000*\dy})
	-- ({11.2594*\dx},{0.0000*\dy})
	-- ({11.2694*\dx},{0.0000*\dy})
	-- ({11.2794*\dx},{0.0000*\dy})
	-- ({11.2894*\dx},{0.0000*\dy})
	-- ({11.2994*\dx},{0.0000*\dy})
	-- ({11.3094*\dx},{0.0000*\dy})
	-- ({11.3194*\dx},{0.0000*\dy})
	-- ({11.3294*\dx},{0.0000*\dy})
	-- ({11.3394*\dx},{0.0000*\dy})
	-- ({11.3495*\dx},{0.0000*\dy})
	-- ({11.3595*\dx},{0.0000*\dy})
	-- ({11.3695*\dx},{0.0000*\dy})
	-- ({11.3795*\dx},{0.0000*\dy})
	-- ({11.3895*\dx},{0.0000*\dy})
	-- ({11.3995*\dx},{0.0000*\dy})
	-- ({11.4095*\dx},{0.0000*\dy})
	-- ({11.4195*\dx},{0.0000*\dy})
	-- ({11.4295*\dx},{0.0000*\dy})
	-- ({11.4395*\dx},{0.0000*\dy})
	-- ({11.4495*\dx},{0.0000*\dy})
	-- ({11.4595*\dx},{0.0000*\dy})
	-- ({11.4696*\dx},{0.0000*\dy})
	-- ({11.4796*\dx},{0.0000*\dy})
	-- ({11.4896*\dx},{0.0000*\dy})
	-- ({11.4996*\dx},{0.0000*\dy})
	-- ({11.5096*\dx},{0.0000*\dy})
	-- ({11.5196*\dx},{0.0000*\dy})
	-- ({11.5296*\dx},{0.0000*\dy})
	-- ({11.5396*\dx},{0.0000*\dy})
	-- ({11.5496*\dx},{0.0000*\dy})
	-- ({11.5596*\dx},{0.0000*\dy})
	-- ({11.5696*\dx},{0.0000*\dy})
	-- ({11.5796*\dx},{0.0000*\dy})
	-- ({11.5897*\dx},{0.0000*\dy})
	-- ({11.5997*\dx},{0.0000*\dy})
	-- ({11.6097*\dx},{0.0000*\dy})
	-- ({11.6197*\dx},{0.0000*\dy})
	-- ({11.6297*\dx},{0.0000*\dy})
	-- ({11.6397*\dx},{0.0000*\dy})
	-- ({11.6497*\dx},{0.0000*\dy})
	-- ({11.6597*\dx},{0.0000*\dy})
	-- ({11.6697*\dx},{0.0000*\dy})
	-- ({11.6797*\dx},{0.0000*\dy})
	-- ({11.6897*\dx},{0.0000*\dy})
	-- ({11.6997*\dx},{0.0000*\dy})
	-- ({11.7098*\dx},{0.0000*\dy})
	-- ({11.7198*\dx},{0.0000*\dy})
	-- ({11.7298*\dx},{0.0000*\dy})
	-- ({11.7398*\dx},{0.0000*\dy})
	-- ({11.7498*\dx},{0.0000*\dy})
	-- ({11.7598*\dx},{0.0000*\dy})
	-- ({11.7698*\dx},{0.0000*\dy})
	-- ({11.7798*\dx},{0.0000*\dy})
	-- ({11.7898*\dx},{0.0000*\dy})
	-- ({11.7998*\dx},{0.0000*\dy})
	-- ({11.8098*\dx},{0.0000*\dy})
	-- ({11.8198*\dx},{0.0000*\dy})
	-- ({11.8299*\dx},{0.0000*\dy})
	-- ({11.8399*\dx},{0.0000*\dy})
	-- ({11.8499*\dx},{0.0000*\dy})
	-- ({11.8599*\dx},{0.0000*\dy})
	-- ({11.8699*\dx},{0.0000*\dy})
	-- ({11.8799*\dx},{0.0000*\dy})
	-- ({11.8899*\dx},{0.0000*\dy})
	-- ({11.8999*\dx},{0.0000*\dy})
	-- ({11.9099*\dx},{0.0000*\dy})
	-- ({11.9199*\dx},{0.0000*\dy})
	-- ({11.9299*\dx},{0.0000*\dy})
	-- ({11.9399*\dx},{0.0000*\dy})
	-- ({11.9500*\dx},{0.0000*\dy})
	-- ({11.9600*\dx},{0.0000*\dy})
	-- ({11.9700*\dx},{0.0000*\dy})
	-- ({11.9800*\dx},{0.0000*\dy})
	-- ({11.9900*\dx},{0.0000*\dy})
	-- ({12.0000*\dx},{0.0000*\dy})
}
\def\psifive{
	({0.0000*\dx},{0.0000*\dy})
	-- ({0.0100*\dx},{0.0000*\dy})
	-- ({0.0200*\dx},{0.0000*\dy})
	-- ({0.0300*\dx},{0.0000*\dy})
	-- ({0.0400*\dx},{0.0000*\dy})
	-- ({0.0500*\dx},{0.0000*\dy})
	-- ({0.0601*\dx},{0.0000*\dy})
	-- ({0.0701*\dx},{0.0000*\dy})
	-- ({0.0801*\dx},{0.0000*\dy})
	-- ({0.0901*\dx},{0.0000*\dy})
	-- ({0.1001*\dx},{0.0000*\dy})
	-- ({0.1101*\dx},{0.0000*\dy})
	-- ({0.1201*\dx},{0.0000*\dy})
	-- ({0.1301*\dx},{0.0000*\dy})
	-- ({0.1401*\dx},{0.0000*\dy})
	-- ({0.1501*\dx},{0.0000*\dy})
	-- ({0.1601*\dx},{0.0000*\dy})
	-- ({0.1701*\dx},{0.0000*\dy})
	-- ({0.1802*\dx},{0.0000*\dy})
	-- ({0.1902*\dx},{0.0000*\dy})
	-- ({0.2002*\dx},{0.0000*\dy})
	-- ({0.2102*\dx},{0.0000*\dy})
	-- ({0.2202*\dx},{0.0000*\dy})
	-- ({0.2302*\dx},{0.0000*\dy})
	-- ({0.2402*\dx},{0.0000*\dy})
	-- ({0.2502*\dx},{0.0000*\dy})
	-- ({0.2602*\dx},{0.0000*\dy})
	-- ({0.2702*\dx},{0.0000*\dy})
	-- ({0.2802*\dx},{0.0000*\dy})
	-- ({0.2902*\dx},{0.0000*\dy})
	-- ({0.3003*\dx},{0.0000*\dy})
	-- ({0.3103*\dx},{0.0000*\dy})
	-- ({0.3203*\dx},{0.0000*\dy})
	-- ({0.3303*\dx},{0.0000*\dy})
	-- ({0.3403*\dx},{0.0000*\dy})
	-- ({0.3503*\dx},{0.0000*\dy})
	-- ({0.3603*\dx},{0.0000*\dy})
	-- ({0.3703*\dx},{0.0000*\dy})
	-- ({0.3803*\dx},{0.0000*\dy})
	-- ({0.3903*\dx},{0.0000*\dy})
	-- ({0.4003*\dx},{0.0000*\dy})
	-- ({0.4103*\dx},{0.0000*\dy})
	-- ({0.4204*\dx},{0.0000*\dy})
	-- ({0.4304*\dx},{0.0000*\dy})
	-- ({0.4404*\dx},{0.0000*\dy})
	-- ({0.4504*\dx},{0.0000*\dy})
	-- ({0.4604*\dx},{0.0000*\dy})
	-- ({0.4704*\dx},{0.0000*\dy})
	-- ({0.4804*\dx},{0.0000*\dy})
	-- ({0.4904*\dx},{0.0000*\dy})
	-- ({0.5004*\dx},{0.0000*\dy})
	-- ({0.5104*\dx},{0.0000*\dy})
	-- ({0.5204*\dx},{0.0000*\dy})
	-- ({0.5304*\dx},{0.0000*\dy})
	-- ({0.5405*\dx},{0.0000*\dy})
	-- ({0.5505*\dx},{0.0000*\dy})
	-- ({0.5605*\dx},{0.0000*\dy})
	-- ({0.5705*\dx},{0.0000*\dy})
	-- ({0.5805*\dx},{0.0000*\dy})
	-- ({0.5905*\dx},{0.0000*\dy})
	-- ({0.6005*\dx},{0.0000*\dy})
	-- ({0.6105*\dx},{0.0000*\dy})
	-- ({0.6205*\dx},{0.0000*\dy})
	-- ({0.6305*\dx},{0.0000*\dy})
	-- ({0.6405*\dx},{0.0000*\dy})
	-- ({0.6505*\dx},{0.0000*\dy})
	-- ({0.6606*\dx},{0.0000*\dy})
	-- ({0.6706*\dx},{0.0000*\dy})
	-- ({0.6806*\dx},{0.0000*\dy})
	-- ({0.6906*\dx},{0.0000*\dy})
	-- ({0.7006*\dx},{0.0000*\dy})
	-- ({0.7106*\dx},{0.0000*\dy})
	-- ({0.7206*\dx},{0.0000*\dy})
	-- ({0.7306*\dx},{0.0000*\dy})
	-- ({0.7406*\dx},{0.0000*\dy})
	-- ({0.7506*\dx},{0.0000*\dy})
	-- ({0.7606*\dx},{0.0000*\dy})
	-- ({0.7706*\dx},{0.0000*\dy})
	-- ({0.7807*\dx},{0.0000*\dy})
	-- ({0.7907*\dx},{0.0000*\dy})
	-- ({0.8007*\dx},{0.0000*\dy})
	-- ({0.8107*\dx},{0.0000*\dy})
	-- ({0.8207*\dx},{0.0000*\dy})
	-- ({0.8307*\dx},{0.0000*\dy})
	-- ({0.8407*\dx},{0.0000*\dy})
	-- ({0.8507*\dx},{0.0000*\dy})
	-- ({0.8607*\dx},{0.0000*\dy})
	-- ({0.8707*\dx},{0.0000*\dy})
	-- ({0.8807*\dx},{0.0000*\dy})
	-- ({0.8907*\dx},{0.0000*\dy})
	-- ({0.9008*\dx},{0.0000*\dy})
	-- ({0.9108*\dx},{0.0000*\dy})
	-- ({0.9208*\dx},{0.0000*\dy})
	-- ({0.9308*\dx},{0.0000*\dy})
	-- ({0.9408*\dx},{0.0000*\dy})
	-- ({0.9508*\dx},{0.0000*\dy})
	-- ({0.9608*\dx},{0.0000*\dy})
	-- ({0.9708*\dx},{0.0000*\dy})
	-- ({0.9808*\dx},{0.0000*\dy})
	-- ({0.9908*\dx},{0.0000*\dy})
	-- ({1.0008*\dx},{0.0000*\dy})
	-- ({1.0108*\dx},{0.0000*\dy})
	-- ({1.0209*\dx},{0.0000*\dy})
	-- ({1.0309*\dx},{0.0000*\dy})
	-- ({1.0409*\dx},{0.0000*\dy})
	-- ({1.0509*\dx},{0.0000*\dy})
	-- ({1.0609*\dx},{0.0000*\dy})
	-- ({1.0709*\dx},{0.0000*\dy})
	-- ({1.0809*\dx},{0.0000*\dy})
	-- ({1.0909*\dx},{0.0000*\dy})
	-- ({1.1009*\dx},{0.0000*\dy})
	-- ({1.1109*\dx},{0.0000*\dy})
	-- ({1.1209*\dx},{0.0000*\dy})
	-- ({1.1309*\dx},{0.0000*\dy})
	-- ({1.1410*\dx},{0.0000*\dy})
	-- ({1.1510*\dx},{0.0000*\dy})
	-- ({1.1610*\dx},{0.0000*\dy})
	-- ({1.1710*\dx},{0.0000*\dy})
	-- ({1.1810*\dx},{0.0000*\dy})
	-- ({1.1910*\dx},{0.0000*\dy})
	-- ({1.2010*\dx},{0.0000*\dy})
	-- ({1.2110*\dx},{0.0000*\dy})
	-- ({1.2210*\dx},{0.0000*\dy})
	-- ({1.2310*\dx},{0.0000*\dy})
	-- ({1.2410*\dx},{0.0000*\dy})
	-- ({1.2510*\dx},{0.0000*\dy})
	-- ({1.2611*\dx},{0.0000*\dy})
	-- ({1.2711*\dx},{0.0000*\dy})
	-- ({1.2811*\dx},{0.0000*\dy})
	-- ({1.2911*\dx},{0.0000*\dy})
	-- ({1.3011*\dx},{0.0000*\dy})
	-- ({1.3111*\dx},{0.0000*\dy})
	-- ({1.3211*\dx},{0.0000*\dy})
	-- ({1.3311*\dx},{0.0000*\dy})
	-- ({1.3411*\dx},{0.0000*\dy})
	-- ({1.3511*\dx},{0.0000*\dy})
	-- ({1.3611*\dx},{0.0000*\dy})
	-- ({1.3711*\dx},{0.0000*\dy})
	-- ({1.3812*\dx},{0.0000*\dy})
	-- ({1.3912*\dx},{0.0000*\dy})
	-- ({1.4012*\dx},{0.0000*\dy})
	-- ({1.4112*\dx},{0.0000*\dy})
	-- ({1.4212*\dx},{0.0000*\dy})
	-- ({1.4312*\dx},{0.0000*\dy})
	-- ({1.4412*\dx},{0.0000*\dy})
	-- ({1.4512*\dx},{0.0000*\dy})
	-- ({1.4612*\dx},{0.0000*\dy})
	-- ({1.4712*\dx},{0.0000*\dy})
	-- ({1.4812*\dx},{0.0000*\dy})
	-- ({1.4912*\dx},{0.0000*\dy})
	-- ({1.5013*\dx},{0.0000*\dy})
	-- ({1.5113*\dx},{0.0000*\dy})
	-- ({1.5213*\dx},{0.0000*\dy})
	-- ({1.5313*\dx},{0.0000*\dy})
	-- ({1.5413*\dx},{0.0000*\dy})
	-- ({1.5513*\dx},{0.0000*\dy})
	-- ({1.5613*\dx},{0.0000*\dy})
	-- ({1.5713*\dx},{0.0000*\dy})
	-- ({1.5813*\dx},{0.0000*\dy})
	-- ({1.5913*\dx},{0.0000*\dy})
	-- ({1.6013*\dx},{0.0000*\dy})
	-- ({1.6113*\dx},{0.0000*\dy})
	-- ({1.6214*\dx},{0.0000*\dy})
	-- ({1.6314*\dx},{0.0000*\dy})
	-- ({1.6414*\dx},{0.0000*\dy})
	-- ({1.6514*\dx},{0.0000*\dy})
	-- ({1.6614*\dx},{0.0000*\dy})
	-- ({1.6714*\dx},{0.0000*\dy})
	-- ({1.6814*\dx},{0.0000*\dy})
	-- ({1.6914*\dx},{0.0000*\dy})
	-- ({1.7014*\dx},{0.0000*\dy})
	-- ({1.7114*\dx},{0.0000*\dy})
	-- ({1.7214*\dx},{0.0000*\dy})
	-- ({1.7314*\dx},{0.0000*\dy})
	-- ({1.7415*\dx},{0.0000*\dy})
	-- ({1.7515*\dx},{0.0000*\dy})
	-- ({1.7615*\dx},{0.0000*\dy})
	-- ({1.7715*\dx},{0.0000*\dy})
	-- ({1.7815*\dx},{0.0000*\dy})
	-- ({1.7915*\dx},{0.0000*\dy})
	-- ({1.8015*\dx},{0.0000*\dy})
	-- ({1.8115*\dx},{0.0000*\dy})
	-- ({1.8215*\dx},{0.0000*\dy})
	-- ({1.8315*\dx},{0.0000*\dy})
	-- ({1.8415*\dx},{0.0000*\dy})
	-- ({1.8515*\dx},{0.0000*\dy})
	-- ({1.8616*\dx},{0.0000*\dy})
	-- ({1.8716*\dx},{0.0000*\dy})
	-- ({1.8816*\dx},{0.0000*\dy})
	-- ({1.8916*\dx},{0.0000*\dy})
	-- ({1.9016*\dx},{0.0000*\dy})
	-- ({1.9116*\dx},{0.0000*\dy})
	-- ({1.9216*\dx},{0.0000*\dy})
	-- ({1.9316*\dx},{0.0000*\dy})
	-- ({1.9416*\dx},{0.0000*\dy})
	-- ({1.9516*\dx},{0.0000*\dy})
	-- ({1.9616*\dx},{0.0000*\dy})
	-- ({1.9716*\dx},{0.0000*\dy})
	-- ({1.9817*\dx},{0.0000*\dy})
	-- ({1.9917*\dx},{0.0000*\dy})
	-- ({2.0017*\dx},{0.0000*\dy})
	-- ({2.0117*\dx},{0.0000*\dy})
	-- ({2.0217*\dx},{0.0000*\dy})
	-- ({2.0317*\dx},{0.0000*\dy})
	-- ({2.0417*\dx},{0.0000*\dy})
	-- ({2.0517*\dx},{0.0000*\dy})
	-- ({2.0617*\dx},{0.0000*\dy})
	-- ({2.0717*\dx},{0.0000*\dy})
	-- ({2.0817*\dx},{0.0000*\dy})
	-- ({2.0917*\dx},{0.0000*\dy})
	-- ({2.1018*\dx},{0.0000*\dy})
	-- ({2.1118*\dx},{0.0000*\dy})
	-- ({2.1218*\dx},{0.0000*\dy})
	-- ({2.1318*\dx},{0.0000*\dy})
	-- ({2.1418*\dx},{0.0000*\dy})
	-- ({2.1518*\dx},{0.0000*\dy})
	-- ({2.1618*\dx},{0.0000*\dy})
	-- ({2.1718*\dx},{0.0000*\dy})
	-- ({2.1818*\dx},{0.0000*\dy})
	-- ({2.1918*\dx},{0.0000*\dy})
	-- ({2.2018*\dx},{0.0000*\dy})
	-- ({2.2118*\dx},{0.0000*\dy})
	-- ({2.2219*\dx},{0.0000*\dy})
	-- ({2.2319*\dx},{0.0000*\dy})
	-- ({2.2419*\dx},{0.0000*\dy})
	-- ({2.2519*\dx},{0.0000*\dy})
	-- ({2.2619*\dx},{0.0000*\dy})
	-- ({2.2719*\dx},{0.0000*\dy})
	-- ({2.2819*\dx},{0.0000*\dy})
	-- ({2.2919*\dx},{0.0000*\dy})
	-- ({2.3019*\dx},{0.0000*\dy})
	-- ({2.3119*\dx},{0.0000*\dy})
	-- ({2.3219*\dx},{0.0000*\dy})
	-- ({2.3319*\dx},{0.0000*\dy})
	-- ({2.3420*\dx},{0.0000*\dy})
	-- ({2.3520*\dx},{0.0000*\dy})
	-- ({2.3620*\dx},{0.0000*\dy})
	-- ({2.3720*\dx},{0.0000*\dy})
	-- ({2.3820*\dx},{0.0000*\dy})
	-- ({2.3920*\dx},{0.0000*\dy})
	-- ({2.4020*\dx},{0.0000*\dy})
	-- ({2.4120*\dx},{0.0000*\dy})
	-- ({2.4220*\dx},{0.0000*\dy})
	-- ({2.4320*\dx},{0.0000*\dy})
	-- ({2.4420*\dx},{0.0000*\dy})
	-- ({2.4520*\dx},{0.0000*\dy})
	-- ({2.4621*\dx},{0.0000*\dy})
	-- ({2.4721*\dx},{0.0000*\dy})
	-- ({2.4821*\dx},{0.0000*\dy})
	-- ({2.4921*\dx},{0.0000*\dy})
	-- ({2.5021*\dx},{0.0000*\dy})
	-- ({2.5121*\dx},{0.0000*\dy})
	-- ({2.5221*\dx},{0.0000*\dy})
	-- ({2.5321*\dx},{0.0000*\dy})
	-- ({2.5421*\dx},{0.0000*\dy})
	-- ({2.5521*\dx},{0.0000*\dy})
	-- ({2.5621*\dx},{0.0000*\dy})
	-- ({2.5721*\dx},{0.0000*\dy})
	-- ({2.5822*\dx},{0.0000*\dy})
	-- ({2.5922*\dx},{0.0000*\dy})
	-- ({2.6022*\dx},{0.0000*\dy})
	-- ({2.6122*\dx},{0.0000*\dy})
	-- ({2.6222*\dx},{0.0000*\dy})
	-- ({2.6322*\dx},{0.0000*\dy})
	-- ({2.6422*\dx},{0.0000*\dy})
	-- ({2.6522*\dx},{0.0000*\dy})
	-- ({2.6622*\dx},{0.0000*\dy})
	-- ({2.6722*\dx},{0.0000*\dy})
	-- ({2.6822*\dx},{0.0000*\dy})
	-- ({2.6922*\dx},{0.0000*\dy})
	-- ({2.7023*\dx},{0.0000*\dy})
	-- ({2.7123*\dx},{0.0000*\dy})
	-- ({2.7223*\dx},{0.0000*\dy})
	-- ({2.7323*\dx},{0.0000*\dy})
	-- ({2.7423*\dx},{0.0000*\dy})
	-- ({2.7523*\dx},{0.0000*\dy})
	-- ({2.7623*\dx},{0.0000*\dy})
	-- ({2.7723*\dx},{0.0000*\dy})
	-- ({2.7823*\dx},{0.0000*\dy})
	-- ({2.7923*\dx},{0.0000*\dy})
	-- ({2.8023*\dx},{0.0000*\dy})
	-- ({2.8123*\dx},{0.0000*\dy})
	-- ({2.8224*\dx},{0.0000*\dy})
	-- ({2.8324*\dx},{0.0000*\dy})
	-- ({2.8424*\dx},{0.0000*\dy})
	-- ({2.8524*\dx},{0.0000*\dy})
	-- ({2.8624*\dx},{0.0000*\dy})
	-- ({2.8724*\dx},{0.0000*\dy})
	-- ({2.8824*\dx},{0.0000*\dy})
	-- ({2.8924*\dx},{0.0000*\dy})
	-- ({2.9024*\dx},{0.0000*\dy})
	-- ({2.9124*\dx},{0.0000*\dy})
	-- ({2.9224*\dx},{0.0000*\dy})
	-- ({2.9324*\dx},{0.0000*\dy})
	-- ({2.9425*\dx},{0.0000*\dy})
	-- ({2.9525*\dx},{0.0000*\dy})
	-- ({2.9625*\dx},{0.0000*\dy})
	-- ({2.9725*\dx},{0.0000*\dy})
	-- ({2.9825*\dx},{0.0000*\dy})
	-- ({2.9925*\dx},{0.0000*\dy})
	-- ({3.0025*\dx},{0.0000*\dy})
	-- ({3.0125*\dx},{0.0000*\dy})
	-- ({3.0225*\dx},{0.0000*\dy})
	-- ({3.0325*\dx},{0.0000*\dy})
	-- ({3.0425*\dx},{0.0000*\dy})
	-- ({3.0525*\dx},{0.0000*\dy})
	-- ({3.0626*\dx},{0.0000*\dy})
	-- ({3.0726*\dx},{0.0000*\dy})
	-- ({3.0826*\dx},{0.0000*\dy})
	-- ({3.0926*\dx},{0.0000*\dy})
	-- ({3.1026*\dx},{0.0000*\dy})
	-- ({3.1126*\dx},{0.0000*\dy})
	-- ({3.1226*\dx},{0.0000*\dy})
	-- ({3.1326*\dx},{0.0000*\dy})
	-- ({3.1426*\dx},{0.0000*\dy})
	-- ({3.1526*\dx},{0.0000*\dy})
	-- ({3.1626*\dx},{0.0000*\dy})
	-- ({3.1726*\dx},{0.0000*\dy})
	-- ({3.1827*\dx},{0.0000*\dy})
	-- ({3.1927*\dx},{0.0000*\dy})
	-- ({3.2027*\dx},{0.0000*\dy})
	-- ({3.2127*\dx},{0.0000*\dy})
	-- ({3.2227*\dx},{0.0000*\dy})
	-- ({3.2327*\dx},{0.0000*\dy})
	-- ({3.2427*\dx},{0.0000*\dy})
	-- ({3.2527*\dx},{0.0000*\dy})
	-- ({3.2627*\dx},{0.0000*\dy})
	-- ({3.2727*\dx},{0.0000*\dy})
	-- ({3.2827*\dx},{0.0000*\dy})
	-- ({3.2927*\dx},{0.0000*\dy})
	-- ({3.3028*\dx},{0.0000*\dy})
	-- ({3.3128*\dx},{0.0000*\dy})
	-- ({3.3228*\dx},{0.0000*\dy})
	-- ({3.3328*\dx},{0.0000*\dy})
	-- ({3.3428*\dx},{0.0000*\dy})
	-- ({3.3528*\dx},{0.0000*\dy})
	-- ({3.3628*\dx},{0.0000*\dy})
	-- ({3.3728*\dx},{0.0000*\dy})
	-- ({3.3828*\dx},{0.0000*\dy})
	-- ({3.3928*\dx},{0.0000*\dy})
	-- ({3.4028*\dx},{0.0000*\dy})
	-- ({3.4128*\dx},{0.0000*\dy})
	-- ({3.4229*\dx},{0.0000*\dy})
	-- ({3.4329*\dx},{0.0000*\dy})
	-- ({3.4429*\dx},{0.0000*\dy})
	-- ({3.4529*\dx},{0.0000*\dy})
	-- ({3.4629*\dx},{0.0000*\dy})
	-- ({3.4729*\dx},{0.0000*\dy})
	-- ({3.4829*\dx},{0.0000*\dy})
	-- ({3.4929*\dx},{0.0000*\dy})
	-- ({3.5029*\dx},{0.0000*\dy})
	-- ({3.5129*\dx},{0.0000*\dy})
	-- ({3.5229*\dx},{0.0000*\dy})
	-- ({3.5329*\dx},{0.0000*\dy})
	-- ({3.5430*\dx},{0.0000*\dy})
	-- ({3.5530*\dx},{0.0000*\dy})
	-- ({3.5630*\dx},{0.0000*\dy})
	-- ({3.5730*\dx},{0.0000*\dy})
	-- ({3.5830*\dx},{0.0000*\dy})
	-- ({3.5930*\dx},{0.0000*\dy})
	-- ({3.6030*\dx},{0.0000*\dy})
	-- ({3.6130*\dx},{0.0000*\dy})
	-- ({3.6230*\dx},{0.0000*\dy})
	-- ({3.6330*\dx},{0.0000*\dy})
	-- ({3.6430*\dx},{0.0000*\dy})
	-- ({3.6530*\dx},{0.0000*\dy})
	-- ({3.6631*\dx},{0.0000*\dy})
	-- ({3.6731*\dx},{0.0000*\dy})
	-- ({3.6831*\dx},{0.0000*\dy})
	-- ({3.6931*\dx},{0.0000*\dy})
	-- ({3.7031*\dx},{0.0000*\dy})
	-- ({3.7131*\dx},{0.0000*\dy})
	-- ({3.7231*\dx},{0.0000*\dy})
	-- ({3.7331*\dx},{0.0000*\dy})
	-- ({3.7431*\dx},{0.0000*\dy})
	-- ({3.7531*\dx},{0.0000*\dy})
	-- ({3.7631*\dx},{0.0000*\dy})
	-- ({3.7731*\dx},{0.0000*\dy})
	-- ({3.7832*\dx},{0.0000*\dy})
	-- ({3.7932*\dx},{0.0000*\dy})
	-- ({3.8032*\dx},{0.0000*\dy})
	-- ({3.8132*\dx},{0.0000*\dy})
	-- ({3.8232*\dx},{0.0000*\dy})
	-- ({3.8332*\dx},{0.0000*\dy})
	-- ({3.8432*\dx},{0.0000*\dy})
	-- ({3.8532*\dx},{0.0000*\dy})
	-- ({3.8632*\dx},{0.0000*\dy})
	-- ({3.8732*\dx},{0.0000*\dy})
	-- ({3.8832*\dx},{0.0000*\dy})
	-- ({3.8932*\dx},{0.0000*\dy})
	-- ({3.9033*\dx},{0.0000*\dy})
	-- ({3.9133*\dx},{0.0000*\dy})
	-- ({3.9233*\dx},{0.0000*\dy})
	-- ({3.9333*\dx},{0.0000*\dy})
	-- ({3.9433*\dx},{0.0000*\dy})
	-- ({3.9533*\dx},{0.0000*\dy})
	-- ({3.9633*\dx},{0.0000*\dy})
	-- ({3.9733*\dx},{0.0000*\dy})
	-- ({3.9833*\dx},{0.0000*\dy})
	-- ({3.9933*\dx},{0.0000*\dy})
	-- ({4.0033*\dx},{0.0000*\dy})
	-- ({4.0133*\dx},{0.0000*\dy})
	-- ({4.0234*\dx},{0.0000*\dy})
	-- ({4.0334*\dx},{0.0000*\dy})
	-- ({4.0434*\dx},{0.0000*\dy})
	-- ({4.0534*\dx},{0.0000*\dy})
	-- ({4.0634*\dx},{0.0000*\dy})
	-- ({4.0734*\dx},{0.0000*\dy})
	-- ({4.0834*\dx},{0.0000*\dy})
	-- ({4.0934*\dx},{0.0000*\dy})
	-- ({4.1034*\dx},{0.0000*\dy})
	-- ({4.1134*\dx},{0.0000*\dy})
	-- ({4.1234*\dx},{0.0000*\dy})
	-- ({4.1334*\dx},{0.0000*\dy})
	-- ({4.1435*\dx},{0.0000*\dy})
	-- ({4.1535*\dx},{0.0000*\dy})
	-- ({4.1635*\dx},{0.0000*\dy})
	-- ({4.1735*\dx},{0.0000*\dy})
	-- ({4.1835*\dx},{0.0000*\dy})
	-- ({4.1935*\dx},{0.0000*\dy})
	-- ({4.2035*\dx},{0.0000*\dy})
	-- ({4.2135*\dx},{0.0000*\dy})
	-- ({4.2235*\dx},{0.0000*\dy})
	-- ({4.2335*\dx},{0.0000*\dy})
	-- ({4.2435*\dx},{0.0000*\dy})
	-- ({4.2535*\dx},{0.0000*\dy})
	-- ({4.2636*\dx},{0.0000*\dy})
	-- ({4.2736*\dx},{0.0000*\dy})
	-- ({4.2836*\dx},{0.0000*\dy})
	-- ({4.2936*\dx},{0.0000*\dy})
	-- ({4.3036*\dx},{0.0000*\dy})
	-- ({4.3136*\dx},{0.0000*\dy})
	-- ({4.3236*\dx},{0.0000*\dy})
	-- ({4.3336*\dx},{0.0000*\dy})
	-- ({4.3436*\dx},{0.0000*\dy})
	-- ({4.3536*\dx},{0.0000*\dy})
	-- ({4.3636*\dx},{0.0000*\dy})
	-- ({4.3736*\dx},{0.0000*\dy})
	-- ({4.3837*\dx},{0.0000*\dy})
	-- ({4.3937*\dx},{0.0000*\dy})
	-- ({4.4037*\dx},{0.0000*\dy})
	-- ({4.4137*\dx},{0.0000*\dy})
	-- ({4.4237*\dx},{0.0000*\dy})
	-- ({4.4337*\dx},{0.0000*\dy})
	-- ({4.4437*\dx},{0.0000*\dy})
	-- ({4.4537*\dx},{0.0000*\dy})
	-- ({4.4637*\dx},{0.0000*\dy})
	-- ({4.4737*\dx},{0.0000*\dy})
	-- ({4.4837*\dx},{0.0000*\dy})
	-- ({4.4937*\dx},{0.0000*\dy})
	-- ({4.5038*\dx},{0.0000*\dy})
	-- ({4.5138*\dx},{0.0000*\dy})
	-- ({4.5238*\dx},{0.0000*\dy})
	-- ({4.5338*\dx},{0.0000*\dy})
	-- ({4.5438*\dx},{0.0000*\dy})
	-- ({4.5538*\dx},{0.0000*\dy})
	-- ({4.5638*\dx},{0.0000*\dy})
	-- ({4.5738*\dx},{0.0000*\dy})
	-- ({4.5838*\dx},{0.0000*\dy})
	-- ({4.5938*\dx},{0.0000*\dy})
	-- ({4.6038*\dx},{0.0000*\dy})
	-- ({4.6138*\dx},{0.0000*\dy})
	-- ({4.6239*\dx},{0.0000*\dy})
	-- ({4.6339*\dx},{0.0000*\dy})
	-- ({4.6439*\dx},{0.0000*\dy})
	-- ({4.6539*\dx},{0.0000*\dy})
	-- ({4.6639*\dx},{0.0000*\dy})
	-- ({4.6739*\dx},{0.0000*\dy})
	-- ({4.6839*\dx},{0.0000*\dy})
	-- ({4.6939*\dx},{0.0000*\dy})
	-- ({4.7039*\dx},{0.0000*\dy})
	-- ({4.7139*\dx},{0.0000*\dy})
	-- ({4.7239*\dx},{0.0000*\dy})
	-- ({4.7339*\dx},{0.0000*\dy})
	-- ({4.7440*\dx},{0.0000*\dy})
	-- ({4.7540*\dx},{0.0000*\dy})
	-- ({4.7640*\dx},{0.0000*\dy})
	-- ({4.7740*\dx},{0.0000*\dy})
	-- ({4.7840*\dx},{0.0000*\dy})
	-- ({4.7940*\dx},{0.0000*\dy})
	-- ({4.8040*\dx},{0.0000*\dy})
	-- ({4.8140*\dx},{0.0000*\dy})
	-- ({4.8240*\dx},{0.0000*\dy})
	-- ({4.8340*\dx},{0.0000*\dy})
	-- ({4.8440*\dx},{0.0000*\dy})
	-- ({4.8540*\dx},{0.0000*\dy})
	-- ({4.8641*\dx},{0.0000*\dy})
	-- ({4.8741*\dx},{0.0000*\dy})
	-- ({4.8841*\dx},{0.0000*\dy})
	-- ({4.8941*\dx},{0.0000*\dy})
	-- ({4.9041*\dx},{0.0000*\dy})
	-- ({4.9141*\dx},{0.0000*\dy})
	-- ({4.9241*\dx},{0.0000*\dy})
	-- ({4.9341*\dx},{0.0000*\dy})
	-- ({4.9441*\dx},{0.0000*\dy})
	-- ({4.9541*\dx},{0.0000*\dy})
	-- ({4.9641*\dx},{0.0000*\dy})
	-- ({4.9741*\dx},{0.0000*\dy})
	-- ({4.9842*\dx},{0.0000*\dy})
	-- ({4.9942*\dx},{0.0000*\dy})
	-- ({5.0042*\dx},{0.0000*\dy})
	-- ({5.0142*\dx},{0.0001*\dy})
	-- ({5.0242*\dx},{0.0002*\dy})
	-- ({5.0342*\dx},{0.0004*\dy})
	-- ({5.0442*\dx},{0.0006*\dy})
	-- ({5.0542*\dx},{0.0009*\dy})
	-- ({5.0642*\dx},{0.0013*\dy})
	-- ({5.0742*\dx},{0.0018*\dy})
	-- ({5.0842*\dx},{0.0023*\dy})
	-- ({5.0942*\dx},{0.0029*\dy})
	-- ({5.1043*\dx},{0.0035*\dy})
	-- ({5.1143*\dx},{0.0042*\dy})
	-- ({5.1243*\dx},{0.0050*\dy})
	-- ({5.1343*\dx},{0.0059*\dy})
	-- ({5.1443*\dx},{0.0068*\dy})
	-- ({5.1543*\dx},{0.0078*\dy})
	-- ({5.1643*\dx},{0.0088*\dy})
	-- ({5.1743*\dx},{0.0099*\dy})
	-- ({5.1843*\dx},{0.0111*\dy})
	-- ({5.1943*\dx},{0.0124*\dy})
	-- ({5.2043*\dx},{0.0137*\dy})
	-- ({5.2143*\dx},{0.0151*\dy})
	-- ({5.2244*\dx},{0.0166*\dy})
	-- ({5.2344*\dx},{0.0181*\dy})
	-- ({5.2444*\dx},{0.0197*\dy})
	-- ({5.2544*\dx},{0.0214*\dy})
	-- ({5.2644*\dx},{0.0231*\dy})
	-- ({5.2744*\dx},{0.0249*\dy})
	-- ({5.2844*\dx},{0.0267*\dy})
	-- ({5.2944*\dx},{0.0287*\dy})
	-- ({5.3044*\dx},{0.0306*\dy})
	-- ({5.3144*\dx},{0.0327*\dy})
	-- ({5.3244*\dx},{0.0348*\dy})
	-- ({5.3344*\dx},{0.0370*\dy})
	-- ({5.3445*\dx},{0.0392*\dy})
	-- ({5.3545*\dx},{0.0415*\dy})
	-- ({5.3645*\dx},{0.0439*\dy})
	-- ({5.3745*\dx},{0.0463*\dy})
	-- ({5.3845*\dx},{0.0488*\dy})
	-- ({5.3945*\dx},{0.0513*\dy})
	-- ({5.4045*\dx},{0.0539*\dy})
	-- ({5.4145*\dx},{0.0566*\dy})
	-- ({5.4245*\dx},{0.0593*\dy})
	-- ({5.4345*\dx},{0.0620*\dy})
	-- ({5.4445*\dx},{0.0648*\dy})
	-- ({5.4545*\dx},{0.0677*\dy})
	-- ({5.4646*\dx},{0.0706*\dy})
	-- ({5.4746*\dx},{0.0736*\dy})
	-- ({5.4846*\dx},{0.0766*\dy})
	-- ({5.4946*\dx},{0.0796*\dy})
	-- ({5.5046*\dx},{0.0827*\dy})
	-- ({5.5146*\dx},{0.0858*\dy})
	-- ({5.5246*\dx},{0.0890*\dy})
	-- ({5.5346*\dx},{0.0922*\dy})
	-- ({5.5446*\dx},{0.0955*\dy})
	-- ({5.5546*\dx},{0.0988*\dy})
	-- ({5.5646*\dx},{0.1021*\dy})
	-- ({5.5746*\dx},{0.1054*\dy})
	-- ({5.5847*\dx},{0.1088*\dy})
	-- ({5.5947*\dx},{0.1123*\dy})
	-- ({5.6047*\dx},{0.1157*\dy})
	-- ({5.6147*\dx},{0.1192*\dy})
	-- ({5.6247*\dx},{0.1227*\dy})
	-- ({5.6347*\dx},{0.1262*\dy})
	-- ({5.6447*\dx},{0.1298*\dy})
	-- ({5.6547*\dx},{0.1333*\dy})
	-- ({5.6647*\dx},{0.1369*\dy})
	-- ({5.6747*\dx},{0.1405*\dy})
	-- ({5.6847*\dx},{0.1441*\dy})
	-- ({5.6947*\dx},{0.1478*\dy})
	-- ({5.7048*\dx},{0.1514*\dy})
	-- ({5.7148*\dx},{0.1550*\dy})
	-- ({5.7248*\dx},{0.1587*\dy})
	-- ({5.7348*\dx},{0.1624*\dy})
	-- ({5.7448*\dx},{0.1660*\dy})
	-- ({5.7548*\dx},{0.1697*\dy})
	-- ({5.7648*\dx},{0.1734*\dy})
	-- ({5.7748*\dx},{0.1771*\dy})
	-- ({5.7848*\dx},{0.1808*\dy})
	-- ({5.7948*\dx},{0.1844*\dy})
	-- ({5.8048*\dx},{0.1881*\dy})
	-- ({5.8148*\dx},{0.1918*\dy})
	-- ({5.8249*\dx},{0.1954*\dy})
	-- ({5.8349*\dx},{0.1991*\dy})
	-- ({5.8449*\dx},{0.2027*\dy})
	-- ({5.8549*\dx},{0.2063*\dy})
	-- ({5.8649*\dx},{0.2099*\dy})
	-- ({5.8749*\dx},{0.2135*\dy})
	-- ({5.8849*\dx},{0.2171*\dy})
	-- ({5.8949*\dx},{0.2207*\dy})
	-- ({5.9049*\dx},{0.2242*\dy})
	-- ({5.9149*\dx},{0.2277*\dy})
	-- ({5.9249*\dx},{0.2312*\dy})
	-- ({5.9349*\dx},{0.2347*\dy})
	-- ({5.9450*\dx},{0.2382*\dy})
	-- ({5.9550*\dx},{0.2416*\dy})
	-- ({5.9650*\dx},{0.2450*\dy})
	-- ({5.9750*\dx},{0.2484*\dy})
	-- ({5.9850*\dx},{0.2518*\dy})
	-- ({5.9950*\dx},{0.2551*\dy})
	-- ({6.0050*\dx},{0.2584*\dy})
	-- ({6.0150*\dx},{0.2617*\dy})
	-- ({6.0250*\dx},{0.2650*\dy})
	-- ({6.0350*\dx},{0.2683*\dy})
	-- ({6.0450*\dx},{0.2716*\dy})
	-- ({6.0550*\dx},{0.2749*\dy})
	-- ({6.0651*\dx},{0.2781*\dy})
	-- ({6.0751*\dx},{0.2814*\dy})
	-- ({6.0851*\dx},{0.2846*\dy})
	-- ({6.0951*\dx},{0.2879*\dy})
	-- ({6.1051*\dx},{0.2911*\dy})
	-- ({6.1151*\dx},{0.2944*\dy})
	-- ({6.1251*\dx},{0.2976*\dy})
	-- ({6.1351*\dx},{0.3008*\dy})
	-- ({6.1451*\dx},{0.3040*\dy})
	-- ({6.1551*\dx},{0.3072*\dy})
	-- ({6.1651*\dx},{0.3104*\dy})
	-- ({6.1751*\dx},{0.3136*\dy})
	-- ({6.1852*\dx},{0.3168*\dy})
	-- ({6.1952*\dx},{0.3199*\dy})
	-- ({6.2052*\dx},{0.3231*\dy})
	-- ({6.2152*\dx},{0.3263*\dy})
	-- ({6.2252*\dx},{0.3294*\dy})
	-- ({6.2352*\dx},{0.3325*\dy})
	-- ({6.2452*\dx},{0.3357*\dy})
	-- ({6.2552*\dx},{0.3388*\dy})
	-- ({6.2652*\dx},{0.3419*\dy})
	-- ({6.2752*\dx},{0.3450*\dy})
	-- ({6.2852*\dx},{0.3481*\dy})
	-- ({6.2952*\dx},{0.3512*\dy})
	-- ({6.3053*\dx},{0.3542*\dy})
	-- ({6.3153*\dx},{0.3573*\dy})
	-- ({6.3253*\dx},{0.3604*\dy})
	-- ({6.3353*\dx},{0.3634*\dy})
	-- ({6.3453*\dx},{0.3665*\dy})
	-- ({6.3553*\dx},{0.3695*\dy})
	-- ({6.3653*\dx},{0.3725*\dy})
	-- ({6.3753*\dx},{0.3755*\dy})
	-- ({6.3853*\dx},{0.3785*\dy})
	-- ({6.3953*\dx},{0.3815*\dy})
	-- ({6.4053*\dx},{0.3845*\dy})
	-- ({6.4153*\dx},{0.3875*\dy})
	-- ({6.4254*\dx},{0.3904*\dy})
	-- ({6.4354*\dx},{0.3934*\dy})
	-- ({6.4454*\dx},{0.3963*\dy})
	-- ({6.4554*\dx},{0.3993*\dy})
	-- ({6.4654*\dx},{0.4022*\dy})
	-- ({6.4754*\dx},{0.4051*\dy})
	-- ({6.4854*\dx},{0.4081*\dy})
	-- ({6.4954*\dx},{0.4110*\dy})
	-- ({6.5054*\dx},{0.4138*\dy})
	-- ({6.5154*\dx},{0.4167*\dy})
	-- ({6.5254*\dx},{0.4196*\dy})
	-- ({6.5354*\dx},{0.4225*\dy})
	-- ({6.5455*\dx},{0.4253*\dy})
	-- ({6.5555*\dx},{0.4282*\dy})
	-- ({6.5655*\dx},{0.4310*\dy})
	-- ({6.5755*\dx},{0.4339*\dy})
	-- ({6.5855*\dx},{0.4367*\dy})
	-- ({6.5955*\dx},{0.4395*\dy})
	-- ({6.6055*\dx},{0.4423*\dy})
	-- ({6.6155*\dx},{0.4451*\dy})
	-- ({6.6255*\dx},{0.4479*\dy})
	-- ({6.6355*\dx},{0.4507*\dy})
	-- ({6.6455*\dx},{0.4534*\dy})
	-- ({6.6555*\dx},{0.4562*\dy})
	-- ({6.6656*\dx},{0.4590*\dy})
	-- ({6.6756*\dx},{0.4617*\dy})
	-- ({6.6856*\dx},{0.4645*\dy})
	-- ({6.6956*\dx},{0.4672*\dy})
	-- ({6.7056*\dx},{0.4699*\dy})
	-- ({6.7156*\dx},{0.4726*\dy})
	-- ({6.7256*\dx},{0.4753*\dy})
	-- ({6.7356*\dx},{0.4780*\dy})
	-- ({6.7456*\dx},{0.4807*\dy})
	-- ({6.7556*\dx},{0.4834*\dy})
	-- ({6.7656*\dx},{0.4861*\dy})
	-- ({6.7756*\dx},{0.4887*\dy})
	-- ({6.7857*\dx},{0.4914*\dy})
	-- ({6.7957*\dx},{0.4941*\dy})
	-- ({6.8057*\dx},{0.4967*\dy})
	-- ({6.8157*\dx},{0.4993*\dy})
	-- ({6.8257*\dx},{0.5020*\dy})
	-- ({6.8357*\dx},{0.5046*\dy})
	-- ({6.8457*\dx},{0.5072*\dy})
	-- ({6.8557*\dx},{0.5098*\dy})
	-- ({6.8657*\dx},{0.5124*\dy})
	-- ({6.8757*\dx},{0.5150*\dy})
	-- ({6.8857*\dx},{0.5176*\dy})
	-- ({6.8957*\dx},{0.5202*\dy})
	-- ({6.9058*\dx},{0.5228*\dy})
	-- ({6.9158*\dx},{0.5253*\dy})
	-- ({6.9258*\dx},{0.5279*\dy})
	-- ({6.9358*\dx},{0.5304*\dy})
	-- ({6.9458*\dx},{0.5330*\dy})
	-- ({6.9558*\dx},{0.5355*\dy})
	-- ({6.9658*\dx},{0.5381*\dy})
	-- ({6.9758*\dx},{0.5406*\dy})
	-- ({6.9858*\dx},{0.5431*\dy})
	-- ({6.9958*\dx},{0.5456*\dy})
	-- ({7.0058*\dx},{0.5482*\dy})
	-- ({7.0158*\dx},{0.5507*\dy})
	-- ({7.0259*\dx},{0.5532*\dy})
	-- ({7.0359*\dx},{0.5557*\dy})
	-- ({7.0459*\dx},{0.5581*\dy})
	-- ({7.0559*\dx},{0.5606*\dy})
	-- ({7.0659*\dx},{0.5631*\dy})
	-- ({7.0759*\dx},{0.5656*\dy})
	-- ({7.0859*\dx},{0.5680*\dy})
	-- ({7.0959*\dx},{0.5705*\dy})
	-- ({7.1059*\dx},{0.5730*\dy})
	-- ({7.1159*\dx},{0.5754*\dy})
	-- ({7.1259*\dx},{0.5779*\dy})
	-- ({7.1359*\dx},{0.5803*\dy})
	-- ({7.1460*\dx},{0.5827*\dy})
	-- ({7.1560*\dx},{0.5852*\dy})
	-- ({7.1660*\dx},{0.5876*\dy})
	-- ({7.1760*\dx},{0.5900*\dy})
	-- ({7.1860*\dx},{0.5925*\dy})
	-- ({7.1960*\dx},{0.5949*\dy})
	-- ({7.2060*\dx},{0.5973*\dy})
	-- ({7.2160*\dx},{0.5997*\dy})
	-- ({7.2260*\dx},{0.6021*\dy})
	-- ({7.2360*\dx},{0.6045*\dy})
	-- ({7.2460*\dx},{0.6069*\dy})
	-- ({7.2560*\dx},{0.6093*\dy})
	-- ({7.2661*\dx},{0.6117*\dy})
	-- ({7.2761*\dx},{0.6140*\dy})
	-- ({7.2861*\dx},{0.6164*\dy})
	-- ({7.2961*\dx},{0.6188*\dy})
	-- ({7.3061*\dx},{0.6212*\dy})
	-- ({7.3161*\dx},{0.6235*\dy})
	-- ({7.3261*\dx},{0.6259*\dy})
	-- ({7.3361*\dx},{0.6283*\dy})
	-- ({7.3461*\dx},{0.6306*\dy})
	-- ({7.3561*\dx},{0.6330*\dy})
	-- ({7.3661*\dx},{0.6354*\dy})
	-- ({7.3761*\dx},{0.6377*\dy})
	-- ({7.3862*\dx},{0.6401*\dy})
	-- ({7.3962*\dx},{0.6424*\dy})
	-- ({7.4062*\dx},{0.6448*\dy})
	-- ({7.4162*\dx},{0.6471*\dy})
	-- ({7.4262*\dx},{0.6494*\dy})
	-- ({7.4362*\dx},{0.6518*\dy})
	-- ({7.4462*\dx},{0.6541*\dy})
	-- ({7.4562*\dx},{0.6565*\dy})
	-- ({7.4662*\dx},{0.6588*\dy})
	-- ({7.4762*\dx},{0.6611*\dy})
	-- ({7.4862*\dx},{0.6635*\dy})
	-- ({7.4962*\dx},{0.6658*\dy})
	-- ({7.5063*\dx},{0.6681*\dy})
	-- ({7.5163*\dx},{0.6704*\dy})
	-- ({7.5263*\dx},{0.6728*\dy})
	-- ({7.5363*\dx},{0.6751*\dy})
	-- ({7.5463*\dx},{0.6774*\dy})
	-- ({7.5563*\dx},{0.6797*\dy})
	-- ({7.5663*\dx},{0.6821*\dy})
	-- ({7.5763*\dx},{0.6844*\dy})
	-- ({7.5863*\dx},{0.6867*\dy})
	-- ({7.5963*\dx},{0.6890*\dy})
	-- ({7.6063*\dx},{0.6914*\dy})
	-- ({7.6163*\dx},{0.6937*\dy})
	-- ({7.6264*\dx},{0.6960*\dy})
	-- ({7.6364*\dx},{0.6983*\dy})
	-- ({7.6464*\dx},{0.7006*\dy})
	-- ({7.6564*\dx},{0.7030*\dy})
	-- ({7.6664*\dx},{0.7053*\dy})
	-- ({7.6764*\dx},{0.7076*\dy})
	-- ({7.6864*\dx},{0.7099*\dy})
	-- ({7.6964*\dx},{0.7122*\dy})
	-- ({7.7064*\dx},{0.7146*\dy})
	-- ({7.7164*\dx},{0.7169*\dy})
	-- ({7.7264*\dx},{0.7192*\dy})
	-- ({7.7364*\dx},{0.7215*\dy})
	-- ({7.7465*\dx},{0.7239*\dy})
	-- ({7.7565*\dx},{0.7262*\dy})
	-- ({7.7665*\dx},{0.7285*\dy})
	-- ({7.7765*\dx},{0.7309*\dy})
	-- ({7.7865*\dx},{0.7332*\dy})
	-- ({7.7965*\dx},{0.7355*\dy})
	-- ({7.8065*\dx},{0.7379*\dy})
	-- ({7.8165*\dx},{0.7402*\dy})
	-- ({7.8265*\dx},{0.7425*\dy})
	-- ({7.8365*\dx},{0.7449*\dy})
	-- ({7.8465*\dx},{0.7472*\dy})
	-- ({7.8565*\dx},{0.7496*\dy})
	-- ({7.8666*\dx},{0.7519*\dy})
	-- ({7.8766*\dx},{0.7543*\dy})
	-- ({7.8866*\dx},{0.7566*\dy})
	-- ({7.8966*\dx},{0.7590*\dy})
	-- ({7.9066*\dx},{0.7613*\dy})
	-- ({7.9166*\dx},{0.7637*\dy})
	-- ({7.9266*\dx},{0.7660*\dy})
	-- ({7.9366*\dx},{0.7684*\dy})
	-- ({7.9466*\dx},{0.7708*\dy})
	-- ({7.9566*\dx},{0.7731*\dy})
	-- ({7.9666*\dx},{0.7755*\dy})
	-- ({7.9766*\dx},{0.7779*\dy})
	-- ({7.9867*\dx},{0.7803*\dy})
	-- ({7.9967*\dx},{0.7827*\dy})
	-- ({8.0067*\dx},{0.7850*\dy})
	-- ({8.0167*\dx},{0.7872*\dy})
	-- ({8.0267*\dx},{0.7893*\dy})
	-- ({8.0367*\dx},{0.7912*\dy})
	-- ({8.0467*\dx},{0.7929*\dy})
	-- ({8.0567*\dx},{0.7945*\dy})
	-- ({8.0667*\dx},{0.7959*\dy})
	-- ({8.0767*\dx},{0.7971*\dy})
	-- ({8.0867*\dx},{0.7982*\dy})
	-- ({8.0967*\dx},{0.7991*\dy})
	-- ({8.1068*\dx},{0.7998*\dy})
	-- ({8.1168*\dx},{0.8003*\dy})
	-- ({8.1268*\dx},{0.8006*\dy})
	-- ({8.1368*\dx},{0.8007*\dy})
	-- ({8.1468*\dx},{0.8006*\dy})
	-- ({8.1568*\dx},{0.8004*\dy})
	-- ({8.1668*\dx},{0.7999*\dy})
	-- ({8.1768*\dx},{0.7992*\dy})
	-- ({8.1868*\dx},{0.7984*\dy})
	-- ({8.1968*\dx},{0.7973*\dy})
	-- ({8.2068*\dx},{0.7960*\dy})
	-- ({8.2168*\dx},{0.7945*\dy})
	-- ({8.2269*\dx},{0.7928*\dy})
	-- ({8.2369*\dx},{0.7909*\dy})
	-- ({8.2469*\dx},{0.7888*\dy})
	-- ({8.2569*\dx},{0.7865*\dy})
	-- ({8.2669*\dx},{0.7840*\dy})
	-- ({8.2769*\dx},{0.7813*\dy})
	-- ({8.2869*\dx},{0.7784*\dy})
	-- ({8.2969*\dx},{0.7752*\dy})
	-- ({8.3069*\dx},{0.7719*\dy})
	-- ({8.3169*\dx},{0.7684*\dy})
	-- ({8.3269*\dx},{0.7647*\dy})
	-- ({8.3369*\dx},{0.7608*\dy})
	-- ({8.3470*\dx},{0.7567*\dy})
	-- ({8.3570*\dx},{0.7524*\dy})
	-- ({8.3670*\dx},{0.7480*\dy})
	-- ({8.3770*\dx},{0.7434*\dy})
	-- ({8.3870*\dx},{0.7386*\dy})
	-- ({8.3970*\dx},{0.7336*\dy})
	-- ({8.4070*\dx},{0.7284*\dy})
	-- ({8.4170*\dx},{0.7231*\dy})
	-- ({8.4270*\dx},{0.7177*\dy})
	-- ({8.4370*\dx},{0.7121*\dy})
	-- ({8.4470*\dx},{0.7063*\dy})
	-- ({8.4570*\dx},{0.7005*\dy})
	-- ({8.4671*\dx},{0.6944*\dy})
	-- ({8.4771*\dx},{0.6883*\dy})
	-- ({8.4871*\dx},{0.6820*\dy})
	-- ({8.4971*\dx},{0.6756*\dy})
	-- ({8.5071*\dx},{0.6691*\dy})
	-- ({8.5171*\dx},{0.6625*\dy})
	-- ({8.5271*\dx},{0.6558*\dy})
	-- ({8.5371*\dx},{0.6489*\dy})
	-- ({8.5471*\dx},{0.6420*\dy})
	-- ({8.5571*\dx},{0.6351*\dy})
	-- ({8.5671*\dx},{0.6280*\dy})
	-- ({8.5771*\dx},{0.6209*\dy})
	-- ({8.5872*\dx},{0.6137*\dy})
	-- ({8.5972*\dx},{0.6064*\dy})
	-- ({8.6072*\dx},{0.5991*\dy})
	-- ({8.6172*\dx},{0.5917*\dy})
	-- ({8.6272*\dx},{0.5843*\dy})
	-- ({8.6372*\dx},{0.5768*\dy})
	-- ({8.6472*\dx},{0.5693*\dy})
	-- ({8.6572*\dx},{0.5618*\dy})
	-- ({8.6672*\dx},{0.5543*\dy})
	-- ({8.6772*\dx},{0.5467*\dy})
	-- ({8.6872*\dx},{0.5392*\dy})
	-- ({8.6972*\dx},{0.5316*\dy})
	-- ({8.7073*\dx},{0.5240*\dy})
	-- ({8.7173*\dx},{0.5164*\dy})
	-- ({8.7273*\dx},{0.5088*\dy})
	-- ({8.7373*\dx},{0.5012*\dy})
	-- ({8.7473*\dx},{0.4937*\dy})
	-- ({8.7573*\dx},{0.4861*\dy})
	-- ({8.7673*\dx},{0.4786*\dy})
	-- ({8.7773*\dx},{0.4711*\dy})
	-- ({8.7873*\dx},{0.4636*\dy})
	-- ({8.7973*\dx},{0.4561*\dy})
	-- ({8.8073*\dx},{0.4487*\dy})
	-- ({8.8173*\dx},{0.4413*\dy})
	-- ({8.8274*\dx},{0.4340*\dy})
	-- ({8.8374*\dx},{0.4267*\dy})
	-- ({8.8474*\dx},{0.4194*\dy})
	-- ({8.8574*\dx},{0.4122*\dy})
	-- ({8.8674*\dx},{0.4050*\dy})
	-- ({8.8774*\dx},{0.3979*\dy})
	-- ({8.8874*\dx},{0.3908*\dy})
	-- ({8.8974*\dx},{0.3838*\dy})
	-- ({8.9074*\dx},{0.3769*\dy})
	-- ({8.9174*\dx},{0.3700*\dy})
	-- ({8.9274*\dx},{0.3631*\dy})
	-- ({8.9374*\dx},{0.3563*\dy})
	-- ({8.9475*\dx},{0.3496*\dy})
	-- ({8.9575*\dx},{0.3430*\dy})
	-- ({8.9675*\dx},{0.3364*\dy})
	-- ({8.9775*\dx},{0.3298*\dy})
	-- ({8.9875*\dx},{0.3234*\dy})
	-- ({8.9975*\dx},{0.3170*\dy})
	-- ({9.0075*\dx},{0.3106*\dy})
	-- ({9.0175*\dx},{0.3044*\dy})
	-- ({9.0275*\dx},{0.2982*\dy})
	-- ({9.0375*\dx},{0.2920*\dy})
	-- ({9.0475*\dx},{0.2860*\dy})
	-- ({9.0575*\dx},{0.2800*\dy})
	-- ({9.0676*\dx},{0.2741*\dy})
	-- ({9.0776*\dx},{0.2682*\dy})
	-- ({9.0876*\dx},{0.2624*\dy})
	-- ({9.0976*\dx},{0.2567*\dy})
	-- ({9.1076*\dx},{0.2511*\dy})
	-- ({9.1176*\dx},{0.2455*\dy})
	-- ({9.1276*\dx},{0.2400*\dy})
	-- ({9.1376*\dx},{0.2346*\dy})
	-- ({9.1476*\dx},{0.2292*\dy})
	-- ({9.1576*\dx},{0.2239*\dy})
	-- ({9.1676*\dx},{0.2187*\dy})
	-- ({9.1776*\dx},{0.2135*\dy})
	-- ({9.1877*\dx},{0.2084*\dy})
	-- ({9.1977*\dx},{0.2034*\dy})
	-- ({9.2077*\dx},{0.1985*\dy})
	-- ({9.2177*\dx},{0.1936*\dy})
	-- ({9.2277*\dx},{0.1888*\dy})
	-- ({9.2377*\dx},{0.1840*\dy})
	-- ({9.2477*\dx},{0.1793*\dy})
	-- ({9.2577*\dx},{0.1747*\dy})
	-- ({9.2677*\dx},{0.1702*\dy})
	-- ({9.2777*\dx},{0.1657*\dy})
	-- ({9.2877*\dx},{0.1613*\dy})
	-- ({9.2977*\dx},{0.1569*\dy})
	-- ({9.3078*\dx},{0.1526*\dy})
	-- ({9.3178*\dx},{0.1484*\dy})
	-- ({9.3278*\dx},{0.1442*\dy})
	-- ({9.3378*\dx},{0.1401*\dy})
	-- ({9.3478*\dx},{0.1361*\dy})
	-- ({9.3578*\dx},{0.1321*\dy})
	-- ({9.3678*\dx},{0.1282*\dy})
	-- ({9.3778*\dx},{0.1244*\dy})
	-- ({9.3878*\dx},{0.1206*\dy})
	-- ({9.3978*\dx},{0.1168*\dy})
	-- ({9.4078*\dx},{0.1132*\dy})
	-- ({9.4178*\dx},{0.1096*\dy})
	-- ({9.4279*\dx},{0.1060*\dy})
	-- ({9.4379*\dx},{0.1025*\dy})
	-- ({9.4479*\dx},{0.0991*\dy})
	-- ({9.4579*\dx},{0.0957*\dy})
	-- ({9.4679*\dx},{0.0924*\dy})
	-- ({9.4779*\dx},{0.0891*\dy})
	-- ({9.4879*\dx},{0.0859*\dy})
	-- ({9.4979*\dx},{0.0828*\dy})
	-- ({9.5079*\dx},{0.0797*\dy})
	-- ({9.5179*\dx},{0.0767*\dy})
	-- ({9.5279*\dx},{0.0737*\dy})
	-- ({9.5379*\dx},{0.0708*\dy})
	-- ({9.5480*\dx},{0.0679*\dy})
	-- ({9.5580*\dx},{0.0651*\dy})
	-- ({9.5680*\dx},{0.0624*\dy})
	-- ({9.5780*\dx},{0.0597*\dy})
	-- ({9.5880*\dx},{0.0570*\dy})
	-- ({9.5980*\dx},{0.0545*\dy})
	-- ({9.6080*\dx},{0.0519*\dy})
	-- ({9.6180*\dx},{0.0495*\dy})
	-- ({9.6280*\dx},{0.0470*\dy})
	-- ({9.6380*\dx},{0.0447*\dy})
	-- ({9.6480*\dx},{0.0424*\dy})
	-- ({9.6580*\dx},{0.0401*\dy})
	-- ({9.6681*\dx},{0.0380*\dy})
	-- ({9.6781*\dx},{0.0358*\dy})
	-- ({9.6881*\dx},{0.0337*\dy})
	-- ({9.6981*\dx},{0.0317*\dy})
	-- ({9.7081*\dx},{0.0298*\dy})
	-- ({9.7181*\dx},{0.0279*\dy})
	-- ({9.7281*\dx},{0.0260*\dy})
	-- ({9.7381*\dx},{0.0242*\dy})
	-- ({9.7481*\dx},{0.0225*\dy})
	-- ({9.7581*\dx},{0.0208*\dy})
	-- ({9.7681*\dx},{0.0192*\dy})
	-- ({9.7781*\dx},{0.0176*\dy})
	-- ({9.7882*\dx},{0.0162*\dy})
	-- ({9.7982*\dx},{0.0147*\dy})
	-- ({9.8082*\dx},{0.0134*\dy})
	-- ({9.8182*\dx},{0.0120*\dy})
	-- ({9.8282*\dx},{0.0108*\dy})
	-- ({9.8382*\dx},{0.0096*\dy})
	-- ({9.8482*\dx},{0.0085*\dy})
	-- ({9.8582*\dx},{0.0075*\dy})
	-- ({9.8682*\dx},{0.0065*\dy})
	-- ({9.8782*\dx},{0.0055*\dy})
	-- ({9.8882*\dx},{0.0047*\dy})
	-- ({9.8982*\dx},{0.0039*\dy})
	-- ({9.9083*\dx},{0.0032*\dy})
	-- ({9.9183*\dx},{0.0025*\dy})
	-- ({9.9283*\dx},{0.0020*\dy})
	-- ({9.9383*\dx},{0.0015*\dy})
	-- ({9.9483*\dx},{0.0010*\dy})
	-- ({9.9583*\dx},{0.0007*\dy})
	-- ({9.9683*\dx},{0.0004*\dy})
	-- ({9.9783*\dx},{0.0002*\dy})
	-- ({9.9883*\dx},{0.0001*\dy})
	-- ({9.9983*\dx},{0.0000*\dy})
	-- ({10.0083*\dx},{0.0000*\dy})
	-- ({10.0183*\dx},{0.0000*\dy})
	-- ({10.0284*\dx},{0.0000*\dy})
	-- ({10.0384*\dx},{0.0000*\dy})
	-- ({10.0484*\dx},{0.0000*\dy})
	-- ({10.0584*\dx},{0.0000*\dy})
	-- ({10.0684*\dx},{0.0000*\dy})
	-- ({10.0784*\dx},{0.0000*\dy})
	-- ({10.0884*\dx},{0.0000*\dy})
	-- ({10.0984*\dx},{0.0000*\dy})
	-- ({10.1084*\dx},{0.0000*\dy})
	-- ({10.1184*\dx},{0.0000*\dy})
	-- ({10.1284*\dx},{0.0000*\dy})
	-- ({10.1384*\dx},{0.0000*\dy})
	-- ({10.1485*\dx},{0.0000*\dy})
	-- ({10.1585*\dx},{0.0000*\dy})
	-- ({10.1685*\dx},{0.0000*\dy})
	-- ({10.1785*\dx},{0.0000*\dy})
	-- ({10.1885*\dx},{0.0000*\dy})
	-- ({10.1985*\dx},{0.0000*\dy})
	-- ({10.2085*\dx},{0.0000*\dy})
	-- ({10.2185*\dx},{0.0000*\dy})
	-- ({10.2285*\dx},{0.0000*\dy})
	-- ({10.2385*\dx},{0.0000*\dy})
	-- ({10.2485*\dx},{0.0000*\dy})
	-- ({10.2585*\dx},{0.0000*\dy})
	-- ({10.2686*\dx},{0.0000*\dy})
	-- ({10.2786*\dx},{0.0000*\dy})
	-- ({10.2886*\dx},{0.0000*\dy})
	-- ({10.2986*\dx},{0.0000*\dy})
	-- ({10.3086*\dx},{0.0000*\dy})
	-- ({10.3186*\dx},{0.0000*\dy})
	-- ({10.3286*\dx},{0.0000*\dy})
	-- ({10.3386*\dx},{0.0000*\dy})
	-- ({10.3486*\dx},{0.0000*\dy})
	-- ({10.3586*\dx},{0.0000*\dy})
	-- ({10.3686*\dx},{0.0000*\dy})
	-- ({10.3786*\dx},{0.0000*\dy})
	-- ({10.3887*\dx},{0.0000*\dy})
	-- ({10.3987*\dx},{0.0000*\dy})
	-- ({10.4087*\dx},{0.0000*\dy})
	-- ({10.4187*\dx},{0.0000*\dy})
	-- ({10.4287*\dx},{0.0000*\dy})
	-- ({10.4387*\dx},{0.0000*\dy})
	-- ({10.4487*\dx},{0.0000*\dy})
	-- ({10.4587*\dx},{0.0000*\dy})
	-- ({10.4687*\dx},{0.0000*\dy})
	-- ({10.4787*\dx},{0.0000*\dy})
	-- ({10.4887*\dx},{0.0000*\dy})
	-- ({10.4987*\dx},{0.0000*\dy})
	-- ({10.5088*\dx},{0.0000*\dy})
	-- ({10.5188*\dx},{0.0000*\dy})
	-- ({10.5288*\dx},{0.0000*\dy})
	-- ({10.5388*\dx},{0.0000*\dy})
	-- ({10.5488*\dx},{0.0000*\dy})
	-- ({10.5588*\dx},{0.0000*\dy})
	-- ({10.5688*\dx},{0.0000*\dy})
	-- ({10.5788*\dx},{0.0000*\dy})
	-- ({10.5888*\dx},{0.0000*\dy})
	-- ({10.5988*\dx},{0.0000*\dy})
	-- ({10.6088*\dx},{0.0000*\dy})
	-- ({10.6188*\dx},{0.0000*\dy})
	-- ({10.6289*\dx},{0.0000*\dy})
	-- ({10.6389*\dx},{0.0000*\dy})
	-- ({10.6489*\dx},{0.0000*\dy})
	-- ({10.6589*\dx},{0.0000*\dy})
	-- ({10.6689*\dx},{0.0000*\dy})
	-- ({10.6789*\dx},{0.0000*\dy})
	-- ({10.6889*\dx},{0.0000*\dy})
	-- ({10.6989*\dx},{0.0000*\dy})
	-- ({10.7089*\dx},{0.0000*\dy})
	-- ({10.7189*\dx},{0.0000*\dy})
	-- ({10.7289*\dx},{0.0000*\dy})
	-- ({10.7389*\dx},{0.0000*\dy})
	-- ({10.7490*\dx},{0.0000*\dy})
	-- ({10.7590*\dx},{0.0000*\dy})
	-- ({10.7690*\dx},{0.0000*\dy})
	-- ({10.7790*\dx},{0.0000*\dy})
	-- ({10.7890*\dx},{0.0000*\dy})
	-- ({10.7990*\dx},{0.0000*\dy})
	-- ({10.8090*\dx},{0.0000*\dy})
	-- ({10.8190*\dx},{0.0000*\dy})
	-- ({10.8290*\dx},{0.0000*\dy})
	-- ({10.8390*\dx},{0.0000*\dy})
	-- ({10.8490*\dx},{0.0000*\dy})
	-- ({10.8590*\dx},{0.0000*\dy})
	-- ({10.8691*\dx},{0.0000*\dy})
	-- ({10.8791*\dx},{0.0000*\dy})
	-- ({10.8891*\dx},{0.0000*\dy})
	-- ({10.8991*\dx},{0.0000*\dy})
	-- ({10.9091*\dx},{0.0000*\dy})
	-- ({10.9191*\dx},{0.0000*\dy})
	-- ({10.9291*\dx},{0.0000*\dy})
	-- ({10.9391*\dx},{0.0000*\dy})
	-- ({10.9491*\dx},{0.0000*\dy})
	-- ({10.9591*\dx},{0.0000*\dy})
	-- ({10.9691*\dx},{0.0000*\dy})
	-- ({10.9791*\dx},{0.0000*\dy})
	-- ({10.9892*\dx},{0.0000*\dy})
	-- ({10.9992*\dx},{0.0000*\dy})
	-- ({11.0092*\dx},{0.0000*\dy})
	-- ({11.0192*\dx},{0.0000*\dy})
	-- ({11.0292*\dx},{0.0000*\dy})
	-- ({11.0392*\dx},{0.0000*\dy})
	-- ({11.0492*\dx},{0.0000*\dy})
	-- ({11.0592*\dx},{0.0000*\dy})
	-- ({11.0692*\dx},{0.0000*\dy})
	-- ({11.0792*\dx},{0.0000*\dy})
	-- ({11.0892*\dx},{0.0000*\dy})
	-- ({11.0992*\dx},{0.0000*\dy})
	-- ({11.1093*\dx},{0.0000*\dy})
	-- ({11.1193*\dx},{0.0000*\dy})
	-- ({11.1293*\dx},{0.0000*\dy})
	-- ({11.1393*\dx},{0.0000*\dy})
	-- ({11.1493*\dx},{0.0000*\dy})
	-- ({11.1593*\dx},{0.0000*\dy})
	-- ({11.1693*\dx},{0.0000*\dy})
	-- ({11.1793*\dx},{0.0000*\dy})
	-- ({11.1893*\dx},{0.0000*\dy})
	-- ({11.1993*\dx},{0.0000*\dy})
	-- ({11.2093*\dx},{0.0000*\dy})
	-- ({11.2193*\dx},{0.0000*\dy})
	-- ({11.2294*\dx},{0.0000*\dy})
	-- ({11.2394*\dx},{0.0000*\dy})
	-- ({11.2494*\dx},{0.0000*\dy})
	-- ({11.2594*\dx},{0.0000*\dy})
	-- ({11.2694*\dx},{0.0000*\dy})
	-- ({11.2794*\dx},{0.0000*\dy})
	-- ({11.2894*\dx},{0.0000*\dy})
	-- ({11.2994*\dx},{0.0000*\dy})
	-- ({11.3094*\dx},{0.0000*\dy})
	-- ({11.3194*\dx},{0.0000*\dy})
	-- ({11.3294*\dx},{0.0000*\dy})
	-- ({11.3394*\dx},{0.0000*\dy})
	-- ({11.3495*\dx},{0.0000*\dy})
	-- ({11.3595*\dx},{0.0000*\dy})
	-- ({11.3695*\dx},{0.0000*\dy})
	-- ({11.3795*\dx},{0.0000*\dy})
	-- ({11.3895*\dx},{0.0000*\dy})
	-- ({11.3995*\dx},{0.0000*\dy})
	-- ({11.4095*\dx},{0.0000*\dy})
	-- ({11.4195*\dx},{0.0000*\dy})
	-- ({11.4295*\dx},{0.0000*\dy})
	-- ({11.4395*\dx},{0.0000*\dy})
	-- ({11.4495*\dx},{0.0000*\dy})
	-- ({11.4595*\dx},{0.0000*\dy})
	-- ({11.4696*\dx},{0.0000*\dy})
	-- ({11.4796*\dx},{0.0000*\dy})
	-- ({11.4896*\dx},{0.0000*\dy})
	-- ({11.4996*\dx},{0.0000*\dy})
	-- ({11.5096*\dx},{0.0000*\dy})
	-- ({11.5196*\dx},{0.0000*\dy})
	-- ({11.5296*\dx},{0.0000*\dy})
	-- ({11.5396*\dx},{0.0000*\dy})
	-- ({11.5496*\dx},{0.0000*\dy})
	-- ({11.5596*\dx},{0.0000*\dy})
	-- ({11.5696*\dx},{0.0000*\dy})
	-- ({11.5796*\dx},{0.0000*\dy})
	-- ({11.5897*\dx},{0.0000*\dy})
	-- ({11.5997*\dx},{0.0000*\dy})
	-- ({11.6097*\dx},{0.0000*\dy})
	-- ({11.6197*\dx},{0.0000*\dy})
	-- ({11.6297*\dx},{0.0000*\dy})
	-- ({11.6397*\dx},{0.0000*\dy})
	-- ({11.6497*\dx},{0.0000*\dy})
	-- ({11.6597*\dx},{0.0000*\dy})
	-- ({11.6697*\dx},{0.0000*\dy})
	-- ({11.6797*\dx},{0.0000*\dy})
	-- ({11.6897*\dx},{0.0000*\dy})
	-- ({11.6997*\dx},{0.0000*\dy})
	-- ({11.7098*\dx},{0.0000*\dy})
	-- ({11.7198*\dx},{0.0000*\dy})
	-- ({11.7298*\dx},{0.0000*\dy})
	-- ({11.7398*\dx},{0.0000*\dy})
	-- ({11.7498*\dx},{0.0000*\dy})
	-- ({11.7598*\dx},{0.0000*\dy})
	-- ({11.7698*\dx},{0.0000*\dy})
	-- ({11.7798*\dx},{0.0000*\dy})
	-- ({11.7898*\dx},{0.0000*\dy})
	-- ({11.7998*\dx},{0.0000*\dy})
	-- ({11.8098*\dx},{0.0000*\dy})
	-- ({11.8198*\dx},{0.0000*\dy})
	-- ({11.8299*\dx},{0.0000*\dy})
	-- ({11.8399*\dx},{0.0000*\dy})
	-- ({11.8499*\dx},{0.0000*\dy})
	-- ({11.8599*\dx},{0.0000*\dy})
	-- ({11.8699*\dx},{0.0000*\dy})
	-- ({11.8799*\dx},{0.0000*\dy})
	-- ({11.8899*\dx},{0.0000*\dy})
	-- ({11.8999*\dx},{0.0000*\dy})
	-- ({11.9099*\dx},{0.0000*\dy})
	-- ({11.9199*\dx},{0.0000*\dy})
	-- ({11.9299*\dx},{0.0000*\dy})
	-- ({11.9399*\dx},{0.0000*\dy})
	-- ({11.9500*\dx},{0.0000*\dy})
	-- ({11.9600*\dx},{0.0000*\dy})
	-- ({11.9700*\dx},{0.0000*\dy})
	-- ({11.9800*\dx},{0.0000*\dy})
	-- ({11.9900*\dx},{0.0000*\dy})
	-- ({12.0000*\dx},{0.0000*\dy})
}
\def\psisix{
	({0.0000*\dx},{0.0000*\dy})
	-- ({0.0100*\dx},{0.0000*\dy})
	-- ({0.0200*\dx},{0.0000*\dy})
	-- ({0.0300*\dx},{0.0000*\dy})
	-- ({0.0400*\dx},{0.0000*\dy})
	-- ({0.0500*\dx},{0.0000*\dy})
	-- ({0.0601*\dx},{0.0000*\dy})
	-- ({0.0701*\dx},{0.0000*\dy})
	-- ({0.0801*\dx},{0.0000*\dy})
	-- ({0.0901*\dx},{0.0000*\dy})
	-- ({0.1001*\dx},{0.0000*\dy})
	-- ({0.1101*\dx},{0.0000*\dy})
	-- ({0.1201*\dx},{0.0000*\dy})
	-- ({0.1301*\dx},{0.0000*\dy})
	-- ({0.1401*\dx},{0.0000*\dy})
	-- ({0.1501*\dx},{0.0000*\dy})
	-- ({0.1601*\dx},{0.0000*\dy})
	-- ({0.1701*\dx},{0.0000*\dy})
	-- ({0.1802*\dx},{0.0000*\dy})
	-- ({0.1902*\dx},{0.0000*\dy})
	-- ({0.2002*\dx},{0.0000*\dy})
	-- ({0.2102*\dx},{0.0000*\dy})
	-- ({0.2202*\dx},{0.0000*\dy})
	-- ({0.2302*\dx},{0.0000*\dy})
	-- ({0.2402*\dx},{0.0000*\dy})
	-- ({0.2502*\dx},{0.0000*\dy})
	-- ({0.2602*\dx},{0.0000*\dy})
	-- ({0.2702*\dx},{0.0000*\dy})
	-- ({0.2802*\dx},{0.0000*\dy})
	-- ({0.2902*\dx},{0.0000*\dy})
	-- ({0.3003*\dx},{0.0000*\dy})
	-- ({0.3103*\dx},{0.0000*\dy})
	-- ({0.3203*\dx},{0.0000*\dy})
	-- ({0.3303*\dx},{0.0000*\dy})
	-- ({0.3403*\dx},{0.0000*\dy})
	-- ({0.3503*\dx},{0.0000*\dy})
	-- ({0.3603*\dx},{0.0000*\dy})
	-- ({0.3703*\dx},{0.0000*\dy})
	-- ({0.3803*\dx},{0.0000*\dy})
	-- ({0.3903*\dx},{0.0000*\dy})
	-- ({0.4003*\dx},{0.0000*\dy})
	-- ({0.4103*\dx},{0.0000*\dy})
	-- ({0.4204*\dx},{0.0000*\dy})
	-- ({0.4304*\dx},{0.0000*\dy})
	-- ({0.4404*\dx},{0.0000*\dy})
	-- ({0.4504*\dx},{0.0000*\dy})
	-- ({0.4604*\dx},{0.0000*\dy})
	-- ({0.4704*\dx},{0.0000*\dy})
	-- ({0.4804*\dx},{0.0000*\dy})
	-- ({0.4904*\dx},{0.0000*\dy})
	-- ({0.5004*\dx},{0.0000*\dy})
	-- ({0.5104*\dx},{0.0000*\dy})
	-- ({0.5204*\dx},{0.0000*\dy})
	-- ({0.5304*\dx},{0.0000*\dy})
	-- ({0.5405*\dx},{0.0000*\dy})
	-- ({0.5505*\dx},{0.0000*\dy})
	-- ({0.5605*\dx},{0.0000*\dy})
	-- ({0.5705*\dx},{0.0000*\dy})
	-- ({0.5805*\dx},{0.0000*\dy})
	-- ({0.5905*\dx},{0.0000*\dy})
	-- ({0.6005*\dx},{0.0000*\dy})
	-- ({0.6105*\dx},{0.0000*\dy})
	-- ({0.6205*\dx},{0.0000*\dy})
	-- ({0.6305*\dx},{0.0000*\dy})
	-- ({0.6405*\dx},{0.0000*\dy})
	-- ({0.6505*\dx},{0.0000*\dy})
	-- ({0.6606*\dx},{0.0000*\dy})
	-- ({0.6706*\dx},{0.0000*\dy})
	-- ({0.6806*\dx},{0.0000*\dy})
	-- ({0.6906*\dx},{0.0000*\dy})
	-- ({0.7006*\dx},{0.0000*\dy})
	-- ({0.7106*\dx},{0.0000*\dy})
	-- ({0.7206*\dx},{0.0000*\dy})
	-- ({0.7306*\dx},{0.0000*\dy})
	-- ({0.7406*\dx},{0.0000*\dy})
	-- ({0.7506*\dx},{0.0000*\dy})
	-- ({0.7606*\dx},{0.0000*\dy})
	-- ({0.7706*\dx},{0.0000*\dy})
	-- ({0.7807*\dx},{0.0000*\dy})
	-- ({0.7907*\dx},{0.0000*\dy})
	-- ({0.8007*\dx},{0.0000*\dy})
	-- ({0.8107*\dx},{0.0000*\dy})
	-- ({0.8207*\dx},{0.0000*\dy})
	-- ({0.8307*\dx},{0.0000*\dy})
	-- ({0.8407*\dx},{0.0000*\dy})
	-- ({0.8507*\dx},{0.0000*\dy})
	-- ({0.8607*\dx},{0.0000*\dy})
	-- ({0.8707*\dx},{0.0000*\dy})
	-- ({0.8807*\dx},{0.0000*\dy})
	-- ({0.8907*\dx},{0.0000*\dy})
	-- ({0.9008*\dx},{0.0000*\dy})
	-- ({0.9108*\dx},{0.0000*\dy})
	-- ({0.9208*\dx},{0.0000*\dy})
	-- ({0.9308*\dx},{0.0000*\dy})
	-- ({0.9408*\dx},{0.0000*\dy})
	-- ({0.9508*\dx},{0.0000*\dy})
	-- ({0.9608*\dx},{0.0000*\dy})
	-- ({0.9708*\dx},{0.0000*\dy})
	-- ({0.9808*\dx},{0.0000*\dy})
	-- ({0.9908*\dx},{0.0000*\dy})
	-- ({1.0008*\dx},{0.0000*\dy})
	-- ({1.0108*\dx},{0.0000*\dy})
	-- ({1.0209*\dx},{0.0000*\dy})
	-- ({1.0309*\dx},{0.0000*\dy})
	-- ({1.0409*\dx},{0.0000*\dy})
	-- ({1.0509*\dx},{0.0000*\dy})
	-- ({1.0609*\dx},{0.0000*\dy})
	-- ({1.0709*\dx},{0.0000*\dy})
	-- ({1.0809*\dx},{0.0000*\dy})
	-- ({1.0909*\dx},{0.0000*\dy})
	-- ({1.1009*\dx},{0.0000*\dy})
	-- ({1.1109*\dx},{0.0000*\dy})
	-- ({1.1209*\dx},{0.0000*\dy})
	-- ({1.1309*\dx},{0.0000*\dy})
	-- ({1.1410*\dx},{0.0000*\dy})
	-- ({1.1510*\dx},{0.0000*\dy})
	-- ({1.1610*\dx},{0.0000*\dy})
	-- ({1.1710*\dx},{0.0000*\dy})
	-- ({1.1810*\dx},{0.0000*\dy})
	-- ({1.1910*\dx},{0.0000*\dy})
	-- ({1.2010*\dx},{0.0000*\dy})
	-- ({1.2110*\dx},{0.0000*\dy})
	-- ({1.2210*\dx},{0.0000*\dy})
	-- ({1.2310*\dx},{0.0000*\dy})
	-- ({1.2410*\dx},{0.0000*\dy})
	-- ({1.2510*\dx},{0.0000*\dy})
	-- ({1.2611*\dx},{0.0000*\dy})
	-- ({1.2711*\dx},{0.0000*\dy})
	-- ({1.2811*\dx},{0.0000*\dy})
	-- ({1.2911*\dx},{0.0000*\dy})
	-- ({1.3011*\dx},{0.0000*\dy})
	-- ({1.3111*\dx},{0.0000*\dy})
	-- ({1.3211*\dx},{0.0000*\dy})
	-- ({1.3311*\dx},{0.0000*\dy})
	-- ({1.3411*\dx},{0.0000*\dy})
	-- ({1.3511*\dx},{0.0000*\dy})
	-- ({1.3611*\dx},{0.0000*\dy})
	-- ({1.3711*\dx},{0.0000*\dy})
	-- ({1.3812*\dx},{0.0000*\dy})
	-- ({1.3912*\dx},{0.0000*\dy})
	-- ({1.4012*\dx},{0.0000*\dy})
	-- ({1.4112*\dx},{0.0000*\dy})
	-- ({1.4212*\dx},{0.0000*\dy})
	-- ({1.4312*\dx},{0.0000*\dy})
	-- ({1.4412*\dx},{0.0000*\dy})
	-- ({1.4512*\dx},{0.0000*\dy})
	-- ({1.4612*\dx},{0.0000*\dy})
	-- ({1.4712*\dx},{0.0000*\dy})
	-- ({1.4812*\dx},{0.0000*\dy})
	-- ({1.4912*\dx},{0.0000*\dy})
	-- ({1.5013*\dx},{0.0000*\dy})
	-- ({1.5113*\dx},{0.0000*\dy})
	-- ({1.5213*\dx},{0.0000*\dy})
	-- ({1.5313*\dx},{0.0000*\dy})
	-- ({1.5413*\dx},{0.0000*\dy})
	-- ({1.5513*\dx},{0.0000*\dy})
	-- ({1.5613*\dx},{0.0000*\dy})
	-- ({1.5713*\dx},{0.0000*\dy})
	-- ({1.5813*\dx},{0.0000*\dy})
	-- ({1.5913*\dx},{0.0000*\dy})
	-- ({1.6013*\dx},{0.0000*\dy})
	-- ({1.6113*\dx},{0.0000*\dy})
	-- ({1.6214*\dx},{0.0000*\dy})
	-- ({1.6314*\dx},{0.0000*\dy})
	-- ({1.6414*\dx},{0.0000*\dy})
	-- ({1.6514*\dx},{0.0000*\dy})
	-- ({1.6614*\dx},{0.0000*\dy})
	-- ({1.6714*\dx},{0.0000*\dy})
	-- ({1.6814*\dx},{0.0000*\dy})
	-- ({1.6914*\dx},{0.0000*\dy})
	-- ({1.7014*\dx},{0.0000*\dy})
	-- ({1.7114*\dx},{0.0000*\dy})
	-- ({1.7214*\dx},{0.0000*\dy})
	-- ({1.7314*\dx},{0.0000*\dy})
	-- ({1.7415*\dx},{0.0000*\dy})
	-- ({1.7515*\dx},{0.0000*\dy})
	-- ({1.7615*\dx},{0.0000*\dy})
	-- ({1.7715*\dx},{0.0000*\dy})
	-- ({1.7815*\dx},{0.0000*\dy})
	-- ({1.7915*\dx},{0.0000*\dy})
	-- ({1.8015*\dx},{0.0000*\dy})
	-- ({1.8115*\dx},{0.0000*\dy})
	-- ({1.8215*\dx},{0.0000*\dy})
	-- ({1.8315*\dx},{0.0000*\dy})
	-- ({1.8415*\dx},{0.0000*\dy})
	-- ({1.8515*\dx},{0.0000*\dy})
	-- ({1.8616*\dx},{0.0000*\dy})
	-- ({1.8716*\dx},{0.0000*\dy})
	-- ({1.8816*\dx},{0.0000*\dy})
	-- ({1.8916*\dx},{0.0000*\dy})
	-- ({1.9016*\dx},{0.0000*\dy})
	-- ({1.9116*\dx},{0.0000*\dy})
	-- ({1.9216*\dx},{0.0000*\dy})
	-- ({1.9316*\dx},{0.0000*\dy})
	-- ({1.9416*\dx},{0.0000*\dy})
	-- ({1.9516*\dx},{0.0000*\dy})
	-- ({1.9616*\dx},{0.0000*\dy})
	-- ({1.9716*\dx},{0.0000*\dy})
	-- ({1.9817*\dx},{0.0000*\dy})
	-- ({1.9917*\dx},{0.0000*\dy})
	-- ({2.0017*\dx},{0.0000*\dy})
	-- ({2.0117*\dx},{0.0000*\dy})
	-- ({2.0217*\dx},{0.0000*\dy})
	-- ({2.0317*\dx},{0.0000*\dy})
	-- ({2.0417*\dx},{0.0000*\dy})
	-- ({2.0517*\dx},{0.0000*\dy})
	-- ({2.0617*\dx},{0.0000*\dy})
	-- ({2.0717*\dx},{0.0000*\dy})
	-- ({2.0817*\dx},{0.0000*\dy})
	-- ({2.0917*\dx},{0.0000*\dy})
	-- ({2.1018*\dx},{0.0000*\dy})
	-- ({2.1118*\dx},{0.0000*\dy})
	-- ({2.1218*\dx},{0.0000*\dy})
	-- ({2.1318*\dx},{0.0000*\dy})
	-- ({2.1418*\dx},{0.0000*\dy})
	-- ({2.1518*\dx},{0.0000*\dy})
	-- ({2.1618*\dx},{0.0000*\dy})
	-- ({2.1718*\dx},{0.0000*\dy})
	-- ({2.1818*\dx},{0.0000*\dy})
	-- ({2.1918*\dx},{0.0000*\dy})
	-- ({2.2018*\dx},{0.0000*\dy})
	-- ({2.2118*\dx},{0.0000*\dy})
	-- ({2.2219*\dx},{0.0000*\dy})
	-- ({2.2319*\dx},{0.0000*\dy})
	-- ({2.2419*\dx},{0.0000*\dy})
	-- ({2.2519*\dx},{0.0000*\dy})
	-- ({2.2619*\dx},{0.0000*\dy})
	-- ({2.2719*\dx},{0.0000*\dy})
	-- ({2.2819*\dx},{0.0000*\dy})
	-- ({2.2919*\dx},{0.0000*\dy})
	-- ({2.3019*\dx},{0.0000*\dy})
	-- ({2.3119*\dx},{0.0000*\dy})
	-- ({2.3219*\dx},{0.0000*\dy})
	-- ({2.3319*\dx},{0.0000*\dy})
	-- ({2.3420*\dx},{0.0000*\dy})
	-- ({2.3520*\dx},{0.0000*\dy})
	-- ({2.3620*\dx},{0.0000*\dy})
	-- ({2.3720*\dx},{0.0000*\dy})
	-- ({2.3820*\dx},{0.0000*\dy})
	-- ({2.3920*\dx},{0.0000*\dy})
	-- ({2.4020*\dx},{0.0000*\dy})
	-- ({2.4120*\dx},{0.0000*\dy})
	-- ({2.4220*\dx},{0.0000*\dy})
	-- ({2.4320*\dx},{0.0000*\dy})
	-- ({2.4420*\dx},{0.0000*\dy})
	-- ({2.4520*\dx},{0.0000*\dy})
	-- ({2.4621*\dx},{0.0000*\dy})
	-- ({2.4721*\dx},{0.0000*\dy})
	-- ({2.4821*\dx},{0.0000*\dy})
	-- ({2.4921*\dx},{0.0000*\dy})
	-- ({2.5021*\dx},{0.0000*\dy})
	-- ({2.5121*\dx},{0.0000*\dy})
	-- ({2.5221*\dx},{0.0000*\dy})
	-- ({2.5321*\dx},{0.0000*\dy})
	-- ({2.5421*\dx},{0.0000*\dy})
	-- ({2.5521*\dx},{0.0000*\dy})
	-- ({2.5621*\dx},{0.0000*\dy})
	-- ({2.5721*\dx},{0.0000*\dy})
	-- ({2.5822*\dx},{0.0000*\dy})
	-- ({2.5922*\dx},{0.0000*\dy})
	-- ({2.6022*\dx},{0.0000*\dy})
	-- ({2.6122*\dx},{0.0000*\dy})
	-- ({2.6222*\dx},{0.0000*\dy})
	-- ({2.6322*\dx},{0.0000*\dy})
	-- ({2.6422*\dx},{0.0000*\dy})
	-- ({2.6522*\dx},{0.0000*\dy})
	-- ({2.6622*\dx},{0.0000*\dy})
	-- ({2.6722*\dx},{0.0000*\dy})
	-- ({2.6822*\dx},{0.0000*\dy})
	-- ({2.6922*\dx},{0.0000*\dy})
	-- ({2.7023*\dx},{0.0000*\dy})
	-- ({2.7123*\dx},{0.0000*\dy})
	-- ({2.7223*\dx},{0.0000*\dy})
	-- ({2.7323*\dx},{0.0000*\dy})
	-- ({2.7423*\dx},{0.0000*\dy})
	-- ({2.7523*\dx},{0.0000*\dy})
	-- ({2.7623*\dx},{0.0000*\dy})
	-- ({2.7723*\dx},{0.0000*\dy})
	-- ({2.7823*\dx},{0.0000*\dy})
	-- ({2.7923*\dx},{0.0000*\dy})
	-- ({2.8023*\dx},{0.0000*\dy})
	-- ({2.8123*\dx},{0.0000*\dy})
	-- ({2.8224*\dx},{0.0000*\dy})
	-- ({2.8324*\dx},{0.0000*\dy})
	-- ({2.8424*\dx},{0.0000*\dy})
	-- ({2.8524*\dx},{0.0000*\dy})
	-- ({2.8624*\dx},{0.0000*\dy})
	-- ({2.8724*\dx},{0.0000*\dy})
	-- ({2.8824*\dx},{0.0000*\dy})
	-- ({2.8924*\dx},{0.0000*\dy})
	-- ({2.9024*\dx},{0.0000*\dy})
	-- ({2.9124*\dx},{0.0000*\dy})
	-- ({2.9224*\dx},{0.0000*\dy})
	-- ({2.9324*\dx},{0.0000*\dy})
	-- ({2.9425*\dx},{0.0000*\dy})
	-- ({2.9525*\dx},{0.0000*\dy})
	-- ({2.9625*\dx},{0.0000*\dy})
	-- ({2.9725*\dx},{0.0000*\dy})
	-- ({2.9825*\dx},{0.0000*\dy})
	-- ({2.9925*\dx},{0.0000*\dy})
	-- ({3.0025*\dx},{0.0000*\dy})
	-- ({3.0125*\dx},{0.0000*\dy})
	-- ({3.0225*\dx},{0.0000*\dy})
	-- ({3.0325*\dx},{0.0000*\dy})
	-- ({3.0425*\dx},{0.0000*\dy})
	-- ({3.0525*\dx},{0.0000*\dy})
	-- ({3.0626*\dx},{0.0000*\dy})
	-- ({3.0726*\dx},{0.0000*\dy})
	-- ({3.0826*\dx},{0.0000*\dy})
	-- ({3.0926*\dx},{0.0000*\dy})
	-- ({3.1026*\dx},{0.0000*\dy})
	-- ({3.1126*\dx},{0.0000*\dy})
	-- ({3.1226*\dx},{0.0000*\dy})
	-- ({3.1326*\dx},{0.0000*\dy})
	-- ({3.1426*\dx},{0.0000*\dy})
	-- ({3.1526*\dx},{0.0000*\dy})
	-- ({3.1626*\dx},{0.0000*\dy})
	-- ({3.1726*\dx},{0.0000*\dy})
	-- ({3.1827*\dx},{0.0000*\dy})
	-- ({3.1927*\dx},{0.0000*\dy})
	-- ({3.2027*\dx},{0.0000*\dy})
	-- ({3.2127*\dx},{0.0000*\dy})
	-- ({3.2227*\dx},{0.0000*\dy})
	-- ({3.2327*\dx},{0.0000*\dy})
	-- ({3.2427*\dx},{0.0000*\dy})
	-- ({3.2527*\dx},{0.0000*\dy})
	-- ({3.2627*\dx},{0.0000*\dy})
	-- ({3.2727*\dx},{0.0000*\dy})
	-- ({3.2827*\dx},{0.0000*\dy})
	-- ({3.2927*\dx},{0.0000*\dy})
	-- ({3.3028*\dx},{0.0000*\dy})
	-- ({3.3128*\dx},{0.0000*\dy})
	-- ({3.3228*\dx},{0.0000*\dy})
	-- ({3.3328*\dx},{0.0000*\dy})
	-- ({3.3428*\dx},{0.0000*\dy})
	-- ({3.3528*\dx},{0.0000*\dy})
	-- ({3.3628*\dx},{0.0000*\dy})
	-- ({3.3728*\dx},{0.0000*\dy})
	-- ({3.3828*\dx},{0.0000*\dy})
	-- ({3.3928*\dx},{0.0000*\dy})
	-- ({3.4028*\dx},{0.0000*\dy})
	-- ({3.4128*\dx},{0.0000*\dy})
	-- ({3.4229*\dx},{0.0000*\dy})
	-- ({3.4329*\dx},{0.0000*\dy})
	-- ({3.4429*\dx},{0.0000*\dy})
	-- ({3.4529*\dx},{0.0000*\dy})
	-- ({3.4629*\dx},{0.0000*\dy})
	-- ({3.4729*\dx},{0.0000*\dy})
	-- ({3.4829*\dx},{0.0000*\dy})
	-- ({3.4929*\dx},{0.0000*\dy})
	-- ({3.5029*\dx},{0.0000*\dy})
	-- ({3.5129*\dx},{0.0000*\dy})
	-- ({3.5229*\dx},{0.0000*\dy})
	-- ({3.5329*\dx},{0.0000*\dy})
	-- ({3.5430*\dx},{0.0000*\dy})
	-- ({3.5530*\dx},{0.0000*\dy})
	-- ({3.5630*\dx},{0.0000*\dy})
	-- ({3.5730*\dx},{0.0000*\dy})
	-- ({3.5830*\dx},{0.0000*\dy})
	-- ({3.5930*\dx},{0.0000*\dy})
	-- ({3.6030*\dx},{0.0000*\dy})
	-- ({3.6130*\dx},{0.0000*\dy})
	-- ({3.6230*\dx},{0.0000*\dy})
	-- ({3.6330*\dx},{0.0000*\dy})
	-- ({3.6430*\dx},{0.0000*\dy})
	-- ({3.6530*\dx},{0.0000*\dy})
	-- ({3.6631*\dx},{0.0000*\dy})
	-- ({3.6731*\dx},{0.0000*\dy})
	-- ({3.6831*\dx},{0.0000*\dy})
	-- ({3.6931*\dx},{0.0000*\dy})
	-- ({3.7031*\dx},{0.0000*\dy})
	-- ({3.7131*\dx},{0.0000*\dy})
	-- ({3.7231*\dx},{0.0000*\dy})
	-- ({3.7331*\dx},{0.0000*\dy})
	-- ({3.7431*\dx},{0.0000*\dy})
	-- ({3.7531*\dx},{0.0000*\dy})
	-- ({3.7631*\dx},{0.0000*\dy})
	-- ({3.7731*\dx},{0.0000*\dy})
	-- ({3.7832*\dx},{0.0000*\dy})
	-- ({3.7932*\dx},{0.0000*\dy})
	-- ({3.8032*\dx},{0.0000*\dy})
	-- ({3.8132*\dx},{0.0000*\dy})
	-- ({3.8232*\dx},{0.0000*\dy})
	-- ({3.8332*\dx},{0.0000*\dy})
	-- ({3.8432*\dx},{0.0000*\dy})
	-- ({3.8532*\dx},{0.0000*\dy})
	-- ({3.8632*\dx},{0.0000*\dy})
	-- ({3.8732*\dx},{0.0000*\dy})
	-- ({3.8832*\dx},{0.0000*\dy})
	-- ({3.8932*\dx},{0.0000*\dy})
	-- ({3.9033*\dx},{0.0000*\dy})
	-- ({3.9133*\dx},{0.0000*\dy})
	-- ({3.9233*\dx},{0.0000*\dy})
	-- ({3.9333*\dx},{0.0000*\dy})
	-- ({3.9433*\dx},{0.0000*\dy})
	-- ({3.9533*\dx},{0.0000*\dy})
	-- ({3.9633*\dx},{0.0000*\dy})
	-- ({3.9733*\dx},{0.0000*\dy})
	-- ({3.9833*\dx},{0.0000*\dy})
	-- ({3.9933*\dx},{0.0000*\dy})
	-- ({4.0033*\dx},{0.0000*\dy})
	-- ({4.0133*\dx},{0.0000*\dy})
	-- ({4.0234*\dx},{0.0000*\dy})
	-- ({4.0334*\dx},{0.0000*\dy})
	-- ({4.0434*\dx},{0.0000*\dy})
	-- ({4.0534*\dx},{0.0000*\dy})
	-- ({4.0634*\dx},{0.0000*\dy})
	-- ({4.0734*\dx},{0.0000*\dy})
	-- ({4.0834*\dx},{0.0000*\dy})
	-- ({4.0934*\dx},{0.0000*\dy})
	-- ({4.1034*\dx},{0.0000*\dy})
	-- ({4.1134*\dx},{0.0000*\dy})
	-- ({4.1234*\dx},{0.0000*\dy})
	-- ({4.1334*\dx},{0.0000*\dy})
	-- ({4.1435*\dx},{0.0000*\dy})
	-- ({4.1535*\dx},{0.0000*\dy})
	-- ({4.1635*\dx},{0.0000*\dy})
	-- ({4.1735*\dx},{0.0000*\dy})
	-- ({4.1835*\dx},{0.0000*\dy})
	-- ({4.1935*\dx},{0.0000*\dy})
	-- ({4.2035*\dx},{0.0000*\dy})
	-- ({4.2135*\dx},{0.0000*\dy})
	-- ({4.2235*\dx},{0.0000*\dy})
	-- ({4.2335*\dx},{0.0000*\dy})
	-- ({4.2435*\dx},{0.0000*\dy})
	-- ({4.2535*\dx},{0.0000*\dy})
	-- ({4.2636*\dx},{0.0000*\dy})
	-- ({4.2736*\dx},{0.0000*\dy})
	-- ({4.2836*\dx},{0.0000*\dy})
	-- ({4.2936*\dx},{0.0000*\dy})
	-- ({4.3036*\dx},{0.0000*\dy})
	-- ({4.3136*\dx},{0.0000*\dy})
	-- ({4.3236*\dx},{0.0000*\dy})
	-- ({4.3336*\dx},{0.0000*\dy})
	-- ({4.3436*\dx},{0.0000*\dy})
	-- ({4.3536*\dx},{0.0000*\dy})
	-- ({4.3636*\dx},{0.0000*\dy})
	-- ({4.3736*\dx},{0.0000*\dy})
	-- ({4.3837*\dx},{0.0000*\dy})
	-- ({4.3937*\dx},{0.0000*\dy})
	-- ({4.4037*\dx},{0.0000*\dy})
	-- ({4.4137*\dx},{0.0000*\dy})
	-- ({4.4237*\dx},{0.0000*\dy})
	-- ({4.4337*\dx},{0.0000*\dy})
	-- ({4.4437*\dx},{0.0000*\dy})
	-- ({4.4537*\dx},{0.0000*\dy})
	-- ({4.4637*\dx},{0.0000*\dy})
	-- ({4.4737*\dx},{0.0000*\dy})
	-- ({4.4837*\dx},{0.0000*\dy})
	-- ({4.4937*\dx},{0.0000*\dy})
	-- ({4.5038*\dx},{0.0000*\dy})
	-- ({4.5138*\dx},{0.0000*\dy})
	-- ({4.5238*\dx},{0.0000*\dy})
	-- ({4.5338*\dx},{0.0000*\dy})
	-- ({4.5438*\dx},{0.0000*\dy})
	-- ({4.5538*\dx},{0.0000*\dy})
	-- ({4.5638*\dx},{0.0000*\dy})
	-- ({4.5738*\dx},{0.0000*\dy})
	-- ({4.5838*\dx},{0.0000*\dy})
	-- ({4.5938*\dx},{0.0000*\dy})
	-- ({4.6038*\dx},{0.0000*\dy})
	-- ({4.6138*\dx},{0.0000*\dy})
	-- ({4.6239*\dx},{0.0000*\dy})
	-- ({4.6339*\dx},{0.0000*\dy})
	-- ({4.6439*\dx},{0.0000*\dy})
	-- ({4.6539*\dx},{0.0000*\dy})
	-- ({4.6639*\dx},{0.0000*\dy})
	-- ({4.6739*\dx},{0.0000*\dy})
	-- ({4.6839*\dx},{0.0000*\dy})
	-- ({4.6939*\dx},{0.0000*\dy})
	-- ({4.7039*\dx},{0.0000*\dy})
	-- ({4.7139*\dx},{0.0000*\dy})
	-- ({4.7239*\dx},{0.0000*\dy})
	-- ({4.7339*\dx},{0.0000*\dy})
	-- ({4.7440*\dx},{0.0000*\dy})
	-- ({4.7540*\dx},{0.0000*\dy})
	-- ({4.7640*\dx},{0.0000*\dy})
	-- ({4.7740*\dx},{0.0000*\dy})
	-- ({4.7840*\dx},{0.0000*\dy})
	-- ({4.7940*\dx},{0.0000*\dy})
	-- ({4.8040*\dx},{0.0000*\dy})
	-- ({4.8140*\dx},{0.0000*\dy})
	-- ({4.8240*\dx},{0.0000*\dy})
	-- ({4.8340*\dx},{0.0000*\dy})
	-- ({4.8440*\dx},{0.0000*\dy})
	-- ({4.8540*\dx},{0.0000*\dy})
	-- ({4.8641*\dx},{0.0000*\dy})
	-- ({4.8741*\dx},{0.0000*\dy})
	-- ({4.8841*\dx},{0.0000*\dy})
	-- ({4.8941*\dx},{0.0000*\dy})
	-- ({4.9041*\dx},{0.0000*\dy})
	-- ({4.9141*\dx},{0.0000*\dy})
	-- ({4.9241*\dx},{0.0000*\dy})
	-- ({4.9341*\dx},{0.0000*\dy})
	-- ({4.9441*\dx},{0.0000*\dy})
	-- ({4.9541*\dx},{0.0000*\dy})
	-- ({4.9641*\dx},{0.0000*\dy})
	-- ({4.9741*\dx},{0.0000*\dy})
	-- ({4.9842*\dx},{0.0000*\dy})
	-- ({4.9942*\dx},{0.0000*\dy})
	-- ({5.0042*\dx},{0.0000*\dy})
	-- ({5.0142*\dx},{0.0000*\dy})
	-- ({5.0242*\dx},{0.0000*\dy})
	-- ({5.0342*\dx},{0.0000*\dy})
	-- ({5.0442*\dx},{0.0000*\dy})
	-- ({5.0542*\dx},{0.0000*\dy})
	-- ({5.0642*\dx},{0.0000*\dy})
	-- ({5.0742*\dx},{0.0000*\dy})
	-- ({5.0842*\dx},{0.0000*\dy})
	-- ({5.0942*\dx},{0.0000*\dy})
	-- ({5.1043*\dx},{0.0000*\dy})
	-- ({5.1143*\dx},{0.0000*\dy})
	-- ({5.1243*\dx},{0.0000*\dy})
	-- ({5.1343*\dx},{0.0000*\dy})
	-- ({5.1443*\dx},{0.0000*\dy})
	-- ({5.1543*\dx},{0.0000*\dy})
	-- ({5.1643*\dx},{0.0000*\dy})
	-- ({5.1743*\dx},{0.0000*\dy})
	-- ({5.1843*\dx},{0.0000*\dy})
	-- ({5.1943*\dx},{0.0000*\dy})
	-- ({5.2043*\dx},{0.0000*\dy})
	-- ({5.2143*\dx},{0.0000*\dy})
	-- ({5.2244*\dx},{0.0000*\dy})
	-- ({5.2344*\dx},{0.0000*\dy})
	-- ({5.2444*\dx},{0.0000*\dy})
	-- ({5.2544*\dx},{0.0000*\dy})
	-- ({5.2644*\dx},{0.0000*\dy})
	-- ({5.2744*\dx},{0.0000*\dy})
	-- ({5.2844*\dx},{0.0000*\dy})
	-- ({5.2944*\dx},{0.0000*\dy})
	-- ({5.3044*\dx},{0.0000*\dy})
	-- ({5.3144*\dx},{0.0000*\dy})
	-- ({5.3244*\dx},{0.0000*\dy})
	-- ({5.3344*\dx},{0.0000*\dy})
	-- ({5.3445*\dx},{0.0000*\dy})
	-- ({5.3545*\dx},{0.0000*\dy})
	-- ({5.3645*\dx},{0.0000*\dy})
	-- ({5.3745*\dx},{0.0000*\dy})
	-- ({5.3845*\dx},{0.0000*\dy})
	-- ({5.3945*\dx},{0.0000*\dy})
	-- ({5.4045*\dx},{0.0000*\dy})
	-- ({5.4145*\dx},{0.0000*\dy})
	-- ({5.4245*\dx},{0.0000*\dy})
	-- ({5.4345*\dx},{0.0000*\dy})
	-- ({5.4445*\dx},{0.0000*\dy})
	-- ({5.4545*\dx},{0.0000*\dy})
	-- ({5.4646*\dx},{0.0000*\dy})
	-- ({5.4746*\dx},{0.0000*\dy})
	-- ({5.4846*\dx},{0.0000*\dy})
	-- ({5.4946*\dx},{0.0000*\dy})
	-- ({5.5046*\dx},{0.0000*\dy})
	-- ({5.5146*\dx},{0.0000*\dy})
	-- ({5.5246*\dx},{0.0000*\dy})
	-- ({5.5346*\dx},{0.0000*\dy})
	-- ({5.5446*\dx},{0.0000*\dy})
	-- ({5.5546*\dx},{0.0000*\dy})
	-- ({5.5646*\dx},{0.0000*\dy})
	-- ({5.5746*\dx},{0.0000*\dy})
	-- ({5.5847*\dx},{0.0000*\dy})
	-- ({5.5947*\dx},{0.0000*\dy})
	-- ({5.6047*\dx},{0.0000*\dy})
	-- ({5.6147*\dx},{0.0000*\dy})
	-- ({5.6247*\dx},{0.0000*\dy})
	-- ({5.6347*\dx},{0.0000*\dy})
	-- ({5.6447*\dx},{0.0000*\dy})
	-- ({5.6547*\dx},{0.0000*\dy})
	-- ({5.6647*\dx},{0.0000*\dy})
	-- ({5.6747*\dx},{0.0000*\dy})
	-- ({5.6847*\dx},{0.0000*\dy})
	-- ({5.6947*\dx},{0.0000*\dy})
	-- ({5.7048*\dx},{0.0000*\dy})
	-- ({5.7148*\dx},{0.0000*\dy})
	-- ({5.7248*\dx},{0.0000*\dy})
	-- ({5.7348*\dx},{0.0000*\dy})
	-- ({5.7448*\dx},{0.0000*\dy})
	-- ({5.7548*\dx},{0.0000*\dy})
	-- ({5.7648*\dx},{0.0000*\dy})
	-- ({5.7748*\dx},{0.0000*\dy})
	-- ({5.7848*\dx},{0.0000*\dy})
	-- ({5.7948*\dx},{0.0000*\dy})
	-- ({5.8048*\dx},{0.0000*\dy})
	-- ({5.8148*\dx},{0.0000*\dy})
	-- ({5.8249*\dx},{0.0000*\dy})
	-- ({5.8349*\dx},{0.0000*\dy})
	-- ({5.8449*\dx},{0.0000*\dy})
	-- ({5.8549*\dx},{0.0000*\dy})
	-- ({5.8649*\dx},{0.0000*\dy})
	-- ({5.8749*\dx},{0.0000*\dy})
	-- ({5.8849*\dx},{0.0000*\dy})
	-- ({5.8949*\dx},{0.0000*\dy})
	-- ({5.9049*\dx},{0.0000*\dy})
	-- ({5.9149*\dx},{0.0000*\dy})
	-- ({5.9249*\dx},{0.0000*\dy})
	-- ({5.9349*\dx},{0.0000*\dy})
	-- ({5.9450*\dx},{0.0000*\dy})
	-- ({5.9550*\dx},{0.0000*\dy})
	-- ({5.9650*\dx},{0.0000*\dy})
	-- ({5.9750*\dx},{0.0000*\dy})
	-- ({5.9850*\dx},{0.0000*\dy})
	-- ({5.9950*\dx},{0.0000*\dy})
	-- ({6.0050*\dx},{0.0000*\dy})
	-- ({6.0150*\dx},{0.0000*\dy})
	-- ({6.0250*\dx},{0.0000*\dy})
	-- ({6.0350*\dx},{0.0000*\dy})
	-- ({6.0450*\dx},{0.0000*\dy})
	-- ({6.0550*\dx},{0.0000*\dy})
	-- ({6.0651*\dx},{0.0000*\dy})
	-- ({6.0751*\dx},{0.0000*\dy})
	-- ({6.0851*\dx},{0.0000*\dy})
	-- ({6.0951*\dx},{0.0000*\dy})
	-- ({6.1051*\dx},{0.0000*\dy})
	-- ({6.1151*\dx},{0.0000*\dy})
	-- ({6.1251*\dx},{0.0000*\dy})
	-- ({6.1351*\dx},{0.0000*\dy})
	-- ({6.1451*\dx},{0.0000*\dy})
	-- ({6.1551*\dx},{0.0000*\dy})
	-- ({6.1651*\dx},{0.0000*\dy})
	-- ({6.1751*\dx},{0.0000*\dy})
	-- ({6.1852*\dx},{0.0000*\dy})
	-- ({6.1952*\dx},{0.0000*\dy})
	-- ({6.2052*\dx},{0.0000*\dy})
	-- ({6.2152*\dx},{0.0000*\dy})
	-- ({6.2252*\dx},{0.0000*\dy})
	-- ({6.2352*\dx},{0.0000*\dy})
	-- ({6.2452*\dx},{0.0000*\dy})
	-- ({6.2552*\dx},{0.0000*\dy})
	-- ({6.2652*\dx},{0.0000*\dy})
	-- ({6.2752*\dx},{0.0000*\dy})
	-- ({6.2852*\dx},{0.0000*\dy})
	-- ({6.2952*\dx},{0.0000*\dy})
	-- ({6.3053*\dx},{0.0000*\dy})
	-- ({6.3153*\dx},{0.0000*\dy})
	-- ({6.3253*\dx},{0.0000*\dy})
	-- ({6.3353*\dx},{0.0000*\dy})
	-- ({6.3453*\dx},{0.0000*\dy})
	-- ({6.3553*\dx},{0.0000*\dy})
	-- ({6.3653*\dx},{0.0000*\dy})
	-- ({6.3753*\dx},{0.0000*\dy})
	-- ({6.3853*\dx},{0.0000*\dy})
	-- ({6.3953*\dx},{0.0000*\dy})
	-- ({6.4053*\dx},{0.0000*\dy})
	-- ({6.4153*\dx},{0.0000*\dy})
	-- ({6.4254*\dx},{0.0000*\dy})
	-- ({6.4354*\dx},{0.0000*\dy})
	-- ({6.4454*\dx},{0.0000*\dy})
	-- ({6.4554*\dx},{0.0000*\dy})
	-- ({6.4654*\dx},{0.0000*\dy})
	-- ({6.4754*\dx},{0.0000*\dy})
	-- ({6.4854*\dx},{0.0000*\dy})
	-- ({6.4954*\dx},{0.0000*\dy})
	-- ({6.5054*\dx},{0.0000*\dy})
	-- ({6.5154*\dx},{0.0000*\dy})
	-- ({6.5254*\dx},{0.0000*\dy})
	-- ({6.5354*\dx},{0.0000*\dy})
	-- ({6.5455*\dx},{0.0000*\dy})
	-- ({6.5555*\dx},{0.0000*\dy})
	-- ({6.5655*\dx},{0.0000*\dy})
	-- ({6.5755*\dx},{0.0000*\dy})
	-- ({6.5855*\dx},{0.0000*\dy})
	-- ({6.5955*\dx},{0.0000*\dy})
	-- ({6.6055*\dx},{0.0000*\dy})
	-- ({6.6155*\dx},{0.0000*\dy})
	-- ({6.6255*\dx},{0.0000*\dy})
	-- ({6.6355*\dx},{0.0000*\dy})
	-- ({6.6455*\dx},{0.0000*\dy})
	-- ({6.6555*\dx},{0.0000*\dy})
	-- ({6.6656*\dx},{0.0000*\dy})
	-- ({6.6756*\dx},{0.0000*\dy})
	-- ({6.6856*\dx},{0.0000*\dy})
	-- ({6.6956*\dx},{0.0000*\dy})
	-- ({6.7056*\dx},{0.0000*\dy})
	-- ({6.7156*\dx},{0.0000*\dy})
	-- ({6.7256*\dx},{0.0000*\dy})
	-- ({6.7356*\dx},{0.0000*\dy})
	-- ({6.7456*\dx},{0.0000*\dy})
	-- ({6.7556*\dx},{0.0000*\dy})
	-- ({6.7656*\dx},{0.0000*\dy})
	-- ({6.7756*\dx},{0.0000*\dy})
	-- ({6.7857*\dx},{0.0000*\dy})
	-- ({6.7957*\dx},{0.0000*\dy})
	-- ({6.8057*\dx},{0.0000*\dy})
	-- ({6.8157*\dx},{0.0000*\dy})
	-- ({6.8257*\dx},{0.0000*\dy})
	-- ({6.8357*\dx},{0.0000*\dy})
	-- ({6.8457*\dx},{0.0000*\dy})
	-- ({6.8557*\dx},{0.0000*\dy})
	-- ({6.8657*\dx},{0.0000*\dy})
	-- ({6.8757*\dx},{0.0000*\dy})
	-- ({6.8857*\dx},{0.0000*\dy})
	-- ({6.8957*\dx},{0.0000*\dy})
	-- ({6.9058*\dx},{0.0000*\dy})
	-- ({6.9158*\dx},{0.0000*\dy})
	-- ({6.9258*\dx},{0.0000*\dy})
	-- ({6.9358*\dx},{0.0000*\dy})
	-- ({6.9458*\dx},{0.0000*\dy})
	-- ({6.9558*\dx},{0.0000*\dy})
	-- ({6.9658*\dx},{0.0000*\dy})
	-- ({6.9758*\dx},{0.0000*\dy})
	-- ({6.9858*\dx},{0.0000*\dy})
	-- ({6.9958*\dx},{0.0000*\dy})
	-- ({7.0058*\dx},{0.0000*\dy})
	-- ({7.0158*\dx},{0.0000*\dy})
	-- ({7.0259*\dx},{0.0000*\dy})
	-- ({7.0359*\dx},{0.0000*\dy})
	-- ({7.0459*\dx},{0.0000*\dy})
	-- ({7.0559*\dx},{0.0000*\dy})
	-- ({7.0659*\dx},{0.0000*\dy})
	-- ({7.0759*\dx},{0.0000*\dy})
	-- ({7.0859*\dx},{0.0000*\dy})
	-- ({7.0959*\dx},{0.0000*\dy})
	-- ({7.1059*\dx},{0.0000*\dy})
	-- ({7.1159*\dx},{0.0000*\dy})
	-- ({7.1259*\dx},{0.0000*\dy})
	-- ({7.1359*\dx},{0.0000*\dy})
	-- ({7.1460*\dx},{0.0000*\dy})
	-- ({7.1560*\dx},{0.0000*\dy})
	-- ({7.1660*\dx},{0.0000*\dy})
	-- ({7.1760*\dx},{0.0000*\dy})
	-- ({7.1860*\dx},{0.0000*\dy})
	-- ({7.1960*\dx},{0.0000*\dy})
	-- ({7.2060*\dx},{0.0000*\dy})
	-- ({7.2160*\dx},{0.0000*\dy})
	-- ({7.2260*\dx},{0.0000*\dy})
	-- ({7.2360*\dx},{0.0000*\dy})
	-- ({7.2460*\dx},{0.0000*\dy})
	-- ({7.2560*\dx},{0.0000*\dy})
	-- ({7.2661*\dx},{0.0000*\dy})
	-- ({7.2761*\dx},{0.0000*\dy})
	-- ({7.2861*\dx},{0.0000*\dy})
	-- ({7.2961*\dx},{0.0000*\dy})
	-- ({7.3061*\dx},{0.0000*\dy})
	-- ({7.3161*\dx},{0.0000*\dy})
	-- ({7.3261*\dx},{0.0000*\dy})
	-- ({7.3361*\dx},{0.0000*\dy})
	-- ({7.3461*\dx},{0.0000*\dy})
	-- ({7.3561*\dx},{0.0000*\dy})
	-- ({7.3661*\dx},{0.0000*\dy})
	-- ({7.3761*\dx},{0.0000*\dy})
	-- ({7.3862*\dx},{0.0000*\dy})
	-- ({7.3962*\dx},{0.0000*\dy})
	-- ({7.4062*\dx},{0.0000*\dy})
	-- ({7.4162*\dx},{0.0000*\dy})
	-- ({7.4262*\dx},{0.0000*\dy})
	-- ({7.4362*\dx},{0.0000*\dy})
	-- ({7.4462*\dx},{0.0000*\dy})
	-- ({7.4562*\dx},{0.0000*\dy})
	-- ({7.4662*\dx},{0.0000*\dy})
	-- ({7.4762*\dx},{0.0000*\dy})
	-- ({7.4862*\dx},{0.0000*\dy})
	-- ({7.4962*\dx},{0.0000*\dy})
	-- ({7.5063*\dx},{0.0000*\dy})
	-- ({7.5163*\dx},{0.0000*\dy})
	-- ({7.5263*\dx},{0.0000*\dy})
	-- ({7.5363*\dx},{0.0000*\dy})
	-- ({7.5463*\dx},{0.0000*\dy})
	-- ({7.5563*\dx},{0.0000*\dy})
	-- ({7.5663*\dx},{0.0000*\dy})
	-- ({7.5763*\dx},{0.0000*\dy})
	-- ({7.5863*\dx},{0.0000*\dy})
	-- ({7.5963*\dx},{0.0000*\dy})
	-- ({7.6063*\dx},{0.0000*\dy})
	-- ({7.6163*\dx},{0.0000*\dy})
	-- ({7.6264*\dx},{0.0000*\dy})
	-- ({7.6364*\dx},{0.0000*\dy})
	-- ({7.6464*\dx},{0.0000*\dy})
	-- ({7.6564*\dx},{0.0000*\dy})
	-- ({7.6664*\dx},{0.0000*\dy})
	-- ({7.6764*\dx},{0.0000*\dy})
	-- ({7.6864*\dx},{0.0000*\dy})
	-- ({7.6964*\dx},{0.0000*\dy})
	-- ({7.7064*\dx},{0.0000*\dy})
	-- ({7.7164*\dx},{0.0000*\dy})
	-- ({7.7264*\dx},{0.0000*\dy})
	-- ({7.7364*\dx},{0.0000*\dy})
	-- ({7.7465*\dx},{0.0000*\dy})
	-- ({7.7565*\dx},{0.0000*\dy})
	-- ({7.7665*\dx},{0.0000*\dy})
	-- ({7.7765*\dx},{0.0000*\dy})
	-- ({7.7865*\dx},{0.0000*\dy})
	-- ({7.7965*\dx},{0.0000*\dy})
	-- ({7.8065*\dx},{0.0000*\dy})
	-- ({7.8165*\dx},{0.0000*\dy})
	-- ({7.8265*\dx},{0.0000*\dy})
	-- ({7.8365*\dx},{0.0000*\dy})
	-- ({7.8465*\dx},{0.0000*\dy})
	-- ({7.8565*\dx},{0.0000*\dy})
	-- ({7.8666*\dx},{0.0000*\dy})
	-- ({7.8766*\dx},{0.0000*\dy})
	-- ({7.8866*\dx},{0.0000*\dy})
	-- ({7.8966*\dx},{0.0000*\dy})
	-- ({7.9066*\dx},{0.0000*\dy})
	-- ({7.9166*\dx},{0.0000*\dy})
	-- ({7.9266*\dx},{0.0000*\dy})
	-- ({7.9366*\dx},{0.0000*\dy})
	-- ({7.9466*\dx},{0.0000*\dy})
	-- ({7.9566*\dx},{0.0000*\dy})
	-- ({7.9666*\dx},{0.0000*\dy})
	-- ({7.9766*\dx},{0.0000*\dy})
	-- ({7.9867*\dx},{0.0000*\dy})
	-- ({7.9967*\dx},{0.0000*\dy})
	-- ({8.0067*\dx},{0.0000*\dy})
	-- ({8.0167*\dx},{0.0003*\dy})
	-- ({8.0267*\dx},{0.0007*\dy})
	-- ({8.0367*\dx},{0.0013*\dy})
	-- ({8.0467*\dx},{0.0021*\dy})
	-- ({8.0567*\dx},{0.0032*\dy})
	-- ({8.0667*\dx},{0.0044*\dy})
	-- ({8.0767*\dx},{0.0059*\dy})
	-- ({8.0867*\dx},{0.0075*\dy})
	-- ({8.0967*\dx},{0.0094*\dy})
	-- ({8.1068*\dx},{0.0116*\dy})
	-- ({8.1168*\dx},{0.0139*\dy})
	-- ({8.1268*\dx},{0.0164*\dy})
	-- ({8.1368*\dx},{0.0192*\dy})
	-- ({8.1468*\dx},{0.0222*\dy})
	-- ({8.1568*\dx},{0.0255*\dy})
	-- ({8.1668*\dx},{0.0290*\dy})
	-- ({8.1768*\dx},{0.0326*\dy})
	-- ({8.1868*\dx},{0.0366*\dy})
	-- ({8.1968*\dx},{0.0407*\dy})
	-- ({8.2068*\dx},{0.0451*\dy})
	-- ({8.2168*\dx},{0.0497*\dy})
	-- ({8.2269*\dx},{0.0545*\dy})
	-- ({8.2369*\dx},{0.0596*\dy})
	-- ({8.2469*\dx},{0.0648*\dy})
	-- ({8.2569*\dx},{0.0703*\dy})
	-- ({8.2669*\dx},{0.0760*\dy})
	-- ({8.2769*\dx},{0.0819*\dy})
	-- ({8.2869*\dx},{0.0880*\dy})
	-- ({8.2969*\dx},{0.0943*\dy})
	-- ({8.3069*\dx},{0.1008*\dy})
	-- ({8.3169*\dx},{0.1075*\dy})
	-- ({8.3269*\dx},{0.1144*\dy})
	-- ({8.3369*\dx},{0.1215*\dy})
	-- ({8.3470*\dx},{0.1287*\dy})
	-- ({8.3570*\dx},{0.1361*\dy})
	-- ({8.3670*\dx},{0.1437*\dy})
	-- ({8.3770*\dx},{0.1515*\dy})
	-- ({8.3870*\dx},{0.1594*\dy})
	-- ({8.3970*\dx},{0.1674*\dy})
	-- ({8.4070*\dx},{0.1756*\dy})
	-- ({8.4170*\dx},{0.1840*\dy})
	-- ({8.4270*\dx},{0.1924*\dy})
	-- ({8.4370*\dx},{0.2010*\dy})
	-- ({8.4470*\dx},{0.2097*\dy})
	-- ({8.4570*\dx},{0.2185*\dy})
	-- ({8.4671*\dx},{0.2274*\dy})
	-- ({8.4771*\dx},{0.2364*\dy})
	-- ({8.4871*\dx},{0.2455*\dy})
	-- ({8.4971*\dx},{0.2546*\dy})
	-- ({8.5071*\dx},{0.2639*\dy})
	-- ({8.5171*\dx},{0.2732*\dy})
	-- ({8.5271*\dx},{0.2825*\dy})
	-- ({8.5371*\dx},{0.2919*\dy})
	-- ({8.5471*\dx},{0.3014*\dy})
	-- ({8.5571*\dx},{0.3108*\dy})
	-- ({8.5671*\dx},{0.3204*\dy})
	-- ({8.5771*\dx},{0.3299*\dy})
	-- ({8.5872*\dx},{0.3394*\dy})
	-- ({8.5972*\dx},{0.3490*\dy})
	-- ({8.6072*\dx},{0.3585*\dy})
	-- ({8.6172*\dx},{0.3681*\dy})
	-- ({8.6272*\dx},{0.3776*\dy})
	-- ({8.6372*\dx},{0.3871*\dy})
	-- ({8.6472*\dx},{0.3966*\dy})
	-- ({8.6572*\dx},{0.4061*\dy})
	-- ({8.6672*\dx},{0.4155*\dy})
	-- ({8.6772*\dx},{0.4249*\dy})
	-- ({8.6872*\dx},{0.4343*\dy})
	-- ({8.6972*\dx},{0.4436*\dy})
	-- ({8.7073*\dx},{0.4528*\dy})
	-- ({8.7173*\dx},{0.4620*\dy})
	-- ({8.7273*\dx},{0.4711*\dy})
	-- ({8.7373*\dx},{0.4802*\dy})
	-- ({8.7473*\dx},{0.4892*\dy})
	-- ({8.7573*\dx},{0.4981*\dy})
	-- ({8.7673*\dx},{0.5070*\dy})
	-- ({8.7773*\dx},{0.5157*\dy})
	-- ({8.7873*\dx},{0.5244*\dy})
	-- ({8.7973*\dx},{0.5330*\dy})
	-- ({8.8073*\dx},{0.5415*\dy})
	-- ({8.8173*\dx},{0.5499*\dy})
	-- ({8.8274*\dx},{0.5582*\dy})
	-- ({8.8374*\dx},{0.5664*\dy})
	-- ({8.8474*\dx},{0.5745*\dy})
	-- ({8.8574*\dx},{0.5825*\dy})
	-- ({8.8674*\dx},{0.5904*\dy})
	-- ({8.8774*\dx},{0.5982*\dy})
	-- ({8.8874*\dx},{0.6059*\dy})
	-- ({8.8974*\dx},{0.6135*\dy})
	-- ({8.9074*\dx},{0.6209*\dy})
	-- ({8.9174*\dx},{0.6283*\dy})
	-- ({8.9274*\dx},{0.6355*\dy})
	-- ({8.9374*\dx},{0.6427*\dy})
	-- ({8.9475*\dx},{0.6497*\dy})
	-- ({8.9575*\dx},{0.6566*\dy})
	-- ({8.9675*\dx},{0.6634*\dy})
	-- ({8.9775*\dx},{0.6700*\dy})
	-- ({8.9875*\dx},{0.6766*\dy})
	-- ({8.9975*\dx},{0.6830*\dy})
	-- ({9.0075*\dx},{0.6894*\dy})
	-- ({9.0175*\dx},{0.6956*\dy})
	-- ({9.0275*\dx},{0.7016*\dy})
	-- ({9.0375*\dx},{0.7076*\dy})
	-- ({9.0475*\dx},{0.7135*\dy})
	-- ({9.0575*\dx},{0.7192*\dy})
	-- ({9.0676*\dx},{0.7248*\dy})
	-- ({9.0776*\dx},{0.7303*\dy})
	-- ({9.0876*\dx},{0.7357*\dy})
	-- ({9.0976*\dx},{0.7410*\dy})
	-- ({9.1076*\dx},{0.7461*\dy})
	-- ({9.1176*\dx},{0.7512*\dy})
	-- ({9.1276*\dx},{0.7561*\dy})
	-- ({9.1376*\dx},{0.7609*\dy})
	-- ({9.1476*\dx},{0.7656*\dy})
	-- ({9.1576*\dx},{0.7701*\dy})
	-- ({9.1676*\dx},{0.7746*\dy})
	-- ({9.1776*\dx},{0.7789*\dy})
	-- ({9.1877*\dx},{0.7832*\dy})
	-- ({9.1977*\dx},{0.7873*\dy})
	-- ({9.2077*\dx},{0.7913*\dy})
	-- ({9.2177*\dx},{0.7952*\dy})
	-- ({9.2277*\dx},{0.7990*\dy})
	-- ({9.2377*\dx},{0.8026*\dy})
	-- ({9.2477*\dx},{0.8062*\dy})
	-- ({9.2577*\dx},{0.8096*\dy})
	-- ({9.2677*\dx},{0.8130*\dy})
	-- ({9.2777*\dx},{0.8162*\dy})
	-- ({9.2877*\dx},{0.8193*\dy})
	-- ({9.2977*\dx},{0.8223*\dy})
	-- ({9.3078*\dx},{0.8252*\dy})
	-- ({9.3178*\dx},{0.8280*\dy})
	-- ({9.3278*\dx},{0.8307*\dy})
	-- ({9.3378*\dx},{0.8333*\dy})
	-- ({9.3478*\dx},{0.8358*\dy})
	-- ({9.3578*\dx},{0.8381*\dy})
	-- ({9.3678*\dx},{0.8404*\dy})
	-- ({9.3778*\dx},{0.8426*\dy})
	-- ({9.3878*\dx},{0.8446*\dy})
	-- ({9.3978*\dx},{0.8465*\dy})
	-- ({9.4078*\dx},{0.8484*\dy})
	-- ({9.4178*\dx},{0.8501*\dy})
	-- ({9.4279*\dx},{0.8517*\dy})
	-- ({9.4379*\dx},{0.8532*\dy})
	-- ({9.4479*\dx},{0.8546*\dy})
	-- ({9.4579*\dx},{0.8559*\dy})
	-- ({9.4679*\dx},{0.8571*\dy})
	-- ({9.4779*\dx},{0.8582*\dy})
	-- ({9.4879*\dx},{0.8592*\dy})
	-- ({9.4979*\dx},{0.8601*\dy})
	-- ({9.5079*\dx},{0.8608*\dy})
	-- ({9.5179*\dx},{0.8615*\dy})
	-- ({9.5279*\dx},{0.8620*\dy})
	-- ({9.5379*\dx},{0.8625*\dy})
	-- ({9.5480*\dx},{0.8628*\dy})
	-- ({9.5580*\dx},{0.8630*\dy})
	-- ({9.5680*\dx},{0.8631*\dy})
	-- ({9.5780*\dx},{0.8631*\dy})
	-- ({9.5880*\dx},{0.8630*\dy})
	-- ({9.5980*\dx},{0.8628*\dy})
	-- ({9.6080*\dx},{0.8625*\dy})
	-- ({9.6180*\dx},{0.8620*\dy})
	-- ({9.6280*\dx},{0.8615*\dy})
	-- ({9.6380*\dx},{0.8608*\dy})
	-- ({9.6480*\dx},{0.8600*\dy})
	-- ({9.6580*\dx},{0.8591*\dy})
	-- ({9.6681*\dx},{0.8581*\dy})
	-- ({9.6781*\dx},{0.8569*\dy})
	-- ({9.6881*\dx},{0.8557*\dy})
	-- ({9.6981*\dx},{0.8543*\dy})
	-- ({9.7081*\dx},{0.8528*\dy})
	-- ({9.7181*\dx},{0.8512*\dy})
	-- ({9.7281*\dx},{0.8494*\dy})
	-- ({9.7381*\dx},{0.8476*\dy})
	-- ({9.7481*\dx},{0.8456*\dy})
	-- ({9.7581*\dx},{0.8434*\dy})
	-- ({9.7681*\dx},{0.8412*\dy})
	-- ({9.7781*\dx},{0.8388*\dy})
	-- ({9.7882*\dx},{0.8363*\dy})
	-- ({9.7982*\dx},{0.8336*\dy})
	-- ({9.8082*\dx},{0.8309*\dy})
	-- ({9.8182*\dx},{0.8280*\dy})
	-- ({9.8282*\dx},{0.8249*\dy})
	-- ({9.8382*\dx},{0.8217*\dy})
	-- ({9.8482*\dx},{0.8184*\dy})
	-- ({9.8582*\dx},{0.8149*\dy})
	-- ({9.8682*\dx},{0.8113*\dy})
	-- ({9.8782*\dx},{0.8076*\dy})
	-- ({9.8882*\dx},{0.8037*\dy})
	-- ({9.8982*\dx},{0.7996*\dy})
	-- ({9.9083*\dx},{0.7955*\dy})
	-- ({9.9183*\dx},{0.7911*\dy})
	-- ({9.9283*\dx},{0.7866*\dy})
	-- ({9.9383*\dx},{0.7820*\dy})
	-- ({9.9483*\dx},{0.7772*\dy})
	-- ({9.9583*\dx},{0.7723*\dy})
	-- ({9.9683*\dx},{0.7672*\dy})
	-- ({9.9783*\dx},{0.7619*\dy})
	-- ({9.9883*\dx},{0.7565*\dy})
	-- ({9.9983*\dx},{0.7509*\dy})
	-- ({10.0083*\dx},{0.7452*\dy})
	-- ({10.0183*\dx},{0.7394*\dy})
	-- ({10.0284*\dx},{0.7335*\dy})
	-- ({10.0384*\dx},{0.7275*\dy})
	-- ({10.0484*\dx},{0.7214*\dy})
	-- ({10.0584*\dx},{0.7152*\dy})
	-- ({10.0684*\dx},{0.7088*\dy})
	-- ({10.0784*\dx},{0.7024*\dy})
	-- ({10.0884*\dx},{0.6959*\dy})
	-- ({10.0984*\dx},{0.6892*\dy})
	-- ({10.1084*\dx},{0.6824*\dy})
	-- ({10.1184*\dx},{0.6755*\dy})
	-- ({10.1284*\dx},{0.6685*\dy})
	-- ({10.1384*\dx},{0.6614*\dy})
	-- ({10.1485*\dx},{0.6542*\dy})
	-- ({10.1585*\dx},{0.6469*\dy})
	-- ({10.1685*\dx},{0.6394*\dy})
	-- ({10.1785*\dx},{0.6318*\dy})
	-- ({10.1885*\dx},{0.6241*\dy})
	-- ({10.1985*\dx},{0.6163*\dy})
	-- ({10.2085*\dx},{0.6084*\dy})
	-- ({10.2185*\dx},{0.6004*\dy})
	-- ({10.2285*\dx},{0.5922*\dy})
	-- ({10.2385*\dx},{0.5839*\dy})
	-- ({10.2485*\dx},{0.5756*\dy})
	-- ({10.2585*\dx},{0.5671*\dy})
	-- ({10.2686*\dx},{0.5584*\dy})
	-- ({10.2786*\dx},{0.5497*\dy})
	-- ({10.2886*\dx},{0.5409*\dy})
	-- ({10.2986*\dx},{0.5319*\dy})
	-- ({10.3086*\dx},{0.5228*\dy})
	-- ({10.3186*\dx},{0.5137*\dy})
	-- ({10.3286*\dx},{0.5044*\dy})
	-- ({10.3386*\dx},{0.4950*\dy})
	-- ({10.3486*\dx},{0.4856*\dy})
	-- ({10.3586*\dx},{0.4760*\dy})
	-- ({10.3686*\dx},{0.4664*\dy})
	-- ({10.3786*\dx},{0.4566*\dy})
	-- ({10.3887*\dx},{0.4468*\dy})
	-- ({10.3987*\dx},{0.4369*\dy})
	-- ({10.4087*\dx},{0.4269*\dy})
	-- ({10.4187*\dx},{0.4168*\dy})
	-- ({10.4287*\dx},{0.4067*\dy})
	-- ({10.4387*\dx},{0.3966*\dy})
	-- ({10.4487*\dx},{0.3863*\dy})
	-- ({10.4587*\dx},{0.3761*\dy})
	-- ({10.4687*\dx},{0.3657*\dy})
	-- ({10.4787*\dx},{0.3554*\dy})
	-- ({10.4887*\dx},{0.3450*\dy})
	-- ({10.4987*\dx},{0.3346*\dy})
	-- ({10.5088*\dx},{0.3242*\dy})
	-- ({10.5188*\dx},{0.3138*\dy})
	-- ({10.5288*\dx},{0.3034*\dy})
	-- ({10.5388*\dx},{0.2931*\dy})
	-- ({10.5488*\dx},{0.2827*\dy})
	-- ({10.5588*\dx},{0.2724*\dy})
	-- ({10.5688*\dx},{0.2621*\dy})
	-- ({10.5788*\dx},{0.2519*\dy})
	-- ({10.5888*\dx},{0.2418*\dy})
	-- ({10.5988*\dx},{0.2317*\dy})
	-- ({10.6088*\dx},{0.2217*\dy})
	-- ({10.6188*\dx},{0.2118*\dy})
	-- ({10.6289*\dx},{0.2020*\dy})
	-- ({10.6389*\dx},{0.1923*\dy})
	-- ({10.6489*\dx},{0.1828*\dy})
	-- ({10.6589*\dx},{0.1734*\dy})
	-- ({10.6689*\dx},{0.1642*\dy})
	-- ({10.6789*\dx},{0.1551*\dy})
	-- ({10.6889*\dx},{0.1462*\dy})
	-- ({10.6989*\dx},{0.1375*\dy})
	-- ({10.7089*\dx},{0.1289*\dy})
	-- ({10.7189*\dx},{0.1206*\dy})
	-- ({10.7289*\dx},{0.1125*\dy})
	-- ({10.7389*\dx},{0.1046*\dy})
	-- ({10.7490*\dx},{0.0970*\dy})
	-- ({10.7590*\dx},{0.0896*\dy})
	-- ({10.7690*\dx},{0.0824*\dy})
	-- ({10.7790*\dx},{0.0755*\dy})
	-- ({10.7890*\dx},{0.0689*\dy})
	-- ({10.7990*\dx},{0.0626*\dy})
	-- ({10.8090*\dx},{0.0565*\dy})
	-- ({10.8190*\dx},{0.0507*\dy})
	-- ({10.8290*\dx},{0.0453*\dy})
	-- ({10.8390*\dx},{0.0401*\dy})
	-- ({10.8490*\dx},{0.0352*\dy})
	-- ({10.8590*\dx},{0.0307*\dy})
	-- ({10.8691*\dx},{0.0264*\dy})
	-- ({10.8791*\dx},{0.0225*\dy})
	-- ({10.8891*\dx},{0.0189*\dy})
	-- ({10.8991*\dx},{0.0156*\dy})
	-- ({10.9091*\dx},{0.0126*\dy})
	-- ({10.9191*\dx},{0.0099*\dy})
	-- ({10.9291*\dx},{0.0076*\dy})
	-- ({10.9391*\dx},{0.0056*\dy})
	-- ({10.9491*\dx},{0.0039*\dy})
	-- ({10.9591*\dx},{0.0025*\dy})
	-- ({10.9691*\dx},{0.0014*\dy})
	-- ({10.9791*\dx},{0.0006*\dy})
	-- ({10.9892*\dx},{0.0002*\dy})
	-- ({10.9992*\dx},{0.0000*\dy})
	-- ({11.0092*\dx},{0.0000*\dy})
	-- ({11.0192*\dx},{0.0000*\dy})
	-- ({11.0292*\dx},{0.0000*\dy})
	-- ({11.0392*\dx},{0.0000*\dy})
	-- ({11.0492*\dx},{0.0000*\dy})
	-- ({11.0592*\dx},{0.0000*\dy})
	-- ({11.0692*\dx},{0.0000*\dy})
	-- ({11.0792*\dx},{0.0000*\dy})
	-- ({11.0892*\dx},{0.0000*\dy})
	-- ({11.0992*\dx},{0.0000*\dy})
	-- ({11.1093*\dx},{0.0000*\dy})
	-- ({11.1193*\dx},{0.0000*\dy})
	-- ({11.1293*\dx},{0.0000*\dy})
	-- ({11.1393*\dx},{0.0000*\dy})
	-- ({11.1493*\dx},{0.0000*\dy})
	-- ({11.1593*\dx},{0.0000*\dy})
	-- ({11.1693*\dx},{0.0000*\dy})
	-- ({11.1793*\dx},{0.0000*\dy})
	-- ({11.1893*\dx},{0.0000*\dy})
	-- ({11.1993*\dx},{0.0000*\dy})
	-- ({11.2093*\dx},{0.0000*\dy})
	-- ({11.2193*\dx},{0.0000*\dy})
	-- ({11.2294*\dx},{0.0000*\dy})
	-- ({11.2394*\dx},{0.0000*\dy})
	-- ({11.2494*\dx},{0.0000*\dy})
	-- ({11.2594*\dx},{0.0000*\dy})
	-- ({11.2694*\dx},{0.0000*\dy})
	-- ({11.2794*\dx},{0.0000*\dy})
	-- ({11.2894*\dx},{0.0000*\dy})
	-- ({11.2994*\dx},{0.0000*\dy})
	-- ({11.3094*\dx},{0.0000*\dy})
	-- ({11.3194*\dx},{0.0000*\dy})
	-- ({11.3294*\dx},{0.0000*\dy})
	-- ({11.3394*\dx},{0.0000*\dy})
	-- ({11.3495*\dx},{0.0000*\dy})
	-- ({11.3595*\dx},{0.0000*\dy})
	-- ({11.3695*\dx},{0.0000*\dy})
	-- ({11.3795*\dx},{0.0000*\dy})
	-- ({11.3895*\dx},{0.0000*\dy})
	-- ({11.3995*\dx},{0.0000*\dy})
	-- ({11.4095*\dx},{0.0000*\dy})
	-- ({11.4195*\dx},{0.0000*\dy})
	-- ({11.4295*\dx},{0.0000*\dy})
	-- ({11.4395*\dx},{0.0000*\dy})
	-- ({11.4495*\dx},{0.0000*\dy})
	-- ({11.4595*\dx},{0.0000*\dy})
	-- ({11.4696*\dx},{0.0000*\dy})
	-- ({11.4796*\dx},{0.0000*\dy})
	-- ({11.4896*\dx},{0.0000*\dy})
	-- ({11.4996*\dx},{0.0000*\dy})
	-- ({11.5096*\dx},{0.0000*\dy})
	-- ({11.5196*\dx},{0.0000*\dy})
	-- ({11.5296*\dx},{0.0000*\dy})
	-- ({11.5396*\dx},{0.0000*\dy})
	-- ({11.5496*\dx},{0.0000*\dy})
	-- ({11.5596*\dx},{0.0000*\dy})
	-- ({11.5696*\dx},{0.0000*\dy})
	-- ({11.5796*\dx},{0.0000*\dy})
	-- ({11.5897*\dx},{0.0000*\dy})
	-- ({11.5997*\dx},{0.0000*\dy})
	-- ({11.6097*\dx},{0.0000*\dy})
	-- ({11.6197*\dx},{0.0000*\dy})
	-- ({11.6297*\dx},{0.0000*\dy})
	-- ({11.6397*\dx},{0.0000*\dy})
	-- ({11.6497*\dx},{0.0000*\dy})
	-- ({11.6597*\dx},{0.0000*\dy})
	-- ({11.6697*\dx},{0.0000*\dy})
	-- ({11.6797*\dx},{0.0000*\dy})
	-- ({11.6897*\dx},{0.0000*\dy})
	-- ({11.6997*\dx},{0.0000*\dy})
	-- ({11.7098*\dx},{0.0000*\dy})
	-- ({11.7198*\dx},{0.0000*\dy})
	-- ({11.7298*\dx},{0.0000*\dy})
	-- ({11.7398*\dx},{0.0000*\dy})
	-- ({11.7498*\dx},{0.0000*\dy})
	-- ({11.7598*\dx},{0.0000*\dy})
	-- ({11.7698*\dx},{0.0000*\dy})
	-- ({11.7798*\dx},{0.0000*\dy})
	-- ({11.7898*\dx},{0.0000*\dy})
	-- ({11.7998*\dx},{0.0000*\dy})
	-- ({11.8098*\dx},{0.0000*\dy})
	-- ({11.8198*\dx},{0.0000*\dy})
	-- ({11.8299*\dx},{0.0000*\dy})
	-- ({11.8399*\dx},{0.0000*\dy})
	-- ({11.8499*\dx},{0.0000*\dy})
	-- ({11.8599*\dx},{0.0000*\dy})
	-- ({11.8699*\dx},{0.0000*\dy})
	-- ({11.8799*\dx},{0.0000*\dy})
	-- ({11.8899*\dx},{0.0000*\dy})
	-- ({11.8999*\dx},{0.0000*\dy})
	-- ({11.9099*\dx},{0.0000*\dy})
	-- ({11.9199*\dx},{0.0000*\dy})
	-- ({11.9299*\dx},{0.0000*\dy})
	-- ({11.9399*\dx},{0.0000*\dy})
	-- ({11.9500*\dx},{0.0000*\dy})
	-- ({11.9600*\dx},{0.0000*\dy})
	-- ({11.9700*\dx},{0.0000*\dy})
	-- ({11.9800*\dx},{0.0000*\dy})
	-- ({11.9900*\dx},{0.0000*\dy})
	-- ({12.0000*\dx},{0.0000*\dy})
}
\def\cpsione{
	({0.0000*\dx},{1.0000*\dy})
	-- ({0.0100*\dx},{1.0000*\dy})
	-- ({0.0200*\dx},{1.0000*\dy})
	-- ({0.0300*\dx},{1.0000*\dy})
	-- ({0.0400*\dx},{1.0000*\dy})
	-- ({0.0500*\dx},{1.0000*\dy})
	-- ({0.0601*\dx},{1.0000*\dy})
	-- ({0.0701*\dx},{1.0000*\dy})
	-- ({0.0801*\dx},{1.0000*\dy})
	-- ({0.0901*\dx},{1.0000*\dy})
	-- ({0.1001*\dx},{1.0000*\dy})
	-- ({0.1101*\dx},{1.0000*\dy})
	-- ({0.1201*\dx},{1.0000*\dy})
	-- ({0.1301*\dx},{1.0000*\dy})
	-- ({0.1401*\dx},{1.0000*\dy})
	-- ({0.1501*\dx},{1.0000*\dy})
	-- ({0.1601*\dx},{1.0000*\dy})
	-- ({0.1701*\dx},{1.0000*\dy})
	-- ({0.1802*\dx},{1.0000*\dy})
	-- ({0.1902*\dx},{1.0000*\dy})
	-- ({0.2002*\dx},{1.0000*\dy})
	-- ({0.2102*\dx},{1.0000*\dy})
	-- ({0.2202*\dx},{1.0000*\dy})
	-- ({0.2302*\dx},{1.0000*\dy})
	-- ({0.2402*\dx},{1.0000*\dy})
	-- ({0.2502*\dx},{1.0000*\dy})
	-- ({0.2602*\dx},{1.0000*\dy})
	-- ({0.2702*\dx},{1.0000*\dy})
	-- ({0.2802*\dx},{1.0000*\dy})
	-- ({0.2902*\dx},{1.0000*\dy})
	-- ({0.3003*\dx},{1.0000*\dy})
	-- ({0.3103*\dx},{1.0000*\dy})
	-- ({0.3203*\dx},{1.0000*\dy})
	-- ({0.3303*\dx},{1.0000*\dy})
	-- ({0.3403*\dx},{1.0000*\dy})
	-- ({0.3503*\dx},{1.0000*\dy})
	-- ({0.3603*\dx},{1.0000*\dy})
	-- ({0.3703*\dx},{1.0000*\dy})
	-- ({0.3803*\dx},{1.0000*\dy})
	-- ({0.3903*\dx},{1.0000*\dy})
	-- ({0.4003*\dx},{1.0000*\dy})
	-- ({0.4103*\dx},{1.0000*\dy})
	-- ({0.4204*\dx},{1.0000*\dy})
	-- ({0.4304*\dx},{1.0000*\dy})
	-- ({0.4404*\dx},{1.0000*\dy})
	-- ({0.4504*\dx},{1.0000*\dy})
	-- ({0.4604*\dx},{1.0000*\dy})
	-- ({0.4704*\dx},{1.0000*\dy})
	-- ({0.4804*\dx},{1.0000*\dy})
	-- ({0.4904*\dx},{1.0000*\dy})
	-- ({0.5004*\dx},{1.0000*\dy})
	-- ({0.5104*\dx},{1.0000*\dy})
	-- ({0.5204*\dx},{1.0000*\dy})
	-- ({0.5304*\dx},{1.0000*\dy})
	-- ({0.5405*\dx},{1.0000*\dy})
	-- ({0.5505*\dx},{1.0000*\dy})
	-- ({0.5605*\dx},{1.0000*\dy})
	-- ({0.5705*\dx},{1.0000*\dy})
	-- ({0.5805*\dx},{1.0000*\dy})
	-- ({0.5905*\dx},{1.0000*\dy})
	-- ({0.6005*\dx},{1.0000*\dy})
	-- ({0.6105*\dx},{1.0000*\dy})
	-- ({0.6205*\dx},{1.0000*\dy})
	-- ({0.6305*\dx},{1.0000*\dy})
	-- ({0.6405*\dx},{1.0000*\dy})
	-- ({0.6505*\dx},{1.0000*\dy})
	-- ({0.6606*\dx},{1.0000*\dy})
	-- ({0.6706*\dx},{1.0000*\dy})
	-- ({0.6806*\dx},{1.0000*\dy})
	-- ({0.6906*\dx},{1.0000*\dy})
	-- ({0.7006*\dx},{1.0000*\dy})
	-- ({0.7106*\dx},{1.0000*\dy})
	-- ({0.7206*\dx},{1.0000*\dy})
	-- ({0.7306*\dx},{1.0000*\dy})
	-- ({0.7406*\dx},{1.0000*\dy})
	-- ({0.7506*\dx},{1.0000*\dy})
	-- ({0.7606*\dx},{1.0000*\dy})
	-- ({0.7706*\dx},{1.0000*\dy})
	-- ({0.7807*\dx},{1.0000*\dy})
	-- ({0.7907*\dx},{1.0000*\dy})
	-- ({0.8007*\dx},{1.0000*\dy})
	-- ({0.8107*\dx},{1.0000*\dy})
	-- ({0.8207*\dx},{1.0000*\dy})
	-- ({0.8307*\dx},{1.0000*\dy})
	-- ({0.8407*\dx},{1.0000*\dy})
	-- ({0.8507*\dx},{1.0000*\dy})
	-- ({0.8607*\dx},{1.0000*\dy})
	-- ({0.8707*\dx},{1.0000*\dy})
	-- ({0.8807*\dx},{1.0000*\dy})
	-- ({0.8907*\dx},{1.0000*\dy})
	-- ({0.9008*\dx},{1.0000*\dy})
	-- ({0.9108*\dx},{1.0000*\dy})
	-- ({0.9208*\dx},{1.0000*\dy})
	-- ({0.9308*\dx},{1.0000*\dy})
	-- ({0.9408*\dx},{1.0000*\dy})
	-- ({0.9508*\dx},{1.0000*\dy})
	-- ({0.9608*\dx},{1.0000*\dy})
	-- ({0.9708*\dx},{1.0000*\dy})
	-- ({0.9808*\dx},{1.0000*\dy})
	-- ({0.9908*\dx},{1.0000*\dy})
	-- ({1.0008*\dx},{1.0000*\dy})
	-- ({1.0108*\dx},{1.0000*\dy})
	-- ({1.0209*\dx},{1.0000*\dy})
	-- ({1.0309*\dx},{1.0000*\dy})
	-- ({1.0409*\dx},{1.0000*\dy})
	-- ({1.0509*\dx},{1.0000*\dy})
	-- ({1.0609*\dx},{1.0000*\dy})
	-- ({1.0709*\dx},{1.0000*\dy})
	-- ({1.0809*\dx},{1.0000*\dy})
	-- ({1.0909*\dx},{1.0000*\dy})
	-- ({1.1009*\dx},{1.0000*\dy})
	-- ({1.1109*\dx},{1.0000*\dy})
	-- ({1.1209*\dx},{1.0000*\dy})
	-- ({1.1309*\dx},{1.0000*\dy})
	-- ({1.1410*\dx},{1.0000*\dy})
	-- ({1.1510*\dx},{1.0000*\dy})
	-- ({1.1610*\dx},{1.0000*\dy})
	-- ({1.1710*\dx},{1.0000*\dy})
	-- ({1.1810*\dx},{1.0000*\dy})
	-- ({1.1910*\dx},{1.0000*\dy})
	-- ({1.2010*\dx},{1.0000*\dy})
	-- ({1.2110*\dx},{1.0000*\dy})
	-- ({1.2210*\dx},{1.0000*\dy})
	-- ({1.2310*\dx},{1.0000*\dy})
	-- ({1.2410*\dx},{1.0000*\dy})
	-- ({1.2510*\dx},{1.0000*\dy})
	-- ({1.2611*\dx},{1.0000*\dy})
	-- ({1.2711*\dx},{1.0000*\dy})
	-- ({1.2811*\dx},{1.0000*\dy})
	-- ({1.2911*\dx},{1.0000*\dy})
	-- ({1.3011*\dx},{1.0000*\dy})
	-- ({1.3111*\dx},{1.0000*\dy})
	-- ({1.3211*\dx},{1.0000*\dy})
	-- ({1.3311*\dx},{1.0000*\dy})
	-- ({1.3411*\dx},{1.0000*\dy})
	-- ({1.3511*\dx},{1.0000*\dy})
	-- ({1.3611*\dx},{1.0000*\dy})
	-- ({1.3711*\dx},{1.0000*\dy})
	-- ({1.3812*\dx},{1.0000*\dy})
	-- ({1.3912*\dx},{1.0000*\dy})
	-- ({1.4012*\dx},{1.0000*\dy})
	-- ({1.4112*\dx},{1.0000*\dy})
	-- ({1.4212*\dx},{1.0000*\dy})
	-- ({1.4312*\dx},{1.0000*\dy})
	-- ({1.4412*\dx},{1.0000*\dy})
	-- ({1.4512*\dx},{1.0000*\dy})
	-- ({1.4612*\dx},{1.0000*\dy})
	-- ({1.4712*\dx},{1.0000*\dy})
	-- ({1.4812*\dx},{1.0000*\dy})
	-- ({1.4912*\dx},{1.0000*\dy})
	-- ({1.5013*\dx},{1.0000*\dy})
	-- ({1.5113*\dx},{1.0000*\dy})
	-- ({1.5213*\dx},{1.0000*\dy})
	-- ({1.5313*\dx},{1.0000*\dy})
	-- ({1.5413*\dx},{1.0000*\dy})
	-- ({1.5513*\dx},{1.0000*\dy})
	-- ({1.5613*\dx},{1.0000*\dy})
	-- ({1.5713*\dx},{1.0000*\dy})
	-- ({1.5813*\dx},{1.0000*\dy})
	-- ({1.5913*\dx},{1.0000*\dy})
	-- ({1.6013*\dx},{1.0000*\dy})
	-- ({1.6113*\dx},{1.0000*\dy})
	-- ({1.6214*\dx},{1.0000*\dy})
	-- ({1.6314*\dx},{1.0000*\dy})
	-- ({1.6414*\dx},{1.0000*\dy})
	-- ({1.6514*\dx},{1.0000*\dy})
	-- ({1.6614*\dx},{1.0000*\dy})
	-- ({1.6714*\dx},{1.0000*\dy})
	-- ({1.6814*\dx},{1.0000*\dy})
	-- ({1.6914*\dx},{1.0000*\dy})
	-- ({1.7014*\dx},{1.0000*\dy})
	-- ({1.7114*\dx},{1.0000*\dy})
	-- ({1.7214*\dx},{1.0000*\dy})
	-- ({1.7314*\dx},{1.0000*\dy})
	-- ({1.7415*\dx},{1.0000*\dy})
	-- ({1.7515*\dx},{1.0000*\dy})
	-- ({1.7615*\dx},{1.0000*\dy})
	-- ({1.7715*\dx},{1.0000*\dy})
	-- ({1.7815*\dx},{1.0000*\dy})
	-- ({1.7915*\dx},{1.0000*\dy})
	-- ({1.8015*\dx},{1.0000*\dy})
	-- ({1.8115*\dx},{1.0000*\dy})
	-- ({1.8215*\dx},{1.0000*\dy})
	-- ({1.8315*\dx},{1.0000*\dy})
	-- ({1.8415*\dx},{1.0000*\dy})
	-- ({1.8515*\dx},{1.0000*\dy})
	-- ({1.8616*\dx},{1.0000*\dy})
	-- ({1.8716*\dx},{1.0000*\dy})
	-- ({1.8816*\dx},{1.0000*\dy})
	-- ({1.8916*\dx},{1.0000*\dy})
	-- ({1.9016*\dx},{1.0000*\dy})
	-- ({1.9116*\dx},{1.0000*\dy})
	-- ({1.9216*\dx},{1.0000*\dy})
	-- ({1.9316*\dx},{1.0000*\dy})
	-- ({1.9416*\dx},{1.0000*\dy})
	-- ({1.9516*\dx},{1.0000*\dy})
	-- ({1.9616*\dx},{1.0000*\dy})
	-- ({1.9716*\dx},{1.0000*\dy})
	-- ({1.9817*\dx},{1.0000*\dy})
	-- ({1.9917*\dx},{1.0000*\dy})
	-- ({2.0017*\dx},{1.0000*\dy})
	-- ({2.0117*\dx},{0.9997*\dy})
	-- ({2.0217*\dx},{0.9988*\dy})
	-- ({2.0317*\dx},{0.9974*\dy})
	-- ({2.0417*\dx},{0.9954*\dy})
	-- ({2.0517*\dx},{0.9928*\dy})
	-- ({2.0617*\dx},{0.9896*\dy})
	-- ({2.0717*\dx},{0.9857*\dy})
	-- ({2.0817*\dx},{0.9811*\dy})
	-- ({2.0917*\dx},{0.9759*\dy})
	-- ({2.1018*\dx},{0.9699*\dy})
	-- ({2.1118*\dx},{0.9632*\dy})
	-- ({2.1218*\dx},{0.9558*\dy})
	-- ({2.1318*\dx},{0.9475*\dy})
	-- ({2.1418*\dx},{0.9386*\dy})
	-- ({2.1518*\dx},{0.9288*\dy})
	-- ({2.1618*\dx},{0.9183*\dy})
	-- ({2.1718*\dx},{0.9070*\dy})
	-- ({2.1818*\dx},{0.8950*\dy})
	-- ({2.1918*\dx},{0.8822*\dy})
	-- ({2.2018*\dx},{0.8686*\dy})
	-- ({2.2118*\dx},{0.8543*\dy})
	-- ({2.2219*\dx},{0.8393*\dy})
	-- ({2.2319*\dx},{0.8237*\dy})
	-- ({2.2419*\dx},{0.8074*\dy})
	-- ({2.2519*\dx},{0.7905*\dy})
	-- ({2.2619*\dx},{0.7730*\dy})
	-- ({2.2719*\dx},{0.7550*\dy})
	-- ({2.2819*\dx},{0.7366*\dy})
	-- ({2.2919*\dx},{0.7177*\dy})
	-- ({2.3019*\dx},{0.6984*\dy})
	-- ({2.3119*\dx},{0.6787*\dy})
	-- ({2.3219*\dx},{0.6588*\dy})
	-- ({2.3319*\dx},{0.6387*\dy})
	-- ({2.3420*\dx},{0.6184*\dy})
	-- ({2.3520*\dx},{0.5980*\dy})
	-- ({2.3620*\dx},{0.5775*\dy})
	-- ({2.3720*\dx},{0.5570*\dy})
	-- ({2.3820*\dx},{0.5365*\dy})
	-- ({2.3920*\dx},{0.5162*\dy})
	-- ({2.4020*\dx},{0.4960*\dy})
	-- ({2.4120*\dx},{0.4759*\dy})
	-- ({2.4220*\dx},{0.4562*\dy})
	-- ({2.4320*\dx},{0.4366*\dy})
	-- ({2.4420*\dx},{0.4174*\dy})
	-- ({2.4520*\dx},{0.3986*\dy})
	-- ({2.4621*\dx},{0.3801*\dy})
	-- ({2.4721*\dx},{0.3620*\dy})
	-- ({2.4821*\dx},{0.3443*\dy})
	-- ({2.4921*\dx},{0.3271*\dy})
	-- ({2.5021*\dx},{0.3104*\dy})
	-- ({2.5121*\dx},{0.2942*\dy})
	-- ({2.5221*\dx},{0.2784*\dy})
	-- ({2.5321*\dx},{0.2632*\dy})
	-- ({2.5421*\dx},{0.2484*\dy})
	-- ({2.5521*\dx},{0.2342*\dy})
	-- ({2.5621*\dx},{0.2205*\dy})
	-- ({2.5721*\dx},{0.2073*\dy})
	-- ({2.5822*\dx},{0.1947*\dy})
	-- ({2.5922*\dx},{0.1825*\dy})
	-- ({2.6022*\dx},{0.1709*\dy})
	-- ({2.6122*\dx},{0.1598*\dy})
	-- ({2.6222*\dx},{0.1491*\dy})
	-- ({2.6322*\dx},{0.1390*\dy})
	-- ({2.6422*\dx},{0.1293*\dy})
	-- ({2.6522*\dx},{0.1201*\dy})
	-- ({2.6622*\dx},{0.1114*\dy})
	-- ({2.6722*\dx},{0.1031*\dy})
	-- ({2.6822*\dx},{0.0952*\dy})
	-- ({2.6922*\dx},{0.0878*\dy})
	-- ({2.7023*\dx},{0.0807*\dy})
	-- ({2.7123*\dx},{0.0741*\dy})
	-- ({2.7223*\dx},{0.0678*\dy})
	-- ({2.7323*\dx},{0.0619*\dy})
	-- ({2.7423*\dx},{0.0564*\dy})
	-- ({2.7523*\dx},{0.0511*\dy})
	-- ({2.7623*\dx},{0.0463*\dy})
	-- ({2.7723*\dx},{0.0417*\dy})
	-- ({2.7823*\dx},{0.0375*\dy})
	-- ({2.7923*\dx},{0.0335*\dy})
	-- ({2.8023*\dx},{0.0298*\dy})
	-- ({2.8123*\dx},{0.0264*\dy})
	-- ({2.8224*\dx},{0.0233*\dy})
	-- ({2.8324*\dx},{0.0204*\dy})
	-- ({2.8424*\dx},{0.0177*\dy})
	-- ({2.8524*\dx},{0.0152*\dy})
	-- ({2.8624*\dx},{0.0130*\dy})
	-- ({2.8724*\dx},{0.0110*\dy})
	-- ({2.8824*\dx},{0.0092*\dy})
	-- ({2.8924*\dx},{0.0076*\dy})
	-- ({2.9024*\dx},{0.0061*\dy})
	-- ({2.9124*\dx},{0.0048*\dy})
	-- ({2.9224*\dx},{0.0037*\dy})
	-- ({2.9324*\dx},{0.0028*\dy})
	-- ({2.9425*\dx},{0.0020*\dy})
	-- ({2.9525*\dx},{0.0013*\dy})
	-- ({2.9625*\dx},{0.0008*\dy})
	-- ({2.9725*\dx},{0.0004*\dy})
	-- ({2.9825*\dx},{0.0002*\dy})
	-- ({2.9925*\dx},{0.0000*\dy})
	-- ({3.0025*\dx},{0.0000*\dy})
	-- ({3.0125*\dx},{0.0000*\dy})
	-- ({3.0225*\dx},{0.0000*\dy})
	-- ({3.0325*\dx},{0.0000*\dy})
	-- ({3.0425*\dx},{0.0000*\dy})
	-- ({3.0525*\dx},{0.0000*\dy})
	-- ({3.0626*\dx},{0.0000*\dy})
	-- ({3.0726*\dx},{0.0000*\dy})
	-- ({3.0826*\dx},{0.0000*\dy})
	-- ({3.0926*\dx},{0.0000*\dy})
	-- ({3.1026*\dx},{0.0000*\dy})
	-- ({3.1126*\dx},{0.0000*\dy})
	-- ({3.1226*\dx},{0.0000*\dy})
	-- ({3.1326*\dx},{0.0000*\dy})
	-- ({3.1426*\dx},{0.0000*\dy})
	-- ({3.1526*\dx},{0.0000*\dy})
	-- ({3.1626*\dx},{0.0000*\dy})
	-- ({3.1726*\dx},{0.0000*\dy})
	-- ({3.1827*\dx},{0.0000*\dy})
	-- ({3.1927*\dx},{0.0000*\dy})
	-- ({3.2027*\dx},{0.0000*\dy})
	-- ({3.2127*\dx},{0.0000*\dy})
	-- ({3.2227*\dx},{0.0000*\dy})
	-- ({3.2327*\dx},{0.0000*\dy})
	-- ({3.2427*\dx},{0.0000*\dy})
	-- ({3.2527*\dx},{0.0000*\dy})
	-- ({3.2627*\dx},{0.0000*\dy})
	-- ({3.2727*\dx},{0.0000*\dy})
	-- ({3.2827*\dx},{0.0000*\dy})
	-- ({3.2927*\dx},{0.0000*\dy})
	-- ({3.3028*\dx},{0.0000*\dy})
	-- ({3.3128*\dx},{0.0000*\dy})
	-- ({3.3228*\dx},{0.0000*\dy})
	-- ({3.3328*\dx},{0.0000*\dy})
	-- ({3.3428*\dx},{0.0000*\dy})
	-- ({3.3528*\dx},{0.0000*\dy})
	-- ({3.3628*\dx},{0.0000*\dy})
	-- ({3.3728*\dx},{0.0000*\dy})
	-- ({3.3828*\dx},{0.0000*\dy})
	-- ({3.3928*\dx},{0.0000*\dy})
	-- ({3.4028*\dx},{0.0000*\dy})
	-- ({3.4128*\dx},{0.0000*\dy})
	-- ({3.4229*\dx},{0.0000*\dy})
	-- ({3.4329*\dx},{0.0000*\dy})
	-- ({3.4429*\dx},{0.0000*\dy})
	-- ({3.4529*\dx},{0.0000*\dy})
	-- ({3.4629*\dx},{0.0000*\dy})
	-- ({3.4729*\dx},{0.0000*\dy})
	-- ({3.4829*\dx},{0.0000*\dy})
	-- ({3.4929*\dx},{0.0000*\dy})
	-- ({3.5029*\dx},{0.0000*\dy})
	-- ({3.5129*\dx},{0.0000*\dy})
	-- ({3.5229*\dx},{0.0000*\dy})
	-- ({3.5329*\dx},{0.0000*\dy})
	-- ({3.5430*\dx},{0.0000*\dy})
	-- ({3.5530*\dx},{0.0000*\dy})
	-- ({3.5630*\dx},{0.0000*\dy})
	-- ({3.5730*\dx},{0.0000*\dy})
	-- ({3.5830*\dx},{0.0000*\dy})
	-- ({3.5930*\dx},{0.0000*\dy})
	-- ({3.6030*\dx},{0.0000*\dy})
	-- ({3.6130*\dx},{0.0000*\dy})
	-- ({3.6230*\dx},{0.0000*\dy})
	-- ({3.6330*\dx},{0.0000*\dy})
	-- ({3.6430*\dx},{0.0000*\dy})
	-- ({3.6530*\dx},{0.0000*\dy})
	-- ({3.6631*\dx},{0.0000*\dy})
	-- ({3.6731*\dx},{0.0000*\dy})
	-- ({3.6831*\dx},{0.0000*\dy})
	-- ({3.6931*\dx},{0.0000*\dy})
	-- ({3.7031*\dx},{0.0000*\dy})
	-- ({3.7131*\dx},{0.0000*\dy})
	-- ({3.7231*\dx},{0.0000*\dy})
	-- ({3.7331*\dx},{0.0000*\dy})
	-- ({3.7431*\dx},{0.0000*\dy})
	-- ({3.7531*\dx},{0.0000*\dy})
	-- ({3.7631*\dx},{0.0000*\dy})
	-- ({3.7731*\dx},{0.0000*\dy})
	-- ({3.7832*\dx},{0.0000*\dy})
	-- ({3.7932*\dx},{0.0000*\dy})
	-- ({3.8032*\dx},{0.0000*\dy})
	-- ({3.8132*\dx},{0.0000*\dy})
	-- ({3.8232*\dx},{0.0000*\dy})
	-- ({3.8332*\dx},{0.0000*\dy})
	-- ({3.8432*\dx},{0.0000*\dy})
	-- ({3.8532*\dx},{0.0000*\dy})
	-- ({3.8632*\dx},{0.0000*\dy})
	-- ({3.8732*\dx},{0.0000*\dy})
	-- ({3.8832*\dx},{0.0000*\dy})
	-- ({3.8932*\dx},{0.0000*\dy})
	-- ({3.9033*\dx},{0.0000*\dy})
	-- ({3.9133*\dx},{0.0000*\dy})
	-- ({3.9233*\dx},{0.0000*\dy})
	-- ({3.9333*\dx},{0.0000*\dy})
	-- ({3.9433*\dx},{0.0000*\dy})
	-- ({3.9533*\dx},{0.0000*\dy})
	-- ({3.9633*\dx},{0.0000*\dy})
	-- ({3.9733*\dx},{0.0000*\dy})
	-- ({3.9833*\dx},{0.0000*\dy})
	-- ({3.9933*\dx},{0.0000*\dy})
	-- ({4.0033*\dx},{0.0000*\dy})
	-- ({4.0133*\dx},{0.0000*\dy})
	-- ({4.0234*\dx},{0.0000*\dy})
	-- ({4.0334*\dx},{0.0000*\dy})
	-- ({4.0434*\dx},{0.0000*\dy})
	-- ({4.0534*\dx},{0.0000*\dy})
	-- ({4.0634*\dx},{0.0000*\dy})
	-- ({4.0734*\dx},{0.0000*\dy})
	-- ({4.0834*\dx},{0.0000*\dy})
	-- ({4.0934*\dx},{0.0000*\dy})
	-- ({4.1034*\dx},{0.0000*\dy})
	-- ({4.1134*\dx},{0.0000*\dy})
	-- ({4.1234*\dx},{0.0000*\dy})
	-- ({4.1334*\dx},{0.0000*\dy})
	-- ({4.1435*\dx},{0.0000*\dy})
	-- ({4.1535*\dx},{0.0000*\dy})
	-- ({4.1635*\dx},{0.0000*\dy})
	-- ({4.1735*\dx},{0.0000*\dy})
	-- ({4.1835*\dx},{0.0000*\dy})
	-- ({4.1935*\dx},{0.0000*\dy})
	-- ({4.2035*\dx},{0.0000*\dy})
	-- ({4.2135*\dx},{0.0000*\dy})
	-- ({4.2235*\dx},{0.0000*\dy})
	-- ({4.2335*\dx},{0.0000*\dy})
	-- ({4.2435*\dx},{0.0000*\dy})
	-- ({4.2535*\dx},{0.0000*\dy})
	-- ({4.2636*\dx},{0.0000*\dy})
	-- ({4.2736*\dx},{0.0000*\dy})
	-- ({4.2836*\dx},{0.0000*\dy})
	-- ({4.2936*\dx},{0.0000*\dy})
	-- ({4.3036*\dx},{0.0000*\dy})
	-- ({4.3136*\dx},{0.0000*\dy})
	-- ({4.3236*\dx},{0.0000*\dy})
	-- ({4.3336*\dx},{0.0000*\dy})
	-- ({4.3436*\dx},{0.0000*\dy})
	-- ({4.3536*\dx},{0.0000*\dy})
	-- ({4.3636*\dx},{0.0000*\dy})
	-- ({4.3736*\dx},{0.0000*\dy})
	-- ({4.3837*\dx},{0.0000*\dy})
	-- ({4.3937*\dx},{0.0000*\dy})
	-- ({4.4037*\dx},{0.0000*\dy})
	-- ({4.4137*\dx},{0.0000*\dy})
	-- ({4.4237*\dx},{0.0000*\dy})
	-- ({4.4337*\dx},{0.0000*\dy})
	-- ({4.4437*\dx},{0.0000*\dy})
	-- ({4.4537*\dx},{0.0000*\dy})
	-- ({4.4637*\dx},{0.0000*\dy})
	-- ({4.4737*\dx},{0.0000*\dy})
	-- ({4.4837*\dx},{0.0000*\dy})
	-- ({4.4937*\dx},{0.0000*\dy})
	-- ({4.5038*\dx},{0.0000*\dy})
	-- ({4.5138*\dx},{0.0000*\dy})
	-- ({4.5238*\dx},{0.0000*\dy})
	-- ({4.5338*\dx},{0.0000*\dy})
	-- ({4.5438*\dx},{0.0000*\dy})
	-- ({4.5538*\dx},{0.0000*\dy})
	-- ({4.5638*\dx},{0.0000*\dy})
	-- ({4.5738*\dx},{0.0000*\dy})
	-- ({4.5838*\dx},{0.0000*\dy})
	-- ({4.5938*\dx},{0.0000*\dy})
	-- ({4.6038*\dx},{0.0000*\dy})
	-- ({4.6138*\dx},{0.0000*\dy})
	-- ({4.6239*\dx},{0.0000*\dy})
	-- ({4.6339*\dx},{0.0000*\dy})
	-- ({4.6439*\dx},{0.0000*\dy})
	-- ({4.6539*\dx},{0.0000*\dy})
	-- ({4.6639*\dx},{0.0000*\dy})
	-- ({4.6739*\dx},{0.0000*\dy})
	-- ({4.6839*\dx},{0.0000*\dy})
	-- ({4.6939*\dx},{0.0000*\dy})
	-- ({4.7039*\dx},{0.0000*\dy})
	-- ({4.7139*\dx},{0.0000*\dy})
	-- ({4.7239*\dx},{0.0000*\dy})
	-- ({4.7339*\dx},{0.0000*\dy})
	-- ({4.7440*\dx},{0.0000*\dy})
	-- ({4.7540*\dx},{0.0000*\dy})
	-- ({4.7640*\dx},{0.0000*\dy})
	-- ({4.7740*\dx},{0.0000*\dy})
	-- ({4.7840*\dx},{0.0000*\dy})
	-- ({4.7940*\dx},{0.0000*\dy})
	-- ({4.8040*\dx},{0.0000*\dy})
	-- ({4.8140*\dx},{0.0000*\dy})
	-- ({4.8240*\dx},{0.0000*\dy})
	-- ({4.8340*\dx},{0.0000*\dy})
	-- ({4.8440*\dx},{0.0000*\dy})
	-- ({4.8540*\dx},{0.0000*\dy})
	-- ({4.8641*\dx},{0.0000*\dy})
	-- ({4.8741*\dx},{0.0000*\dy})
	-- ({4.8841*\dx},{0.0000*\dy})
	-- ({4.8941*\dx},{0.0000*\dy})
	-- ({4.9041*\dx},{0.0000*\dy})
	-- ({4.9141*\dx},{0.0000*\dy})
	-- ({4.9241*\dx},{0.0000*\dy})
	-- ({4.9341*\dx},{0.0000*\dy})
	-- ({4.9441*\dx},{0.0000*\dy})
	-- ({4.9541*\dx},{0.0000*\dy})
	-- ({4.9641*\dx},{0.0000*\dy})
	-- ({4.9741*\dx},{0.0000*\dy})
	-- ({4.9842*\dx},{0.0000*\dy})
	-- ({4.9942*\dx},{0.0000*\dy})
	-- ({5.0042*\dx},{0.0000*\dy})
	-- ({5.0142*\dx},{0.0000*\dy})
	-- ({5.0242*\dx},{0.0000*\dy})
	-- ({5.0342*\dx},{0.0000*\dy})
	-- ({5.0442*\dx},{0.0000*\dy})
	-- ({5.0542*\dx},{0.0000*\dy})
	-- ({5.0642*\dx},{0.0000*\dy})
	-- ({5.0742*\dx},{0.0000*\dy})
	-- ({5.0842*\dx},{0.0000*\dy})
	-- ({5.0942*\dx},{0.0000*\dy})
	-- ({5.1043*\dx},{0.0000*\dy})
	-- ({5.1143*\dx},{0.0000*\dy})
	-- ({5.1243*\dx},{0.0000*\dy})
	-- ({5.1343*\dx},{0.0000*\dy})
	-- ({5.1443*\dx},{0.0000*\dy})
	-- ({5.1543*\dx},{0.0000*\dy})
	-- ({5.1643*\dx},{0.0000*\dy})
	-- ({5.1743*\dx},{0.0000*\dy})
	-- ({5.1843*\dx},{0.0000*\dy})
	-- ({5.1943*\dx},{0.0000*\dy})
	-- ({5.2043*\dx},{0.0000*\dy})
	-- ({5.2143*\dx},{0.0000*\dy})
	-- ({5.2244*\dx},{0.0000*\dy})
	-- ({5.2344*\dx},{0.0000*\dy})
	-- ({5.2444*\dx},{0.0000*\dy})
	-- ({5.2544*\dx},{0.0000*\dy})
	-- ({5.2644*\dx},{0.0000*\dy})
	-- ({5.2744*\dx},{0.0000*\dy})
	-- ({5.2844*\dx},{0.0000*\dy})
	-- ({5.2944*\dx},{0.0000*\dy})
	-- ({5.3044*\dx},{0.0000*\dy})
	-- ({5.3144*\dx},{0.0000*\dy})
	-- ({5.3244*\dx},{0.0000*\dy})
	-- ({5.3344*\dx},{0.0000*\dy})
	-- ({5.3445*\dx},{0.0000*\dy})
	-- ({5.3545*\dx},{0.0000*\dy})
	-- ({5.3645*\dx},{0.0000*\dy})
	-- ({5.3745*\dx},{0.0000*\dy})
	-- ({5.3845*\dx},{0.0000*\dy})
	-- ({5.3945*\dx},{0.0000*\dy})
	-- ({5.4045*\dx},{0.0000*\dy})
	-- ({5.4145*\dx},{0.0000*\dy})
	-- ({5.4245*\dx},{0.0000*\dy})
	-- ({5.4345*\dx},{0.0000*\dy})
	-- ({5.4445*\dx},{0.0000*\dy})
	-- ({5.4545*\dx},{0.0000*\dy})
	-- ({5.4646*\dx},{0.0000*\dy})
	-- ({5.4746*\dx},{0.0000*\dy})
	-- ({5.4846*\dx},{0.0000*\dy})
	-- ({5.4946*\dx},{0.0000*\dy})
	-- ({5.5046*\dx},{0.0000*\dy})
	-- ({5.5146*\dx},{0.0000*\dy})
	-- ({5.5246*\dx},{0.0000*\dy})
	-- ({5.5346*\dx},{0.0000*\dy})
	-- ({5.5446*\dx},{0.0000*\dy})
	-- ({5.5546*\dx},{0.0000*\dy})
	-- ({5.5646*\dx},{0.0000*\dy})
	-- ({5.5746*\dx},{0.0000*\dy})
	-- ({5.5847*\dx},{0.0000*\dy})
	-- ({5.5947*\dx},{0.0000*\dy})
	-- ({5.6047*\dx},{0.0000*\dy})
	-- ({5.6147*\dx},{0.0000*\dy})
	-- ({5.6247*\dx},{0.0000*\dy})
	-- ({5.6347*\dx},{0.0000*\dy})
	-- ({5.6447*\dx},{0.0000*\dy})
	-- ({5.6547*\dx},{0.0000*\dy})
	-- ({5.6647*\dx},{0.0000*\dy})
	-- ({5.6747*\dx},{0.0000*\dy})
	-- ({5.6847*\dx},{0.0000*\dy})
	-- ({5.6947*\dx},{0.0000*\dy})
	-- ({5.7048*\dx},{0.0000*\dy})
	-- ({5.7148*\dx},{0.0000*\dy})
	-- ({5.7248*\dx},{0.0000*\dy})
	-- ({5.7348*\dx},{0.0000*\dy})
	-- ({5.7448*\dx},{0.0000*\dy})
	-- ({5.7548*\dx},{0.0000*\dy})
	-- ({5.7648*\dx},{0.0000*\dy})
	-- ({5.7748*\dx},{0.0000*\dy})
	-- ({5.7848*\dx},{0.0000*\dy})
	-- ({5.7948*\dx},{0.0000*\dy})
	-- ({5.8048*\dx},{0.0000*\dy})
	-- ({5.8148*\dx},{0.0000*\dy})
	-- ({5.8249*\dx},{0.0000*\dy})
	-- ({5.8349*\dx},{0.0000*\dy})
	-- ({5.8449*\dx},{0.0000*\dy})
	-- ({5.8549*\dx},{0.0000*\dy})
	-- ({5.8649*\dx},{0.0000*\dy})
	-- ({5.8749*\dx},{0.0000*\dy})
	-- ({5.8849*\dx},{0.0000*\dy})
	-- ({5.8949*\dx},{0.0000*\dy})
	-- ({5.9049*\dx},{0.0000*\dy})
	-- ({5.9149*\dx},{0.0000*\dy})
	-- ({5.9249*\dx},{0.0000*\dy})
	-- ({5.9349*\dx},{0.0000*\dy})
	-- ({5.9450*\dx},{0.0000*\dy})
	-- ({5.9550*\dx},{0.0000*\dy})
	-- ({5.9650*\dx},{0.0000*\dy})
	-- ({5.9750*\dx},{0.0000*\dy})
	-- ({5.9850*\dx},{0.0000*\dy})
	-- ({5.9950*\dx},{0.0000*\dy})
	-- ({6.0050*\dx},{0.0000*\dy})
	-- ({6.0150*\dx},{0.0000*\dy})
	-- ({6.0250*\dx},{0.0000*\dy})
	-- ({6.0350*\dx},{0.0000*\dy})
	-- ({6.0450*\dx},{0.0000*\dy})
	-- ({6.0550*\dx},{0.0000*\dy})
	-- ({6.0651*\dx},{0.0000*\dy})
	-- ({6.0751*\dx},{0.0000*\dy})
	-- ({6.0851*\dx},{0.0000*\dy})
	-- ({6.0951*\dx},{0.0000*\dy})
	-- ({6.1051*\dx},{0.0000*\dy})
	-- ({6.1151*\dx},{0.0000*\dy})
	-- ({6.1251*\dx},{0.0000*\dy})
	-- ({6.1351*\dx},{0.0000*\dy})
	-- ({6.1451*\dx},{0.0000*\dy})
	-- ({6.1551*\dx},{0.0000*\dy})
	-- ({6.1651*\dx},{0.0000*\dy})
	-- ({6.1751*\dx},{0.0000*\dy})
	-- ({6.1852*\dx},{0.0000*\dy})
	-- ({6.1952*\dx},{0.0000*\dy})
	-- ({6.2052*\dx},{0.0000*\dy})
	-- ({6.2152*\dx},{0.0000*\dy})
	-- ({6.2252*\dx},{0.0000*\dy})
	-- ({6.2352*\dx},{0.0000*\dy})
	-- ({6.2452*\dx},{0.0000*\dy})
	-- ({6.2552*\dx},{0.0000*\dy})
	-- ({6.2652*\dx},{0.0000*\dy})
	-- ({6.2752*\dx},{0.0000*\dy})
	-- ({6.2852*\dx},{0.0000*\dy})
	-- ({6.2952*\dx},{0.0000*\dy})
	-- ({6.3053*\dx},{0.0000*\dy})
	-- ({6.3153*\dx},{0.0000*\dy})
	-- ({6.3253*\dx},{0.0000*\dy})
	-- ({6.3353*\dx},{0.0000*\dy})
	-- ({6.3453*\dx},{0.0000*\dy})
	-- ({6.3553*\dx},{0.0000*\dy})
	-- ({6.3653*\dx},{0.0000*\dy})
	-- ({6.3753*\dx},{0.0000*\dy})
	-- ({6.3853*\dx},{0.0000*\dy})
	-- ({6.3953*\dx},{0.0000*\dy})
	-- ({6.4053*\dx},{0.0000*\dy})
	-- ({6.4153*\dx},{0.0000*\dy})
	-- ({6.4254*\dx},{0.0000*\dy})
	-- ({6.4354*\dx},{0.0000*\dy})
	-- ({6.4454*\dx},{0.0000*\dy})
	-- ({6.4554*\dx},{0.0000*\dy})
	-- ({6.4654*\dx},{0.0000*\dy})
	-- ({6.4754*\dx},{0.0000*\dy})
	-- ({6.4854*\dx},{0.0000*\dy})
	-- ({6.4954*\dx},{0.0000*\dy})
	-- ({6.5054*\dx},{0.0000*\dy})
	-- ({6.5154*\dx},{0.0000*\dy})
	-- ({6.5254*\dx},{0.0000*\dy})
	-- ({6.5354*\dx},{0.0000*\dy})
	-- ({6.5455*\dx},{0.0000*\dy})
	-- ({6.5555*\dx},{0.0000*\dy})
	-- ({6.5655*\dx},{0.0000*\dy})
	-- ({6.5755*\dx},{0.0000*\dy})
	-- ({6.5855*\dx},{0.0000*\dy})
	-- ({6.5955*\dx},{0.0000*\dy})
	-- ({6.6055*\dx},{0.0000*\dy})
	-- ({6.6155*\dx},{0.0000*\dy})
	-- ({6.6255*\dx},{0.0000*\dy})
	-- ({6.6355*\dx},{0.0000*\dy})
	-- ({6.6455*\dx},{0.0000*\dy})
	-- ({6.6555*\dx},{0.0000*\dy})
	-- ({6.6656*\dx},{0.0000*\dy})
	-- ({6.6756*\dx},{0.0000*\dy})
	-- ({6.6856*\dx},{0.0000*\dy})
	-- ({6.6956*\dx},{0.0000*\dy})
	-- ({6.7056*\dx},{0.0000*\dy})
	-- ({6.7156*\dx},{0.0000*\dy})
	-- ({6.7256*\dx},{0.0000*\dy})
	-- ({6.7356*\dx},{0.0000*\dy})
	-- ({6.7456*\dx},{0.0000*\dy})
	-- ({6.7556*\dx},{0.0000*\dy})
	-- ({6.7656*\dx},{0.0000*\dy})
	-- ({6.7756*\dx},{0.0000*\dy})
	-- ({6.7857*\dx},{0.0000*\dy})
	-- ({6.7957*\dx},{0.0000*\dy})
	-- ({6.8057*\dx},{0.0000*\dy})
	-- ({6.8157*\dx},{0.0000*\dy})
	-- ({6.8257*\dx},{0.0000*\dy})
	-- ({6.8357*\dx},{0.0000*\dy})
	-- ({6.8457*\dx},{0.0000*\dy})
	-- ({6.8557*\dx},{0.0000*\dy})
	-- ({6.8657*\dx},{0.0000*\dy})
	-- ({6.8757*\dx},{0.0000*\dy})
	-- ({6.8857*\dx},{0.0000*\dy})
	-- ({6.8957*\dx},{0.0000*\dy})
	-- ({6.9058*\dx},{0.0000*\dy})
	-- ({6.9158*\dx},{0.0000*\dy})
	-- ({6.9258*\dx},{0.0000*\dy})
	-- ({6.9358*\dx},{0.0000*\dy})
	-- ({6.9458*\dx},{0.0000*\dy})
	-- ({6.9558*\dx},{0.0000*\dy})
	-- ({6.9658*\dx},{0.0000*\dy})
	-- ({6.9758*\dx},{0.0000*\dy})
	-- ({6.9858*\dx},{0.0000*\dy})
	-- ({6.9958*\dx},{0.0000*\dy})
	-- ({7.0058*\dx},{0.0000*\dy})
	-- ({7.0158*\dx},{0.0000*\dy})
	-- ({7.0259*\dx},{0.0000*\dy})
	-- ({7.0359*\dx},{0.0000*\dy})
	-- ({7.0459*\dx},{0.0000*\dy})
	-- ({7.0559*\dx},{0.0000*\dy})
	-- ({7.0659*\dx},{0.0000*\dy})
	-- ({7.0759*\dx},{0.0000*\dy})
	-- ({7.0859*\dx},{0.0000*\dy})
	-- ({7.0959*\dx},{0.0000*\dy})
	-- ({7.1059*\dx},{0.0000*\dy})
	-- ({7.1159*\dx},{0.0000*\dy})
	-- ({7.1259*\dx},{0.0000*\dy})
	-- ({7.1359*\dx},{0.0000*\dy})
	-- ({7.1460*\dx},{0.0000*\dy})
	-- ({7.1560*\dx},{0.0000*\dy})
	-- ({7.1660*\dx},{0.0000*\dy})
	-- ({7.1760*\dx},{0.0000*\dy})
	-- ({7.1860*\dx},{0.0000*\dy})
	-- ({7.1960*\dx},{0.0000*\dy})
	-- ({7.2060*\dx},{0.0000*\dy})
	-- ({7.2160*\dx},{0.0000*\dy})
	-- ({7.2260*\dx},{0.0000*\dy})
	-- ({7.2360*\dx},{0.0000*\dy})
	-- ({7.2460*\dx},{0.0000*\dy})
	-- ({7.2560*\dx},{0.0000*\dy})
	-- ({7.2661*\dx},{0.0000*\dy})
	-- ({7.2761*\dx},{0.0000*\dy})
	-- ({7.2861*\dx},{0.0000*\dy})
	-- ({7.2961*\dx},{0.0000*\dy})
	-- ({7.3061*\dx},{0.0000*\dy})
	-- ({7.3161*\dx},{0.0000*\dy})
	-- ({7.3261*\dx},{0.0000*\dy})
	-- ({7.3361*\dx},{0.0000*\dy})
	-- ({7.3461*\dx},{0.0000*\dy})
	-- ({7.3561*\dx},{0.0000*\dy})
	-- ({7.3661*\dx},{0.0000*\dy})
	-- ({7.3761*\dx},{0.0000*\dy})
	-- ({7.3862*\dx},{0.0000*\dy})
	-- ({7.3962*\dx},{0.0000*\dy})
	-- ({7.4062*\dx},{0.0000*\dy})
	-- ({7.4162*\dx},{0.0000*\dy})
	-- ({7.4262*\dx},{0.0000*\dy})
	-- ({7.4362*\dx},{0.0000*\dy})
	-- ({7.4462*\dx},{0.0000*\dy})
	-- ({7.4562*\dx},{0.0000*\dy})
	-- ({7.4662*\dx},{0.0000*\dy})
	-- ({7.4762*\dx},{0.0000*\dy})
	-- ({7.4862*\dx},{0.0000*\dy})
	-- ({7.4962*\dx},{0.0000*\dy})
	-- ({7.5063*\dx},{0.0000*\dy})
	-- ({7.5163*\dx},{0.0000*\dy})
	-- ({7.5263*\dx},{0.0000*\dy})
	-- ({7.5363*\dx},{0.0000*\dy})
	-- ({7.5463*\dx},{0.0000*\dy})
	-- ({7.5563*\dx},{0.0000*\dy})
	-- ({7.5663*\dx},{0.0000*\dy})
	-- ({7.5763*\dx},{0.0000*\dy})
	-- ({7.5863*\dx},{0.0000*\dy})
	-- ({7.5963*\dx},{0.0000*\dy})
	-- ({7.6063*\dx},{0.0000*\dy})
	-- ({7.6163*\dx},{0.0000*\dy})
	-- ({7.6264*\dx},{0.0000*\dy})
	-- ({7.6364*\dx},{0.0000*\dy})
	-- ({7.6464*\dx},{0.0000*\dy})
	-- ({7.6564*\dx},{0.0000*\dy})
	-- ({7.6664*\dx},{0.0000*\dy})
	-- ({7.6764*\dx},{0.0000*\dy})
	-- ({7.6864*\dx},{0.0000*\dy})
	-- ({7.6964*\dx},{0.0000*\dy})
	-- ({7.7064*\dx},{0.0000*\dy})
	-- ({7.7164*\dx},{0.0000*\dy})
	-- ({7.7264*\dx},{0.0000*\dy})
	-- ({7.7364*\dx},{0.0000*\dy})
	-- ({7.7465*\dx},{0.0000*\dy})
	-- ({7.7565*\dx},{0.0000*\dy})
	-- ({7.7665*\dx},{0.0000*\dy})
	-- ({7.7765*\dx},{0.0000*\dy})
	-- ({7.7865*\dx},{0.0000*\dy})
	-- ({7.7965*\dx},{0.0000*\dy})
	-- ({7.8065*\dx},{0.0000*\dy})
	-- ({7.8165*\dx},{0.0000*\dy})
	-- ({7.8265*\dx},{0.0000*\dy})
	-- ({7.8365*\dx},{0.0000*\dy})
	-- ({7.8465*\dx},{0.0000*\dy})
	-- ({7.8565*\dx},{0.0000*\dy})
	-- ({7.8666*\dx},{0.0000*\dy})
	-- ({7.8766*\dx},{0.0000*\dy})
	-- ({7.8866*\dx},{0.0000*\dy})
	-- ({7.8966*\dx},{0.0000*\dy})
	-- ({7.9066*\dx},{0.0000*\dy})
	-- ({7.9166*\dx},{0.0000*\dy})
	-- ({7.9266*\dx},{0.0000*\dy})
	-- ({7.9366*\dx},{0.0000*\dy})
	-- ({7.9466*\dx},{0.0000*\dy})
	-- ({7.9566*\dx},{0.0000*\dy})
	-- ({7.9666*\dx},{0.0000*\dy})
	-- ({7.9766*\dx},{0.0000*\dy})
	-- ({7.9867*\dx},{0.0000*\dy})
	-- ({7.9967*\dx},{0.0000*\dy})
	-- ({8.0067*\dx},{0.0000*\dy})
	-- ({8.0167*\dx},{0.0000*\dy})
	-- ({8.0267*\dx},{0.0000*\dy})
	-- ({8.0367*\dx},{0.0000*\dy})
	-- ({8.0467*\dx},{0.0000*\dy})
	-- ({8.0567*\dx},{0.0000*\dy})
	-- ({8.0667*\dx},{0.0000*\dy})
	-- ({8.0767*\dx},{0.0000*\dy})
	-- ({8.0867*\dx},{0.0000*\dy})
	-- ({8.0967*\dx},{0.0000*\dy})
	-- ({8.1068*\dx},{0.0000*\dy})
	-- ({8.1168*\dx},{0.0000*\dy})
	-- ({8.1268*\dx},{0.0000*\dy})
	-- ({8.1368*\dx},{0.0000*\dy})
	-- ({8.1468*\dx},{0.0000*\dy})
	-- ({8.1568*\dx},{0.0000*\dy})
	-- ({8.1668*\dx},{0.0000*\dy})
	-- ({8.1768*\dx},{0.0000*\dy})
	-- ({8.1868*\dx},{0.0000*\dy})
	-- ({8.1968*\dx},{0.0000*\dy})
	-- ({8.2068*\dx},{0.0000*\dy})
	-- ({8.2168*\dx},{0.0000*\dy})
	-- ({8.2269*\dx},{0.0000*\dy})
	-- ({8.2369*\dx},{0.0000*\dy})
	-- ({8.2469*\dx},{0.0000*\dy})
	-- ({8.2569*\dx},{0.0000*\dy})
	-- ({8.2669*\dx},{0.0000*\dy})
	-- ({8.2769*\dx},{0.0000*\dy})
	-- ({8.2869*\dx},{0.0000*\dy})
	-- ({8.2969*\dx},{0.0000*\dy})
	-- ({8.3069*\dx},{0.0000*\dy})
	-- ({8.3169*\dx},{0.0000*\dy})
	-- ({8.3269*\dx},{0.0000*\dy})
	-- ({8.3369*\dx},{0.0000*\dy})
	-- ({8.3470*\dx},{0.0000*\dy})
	-- ({8.3570*\dx},{0.0000*\dy})
	-- ({8.3670*\dx},{0.0000*\dy})
	-- ({8.3770*\dx},{0.0000*\dy})
	-- ({8.3870*\dx},{0.0000*\dy})
	-- ({8.3970*\dx},{0.0000*\dy})
	-- ({8.4070*\dx},{0.0000*\dy})
	-- ({8.4170*\dx},{0.0000*\dy})
	-- ({8.4270*\dx},{0.0000*\dy})
	-- ({8.4370*\dx},{0.0000*\dy})
	-- ({8.4470*\dx},{0.0000*\dy})
	-- ({8.4570*\dx},{0.0000*\dy})
	-- ({8.4671*\dx},{0.0000*\dy})
	-- ({8.4771*\dx},{0.0000*\dy})
	-- ({8.4871*\dx},{0.0000*\dy})
	-- ({8.4971*\dx},{0.0000*\dy})
	-- ({8.5071*\dx},{0.0000*\dy})
	-- ({8.5171*\dx},{0.0000*\dy})
	-- ({8.5271*\dx},{0.0000*\dy})
	-- ({8.5371*\dx},{0.0000*\dy})
	-- ({8.5471*\dx},{0.0000*\dy})
	-- ({8.5571*\dx},{0.0000*\dy})
	-- ({8.5671*\dx},{0.0000*\dy})
	-- ({8.5771*\dx},{0.0000*\dy})
	-- ({8.5872*\dx},{0.0000*\dy})
	-- ({8.5972*\dx},{0.0000*\dy})
	-- ({8.6072*\dx},{0.0000*\dy})
	-- ({8.6172*\dx},{0.0000*\dy})
	-- ({8.6272*\dx},{0.0000*\dy})
	-- ({8.6372*\dx},{0.0000*\dy})
	-- ({8.6472*\dx},{0.0000*\dy})
	-- ({8.6572*\dx},{0.0000*\dy})
	-- ({8.6672*\dx},{0.0000*\dy})
	-- ({8.6772*\dx},{0.0000*\dy})
	-- ({8.6872*\dx},{0.0000*\dy})
	-- ({8.6972*\dx},{0.0000*\dy})
	-- ({8.7073*\dx},{0.0000*\dy})
	-- ({8.7173*\dx},{0.0000*\dy})
	-- ({8.7273*\dx},{0.0000*\dy})
	-- ({8.7373*\dx},{0.0000*\dy})
	-- ({8.7473*\dx},{0.0000*\dy})
	-- ({8.7573*\dx},{0.0000*\dy})
	-- ({8.7673*\dx},{0.0000*\dy})
	-- ({8.7773*\dx},{0.0000*\dy})
	-- ({8.7873*\dx},{0.0000*\dy})
	-- ({8.7973*\dx},{0.0000*\dy})
	-- ({8.8073*\dx},{0.0000*\dy})
	-- ({8.8173*\dx},{0.0000*\dy})
	-- ({8.8274*\dx},{0.0000*\dy})
	-- ({8.8374*\dx},{0.0000*\dy})
	-- ({8.8474*\dx},{0.0000*\dy})
	-- ({8.8574*\dx},{0.0000*\dy})
	-- ({8.8674*\dx},{0.0000*\dy})
	-- ({8.8774*\dx},{0.0000*\dy})
	-- ({8.8874*\dx},{0.0000*\dy})
	-- ({8.8974*\dx},{0.0000*\dy})
	-- ({8.9074*\dx},{0.0000*\dy})
	-- ({8.9174*\dx},{0.0000*\dy})
	-- ({8.9274*\dx},{0.0000*\dy})
	-- ({8.9374*\dx},{0.0000*\dy})
	-- ({8.9475*\dx},{0.0000*\dy})
	-- ({8.9575*\dx},{0.0000*\dy})
	-- ({8.9675*\dx},{0.0000*\dy})
	-- ({8.9775*\dx},{0.0000*\dy})
	-- ({8.9875*\dx},{0.0000*\dy})
	-- ({8.9975*\dx},{0.0000*\dy})
	-- ({9.0075*\dx},{0.0000*\dy})
	-- ({9.0175*\dx},{0.0000*\dy})
	-- ({9.0275*\dx},{0.0000*\dy})
	-- ({9.0375*\dx},{0.0000*\dy})
	-- ({9.0475*\dx},{0.0000*\dy})
	-- ({9.0575*\dx},{0.0000*\dy})
	-- ({9.0676*\dx},{0.0000*\dy})
	-- ({9.0776*\dx},{0.0000*\dy})
	-- ({9.0876*\dx},{0.0000*\dy})
	-- ({9.0976*\dx},{0.0000*\dy})
	-- ({9.1076*\dx},{0.0000*\dy})
	-- ({9.1176*\dx},{0.0000*\dy})
	-- ({9.1276*\dx},{0.0000*\dy})
	-- ({9.1376*\dx},{0.0000*\dy})
	-- ({9.1476*\dx},{0.0000*\dy})
	-- ({9.1576*\dx},{0.0000*\dy})
	-- ({9.1676*\dx},{0.0000*\dy})
	-- ({9.1776*\dx},{0.0000*\dy})
	-- ({9.1877*\dx},{0.0000*\dy})
	-- ({9.1977*\dx},{0.0000*\dy})
	-- ({9.2077*\dx},{0.0000*\dy})
	-- ({9.2177*\dx},{0.0000*\dy})
	-- ({9.2277*\dx},{0.0000*\dy})
	-- ({9.2377*\dx},{0.0000*\dy})
	-- ({9.2477*\dx},{0.0000*\dy})
	-- ({9.2577*\dx},{0.0000*\dy})
	-- ({9.2677*\dx},{0.0000*\dy})
	-- ({9.2777*\dx},{0.0000*\dy})
	-- ({9.2877*\dx},{0.0000*\dy})
	-- ({9.2977*\dx},{0.0000*\dy})
	-- ({9.3078*\dx},{0.0000*\dy})
	-- ({9.3178*\dx},{0.0000*\dy})
	-- ({9.3278*\dx},{0.0000*\dy})
	-- ({9.3378*\dx},{0.0000*\dy})
	-- ({9.3478*\dx},{0.0000*\dy})
	-- ({9.3578*\dx},{0.0000*\dy})
	-- ({9.3678*\dx},{0.0000*\dy})
	-- ({9.3778*\dx},{0.0000*\dy})
	-- ({9.3878*\dx},{0.0000*\dy})
	-- ({9.3978*\dx},{0.0000*\dy})
	-- ({9.4078*\dx},{0.0000*\dy})
	-- ({9.4178*\dx},{0.0000*\dy})
	-- ({9.4279*\dx},{0.0000*\dy})
	-- ({9.4379*\dx},{0.0000*\dy})
	-- ({9.4479*\dx},{0.0000*\dy})
	-- ({9.4579*\dx},{0.0000*\dy})
	-- ({9.4679*\dx},{0.0000*\dy})
	-- ({9.4779*\dx},{0.0000*\dy})
	-- ({9.4879*\dx},{0.0000*\dy})
	-- ({9.4979*\dx},{0.0000*\dy})
	-- ({9.5079*\dx},{0.0000*\dy})
	-- ({9.5179*\dx},{0.0000*\dy})
	-- ({9.5279*\dx},{0.0000*\dy})
	-- ({9.5379*\dx},{0.0000*\dy})
	-- ({9.5480*\dx},{0.0000*\dy})
	-- ({9.5580*\dx},{0.0000*\dy})
	-- ({9.5680*\dx},{0.0000*\dy})
	-- ({9.5780*\dx},{0.0000*\dy})
	-- ({9.5880*\dx},{0.0000*\dy})
	-- ({9.5980*\dx},{0.0000*\dy})
	-- ({9.6080*\dx},{0.0000*\dy})
	-- ({9.6180*\dx},{0.0000*\dy})
	-- ({9.6280*\dx},{0.0000*\dy})
	-- ({9.6380*\dx},{0.0000*\dy})
	-- ({9.6480*\dx},{0.0000*\dy})
	-- ({9.6580*\dx},{0.0000*\dy})
	-- ({9.6681*\dx},{0.0000*\dy})
	-- ({9.6781*\dx},{0.0000*\dy})
	-- ({9.6881*\dx},{0.0000*\dy})
	-- ({9.6981*\dx},{0.0000*\dy})
	-- ({9.7081*\dx},{0.0000*\dy})
	-- ({9.7181*\dx},{0.0000*\dy})
	-- ({9.7281*\dx},{0.0000*\dy})
	-- ({9.7381*\dx},{0.0000*\dy})
	-- ({9.7481*\dx},{0.0000*\dy})
	-- ({9.7581*\dx},{0.0000*\dy})
	-- ({9.7681*\dx},{0.0000*\dy})
	-- ({9.7781*\dx},{0.0000*\dy})
	-- ({9.7882*\dx},{0.0000*\dy})
	-- ({9.7982*\dx},{0.0000*\dy})
	-- ({9.8082*\dx},{0.0000*\dy})
	-- ({9.8182*\dx},{0.0000*\dy})
	-- ({9.8282*\dx},{0.0000*\dy})
	-- ({9.8382*\dx},{0.0000*\dy})
	-- ({9.8482*\dx},{0.0000*\dy})
	-- ({9.8582*\dx},{0.0000*\dy})
	-- ({9.8682*\dx},{0.0000*\dy})
	-- ({9.8782*\dx},{0.0000*\dy})
	-- ({9.8882*\dx},{0.0000*\dy})
	-- ({9.8982*\dx},{0.0000*\dy})
	-- ({9.9083*\dx},{0.0000*\dy})
	-- ({9.9183*\dx},{0.0000*\dy})
	-- ({9.9283*\dx},{0.0000*\dy})
	-- ({9.9383*\dx},{0.0000*\dy})
	-- ({9.9483*\dx},{0.0000*\dy})
	-- ({9.9583*\dx},{0.0000*\dy})
	-- ({9.9683*\dx},{0.0000*\dy})
	-- ({9.9783*\dx},{0.0000*\dy})
	-- ({9.9883*\dx},{0.0000*\dy})
	-- ({9.9983*\dx},{0.0000*\dy})
	-- ({10.0083*\dx},{0.0000*\dy})
	-- ({10.0183*\dx},{0.0000*\dy})
	-- ({10.0284*\dx},{0.0000*\dy})
	-- ({10.0384*\dx},{0.0000*\dy})
	-- ({10.0484*\dx},{0.0000*\dy})
	-- ({10.0584*\dx},{0.0000*\dy})
	-- ({10.0684*\dx},{0.0000*\dy})
	-- ({10.0784*\dx},{0.0000*\dy})
	-- ({10.0884*\dx},{0.0000*\dy})
	-- ({10.0984*\dx},{0.0000*\dy})
	-- ({10.1084*\dx},{0.0000*\dy})
	-- ({10.1184*\dx},{0.0000*\dy})
	-- ({10.1284*\dx},{0.0000*\dy})
	-- ({10.1384*\dx},{0.0000*\dy})
	-- ({10.1485*\dx},{0.0000*\dy})
	-- ({10.1585*\dx},{0.0000*\dy})
	-- ({10.1685*\dx},{0.0000*\dy})
	-- ({10.1785*\dx},{0.0000*\dy})
	-- ({10.1885*\dx},{0.0000*\dy})
	-- ({10.1985*\dx},{0.0000*\dy})
	-- ({10.2085*\dx},{0.0000*\dy})
	-- ({10.2185*\dx},{0.0000*\dy})
	-- ({10.2285*\dx},{0.0000*\dy})
	-- ({10.2385*\dx},{0.0000*\dy})
	-- ({10.2485*\dx},{0.0000*\dy})
	-- ({10.2585*\dx},{0.0000*\dy})
	-- ({10.2686*\dx},{0.0000*\dy})
	-- ({10.2786*\dx},{0.0000*\dy})
	-- ({10.2886*\dx},{0.0000*\dy})
	-- ({10.2986*\dx},{0.0000*\dy})
	-- ({10.3086*\dx},{0.0000*\dy})
	-- ({10.3186*\dx},{0.0000*\dy})
	-- ({10.3286*\dx},{0.0000*\dy})
	-- ({10.3386*\dx},{0.0000*\dy})
	-- ({10.3486*\dx},{0.0000*\dy})
	-- ({10.3586*\dx},{0.0000*\dy})
	-- ({10.3686*\dx},{0.0000*\dy})
	-- ({10.3786*\dx},{0.0000*\dy})
	-- ({10.3887*\dx},{0.0000*\dy})
	-- ({10.3987*\dx},{0.0000*\dy})
	-- ({10.4087*\dx},{0.0000*\dy})
	-- ({10.4187*\dx},{0.0000*\dy})
	-- ({10.4287*\dx},{0.0000*\dy})
	-- ({10.4387*\dx},{0.0000*\dy})
	-- ({10.4487*\dx},{0.0000*\dy})
	-- ({10.4587*\dx},{0.0000*\dy})
	-- ({10.4687*\dx},{0.0000*\dy})
	-- ({10.4787*\dx},{0.0000*\dy})
	-- ({10.4887*\dx},{0.0000*\dy})
	-- ({10.4987*\dx},{0.0000*\dy})
	-- ({10.5088*\dx},{0.0000*\dy})
	-- ({10.5188*\dx},{0.0000*\dy})
	-- ({10.5288*\dx},{0.0000*\dy})
	-- ({10.5388*\dx},{0.0000*\dy})
	-- ({10.5488*\dx},{0.0000*\dy})
	-- ({10.5588*\dx},{0.0000*\dy})
	-- ({10.5688*\dx},{0.0000*\dy})
	-- ({10.5788*\dx},{0.0000*\dy})
	-- ({10.5888*\dx},{0.0000*\dy})
	-- ({10.5988*\dx},{0.0000*\dy})
	-- ({10.6088*\dx},{0.0000*\dy})
	-- ({10.6188*\dx},{0.0000*\dy})
	-- ({10.6289*\dx},{0.0000*\dy})
	-- ({10.6389*\dx},{0.0000*\dy})
	-- ({10.6489*\dx},{0.0000*\dy})
	-- ({10.6589*\dx},{0.0000*\dy})
	-- ({10.6689*\dx},{0.0000*\dy})
	-- ({10.6789*\dx},{0.0000*\dy})
	-- ({10.6889*\dx},{0.0000*\dy})
	-- ({10.6989*\dx},{0.0000*\dy})
	-- ({10.7089*\dx},{0.0000*\dy})
	-- ({10.7189*\dx},{0.0000*\dy})
	-- ({10.7289*\dx},{0.0000*\dy})
	-- ({10.7389*\dx},{0.0000*\dy})
	-- ({10.7490*\dx},{0.0000*\dy})
	-- ({10.7590*\dx},{0.0000*\dy})
	-- ({10.7690*\dx},{0.0000*\dy})
	-- ({10.7790*\dx},{0.0000*\dy})
	-- ({10.7890*\dx},{0.0000*\dy})
	-- ({10.7990*\dx},{0.0000*\dy})
	-- ({10.8090*\dx},{0.0000*\dy})
	-- ({10.8190*\dx},{0.0000*\dy})
	-- ({10.8290*\dx},{0.0000*\dy})
	-- ({10.8390*\dx},{0.0000*\dy})
	-- ({10.8490*\dx},{0.0000*\dy})
	-- ({10.8590*\dx},{0.0000*\dy})
	-- ({10.8691*\dx},{0.0000*\dy})
	-- ({10.8791*\dx},{0.0000*\dy})
	-- ({10.8891*\dx},{0.0000*\dy})
	-- ({10.8991*\dx},{0.0000*\dy})
	-- ({10.9091*\dx},{0.0000*\dy})
	-- ({10.9191*\dx},{0.0000*\dy})
	-- ({10.9291*\dx},{0.0000*\dy})
	-- ({10.9391*\dx},{0.0000*\dy})
	-- ({10.9491*\dx},{0.0000*\dy})
	-- ({10.9591*\dx},{0.0000*\dy})
	-- ({10.9691*\dx},{0.0000*\dy})
	-- ({10.9791*\dx},{0.0000*\dy})
	-- ({10.9892*\dx},{0.0000*\dy})
	-- ({10.9992*\dx},{0.0000*\dy})
	-- ({11.0092*\dx},{0.0000*\dy})
	-- ({11.0192*\dx},{0.0000*\dy})
	-- ({11.0292*\dx},{0.0000*\dy})
	-- ({11.0392*\dx},{0.0000*\dy})
	-- ({11.0492*\dx},{0.0000*\dy})
	-- ({11.0592*\dx},{0.0000*\dy})
	-- ({11.0692*\dx},{0.0000*\dy})
	-- ({11.0792*\dx},{0.0000*\dy})
	-- ({11.0892*\dx},{0.0000*\dy})
	-- ({11.0992*\dx},{0.0000*\dy})
	-- ({11.1093*\dx},{0.0000*\dy})
	-- ({11.1193*\dx},{0.0000*\dy})
	-- ({11.1293*\dx},{0.0000*\dy})
	-- ({11.1393*\dx},{0.0000*\dy})
	-- ({11.1493*\dx},{0.0000*\dy})
	-- ({11.1593*\dx},{0.0000*\dy})
	-- ({11.1693*\dx},{0.0000*\dy})
	-- ({11.1793*\dx},{0.0000*\dy})
	-- ({11.1893*\dx},{0.0000*\dy})
	-- ({11.1993*\dx},{0.0000*\dy})
	-- ({11.2093*\dx},{0.0000*\dy})
	-- ({11.2193*\dx},{0.0000*\dy})
	-- ({11.2294*\dx},{0.0000*\dy})
	-- ({11.2394*\dx},{0.0000*\dy})
	-- ({11.2494*\dx},{0.0000*\dy})
	-- ({11.2594*\dx},{0.0000*\dy})
	-- ({11.2694*\dx},{0.0000*\dy})
	-- ({11.2794*\dx},{0.0000*\dy})
	-- ({11.2894*\dx},{0.0000*\dy})
	-- ({11.2994*\dx},{0.0000*\dy})
	-- ({11.3094*\dx},{0.0000*\dy})
	-- ({11.3194*\dx},{0.0000*\dy})
	-- ({11.3294*\dx},{0.0000*\dy})
	-- ({11.3394*\dx},{0.0000*\dy})
	-- ({11.3495*\dx},{0.0000*\dy})
	-- ({11.3595*\dx},{0.0000*\dy})
	-- ({11.3695*\dx},{0.0000*\dy})
	-- ({11.3795*\dx},{0.0000*\dy})
	-- ({11.3895*\dx},{0.0000*\dy})
	-- ({11.3995*\dx},{0.0000*\dy})
	-- ({11.4095*\dx},{0.0000*\dy})
	-- ({11.4195*\dx},{0.0000*\dy})
	-- ({11.4295*\dx},{0.0000*\dy})
	-- ({11.4395*\dx},{0.0000*\dy})
	-- ({11.4495*\dx},{0.0000*\dy})
	-- ({11.4595*\dx},{0.0000*\dy})
	-- ({11.4696*\dx},{0.0000*\dy})
	-- ({11.4796*\dx},{0.0000*\dy})
	-- ({11.4896*\dx},{0.0000*\dy})
	-- ({11.4996*\dx},{0.0000*\dy})
	-- ({11.5096*\dx},{0.0000*\dy})
	-- ({11.5196*\dx},{0.0000*\dy})
	-- ({11.5296*\dx},{0.0000*\dy})
	-- ({11.5396*\dx},{0.0000*\dy})
	-- ({11.5496*\dx},{0.0000*\dy})
	-- ({11.5596*\dx},{0.0000*\dy})
	-- ({11.5696*\dx},{0.0000*\dy})
	-- ({11.5796*\dx},{0.0000*\dy})
	-- ({11.5897*\dx},{0.0000*\dy})
	-- ({11.5997*\dx},{0.0000*\dy})
	-- ({11.6097*\dx},{0.0000*\dy})
	-- ({11.6197*\dx},{0.0000*\dy})
	-- ({11.6297*\dx},{0.0000*\dy})
	-- ({11.6397*\dx},{0.0000*\dy})
	-- ({11.6497*\dx},{0.0000*\dy})
	-- ({11.6597*\dx},{0.0000*\dy})
	-- ({11.6697*\dx},{0.0000*\dy})
	-- ({11.6797*\dx},{0.0000*\dy})
	-- ({11.6897*\dx},{0.0000*\dy})
	-- ({11.6997*\dx},{0.0000*\dy})
	-- ({11.7098*\dx},{0.0000*\dy})
	-- ({11.7198*\dx},{0.0000*\dy})
	-- ({11.7298*\dx},{0.0000*\dy})
	-- ({11.7398*\dx},{0.0000*\dy})
	-- ({11.7498*\dx},{0.0000*\dy})
	-- ({11.7598*\dx},{0.0000*\dy})
	-- ({11.7698*\dx},{0.0000*\dy})
	-- ({11.7798*\dx},{0.0000*\dy})
	-- ({11.7898*\dx},{0.0000*\dy})
	-- ({11.7998*\dx},{0.0000*\dy})
	-- ({11.8098*\dx},{0.0000*\dy})
	-- ({11.8198*\dx},{0.0000*\dy})
	-- ({11.8299*\dx},{0.0000*\dy})
	-- ({11.8399*\dx},{0.0000*\dy})
	-- ({11.8499*\dx},{0.0000*\dy})
	-- ({11.8599*\dx},{0.0000*\dy})
	-- ({11.8699*\dx},{0.0000*\dy})
	-- ({11.8799*\dx},{0.0000*\dy})
	-- ({11.8899*\dx},{0.0000*\dy})
	-- ({11.8999*\dx},{0.0000*\dy})
	-- ({11.9099*\dx},{0.0000*\dy})
	-- ({11.9199*\dx},{0.0000*\dy})
	-- ({11.9299*\dx},{0.0000*\dy})
	-- ({11.9399*\dx},{0.0000*\dy})
	-- ({11.9500*\dx},{0.0000*\dy})
	-- ({11.9600*\dx},{0.0000*\dy})
	-- ({11.9700*\dx},{0.0000*\dy})
	-- ({11.9800*\dx},{0.0000*\dy})
	-- ({11.9900*\dx},{0.0000*\dy})
	-- ({12.0000*\dx},{0.0000*\dy})
}
\def\cpsitwo{
	({0.0000*\dx},{1.0000*\dy})
	-- ({0.0100*\dx},{1.0000*\dy})
	-- ({0.0200*\dx},{1.0000*\dy})
	-- ({0.0300*\dx},{1.0000*\dy})
	-- ({0.0400*\dx},{1.0000*\dy})
	-- ({0.0500*\dx},{1.0000*\dy})
	-- ({0.0601*\dx},{1.0000*\dy})
	-- ({0.0701*\dx},{1.0000*\dy})
	-- ({0.0801*\dx},{1.0000*\dy})
	-- ({0.0901*\dx},{1.0000*\dy})
	-- ({0.1001*\dx},{1.0000*\dy})
	-- ({0.1101*\dx},{1.0000*\dy})
	-- ({0.1201*\dx},{1.0000*\dy})
	-- ({0.1301*\dx},{1.0000*\dy})
	-- ({0.1401*\dx},{1.0000*\dy})
	-- ({0.1501*\dx},{1.0000*\dy})
	-- ({0.1601*\dx},{1.0000*\dy})
	-- ({0.1701*\dx},{1.0000*\dy})
	-- ({0.1802*\dx},{1.0000*\dy})
	-- ({0.1902*\dx},{1.0000*\dy})
	-- ({0.2002*\dx},{1.0000*\dy})
	-- ({0.2102*\dx},{1.0000*\dy})
	-- ({0.2202*\dx},{1.0000*\dy})
	-- ({0.2302*\dx},{1.0000*\dy})
	-- ({0.2402*\dx},{1.0000*\dy})
	-- ({0.2502*\dx},{1.0000*\dy})
	-- ({0.2602*\dx},{1.0000*\dy})
	-- ({0.2702*\dx},{1.0000*\dy})
	-- ({0.2802*\dx},{1.0000*\dy})
	-- ({0.2902*\dx},{1.0000*\dy})
	-- ({0.3003*\dx},{1.0000*\dy})
	-- ({0.3103*\dx},{1.0000*\dy})
	-- ({0.3203*\dx},{1.0000*\dy})
	-- ({0.3303*\dx},{1.0000*\dy})
	-- ({0.3403*\dx},{1.0000*\dy})
	-- ({0.3503*\dx},{1.0000*\dy})
	-- ({0.3603*\dx},{1.0000*\dy})
	-- ({0.3703*\dx},{1.0000*\dy})
	-- ({0.3803*\dx},{1.0000*\dy})
	-- ({0.3903*\dx},{1.0000*\dy})
	-- ({0.4003*\dx},{1.0000*\dy})
	-- ({0.4103*\dx},{1.0000*\dy})
	-- ({0.4204*\dx},{1.0000*\dy})
	-- ({0.4304*\dx},{1.0000*\dy})
	-- ({0.4404*\dx},{1.0000*\dy})
	-- ({0.4504*\dx},{1.0000*\dy})
	-- ({0.4604*\dx},{1.0000*\dy})
	-- ({0.4704*\dx},{1.0000*\dy})
	-- ({0.4804*\dx},{1.0000*\dy})
	-- ({0.4904*\dx},{1.0000*\dy})
	-- ({0.5004*\dx},{1.0000*\dy})
	-- ({0.5104*\dx},{1.0000*\dy})
	-- ({0.5204*\dx},{1.0000*\dy})
	-- ({0.5304*\dx},{1.0000*\dy})
	-- ({0.5405*\dx},{1.0000*\dy})
	-- ({0.5505*\dx},{1.0000*\dy})
	-- ({0.5605*\dx},{1.0000*\dy})
	-- ({0.5705*\dx},{1.0000*\dy})
	-- ({0.5805*\dx},{1.0000*\dy})
	-- ({0.5905*\dx},{1.0000*\dy})
	-- ({0.6005*\dx},{1.0000*\dy})
	-- ({0.6105*\dx},{1.0000*\dy})
	-- ({0.6205*\dx},{1.0000*\dy})
	-- ({0.6305*\dx},{1.0000*\dy})
	-- ({0.6405*\dx},{1.0000*\dy})
	-- ({0.6505*\dx},{1.0000*\dy})
	-- ({0.6606*\dx},{1.0000*\dy})
	-- ({0.6706*\dx},{1.0000*\dy})
	-- ({0.6806*\dx},{1.0000*\dy})
	-- ({0.6906*\dx},{1.0000*\dy})
	-- ({0.7006*\dx},{1.0000*\dy})
	-- ({0.7106*\dx},{1.0000*\dy})
	-- ({0.7206*\dx},{1.0000*\dy})
	-- ({0.7306*\dx},{1.0000*\dy})
	-- ({0.7406*\dx},{1.0000*\dy})
	-- ({0.7506*\dx},{1.0000*\dy})
	-- ({0.7606*\dx},{1.0000*\dy})
	-- ({0.7706*\dx},{1.0000*\dy})
	-- ({0.7807*\dx},{1.0000*\dy})
	-- ({0.7907*\dx},{1.0000*\dy})
	-- ({0.8007*\dx},{1.0000*\dy})
	-- ({0.8107*\dx},{1.0000*\dy})
	-- ({0.8207*\dx},{1.0000*\dy})
	-- ({0.8307*\dx},{1.0000*\dy})
	-- ({0.8407*\dx},{1.0000*\dy})
	-- ({0.8507*\dx},{1.0000*\dy})
	-- ({0.8607*\dx},{1.0000*\dy})
	-- ({0.8707*\dx},{1.0000*\dy})
	-- ({0.8807*\dx},{1.0000*\dy})
	-- ({0.8907*\dx},{1.0000*\dy})
	-- ({0.9008*\dx},{1.0000*\dy})
	-- ({0.9108*\dx},{1.0000*\dy})
	-- ({0.9208*\dx},{1.0000*\dy})
	-- ({0.9308*\dx},{1.0000*\dy})
	-- ({0.9408*\dx},{1.0000*\dy})
	-- ({0.9508*\dx},{1.0000*\dy})
	-- ({0.9608*\dx},{1.0000*\dy})
	-- ({0.9708*\dx},{1.0000*\dy})
	-- ({0.9808*\dx},{1.0000*\dy})
	-- ({0.9908*\dx},{1.0000*\dy})
	-- ({1.0008*\dx},{1.0000*\dy})
	-- ({1.0108*\dx},{1.0000*\dy})
	-- ({1.0209*\dx},{1.0000*\dy})
	-- ({1.0309*\dx},{1.0000*\dy})
	-- ({1.0409*\dx},{1.0000*\dy})
	-- ({1.0509*\dx},{1.0000*\dy})
	-- ({1.0609*\dx},{1.0000*\dy})
	-- ({1.0709*\dx},{1.0000*\dy})
	-- ({1.0809*\dx},{1.0000*\dy})
	-- ({1.0909*\dx},{1.0000*\dy})
	-- ({1.1009*\dx},{1.0000*\dy})
	-- ({1.1109*\dx},{1.0000*\dy})
	-- ({1.1209*\dx},{1.0000*\dy})
	-- ({1.1309*\dx},{1.0000*\dy})
	-- ({1.1410*\dx},{1.0000*\dy})
	-- ({1.1510*\dx},{1.0000*\dy})
	-- ({1.1610*\dx},{1.0000*\dy})
	-- ({1.1710*\dx},{1.0000*\dy})
	-- ({1.1810*\dx},{1.0000*\dy})
	-- ({1.1910*\dx},{1.0000*\dy})
	-- ({1.2010*\dx},{1.0000*\dy})
	-- ({1.2110*\dx},{1.0000*\dy})
	-- ({1.2210*\dx},{1.0000*\dy})
	-- ({1.2310*\dx},{1.0000*\dy})
	-- ({1.2410*\dx},{1.0000*\dy})
	-- ({1.2510*\dx},{1.0000*\dy})
	-- ({1.2611*\dx},{1.0000*\dy})
	-- ({1.2711*\dx},{1.0000*\dy})
	-- ({1.2811*\dx},{1.0000*\dy})
	-- ({1.2911*\dx},{1.0000*\dy})
	-- ({1.3011*\dx},{1.0000*\dy})
	-- ({1.3111*\dx},{1.0000*\dy})
	-- ({1.3211*\dx},{1.0000*\dy})
	-- ({1.3311*\dx},{1.0000*\dy})
	-- ({1.3411*\dx},{1.0000*\dy})
	-- ({1.3511*\dx},{1.0000*\dy})
	-- ({1.3611*\dx},{1.0000*\dy})
	-- ({1.3711*\dx},{1.0000*\dy})
	-- ({1.3812*\dx},{1.0000*\dy})
	-- ({1.3912*\dx},{1.0000*\dy})
	-- ({1.4012*\dx},{1.0000*\dy})
	-- ({1.4112*\dx},{1.0000*\dy})
	-- ({1.4212*\dx},{1.0000*\dy})
	-- ({1.4312*\dx},{1.0000*\dy})
	-- ({1.4412*\dx},{1.0000*\dy})
	-- ({1.4512*\dx},{1.0000*\dy})
	-- ({1.4612*\dx},{1.0000*\dy})
	-- ({1.4712*\dx},{1.0000*\dy})
	-- ({1.4812*\dx},{1.0000*\dy})
	-- ({1.4912*\dx},{1.0000*\dy})
	-- ({1.5013*\dx},{1.0000*\dy})
	-- ({1.5113*\dx},{1.0000*\dy})
	-- ({1.5213*\dx},{1.0000*\dy})
	-- ({1.5313*\dx},{1.0000*\dy})
	-- ({1.5413*\dx},{1.0000*\dy})
	-- ({1.5513*\dx},{1.0000*\dy})
	-- ({1.5613*\dx},{1.0000*\dy})
	-- ({1.5713*\dx},{1.0000*\dy})
	-- ({1.5813*\dx},{1.0000*\dy})
	-- ({1.5913*\dx},{1.0000*\dy})
	-- ({1.6013*\dx},{1.0000*\dy})
	-- ({1.6113*\dx},{1.0000*\dy})
	-- ({1.6214*\dx},{1.0000*\dy})
	-- ({1.6314*\dx},{1.0000*\dy})
	-- ({1.6414*\dx},{1.0000*\dy})
	-- ({1.6514*\dx},{1.0000*\dy})
	-- ({1.6614*\dx},{1.0000*\dy})
	-- ({1.6714*\dx},{1.0000*\dy})
	-- ({1.6814*\dx},{1.0000*\dy})
	-- ({1.6914*\dx},{1.0000*\dy})
	-- ({1.7014*\dx},{1.0000*\dy})
	-- ({1.7114*\dx},{1.0000*\dy})
	-- ({1.7214*\dx},{1.0000*\dy})
	-- ({1.7314*\dx},{1.0000*\dy})
	-- ({1.7415*\dx},{1.0000*\dy})
	-- ({1.7515*\dx},{1.0000*\dy})
	-- ({1.7615*\dx},{1.0000*\dy})
	-- ({1.7715*\dx},{1.0000*\dy})
	-- ({1.7815*\dx},{1.0000*\dy})
	-- ({1.7915*\dx},{1.0000*\dy})
	-- ({1.8015*\dx},{1.0000*\dy})
	-- ({1.8115*\dx},{1.0000*\dy})
	-- ({1.8215*\dx},{1.0000*\dy})
	-- ({1.8315*\dx},{1.0000*\dy})
	-- ({1.8415*\dx},{1.0000*\dy})
	-- ({1.8515*\dx},{1.0000*\dy})
	-- ({1.8616*\dx},{1.0000*\dy})
	-- ({1.8716*\dx},{1.0000*\dy})
	-- ({1.8816*\dx},{1.0000*\dy})
	-- ({1.8916*\dx},{1.0000*\dy})
	-- ({1.9016*\dx},{1.0000*\dy})
	-- ({1.9116*\dx},{1.0000*\dy})
	-- ({1.9216*\dx},{1.0000*\dy})
	-- ({1.9316*\dx},{1.0000*\dy})
	-- ({1.9416*\dx},{1.0000*\dy})
	-- ({1.9516*\dx},{1.0000*\dy})
	-- ({1.9616*\dx},{1.0000*\dy})
	-- ({1.9716*\dx},{1.0000*\dy})
	-- ({1.9817*\dx},{1.0000*\dy})
	-- ({1.9917*\dx},{1.0000*\dy})
	-- ({2.0017*\dx},{1.0000*\dy})
	-- ({2.0117*\dx},{0.9997*\dy})
	-- ({2.0217*\dx},{0.9988*\dy})
	-- ({2.0317*\dx},{0.9974*\dy})
	-- ({2.0417*\dx},{0.9954*\dy})
	-- ({2.0517*\dx},{0.9928*\dy})
	-- ({2.0617*\dx},{0.9896*\dy})
	-- ({2.0717*\dx},{0.9857*\dy})
	-- ({2.0817*\dx},{0.9811*\dy})
	-- ({2.0917*\dx},{0.9759*\dy})
	-- ({2.1018*\dx},{0.9699*\dy})
	-- ({2.1118*\dx},{0.9632*\dy})
	-- ({2.1218*\dx},{0.9558*\dy})
	-- ({2.1318*\dx},{0.9475*\dy})
	-- ({2.1418*\dx},{0.9386*\dy})
	-- ({2.1518*\dx},{0.9288*\dy})
	-- ({2.1618*\dx},{0.9183*\dy})
	-- ({2.1718*\dx},{0.9070*\dy})
	-- ({2.1818*\dx},{0.8950*\dy})
	-- ({2.1918*\dx},{0.8822*\dy})
	-- ({2.2018*\dx},{0.8686*\dy})
	-- ({2.2118*\dx},{0.8543*\dy})
	-- ({2.2219*\dx},{0.8393*\dy})
	-- ({2.2319*\dx},{0.8237*\dy})
	-- ({2.2419*\dx},{0.8074*\dy})
	-- ({2.2519*\dx},{0.7905*\dy})
	-- ({2.2619*\dx},{0.7730*\dy})
	-- ({2.2719*\dx},{0.7550*\dy})
	-- ({2.2819*\dx},{0.7366*\dy})
	-- ({2.2919*\dx},{0.7177*\dy})
	-- ({2.3019*\dx},{0.6984*\dy})
	-- ({2.3119*\dx},{0.6787*\dy})
	-- ({2.3219*\dx},{0.6588*\dy})
	-- ({2.3319*\dx},{0.6387*\dy})
	-- ({2.3420*\dx},{0.6184*\dy})
	-- ({2.3520*\dx},{0.5980*\dy})
	-- ({2.3620*\dx},{0.5775*\dy})
	-- ({2.3720*\dx},{0.5570*\dy})
	-- ({2.3820*\dx},{0.5365*\dy})
	-- ({2.3920*\dx},{0.5162*\dy})
	-- ({2.4020*\dx},{0.4960*\dy})
	-- ({2.4120*\dx},{0.4759*\dy})
	-- ({2.4220*\dx},{0.4562*\dy})
	-- ({2.4320*\dx},{0.4366*\dy})
	-- ({2.4420*\dx},{0.4174*\dy})
	-- ({2.4520*\dx},{0.3986*\dy})
	-- ({2.4621*\dx},{0.3801*\dy})
	-- ({2.4721*\dx},{0.3620*\dy})
	-- ({2.4821*\dx},{0.3443*\dy})
	-- ({2.4921*\dx},{0.3271*\dy})
	-- ({2.5021*\dx},{0.3104*\dy})
	-- ({2.5121*\dx},{0.2942*\dy})
	-- ({2.5221*\dx},{0.2784*\dy})
	-- ({2.5321*\dx},{0.2632*\dy})
	-- ({2.5421*\dx},{0.2484*\dy})
	-- ({2.5521*\dx},{0.2342*\dy})
	-- ({2.5621*\dx},{0.2205*\dy})
	-- ({2.5721*\dx},{0.2073*\dy})
	-- ({2.5822*\dx},{0.1947*\dy})
	-- ({2.5922*\dx},{0.1825*\dy})
	-- ({2.6022*\dx},{0.1709*\dy})
	-- ({2.6122*\dx},{0.1598*\dy})
	-- ({2.6222*\dx},{0.1491*\dy})
	-- ({2.6322*\dx},{0.1390*\dy})
	-- ({2.6422*\dx},{0.1293*\dy})
	-- ({2.6522*\dx},{0.1201*\dy})
	-- ({2.6622*\dx},{0.1114*\dy})
	-- ({2.6722*\dx},{0.1031*\dy})
	-- ({2.6822*\dx},{0.0952*\dy})
	-- ({2.6922*\dx},{0.0878*\dy})
	-- ({2.7023*\dx},{0.0807*\dy})
	-- ({2.7123*\dx},{0.0741*\dy})
	-- ({2.7223*\dx},{0.0678*\dy})
	-- ({2.7323*\dx},{0.0619*\dy})
	-- ({2.7423*\dx},{0.0564*\dy})
	-- ({2.7523*\dx},{0.0511*\dy})
	-- ({2.7623*\dx},{0.0463*\dy})
	-- ({2.7723*\dx},{0.0417*\dy})
	-- ({2.7823*\dx},{0.0375*\dy})
	-- ({2.7923*\dx},{0.0335*\dy})
	-- ({2.8023*\dx},{0.0298*\dy})
	-- ({2.8123*\dx},{0.0264*\dy})
	-- ({2.8224*\dx},{0.0233*\dy})
	-- ({2.8324*\dx},{0.0204*\dy})
	-- ({2.8424*\dx},{0.0177*\dy})
	-- ({2.8524*\dx},{0.0152*\dy})
	-- ({2.8624*\dx},{0.0130*\dy})
	-- ({2.8724*\dx},{0.0110*\dy})
	-- ({2.8824*\dx},{0.0092*\dy})
	-- ({2.8924*\dx},{0.0076*\dy})
	-- ({2.9024*\dx},{0.0061*\dy})
	-- ({2.9124*\dx},{0.0048*\dy})
	-- ({2.9224*\dx},{0.0037*\dy})
	-- ({2.9324*\dx},{0.0028*\dy})
	-- ({2.9425*\dx},{0.0020*\dy})
	-- ({2.9525*\dx},{0.0013*\dy})
	-- ({2.9625*\dx},{0.0008*\dy})
	-- ({2.9725*\dx},{0.0004*\dy})
	-- ({2.9825*\dx},{0.0002*\dy})
	-- ({2.9925*\dx},{0.0000*\dy})
	-- ({3.0025*\dx},{0.0000*\dy})
	-- ({3.0125*\dx},{0.0000*\dy})
	-- ({3.0225*\dx},{0.0000*\dy})
	-- ({3.0325*\dx},{0.0000*\dy})
	-- ({3.0425*\dx},{0.0000*\dy})
	-- ({3.0525*\dx},{0.0000*\dy})
	-- ({3.0626*\dx},{0.0000*\dy})
	-- ({3.0726*\dx},{0.0000*\dy})
	-- ({3.0826*\dx},{0.0000*\dy})
	-- ({3.0926*\dx},{0.0000*\dy})
	-- ({3.1026*\dx},{0.0000*\dy})
	-- ({3.1126*\dx},{0.0000*\dy})
	-- ({3.1226*\dx},{0.0000*\dy})
	-- ({3.1326*\dx},{0.0000*\dy})
	-- ({3.1426*\dx},{0.0000*\dy})
	-- ({3.1526*\dx},{0.0000*\dy})
	-- ({3.1626*\dx},{0.0000*\dy})
	-- ({3.1726*\dx},{0.0000*\dy})
	-- ({3.1827*\dx},{0.0000*\dy})
	-- ({3.1927*\dx},{0.0000*\dy})
	-- ({3.2027*\dx},{0.0000*\dy})
	-- ({3.2127*\dx},{0.0000*\dy})
	-- ({3.2227*\dx},{0.0000*\dy})
	-- ({3.2327*\dx},{0.0000*\dy})
	-- ({3.2427*\dx},{0.0000*\dy})
	-- ({3.2527*\dx},{0.0000*\dy})
	-- ({3.2627*\dx},{0.0000*\dy})
	-- ({3.2727*\dx},{0.0000*\dy})
	-- ({3.2827*\dx},{0.0000*\dy})
	-- ({3.2927*\dx},{0.0000*\dy})
	-- ({3.3028*\dx},{0.0000*\dy})
	-- ({3.3128*\dx},{0.0000*\dy})
	-- ({3.3228*\dx},{0.0000*\dy})
	-- ({3.3328*\dx},{0.0000*\dy})
	-- ({3.3428*\dx},{0.0000*\dy})
	-- ({3.3528*\dx},{0.0000*\dy})
	-- ({3.3628*\dx},{0.0000*\dy})
	-- ({3.3728*\dx},{0.0000*\dy})
	-- ({3.3828*\dx},{0.0000*\dy})
	-- ({3.3928*\dx},{0.0000*\dy})
	-- ({3.4028*\dx},{0.0000*\dy})
	-- ({3.4128*\dx},{0.0000*\dy})
	-- ({3.4229*\dx},{0.0000*\dy})
	-- ({3.4329*\dx},{0.0000*\dy})
	-- ({3.4429*\dx},{0.0000*\dy})
	-- ({3.4529*\dx},{0.0000*\dy})
	-- ({3.4629*\dx},{0.0000*\dy})
	-- ({3.4729*\dx},{0.0000*\dy})
	-- ({3.4829*\dx},{0.0000*\dy})
	-- ({3.4929*\dx},{0.0000*\dy})
	-- ({3.5029*\dx},{0.0000*\dy})
	-- ({3.5129*\dx},{0.0000*\dy})
	-- ({3.5229*\dx},{0.0000*\dy})
	-- ({3.5329*\dx},{0.0000*\dy})
	-- ({3.5430*\dx},{0.0000*\dy})
	-- ({3.5530*\dx},{0.0000*\dy})
	-- ({3.5630*\dx},{0.0000*\dy})
	-- ({3.5730*\dx},{0.0000*\dy})
	-- ({3.5830*\dx},{0.0000*\dy})
	-- ({3.5930*\dx},{0.0000*\dy})
	-- ({3.6030*\dx},{0.0000*\dy})
	-- ({3.6130*\dx},{0.0000*\dy})
	-- ({3.6230*\dx},{0.0000*\dy})
	-- ({3.6330*\dx},{0.0000*\dy})
	-- ({3.6430*\dx},{0.0000*\dy})
	-- ({3.6530*\dx},{0.0000*\dy})
	-- ({3.6631*\dx},{0.0000*\dy})
	-- ({3.6731*\dx},{0.0000*\dy})
	-- ({3.6831*\dx},{0.0000*\dy})
	-- ({3.6931*\dx},{0.0000*\dy})
	-- ({3.7031*\dx},{0.0000*\dy})
	-- ({3.7131*\dx},{0.0000*\dy})
	-- ({3.7231*\dx},{0.0000*\dy})
	-- ({3.7331*\dx},{0.0000*\dy})
	-- ({3.7431*\dx},{0.0000*\dy})
	-- ({3.7531*\dx},{0.0000*\dy})
	-- ({3.7631*\dx},{0.0000*\dy})
	-- ({3.7731*\dx},{0.0000*\dy})
	-- ({3.7832*\dx},{0.0000*\dy})
	-- ({3.7932*\dx},{0.0000*\dy})
	-- ({3.8032*\dx},{0.0000*\dy})
	-- ({3.8132*\dx},{0.0000*\dy})
	-- ({3.8232*\dx},{0.0000*\dy})
	-- ({3.8332*\dx},{0.0000*\dy})
	-- ({3.8432*\dx},{0.0000*\dy})
	-- ({3.8532*\dx},{0.0000*\dy})
	-- ({3.8632*\dx},{0.0000*\dy})
	-- ({3.8732*\dx},{0.0000*\dy})
	-- ({3.8832*\dx},{0.0000*\dy})
	-- ({3.8932*\dx},{0.0000*\dy})
	-- ({3.9033*\dx},{0.0000*\dy})
	-- ({3.9133*\dx},{0.0000*\dy})
	-- ({3.9233*\dx},{0.0000*\dy})
	-- ({3.9333*\dx},{0.0000*\dy})
	-- ({3.9433*\dx},{0.0000*\dy})
	-- ({3.9533*\dx},{0.0000*\dy})
	-- ({3.9633*\dx},{0.0000*\dy})
	-- ({3.9733*\dx},{0.0000*\dy})
	-- ({3.9833*\dx},{0.0000*\dy})
	-- ({3.9933*\dx},{0.0000*\dy})
	-- ({4.0033*\dx},{0.0000*\dy})
	-- ({4.0133*\dx},{0.0000*\dy})
	-- ({4.0234*\dx},{0.0000*\dy})
	-- ({4.0334*\dx},{0.0000*\dy})
	-- ({4.0434*\dx},{0.0000*\dy})
	-- ({4.0534*\dx},{0.0000*\dy})
	-- ({4.0634*\dx},{0.0000*\dy})
	-- ({4.0734*\dx},{0.0000*\dy})
	-- ({4.0834*\dx},{0.0000*\dy})
	-- ({4.0934*\dx},{0.0000*\dy})
	-- ({4.1034*\dx},{0.0000*\dy})
	-- ({4.1134*\dx},{0.0000*\dy})
	-- ({4.1234*\dx},{0.0000*\dy})
	-- ({4.1334*\dx},{0.0000*\dy})
	-- ({4.1435*\dx},{0.0000*\dy})
	-- ({4.1535*\dx},{0.0000*\dy})
	-- ({4.1635*\dx},{0.0000*\dy})
	-- ({4.1735*\dx},{0.0000*\dy})
	-- ({4.1835*\dx},{0.0000*\dy})
	-- ({4.1935*\dx},{0.0000*\dy})
	-- ({4.2035*\dx},{0.0000*\dy})
	-- ({4.2135*\dx},{0.0000*\dy})
	-- ({4.2235*\dx},{0.0000*\dy})
	-- ({4.2335*\dx},{0.0000*\dy})
	-- ({4.2435*\dx},{0.0000*\dy})
	-- ({4.2535*\dx},{0.0000*\dy})
	-- ({4.2636*\dx},{0.0000*\dy})
	-- ({4.2736*\dx},{0.0000*\dy})
	-- ({4.2836*\dx},{0.0000*\dy})
	-- ({4.2936*\dx},{0.0000*\dy})
	-- ({4.3036*\dx},{0.0000*\dy})
	-- ({4.3136*\dx},{0.0000*\dy})
	-- ({4.3236*\dx},{0.0000*\dy})
	-- ({4.3336*\dx},{0.0000*\dy})
	-- ({4.3436*\dx},{0.0000*\dy})
	-- ({4.3536*\dx},{0.0000*\dy})
	-- ({4.3636*\dx},{0.0000*\dy})
	-- ({4.3736*\dx},{0.0000*\dy})
	-- ({4.3837*\dx},{0.0000*\dy})
	-- ({4.3937*\dx},{0.0000*\dy})
	-- ({4.4037*\dx},{0.0000*\dy})
	-- ({4.4137*\dx},{0.0000*\dy})
	-- ({4.4237*\dx},{0.0000*\dy})
	-- ({4.4337*\dx},{0.0000*\dy})
	-- ({4.4437*\dx},{0.0000*\dy})
	-- ({4.4537*\dx},{0.0000*\dy})
	-- ({4.4637*\dx},{0.0000*\dy})
	-- ({4.4737*\dx},{0.0000*\dy})
	-- ({4.4837*\dx},{0.0000*\dy})
	-- ({4.4937*\dx},{0.0000*\dy})
	-- ({4.5038*\dx},{0.0000*\dy})
	-- ({4.5138*\dx},{0.0000*\dy})
	-- ({4.5238*\dx},{0.0000*\dy})
	-- ({4.5338*\dx},{0.0000*\dy})
	-- ({4.5438*\dx},{0.0000*\dy})
	-- ({4.5538*\dx},{0.0000*\dy})
	-- ({4.5638*\dx},{0.0000*\dy})
	-- ({4.5738*\dx},{0.0000*\dy})
	-- ({4.5838*\dx},{0.0000*\dy})
	-- ({4.5938*\dx},{0.0000*\dy})
	-- ({4.6038*\dx},{0.0000*\dy})
	-- ({4.6138*\dx},{0.0000*\dy})
	-- ({4.6239*\dx},{0.0000*\dy})
	-- ({4.6339*\dx},{0.0000*\dy})
	-- ({4.6439*\dx},{0.0000*\dy})
	-- ({4.6539*\dx},{0.0000*\dy})
	-- ({4.6639*\dx},{0.0000*\dy})
	-- ({4.6739*\dx},{0.0000*\dy})
	-- ({4.6839*\dx},{0.0000*\dy})
	-- ({4.6939*\dx},{0.0000*\dy})
	-- ({4.7039*\dx},{0.0000*\dy})
	-- ({4.7139*\dx},{0.0000*\dy})
	-- ({4.7239*\dx},{0.0000*\dy})
	-- ({4.7339*\dx},{0.0000*\dy})
	-- ({4.7440*\dx},{0.0000*\dy})
	-- ({4.7540*\dx},{0.0000*\dy})
	-- ({4.7640*\dx},{0.0000*\dy})
	-- ({4.7740*\dx},{0.0000*\dy})
	-- ({4.7840*\dx},{0.0000*\dy})
	-- ({4.7940*\dx},{0.0000*\dy})
	-- ({4.8040*\dx},{0.0000*\dy})
	-- ({4.8140*\dx},{0.0000*\dy})
	-- ({4.8240*\dx},{0.0000*\dy})
	-- ({4.8340*\dx},{0.0000*\dy})
	-- ({4.8440*\dx},{0.0000*\dy})
	-- ({4.8540*\dx},{0.0000*\dy})
	-- ({4.8641*\dx},{0.0000*\dy})
	-- ({4.8741*\dx},{0.0000*\dy})
	-- ({4.8841*\dx},{0.0000*\dy})
	-- ({4.8941*\dx},{0.0000*\dy})
	-- ({4.9041*\dx},{0.0000*\dy})
	-- ({4.9141*\dx},{0.0000*\dy})
	-- ({4.9241*\dx},{0.0000*\dy})
	-- ({4.9341*\dx},{0.0000*\dy})
	-- ({4.9441*\dx},{0.0000*\dy})
	-- ({4.9541*\dx},{0.0000*\dy})
	-- ({4.9641*\dx},{0.0000*\dy})
	-- ({4.9741*\dx},{0.0000*\dy})
	-- ({4.9842*\dx},{0.0000*\dy})
	-- ({4.9942*\dx},{0.0000*\dy})
	-- ({5.0042*\dx},{0.0000*\dy})
	-- ({5.0142*\dx},{0.0000*\dy})
	-- ({5.0242*\dx},{0.0000*\dy})
	-- ({5.0342*\dx},{0.0000*\dy})
	-- ({5.0442*\dx},{0.0000*\dy})
	-- ({5.0542*\dx},{0.0000*\dy})
	-- ({5.0642*\dx},{0.0000*\dy})
	-- ({5.0742*\dx},{0.0000*\dy})
	-- ({5.0842*\dx},{0.0000*\dy})
	-- ({5.0942*\dx},{0.0000*\dy})
	-- ({5.1043*\dx},{0.0000*\dy})
	-- ({5.1143*\dx},{0.0000*\dy})
	-- ({5.1243*\dx},{0.0000*\dy})
	-- ({5.1343*\dx},{0.0000*\dy})
	-- ({5.1443*\dx},{0.0000*\dy})
	-- ({5.1543*\dx},{0.0000*\dy})
	-- ({5.1643*\dx},{0.0000*\dy})
	-- ({5.1743*\dx},{0.0000*\dy})
	-- ({5.1843*\dx},{0.0000*\dy})
	-- ({5.1943*\dx},{0.0000*\dy})
	-- ({5.2043*\dx},{0.0000*\dy})
	-- ({5.2143*\dx},{0.0000*\dy})
	-- ({5.2244*\dx},{0.0000*\dy})
	-- ({5.2344*\dx},{0.0000*\dy})
	-- ({5.2444*\dx},{0.0000*\dy})
	-- ({5.2544*\dx},{0.0000*\dy})
	-- ({5.2644*\dx},{0.0000*\dy})
	-- ({5.2744*\dx},{0.0000*\dy})
	-- ({5.2844*\dx},{0.0000*\dy})
	-- ({5.2944*\dx},{0.0000*\dy})
	-- ({5.3044*\dx},{0.0000*\dy})
	-- ({5.3144*\dx},{0.0000*\dy})
	-- ({5.3244*\dx},{0.0000*\dy})
	-- ({5.3344*\dx},{0.0000*\dy})
	-- ({5.3445*\dx},{0.0000*\dy})
	-- ({5.3545*\dx},{0.0000*\dy})
	-- ({5.3645*\dx},{0.0000*\dy})
	-- ({5.3745*\dx},{0.0000*\dy})
	-- ({5.3845*\dx},{0.0000*\dy})
	-- ({5.3945*\dx},{0.0000*\dy})
	-- ({5.4045*\dx},{0.0000*\dy})
	-- ({5.4145*\dx},{0.0000*\dy})
	-- ({5.4245*\dx},{0.0000*\dy})
	-- ({5.4345*\dx},{0.0000*\dy})
	-- ({5.4445*\dx},{0.0000*\dy})
	-- ({5.4545*\dx},{0.0000*\dy})
	-- ({5.4646*\dx},{0.0000*\dy})
	-- ({5.4746*\dx},{0.0000*\dy})
	-- ({5.4846*\dx},{0.0000*\dy})
	-- ({5.4946*\dx},{0.0000*\dy})
	-- ({5.5046*\dx},{0.0000*\dy})
	-- ({5.5146*\dx},{0.0000*\dy})
	-- ({5.5246*\dx},{0.0000*\dy})
	-- ({5.5346*\dx},{0.0000*\dy})
	-- ({5.5446*\dx},{0.0000*\dy})
	-- ({5.5546*\dx},{0.0000*\dy})
	-- ({5.5646*\dx},{0.0000*\dy})
	-- ({5.5746*\dx},{0.0000*\dy})
	-- ({5.5847*\dx},{0.0000*\dy})
	-- ({5.5947*\dx},{0.0000*\dy})
	-- ({5.6047*\dx},{0.0000*\dy})
	-- ({5.6147*\dx},{0.0000*\dy})
	-- ({5.6247*\dx},{0.0000*\dy})
	-- ({5.6347*\dx},{0.0000*\dy})
	-- ({5.6447*\dx},{0.0000*\dy})
	-- ({5.6547*\dx},{0.0000*\dy})
	-- ({5.6647*\dx},{0.0000*\dy})
	-- ({5.6747*\dx},{0.0000*\dy})
	-- ({5.6847*\dx},{0.0000*\dy})
	-- ({5.6947*\dx},{0.0000*\dy})
	-- ({5.7048*\dx},{0.0000*\dy})
	-- ({5.7148*\dx},{0.0000*\dy})
	-- ({5.7248*\dx},{0.0000*\dy})
	-- ({5.7348*\dx},{0.0000*\dy})
	-- ({5.7448*\dx},{0.0000*\dy})
	-- ({5.7548*\dx},{0.0000*\dy})
	-- ({5.7648*\dx},{0.0000*\dy})
	-- ({5.7748*\dx},{0.0000*\dy})
	-- ({5.7848*\dx},{0.0000*\dy})
	-- ({5.7948*\dx},{0.0000*\dy})
	-- ({5.8048*\dx},{0.0000*\dy})
	-- ({5.8148*\dx},{0.0000*\dy})
	-- ({5.8249*\dx},{0.0000*\dy})
	-- ({5.8349*\dx},{0.0000*\dy})
	-- ({5.8449*\dx},{0.0000*\dy})
	-- ({5.8549*\dx},{0.0000*\dy})
	-- ({5.8649*\dx},{0.0000*\dy})
	-- ({5.8749*\dx},{0.0000*\dy})
	-- ({5.8849*\dx},{0.0000*\dy})
	-- ({5.8949*\dx},{0.0000*\dy})
	-- ({5.9049*\dx},{0.0000*\dy})
	-- ({5.9149*\dx},{0.0000*\dy})
	-- ({5.9249*\dx},{0.0000*\dy})
	-- ({5.9349*\dx},{0.0000*\dy})
	-- ({5.9450*\dx},{0.0000*\dy})
	-- ({5.9550*\dx},{0.0000*\dy})
	-- ({5.9650*\dx},{0.0000*\dy})
	-- ({5.9750*\dx},{0.0000*\dy})
	-- ({5.9850*\dx},{0.0000*\dy})
	-- ({5.9950*\dx},{0.0000*\dy})
	-- ({6.0050*\dx},{0.0000*\dy})
	-- ({6.0150*\dx},{0.0000*\dy})
	-- ({6.0250*\dx},{0.0000*\dy})
	-- ({6.0350*\dx},{0.0000*\dy})
	-- ({6.0450*\dx},{0.0000*\dy})
	-- ({6.0550*\dx},{0.0000*\dy})
	-- ({6.0651*\dx},{0.0000*\dy})
	-- ({6.0751*\dx},{0.0000*\dy})
	-- ({6.0851*\dx},{0.0000*\dy})
	-- ({6.0951*\dx},{0.0000*\dy})
	-- ({6.1051*\dx},{0.0000*\dy})
	-- ({6.1151*\dx},{0.0000*\dy})
	-- ({6.1251*\dx},{0.0000*\dy})
	-- ({6.1351*\dx},{0.0000*\dy})
	-- ({6.1451*\dx},{0.0000*\dy})
	-- ({6.1551*\dx},{0.0000*\dy})
	-- ({6.1651*\dx},{0.0000*\dy})
	-- ({6.1751*\dx},{0.0000*\dy})
	-- ({6.1852*\dx},{0.0000*\dy})
	-- ({6.1952*\dx},{0.0000*\dy})
	-- ({6.2052*\dx},{0.0000*\dy})
	-- ({6.2152*\dx},{0.0000*\dy})
	-- ({6.2252*\dx},{0.0000*\dy})
	-- ({6.2352*\dx},{0.0000*\dy})
	-- ({6.2452*\dx},{0.0000*\dy})
	-- ({6.2552*\dx},{0.0000*\dy})
	-- ({6.2652*\dx},{0.0000*\dy})
	-- ({6.2752*\dx},{0.0000*\dy})
	-- ({6.2852*\dx},{0.0000*\dy})
	-- ({6.2952*\dx},{0.0000*\dy})
	-- ({6.3053*\dx},{0.0000*\dy})
	-- ({6.3153*\dx},{0.0000*\dy})
	-- ({6.3253*\dx},{0.0000*\dy})
	-- ({6.3353*\dx},{0.0000*\dy})
	-- ({6.3453*\dx},{0.0000*\dy})
	-- ({6.3553*\dx},{0.0000*\dy})
	-- ({6.3653*\dx},{0.0000*\dy})
	-- ({6.3753*\dx},{0.0000*\dy})
	-- ({6.3853*\dx},{0.0000*\dy})
	-- ({6.3953*\dx},{0.0000*\dy})
	-- ({6.4053*\dx},{0.0000*\dy})
	-- ({6.4153*\dx},{0.0000*\dy})
	-- ({6.4254*\dx},{0.0000*\dy})
	-- ({6.4354*\dx},{0.0000*\dy})
	-- ({6.4454*\dx},{0.0000*\dy})
	-- ({6.4554*\dx},{0.0000*\dy})
	-- ({6.4654*\dx},{0.0000*\dy})
	-- ({6.4754*\dx},{0.0000*\dy})
	-- ({6.4854*\dx},{0.0000*\dy})
	-- ({6.4954*\dx},{0.0000*\dy})
	-- ({6.5054*\dx},{0.0000*\dy})
	-- ({6.5154*\dx},{0.0000*\dy})
	-- ({6.5254*\dx},{0.0000*\dy})
	-- ({6.5354*\dx},{0.0000*\dy})
	-- ({6.5455*\dx},{0.0000*\dy})
	-- ({6.5555*\dx},{0.0000*\dy})
	-- ({6.5655*\dx},{0.0000*\dy})
	-- ({6.5755*\dx},{0.0000*\dy})
	-- ({6.5855*\dx},{0.0000*\dy})
	-- ({6.5955*\dx},{0.0000*\dy})
	-- ({6.6055*\dx},{0.0000*\dy})
	-- ({6.6155*\dx},{0.0000*\dy})
	-- ({6.6255*\dx},{0.0000*\dy})
	-- ({6.6355*\dx},{0.0000*\dy})
	-- ({6.6455*\dx},{0.0000*\dy})
	-- ({6.6555*\dx},{0.0000*\dy})
	-- ({6.6656*\dx},{0.0000*\dy})
	-- ({6.6756*\dx},{0.0000*\dy})
	-- ({6.6856*\dx},{0.0000*\dy})
	-- ({6.6956*\dx},{0.0000*\dy})
	-- ({6.7056*\dx},{0.0000*\dy})
	-- ({6.7156*\dx},{0.0000*\dy})
	-- ({6.7256*\dx},{0.0000*\dy})
	-- ({6.7356*\dx},{0.0000*\dy})
	-- ({6.7456*\dx},{0.0000*\dy})
	-- ({6.7556*\dx},{0.0000*\dy})
	-- ({6.7656*\dx},{0.0000*\dy})
	-- ({6.7756*\dx},{0.0000*\dy})
	-- ({6.7857*\dx},{0.0000*\dy})
	-- ({6.7957*\dx},{0.0000*\dy})
	-- ({6.8057*\dx},{0.0000*\dy})
	-- ({6.8157*\dx},{0.0000*\dy})
	-- ({6.8257*\dx},{0.0000*\dy})
	-- ({6.8357*\dx},{0.0000*\dy})
	-- ({6.8457*\dx},{0.0000*\dy})
	-- ({6.8557*\dx},{0.0000*\dy})
	-- ({6.8657*\dx},{0.0000*\dy})
	-- ({6.8757*\dx},{0.0000*\dy})
	-- ({6.8857*\dx},{0.0000*\dy})
	-- ({6.8957*\dx},{0.0000*\dy})
	-- ({6.9058*\dx},{0.0000*\dy})
	-- ({6.9158*\dx},{0.0000*\dy})
	-- ({6.9258*\dx},{0.0000*\dy})
	-- ({6.9358*\dx},{0.0000*\dy})
	-- ({6.9458*\dx},{0.0000*\dy})
	-- ({6.9558*\dx},{0.0000*\dy})
	-- ({6.9658*\dx},{0.0000*\dy})
	-- ({6.9758*\dx},{0.0000*\dy})
	-- ({6.9858*\dx},{0.0000*\dy})
	-- ({6.9958*\dx},{0.0000*\dy})
	-- ({7.0058*\dx},{0.0000*\dy})
	-- ({7.0158*\dx},{0.0000*\dy})
	-- ({7.0259*\dx},{0.0000*\dy})
	-- ({7.0359*\dx},{0.0000*\dy})
	-- ({7.0459*\dx},{0.0000*\dy})
	-- ({7.0559*\dx},{0.0000*\dy})
	-- ({7.0659*\dx},{0.0000*\dy})
	-- ({7.0759*\dx},{0.0000*\dy})
	-- ({7.0859*\dx},{0.0000*\dy})
	-- ({7.0959*\dx},{0.0000*\dy})
	-- ({7.1059*\dx},{0.0000*\dy})
	-- ({7.1159*\dx},{0.0000*\dy})
	-- ({7.1259*\dx},{0.0000*\dy})
	-- ({7.1359*\dx},{0.0000*\dy})
	-- ({7.1460*\dx},{0.0000*\dy})
	-- ({7.1560*\dx},{0.0000*\dy})
	-- ({7.1660*\dx},{0.0000*\dy})
	-- ({7.1760*\dx},{0.0000*\dy})
	-- ({7.1860*\dx},{0.0000*\dy})
	-- ({7.1960*\dx},{0.0000*\dy})
	-- ({7.2060*\dx},{0.0000*\dy})
	-- ({7.2160*\dx},{0.0000*\dy})
	-- ({7.2260*\dx},{0.0000*\dy})
	-- ({7.2360*\dx},{0.0000*\dy})
	-- ({7.2460*\dx},{0.0000*\dy})
	-- ({7.2560*\dx},{0.0000*\dy})
	-- ({7.2661*\dx},{0.0000*\dy})
	-- ({7.2761*\dx},{0.0000*\dy})
	-- ({7.2861*\dx},{0.0000*\dy})
	-- ({7.2961*\dx},{0.0000*\dy})
	-- ({7.3061*\dx},{0.0000*\dy})
	-- ({7.3161*\dx},{0.0000*\dy})
	-- ({7.3261*\dx},{0.0000*\dy})
	-- ({7.3361*\dx},{0.0000*\dy})
	-- ({7.3461*\dx},{0.0000*\dy})
	-- ({7.3561*\dx},{0.0000*\dy})
	-- ({7.3661*\dx},{0.0000*\dy})
	-- ({7.3761*\dx},{0.0000*\dy})
	-- ({7.3862*\dx},{0.0000*\dy})
	-- ({7.3962*\dx},{0.0000*\dy})
	-- ({7.4062*\dx},{0.0000*\dy})
	-- ({7.4162*\dx},{0.0000*\dy})
	-- ({7.4262*\dx},{0.0000*\dy})
	-- ({7.4362*\dx},{0.0000*\dy})
	-- ({7.4462*\dx},{0.0000*\dy})
	-- ({7.4562*\dx},{0.0000*\dy})
	-- ({7.4662*\dx},{0.0000*\dy})
	-- ({7.4762*\dx},{0.0000*\dy})
	-- ({7.4862*\dx},{0.0000*\dy})
	-- ({7.4962*\dx},{0.0000*\dy})
	-- ({7.5063*\dx},{0.0000*\dy})
	-- ({7.5163*\dx},{0.0000*\dy})
	-- ({7.5263*\dx},{0.0000*\dy})
	-- ({7.5363*\dx},{0.0000*\dy})
	-- ({7.5463*\dx},{0.0000*\dy})
	-- ({7.5563*\dx},{0.0000*\dy})
	-- ({7.5663*\dx},{0.0000*\dy})
	-- ({7.5763*\dx},{0.0000*\dy})
	-- ({7.5863*\dx},{0.0000*\dy})
	-- ({7.5963*\dx},{0.0000*\dy})
	-- ({7.6063*\dx},{0.0000*\dy})
	-- ({7.6163*\dx},{0.0000*\dy})
	-- ({7.6264*\dx},{0.0000*\dy})
	-- ({7.6364*\dx},{0.0000*\dy})
	-- ({7.6464*\dx},{0.0000*\dy})
	-- ({7.6564*\dx},{0.0000*\dy})
	-- ({7.6664*\dx},{0.0000*\dy})
	-- ({7.6764*\dx},{0.0000*\dy})
	-- ({7.6864*\dx},{0.0000*\dy})
	-- ({7.6964*\dx},{0.0000*\dy})
	-- ({7.7064*\dx},{0.0000*\dy})
	-- ({7.7164*\dx},{0.0000*\dy})
	-- ({7.7264*\dx},{0.0000*\dy})
	-- ({7.7364*\dx},{0.0000*\dy})
	-- ({7.7465*\dx},{0.0000*\dy})
	-- ({7.7565*\dx},{0.0000*\dy})
	-- ({7.7665*\dx},{0.0000*\dy})
	-- ({7.7765*\dx},{0.0000*\dy})
	-- ({7.7865*\dx},{0.0000*\dy})
	-- ({7.7965*\dx},{0.0000*\dy})
	-- ({7.8065*\dx},{0.0000*\dy})
	-- ({7.8165*\dx},{0.0000*\dy})
	-- ({7.8265*\dx},{0.0000*\dy})
	-- ({7.8365*\dx},{0.0000*\dy})
	-- ({7.8465*\dx},{0.0000*\dy})
	-- ({7.8565*\dx},{0.0000*\dy})
	-- ({7.8666*\dx},{0.0000*\dy})
	-- ({7.8766*\dx},{0.0000*\dy})
	-- ({7.8866*\dx},{0.0000*\dy})
	-- ({7.8966*\dx},{0.0000*\dy})
	-- ({7.9066*\dx},{0.0000*\dy})
	-- ({7.9166*\dx},{0.0000*\dy})
	-- ({7.9266*\dx},{0.0000*\dy})
	-- ({7.9366*\dx},{0.0000*\dy})
	-- ({7.9466*\dx},{0.0000*\dy})
	-- ({7.9566*\dx},{0.0000*\dy})
	-- ({7.9666*\dx},{0.0000*\dy})
	-- ({7.9766*\dx},{0.0000*\dy})
	-- ({7.9867*\dx},{0.0000*\dy})
	-- ({7.9967*\dx},{0.0000*\dy})
	-- ({8.0067*\dx},{0.0000*\dy})
	-- ({8.0167*\dx},{0.0000*\dy})
	-- ({8.0267*\dx},{0.0000*\dy})
	-- ({8.0367*\dx},{0.0000*\dy})
	-- ({8.0467*\dx},{0.0000*\dy})
	-- ({8.0567*\dx},{0.0000*\dy})
	-- ({8.0667*\dx},{0.0000*\dy})
	-- ({8.0767*\dx},{0.0000*\dy})
	-- ({8.0867*\dx},{0.0000*\dy})
	-- ({8.0967*\dx},{0.0000*\dy})
	-- ({8.1068*\dx},{0.0000*\dy})
	-- ({8.1168*\dx},{0.0000*\dy})
	-- ({8.1268*\dx},{0.0000*\dy})
	-- ({8.1368*\dx},{0.0000*\dy})
	-- ({8.1468*\dx},{0.0000*\dy})
	-- ({8.1568*\dx},{0.0000*\dy})
	-- ({8.1668*\dx},{0.0000*\dy})
	-- ({8.1768*\dx},{0.0000*\dy})
	-- ({8.1868*\dx},{0.0000*\dy})
	-- ({8.1968*\dx},{0.0000*\dy})
	-- ({8.2068*\dx},{0.0000*\dy})
	-- ({8.2168*\dx},{0.0000*\dy})
	-- ({8.2269*\dx},{0.0000*\dy})
	-- ({8.2369*\dx},{0.0000*\dy})
	-- ({8.2469*\dx},{0.0000*\dy})
	-- ({8.2569*\dx},{0.0000*\dy})
	-- ({8.2669*\dx},{0.0000*\dy})
	-- ({8.2769*\dx},{0.0000*\dy})
	-- ({8.2869*\dx},{0.0000*\dy})
	-- ({8.2969*\dx},{0.0000*\dy})
	-- ({8.3069*\dx},{0.0000*\dy})
	-- ({8.3169*\dx},{0.0000*\dy})
	-- ({8.3269*\dx},{0.0000*\dy})
	-- ({8.3369*\dx},{0.0000*\dy})
	-- ({8.3470*\dx},{0.0000*\dy})
	-- ({8.3570*\dx},{0.0000*\dy})
	-- ({8.3670*\dx},{0.0000*\dy})
	-- ({8.3770*\dx},{0.0000*\dy})
	-- ({8.3870*\dx},{0.0000*\dy})
	-- ({8.3970*\dx},{0.0000*\dy})
	-- ({8.4070*\dx},{0.0000*\dy})
	-- ({8.4170*\dx},{0.0000*\dy})
	-- ({8.4270*\dx},{0.0000*\dy})
	-- ({8.4370*\dx},{0.0000*\dy})
	-- ({8.4470*\dx},{0.0000*\dy})
	-- ({8.4570*\dx},{0.0000*\dy})
	-- ({8.4671*\dx},{0.0000*\dy})
	-- ({8.4771*\dx},{0.0000*\dy})
	-- ({8.4871*\dx},{0.0000*\dy})
	-- ({8.4971*\dx},{0.0000*\dy})
	-- ({8.5071*\dx},{0.0000*\dy})
	-- ({8.5171*\dx},{0.0000*\dy})
	-- ({8.5271*\dx},{0.0000*\dy})
	-- ({8.5371*\dx},{0.0000*\dy})
	-- ({8.5471*\dx},{0.0000*\dy})
	-- ({8.5571*\dx},{0.0000*\dy})
	-- ({8.5671*\dx},{0.0000*\dy})
	-- ({8.5771*\dx},{0.0000*\dy})
	-- ({8.5872*\dx},{0.0000*\dy})
	-- ({8.5972*\dx},{0.0000*\dy})
	-- ({8.6072*\dx},{0.0000*\dy})
	-- ({8.6172*\dx},{0.0000*\dy})
	-- ({8.6272*\dx},{0.0000*\dy})
	-- ({8.6372*\dx},{0.0000*\dy})
	-- ({8.6472*\dx},{0.0000*\dy})
	-- ({8.6572*\dx},{0.0000*\dy})
	-- ({8.6672*\dx},{0.0000*\dy})
	-- ({8.6772*\dx},{0.0000*\dy})
	-- ({8.6872*\dx},{0.0000*\dy})
	-- ({8.6972*\dx},{0.0000*\dy})
	-- ({8.7073*\dx},{0.0000*\dy})
	-- ({8.7173*\dx},{0.0000*\dy})
	-- ({8.7273*\dx},{0.0000*\dy})
	-- ({8.7373*\dx},{0.0000*\dy})
	-- ({8.7473*\dx},{0.0000*\dy})
	-- ({8.7573*\dx},{0.0000*\dy})
	-- ({8.7673*\dx},{0.0000*\dy})
	-- ({8.7773*\dx},{0.0000*\dy})
	-- ({8.7873*\dx},{0.0000*\dy})
	-- ({8.7973*\dx},{0.0000*\dy})
	-- ({8.8073*\dx},{0.0000*\dy})
	-- ({8.8173*\dx},{0.0000*\dy})
	-- ({8.8274*\dx},{0.0000*\dy})
	-- ({8.8374*\dx},{0.0000*\dy})
	-- ({8.8474*\dx},{0.0000*\dy})
	-- ({8.8574*\dx},{0.0000*\dy})
	-- ({8.8674*\dx},{0.0000*\dy})
	-- ({8.8774*\dx},{0.0000*\dy})
	-- ({8.8874*\dx},{0.0000*\dy})
	-- ({8.8974*\dx},{0.0000*\dy})
	-- ({8.9074*\dx},{0.0000*\dy})
	-- ({8.9174*\dx},{0.0000*\dy})
	-- ({8.9274*\dx},{0.0000*\dy})
	-- ({8.9374*\dx},{0.0000*\dy})
	-- ({8.9475*\dx},{0.0000*\dy})
	-- ({8.9575*\dx},{0.0000*\dy})
	-- ({8.9675*\dx},{0.0000*\dy})
	-- ({8.9775*\dx},{0.0000*\dy})
	-- ({8.9875*\dx},{0.0000*\dy})
	-- ({8.9975*\dx},{0.0000*\dy})
	-- ({9.0075*\dx},{0.0000*\dy})
	-- ({9.0175*\dx},{0.0001*\dy})
	-- ({9.0275*\dx},{0.0002*\dy})
	-- ({9.0375*\dx},{0.0003*\dy})
	-- ({9.0475*\dx},{0.0006*\dy})
	-- ({9.0575*\dx},{0.0008*\dy})
	-- ({9.0676*\dx},{0.0011*\dy})
	-- ({9.0776*\dx},{0.0015*\dy})
	-- ({9.0876*\dx},{0.0019*\dy})
	-- ({9.0976*\dx},{0.0023*\dy})
	-- ({9.1076*\dx},{0.0028*\dy})
	-- ({9.1176*\dx},{0.0034*\dy})
	-- ({9.1276*\dx},{0.0039*\dy})
	-- ({9.1376*\dx},{0.0046*\dy})
	-- ({9.1476*\dx},{0.0052*\dy})
	-- ({9.1576*\dx},{0.0060*\dy})
	-- ({9.1676*\dx},{0.0067*\dy})
	-- ({9.1776*\dx},{0.0075*\dy})
	-- ({9.1877*\dx},{0.0084*\dy})
	-- ({9.1977*\dx},{0.0093*\dy})
	-- ({9.2077*\dx},{0.0103*\dy})
	-- ({9.2177*\dx},{0.0112*\dy})
	-- ({9.2277*\dx},{0.0123*\dy})
	-- ({9.2377*\dx},{0.0134*\dy})
	-- ({9.2477*\dx},{0.0145*\dy})
	-- ({9.2577*\dx},{0.0156*\dy})
	-- ({9.2677*\dx},{0.0168*\dy})
	-- ({9.2777*\dx},{0.0181*\dy})
	-- ({9.2877*\dx},{0.0194*\dy})
	-- ({9.2977*\dx},{0.0207*\dy})
	-- ({9.3078*\dx},{0.0221*\dy})
	-- ({9.3178*\dx},{0.0236*\dy})
	-- ({9.3278*\dx},{0.0250*\dy})
	-- ({9.3378*\dx},{0.0266*\dy})
	-- ({9.3478*\dx},{0.0281*\dy})
	-- ({9.3578*\dx},{0.0297*\dy})
	-- ({9.3678*\dx},{0.0314*\dy})
	-- ({9.3778*\dx},{0.0331*\dy})
	-- ({9.3878*\dx},{0.0348*\dy})
	-- ({9.3978*\dx},{0.0366*\dy})
	-- ({9.4078*\dx},{0.0385*\dy})
	-- ({9.4178*\dx},{0.0403*\dy})
	-- ({9.4279*\dx},{0.0423*\dy})
	-- ({9.4379*\dx},{0.0443*\dy})
	-- ({9.4479*\dx},{0.0463*\dy})
	-- ({9.4579*\dx},{0.0484*\dy})
	-- ({9.4679*\dx},{0.0505*\dy})
	-- ({9.4779*\dx},{0.0527*\dy})
	-- ({9.4879*\dx},{0.0549*\dy})
	-- ({9.4979*\dx},{0.0571*\dy})
	-- ({9.5079*\dx},{0.0595*\dy})
	-- ({9.5179*\dx},{0.0618*\dy})
	-- ({9.5279*\dx},{0.0643*\dy})
	-- ({9.5379*\dx},{0.0667*\dy})
	-- ({9.5480*\dx},{0.0693*\dy})
	-- ({9.5580*\dx},{0.0719*\dy})
	-- ({9.5680*\dx},{0.0745*\dy})
	-- ({9.5780*\dx},{0.0772*\dy})
	-- ({9.5880*\dx},{0.0799*\dy})
	-- ({9.5980*\dx},{0.0827*\dy})
	-- ({9.6080*\dx},{0.0856*\dy})
	-- ({9.6180*\dx},{0.0885*\dy})
	-- ({9.6280*\dx},{0.0915*\dy})
	-- ({9.6380*\dx},{0.0945*\dy})
	-- ({9.6480*\dx},{0.0976*\dy})
	-- ({9.6580*\dx},{0.1008*\dy})
	-- ({9.6681*\dx},{0.1040*\dy})
	-- ({9.6781*\dx},{0.1072*\dy})
	-- ({9.6881*\dx},{0.1106*\dy})
	-- ({9.6981*\dx},{0.1140*\dy})
	-- ({9.7081*\dx},{0.1174*\dy})
	-- ({9.7181*\dx},{0.1210*\dy})
	-- ({9.7281*\dx},{0.1246*\dy})
	-- ({9.7381*\dx},{0.1282*\dy})
	-- ({9.7481*\dx},{0.1320*\dy})
	-- ({9.7581*\dx},{0.1357*\dy})
	-- ({9.7681*\dx},{0.1396*\dy})
	-- ({9.7781*\dx},{0.1435*\dy})
	-- ({9.7882*\dx},{0.1476*\dy})
	-- ({9.7982*\dx},{0.1516*\dy})
	-- ({9.8082*\dx},{0.1558*\dy})
	-- ({9.8182*\dx},{0.1600*\dy})
	-- ({9.8282*\dx},{0.1643*\dy})
	-- ({9.8382*\dx},{0.1687*\dy})
	-- ({9.8482*\dx},{0.1731*\dy})
	-- ({9.8582*\dx},{0.1776*\dy})
	-- ({9.8682*\dx},{0.1822*\dy})
	-- ({9.8782*\dx},{0.1869*\dy})
	-- ({9.8882*\dx},{0.1916*\dy})
	-- ({9.8982*\dx},{0.1965*\dy})
	-- ({9.9083*\dx},{0.2014*\dy})
	-- ({9.9183*\dx},{0.2063*\dy})
	-- ({9.9283*\dx},{0.2114*\dy})
	-- ({9.9383*\dx},{0.2165*\dy})
	-- ({9.9483*\dx},{0.2218*\dy})
	-- ({9.9583*\dx},{0.2271*\dy})
	-- ({9.9683*\dx},{0.2324*\dy})
	-- ({9.9783*\dx},{0.2379*\dy})
	-- ({9.9883*\dx},{0.2434*\dy})
	-- ({9.9983*\dx},{0.2491*\dy})
	-- ({10.0083*\dx},{0.2548*\dy})
	-- ({10.0183*\dx},{0.2606*\dy})
	-- ({10.0284*\dx},{0.2665*\dy})
	-- ({10.0384*\dx},{0.2725*\dy})
	-- ({10.0484*\dx},{0.2786*\dy})
	-- ({10.0584*\dx},{0.2848*\dy})
	-- ({10.0684*\dx},{0.2912*\dy})
	-- ({10.0784*\dx},{0.2976*\dy})
	-- ({10.0884*\dx},{0.3041*\dy})
	-- ({10.0984*\dx},{0.3108*\dy})
	-- ({10.1084*\dx},{0.3176*\dy})
	-- ({10.1184*\dx},{0.3245*\dy})
	-- ({10.1284*\dx},{0.3315*\dy})
	-- ({10.1384*\dx},{0.3386*\dy})
	-- ({10.1485*\dx},{0.3458*\dy})
	-- ({10.1585*\dx},{0.3531*\dy})
	-- ({10.1685*\dx},{0.3606*\dy})
	-- ({10.1785*\dx},{0.3682*\dy})
	-- ({10.1885*\dx},{0.3759*\dy})
	-- ({10.1985*\dx},{0.3837*\dy})
	-- ({10.2085*\dx},{0.3916*\dy})
	-- ({10.2185*\dx},{0.3996*\dy})
	-- ({10.2285*\dx},{0.4078*\dy})
	-- ({10.2385*\dx},{0.4161*\dy})
	-- ({10.2485*\dx},{0.4244*\dy})
	-- ({10.2585*\dx},{0.4329*\dy})
	-- ({10.2686*\dx},{0.4416*\dy})
	-- ({10.2786*\dx},{0.4503*\dy})
	-- ({10.2886*\dx},{0.4591*\dy})
	-- ({10.2986*\dx},{0.4681*\dy})
	-- ({10.3086*\dx},{0.4772*\dy})
	-- ({10.3186*\dx},{0.4863*\dy})
	-- ({10.3286*\dx},{0.4956*\dy})
	-- ({10.3386*\dx},{0.5050*\dy})
	-- ({10.3486*\dx},{0.5144*\dy})
	-- ({10.3586*\dx},{0.5240*\dy})
	-- ({10.3686*\dx},{0.5336*\dy})
	-- ({10.3786*\dx},{0.5434*\dy})
	-- ({10.3887*\dx},{0.5532*\dy})
	-- ({10.3987*\dx},{0.5631*\dy})
	-- ({10.4087*\dx},{0.5731*\dy})
	-- ({10.4187*\dx},{0.5832*\dy})
	-- ({10.4287*\dx},{0.5933*\dy})
	-- ({10.4387*\dx},{0.6034*\dy})
	-- ({10.4487*\dx},{0.6137*\dy})
	-- ({10.4587*\dx},{0.6239*\dy})
	-- ({10.4687*\dx},{0.6343*\dy})
	-- ({10.4787*\dx},{0.6446*\dy})
	-- ({10.4887*\dx},{0.6550*\dy})
	-- ({10.4987*\dx},{0.6654*\dy})
	-- ({10.5088*\dx},{0.6758*\dy})
	-- ({10.5188*\dx},{0.6862*\dy})
	-- ({10.5288*\dx},{0.6966*\dy})
	-- ({10.5388*\dx},{0.7069*\dy})
	-- ({10.5488*\dx},{0.7173*\dy})
	-- ({10.5588*\dx},{0.7276*\dy})
	-- ({10.5688*\dx},{0.7379*\dy})
	-- ({10.5788*\dx},{0.7481*\dy})
	-- ({10.5888*\dx},{0.7582*\dy})
	-- ({10.5988*\dx},{0.7683*\dy})
	-- ({10.6088*\dx},{0.7783*\dy})
	-- ({10.6188*\dx},{0.7882*\dy})
	-- ({10.6289*\dx},{0.7980*\dy})
	-- ({10.6389*\dx},{0.8077*\dy})
	-- ({10.6489*\dx},{0.8172*\dy})
	-- ({10.6589*\dx},{0.8266*\dy})
	-- ({10.6689*\dx},{0.8358*\dy})
	-- ({10.6789*\dx},{0.8449*\dy})
	-- ({10.6889*\dx},{0.8538*\dy})
	-- ({10.6989*\dx},{0.8625*\dy})
	-- ({10.7089*\dx},{0.8711*\dy})
	-- ({10.7189*\dx},{0.8794*\dy})
	-- ({10.7289*\dx},{0.8875*\dy})
	-- ({10.7389*\dx},{0.8954*\dy})
	-- ({10.7490*\dx},{0.9030*\dy})
	-- ({10.7590*\dx},{0.9104*\dy})
	-- ({10.7690*\dx},{0.9176*\dy})
	-- ({10.7790*\dx},{0.9245*\dy})
	-- ({10.7890*\dx},{0.9311*\dy})
	-- ({10.7990*\dx},{0.9374*\dy})
	-- ({10.8090*\dx},{0.9435*\dy})
	-- ({10.8190*\dx},{0.9493*\dy})
	-- ({10.8290*\dx},{0.9547*\dy})
	-- ({10.8390*\dx},{0.9599*\dy})
	-- ({10.8490*\dx},{0.9648*\dy})
	-- ({10.8590*\dx},{0.9693*\dy})
	-- ({10.8691*\dx},{0.9736*\dy})
	-- ({10.8791*\dx},{0.9775*\dy})
	-- ({10.8891*\dx},{0.9811*\dy})
	-- ({10.8991*\dx},{0.9844*\dy})
	-- ({10.9091*\dx},{0.9874*\dy})
	-- ({10.9191*\dx},{0.9901*\dy})
	-- ({10.9291*\dx},{0.9924*\dy})
	-- ({10.9391*\dx},{0.9944*\dy})
	-- ({10.9491*\dx},{0.9961*\dy})
	-- ({10.9591*\dx},{0.9975*\dy})
	-- ({10.9691*\dx},{0.9986*\dy})
	-- ({10.9791*\dx},{0.9994*\dy})
	-- ({10.9892*\dx},{0.9998*\dy})
	-- ({10.9992*\dx},{1.0000*\dy})
	-- ({11.0092*\dx},{1.0000*\dy})
	-- ({11.0192*\dx},{1.0000*\dy})
	-- ({11.0292*\dx},{1.0000*\dy})
	-- ({11.0392*\dx},{1.0000*\dy})
	-- ({11.0492*\dx},{1.0000*\dy})
	-- ({11.0592*\dx},{1.0000*\dy})
	-- ({11.0692*\dx},{1.0000*\dy})
	-- ({11.0792*\dx},{1.0000*\dy})
	-- ({11.0892*\dx},{1.0000*\dy})
	-- ({11.0992*\dx},{1.0000*\dy})
	-- ({11.1093*\dx},{1.0000*\dy})
	-- ({11.1193*\dx},{1.0000*\dy})
	-- ({11.1293*\dx},{1.0000*\dy})
	-- ({11.1393*\dx},{1.0000*\dy})
	-- ({11.1493*\dx},{1.0000*\dy})
	-- ({11.1593*\dx},{1.0000*\dy})
	-- ({11.1693*\dx},{1.0000*\dy})
	-- ({11.1793*\dx},{1.0000*\dy})
	-- ({11.1893*\dx},{1.0000*\dy})
	-- ({11.1993*\dx},{1.0000*\dy})
	-- ({11.2093*\dx},{1.0000*\dy})
	-- ({11.2193*\dx},{1.0000*\dy})
	-- ({11.2294*\dx},{1.0000*\dy})
	-- ({11.2394*\dx},{1.0000*\dy})
	-- ({11.2494*\dx},{1.0000*\dy})
	-- ({11.2594*\dx},{1.0000*\dy})
	-- ({11.2694*\dx},{1.0000*\dy})
	-- ({11.2794*\dx},{1.0000*\dy})
	-- ({11.2894*\dx},{1.0000*\dy})
	-- ({11.2994*\dx},{1.0000*\dy})
	-- ({11.3094*\dx},{1.0000*\dy})
	-- ({11.3194*\dx},{1.0000*\dy})
	-- ({11.3294*\dx},{1.0000*\dy})
	-- ({11.3394*\dx},{1.0000*\dy})
	-- ({11.3495*\dx},{1.0000*\dy})
	-- ({11.3595*\dx},{1.0000*\dy})
	-- ({11.3695*\dx},{1.0000*\dy})
	-- ({11.3795*\dx},{1.0000*\dy})
	-- ({11.3895*\dx},{1.0000*\dy})
	-- ({11.3995*\dx},{1.0000*\dy})
	-- ({11.4095*\dx},{1.0000*\dy})
	-- ({11.4195*\dx},{1.0000*\dy})
	-- ({11.4295*\dx},{1.0000*\dy})
	-- ({11.4395*\dx},{1.0000*\dy})
	-- ({11.4495*\dx},{1.0000*\dy})
	-- ({11.4595*\dx},{1.0000*\dy})
	-- ({11.4696*\dx},{1.0000*\dy})
	-- ({11.4796*\dx},{1.0000*\dy})
	-- ({11.4896*\dx},{1.0000*\dy})
	-- ({11.4996*\dx},{1.0000*\dy})
	-- ({11.5096*\dx},{1.0000*\dy})
	-- ({11.5196*\dx},{1.0000*\dy})
	-- ({11.5296*\dx},{1.0000*\dy})
	-- ({11.5396*\dx},{1.0000*\dy})
	-- ({11.5496*\dx},{1.0000*\dy})
	-- ({11.5596*\dx},{1.0000*\dy})
	-- ({11.5696*\dx},{1.0000*\dy})
	-- ({11.5796*\dx},{1.0000*\dy})
	-- ({11.5897*\dx},{1.0000*\dy})
	-- ({11.5997*\dx},{1.0000*\dy})
	-- ({11.6097*\dx},{1.0000*\dy})
	-- ({11.6197*\dx},{1.0000*\dy})
	-- ({11.6297*\dx},{1.0000*\dy})
	-- ({11.6397*\dx},{1.0000*\dy})
	-- ({11.6497*\dx},{1.0000*\dy})
	-- ({11.6597*\dx},{1.0000*\dy})
	-- ({11.6697*\dx},{1.0000*\dy})
	-- ({11.6797*\dx},{1.0000*\dy})
	-- ({11.6897*\dx},{1.0000*\dy})
	-- ({11.6997*\dx},{1.0000*\dy})
	-- ({11.7098*\dx},{1.0000*\dy})
	-- ({11.7198*\dx},{1.0000*\dy})
	-- ({11.7298*\dx},{1.0000*\dy})
	-- ({11.7398*\dx},{1.0000*\dy})
	-- ({11.7498*\dx},{1.0000*\dy})
	-- ({11.7598*\dx},{1.0000*\dy})
	-- ({11.7698*\dx},{1.0000*\dy})
	-- ({11.7798*\dx},{1.0000*\dy})
	-- ({11.7898*\dx},{1.0000*\dy})
	-- ({11.7998*\dx},{1.0000*\dy})
	-- ({11.8098*\dx},{1.0000*\dy})
	-- ({11.8198*\dx},{1.0000*\dy})
	-- ({11.8299*\dx},{1.0000*\dy})
	-- ({11.8399*\dx},{1.0000*\dy})
	-- ({11.8499*\dx},{1.0000*\dy})
	-- ({11.8599*\dx},{1.0000*\dy})
	-- ({11.8699*\dx},{1.0000*\dy})
	-- ({11.8799*\dx},{1.0000*\dy})
	-- ({11.8899*\dx},{1.0000*\dy})
	-- ({11.8999*\dx},{1.0000*\dy})
	-- ({11.9099*\dx},{1.0000*\dy})
	-- ({11.9199*\dx},{1.0000*\dy})
	-- ({11.9299*\dx},{1.0000*\dy})
	-- ({11.9399*\dx},{1.0000*\dy})
	-- ({11.9500*\dx},{1.0000*\dy})
	-- ({11.9600*\dx},{1.0000*\dy})
	-- ({11.9700*\dx},{1.0000*\dy})
	-- ({11.9800*\dx},{1.0000*\dy})
	-- ({11.9900*\dx},{1.0000*\dy})
	-- ({12.0000*\dx},{1.0000*\dy})
}
\def\cpsithree{
	({0.0000*\dx},{1.0000*\dy})
	-- ({0.0100*\dx},{1.0000*\dy})
	-- ({0.0200*\dx},{1.0000*\dy})
	-- ({0.0300*\dx},{1.0000*\dy})
	-- ({0.0400*\dx},{1.0000*\dy})
	-- ({0.0500*\dx},{1.0000*\dy})
	-- ({0.0601*\dx},{1.0000*\dy})
	-- ({0.0701*\dx},{1.0000*\dy})
	-- ({0.0801*\dx},{1.0000*\dy})
	-- ({0.0901*\dx},{1.0000*\dy})
	-- ({0.1001*\dx},{1.0000*\dy})
	-- ({0.1101*\dx},{1.0000*\dy})
	-- ({0.1201*\dx},{1.0000*\dy})
	-- ({0.1301*\dx},{1.0000*\dy})
	-- ({0.1401*\dx},{1.0000*\dy})
	-- ({0.1501*\dx},{1.0000*\dy})
	-- ({0.1601*\dx},{1.0000*\dy})
	-- ({0.1701*\dx},{1.0000*\dy})
	-- ({0.1802*\dx},{1.0000*\dy})
	-- ({0.1902*\dx},{1.0000*\dy})
	-- ({0.2002*\dx},{1.0000*\dy})
	-- ({0.2102*\dx},{1.0000*\dy})
	-- ({0.2202*\dx},{1.0000*\dy})
	-- ({0.2302*\dx},{1.0000*\dy})
	-- ({0.2402*\dx},{1.0000*\dy})
	-- ({0.2502*\dx},{1.0000*\dy})
	-- ({0.2602*\dx},{1.0000*\dy})
	-- ({0.2702*\dx},{1.0000*\dy})
	-- ({0.2802*\dx},{1.0000*\dy})
	-- ({0.2902*\dx},{1.0000*\dy})
	-- ({0.3003*\dx},{1.0000*\dy})
	-- ({0.3103*\dx},{1.0000*\dy})
	-- ({0.3203*\dx},{1.0000*\dy})
	-- ({0.3303*\dx},{1.0000*\dy})
	-- ({0.3403*\dx},{1.0000*\dy})
	-- ({0.3503*\dx},{1.0000*\dy})
	-- ({0.3603*\dx},{1.0000*\dy})
	-- ({0.3703*\dx},{1.0000*\dy})
	-- ({0.3803*\dx},{1.0000*\dy})
	-- ({0.3903*\dx},{1.0000*\dy})
	-- ({0.4003*\dx},{1.0000*\dy})
	-- ({0.4103*\dx},{1.0000*\dy})
	-- ({0.4204*\dx},{1.0000*\dy})
	-- ({0.4304*\dx},{1.0000*\dy})
	-- ({0.4404*\dx},{1.0000*\dy})
	-- ({0.4504*\dx},{1.0000*\dy})
	-- ({0.4604*\dx},{1.0000*\dy})
	-- ({0.4704*\dx},{1.0000*\dy})
	-- ({0.4804*\dx},{1.0000*\dy})
	-- ({0.4904*\dx},{1.0000*\dy})
	-- ({0.5004*\dx},{1.0000*\dy})
	-- ({0.5104*\dx},{1.0000*\dy})
	-- ({0.5204*\dx},{1.0000*\dy})
	-- ({0.5304*\dx},{1.0000*\dy})
	-- ({0.5405*\dx},{1.0000*\dy})
	-- ({0.5505*\dx},{1.0000*\dy})
	-- ({0.5605*\dx},{1.0000*\dy})
	-- ({0.5705*\dx},{1.0000*\dy})
	-- ({0.5805*\dx},{1.0000*\dy})
	-- ({0.5905*\dx},{1.0000*\dy})
	-- ({0.6005*\dx},{1.0000*\dy})
	-- ({0.6105*\dx},{1.0000*\dy})
	-- ({0.6205*\dx},{1.0000*\dy})
	-- ({0.6305*\dx},{1.0000*\dy})
	-- ({0.6405*\dx},{1.0000*\dy})
	-- ({0.6505*\dx},{1.0000*\dy})
	-- ({0.6606*\dx},{1.0000*\dy})
	-- ({0.6706*\dx},{1.0000*\dy})
	-- ({0.6806*\dx},{1.0000*\dy})
	-- ({0.6906*\dx},{1.0000*\dy})
	-- ({0.7006*\dx},{1.0000*\dy})
	-- ({0.7106*\dx},{1.0000*\dy})
	-- ({0.7206*\dx},{1.0000*\dy})
	-- ({0.7306*\dx},{1.0000*\dy})
	-- ({0.7406*\dx},{1.0000*\dy})
	-- ({0.7506*\dx},{1.0000*\dy})
	-- ({0.7606*\dx},{1.0000*\dy})
	-- ({0.7706*\dx},{1.0000*\dy})
	-- ({0.7807*\dx},{1.0000*\dy})
	-- ({0.7907*\dx},{1.0000*\dy})
	-- ({0.8007*\dx},{1.0000*\dy})
	-- ({0.8107*\dx},{1.0000*\dy})
	-- ({0.8207*\dx},{1.0000*\dy})
	-- ({0.8307*\dx},{1.0000*\dy})
	-- ({0.8407*\dx},{1.0000*\dy})
	-- ({0.8507*\dx},{1.0000*\dy})
	-- ({0.8607*\dx},{1.0000*\dy})
	-- ({0.8707*\dx},{1.0000*\dy})
	-- ({0.8807*\dx},{1.0000*\dy})
	-- ({0.8907*\dx},{1.0000*\dy})
	-- ({0.9008*\dx},{1.0000*\dy})
	-- ({0.9108*\dx},{1.0000*\dy})
	-- ({0.9208*\dx},{1.0000*\dy})
	-- ({0.9308*\dx},{1.0000*\dy})
	-- ({0.9408*\dx},{1.0000*\dy})
	-- ({0.9508*\dx},{1.0000*\dy})
	-- ({0.9608*\dx},{1.0000*\dy})
	-- ({0.9708*\dx},{1.0000*\dy})
	-- ({0.9808*\dx},{1.0000*\dy})
	-- ({0.9908*\dx},{1.0000*\dy})
	-- ({1.0008*\dx},{1.0000*\dy})
	-- ({1.0108*\dx},{1.0000*\dy})
	-- ({1.0209*\dx},{1.0000*\dy})
	-- ({1.0309*\dx},{1.0000*\dy})
	-- ({1.0409*\dx},{1.0000*\dy})
	-- ({1.0509*\dx},{1.0000*\dy})
	-- ({1.0609*\dx},{1.0000*\dy})
	-- ({1.0709*\dx},{1.0000*\dy})
	-- ({1.0809*\dx},{1.0000*\dy})
	-- ({1.0909*\dx},{1.0000*\dy})
	-- ({1.1009*\dx},{1.0000*\dy})
	-- ({1.1109*\dx},{1.0000*\dy})
	-- ({1.1209*\dx},{1.0000*\dy})
	-- ({1.1309*\dx},{1.0000*\dy})
	-- ({1.1410*\dx},{1.0000*\dy})
	-- ({1.1510*\dx},{1.0000*\dy})
	-- ({1.1610*\dx},{1.0000*\dy})
	-- ({1.1710*\dx},{1.0000*\dy})
	-- ({1.1810*\dx},{1.0000*\dy})
	-- ({1.1910*\dx},{1.0000*\dy})
	-- ({1.2010*\dx},{1.0000*\dy})
	-- ({1.2110*\dx},{1.0000*\dy})
	-- ({1.2210*\dx},{1.0000*\dy})
	-- ({1.2310*\dx},{1.0000*\dy})
	-- ({1.2410*\dx},{1.0000*\dy})
	-- ({1.2510*\dx},{1.0000*\dy})
	-- ({1.2611*\dx},{1.0000*\dy})
	-- ({1.2711*\dx},{1.0000*\dy})
	-- ({1.2811*\dx},{1.0000*\dy})
	-- ({1.2911*\dx},{1.0000*\dy})
	-- ({1.3011*\dx},{1.0000*\dy})
	-- ({1.3111*\dx},{1.0000*\dy})
	-- ({1.3211*\dx},{1.0000*\dy})
	-- ({1.3311*\dx},{1.0000*\dy})
	-- ({1.3411*\dx},{1.0000*\dy})
	-- ({1.3511*\dx},{1.0000*\dy})
	-- ({1.3611*\dx},{1.0000*\dy})
	-- ({1.3711*\dx},{1.0000*\dy})
	-- ({1.3812*\dx},{1.0000*\dy})
	-- ({1.3912*\dx},{1.0000*\dy})
	-- ({1.4012*\dx},{1.0000*\dy})
	-- ({1.4112*\dx},{1.0000*\dy})
	-- ({1.4212*\dx},{1.0000*\dy})
	-- ({1.4312*\dx},{1.0000*\dy})
	-- ({1.4412*\dx},{1.0000*\dy})
	-- ({1.4512*\dx},{1.0000*\dy})
	-- ({1.4612*\dx},{1.0000*\dy})
	-- ({1.4712*\dx},{1.0000*\dy})
	-- ({1.4812*\dx},{1.0000*\dy})
	-- ({1.4912*\dx},{1.0000*\dy})
	-- ({1.5013*\dx},{1.0000*\dy})
	-- ({1.5113*\dx},{1.0000*\dy})
	-- ({1.5213*\dx},{1.0000*\dy})
	-- ({1.5313*\dx},{1.0000*\dy})
	-- ({1.5413*\dx},{1.0000*\dy})
	-- ({1.5513*\dx},{1.0000*\dy})
	-- ({1.5613*\dx},{1.0000*\dy})
	-- ({1.5713*\dx},{1.0000*\dy})
	-- ({1.5813*\dx},{1.0000*\dy})
	-- ({1.5913*\dx},{1.0000*\dy})
	-- ({1.6013*\dx},{1.0000*\dy})
	-- ({1.6113*\dx},{1.0000*\dy})
	-- ({1.6214*\dx},{1.0000*\dy})
	-- ({1.6314*\dx},{1.0000*\dy})
	-- ({1.6414*\dx},{1.0000*\dy})
	-- ({1.6514*\dx},{1.0000*\dy})
	-- ({1.6614*\dx},{1.0000*\dy})
	-- ({1.6714*\dx},{1.0000*\dy})
	-- ({1.6814*\dx},{1.0000*\dy})
	-- ({1.6914*\dx},{1.0000*\dy})
	-- ({1.7014*\dx},{1.0000*\dy})
	-- ({1.7114*\dx},{1.0000*\dy})
	-- ({1.7214*\dx},{1.0000*\dy})
	-- ({1.7314*\dx},{1.0000*\dy})
	-- ({1.7415*\dx},{1.0000*\dy})
	-- ({1.7515*\dx},{1.0000*\dy})
	-- ({1.7615*\dx},{1.0000*\dy})
	-- ({1.7715*\dx},{1.0000*\dy})
	-- ({1.7815*\dx},{1.0000*\dy})
	-- ({1.7915*\dx},{1.0000*\dy})
	-- ({1.8015*\dx},{1.0000*\dy})
	-- ({1.8115*\dx},{1.0000*\dy})
	-- ({1.8215*\dx},{1.0000*\dy})
	-- ({1.8315*\dx},{1.0000*\dy})
	-- ({1.8415*\dx},{1.0000*\dy})
	-- ({1.8515*\dx},{1.0000*\dy})
	-- ({1.8616*\dx},{1.0000*\dy})
	-- ({1.8716*\dx},{1.0000*\dy})
	-- ({1.8816*\dx},{1.0000*\dy})
	-- ({1.8916*\dx},{1.0000*\dy})
	-- ({1.9016*\dx},{1.0000*\dy})
	-- ({1.9116*\dx},{1.0000*\dy})
	-- ({1.9216*\dx},{1.0000*\dy})
	-- ({1.9316*\dx},{1.0000*\dy})
	-- ({1.9416*\dx},{1.0000*\dy})
	-- ({1.9516*\dx},{1.0000*\dy})
	-- ({1.9616*\dx},{1.0000*\dy})
	-- ({1.9716*\dx},{1.0000*\dy})
	-- ({1.9817*\dx},{1.0000*\dy})
	-- ({1.9917*\dx},{1.0000*\dy})
	-- ({2.0017*\dx},{1.0000*\dy})
	-- ({2.0117*\dx},{1.0000*\dy})
	-- ({2.0217*\dx},{1.0000*\dy})
	-- ({2.0317*\dx},{1.0000*\dy})
	-- ({2.0417*\dx},{1.0000*\dy})
	-- ({2.0517*\dx},{1.0000*\dy})
	-- ({2.0617*\dx},{1.0000*\dy})
	-- ({2.0717*\dx},{1.0000*\dy})
	-- ({2.0817*\dx},{1.0000*\dy})
	-- ({2.0917*\dx},{1.0000*\dy})
	-- ({2.1018*\dx},{1.0000*\dy})
	-- ({2.1118*\dx},{1.0000*\dy})
	-- ({2.1218*\dx},{1.0000*\dy})
	-- ({2.1318*\dx},{1.0000*\dy})
	-- ({2.1418*\dx},{1.0000*\dy})
	-- ({2.1518*\dx},{1.0000*\dy})
	-- ({2.1618*\dx},{1.0000*\dy})
	-- ({2.1718*\dx},{1.0000*\dy})
	-- ({2.1818*\dx},{1.0000*\dy})
	-- ({2.1918*\dx},{1.0000*\dy})
	-- ({2.2018*\dx},{1.0000*\dy})
	-- ({2.2118*\dx},{1.0000*\dy})
	-- ({2.2219*\dx},{1.0000*\dy})
	-- ({2.2319*\dx},{1.0000*\dy})
	-- ({2.2419*\dx},{1.0000*\dy})
	-- ({2.2519*\dx},{1.0000*\dy})
	-- ({2.2619*\dx},{1.0000*\dy})
	-- ({2.2719*\dx},{1.0000*\dy})
	-- ({2.2819*\dx},{1.0000*\dy})
	-- ({2.2919*\dx},{1.0000*\dy})
	-- ({2.3019*\dx},{1.0000*\dy})
	-- ({2.3119*\dx},{1.0000*\dy})
	-- ({2.3219*\dx},{1.0000*\dy})
	-- ({2.3319*\dx},{1.0000*\dy})
	-- ({2.3420*\dx},{1.0000*\dy})
	-- ({2.3520*\dx},{1.0000*\dy})
	-- ({2.3620*\dx},{1.0000*\dy})
	-- ({2.3720*\dx},{1.0000*\dy})
	-- ({2.3820*\dx},{1.0000*\dy})
	-- ({2.3920*\dx},{1.0000*\dy})
	-- ({2.4020*\dx},{1.0000*\dy})
	-- ({2.4120*\dx},{1.0000*\dy})
	-- ({2.4220*\dx},{1.0000*\dy})
	-- ({2.4320*\dx},{1.0000*\dy})
	-- ({2.4420*\dx},{1.0000*\dy})
	-- ({2.4520*\dx},{1.0000*\dy})
	-- ({2.4621*\dx},{1.0000*\dy})
	-- ({2.4721*\dx},{1.0000*\dy})
	-- ({2.4821*\dx},{1.0000*\dy})
	-- ({2.4921*\dx},{1.0000*\dy})
	-- ({2.5021*\dx},{1.0000*\dy})
	-- ({2.5121*\dx},{1.0000*\dy})
	-- ({2.5221*\dx},{1.0000*\dy})
	-- ({2.5321*\dx},{1.0000*\dy})
	-- ({2.5421*\dx},{1.0000*\dy})
	-- ({2.5521*\dx},{1.0000*\dy})
	-- ({2.5621*\dx},{1.0000*\dy})
	-- ({2.5721*\dx},{1.0000*\dy})
	-- ({2.5822*\dx},{1.0000*\dy})
	-- ({2.5922*\dx},{1.0000*\dy})
	-- ({2.6022*\dx},{1.0000*\dy})
	-- ({2.6122*\dx},{1.0000*\dy})
	-- ({2.6222*\dx},{1.0000*\dy})
	-- ({2.6322*\dx},{1.0000*\dy})
	-- ({2.6422*\dx},{1.0000*\dy})
	-- ({2.6522*\dx},{1.0000*\dy})
	-- ({2.6622*\dx},{1.0000*\dy})
	-- ({2.6722*\dx},{1.0000*\dy})
	-- ({2.6822*\dx},{1.0000*\dy})
	-- ({2.6922*\dx},{1.0000*\dy})
	-- ({2.7023*\dx},{1.0000*\dy})
	-- ({2.7123*\dx},{1.0000*\dy})
	-- ({2.7223*\dx},{1.0000*\dy})
	-- ({2.7323*\dx},{1.0000*\dy})
	-- ({2.7423*\dx},{1.0000*\dy})
	-- ({2.7523*\dx},{1.0000*\dy})
	-- ({2.7623*\dx},{1.0000*\dy})
	-- ({2.7723*\dx},{1.0000*\dy})
	-- ({2.7823*\dx},{1.0000*\dy})
	-- ({2.7923*\dx},{1.0000*\dy})
	-- ({2.8023*\dx},{1.0000*\dy})
	-- ({2.8123*\dx},{1.0000*\dy})
	-- ({2.8224*\dx},{1.0000*\dy})
	-- ({2.8324*\dx},{1.0000*\dy})
	-- ({2.8424*\dx},{1.0000*\dy})
	-- ({2.8524*\dx},{1.0000*\dy})
	-- ({2.8624*\dx},{1.0000*\dy})
	-- ({2.8724*\dx},{1.0000*\dy})
	-- ({2.8824*\dx},{1.0000*\dy})
	-- ({2.8924*\dx},{1.0000*\dy})
	-- ({2.9024*\dx},{1.0000*\dy})
	-- ({2.9124*\dx},{1.0000*\dy})
	-- ({2.9224*\dx},{1.0000*\dy})
	-- ({2.9324*\dx},{1.0000*\dy})
	-- ({2.9425*\dx},{1.0000*\dy})
	-- ({2.9525*\dx},{1.0000*\dy})
	-- ({2.9625*\dx},{1.0000*\dy})
	-- ({2.9725*\dx},{1.0000*\dy})
	-- ({2.9825*\dx},{1.0000*\dy})
	-- ({2.9925*\dx},{1.0000*\dy})
	-- ({3.0025*\dx},{1.0000*\dy})
	-- ({3.0125*\dx},{0.9999*\dy})
	-- ({3.0225*\dx},{0.9997*\dy})
	-- ({3.0325*\dx},{0.9994*\dy})
	-- ({3.0425*\dx},{0.9991*\dy})
	-- ({3.0525*\dx},{0.9986*\dy})
	-- ({3.0626*\dx},{0.9981*\dy})
	-- ({3.0726*\dx},{0.9974*\dy})
	-- ({3.0826*\dx},{0.9967*\dy})
	-- ({3.0926*\dx},{0.9959*\dy})
	-- ({3.1026*\dx},{0.9951*\dy})
	-- ({3.1126*\dx},{0.9941*\dy})
	-- ({3.1226*\dx},{0.9931*\dy})
	-- ({3.1326*\dx},{0.9921*\dy})
	-- ({3.1426*\dx},{0.9910*\dy})
	-- ({3.1526*\dx},{0.9898*\dy})
	-- ({3.1626*\dx},{0.9886*\dy})
	-- ({3.1726*\dx},{0.9873*\dy})
	-- ({3.1827*\dx},{0.9860*\dy})
	-- ({3.1927*\dx},{0.9846*\dy})
	-- ({3.2027*\dx},{0.9832*\dy})
	-- ({3.2127*\dx},{0.9817*\dy})
	-- ({3.2227*\dx},{0.9802*\dy})
	-- ({3.2327*\dx},{0.9787*\dy})
	-- ({3.2427*\dx},{0.9771*\dy})
	-- ({3.2527*\dx},{0.9755*\dy})
	-- ({3.2627*\dx},{0.9739*\dy})
	-- ({3.2727*\dx},{0.9722*\dy})
	-- ({3.2827*\dx},{0.9705*\dy})
	-- ({3.2927*\dx},{0.9687*\dy})
	-- ({3.3028*\dx},{0.9669*\dy})
	-- ({3.3128*\dx},{0.9651*\dy})
	-- ({3.3228*\dx},{0.9633*\dy})
	-- ({3.3328*\dx},{0.9615*\dy})
	-- ({3.3428*\dx},{0.9596*\dy})
	-- ({3.3528*\dx},{0.9577*\dy})
	-- ({3.3628*\dx},{0.9557*\dy})
	-- ({3.3728*\dx},{0.9538*\dy})
	-- ({3.3828*\dx},{0.9518*\dy})
	-- ({3.3928*\dx},{0.9498*\dy})
	-- ({3.4028*\dx},{0.9478*\dy})
	-- ({3.4128*\dx},{0.9457*\dy})
	-- ({3.4229*\dx},{0.9437*\dy})
	-- ({3.4329*\dx},{0.9416*\dy})
	-- ({3.4429*\dx},{0.9395*\dy})
	-- ({3.4529*\dx},{0.9374*\dy})
	-- ({3.4629*\dx},{0.9353*\dy})
	-- ({3.4729*\dx},{0.9331*\dy})
	-- ({3.4829*\dx},{0.9310*\dy})
	-- ({3.4929*\dx},{0.9288*\dy})
	-- ({3.5029*\dx},{0.9266*\dy})
	-- ({3.5129*\dx},{0.9244*\dy})
	-- ({3.5229*\dx},{0.9221*\dy})
	-- ({3.5329*\dx},{0.9199*\dy})
	-- ({3.5430*\dx},{0.9176*\dy})
	-- ({3.5530*\dx},{0.9154*\dy})
	-- ({3.5630*\dx},{0.9131*\dy})
	-- ({3.5730*\dx},{0.9108*\dy})
	-- ({3.5830*\dx},{0.9085*\dy})
	-- ({3.5930*\dx},{0.9061*\dy})
	-- ({3.6030*\dx},{0.9038*\dy})
	-- ({3.6130*\dx},{0.9014*\dy})
	-- ({3.6230*\dx},{0.8991*\dy})
	-- ({3.6330*\dx},{0.8967*\dy})
	-- ({3.6430*\dx},{0.8943*\dy})
	-- ({3.6530*\dx},{0.8919*\dy})
	-- ({3.6631*\dx},{0.8895*\dy})
	-- ({3.6731*\dx},{0.8870*\dy})
	-- ({3.6831*\dx},{0.8846*\dy})
	-- ({3.6931*\dx},{0.8821*\dy})
	-- ({3.7031*\dx},{0.8796*\dy})
	-- ({3.7131*\dx},{0.8772*\dy})
	-- ({3.7231*\dx},{0.8747*\dy})
	-- ({3.7331*\dx},{0.8721*\dy})
	-- ({3.7431*\dx},{0.8696*\dy})
	-- ({3.7531*\dx},{0.8671*\dy})
	-- ({3.7631*\dx},{0.8645*\dy})
	-- ({3.7731*\dx},{0.8620*\dy})
	-- ({3.7832*\dx},{0.8594*\dy})
	-- ({3.7932*\dx},{0.8568*\dy})
	-- ({3.8032*\dx},{0.8542*\dy})
	-- ({3.8132*\dx},{0.8516*\dy})
	-- ({3.8232*\dx},{0.8489*\dy})
	-- ({3.8332*\dx},{0.8463*\dy})
	-- ({3.8432*\dx},{0.8436*\dy})
	-- ({3.8532*\dx},{0.8409*\dy})
	-- ({3.8632*\dx},{0.8382*\dy})
	-- ({3.8732*\dx},{0.8355*\dy})
	-- ({3.8832*\dx},{0.8328*\dy})
	-- ({3.8932*\dx},{0.8301*\dy})
	-- ({3.9033*\dx},{0.8273*\dy})
	-- ({3.9133*\dx},{0.8246*\dy})
	-- ({3.9233*\dx},{0.8218*\dy})
	-- ({3.9333*\dx},{0.8190*\dy})
	-- ({3.9433*\dx},{0.8162*\dy})
	-- ({3.9533*\dx},{0.8134*\dy})
	-- ({3.9633*\dx},{0.8105*\dy})
	-- ({3.9733*\dx},{0.8077*\dy})
	-- ({3.9833*\dx},{0.8048*\dy})
	-- ({3.9933*\dx},{0.8019*\dy})
	-- ({4.0033*\dx},{0.7990*\dy})
	-- ({4.0133*\dx},{0.7961*\dy})
	-- ({4.0234*\dx},{0.7932*\dy})
	-- ({4.0334*\dx},{0.7902*\dy})
	-- ({4.0434*\dx},{0.7873*\dy})
	-- ({4.0534*\dx},{0.7843*\dy})
	-- ({4.0634*\dx},{0.7813*\dy})
	-- ({4.0734*\dx},{0.7783*\dy})
	-- ({4.0834*\dx},{0.7752*\dy})
	-- ({4.0934*\dx},{0.7722*\dy})
	-- ({4.1034*\dx},{0.7691*\dy})
	-- ({4.1134*\dx},{0.7660*\dy})
	-- ({4.1234*\dx},{0.7629*\dy})
	-- ({4.1334*\dx},{0.7598*\dy})
	-- ({4.1435*\dx},{0.7566*\dy})
	-- ({4.1535*\dx},{0.7534*\dy})
	-- ({4.1635*\dx},{0.7503*\dy})
	-- ({4.1735*\dx},{0.7471*\dy})
	-- ({4.1835*\dx},{0.7438*\dy})
	-- ({4.1935*\dx},{0.7406*\dy})
	-- ({4.2035*\dx},{0.7373*\dy})
	-- ({4.2135*\dx},{0.7340*\dy})
	-- ({4.2235*\dx},{0.7307*\dy})
	-- ({4.2335*\dx},{0.7274*\dy})
	-- ({4.2435*\dx},{0.7241*\dy})
	-- ({4.2535*\dx},{0.7207*\dy})
	-- ({4.2636*\dx},{0.7173*\dy})
	-- ({4.2736*\dx},{0.7139*\dy})
	-- ({4.2836*\dx},{0.7105*\dy})
	-- ({4.2936*\dx},{0.7070*\dy})
	-- ({4.3036*\dx},{0.7035*\dy})
	-- ({4.3136*\dx},{0.7000*\dy})
	-- ({4.3236*\dx},{0.6965*\dy})
	-- ({4.3336*\dx},{0.6930*\dy})
	-- ({4.3436*\dx},{0.6894*\dy})
	-- ({4.3536*\dx},{0.6858*\dy})
	-- ({4.3636*\dx},{0.6822*\dy})
	-- ({4.3736*\dx},{0.6786*\dy})
	-- ({4.3837*\dx},{0.6749*\dy})
	-- ({4.3937*\dx},{0.6712*\dy})
	-- ({4.4037*\dx},{0.6675*\dy})
	-- ({4.4137*\dx},{0.6638*\dy})
	-- ({4.4237*\dx},{0.6600*\dy})
	-- ({4.4337*\dx},{0.6563*\dy})
	-- ({4.4437*\dx},{0.6525*\dy})
	-- ({4.4537*\dx},{0.6486*\dy})
	-- ({4.4637*\dx},{0.6448*\dy})
	-- ({4.4737*\dx},{0.6409*\dy})
	-- ({4.4837*\dx},{0.6370*\dy})
	-- ({4.4937*\dx},{0.6331*\dy})
	-- ({4.5038*\dx},{0.6291*\dy})
	-- ({4.5138*\dx},{0.6251*\dy})
	-- ({4.5238*\dx},{0.6211*\dy})
	-- ({4.5338*\dx},{0.6171*\dy})
	-- ({4.5438*\dx},{0.6130*\dy})
	-- ({4.5538*\dx},{0.6090*\dy})
	-- ({4.5638*\dx},{0.6048*\dy})
	-- ({4.5738*\dx},{0.6007*\dy})
	-- ({4.5838*\dx},{0.5965*\dy})
	-- ({4.5938*\dx},{0.5923*\dy})
	-- ({4.6038*\dx},{0.5881*\dy})
	-- ({4.6138*\dx},{0.5839*\dy})
	-- ({4.6239*\dx},{0.5796*\dy})
	-- ({4.6339*\dx},{0.5753*\dy})
	-- ({4.6439*\dx},{0.5710*\dy})
	-- ({4.6539*\dx},{0.5666*\dy})
	-- ({4.6639*\dx},{0.5623*\dy})
	-- ({4.6739*\dx},{0.5578*\dy})
	-- ({4.6839*\dx},{0.5534*\dy})
	-- ({4.6939*\dx},{0.5489*\dy})
	-- ({4.7039*\dx},{0.5445*\dy})
	-- ({4.7139*\dx},{0.5399*\dy})
	-- ({4.7239*\dx},{0.5354*\dy})
	-- ({4.7339*\dx},{0.5308*\dy})
	-- ({4.7440*\dx},{0.5262*\dy})
	-- ({4.7540*\dx},{0.5216*\dy})
	-- ({4.7640*\dx},{0.5170*\dy})
	-- ({4.7740*\dx},{0.5123*\dy})
	-- ({4.7840*\dx},{0.5076*\dy})
	-- ({4.7940*\dx},{0.5029*\dy})
	-- ({4.8040*\dx},{0.4981*\dy})
	-- ({4.8140*\dx},{0.4933*\dy})
	-- ({4.8240*\dx},{0.4885*\dy})
	-- ({4.8340*\dx},{0.4837*\dy})
	-- ({4.8440*\dx},{0.4788*\dy})
	-- ({4.8540*\dx},{0.4739*\dy})
	-- ({4.8641*\dx},{0.4690*\dy})
	-- ({4.8741*\dx},{0.4641*\dy})
	-- ({4.8841*\dx},{0.4591*\dy})
	-- ({4.8941*\dx},{0.4541*\dy})
	-- ({4.9041*\dx},{0.4491*\dy})
	-- ({4.9141*\dx},{0.4441*\dy})
	-- ({4.9241*\dx},{0.4390*\dy})
	-- ({4.9341*\dx},{0.4340*\dy})
	-- ({4.9441*\dx},{0.4289*\dy})
	-- ({4.9541*\dx},{0.4237*\dy})
	-- ({4.9641*\dx},{0.4186*\dy})
	-- ({4.9741*\dx},{0.4134*\dy})
	-- ({4.9842*\dx},{0.4082*\dy})
	-- ({4.9942*\dx},{0.4030*\dy})
	-- ({5.0042*\dx},{0.3978*\dy})
	-- ({5.0142*\dx},{0.3926*\dy})
	-- ({5.0242*\dx},{0.3872*\dy})
	-- ({5.0342*\dx},{0.3819*\dy})
	-- ({5.0442*\dx},{0.3765*\dy})
	-- ({5.0542*\dx},{0.3711*\dy})
	-- ({5.0642*\dx},{0.3656*\dy})
	-- ({5.0742*\dx},{0.3601*\dy})
	-- ({5.0842*\dx},{0.3546*\dy})
	-- ({5.0942*\dx},{0.3490*\dy})
	-- ({5.1043*\dx},{0.3435*\dy})
	-- ({5.1143*\dx},{0.3378*\dy})
	-- ({5.1243*\dx},{0.3322*\dy})
	-- ({5.1343*\dx},{0.3265*\dy})
	-- ({5.1443*\dx},{0.3209*\dy})
	-- ({5.1543*\dx},{0.3152*\dy})
	-- ({5.1643*\dx},{0.3094*\dy})
	-- ({5.1743*\dx},{0.3037*\dy})
	-- ({5.1843*\dx},{0.2980*\dy})
	-- ({5.1943*\dx},{0.2922*\dy})
	-- ({5.2043*\dx},{0.2864*\dy})
	-- ({5.2143*\dx},{0.2807*\dy})
	-- ({5.2244*\dx},{0.2749*\dy})
	-- ({5.2344*\dx},{0.2691*\dy})
	-- ({5.2444*\dx},{0.2633*\dy})
	-- ({5.2544*\dx},{0.2576*\dy})
	-- ({5.2644*\dx},{0.2518*\dy})
	-- ({5.2744*\dx},{0.2461*\dy})
	-- ({5.2844*\dx},{0.2403*\dy})
	-- ({5.2944*\dx},{0.2346*\dy})
	-- ({5.3044*\dx},{0.2289*\dy})
	-- ({5.3144*\dx},{0.2232*\dy})
	-- ({5.3244*\dx},{0.2176*\dy})
	-- ({5.3344*\dx},{0.2119*\dy})
	-- ({5.3445*\dx},{0.2063*\dy})
	-- ({5.3545*\dx},{0.2007*\dy})
	-- ({5.3645*\dx},{0.1952*\dy})
	-- ({5.3745*\dx},{0.1897*\dy})
	-- ({5.3845*\dx},{0.1842*\dy})
	-- ({5.3945*\dx},{0.1788*\dy})
	-- ({5.4045*\dx},{0.1734*\dy})
	-- ({5.4145*\dx},{0.1680*\dy})
	-- ({5.4245*\dx},{0.1627*\dy})
	-- ({5.4345*\dx},{0.1575*\dy})
	-- ({5.4445*\dx},{0.1523*\dy})
	-- ({5.4545*\dx},{0.1472*\dy})
	-- ({5.4646*\dx},{0.1421*\dy})
	-- ({5.4746*\dx},{0.1371*\dy})
	-- ({5.4846*\dx},{0.1321*\dy})
	-- ({5.4946*\dx},{0.1273*\dy})
	-- ({5.5046*\dx},{0.1224*\dy})
	-- ({5.5146*\dx},{0.1177*\dy})
	-- ({5.5246*\dx},{0.1130*\dy})
	-- ({5.5346*\dx},{0.1084*\dy})
	-- ({5.5446*\dx},{0.1039*\dy})
	-- ({5.5546*\dx},{0.0995*\dy})
	-- ({5.5646*\dx},{0.0951*\dy})
	-- ({5.5746*\dx},{0.0908*\dy})
	-- ({5.5847*\dx},{0.0867*\dy})
	-- ({5.5947*\dx},{0.0825*\dy})
	-- ({5.6047*\dx},{0.0785*\dy})
	-- ({5.6147*\dx},{0.0746*\dy})
	-- ({5.6247*\dx},{0.0708*\dy})
	-- ({5.6347*\dx},{0.0670*\dy})
	-- ({5.6447*\dx},{0.0634*\dy})
	-- ({5.6547*\dx},{0.0598*\dy})
	-- ({5.6647*\dx},{0.0564*\dy})
	-- ({5.6747*\dx},{0.0530*\dy})
	-- ({5.6847*\dx},{0.0498*\dy})
	-- ({5.6947*\dx},{0.0466*\dy})
	-- ({5.7048*\dx},{0.0436*\dy})
	-- ({5.7148*\dx},{0.0406*\dy})
	-- ({5.7248*\dx},{0.0377*\dy})
	-- ({5.7348*\dx},{0.0350*\dy})
	-- ({5.7448*\dx},{0.0324*\dy})
	-- ({5.7548*\dx},{0.0298*\dy})
	-- ({5.7648*\dx},{0.0274*\dy})
	-- ({5.7748*\dx},{0.0250*\dy})
	-- ({5.7848*\dx},{0.0228*\dy})
	-- ({5.7948*\dx},{0.0207*\dy})
	-- ({5.8048*\dx},{0.0187*\dy})
	-- ({5.8148*\dx},{0.0168*\dy})
	-- ({5.8249*\dx},{0.0150*\dy})
	-- ({5.8349*\dx},{0.0133*\dy})
	-- ({5.8449*\dx},{0.0117*\dy})
	-- ({5.8549*\dx},{0.0102*\dy})
	-- ({5.8649*\dx},{0.0088*\dy})
	-- ({5.8749*\dx},{0.0075*\dy})
	-- ({5.8849*\dx},{0.0064*\dy})
	-- ({5.8949*\dx},{0.0053*\dy})
	-- ({5.9049*\dx},{0.0043*\dy})
	-- ({5.9149*\dx},{0.0034*\dy})
	-- ({5.9249*\dx},{0.0027*\dy})
	-- ({5.9349*\dx},{0.0020*\dy})
	-- ({5.9450*\dx},{0.0014*\dy})
	-- ({5.9550*\dx},{0.0009*\dy})
	-- ({5.9650*\dx},{0.0006*\dy})
	-- ({5.9750*\dx},{0.0003*\dy})
	-- ({5.9850*\dx},{0.0001*\dy})
	-- ({5.9950*\dx},{0.0000*\dy})
	-- ({6.0050*\dx},{0.0000*\dy})
	-- ({6.0150*\dx},{0.0000*\dy})
	-- ({6.0250*\dx},{0.0000*\dy})
	-- ({6.0350*\dx},{0.0000*\dy})
	-- ({6.0450*\dx},{0.0000*\dy})
	-- ({6.0550*\dx},{0.0000*\dy})
	-- ({6.0651*\dx},{0.0000*\dy})
	-- ({6.0751*\dx},{0.0000*\dy})
	-- ({6.0851*\dx},{0.0000*\dy})
	-- ({6.0951*\dx},{0.0000*\dy})
	-- ({6.1051*\dx},{0.0000*\dy})
	-- ({6.1151*\dx},{0.0000*\dy})
	-- ({6.1251*\dx},{0.0000*\dy})
	-- ({6.1351*\dx},{0.0000*\dy})
	-- ({6.1451*\dx},{0.0000*\dy})
	-- ({6.1551*\dx},{0.0000*\dy})
	-- ({6.1651*\dx},{0.0000*\dy})
	-- ({6.1751*\dx},{0.0000*\dy})
	-- ({6.1852*\dx},{0.0000*\dy})
	-- ({6.1952*\dx},{0.0000*\dy})
	-- ({6.2052*\dx},{0.0000*\dy})
	-- ({6.2152*\dx},{0.0000*\dy})
	-- ({6.2252*\dx},{0.0000*\dy})
	-- ({6.2352*\dx},{0.0000*\dy})
	-- ({6.2452*\dx},{0.0000*\dy})
	-- ({6.2552*\dx},{0.0000*\dy})
	-- ({6.2652*\dx},{0.0000*\dy})
	-- ({6.2752*\dx},{0.0000*\dy})
	-- ({6.2852*\dx},{0.0000*\dy})
	-- ({6.2952*\dx},{0.0000*\dy})
	-- ({6.3053*\dx},{0.0000*\dy})
	-- ({6.3153*\dx},{0.0000*\dy})
	-- ({6.3253*\dx},{0.0000*\dy})
	-- ({6.3353*\dx},{0.0000*\dy})
	-- ({6.3453*\dx},{0.0000*\dy})
	-- ({6.3553*\dx},{0.0000*\dy})
	-- ({6.3653*\dx},{0.0000*\dy})
	-- ({6.3753*\dx},{0.0000*\dy})
	-- ({6.3853*\dx},{0.0000*\dy})
	-- ({6.3953*\dx},{0.0000*\dy})
	-- ({6.4053*\dx},{0.0000*\dy})
	-- ({6.4153*\dx},{0.0000*\dy})
	-- ({6.4254*\dx},{0.0000*\dy})
	-- ({6.4354*\dx},{0.0000*\dy})
	-- ({6.4454*\dx},{0.0000*\dy})
	-- ({6.4554*\dx},{0.0000*\dy})
	-- ({6.4654*\dx},{0.0000*\dy})
	-- ({6.4754*\dx},{0.0000*\dy})
	-- ({6.4854*\dx},{0.0000*\dy})
	-- ({6.4954*\dx},{0.0000*\dy})
	-- ({6.5054*\dx},{0.0000*\dy})
	-- ({6.5154*\dx},{0.0000*\dy})
	-- ({6.5254*\dx},{0.0000*\dy})
	-- ({6.5354*\dx},{0.0000*\dy})
	-- ({6.5455*\dx},{0.0000*\dy})
	-- ({6.5555*\dx},{0.0000*\dy})
	-- ({6.5655*\dx},{0.0000*\dy})
	-- ({6.5755*\dx},{0.0000*\dy})
	-- ({6.5855*\dx},{0.0000*\dy})
	-- ({6.5955*\dx},{0.0000*\dy})
	-- ({6.6055*\dx},{0.0000*\dy})
	-- ({6.6155*\dx},{0.0000*\dy})
	-- ({6.6255*\dx},{0.0000*\dy})
	-- ({6.6355*\dx},{0.0000*\dy})
	-- ({6.6455*\dx},{0.0000*\dy})
	-- ({6.6555*\dx},{0.0000*\dy})
	-- ({6.6656*\dx},{0.0000*\dy})
	-- ({6.6756*\dx},{0.0000*\dy})
	-- ({6.6856*\dx},{0.0000*\dy})
	-- ({6.6956*\dx},{0.0000*\dy})
	-- ({6.7056*\dx},{0.0000*\dy})
	-- ({6.7156*\dx},{0.0000*\dy})
	-- ({6.7256*\dx},{0.0000*\dy})
	-- ({6.7356*\dx},{0.0000*\dy})
	-- ({6.7456*\dx},{0.0000*\dy})
	-- ({6.7556*\dx},{0.0000*\dy})
	-- ({6.7656*\dx},{0.0000*\dy})
	-- ({6.7756*\dx},{0.0000*\dy})
	-- ({6.7857*\dx},{0.0000*\dy})
	-- ({6.7957*\dx},{0.0000*\dy})
	-- ({6.8057*\dx},{0.0000*\dy})
	-- ({6.8157*\dx},{0.0000*\dy})
	-- ({6.8257*\dx},{0.0000*\dy})
	-- ({6.8357*\dx},{0.0000*\dy})
	-- ({6.8457*\dx},{0.0000*\dy})
	-- ({6.8557*\dx},{0.0000*\dy})
	-- ({6.8657*\dx},{0.0000*\dy})
	-- ({6.8757*\dx},{0.0000*\dy})
	-- ({6.8857*\dx},{0.0000*\dy})
	-- ({6.8957*\dx},{0.0000*\dy})
	-- ({6.9058*\dx},{0.0000*\dy})
	-- ({6.9158*\dx},{0.0000*\dy})
	-- ({6.9258*\dx},{0.0000*\dy})
	-- ({6.9358*\dx},{0.0000*\dy})
	-- ({6.9458*\dx},{0.0000*\dy})
	-- ({6.9558*\dx},{0.0000*\dy})
	-- ({6.9658*\dx},{0.0000*\dy})
	-- ({6.9758*\dx},{0.0000*\dy})
	-- ({6.9858*\dx},{0.0000*\dy})
	-- ({6.9958*\dx},{0.0000*\dy})
	-- ({7.0058*\dx},{0.0000*\dy})
	-- ({7.0158*\dx},{0.0000*\dy})
	-- ({7.0259*\dx},{0.0000*\dy})
	-- ({7.0359*\dx},{0.0000*\dy})
	-- ({7.0459*\dx},{0.0000*\dy})
	-- ({7.0559*\dx},{0.0000*\dy})
	-- ({7.0659*\dx},{0.0000*\dy})
	-- ({7.0759*\dx},{0.0000*\dy})
	-- ({7.0859*\dx},{0.0000*\dy})
	-- ({7.0959*\dx},{0.0000*\dy})
	-- ({7.1059*\dx},{0.0000*\dy})
	-- ({7.1159*\dx},{0.0000*\dy})
	-- ({7.1259*\dx},{0.0000*\dy})
	-- ({7.1359*\dx},{0.0000*\dy})
	-- ({7.1460*\dx},{0.0000*\dy})
	-- ({7.1560*\dx},{0.0000*\dy})
	-- ({7.1660*\dx},{0.0000*\dy})
	-- ({7.1760*\dx},{0.0000*\dy})
	-- ({7.1860*\dx},{0.0000*\dy})
	-- ({7.1960*\dx},{0.0000*\dy})
	-- ({7.2060*\dx},{0.0000*\dy})
	-- ({7.2160*\dx},{0.0000*\dy})
	-- ({7.2260*\dx},{0.0000*\dy})
	-- ({7.2360*\dx},{0.0000*\dy})
	-- ({7.2460*\dx},{0.0000*\dy})
	-- ({7.2560*\dx},{0.0000*\dy})
	-- ({7.2661*\dx},{0.0000*\dy})
	-- ({7.2761*\dx},{0.0000*\dy})
	-- ({7.2861*\dx},{0.0000*\dy})
	-- ({7.2961*\dx},{0.0000*\dy})
	-- ({7.3061*\dx},{0.0000*\dy})
	-- ({7.3161*\dx},{0.0000*\dy})
	-- ({7.3261*\dx},{0.0000*\dy})
	-- ({7.3361*\dx},{0.0000*\dy})
	-- ({7.3461*\dx},{0.0000*\dy})
	-- ({7.3561*\dx},{0.0000*\dy})
	-- ({7.3661*\dx},{0.0000*\dy})
	-- ({7.3761*\dx},{0.0000*\dy})
	-- ({7.3862*\dx},{0.0000*\dy})
	-- ({7.3962*\dx},{0.0000*\dy})
	-- ({7.4062*\dx},{0.0000*\dy})
	-- ({7.4162*\dx},{0.0000*\dy})
	-- ({7.4262*\dx},{0.0000*\dy})
	-- ({7.4362*\dx},{0.0000*\dy})
	-- ({7.4462*\dx},{0.0000*\dy})
	-- ({7.4562*\dx},{0.0000*\dy})
	-- ({7.4662*\dx},{0.0000*\dy})
	-- ({7.4762*\dx},{0.0000*\dy})
	-- ({7.4862*\dx},{0.0000*\dy})
	-- ({7.4962*\dx},{0.0000*\dy})
	-- ({7.5063*\dx},{0.0000*\dy})
	-- ({7.5163*\dx},{0.0000*\dy})
	-- ({7.5263*\dx},{0.0000*\dy})
	-- ({7.5363*\dx},{0.0000*\dy})
	-- ({7.5463*\dx},{0.0000*\dy})
	-- ({7.5563*\dx},{0.0000*\dy})
	-- ({7.5663*\dx},{0.0000*\dy})
	-- ({7.5763*\dx},{0.0000*\dy})
	-- ({7.5863*\dx},{0.0000*\dy})
	-- ({7.5963*\dx},{0.0000*\dy})
	-- ({7.6063*\dx},{0.0000*\dy})
	-- ({7.6163*\dx},{0.0000*\dy})
	-- ({7.6264*\dx},{0.0000*\dy})
	-- ({7.6364*\dx},{0.0000*\dy})
	-- ({7.6464*\dx},{0.0000*\dy})
	-- ({7.6564*\dx},{0.0000*\dy})
	-- ({7.6664*\dx},{0.0000*\dy})
	-- ({7.6764*\dx},{0.0000*\dy})
	-- ({7.6864*\dx},{0.0000*\dy})
	-- ({7.6964*\dx},{0.0000*\dy})
	-- ({7.7064*\dx},{0.0000*\dy})
	-- ({7.7164*\dx},{0.0000*\dy})
	-- ({7.7264*\dx},{0.0000*\dy})
	-- ({7.7364*\dx},{0.0000*\dy})
	-- ({7.7465*\dx},{0.0000*\dy})
	-- ({7.7565*\dx},{0.0000*\dy})
	-- ({7.7665*\dx},{0.0000*\dy})
	-- ({7.7765*\dx},{0.0000*\dy})
	-- ({7.7865*\dx},{0.0000*\dy})
	-- ({7.7965*\dx},{0.0000*\dy})
	-- ({7.8065*\dx},{0.0000*\dy})
	-- ({7.8165*\dx},{0.0000*\dy})
	-- ({7.8265*\dx},{0.0000*\dy})
	-- ({7.8365*\dx},{0.0000*\dy})
	-- ({7.8465*\dx},{0.0000*\dy})
	-- ({7.8565*\dx},{0.0000*\dy})
	-- ({7.8666*\dx},{0.0000*\dy})
	-- ({7.8766*\dx},{0.0000*\dy})
	-- ({7.8866*\dx},{0.0000*\dy})
	-- ({7.8966*\dx},{0.0000*\dy})
	-- ({7.9066*\dx},{0.0000*\dy})
	-- ({7.9166*\dx},{0.0000*\dy})
	-- ({7.9266*\dx},{0.0000*\dy})
	-- ({7.9366*\dx},{0.0000*\dy})
	-- ({7.9466*\dx},{0.0000*\dy})
	-- ({7.9566*\dx},{0.0000*\dy})
	-- ({7.9666*\dx},{0.0000*\dy})
	-- ({7.9766*\dx},{0.0000*\dy})
	-- ({7.9867*\dx},{0.0000*\dy})
	-- ({7.9967*\dx},{0.0000*\dy})
	-- ({8.0067*\dx},{0.0000*\dy})
	-- ({8.0167*\dx},{0.0000*\dy})
	-- ({8.0267*\dx},{0.0000*\dy})
	-- ({8.0367*\dx},{0.0000*\dy})
	-- ({8.0467*\dx},{0.0000*\dy})
	-- ({8.0567*\dx},{0.0000*\dy})
	-- ({8.0667*\dx},{0.0000*\dy})
	-- ({8.0767*\dx},{0.0000*\dy})
	-- ({8.0867*\dx},{0.0000*\dy})
	-- ({8.0967*\dx},{0.0000*\dy})
	-- ({8.1068*\dx},{0.0000*\dy})
	-- ({8.1168*\dx},{0.0000*\dy})
	-- ({8.1268*\dx},{0.0000*\dy})
	-- ({8.1368*\dx},{0.0000*\dy})
	-- ({8.1468*\dx},{0.0000*\dy})
	-- ({8.1568*\dx},{0.0000*\dy})
	-- ({8.1668*\dx},{0.0000*\dy})
	-- ({8.1768*\dx},{0.0000*\dy})
	-- ({8.1868*\dx},{0.0000*\dy})
	-- ({8.1968*\dx},{0.0000*\dy})
	-- ({8.2068*\dx},{0.0000*\dy})
	-- ({8.2168*\dx},{0.0000*\dy})
	-- ({8.2269*\dx},{0.0000*\dy})
	-- ({8.2369*\dx},{0.0000*\dy})
	-- ({8.2469*\dx},{0.0000*\dy})
	-- ({8.2569*\dx},{0.0000*\dy})
	-- ({8.2669*\dx},{0.0000*\dy})
	-- ({8.2769*\dx},{0.0000*\dy})
	-- ({8.2869*\dx},{0.0000*\dy})
	-- ({8.2969*\dx},{0.0000*\dy})
	-- ({8.3069*\dx},{0.0000*\dy})
	-- ({8.3169*\dx},{0.0000*\dy})
	-- ({8.3269*\dx},{0.0000*\dy})
	-- ({8.3369*\dx},{0.0000*\dy})
	-- ({8.3470*\dx},{0.0000*\dy})
	-- ({8.3570*\dx},{0.0000*\dy})
	-- ({8.3670*\dx},{0.0000*\dy})
	-- ({8.3770*\dx},{0.0000*\dy})
	-- ({8.3870*\dx},{0.0000*\dy})
	-- ({8.3970*\dx},{0.0000*\dy})
	-- ({8.4070*\dx},{0.0000*\dy})
	-- ({8.4170*\dx},{0.0000*\dy})
	-- ({8.4270*\dx},{0.0000*\dy})
	-- ({8.4370*\dx},{0.0000*\dy})
	-- ({8.4470*\dx},{0.0000*\dy})
	-- ({8.4570*\dx},{0.0000*\dy})
	-- ({8.4671*\dx},{0.0000*\dy})
	-- ({8.4771*\dx},{0.0000*\dy})
	-- ({8.4871*\dx},{0.0000*\dy})
	-- ({8.4971*\dx},{0.0000*\dy})
	-- ({8.5071*\dx},{0.0000*\dy})
	-- ({8.5171*\dx},{0.0000*\dy})
	-- ({8.5271*\dx},{0.0000*\dy})
	-- ({8.5371*\dx},{0.0000*\dy})
	-- ({8.5471*\dx},{0.0000*\dy})
	-- ({8.5571*\dx},{0.0000*\dy})
	-- ({8.5671*\dx},{0.0000*\dy})
	-- ({8.5771*\dx},{0.0000*\dy})
	-- ({8.5872*\dx},{0.0000*\dy})
	-- ({8.5972*\dx},{0.0000*\dy})
	-- ({8.6072*\dx},{0.0000*\dy})
	-- ({8.6172*\dx},{0.0000*\dy})
	-- ({8.6272*\dx},{0.0000*\dy})
	-- ({8.6372*\dx},{0.0000*\dy})
	-- ({8.6472*\dx},{0.0000*\dy})
	-- ({8.6572*\dx},{0.0000*\dy})
	-- ({8.6672*\dx},{0.0000*\dy})
	-- ({8.6772*\dx},{0.0000*\dy})
	-- ({8.6872*\dx},{0.0000*\dy})
	-- ({8.6972*\dx},{0.0000*\dy})
	-- ({8.7073*\dx},{0.0000*\dy})
	-- ({8.7173*\dx},{0.0000*\dy})
	-- ({8.7273*\dx},{0.0000*\dy})
	-- ({8.7373*\dx},{0.0000*\dy})
	-- ({8.7473*\dx},{0.0000*\dy})
	-- ({8.7573*\dx},{0.0000*\dy})
	-- ({8.7673*\dx},{0.0000*\dy})
	-- ({8.7773*\dx},{0.0000*\dy})
	-- ({8.7873*\dx},{0.0000*\dy})
	-- ({8.7973*\dx},{0.0000*\dy})
	-- ({8.8073*\dx},{0.0000*\dy})
	-- ({8.8173*\dx},{0.0000*\dy})
	-- ({8.8274*\dx},{0.0000*\dy})
	-- ({8.8374*\dx},{0.0000*\dy})
	-- ({8.8474*\dx},{0.0000*\dy})
	-- ({8.8574*\dx},{0.0000*\dy})
	-- ({8.8674*\dx},{0.0000*\dy})
	-- ({8.8774*\dx},{0.0000*\dy})
	-- ({8.8874*\dx},{0.0000*\dy})
	-- ({8.8974*\dx},{0.0000*\dy})
	-- ({8.9074*\dx},{0.0000*\dy})
	-- ({8.9174*\dx},{0.0000*\dy})
	-- ({8.9274*\dx},{0.0000*\dy})
	-- ({8.9374*\dx},{0.0000*\dy})
	-- ({8.9475*\dx},{0.0000*\dy})
	-- ({8.9575*\dx},{0.0000*\dy})
	-- ({8.9675*\dx},{0.0000*\dy})
	-- ({8.9775*\dx},{0.0000*\dy})
	-- ({8.9875*\dx},{0.0000*\dy})
	-- ({8.9975*\dx},{0.0000*\dy})
	-- ({9.0075*\dx},{0.0000*\dy})
	-- ({9.0175*\dx},{0.0001*\dy})
	-- ({9.0275*\dx},{0.0002*\dy})
	-- ({9.0375*\dx},{0.0003*\dy})
	-- ({9.0475*\dx},{0.0006*\dy})
	-- ({9.0575*\dx},{0.0008*\dy})
	-- ({9.0676*\dx},{0.0011*\dy})
	-- ({9.0776*\dx},{0.0015*\dy})
	-- ({9.0876*\dx},{0.0019*\dy})
	-- ({9.0976*\dx},{0.0023*\dy})
	-- ({9.1076*\dx},{0.0028*\dy})
	-- ({9.1176*\dx},{0.0034*\dy})
	-- ({9.1276*\dx},{0.0039*\dy})
	-- ({9.1376*\dx},{0.0046*\dy})
	-- ({9.1476*\dx},{0.0052*\dy})
	-- ({9.1576*\dx},{0.0060*\dy})
	-- ({9.1676*\dx},{0.0067*\dy})
	-- ({9.1776*\dx},{0.0075*\dy})
	-- ({9.1877*\dx},{0.0084*\dy})
	-- ({9.1977*\dx},{0.0093*\dy})
	-- ({9.2077*\dx},{0.0103*\dy})
	-- ({9.2177*\dx},{0.0112*\dy})
	-- ({9.2277*\dx},{0.0123*\dy})
	-- ({9.2377*\dx},{0.0134*\dy})
	-- ({9.2477*\dx},{0.0145*\dy})
	-- ({9.2577*\dx},{0.0156*\dy})
	-- ({9.2677*\dx},{0.0168*\dy})
	-- ({9.2777*\dx},{0.0181*\dy})
	-- ({9.2877*\dx},{0.0194*\dy})
	-- ({9.2977*\dx},{0.0207*\dy})
	-- ({9.3078*\dx},{0.0221*\dy})
	-- ({9.3178*\dx},{0.0236*\dy})
	-- ({9.3278*\dx},{0.0250*\dy})
	-- ({9.3378*\dx},{0.0266*\dy})
	-- ({9.3478*\dx},{0.0281*\dy})
	-- ({9.3578*\dx},{0.0297*\dy})
	-- ({9.3678*\dx},{0.0314*\dy})
	-- ({9.3778*\dx},{0.0331*\dy})
	-- ({9.3878*\dx},{0.0348*\dy})
	-- ({9.3978*\dx},{0.0366*\dy})
	-- ({9.4078*\dx},{0.0385*\dy})
	-- ({9.4178*\dx},{0.0403*\dy})
	-- ({9.4279*\dx},{0.0423*\dy})
	-- ({9.4379*\dx},{0.0443*\dy})
	-- ({9.4479*\dx},{0.0463*\dy})
	-- ({9.4579*\dx},{0.0484*\dy})
	-- ({9.4679*\dx},{0.0505*\dy})
	-- ({9.4779*\dx},{0.0527*\dy})
	-- ({9.4879*\dx},{0.0549*\dy})
	-- ({9.4979*\dx},{0.0571*\dy})
	-- ({9.5079*\dx},{0.0595*\dy})
	-- ({9.5179*\dx},{0.0618*\dy})
	-- ({9.5279*\dx},{0.0643*\dy})
	-- ({9.5379*\dx},{0.0667*\dy})
	-- ({9.5480*\dx},{0.0693*\dy})
	-- ({9.5580*\dx},{0.0719*\dy})
	-- ({9.5680*\dx},{0.0745*\dy})
	-- ({9.5780*\dx},{0.0772*\dy})
	-- ({9.5880*\dx},{0.0799*\dy})
	-- ({9.5980*\dx},{0.0827*\dy})
	-- ({9.6080*\dx},{0.0856*\dy})
	-- ({9.6180*\dx},{0.0885*\dy})
	-- ({9.6280*\dx},{0.0915*\dy})
	-- ({9.6380*\dx},{0.0945*\dy})
	-- ({9.6480*\dx},{0.0976*\dy})
	-- ({9.6580*\dx},{0.1008*\dy})
	-- ({9.6681*\dx},{0.1040*\dy})
	-- ({9.6781*\dx},{0.1072*\dy})
	-- ({9.6881*\dx},{0.1106*\dy})
	-- ({9.6981*\dx},{0.1140*\dy})
	-- ({9.7081*\dx},{0.1174*\dy})
	-- ({9.7181*\dx},{0.1210*\dy})
	-- ({9.7281*\dx},{0.1246*\dy})
	-- ({9.7381*\dx},{0.1282*\dy})
	-- ({9.7481*\dx},{0.1320*\dy})
	-- ({9.7581*\dx},{0.1357*\dy})
	-- ({9.7681*\dx},{0.1396*\dy})
	-- ({9.7781*\dx},{0.1435*\dy})
	-- ({9.7882*\dx},{0.1476*\dy})
	-- ({9.7982*\dx},{0.1516*\dy})
	-- ({9.8082*\dx},{0.1558*\dy})
	-- ({9.8182*\dx},{0.1600*\dy})
	-- ({9.8282*\dx},{0.1643*\dy})
	-- ({9.8382*\dx},{0.1687*\dy})
	-- ({9.8482*\dx},{0.1731*\dy})
	-- ({9.8582*\dx},{0.1776*\dy})
	-- ({9.8682*\dx},{0.1822*\dy})
	-- ({9.8782*\dx},{0.1869*\dy})
	-- ({9.8882*\dx},{0.1916*\dy})
	-- ({9.8982*\dx},{0.1965*\dy})
	-- ({9.9083*\dx},{0.2014*\dy})
	-- ({9.9183*\dx},{0.2063*\dy})
	-- ({9.9283*\dx},{0.2114*\dy})
	-- ({9.9383*\dx},{0.2165*\dy})
	-- ({9.9483*\dx},{0.2218*\dy})
	-- ({9.9583*\dx},{0.2271*\dy})
	-- ({9.9683*\dx},{0.2324*\dy})
	-- ({9.9783*\dx},{0.2379*\dy})
	-- ({9.9883*\dx},{0.2434*\dy})
	-- ({9.9983*\dx},{0.2491*\dy})
	-- ({10.0083*\dx},{0.2548*\dy})
	-- ({10.0183*\dx},{0.2606*\dy})
	-- ({10.0284*\dx},{0.2665*\dy})
	-- ({10.0384*\dx},{0.2725*\dy})
	-- ({10.0484*\dx},{0.2786*\dy})
	-- ({10.0584*\dx},{0.2848*\dy})
	-- ({10.0684*\dx},{0.2912*\dy})
	-- ({10.0784*\dx},{0.2976*\dy})
	-- ({10.0884*\dx},{0.3041*\dy})
	-- ({10.0984*\dx},{0.3108*\dy})
	-- ({10.1084*\dx},{0.3176*\dy})
	-- ({10.1184*\dx},{0.3245*\dy})
	-- ({10.1284*\dx},{0.3315*\dy})
	-- ({10.1384*\dx},{0.3386*\dy})
	-- ({10.1485*\dx},{0.3458*\dy})
	-- ({10.1585*\dx},{0.3531*\dy})
	-- ({10.1685*\dx},{0.3606*\dy})
	-- ({10.1785*\dx},{0.3682*\dy})
	-- ({10.1885*\dx},{0.3759*\dy})
	-- ({10.1985*\dx},{0.3837*\dy})
	-- ({10.2085*\dx},{0.3916*\dy})
	-- ({10.2185*\dx},{0.3996*\dy})
	-- ({10.2285*\dx},{0.4078*\dy})
	-- ({10.2385*\dx},{0.4161*\dy})
	-- ({10.2485*\dx},{0.4244*\dy})
	-- ({10.2585*\dx},{0.4329*\dy})
	-- ({10.2686*\dx},{0.4416*\dy})
	-- ({10.2786*\dx},{0.4503*\dy})
	-- ({10.2886*\dx},{0.4591*\dy})
	-- ({10.2986*\dx},{0.4681*\dy})
	-- ({10.3086*\dx},{0.4772*\dy})
	-- ({10.3186*\dx},{0.4863*\dy})
	-- ({10.3286*\dx},{0.4956*\dy})
	-- ({10.3386*\dx},{0.5050*\dy})
	-- ({10.3486*\dx},{0.5144*\dy})
	-- ({10.3586*\dx},{0.5240*\dy})
	-- ({10.3686*\dx},{0.5336*\dy})
	-- ({10.3786*\dx},{0.5434*\dy})
	-- ({10.3887*\dx},{0.5532*\dy})
	-- ({10.3987*\dx},{0.5631*\dy})
	-- ({10.4087*\dx},{0.5731*\dy})
	-- ({10.4187*\dx},{0.5832*\dy})
	-- ({10.4287*\dx},{0.5933*\dy})
	-- ({10.4387*\dx},{0.6034*\dy})
	-- ({10.4487*\dx},{0.6137*\dy})
	-- ({10.4587*\dx},{0.6239*\dy})
	-- ({10.4687*\dx},{0.6343*\dy})
	-- ({10.4787*\dx},{0.6446*\dy})
	-- ({10.4887*\dx},{0.6550*\dy})
	-- ({10.4987*\dx},{0.6654*\dy})
	-- ({10.5088*\dx},{0.6758*\dy})
	-- ({10.5188*\dx},{0.6862*\dy})
	-- ({10.5288*\dx},{0.6966*\dy})
	-- ({10.5388*\dx},{0.7069*\dy})
	-- ({10.5488*\dx},{0.7173*\dy})
	-- ({10.5588*\dx},{0.7276*\dy})
	-- ({10.5688*\dx},{0.7379*\dy})
	-- ({10.5788*\dx},{0.7481*\dy})
	-- ({10.5888*\dx},{0.7582*\dy})
	-- ({10.5988*\dx},{0.7683*\dy})
	-- ({10.6088*\dx},{0.7783*\dy})
	-- ({10.6188*\dx},{0.7882*\dy})
	-- ({10.6289*\dx},{0.7980*\dy})
	-- ({10.6389*\dx},{0.8077*\dy})
	-- ({10.6489*\dx},{0.8172*\dy})
	-- ({10.6589*\dx},{0.8266*\dy})
	-- ({10.6689*\dx},{0.8358*\dy})
	-- ({10.6789*\dx},{0.8449*\dy})
	-- ({10.6889*\dx},{0.8538*\dy})
	-- ({10.6989*\dx},{0.8625*\dy})
	-- ({10.7089*\dx},{0.8711*\dy})
	-- ({10.7189*\dx},{0.8794*\dy})
	-- ({10.7289*\dx},{0.8875*\dy})
	-- ({10.7389*\dx},{0.8954*\dy})
	-- ({10.7490*\dx},{0.9030*\dy})
	-- ({10.7590*\dx},{0.9104*\dy})
	-- ({10.7690*\dx},{0.9176*\dy})
	-- ({10.7790*\dx},{0.9245*\dy})
	-- ({10.7890*\dx},{0.9311*\dy})
	-- ({10.7990*\dx},{0.9374*\dy})
	-- ({10.8090*\dx},{0.9435*\dy})
	-- ({10.8190*\dx},{0.9493*\dy})
	-- ({10.8290*\dx},{0.9547*\dy})
	-- ({10.8390*\dx},{0.9599*\dy})
	-- ({10.8490*\dx},{0.9648*\dy})
	-- ({10.8590*\dx},{0.9693*\dy})
	-- ({10.8691*\dx},{0.9736*\dy})
	-- ({10.8791*\dx},{0.9775*\dy})
	-- ({10.8891*\dx},{0.9811*\dy})
	-- ({10.8991*\dx},{0.9844*\dy})
	-- ({10.9091*\dx},{0.9874*\dy})
	-- ({10.9191*\dx},{0.9901*\dy})
	-- ({10.9291*\dx},{0.9924*\dy})
	-- ({10.9391*\dx},{0.9944*\dy})
	-- ({10.9491*\dx},{0.9961*\dy})
	-- ({10.9591*\dx},{0.9975*\dy})
	-- ({10.9691*\dx},{0.9986*\dy})
	-- ({10.9791*\dx},{0.9994*\dy})
	-- ({10.9892*\dx},{0.9998*\dy})
	-- ({10.9992*\dx},{1.0000*\dy})
	-- ({11.0092*\dx},{1.0000*\dy})
	-- ({11.0192*\dx},{1.0000*\dy})
	-- ({11.0292*\dx},{1.0000*\dy})
	-- ({11.0392*\dx},{1.0000*\dy})
	-- ({11.0492*\dx},{1.0000*\dy})
	-- ({11.0592*\dx},{1.0000*\dy})
	-- ({11.0692*\dx},{1.0000*\dy})
	-- ({11.0792*\dx},{1.0000*\dy})
	-- ({11.0892*\dx},{1.0000*\dy})
	-- ({11.0992*\dx},{1.0000*\dy})
	-- ({11.1093*\dx},{1.0000*\dy})
	-- ({11.1193*\dx},{1.0000*\dy})
	-- ({11.1293*\dx},{1.0000*\dy})
	-- ({11.1393*\dx},{1.0000*\dy})
	-- ({11.1493*\dx},{1.0000*\dy})
	-- ({11.1593*\dx},{1.0000*\dy})
	-- ({11.1693*\dx},{1.0000*\dy})
	-- ({11.1793*\dx},{1.0000*\dy})
	-- ({11.1893*\dx},{1.0000*\dy})
	-- ({11.1993*\dx},{1.0000*\dy})
	-- ({11.2093*\dx},{1.0000*\dy})
	-- ({11.2193*\dx},{1.0000*\dy})
	-- ({11.2294*\dx},{1.0000*\dy})
	-- ({11.2394*\dx},{1.0000*\dy})
	-- ({11.2494*\dx},{1.0000*\dy})
	-- ({11.2594*\dx},{1.0000*\dy})
	-- ({11.2694*\dx},{1.0000*\dy})
	-- ({11.2794*\dx},{1.0000*\dy})
	-- ({11.2894*\dx},{1.0000*\dy})
	-- ({11.2994*\dx},{1.0000*\dy})
	-- ({11.3094*\dx},{1.0000*\dy})
	-- ({11.3194*\dx},{1.0000*\dy})
	-- ({11.3294*\dx},{1.0000*\dy})
	-- ({11.3394*\dx},{1.0000*\dy})
	-- ({11.3495*\dx},{1.0000*\dy})
	-- ({11.3595*\dx},{1.0000*\dy})
	-- ({11.3695*\dx},{1.0000*\dy})
	-- ({11.3795*\dx},{1.0000*\dy})
	-- ({11.3895*\dx},{1.0000*\dy})
	-- ({11.3995*\dx},{1.0000*\dy})
	-- ({11.4095*\dx},{1.0000*\dy})
	-- ({11.4195*\dx},{1.0000*\dy})
	-- ({11.4295*\dx},{1.0000*\dy})
	-- ({11.4395*\dx},{1.0000*\dy})
	-- ({11.4495*\dx},{1.0000*\dy})
	-- ({11.4595*\dx},{1.0000*\dy})
	-- ({11.4696*\dx},{1.0000*\dy})
	-- ({11.4796*\dx},{1.0000*\dy})
	-- ({11.4896*\dx},{1.0000*\dy})
	-- ({11.4996*\dx},{1.0000*\dy})
	-- ({11.5096*\dx},{1.0000*\dy})
	-- ({11.5196*\dx},{1.0000*\dy})
	-- ({11.5296*\dx},{1.0000*\dy})
	-- ({11.5396*\dx},{1.0000*\dy})
	-- ({11.5496*\dx},{1.0000*\dy})
	-- ({11.5596*\dx},{1.0000*\dy})
	-- ({11.5696*\dx},{1.0000*\dy})
	-- ({11.5796*\dx},{1.0000*\dy})
	-- ({11.5897*\dx},{1.0000*\dy})
	-- ({11.5997*\dx},{1.0000*\dy})
	-- ({11.6097*\dx},{1.0000*\dy})
	-- ({11.6197*\dx},{1.0000*\dy})
	-- ({11.6297*\dx},{1.0000*\dy})
	-- ({11.6397*\dx},{1.0000*\dy})
	-- ({11.6497*\dx},{1.0000*\dy})
	-- ({11.6597*\dx},{1.0000*\dy})
	-- ({11.6697*\dx},{1.0000*\dy})
	-- ({11.6797*\dx},{1.0000*\dy})
	-- ({11.6897*\dx},{1.0000*\dy})
	-- ({11.6997*\dx},{1.0000*\dy})
	-- ({11.7098*\dx},{1.0000*\dy})
	-- ({11.7198*\dx},{1.0000*\dy})
	-- ({11.7298*\dx},{1.0000*\dy})
	-- ({11.7398*\dx},{1.0000*\dy})
	-- ({11.7498*\dx},{1.0000*\dy})
	-- ({11.7598*\dx},{1.0000*\dy})
	-- ({11.7698*\dx},{1.0000*\dy})
	-- ({11.7798*\dx},{1.0000*\dy})
	-- ({11.7898*\dx},{1.0000*\dy})
	-- ({11.7998*\dx},{1.0000*\dy})
	-- ({11.8098*\dx},{1.0000*\dy})
	-- ({11.8198*\dx},{1.0000*\dy})
	-- ({11.8299*\dx},{1.0000*\dy})
	-- ({11.8399*\dx},{1.0000*\dy})
	-- ({11.8499*\dx},{1.0000*\dy})
	-- ({11.8599*\dx},{1.0000*\dy})
	-- ({11.8699*\dx},{1.0000*\dy})
	-- ({11.8799*\dx},{1.0000*\dy})
	-- ({11.8899*\dx},{1.0000*\dy})
	-- ({11.8999*\dx},{1.0000*\dy})
	-- ({11.9099*\dx},{1.0000*\dy})
	-- ({11.9199*\dx},{1.0000*\dy})
	-- ({11.9299*\dx},{1.0000*\dy})
	-- ({11.9399*\dx},{1.0000*\dy})
	-- ({11.9500*\dx},{1.0000*\dy})
	-- ({11.9600*\dx},{1.0000*\dy})
	-- ({11.9700*\dx},{1.0000*\dy})
	-- ({11.9800*\dx},{1.0000*\dy})
	-- ({11.9900*\dx},{1.0000*\dy})
	-- ({12.0000*\dx},{1.0000*\dy})
}
\def\cpsifour{
	({0.0000*\dx},{1.0000*\dy})
	-- ({0.0100*\dx},{1.0000*\dy})
	-- ({0.0200*\dx},{1.0000*\dy})
	-- ({0.0300*\dx},{1.0000*\dy})
	-- ({0.0400*\dx},{1.0000*\dy})
	-- ({0.0500*\dx},{1.0000*\dy})
	-- ({0.0601*\dx},{1.0000*\dy})
	-- ({0.0701*\dx},{1.0000*\dy})
	-- ({0.0801*\dx},{1.0000*\dy})
	-- ({0.0901*\dx},{1.0000*\dy})
	-- ({0.1001*\dx},{1.0000*\dy})
	-- ({0.1101*\dx},{1.0000*\dy})
	-- ({0.1201*\dx},{1.0000*\dy})
	-- ({0.1301*\dx},{1.0000*\dy})
	-- ({0.1401*\dx},{1.0000*\dy})
	-- ({0.1501*\dx},{1.0000*\dy})
	-- ({0.1601*\dx},{1.0000*\dy})
	-- ({0.1701*\dx},{1.0000*\dy})
	-- ({0.1802*\dx},{1.0000*\dy})
	-- ({0.1902*\dx},{1.0000*\dy})
	-- ({0.2002*\dx},{1.0000*\dy})
	-- ({0.2102*\dx},{1.0000*\dy})
	-- ({0.2202*\dx},{1.0000*\dy})
	-- ({0.2302*\dx},{1.0000*\dy})
	-- ({0.2402*\dx},{1.0000*\dy})
	-- ({0.2502*\dx},{1.0000*\dy})
	-- ({0.2602*\dx},{1.0000*\dy})
	-- ({0.2702*\dx},{1.0000*\dy})
	-- ({0.2802*\dx},{1.0000*\dy})
	-- ({0.2902*\dx},{1.0000*\dy})
	-- ({0.3003*\dx},{1.0000*\dy})
	-- ({0.3103*\dx},{1.0000*\dy})
	-- ({0.3203*\dx},{1.0000*\dy})
	-- ({0.3303*\dx},{1.0000*\dy})
	-- ({0.3403*\dx},{1.0000*\dy})
	-- ({0.3503*\dx},{1.0000*\dy})
	-- ({0.3603*\dx},{1.0000*\dy})
	-- ({0.3703*\dx},{1.0000*\dy})
	-- ({0.3803*\dx},{1.0000*\dy})
	-- ({0.3903*\dx},{1.0000*\dy})
	-- ({0.4003*\dx},{1.0000*\dy})
	-- ({0.4103*\dx},{1.0000*\dy})
	-- ({0.4204*\dx},{1.0000*\dy})
	-- ({0.4304*\dx},{1.0000*\dy})
	-- ({0.4404*\dx},{1.0000*\dy})
	-- ({0.4504*\dx},{1.0000*\dy})
	-- ({0.4604*\dx},{1.0000*\dy})
	-- ({0.4704*\dx},{1.0000*\dy})
	-- ({0.4804*\dx},{1.0000*\dy})
	-- ({0.4904*\dx},{1.0000*\dy})
	-- ({0.5004*\dx},{1.0000*\dy})
	-- ({0.5104*\dx},{1.0000*\dy})
	-- ({0.5204*\dx},{1.0000*\dy})
	-- ({0.5304*\dx},{1.0000*\dy})
	-- ({0.5405*\dx},{1.0000*\dy})
	-- ({0.5505*\dx},{1.0000*\dy})
	-- ({0.5605*\dx},{1.0000*\dy})
	-- ({0.5705*\dx},{1.0000*\dy})
	-- ({0.5805*\dx},{1.0000*\dy})
	-- ({0.5905*\dx},{1.0000*\dy})
	-- ({0.6005*\dx},{1.0000*\dy})
	-- ({0.6105*\dx},{1.0000*\dy})
	-- ({0.6205*\dx},{1.0000*\dy})
	-- ({0.6305*\dx},{1.0000*\dy})
	-- ({0.6405*\dx},{1.0000*\dy})
	-- ({0.6505*\dx},{1.0000*\dy})
	-- ({0.6606*\dx},{1.0000*\dy})
	-- ({0.6706*\dx},{1.0000*\dy})
	-- ({0.6806*\dx},{1.0000*\dy})
	-- ({0.6906*\dx},{1.0000*\dy})
	-- ({0.7006*\dx},{1.0000*\dy})
	-- ({0.7106*\dx},{1.0000*\dy})
	-- ({0.7206*\dx},{1.0000*\dy})
	-- ({0.7306*\dx},{1.0000*\dy})
	-- ({0.7406*\dx},{1.0000*\dy})
	-- ({0.7506*\dx},{1.0000*\dy})
	-- ({0.7606*\dx},{1.0000*\dy})
	-- ({0.7706*\dx},{1.0000*\dy})
	-- ({0.7807*\dx},{1.0000*\dy})
	-- ({0.7907*\dx},{1.0000*\dy})
	-- ({0.8007*\dx},{1.0000*\dy})
	-- ({0.8107*\dx},{1.0000*\dy})
	-- ({0.8207*\dx},{1.0000*\dy})
	-- ({0.8307*\dx},{1.0000*\dy})
	-- ({0.8407*\dx},{1.0000*\dy})
	-- ({0.8507*\dx},{1.0000*\dy})
	-- ({0.8607*\dx},{1.0000*\dy})
	-- ({0.8707*\dx},{1.0000*\dy})
	-- ({0.8807*\dx},{1.0000*\dy})
	-- ({0.8907*\dx},{1.0000*\dy})
	-- ({0.9008*\dx},{1.0000*\dy})
	-- ({0.9108*\dx},{1.0000*\dy})
	-- ({0.9208*\dx},{1.0000*\dy})
	-- ({0.9308*\dx},{1.0000*\dy})
	-- ({0.9408*\dx},{1.0000*\dy})
	-- ({0.9508*\dx},{1.0000*\dy})
	-- ({0.9608*\dx},{1.0000*\dy})
	-- ({0.9708*\dx},{1.0000*\dy})
	-- ({0.9808*\dx},{1.0000*\dy})
	-- ({0.9908*\dx},{1.0000*\dy})
	-- ({1.0008*\dx},{1.0000*\dy})
	-- ({1.0108*\dx},{1.0000*\dy})
	-- ({1.0209*\dx},{1.0000*\dy})
	-- ({1.0309*\dx},{1.0000*\dy})
	-- ({1.0409*\dx},{1.0000*\dy})
	-- ({1.0509*\dx},{1.0000*\dy})
	-- ({1.0609*\dx},{1.0000*\dy})
	-- ({1.0709*\dx},{1.0000*\dy})
	-- ({1.0809*\dx},{1.0000*\dy})
	-- ({1.0909*\dx},{1.0000*\dy})
	-- ({1.1009*\dx},{1.0000*\dy})
	-- ({1.1109*\dx},{1.0000*\dy})
	-- ({1.1209*\dx},{1.0000*\dy})
	-- ({1.1309*\dx},{1.0000*\dy})
	-- ({1.1410*\dx},{1.0000*\dy})
	-- ({1.1510*\dx},{1.0000*\dy})
	-- ({1.1610*\dx},{1.0000*\dy})
	-- ({1.1710*\dx},{1.0000*\dy})
	-- ({1.1810*\dx},{1.0000*\dy})
	-- ({1.1910*\dx},{1.0000*\dy})
	-- ({1.2010*\dx},{1.0000*\dy})
	-- ({1.2110*\dx},{1.0000*\dy})
	-- ({1.2210*\dx},{1.0000*\dy})
	-- ({1.2310*\dx},{1.0000*\dy})
	-- ({1.2410*\dx},{1.0000*\dy})
	-- ({1.2510*\dx},{1.0000*\dy})
	-- ({1.2611*\dx},{1.0000*\dy})
	-- ({1.2711*\dx},{1.0000*\dy})
	-- ({1.2811*\dx},{1.0000*\dy})
	-- ({1.2911*\dx},{1.0000*\dy})
	-- ({1.3011*\dx},{1.0000*\dy})
	-- ({1.3111*\dx},{1.0000*\dy})
	-- ({1.3211*\dx},{1.0000*\dy})
	-- ({1.3311*\dx},{1.0000*\dy})
	-- ({1.3411*\dx},{1.0000*\dy})
	-- ({1.3511*\dx},{1.0000*\dy})
	-- ({1.3611*\dx},{1.0000*\dy})
	-- ({1.3711*\dx},{1.0000*\dy})
	-- ({1.3812*\dx},{1.0000*\dy})
	-- ({1.3912*\dx},{1.0000*\dy})
	-- ({1.4012*\dx},{1.0000*\dy})
	-- ({1.4112*\dx},{1.0000*\dy})
	-- ({1.4212*\dx},{1.0000*\dy})
	-- ({1.4312*\dx},{1.0000*\dy})
	-- ({1.4412*\dx},{1.0000*\dy})
	-- ({1.4512*\dx},{1.0000*\dy})
	-- ({1.4612*\dx},{1.0000*\dy})
	-- ({1.4712*\dx},{1.0000*\dy})
	-- ({1.4812*\dx},{1.0000*\dy})
	-- ({1.4912*\dx},{1.0000*\dy})
	-- ({1.5013*\dx},{1.0000*\dy})
	-- ({1.5113*\dx},{1.0000*\dy})
	-- ({1.5213*\dx},{1.0000*\dy})
	-- ({1.5313*\dx},{1.0000*\dy})
	-- ({1.5413*\dx},{1.0000*\dy})
	-- ({1.5513*\dx},{1.0000*\dy})
	-- ({1.5613*\dx},{1.0000*\dy})
	-- ({1.5713*\dx},{1.0000*\dy})
	-- ({1.5813*\dx},{1.0000*\dy})
	-- ({1.5913*\dx},{1.0000*\dy})
	-- ({1.6013*\dx},{1.0000*\dy})
	-- ({1.6113*\dx},{1.0000*\dy})
	-- ({1.6214*\dx},{1.0000*\dy})
	-- ({1.6314*\dx},{1.0000*\dy})
	-- ({1.6414*\dx},{1.0000*\dy})
	-- ({1.6514*\dx},{1.0000*\dy})
	-- ({1.6614*\dx},{1.0000*\dy})
	-- ({1.6714*\dx},{1.0000*\dy})
	-- ({1.6814*\dx},{1.0000*\dy})
	-- ({1.6914*\dx},{1.0000*\dy})
	-- ({1.7014*\dx},{1.0000*\dy})
	-- ({1.7114*\dx},{1.0000*\dy})
	-- ({1.7214*\dx},{1.0000*\dy})
	-- ({1.7314*\dx},{1.0000*\dy})
	-- ({1.7415*\dx},{1.0000*\dy})
	-- ({1.7515*\dx},{1.0000*\dy})
	-- ({1.7615*\dx},{1.0000*\dy})
	-- ({1.7715*\dx},{1.0000*\dy})
	-- ({1.7815*\dx},{1.0000*\dy})
	-- ({1.7915*\dx},{1.0000*\dy})
	-- ({1.8015*\dx},{1.0000*\dy})
	-- ({1.8115*\dx},{1.0000*\dy})
	-- ({1.8215*\dx},{1.0000*\dy})
	-- ({1.8315*\dx},{1.0000*\dy})
	-- ({1.8415*\dx},{1.0000*\dy})
	-- ({1.8515*\dx},{1.0000*\dy})
	-- ({1.8616*\dx},{1.0000*\dy})
	-- ({1.8716*\dx},{1.0000*\dy})
	-- ({1.8816*\dx},{1.0000*\dy})
	-- ({1.8916*\dx},{1.0000*\dy})
	-- ({1.9016*\dx},{1.0000*\dy})
	-- ({1.9116*\dx},{1.0000*\dy})
	-- ({1.9216*\dx},{1.0000*\dy})
	-- ({1.9316*\dx},{1.0000*\dy})
	-- ({1.9416*\dx},{1.0000*\dy})
	-- ({1.9516*\dx},{1.0000*\dy})
	-- ({1.9616*\dx},{1.0000*\dy})
	-- ({1.9716*\dx},{1.0000*\dy})
	-- ({1.9817*\dx},{1.0000*\dy})
	-- ({1.9917*\dx},{1.0000*\dy})
	-- ({2.0017*\dx},{1.0000*\dy})
	-- ({2.0117*\dx},{1.0000*\dy})
	-- ({2.0217*\dx},{1.0000*\dy})
	-- ({2.0317*\dx},{1.0000*\dy})
	-- ({2.0417*\dx},{1.0000*\dy})
	-- ({2.0517*\dx},{1.0000*\dy})
	-- ({2.0617*\dx},{1.0000*\dy})
	-- ({2.0717*\dx},{1.0000*\dy})
	-- ({2.0817*\dx},{1.0000*\dy})
	-- ({2.0917*\dx},{1.0000*\dy})
	-- ({2.1018*\dx},{1.0000*\dy})
	-- ({2.1118*\dx},{1.0000*\dy})
	-- ({2.1218*\dx},{1.0000*\dy})
	-- ({2.1318*\dx},{1.0000*\dy})
	-- ({2.1418*\dx},{1.0000*\dy})
	-- ({2.1518*\dx},{1.0000*\dy})
	-- ({2.1618*\dx},{1.0000*\dy})
	-- ({2.1718*\dx},{1.0000*\dy})
	-- ({2.1818*\dx},{1.0000*\dy})
	-- ({2.1918*\dx},{1.0000*\dy})
	-- ({2.2018*\dx},{1.0000*\dy})
	-- ({2.2118*\dx},{1.0000*\dy})
	-- ({2.2219*\dx},{1.0000*\dy})
	-- ({2.2319*\dx},{1.0000*\dy})
	-- ({2.2419*\dx},{1.0000*\dy})
	-- ({2.2519*\dx},{1.0000*\dy})
	-- ({2.2619*\dx},{1.0000*\dy})
	-- ({2.2719*\dx},{1.0000*\dy})
	-- ({2.2819*\dx},{1.0000*\dy})
	-- ({2.2919*\dx},{1.0000*\dy})
	-- ({2.3019*\dx},{1.0000*\dy})
	-- ({2.3119*\dx},{1.0000*\dy})
	-- ({2.3219*\dx},{1.0000*\dy})
	-- ({2.3319*\dx},{1.0000*\dy})
	-- ({2.3420*\dx},{1.0000*\dy})
	-- ({2.3520*\dx},{1.0000*\dy})
	-- ({2.3620*\dx},{1.0000*\dy})
	-- ({2.3720*\dx},{1.0000*\dy})
	-- ({2.3820*\dx},{1.0000*\dy})
	-- ({2.3920*\dx},{1.0000*\dy})
	-- ({2.4020*\dx},{1.0000*\dy})
	-- ({2.4120*\dx},{1.0000*\dy})
	-- ({2.4220*\dx},{1.0000*\dy})
	-- ({2.4320*\dx},{1.0000*\dy})
	-- ({2.4420*\dx},{1.0000*\dy})
	-- ({2.4520*\dx},{1.0000*\dy})
	-- ({2.4621*\dx},{1.0000*\dy})
	-- ({2.4721*\dx},{1.0000*\dy})
	-- ({2.4821*\dx},{1.0000*\dy})
	-- ({2.4921*\dx},{1.0000*\dy})
	-- ({2.5021*\dx},{1.0000*\dy})
	-- ({2.5121*\dx},{1.0000*\dy})
	-- ({2.5221*\dx},{1.0000*\dy})
	-- ({2.5321*\dx},{1.0000*\dy})
	-- ({2.5421*\dx},{1.0000*\dy})
	-- ({2.5521*\dx},{1.0000*\dy})
	-- ({2.5621*\dx},{1.0000*\dy})
	-- ({2.5721*\dx},{1.0000*\dy})
	-- ({2.5822*\dx},{1.0000*\dy})
	-- ({2.5922*\dx},{1.0000*\dy})
	-- ({2.6022*\dx},{1.0000*\dy})
	-- ({2.6122*\dx},{1.0000*\dy})
	-- ({2.6222*\dx},{1.0000*\dy})
	-- ({2.6322*\dx},{1.0000*\dy})
	-- ({2.6422*\dx},{1.0000*\dy})
	-- ({2.6522*\dx},{1.0000*\dy})
	-- ({2.6622*\dx},{1.0000*\dy})
	-- ({2.6722*\dx},{1.0000*\dy})
	-- ({2.6822*\dx},{1.0000*\dy})
	-- ({2.6922*\dx},{1.0000*\dy})
	-- ({2.7023*\dx},{1.0000*\dy})
	-- ({2.7123*\dx},{1.0000*\dy})
	-- ({2.7223*\dx},{1.0000*\dy})
	-- ({2.7323*\dx},{1.0000*\dy})
	-- ({2.7423*\dx},{1.0000*\dy})
	-- ({2.7523*\dx},{1.0000*\dy})
	-- ({2.7623*\dx},{1.0000*\dy})
	-- ({2.7723*\dx},{1.0000*\dy})
	-- ({2.7823*\dx},{1.0000*\dy})
	-- ({2.7923*\dx},{1.0000*\dy})
	-- ({2.8023*\dx},{1.0000*\dy})
	-- ({2.8123*\dx},{1.0000*\dy})
	-- ({2.8224*\dx},{1.0000*\dy})
	-- ({2.8324*\dx},{1.0000*\dy})
	-- ({2.8424*\dx},{1.0000*\dy})
	-- ({2.8524*\dx},{1.0000*\dy})
	-- ({2.8624*\dx},{1.0000*\dy})
	-- ({2.8724*\dx},{1.0000*\dy})
	-- ({2.8824*\dx},{1.0000*\dy})
	-- ({2.8924*\dx},{1.0000*\dy})
	-- ({2.9024*\dx},{1.0000*\dy})
	-- ({2.9124*\dx},{1.0000*\dy})
	-- ({2.9224*\dx},{1.0000*\dy})
	-- ({2.9324*\dx},{1.0000*\dy})
	-- ({2.9425*\dx},{1.0000*\dy})
	-- ({2.9525*\dx},{1.0000*\dy})
	-- ({2.9625*\dx},{1.0000*\dy})
	-- ({2.9725*\dx},{1.0000*\dy})
	-- ({2.9825*\dx},{1.0000*\dy})
	-- ({2.9925*\dx},{1.0000*\dy})
	-- ({3.0025*\dx},{1.0000*\dy})
	-- ({3.0125*\dx},{1.0000*\dy})
	-- ({3.0225*\dx},{1.0000*\dy})
	-- ({3.0325*\dx},{1.0000*\dy})
	-- ({3.0425*\dx},{1.0000*\dy})
	-- ({3.0525*\dx},{1.0000*\dy})
	-- ({3.0626*\dx},{1.0000*\dy})
	-- ({3.0726*\dx},{1.0000*\dy})
	-- ({3.0826*\dx},{1.0000*\dy})
	-- ({3.0926*\dx},{1.0000*\dy})
	-- ({3.1026*\dx},{1.0000*\dy})
	-- ({3.1126*\dx},{1.0000*\dy})
	-- ({3.1226*\dx},{1.0000*\dy})
	-- ({3.1326*\dx},{1.0000*\dy})
	-- ({3.1426*\dx},{1.0000*\dy})
	-- ({3.1526*\dx},{1.0000*\dy})
	-- ({3.1626*\dx},{1.0000*\dy})
	-- ({3.1726*\dx},{1.0000*\dy})
	-- ({3.1827*\dx},{1.0000*\dy})
	-- ({3.1927*\dx},{1.0000*\dy})
	-- ({3.2027*\dx},{1.0000*\dy})
	-- ({3.2127*\dx},{1.0000*\dy})
	-- ({3.2227*\dx},{1.0000*\dy})
	-- ({3.2327*\dx},{1.0000*\dy})
	-- ({3.2427*\dx},{1.0000*\dy})
	-- ({3.2527*\dx},{1.0000*\dy})
	-- ({3.2627*\dx},{1.0000*\dy})
	-- ({3.2727*\dx},{1.0000*\dy})
	-- ({3.2827*\dx},{1.0000*\dy})
	-- ({3.2927*\dx},{1.0000*\dy})
	-- ({3.3028*\dx},{1.0000*\dy})
	-- ({3.3128*\dx},{1.0000*\dy})
	-- ({3.3228*\dx},{1.0000*\dy})
	-- ({3.3328*\dx},{1.0000*\dy})
	-- ({3.3428*\dx},{1.0000*\dy})
	-- ({3.3528*\dx},{1.0000*\dy})
	-- ({3.3628*\dx},{1.0000*\dy})
	-- ({3.3728*\dx},{1.0000*\dy})
	-- ({3.3828*\dx},{1.0000*\dy})
	-- ({3.3928*\dx},{1.0000*\dy})
	-- ({3.4028*\dx},{1.0000*\dy})
	-- ({3.4128*\dx},{1.0000*\dy})
	-- ({3.4229*\dx},{1.0000*\dy})
	-- ({3.4329*\dx},{1.0000*\dy})
	-- ({3.4429*\dx},{1.0000*\dy})
	-- ({3.4529*\dx},{1.0000*\dy})
	-- ({3.4629*\dx},{1.0000*\dy})
	-- ({3.4729*\dx},{1.0000*\dy})
	-- ({3.4829*\dx},{1.0000*\dy})
	-- ({3.4929*\dx},{1.0000*\dy})
	-- ({3.5029*\dx},{1.0000*\dy})
	-- ({3.5129*\dx},{1.0000*\dy})
	-- ({3.5229*\dx},{1.0000*\dy})
	-- ({3.5329*\dx},{1.0000*\dy})
	-- ({3.5430*\dx},{1.0000*\dy})
	-- ({3.5530*\dx},{1.0000*\dy})
	-- ({3.5630*\dx},{1.0000*\dy})
	-- ({3.5730*\dx},{1.0000*\dy})
	-- ({3.5830*\dx},{1.0000*\dy})
	-- ({3.5930*\dx},{1.0000*\dy})
	-- ({3.6030*\dx},{1.0000*\dy})
	-- ({3.6130*\dx},{1.0000*\dy})
	-- ({3.6230*\dx},{1.0000*\dy})
	-- ({3.6330*\dx},{1.0000*\dy})
	-- ({3.6430*\dx},{1.0000*\dy})
	-- ({3.6530*\dx},{1.0000*\dy})
	-- ({3.6631*\dx},{1.0000*\dy})
	-- ({3.6731*\dx},{1.0000*\dy})
	-- ({3.6831*\dx},{1.0000*\dy})
	-- ({3.6931*\dx},{1.0000*\dy})
	-- ({3.7031*\dx},{1.0000*\dy})
	-- ({3.7131*\dx},{1.0000*\dy})
	-- ({3.7231*\dx},{1.0000*\dy})
	-- ({3.7331*\dx},{1.0000*\dy})
	-- ({3.7431*\dx},{1.0000*\dy})
	-- ({3.7531*\dx},{1.0000*\dy})
	-- ({3.7631*\dx},{1.0000*\dy})
	-- ({3.7731*\dx},{1.0000*\dy})
	-- ({3.7832*\dx},{1.0000*\dy})
	-- ({3.7932*\dx},{1.0000*\dy})
	-- ({3.8032*\dx},{1.0000*\dy})
	-- ({3.8132*\dx},{1.0000*\dy})
	-- ({3.8232*\dx},{1.0000*\dy})
	-- ({3.8332*\dx},{1.0000*\dy})
	-- ({3.8432*\dx},{1.0000*\dy})
	-- ({3.8532*\dx},{1.0000*\dy})
	-- ({3.8632*\dx},{1.0000*\dy})
	-- ({3.8732*\dx},{1.0000*\dy})
	-- ({3.8832*\dx},{1.0000*\dy})
	-- ({3.8932*\dx},{1.0000*\dy})
	-- ({3.9033*\dx},{1.0000*\dy})
	-- ({3.9133*\dx},{1.0000*\dy})
	-- ({3.9233*\dx},{1.0000*\dy})
	-- ({3.9333*\dx},{1.0000*\dy})
	-- ({3.9433*\dx},{1.0000*\dy})
	-- ({3.9533*\dx},{1.0000*\dy})
	-- ({3.9633*\dx},{1.0000*\dy})
	-- ({3.9733*\dx},{1.0000*\dy})
	-- ({3.9833*\dx},{1.0000*\dy})
	-- ({3.9933*\dx},{1.0000*\dy})
	-- ({4.0033*\dx},{1.0000*\dy})
	-- ({4.0133*\dx},{1.0000*\dy})
	-- ({4.0234*\dx},{1.0000*\dy})
	-- ({4.0334*\dx},{1.0000*\dy})
	-- ({4.0434*\dx},{1.0000*\dy})
	-- ({4.0534*\dx},{1.0000*\dy})
	-- ({4.0634*\dx},{1.0000*\dy})
	-- ({4.0734*\dx},{1.0000*\dy})
	-- ({4.0834*\dx},{1.0000*\dy})
	-- ({4.0934*\dx},{1.0000*\dy})
	-- ({4.1034*\dx},{1.0000*\dy})
	-- ({4.1134*\dx},{1.0000*\dy})
	-- ({4.1234*\dx},{1.0000*\dy})
	-- ({4.1334*\dx},{1.0000*\dy})
	-- ({4.1435*\dx},{1.0000*\dy})
	-- ({4.1535*\dx},{1.0000*\dy})
	-- ({4.1635*\dx},{1.0000*\dy})
	-- ({4.1735*\dx},{1.0000*\dy})
	-- ({4.1835*\dx},{1.0000*\dy})
	-- ({4.1935*\dx},{1.0000*\dy})
	-- ({4.2035*\dx},{1.0000*\dy})
	-- ({4.2135*\dx},{1.0000*\dy})
	-- ({4.2235*\dx},{1.0000*\dy})
	-- ({4.2335*\dx},{1.0000*\dy})
	-- ({4.2435*\dx},{1.0000*\dy})
	-- ({4.2535*\dx},{1.0000*\dy})
	-- ({4.2636*\dx},{1.0000*\dy})
	-- ({4.2736*\dx},{1.0000*\dy})
	-- ({4.2836*\dx},{1.0000*\dy})
	-- ({4.2936*\dx},{1.0000*\dy})
	-- ({4.3036*\dx},{1.0000*\dy})
	-- ({4.3136*\dx},{1.0000*\dy})
	-- ({4.3236*\dx},{1.0000*\dy})
	-- ({4.3336*\dx},{1.0000*\dy})
	-- ({4.3436*\dx},{1.0000*\dy})
	-- ({4.3536*\dx},{1.0000*\dy})
	-- ({4.3636*\dx},{1.0000*\dy})
	-- ({4.3736*\dx},{1.0000*\dy})
	-- ({4.3837*\dx},{1.0000*\dy})
	-- ({4.3937*\dx},{1.0000*\dy})
	-- ({4.4037*\dx},{1.0000*\dy})
	-- ({4.4137*\dx},{1.0000*\dy})
	-- ({4.4237*\dx},{1.0000*\dy})
	-- ({4.4337*\dx},{1.0000*\dy})
	-- ({4.4437*\dx},{1.0000*\dy})
	-- ({4.4537*\dx},{1.0000*\dy})
	-- ({4.4637*\dx},{1.0000*\dy})
	-- ({4.4737*\dx},{1.0000*\dy})
	-- ({4.4837*\dx},{1.0000*\dy})
	-- ({4.4937*\dx},{1.0000*\dy})
	-- ({4.5038*\dx},{1.0000*\dy})
	-- ({4.5138*\dx},{1.0000*\dy})
	-- ({4.5238*\dx},{1.0000*\dy})
	-- ({4.5338*\dx},{1.0000*\dy})
	-- ({4.5438*\dx},{1.0000*\dy})
	-- ({4.5538*\dx},{1.0000*\dy})
	-- ({4.5638*\dx},{1.0000*\dy})
	-- ({4.5738*\dx},{1.0000*\dy})
	-- ({4.5838*\dx},{1.0000*\dy})
	-- ({4.5938*\dx},{1.0000*\dy})
	-- ({4.6038*\dx},{1.0000*\dy})
	-- ({4.6138*\dx},{1.0000*\dy})
	-- ({4.6239*\dx},{1.0000*\dy})
	-- ({4.6339*\dx},{1.0000*\dy})
	-- ({4.6439*\dx},{1.0000*\dy})
	-- ({4.6539*\dx},{1.0000*\dy})
	-- ({4.6639*\dx},{1.0000*\dy})
	-- ({4.6739*\dx},{1.0000*\dy})
	-- ({4.6839*\dx},{1.0000*\dy})
	-- ({4.6939*\dx},{1.0000*\dy})
	-- ({4.7039*\dx},{1.0000*\dy})
	-- ({4.7139*\dx},{1.0000*\dy})
	-- ({4.7239*\dx},{1.0000*\dy})
	-- ({4.7339*\dx},{1.0000*\dy})
	-- ({4.7440*\dx},{1.0000*\dy})
	-- ({4.7540*\dx},{1.0000*\dy})
	-- ({4.7640*\dx},{1.0000*\dy})
	-- ({4.7740*\dx},{1.0000*\dy})
	-- ({4.7840*\dx},{1.0000*\dy})
	-- ({4.7940*\dx},{1.0000*\dy})
	-- ({4.8040*\dx},{1.0000*\dy})
	-- ({4.8140*\dx},{1.0000*\dy})
	-- ({4.8240*\dx},{1.0000*\dy})
	-- ({4.8340*\dx},{1.0000*\dy})
	-- ({4.8440*\dx},{1.0000*\dy})
	-- ({4.8540*\dx},{1.0000*\dy})
	-- ({4.8641*\dx},{1.0000*\dy})
	-- ({4.8741*\dx},{1.0000*\dy})
	-- ({4.8841*\dx},{1.0000*\dy})
	-- ({4.8941*\dx},{1.0000*\dy})
	-- ({4.9041*\dx},{1.0000*\dy})
	-- ({4.9141*\dx},{1.0000*\dy})
	-- ({4.9241*\dx},{1.0000*\dy})
	-- ({4.9341*\dx},{1.0000*\dy})
	-- ({4.9441*\dx},{1.0000*\dy})
	-- ({4.9541*\dx},{1.0000*\dy})
	-- ({4.9641*\dx},{1.0000*\dy})
	-- ({4.9741*\dx},{1.0000*\dy})
	-- ({4.9842*\dx},{1.0000*\dy})
	-- ({4.9942*\dx},{1.0000*\dy})
	-- ({5.0042*\dx},{1.0000*\dy})
	-- ({5.0142*\dx},{0.9999*\dy})
	-- ({5.0242*\dx},{0.9998*\dy})
	-- ({5.0342*\dx},{0.9996*\dy})
	-- ({5.0442*\dx},{0.9994*\dy})
	-- ({5.0542*\dx},{0.9991*\dy})
	-- ({5.0642*\dx},{0.9987*\dy})
	-- ({5.0742*\dx},{0.9982*\dy})
	-- ({5.0842*\dx},{0.9977*\dy})
	-- ({5.0942*\dx},{0.9971*\dy})
	-- ({5.1043*\dx},{0.9965*\dy})
	-- ({5.1143*\dx},{0.9958*\dy})
	-- ({5.1243*\dx},{0.9950*\dy})
	-- ({5.1343*\dx},{0.9941*\dy})
	-- ({5.1443*\dx},{0.9932*\dy})
	-- ({5.1543*\dx},{0.9922*\dy})
	-- ({5.1643*\dx},{0.9912*\dy})
	-- ({5.1743*\dx},{0.9901*\dy})
	-- ({5.1843*\dx},{0.9889*\dy})
	-- ({5.1943*\dx},{0.9876*\dy})
	-- ({5.2043*\dx},{0.9863*\dy})
	-- ({5.2143*\dx},{0.9849*\dy})
	-- ({5.2244*\dx},{0.9834*\dy})
	-- ({5.2344*\dx},{0.9819*\dy})
	-- ({5.2444*\dx},{0.9803*\dy})
	-- ({5.2544*\dx},{0.9786*\dy})
	-- ({5.2644*\dx},{0.9769*\dy})
	-- ({5.2744*\dx},{0.9751*\dy})
	-- ({5.2844*\dx},{0.9733*\dy})
	-- ({5.2944*\dx},{0.9713*\dy})
	-- ({5.3044*\dx},{0.9694*\dy})
	-- ({5.3144*\dx},{0.9673*\dy})
	-- ({5.3244*\dx},{0.9652*\dy})
	-- ({5.3344*\dx},{0.9630*\dy})
	-- ({5.3445*\dx},{0.9608*\dy})
	-- ({5.3545*\dx},{0.9585*\dy})
	-- ({5.3645*\dx},{0.9561*\dy})
	-- ({5.3745*\dx},{0.9537*\dy})
	-- ({5.3845*\dx},{0.9512*\dy})
	-- ({5.3945*\dx},{0.9487*\dy})
	-- ({5.4045*\dx},{0.9461*\dy})
	-- ({5.4145*\dx},{0.9434*\dy})
	-- ({5.4245*\dx},{0.9407*\dy})
	-- ({5.4345*\dx},{0.9380*\dy})
	-- ({5.4445*\dx},{0.9352*\dy})
	-- ({5.4545*\dx},{0.9323*\dy})
	-- ({5.4646*\dx},{0.9294*\dy})
	-- ({5.4746*\dx},{0.9264*\dy})
	-- ({5.4846*\dx},{0.9234*\dy})
	-- ({5.4946*\dx},{0.9204*\dy})
	-- ({5.5046*\dx},{0.9173*\dy})
	-- ({5.5146*\dx},{0.9142*\dy})
	-- ({5.5246*\dx},{0.9110*\dy})
	-- ({5.5346*\dx},{0.9078*\dy})
	-- ({5.5446*\dx},{0.9045*\dy})
	-- ({5.5546*\dx},{0.9012*\dy})
	-- ({5.5646*\dx},{0.8979*\dy})
	-- ({5.5746*\dx},{0.8946*\dy})
	-- ({5.5847*\dx},{0.8912*\dy})
	-- ({5.5947*\dx},{0.8877*\dy})
	-- ({5.6047*\dx},{0.8843*\dy})
	-- ({5.6147*\dx},{0.8808*\dy})
	-- ({5.6247*\dx},{0.8773*\dy})
	-- ({5.6347*\dx},{0.8738*\dy})
	-- ({5.6447*\dx},{0.8702*\dy})
	-- ({5.6547*\dx},{0.8667*\dy})
	-- ({5.6647*\dx},{0.8631*\dy})
	-- ({5.6747*\dx},{0.8595*\dy})
	-- ({5.6847*\dx},{0.8559*\dy})
	-- ({5.6947*\dx},{0.8522*\dy})
	-- ({5.7048*\dx},{0.8486*\dy})
	-- ({5.7148*\dx},{0.8450*\dy})
	-- ({5.7248*\dx},{0.8413*\dy})
	-- ({5.7348*\dx},{0.8376*\dy})
	-- ({5.7448*\dx},{0.8340*\dy})
	-- ({5.7548*\dx},{0.8303*\dy})
	-- ({5.7648*\dx},{0.8266*\dy})
	-- ({5.7748*\dx},{0.8229*\dy})
	-- ({5.7848*\dx},{0.8192*\dy})
	-- ({5.7948*\dx},{0.8156*\dy})
	-- ({5.8048*\dx},{0.8119*\dy})
	-- ({5.8148*\dx},{0.8082*\dy})
	-- ({5.8249*\dx},{0.8046*\dy})
	-- ({5.8349*\dx},{0.8009*\dy})
	-- ({5.8449*\dx},{0.7973*\dy})
	-- ({5.8549*\dx},{0.7937*\dy})
	-- ({5.8649*\dx},{0.7901*\dy})
	-- ({5.8749*\dx},{0.7865*\dy})
	-- ({5.8849*\dx},{0.7829*\dy})
	-- ({5.8949*\dx},{0.7793*\dy})
	-- ({5.9049*\dx},{0.7758*\dy})
	-- ({5.9149*\dx},{0.7723*\dy})
	-- ({5.9249*\dx},{0.7688*\dy})
	-- ({5.9349*\dx},{0.7653*\dy})
	-- ({5.9450*\dx},{0.7618*\dy})
	-- ({5.9550*\dx},{0.7584*\dy})
	-- ({5.9650*\dx},{0.7550*\dy})
	-- ({5.9750*\dx},{0.7516*\dy})
	-- ({5.9850*\dx},{0.7482*\dy})
	-- ({5.9950*\dx},{0.7449*\dy})
	-- ({6.0050*\dx},{0.7416*\dy})
	-- ({6.0150*\dx},{0.7383*\dy})
	-- ({6.0250*\dx},{0.7350*\dy})
	-- ({6.0350*\dx},{0.7317*\dy})
	-- ({6.0450*\dx},{0.7284*\dy})
	-- ({6.0550*\dx},{0.7251*\dy})
	-- ({6.0651*\dx},{0.7219*\dy})
	-- ({6.0751*\dx},{0.7186*\dy})
	-- ({6.0851*\dx},{0.7154*\dy})
	-- ({6.0951*\dx},{0.7121*\dy})
	-- ({6.1051*\dx},{0.7089*\dy})
	-- ({6.1151*\dx},{0.7056*\dy})
	-- ({6.1251*\dx},{0.7024*\dy})
	-- ({6.1351*\dx},{0.6992*\dy})
	-- ({6.1451*\dx},{0.6960*\dy})
	-- ({6.1551*\dx},{0.6928*\dy})
	-- ({6.1651*\dx},{0.6896*\dy})
	-- ({6.1751*\dx},{0.6864*\dy})
	-- ({6.1852*\dx},{0.6832*\dy})
	-- ({6.1952*\dx},{0.6801*\dy})
	-- ({6.2052*\dx},{0.6769*\dy})
	-- ({6.2152*\dx},{0.6737*\dy})
	-- ({6.2252*\dx},{0.6706*\dy})
	-- ({6.2352*\dx},{0.6675*\dy})
	-- ({6.2452*\dx},{0.6643*\dy})
	-- ({6.2552*\dx},{0.6612*\dy})
	-- ({6.2652*\dx},{0.6581*\dy})
	-- ({6.2752*\dx},{0.6550*\dy})
	-- ({6.2852*\dx},{0.6519*\dy})
	-- ({6.2952*\dx},{0.6488*\dy})
	-- ({6.3053*\dx},{0.6458*\dy})
	-- ({6.3153*\dx},{0.6427*\dy})
	-- ({6.3253*\dx},{0.6396*\dy})
	-- ({6.3353*\dx},{0.6366*\dy})
	-- ({6.3453*\dx},{0.6335*\dy})
	-- ({6.3553*\dx},{0.6305*\dy})
	-- ({6.3653*\dx},{0.6275*\dy})
	-- ({6.3753*\dx},{0.6245*\dy})
	-- ({6.3853*\dx},{0.6215*\dy})
	-- ({6.3953*\dx},{0.6185*\dy})
	-- ({6.4053*\dx},{0.6155*\dy})
	-- ({6.4153*\dx},{0.6125*\dy})
	-- ({6.4254*\dx},{0.6096*\dy})
	-- ({6.4354*\dx},{0.6066*\dy})
	-- ({6.4454*\dx},{0.6037*\dy})
	-- ({6.4554*\dx},{0.6007*\dy})
	-- ({6.4654*\dx},{0.5978*\dy})
	-- ({6.4754*\dx},{0.5949*\dy})
	-- ({6.4854*\dx},{0.5919*\dy})
	-- ({6.4954*\dx},{0.5890*\dy})
	-- ({6.5054*\dx},{0.5862*\dy})
	-- ({6.5154*\dx},{0.5833*\dy})
	-- ({6.5254*\dx},{0.5804*\dy})
	-- ({6.5354*\dx},{0.5775*\dy})
	-- ({6.5455*\dx},{0.5747*\dy})
	-- ({6.5555*\dx},{0.5718*\dy})
	-- ({6.5655*\dx},{0.5690*\dy})
	-- ({6.5755*\dx},{0.5661*\dy})
	-- ({6.5855*\dx},{0.5633*\dy})
	-- ({6.5955*\dx},{0.5605*\dy})
	-- ({6.6055*\dx},{0.5577*\dy})
	-- ({6.6155*\dx},{0.5549*\dy})
	-- ({6.6255*\dx},{0.5521*\dy})
	-- ({6.6355*\dx},{0.5493*\dy})
	-- ({6.6455*\dx},{0.5466*\dy})
	-- ({6.6555*\dx},{0.5438*\dy})
	-- ({6.6656*\dx},{0.5410*\dy})
	-- ({6.6756*\dx},{0.5383*\dy})
	-- ({6.6856*\dx},{0.5355*\dy})
	-- ({6.6956*\dx},{0.5328*\dy})
	-- ({6.7056*\dx},{0.5301*\dy})
	-- ({6.7156*\dx},{0.5274*\dy})
	-- ({6.7256*\dx},{0.5247*\dy})
	-- ({6.7356*\dx},{0.5220*\dy})
	-- ({6.7456*\dx},{0.5193*\dy})
	-- ({6.7556*\dx},{0.5166*\dy})
	-- ({6.7656*\dx},{0.5139*\dy})
	-- ({6.7756*\dx},{0.5113*\dy})
	-- ({6.7857*\dx},{0.5086*\dy})
	-- ({6.7957*\dx},{0.5059*\dy})
	-- ({6.8057*\dx},{0.5033*\dy})
	-- ({6.8157*\dx},{0.5007*\dy})
	-- ({6.8257*\dx},{0.4980*\dy})
	-- ({6.8357*\dx},{0.4954*\dy})
	-- ({6.8457*\dx},{0.4928*\dy})
	-- ({6.8557*\dx},{0.4902*\dy})
	-- ({6.8657*\dx},{0.4876*\dy})
	-- ({6.8757*\dx},{0.4850*\dy})
	-- ({6.8857*\dx},{0.4824*\dy})
	-- ({6.8957*\dx},{0.4798*\dy})
	-- ({6.9058*\dx},{0.4772*\dy})
	-- ({6.9158*\dx},{0.4747*\dy})
	-- ({6.9258*\dx},{0.4721*\dy})
	-- ({6.9358*\dx},{0.4696*\dy})
	-- ({6.9458*\dx},{0.4670*\dy})
	-- ({6.9558*\dx},{0.4645*\dy})
	-- ({6.9658*\dx},{0.4619*\dy})
	-- ({6.9758*\dx},{0.4594*\dy})
	-- ({6.9858*\dx},{0.4569*\dy})
	-- ({6.9958*\dx},{0.4544*\dy})
	-- ({7.0058*\dx},{0.4518*\dy})
	-- ({7.0158*\dx},{0.4493*\dy})
	-- ({7.0259*\dx},{0.4468*\dy})
	-- ({7.0359*\dx},{0.4443*\dy})
	-- ({7.0459*\dx},{0.4419*\dy})
	-- ({7.0559*\dx},{0.4394*\dy})
	-- ({7.0659*\dx},{0.4369*\dy})
	-- ({7.0759*\dx},{0.4344*\dy})
	-- ({7.0859*\dx},{0.4320*\dy})
	-- ({7.0959*\dx},{0.4295*\dy})
	-- ({7.1059*\dx},{0.4270*\dy})
	-- ({7.1159*\dx},{0.4246*\dy})
	-- ({7.1259*\dx},{0.4221*\dy})
	-- ({7.1359*\dx},{0.4197*\dy})
	-- ({7.1460*\dx},{0.4173*\dy})
	-- ({7.1560*\dx},{0.4148*\dy})
	-- ({7.1660*\dx},{0.4124*\dy})
	-- ({7.1760*\dx},{0.4100*\dy})
	-- ({7.1860*\dx},{0.4075*\dy})
	-- ({7.1960*\dx},{0.4051*\dy})
	-- ({7.2060*\dx},{0.4027*\dy})
	-- ({7.2160*\dx},{0.4003*\dy})
	-- ({7.2260*\dx},{0.3979*\dy})
	-- ({7.2360*\dx},{0.3955*\dy})
	-- ({7.2460*\dx},{0.3931*\dy})
	-- ({7.2560*\dx},{0.3907*\dy})
	-- ({7.2661*\dx},{0.3883*\dy})
	-- ({7.2761*\dx},{0.3860*\dy})
	-- ({7.2861*\dx},{0.3836*\dy})
	-- ({7.2961*\dx},{0.3812*\dy})
	-- ({7.3061*\dx},{0.3788*\dy})
	-- ({7.3161*\dx},{0.3765*\dy})
	-- ({7.3261*\dx},{0.3741*\dy})
	-- ({7.3361*\dx},{0.3717*\dy})
	-- ({7.3461*\dx},{0.3694*\dy})
	-- ({7.3561*\dx},{0.3670*\dy})
	-- ({7.3661*\dx},{0.3646*\dy})
	-- ({7.3761*\dx},{0.3623*\dy})
	-- ({7.3862*\dx},{0.3599*\dy})
	-- ({7.3962*\dx},{0.3576*\dy})
	-- ({7.4062*\dx},{0.3552*\dy})
	-- ({7.4162*\dx},{0.3529*\dy})
	-- ({7.4262*\dx},{0.3506*\dy})
	-- ({7.4362*\dx},{0.3482*\dy})
	-- ({7.4462*\dx},{0.3459*\dy})
	-- ({7.4562*\dx},{0.3435*\dy})
	-- ({7.4662*\dx},{0.3412*\dy})
	-- ({7.4762*\dx},{0.3389*\dy})
	-- ({7.4862*\dx},{0.3365*\dy})
	-- ({7.4962*\dx},{0.3342*\dy})
	-- ({7.5063*\dx},{0.3319*\dy})
	-- ({7.5163*\dx},{0.3296*\dy})
	-- ({7.5263*\dx},{0.3272*\dy})
	-- ({7.5363*\dx},{0.3249*\dy})
	-- ({7.5463*\dx},{0.3226*\dy})
	-- ({7.5563*\dx},{0.3203*\dy})
	-- ({7.5663*\dx},{0.3179*\dy})
	-- ({7.5763*\dx},{0.3156*\dy})
	-- ({7.5863*\dx},{0.3133*\dy})
	-- ({7.5963*\dx},{0.3110*\dy})
	-- ({7.6063*\dx},{0.3086*\dy})
	-- ({7.6163*\dx},{0.3063*\dy})
	-- ({7.6264*\dx},{0.3040*\dy})
	-- ({7.6364*\dx},{0.3017*\dy})
	-- ({7.6464*\dx},{0.2994*\dy})
	-- ({7.6564*\dx},{0.2970*\dy})
	-- ({7.6664*\dx},{0.2947*\dy})
	-- ({7.6764*\dx},{0.2924*\dy})
	-- ({7.6864*\dx},{0.2901*\dy})
	-- ({7.6964*\dx},{0.2878*\dy})
	-- ({7.7064*\dx},{0.2854*\dy})
	-- ({7.7164*\dx},{0.2831*\dy})
	-- ({7.7264*\dx},{0.2808*\dy})
	-- ({7.7364*\dx},{0.2785*\dy})
	-- ({7.7465*\dx},{0.2761*\dy})
	-- ({7.7565*\dx},{0.2738*\dy})
	-- ({7.7665*\dx},{0.2715*\dy})
	-- ({7.7765*\dx},{0.2691*\dy})
	-- ({7.7865*\dx},{0.2668*\dy})
	-- ({7.7965*\dx},{0.2645*\dy})
	-- ({7.8065*\dx},{0.2621*\dy})
	-- ({7.8165*\dx},{0.2598*\dy})
	-- ({7.8265*\dx},{0.2575*\dy})
	-- ({7.8365*\dx},{0.2551*\dy})
	-- ({7.8465*\dx},{0.2528*\dy})
	-- ({7.8565*\dx},{0.2504*\dy})
	-- ({7.8666*\dx},{0.2481*\dy})
	-- ({7.8766*\dx},{0.2457*\dy})
	-- ({7.8866*\dx},{0.2434*\dy})
	-- ({7.8966*\dx},{0.2410*\dy})
	-- ({7.9066*\dx},{0.2387*\dy})
	-- ({7.9166*\dx},{0.2363*\dy})
	-- ({7.9266*\dx},{0.2340*\dy})
	-- ({7.9366*\dx},{0.2316*\dy})
	-- ({7.9466*\dx},{0.2292*\dy})
	-- ({7.9566*\dx},{0.2269*\dy})
	-- ({7.9666*\dx},{0.2245*\dy})
	-- ({7.9766*\dx},{0.2221*\dy})
	-- ({7.9867*\dx},{0.2197*\dy})
	-- ({7.9967*\dx},{0.2173*\dy})
	-- ({8.0067*\dx},{0.2149*\dy})
	-- ({8.0167*\dx},{0.2125*\dy})
	-- ({8.0267*\dx},{0.2100*\dy})
	-- ({8.0367*\dx},{0.2075*\dy})
	-- ({8.0467*\dx},{0.2049*\dy})
	-- ({8.0567*\dx},{0.2023*\dy})
	-- ({8.0667*\dx},{0.1997*\dy})
	-- ({8.0767*\dx},{0.1970*\dy})
	-- ({8.0867*\dx},{0.1942*\dy})
	-- ({8.0967*\dx},{0.1915*\dy})
	-- ({8.1068*\dx},{0.1887*\dy})
	-- ({8.1168*\dx},{0.1858*\dy})
	-- ({8.1268*\dx},{0.1830*\dy})
	-- ({8.1368*\dx},{0.1801*\dy})
	-- ({8.1468*\dx},{0.1771*\dy})
	-- ({8.1568*\dx},{0.1741*\dy})
	-- ({8.1668*\dx},{0.1711*\dy})
	-- ({8.1768*\dx},{0.1681*\dy})
	-- ({8.1868*\dx},{0.1651*\dy})
	-- ({8.1968*\dx},{0.1620*\dy})
	-- ({8.2068*\dx},{0.1589*\dy})
	-- ({8.2168*\dx},{0.1558*\dy})
	-- ({8.2269*\dx},{0.1527*\dy})
	-- ({8.2369*\dx},{0.1495*\dy})
	-- ({8.2469*\dx},{0.1464*\dy})
	-- ({8.2569*\dx},{0.1432*\dy})
	-- ({8.2669*\dx},{0.1400*\dy})
	-- ({8.2769*\dx},{0.1368*\dy})
	-- ({8.2869*\dx},{0.1337*\dy})
	-- ({8.2969*\dx},{0.1305*\dy})
	-- ({8.3069*\dx},{0.1273*\dy})
	-- ({8.3169*\dx},{0.1241*\dy})
	-- ({8.3269*\dx},{0.1209*\dy})
	-- ({8.3369*\dx},{0.1177*\dy})
	-- ({8.3470*\dx},{0.1146*\dy})
	-- ({8.3570*\dx},{0.1114*\dy})
	-- ({8.3670*\dx},{0.1083*\dy})
	-- ({8.3770*\dx},{0.1052*\dy})
	-- ({8.3870*\dx},{0.1021*\dy})
	-- ({8.3970*\dx},{0.0990*\dy})
	-- ({8.4070*\dx},{0.0959*\dy})
	-- ({8.4170*\dx},{0.0929*\dy})
	-- ({8.4270*\dx},{0.0899*\dy})
	-- ({8.4370*\dx},{0.0869*\dy})
	-- ({8.4470*\dx},{0.0840*\dy})
	-- ({8.4570*\dx},{0.0810*\dy})
	-- ({8.4671*\dx},{0.0782*\dy})
	-- ({8.4771*\dx},{0.0753*\dy})
	-- ({8.4871*\dx},{0.0725*\dy})
	-- ({8.4971*\dx},{0.0698*\dy})
	-- ({8.5071*\dx},{0.0670*\dy})
	-- ({8.5171*\dx},{0.0644*\dy})
	-- ({8.5271*\dx},{0.0617*\dy})
	-- ({8.5371*\dx},{0.0591*\dy})
	-- ({8.5471*\dx},{0.0566*\dy})
	-- ({8.5571*\dx},{0.0541*\dy})
	-- ({8.5671*\dx},{0.0517*\dy})
	-- ({8.5771*\dx},{0.0493*\dy})
	-- ({8.5872*\dx},{0.0469*\dy})
	-- ({8.5972*\dx},{0.0446*\dy})
	-- ({8.6072*\dx},{0.0424*\dy})
	-- ({8.6172*\dx},{0.0402*\dy})
	-- ({8.6272*\dx},{0.0381*\dy})
	-- ({8.6372*\dx},{0.0360*\dy})
	-- ({8.6472*\dx},{0.0340*\dy})
	-- ({8.6572*\dx},{0.0321*\dy})
	-- ({8.6672*\dx},{0.0302*\dy})
	-- ({8.6772*\dx},{0.0283*\dy})
	-- ({8.6872*\dx},{0.0266*\dy})
	-- ({8.6972*\dx},{0.0248*\dy})
	-- ({8.7073*\dx},{0.0232*\dy})
	-- ({8.7173*\dx},{0.0216*\dy})
	-- ({8.7273*\dx},{0.0200*\dy})
	-- ({8.7373*\dx},{0.0186*\dy})
	-- ({8.7473*\dx},{0.0171*\dy})
	-- ({8.7573*\dx},{0.0158*\dy})
	-- ({8.7673*\dx},{0.0145*\dy})
	-- ({8.7773*\dx},{0.0132*\dy})
	-- ({8.7873*\dx},{0.0120*\dy})
	-- ({8.7973*\dx},{0.0109*\dy})
	-- ({8.8073*\dx},{0.0098*\dy})
	-- ({8.8173*\dx},{0.0088*\dy})
	-- ({8.8274*\dx},{0.0078*\dy})
	-- ({8.8374*\dx},{0.0069*\dy})
	-- ({8.8474*\dx},{0.0061*\dy})
	-- ({8.8574*\dx},{0.0053*\dy})
	-- ({8.8674*\dx},{0.0046*\dy})
	-- ({8.8774*\dx},{0.0039*\dy})
	-- ({8.8874*\dx},{0.0033*\dy})
	-- ({8.8974*\dx},{0.0027*\dy})
	-- ({8.9074*\dx},{0.0022*\dy})
	-- ({8.9174*\dx},{0.0017*\dy})
	-- ({8.9274*\dx},{0.0013*\dy})
	-- ({8.9374*\dx},{0.0010*\dy})
	-- ({8.9475*\dx},{0.0007*\dy})
	-- ({8.9575*\dx},{0.0005*\dy})
	-- ({8.9675*\dx},{0.0003*\dy})
	-- ({8.9775*\dx},{0.0001*\dy})
	-- ({8.9875*\dx},{0.0000*\dy})
	-- ({8.9975*\dx},{0.0000*\dy})
	-- ({9.0075*\dx},{0.0000*\dy})
	-- ({9.0175*\dx},{0.0001*\dy})
	-- ({9.0275*\dx},{0.0002*\dy})
	-- ({9.0375*\dx},{0.0003*\dy})
	-- ({9.0475*\dx},{0.0006*\dy})
	-- ({9.0575*\dx},{0.0008*\dy})
	-- ({9.0676*\dx},{0.0011*\dy})
	-- ({9.0776*\dx},{0.0015*\dy})
	-- ({9.0876*\dx},{0.0019*\dy})
	-- ({9.0976*\dx},{0.0023*\dy})
	-- ({9.1076*\dx},{0.0028*\dy})
	-- ({9.1176*\dx},{0.0034*\dy})
	-- ({9.1276*\dx},{0.0039*\dy})
	-- ({9.1376*\dx},{0.0046*\dy})
	-- ({9.1476*\dx},{0.0052*\dy})
	-- ({9.1576*\dx},{0.0060*\dy})
	-- ({9.1676*\dx},{0.0067*\dy})
	-- ({9.1776*\dx},{0.0075*\dy})
	-- ({9.1877*\dx},{0.0084*\dy})
	-- ({9.1977*\dx},{0.0093*\dy})
	-- ({9.2077*\dx},{0.0103*\dy})
	-- ({9.2177*\dx},{0.0112*\dy})
	-- ({9.2277*\dx},{0.0123*\dy})
	-- ({9.2377*\dx},{0.0134*\dy})
	-- ({9.2477*\dx},{0.0145*\dy})
	-- ({9.2577*\dx},{0.0156*\dy})
	-- ({9.2677*\dx},{0.0168*\dy})
	-- ({9.2777*\dx},{0.0181*\dy})
	-- ({9.2877*\dx},{0.0194*\dy})
	-- ({9.2977*\dx},{0.0207*\dy})
	-- ({9.3078*\dx},{0.0221*\dy})
	-- ({9.3178*\dx},{0.0236*\dy})
	-- ({9.3278*\dx},{0.0250*\dy})
	-- ({9.3378*\dx},{0.0266*\dy})
	-- ({9.3478*\dx},{0.0281*\dy})
	-- ({9.3578*\dx},{0.0297*\dy})
	-- ({9.3678*\dx},{0.0314*\dy})
	-- ({9.3778*\dx},{0.0331*\dy})
	-- ({9.3878*\dx},{0.0348*\dy})
	-- ({9.3978*\dx},{0.0366*\dy})
	-- ({9.4078*\dx},{0.0385*\dy})
	-- ({9.4178*\dx},{0.0403*\dy})
	-- ({9.4279*\dx},{0.0423*\dy})
	-- ({9.4379*\dx},{0.0443*\dy})
	-- ({9.4479*\dx},{0.0463*\dy})
	-- ({9.4579*\dx},{0.0484*\dy})
	-- ({9.4679*\dx},{0.0505*\dy})
	-- ({9.4779*\dx},{0.0527*\dy})
	-- ({9.4879*\dx},{0.0549*\dy})
	-- ({9.4979*\dx},{0.0571*\dy})
	-- ({9.5079*\dx},{0.0595*\dy})
	-- ({9.5179*\dx},{0.0618*\dy})
	-- ({9.5279*\dx},{0.0643*\dy})
	-- ({9.5379*\dx},{0.0667*\dy})
	-- ({9.5480*\dx},{0.0693*\dy})
	-- ({9.5580*\dx},{0.0719*\dy})
	-- ({9.5680*\dx},{0.0745*\dy})
	-- ({9.5780*\dx},{0.0772*\dy})
	-- ({9.5880*\dx},{0.0799*\dy})
	-- ({9.5980*\dx},{0.0827*\dy})
	-- ({9.6080*\dx},{0.0856*\dy})
	-- ({9.6180*\dx},{0.0885*\dy})
	-- ({9.6280*\dx},{0.0915*\dy})
	-- ({9.6380*\dx},{0.0945*\dy})
	-- ({9.6480*\dx},{0.0976*\dy})
	-- ({9.6580*\dx},{0.1008*\dy})
	-- ({9.6681*\dx},{0.1040*\dy})
	-- ({9.6781*\dx},{0.1072*\dy})
	-- ({9.6881*\dx},{0.1106*\dy})
	-- ({9.6981*\dx},{0.1140*\dy})
	-- ({9.7081*\dx},{0.1174*\dy})
	-- ({9.7181*\dx},{0.1210*\dy})
	-- ({9.7281*\dx},{0.1246*\dy})
	-- ({9.7381*\dx},{0.1282*\dy})
	-- ({9.7481*\dx},{0.1320*\dy})
	-- ({9.7581*\dx},{0.1357*\dy})
	-- ({9.7681*\dx},{0.1396*\dy})
	-- ({9.7781*\dx},{0.1435*\dy})
	-- ({9.7882*\dx},{0.1476*\dy})
	-- ({9.7982*\dx},{0.1516*\dy})
	-- ({9.8082*\dx},{0.1558*\dy})
	-- ({9.8182*\dx},{0.1600*\dy})
	-- ({9.8282*\dx},{0.1643*\dy})
	-- ({9.8382*\dx},{0.1687*\dy})
	-- ({9.8482*\dx},{0.1731*\dy})
	-- ({9.8582*\dx},{0.1776*\dy})
	-- ({9.8682*\dx},{0.1822*\dy})
	-- ({9.8782*\dx},{0.1869*\dy})
	-- ({9.8882*\dx},{0.1916*\dy})
	-- ({9.8982*\dx},{0.1965*\dy})
	-- ({9.9083*\dx},{0.2014*\dy})
	-- ({9.9183*\dx},{0.2063*\dy})
	-- ({9.9283*\dx},{0.2114*\dy})
	-- ({9.9383*\dx},{0.2165*\dy})
	-- ({9.9483*\dx},{0.2218*\dy})
	-- ({9.9583*\dx},{0.2271*\dy})
	-- ({9.9683*\dx},{0.2324*\dy})
	-- ({9.9783*\dx},{0.2379*\dy})
	-- ({9.9883*\dx},{0.2434*\dy})
	-- ({9.9983*\dx},{0.2491*\dy})
	-- ({10.0083*\dx},{0.2548*\dy})
	-- ({10.0183*\dx},{0.2606*\dy})
	-- ({10.0284*\dx},{0.2665*\dy})
	-- ({10.0384*\dx},{0.2725*\dy})
	-- ({10.0484*\dx},{0.2786*\dy})
	-- ({10.0584*\dx},{0.2848*\dy})
	-- ({10.0684*\dx},{0.2912*\dy})
	-- ({10.0784*\dx},{0.2976*\dy})
	-- ({10.0884*\dx},{0.3041*\dy})
	-- ({10.0984*\dx},{0.3108*\dy})
	-- ({10.1084*\dx},{0.3176*\dy})
	-- ({10.1184*\dx},{0.3245*\dy})
	-- ({10.1284*\dx},{0.3315*\dy})
	-- ({10.1384*\dx},{0.3386*\dy})
	-- ({10.1485*\dx},{0.3458*\dy})
	-- ({10.1585*\dx},{0.3531*\dy})
	-- ({10.1685*\dx},{0.3606*\dy})
	-- ({10.1785*\dx},{0.3682*\dy})
	-- ({10.1885*\dx},{0.3759*\dy})
	-- ({10.1985*\dx},{0.3837*\dy})
	-- ({10.2085*\dx},{0.3916*\dy})
	-- ({10.2185*\dx},{0.3996*\dy})
	-- ({10.2285*\dx},{0.4078*\dy})
	-- ({10.2385*\dx},{0.4161*\dy})
	-- ({10.2485*\dx},{0.4244*\dy})
	-- ({10.2585*\dx},{0.4329*\dy})
	-- ({10.2686*\dx},{0.4416*\dy})
	-- ({10.2786*\dx},{0.4503*\dy})
	-- ({10.2886*\dx},{0.4591*\dy})
	-- ({10.2986*\dx},{0.4681*\dy})
	-- ({10.3086*\dx},{0.4772*\dy})
	-- ({10.3186*\dx},{0.4863*\dy})
	-- ({10.3286*\dx},{0.4956*\dy})
	-- ({10.3386*\dx},{0.5050*\dy})
	-- ({10.3486*\dx},{0.5144*\dy})
	-- ({10.3586*\dx},{0.5240*\dy})
	-- ({10.3686*\dx},{0.5336*\dy})
	-- ({10.3786*\dx},{0.5434*\dy})
	-- ({10.3887*\dx},{0.5532*\dy})
	-- ({10.3987*\dx},{0.5631*\dy})
	-- ({10.4087*\dx},{0.5731*\dy})
	-- ({10.4187*\dx},{0.5832*\dy})
	-- ({10.4287*\dx},{0.5933*\dy})
	-- ({10.4387*\dx},{0.6034*\dy})
	-- ({10.4487*\dx},{0.6137*\dy})
	-- ({10.4587*\dx},{0.6239*\dy})
	-- ({10.4687*\dx},{0.6343*\dy})
	-- ({10.4787*\dx},{0.6446*\dy})
	-- ({10.4887*\dx},{0.6550*\dy})
	-- ({10.4987*\dx},{0.6654*\dy})
	-- ({10.5088*\dx},{0.6758*\dy})
	-- ({10.5188*\dx},{0.6862*\dy})
	-- ({10.5288*\dx},{0.6966*\dy})
	-- ({10.5388*\dx},{0.7069*\dy})
	-- ({10.5488*\dx},{0.7173*\dy})
	-- ({10.5588*\dx},{0.7276*\dy})
	-- ({10.5688*\dx},{0.7379*\dy})
	-- ({10.5788*\dx},{0.7481*\dy})
	-- ({10.5888*\dx},{0.7582*\dy})
	-- ({10.5988*\dx},{0.7683*\dy})
	-- ({10.6088*\dx},{0.7783*\dy})
	-- ({10.6188*\dx},{0.7882*\dy})
	-- ({10.6289*\dx},{0.7980*\dy})
	-- ({10.6389*\dx},{0.8077*\dy})
	-- ({10.6489*\dx},{0.8172*\dy})
	-- ({10.6589*\dx},{0.8266*\dy})
	-- ({10.6689*\dx},{0.8358*\dy})
	-- ({10.6789*\dx},{0.8449*\dy})
	-- ({10.6889*\dx},{0.8538*\dy})
	-- ({10.6989*\dx},{0.8625*\dy})
	-- ({10.7089*\dx},{0.8711*\dy})
	-- ({10.7189*\dx},{0.8794*\dy})
	-- ({10.7289*\dx},{0.8875*\dy})
	-- ({10.7389*\dx},{0.8954*\dy})
	-- ({10.7490*\dx},{0.9030*\dy})
	-- ({10.7590*\dx},{0.9104*\dy})
	-- ({10.7690*\dx},{0.9176*\dy})
	-- ({10.7790*\dx},{0.9245*\dy})
	-- ({10.7890*\dx},{0.9311*\dy})
	-- ({10.7990*\dx},{0.9374*\dy})
	-- ({10.8090*\dx},{0.9435*\dy})
	-- ({10.8190*\dx},{0.9493*\dy})
	-- ({10.8290*\dx},{0.9547*\dy})
	-- ({10.8390*\dx},{0.9599*\dy})
	-- ({10.8490*\dx},{0.9648*\dy})
	-- ({10.8590*\dx},{0.9693*\dy})
	-- ({10.8691*\dx},{0.9736*\dy})
	-- ({10.8791*\dx},{0.9775*\dy})
	-- ({10.8891*\dx},{0.9811*\dy})
	-- ({10.8991*\dx},{0.9844*\dy})
	-- ({10.9091*\dx},{0.9874*\dy})
	-- ({10.9191*\dx},{0.9901*\dy})
	-- ({10.9291*\dx},{0.9924*\dy})
	-- ({10.9391*\dx},{0.9944*\dy})
	-- ({10.9491*\dx},{0.9961*\dy})
	-- ({10.9591*\dx},{0.9975*\dy})
	-- ({10.9691*\dx},{0.9986*\dy})
	-- ({10.9791*\dx},{0.9994*\dy})
	-- ({10.9892*\dx},{0.9998*\dy})
	-- ({10.9992*\dx},{1.0000*\dy})
	-- ({11.0092*\dx},{1.0000*\dy})
	-- ({11.0192*\dx},{1.0000*\dy})
	-- ({11.0292*\dx},{1.0000*\dy})
	-- ({11.0392*\dx},{1.0000*\dy})
	-- ({11.0492*\dx},{1.0000*\dy})
	-- ({11.0592*\dx},{1.0000*\dy})
	-- ({11.0692*\dx},{1.0000*\dy})
	-- ({11.0792*\dx},{1.0000*\dy})
	-- ({11.0892*\dx},{1.0000*\dy})
	-- ({11.0992*\dx},{1.0000*\dy})
	-- ({11.1093*\dx},{1.0000*\dy})
	-- ({11.1193*\dx},{1.0000*\dy})
	-- ({11.1293*\dx},{1.0000*\dy})
	-- ({11.1393*\dx},{1.0000*\dy})
	-- ({11.1493*\dx},{1.0000*\dy})
	-- ({11.1593*\dx},{1.0000*\dy})
	-- ({11.1693*\dx},{1.0000*\dy})
	-- ({11.1793*\dx},{1.0000*\dy})
	-- ({11.1893*\dx},{1.0000*\dy})
	-- ({11.1993*\dx},{1.0000*\dy})
	-- ({11.2093*\dx},{1.0000*\dy})
	-- ({11.2193*\dx},{1.0000*\dy})
	-- ({11.2294*\dx},{1.0000*\dy})
	-- ({11.2394*\dx},{1.0000*\dy})
	-- ({11.2494*\dx},{1.0000*\dy})
	-- ({11.2594*\dx},{1.0000*\dy})
	-- ({11.2694*\dx},{1.0000*\dy})
	-- ({11.2794*\dx},{1.0000*\dy})
	-- ({11.2894*\dx},{1.0000*\dy})
	-- ({11.2994*\dx},{1.0000*\dy})
	-- ({11.3094*\dx},{1.0000*\dy})
	-- ({11.3194*\dx},{1.0000*\dy})
	-- ({11.3294*\dx},{1.0000*\dy})
	-- ({11.3394*\dx},{1.0000*\dy})
	-- ({11.3495*\dx},{1.0000*\dy})
	-- ({11.3595*\dx},{1.0000*\dy})
	-- ({11.3695*\dx},{1.0000*\dy})
	-- ({11.3795*\dx},{1.0000*\dy})
	-- ({11.3895*\dx},{1.0000*\dy})
	-- ({11.3995*\dx},{1.0000*\dy})
	-- ({11.4095*\dx},{1.0000*\dy})
	-- ({11.4195*\dx},{1.0000*\dy})
	-- ({11.4295*\dx},{1.0000*\dy})
	-- ({11.4395*\dx},{1.0000*\dy})
	-- ({11.4495*\dx},{1.0000*\dy})
	-- ({11.4595*\dx},{1.0000*\dy})
	-- ({11.4696*\dx},{1.0000*\dy})
	-- ({11.4796*\dx},{1.0000*\dy})
	-- ({11.4896*\dx},{1.0000*\dy})
	-- ({11.4996*\dx},{1.0000*\dy})
	-- ({11.5096*\dx},{1.0000*\dy})
	-- ({11.5196*\dx},{1.0000*\dy})
	-- ({11.5296*\dx},{1.0000*\dy})
	-- ({11.5396*\dx},{1.0000*\dy})
	-- ({11.5496*\dx},{1.0000*\dy})
	-- ({11.5596*\dx},{1.0000*\dy})
	-- ({11.5696*\dx},{1.0000*\dy})
	-- ({11.5796*\dx},{1.0000*\dy})
	-- ({11.5897*\dx},{1.0000*\dy})
	-- ({11.5997*\dx},{1.0000*\dy})
	-- ({11.6097*\dx},{1.0000*\dy})
	-- ({11.6197*\dx},{1.0000*\dy})
	-- ({11.6297*\dx},{1.0000*\dy})
	-- ({11.6397*\dx},{1.0000*\dy})
	-- ({11.6497*\dx},{1.0000*\dy})
	-- ({11.6597*\dx},{1.0000*\dy})
	-- ({11.6697*\dx},{1.0000*\dy})
	-- ({11.6797*\dx},{1.0000*\dy})
	-- ({11.6897*\dx},{1.0000*\dy})
	-- ({11.6997*\dx},{1.0000*\dy})
	-- ({11.7098*\dx},{1.0000*\dy})
	-- ({11.7198*\dx},{1.0000*\dy})
	-- ({11.7298*\dx},{1.0000*\dy})
	-- ({11.7398*\dx},{1.0000*\dy})
	-- ({11.7498*\dx},{1.0000*\dy})
	-- ({11.7598*\dx},{1.0000*\dy})
	-- ({11.7698*\dx},{1.0000*\dy})
	-- ({11.7798*\dx},{1.0000*\dy})
	-- ({11.7898*\dx},{1.0000*\dy})
	-- ({11.7998*\dx},{1.0000*\dy})
	-- ({11.8098*\dx},{1.0000*\dy})
	-- ({11.8198*\dx},{1.0000*\dy})
	-- ({11.8299*\dx},{1.0000*\dy})
	-- ({11.8399*\dx},{1.0000*\dy})
	-- ({11.8499*\dx},{1.0000*\dy})
	-- ({11.8599*\dx},{1.0000*\dy})
	-- ({11.8699*\dx},{1.0000*\dy})
	-- ({11.8799*\dx},{1.0000*\dy})
	-- ({11.8899*\dx},{1.0000*\dy})
	-- ({11.8999*\dx},{1.0000*\dy})
	-- ({11.9099*\dx},{1.0000*\dy})
	-- ({11.9199*\dx},{1.0000*\dy})
	-- ({11.9299*\dx},{1.0000*\dy})
	-- ({11.9399*\dx},{1.0000*\dy})
	-- ({11.9500*\dx},{1.0000*\dy})
	-- ({11.9600*\dx},{1.0000*\dy})
	-- ({11.9700*\dx},{1.0000*\dy})
	-- ({11.9800*\dx},{1.0000*\dy})
	-- ({11.9900*\dx},{1.0000*\dy})
	-- ({12.0000*\dx},{1.0000*\dy})
}
\def\cpsifive{
	({0.0000*\dx},{1.0000*\dy})
	-- ({0.0100*\dx},{1.0000*\dy})
	-- ({0.0200*\dx},{1.0000*\dy})
	-- ({0.0300*\dx},{1.0000*\dy})
	-- ({0.0400*\dx},{1.0000*\dy})
	-- ({0.0500*\dx},{1.0000*\dy})
	-- ({0.0601*\dx},{1.0000*\dy})
	-- ({0.0701*\dx},{1.0000*\dy})
	-- ({0.0801*\dx},{1.0000*\dy})
	-- ({0.0901*\dx},{1.0000*\dy})
	-- ({0.1001*\dx},{1.0000*\dy})
	-- ({0.1101*\dx},{1.0000*\dy})
	-- ({0.1201*\dx},{1.0000*\dy})
	-- ({0.1301*\dx},{1.0000*\dy})
	-- ({0.1401*\dx},{1.0000*\dy})
	-- ({0.1501*\dx},{1.0000*\dy})
	-- ({0.1601*\dx},{1.0000*\dy})
	-- ({0.1701*\dx},{1.0000*\dy})
	-- ({0.1802*\dx},{1.0000*\dy})
	-- ({0.1902*\dx},{1.0000*\dy})
	-- ({0.2002*\dx},{1.0000*\dy})
	-- ({0.2102*\dx},{1.0000*\dy})
	-- ({0.2202*\dx},{1.0000*\dy})
	-- ({0.2302*\dx},{1.0000*\dy})
	-- ({0.2402*\dx},{1.0000*\dy})
	-- ({0.2502*\dx},{1.0000*\dy})
	-- ({0.2602*\dx},{1.0000*\dy})
	-- ({0.2702*\dx},{1.0000*\dy})
	-- ({0.2802*\dx},{1.0000*\dy})
	-- ({0.2902*\dx},{1.0000*\dy})
	-- ({0.3003*\dx},{1.0000*\dy})
	-- ({0.3103*\dx},{1.0000*\dy})
	-- ({0.3203*\dx},{1.0000*\dy})
	-- ({0.3303*\dx},{1.0000*\dy})
	-- ({0.3403*\dx},{1.0000*\dy})
	-- ({0.3503*\dx},{1.0000*\dy})
	-- ({0.3603*\dx},{1.0000*\dy})
	-- ({0.3703*\dx},{1.0000*\dy})
	-- ({0.3803*\dx},{1.0000*\dy})
	-- ({0.3903*\dx},{1.0000*\dy})
	-- ({0.4003*\dx},{1.0000*\dy})
	-- ({0.4103*\dx},{1.0000*\dy})
	-- ({0.4204*\dx},{1.0000*\dy})
	-- ({0.4304*\dx},{1.0000*\dy})
	-- ({0.4404*\dx},{1.0000*\dy})
	-- ({0.4504*\dx},{1.0000*\dy})
	-- ({0.4604*\dx},{1.0000*\dy})
	-- ({0.4704*\dx},{1.0000*\dy})
	-- ({0.4804*\dx},{1.0000*\dy})
	-- ({0.4904*\dx},{1.0000*\dy})
	-- ({0.5004*\dx},{1.0000*\dy})
	-- ({0.5104*\dx},{1.0000*\dy})
	-- ({0.5204*\dx},{1.0000*\dy})
	-- ({0.5304*\dx},{1.0000*\dy})
	-- ({0.5405*\dx},{1.0000*\dy})
	-- ({0.5505*\dx},{1.0000*\dy})
	-- ({0.5605*\dx},{1.0000*\dy})
	-- ({0.5705*\dx},{1.0000*\dy})
	-- ({0.5805*\dx},{1.0000*\dy})
	-- ({0.5905*\dx},{1.0000*\dy})
	-- ({0.6005*\dx},{1.0000*\dy})
	-- ({0.6105*\dx},{1.0000*\dy})
	-- ({0.6205*\dx},{1.0000*\dy})
	-- ({0.6305*\dx},{1.0000*\dy})
	-- ({0.6405*\dx},{1.0000*\dy})
	-- ({0.6505*\dx},{1.0000*\dy})
	-- ({0.6606*\dx},{1.0000*\dy})
	-- ({0.6706*\dx},{1.0000*\dy})
	-- ({0.6806*\dx},{1.0000*\dy})
	-- ({0.6906*\dx},{1.0000*\dy})
	-- ({0.7006*\dx},{1.0000*\dy})
	-- ({0.7106*\dx},{1.0000*\dy})
	-- ({0.7206*\dx},{1.0000*\dy})
	-- ({0.7306*\dx},{1.0000*\dy})
	-- ({0.7406*\dx},{1.0000*\dy})
	-- ({0.7506*\dx},{1.0000*\dy})
	-- ({0.7606*\dx},{1.0000*\dy})
	-- ({0.7706*\dx},{1.0000*\dy})
	-- ({0.7807*\dx},{1.0000*\dy})
	-- ({0.7907*\dx},{1.0000*\dy})
	-- ({0.8007*\dx},{1.0000*\dy})
	-- ({0.8107*\dx},{1.0000*\dy})
	-- ({0.8207*\dx},{1.0000*\dy})
	-- ({0.8307*\dx},{1.0000*\dy})
	-- ({0.8407*\dx},{1.0000*\dy})
	-- ({0.8507*\dx},{1.0000*\dy})
	-- ({0.8607*\dx},{1.0000*\dy})
	-- ({0.8707*\dx},{1.0000*\dy})
	-- ({0.8807*\dx},{1.0000*\dy})
	-- ({0.8907*\dx},{1.0000*\dy})
	-- ({0.9008*\dx},{1.0000*\dy})
	-- ({0.9108*\dx},{1.0000*\dy})
	-- ({0.9208*\dx},{1.0000*\dy})
	-- ({0.9308*\dx},{1.0000*\dy})
	-- ({0.9408*\dx},{1.0000*\dy})
	-- ({0.9508*\dx},{1.0000*\dy})
	-- ({0.9608*\dx},{1.0000*\dy})
	-- ({0.9708*\dx},{1.0000*\dy})
	-- ({0.9808*\dx},{1.0000*\dy})
	-- ({0.9908*\dx},{1.0000*\dy})
	-- ({1.0008*\dx},{1.0000*\dy})
	-- ({1.0108*\dx},{1.0000*\dy})
	-- ({1.0209*\dx},{1.0000*\dy})
	-- ({1.0309*\dx},{1.0000*\dy})
	-- ({1.0409*\dx},{1.0000*\dy})
	-- ({1.0509*\dx},{1.0000*\dy})
	-- ({1.0609*\dx},{1.0000*\dy})
	-- ({1.0709*\dx},{1.0000*\dy})
	-- ({1.0809*\dx},{1.0000*\dy})
	-- ({1.0909*\dx},{1.0000*\dy})
	-- ({1.1009*\dx},{1.0000*\dy})
	-- ({1.1109*\dx},{1.0000*\dy})
	-- ({1.1209*\dx},{1.0000*\dy})
	-- ({1.1309*\dx},{1.0000*\dy})
	-- ({1.1410*\dx},{1.0000*\dy})
	-- ({1.1510*\dx},{1.0000*\dy})
	-- ({1.1610*\dx},{1.0000*\dy})
	-- ({1.1710*\dx},{1.0000*\dy})
	-- ({1.1810*\dx},{1.0000*\dy})
	-- ({1.1910*\dx},{1.0000*\dy})
	-- ({1.2010*\dx},{1.0000*\dy})
	-- ({1.2110*\dx},{1.0000*\dy})
	-- ({1.2210*\dx},{1.0000*\dy})
	-- ({1.2310*\dx},{1.0000*\dy})
	-- ({1.2410*\dx},{1.0000*\dy})
	-- ({1.2510*\dx},{1.0000*\dy})
	-- ({1.2611*\dx},{1.0000*\dy})
	-- ({1.2711*\dx},{1.0000*\dy})
	-- ({1.2811*\dx},{1.0000*\dy})
	-- ({1.2911*\dx},{1.0000*\dy})
	-- ({1.3011*\dx},{1.0000*\dy})
	-- ({1.3111*\dx},{1.0000*\dy})
	-- ({1.3211*\dx},{1.0000*\dy})
	-- ({1.3311*\dx},{1.0000*\dy})
	-- ({1.3411*\dx},{1.0000*\dy})
	-- ({1.3511*\dx},{1.0000*\dy})
	-- ({1.3611*\dx},{1.0000*\dy})
	-- ({1.3711*\dx},{1.0000*\dy})
	-- ({1.3812*\dx},{1.0000*\dy})
	-- ({1.3912*\dx},{1.0000*\dy})
	-- ({1.4012*\dx},{1.0000*\dy})
	-- ({1.4112*\dx},{1.0000*\dy})
	-- ({1.4212*\dx},{1.0000*\dy})
	-- ({1.4312*\dx},{1.0000*\dy})
	-- ({1.4412*\dx},{1.0000*\dy})
	-- ({1.4512*\dx},{1.0000*\dy})
	-- ({1.4612*\dx},{1.0000*\dy})
	-- ({1.4712*\dx},{1.0000*\dy})
	-- ({1.4812*\dx},{1.0000*\dy})
	-- ({1.4912*\dx},{1.0000*\dy})
	-- ({1.5013*\dx},{1.0000*\dy})
	-- ({1.5113*\dx},{1.0000*\dy})
	-- ({1.5213*\dx},{1.0000*\dy})
	-- ({1.5313*\dx},{1.0000*\dy})
	-- ({1.5413*\dx},{1.0000*\dy})
	-- ({1.5513*\dx},{1.0000*\dy})
	-- ({1.5613*\dx},{1.0000*\dy})
	-- ({1.5713*\dx},{1.0000*\dy})
	-- ({1.5813*\dx},{1.0000*\dy})
	-- ({1.5913*\dx},{1.0000*\dy})
	-- ({1.6013*\dx},{1.0000*\dy})
	-- ({1.6113*\dx},{1.0000*\dy})
	-- ({1.6214*\dx},{1.0000*\dy})
	-- ({1.6314*\dx},{1.0000*\dy})
	-- ({1.6414*\dx},{1.0000*\dy})
	-- ({1.6514*\dx},{1.0000*\dy})
	-- ({1.6614*\dx},{1.0000*\dy})
	-- ({1.6714*\dx},{1.0000*\dy})
	-- ({1.6814*\dx},{1.0000*\dy})
	-- ({1.6914*\dx},{1.0000*\dy})
	-- ({1.7014*\dx},{1.0000*\dy})
	-- ({1.7114*\dx},{1.0000*\dy})
	-- ({1.7214*\dx},{1.0000*\dy})
	-- ({1.7314*\dx},{1.0000*\dy})
	-- ({1.7415*\dx},{1.0000*\dy})
	-- ({1.7515*\dx},{1.0000*\dy})
	-- ({1.7615*\dx},{1.0000*\dy})
	-- ({1.7715*\dx},{1.0000*\dy})
	-- ({1.7815*\dx},{1.0000*\dy})
	-- ({1.7915*\dx},{1.0000*\dy})
	-- ({1.8015*\dx},{1.0000*\dy})
	-- ({1.8115*\dx},{1.0000*\dy})
	-- ({1.8215*\dx},{1.0000*\dy})
	-- ({1.8315*\dx},{1.0000*\dy})
	-- ({1.8415*\dx},{1.0000*\dy})
	-- ({1.8515*\dx},{1.0000*\dy})
	-- ({1.8616*\dx},{1.0000*\dy})
	-- ({1.8716*\dx},{1.0000*\dy})
	-- ({1.8816*\dx},{1.0000*\dy})
	-- ({1.8916*\dx},{1.0000*\dy})
	-- ({1.9016*\dx},{1.0000*\dy})
	-- ({1.9116*\dx},{1.0000*\dy})
	-- ({1.9216*\dx},{1.0000*\dy})
	-- ({1.9316*\dx},{1.0000*\dy})
	-- ({1.9416*\dx},{1.0000*\dy})
	-- ({1.9516*\dx},{1.0000*\dy})
	-- ({1.9616*\dx},{1.0000*\dy})
	-- ({1.9716*\dx},{1.0000*\dy})
	-- ({1.9817*\dx},{1.0000*\dy})
	-- ({1.9917*\dx},{1.0000*\dy})
	-- ({2.0017*\dx},{1.0000*\dy})
	-- ({2.0117*\dx},{1.0000*\dy})
	-- ({2.0217*\dx},{1.0000*\dy})
	-- ({2.0317*\dx},{1.0000*\dy})
	-- ({2.0417*\dx},{1.0000*\dy})
	-- ({2.0517*\dx},{1.0000*\dy})
	-- ({2.0617*\dx},{1.0000*\dy})
	-- ({2.0717*\dx},{1.0000*\dy})
	-- ({2.0817*\dx},{1.0000*\dy})
	-- ({2.0917*\dx},{1.0000*\dy})
	-- ({2.1018*\dx},{1.0000*\dy})
	-- ({2.1118*\dx},{1.0000*\dy})
	-- ({2.1218*\dx},{1.0000*\dy})
	-- ({2.1318*\dx},{1.0000*\dy})
	-- ({2.1418*\dx},{1.0000*\dy})
	-- ({2.1518*\dx},{1.0000*\dy})
	-- ({2.1618*\dx},{1.0000*\dy})
	-- ({2.1718*\dx},{1.0000*\dy})
	-- ({2.1818*\dx},{1.0000*\dy})
	-- ({2.1918*\dx},{1.0000*\dy})
	-- ({2.2018*\dx},{1.0000*\dy})
	-- ({2.2118*\dx},{1.0000*\dy})
	-- ({2.2219*\dx},{1.0000*\dy})
	-- ({2.2319*\dx},{1.0000*\dy})
	-- ({2.2419*\dx},{1.0000*\dy})
	-- ({2.2519*\dx},{1.0000*\dy})
	-- ({2.2619*\dx},{1.0000*\dy})
	-- ({2.2719*\dx},{1.0000*\dy})
	-- ({2.2819*\dx},{1.0000*\dy})
	-- ({2.2919*\dx},{1.0000*\dy})
	-- ({2.3019*\dx},{1.0000*\dy})
	-- ({2.3119*\dx},{1.0000*\dy})
	-- ({2.3219*\dx},{1.0000*\dy})
	-- ({2.3319*\dx},{1.0000*\dy})
	-- ({2.3420*\dx},{1.0000*\dy})
	-- ({2.3520*\dx},{1.0000*\dy})
	-- ({2.3620*\dx},{1.0000*\dy})
	-- ({2.3720*\dx},{1.0000*\dy})
	-- ({2.3820*\dx},{1.0000*\dy})
	-- ({2.3920*\dx},{1.0000*\dy})
	-- ({2.4020*\dx},{1.0000*\dy})
	-- ({2.4120*\dx},{1.0000*\dy})
	-- ({2.4220*\dx},{1.0000*\dy})
	-- ({2.4320*\dx},{1.0000*\dy})
	-- ({2.4420*\dx},{1.0000*\dy})
	-- ({2.4520*\dx},{1.0000*\dy})
	-- ({2.4621*\dx},{1.0000*\dy})
	-- ({2.4721*\dx},{1.0000*\dy})
	-- ({2.4821*\dx},{1.0000*\dy})
	-- ({2.4921*\dx},{1.0000*\dy})
	-- ({2.5021*\dx},{1.0000*\dy})
	-- ({2.5121*\dx},{1.0000*\dy})
	-- ({2.5221*\dx},{1.0000*\dy})
	-- ({2.5321*\dx},{1.0000*\dy})
	-- ({2.5421*\dx},{1.0000*\dy})
	-- ({2.5521*\dx},{1.0000*\dy})
	-- ({2.5621*\dx},{1.0000*\dy})
	-- ({2.5721*\dx},{1.0000*\dy})
	-- ({2.5822*\dx},{1.0000*\dy})
	-- ({2.5922*\dx},{1.0000*\dy})
	-- ({2.6022*\dx},{1.0000*\dy})
	-- ({2.6122*\dx},{1.0000*\dy})
	-- ({2.6222*\dx},{1.0000*\dy})
	-- ({2.6322*\dx},{1.0000*\dy})
	-- ({2.6422*\dx},{1.0000*\dy})
	-- ({2.6522*\dx},{1.0000*\dy})
	-- ({2.6622*\dx},{1.0000*\dy})
	-- ({2.6722*\dx},{1.0000*\dy})
	-- ({2.6822*\dx},{1.0000*\dy})
	-- ({2.6922*\dx},{1.0000*\dy})
	-- ({2.7023*\dx},{1.0000*\dy})
	-- ({2.7123*\dx},{1.0000*\dy})
	-- ({2.7223*\dx},{1.0000*\dy})
	-- ({2.7323*\dx},{1.0000*\dy})
	-- ({2.7423*\dx},{1.0000*\dy})
	-- ({2.7523*\dx},{1.0000*\dy})
	-- ({2.7623*\dx},{1.0000*\dy})
	-- ({2.7723*\dx},{1.0000*\dy})
	-- ({2.7823*\dx},{1.0000*\dy})
	-- ({2.7923*\dx},{1.0000*\dy})
	-- ({2.8023*\dx},{1.0000*\dy})
	-- ({2.8123*\dx},{1.0000*\dy})
	-- ({2.8224*\dx},{1.0000*\dy})
	-- ({2.8324*\dx},{1.0000*\dy})
	-- ({2.8424*\dx},{1.0000*\dy})
	-- ({2.8524*\dx},{1.0000*\dy})
	-- ({2.8624*\dx},{1.0000*\dy})
	-- ({2.8724*\dx},{1.0000*\dy})
	-- ({2.8824*\dx},{1.0000*\dy})
	-- ({2.8924*\dx},{1.0000*\dy})
	-- ({2.9024*\dx},{1.0000*\dy})
	-- ({2.9124*\dx},{1.0000*\dy})
	-- ({2.9224*\dx},{1.0000*\dy})
	-- ({2.9324*\dx},{1.0000*\dy})
	-- ({2.9425*\dx},{1.0000*\dy})
	-- ({2.9525*\dx},{1.0000*\dy})
	-- ({2.9625*\dx},{1.0000*\dy})
	-- ({2.9725*\dx},{1.0000*\dy})
	-- ({2.9825*\dx},{1.0000*\dy})
	-- ({2.9925*\dx},{1.0000*\dy})
	-- ({3.0025*\dx},{1.0000*\dy})
	-- ({3.0125*\dx},{1.0000*\dy})
	-- ({3.0225*\dx},{1.0000*\dy})
	-- ({3.0325*\dx},{1.0000*\dy})
	-- ({3.0425*\dx},{1.0000*\dy})
	-- ({3.0525*\dx},{1.0000*\dy})
	-- ({3.0626*\dx},{1.0000*\dy})
	-- ({3.0726*\dx},{1.0000*\dy})
	-- ({3.0826*\dx},{1.0000*\dy})
	-- ({3.0926*\dx},{1.0000*\dy})
	-- ({3.1026*\dx},{1.0000*\dy})
	-- ({3.1126*\dx},{1.0000*\dy})
	-- ({3.1226*\dx},{1.0000*\dy})
	-- ({3.1326*\dx},{1.0000*\dy})
	-- ({3.1426*\dx},{1.0000*\dy})
	-- ({3.1526*\dx},{1.0000*\dy})
	-- ({3.1626*\dx},{1.0000*\dy})
	-- ({3.1726*\dx},{1.0000*\dy})
	-- ({3.1827*\dx},{1.0000*\dy})
	-- ({3.1927*\dx},{1.0000*\dy})
	-- ({3.2027*\dx},{1.0000*\dy})
	-- ({3.2127*\dx},{1.0000*\dy})
	-- ({3.2227*\dx},{1.0000*\dy})
	-- ({3.2327*\dx},{1.0000*\dy})
	-- ({3.2427*\dx},{1.0000*\dy})
	-- ({3.2527*\dx},{1.0000*\dy})
	-- ({3.2627*\dx},{1.0000*\dy})
	-- ({3.2727*\dx},{1.0000*\dy})
	-- ({3.2827*\dx},{1.0000*\dy})
	-- ({3.2927*\dx},{1.0000*\dy})
	-- ({3.3028*\dx},{1.0000*\dy})
	-- ({3.3128*\dx},{1.0000*\dy})
	-- ({3.3228*\dx},{1.0000*\dy})
	-- ({3.3328*\dx},{1.0000*\dy})
	-- ({3.3428*\dx},{1.0000*\dy})
	-- ({3.3528*\dx},{1.0000*\dy})
	-- ({3.3628*\dx},{1.0000*\dy})
	-- ({3.3728*\dx},{1.0000*\dy})
	-- ({3.3828*\dx},{1.0000*\dy})
	-- ({3.3928*\dx},{1.0000*\dy})
	-- ({3.4028*\dx},{1.0000*\dy})
	-- ({3.4128*\dx},{1.0000*\dy})
	-- ({3.4229*\dx},{1.0000*\dy})
	-- ({3.4329*\dx},{1.0000*\dy})
	-- ({3.4429*\dx},{1.0000*\dy})
	-- ({3.4529*\dx},{1.0000*\dy})
	-- ({3.4629*\dx},{1.0000*\dy})
	-- ({3.4729*\dx},{1.0000*\dy})
	-- ({3.4829*\dx},{1.0000*\dy})
	-- ({3.4929*\dx},{1.0000*\dy})
	-- ({3.5029*\dx},{1.0000*\dy})
	-- ({3.5129*\dx},{1.0000*\dy})
	-- ({3.5229*\dx},{1.0000*\dy})
	-- ({3.5329*\dx},{1.0000*\dy})
	-- ({3.5430*\dx},{1.0000*\dy})
	-- ({3.5530*\dx},{1.0000*\dy})
	-- ({3.5630*\dx},{1.0000*\dy})
	-- ({3.5730*\dx},{1.0000*\dy})
	-- ({3.5830*\dx},{1.0000*\dy})
	-- ({3.5930*\dx},{1.0000*\dy})
	-- ({3.6030*\dx},{1.0000*\dy})
	-- ({3.6130*\dx},{1.0000*\dy})
	-- ({3.6230*\dx},{1.0000*\dy})
	-- ({3.6330*\dx},{1.0000*\dy})
	-- ({3.6430*\dx},{1.0000*\dy})
	-- ({3.6530*\dx},{1.0000*\dy})
	-- ({3.6631*\dx},{1.0000*\dy})
	-- ({3.6731*\dx},{1.0000*\dy})
	-- ({3.6831*\dx},{1.0000*\dy})
	-- ({3.6931*\dx},{1.0000*\dy})
	-- ({3.7031*\dx},{1.0000*\dy})
	-- ({3.7131*\dx},{1.0000*\dy})
	-- ({3.7231*\dx},{1.0000*\dy})
	-- ({3.7331*\dx},{1.0000*\dy})
	-- ({3.7431*\dx},{1.0000*\dy})
	-- ({3.7531*\dx},{1.0000*\dy})
	-- ({3.7631*\dx},{1.0000*\dy})
	-- ({3.7731*\dx},{1.0000*\dy})
	-- ({3.7832*\dx},{1.0000*\dy})
	-- ({3.7932*\dx},{1.0000*\dy})
	-- ({3.8032*\dx},{1.0000*\dy})
	-- ({3.8132*\dx},{1.0000*\dy})
	-- ({3.8232*\dx},{1.0000*\dy})
	-- ({3.8332*\dx},{1.0000*\dy})
	-- ({3.8432*\dx},{1.0000*\dy})
	-- ({3.8532*\dx},{1.0000*\dy})
	-- ({3.8632*\dx},{1.0000*\dy})
	-- ({3.8732*\dx},{1.0000*\dy})
	-- ({3.8832*\dx},{1.0000*\dy})
	-- ({3.8932*\dx},{1.0000*\dy})
	-- ({3.9033*\dx},{1.0000*\dy})
	-- ({3.9133*\dx},{1.0000*\dy})
	-- ({3.9233*\dx},{1.0000*\dy})
	-- ({3.9333*\dx},{1.0000*\dy})
	-- ({3.9433*\dx},{1.0000*\dy})
	-- ({3.9533*\dx},{1.0000*\dy})
	-- ({3.9633*\dx},{1.0000*\dy})
	-- ({3.9733*\dx},{1.0000*\dy})
	-- ({3.9833*\dx},{1.0000*\dy})
	-- ({3.9933*\dx},{1.0000*\dy})
	-- ({4.0033*\dx},{1.0000*\dy})
	-- ({4.0133*\dx},{1.0000*\dy})
	-- ({4.0234*\dx},{1.0000*\dy})
	-- ({4.0334*\dx},{1.0000*\dy})
	-- ({4.0434*\dx},{1.0000*\dy})
	-- ({4.0534*\dx},{1.0000*\dy})
	-- ({4.0634*\dx},{1.0000*\dy})
	-- ({4.0734*\dx},{1.0000*\dy})
	-- ({4.0834*\dx},{1.0000*\dy})
	-- ({4.0934*\dx},{1.0000*\dy})
	-- ({4.1034*\dx},{1.0000*\dy})
	-- ({4.1134*\dx},{1.0000*\dy})
	-- ({4.1234*\dx},{1.0000*\dy})
	-- ({4.1334*\dx},{1.0000*\dy})
	-- ({4.1435*\dx},{1.0000*\dy})
	-- ({4.1535*\dx},{1.0000*\dy})
	-- ({4.1635*\dx},{1.0000*\dy})
	-- ({4.1735*\dx},{1.0000*\dy})
	-- ({4.1835*\dx},{1.0000*\dy})
	-- ({4.1935*\dx},{1.0000*\dy})
	-- ({4.2035*\dx},{1.0000*\dy})
	-- ({4.2135*\dx},{1.0000*\dy})
	-- ({4.2235*\dx},{1.0000*\dy})
	-- ({4.2335*\dx},{1.0000*\dy})
	-- ({4.2435*\dx},{1.0000*\dy})
	-- ({4.2535*\dx},{1.0000*\dy})
	-- ({4.2636*\dx},{1.0000*\dy})
	-- ({4.2736*\dx},{1.0000*\dy})
	-- ({4.2836*\dx},{1.0000*\dy})
	-- ({4.2936*\dx},{1.0000*\dy})
	-- ({4.3036*\dx},{1.0000*\dy})
	-- ({4.3136*\dx},{1.0000*\dy})
	-- ({4.3236*\dx},{1.0000*\dy})
	-- ({4.3336*\dx},{1.0000*\dy})
	-- ({4.3436*\dx},{1.0000*\dy})
	-- ({4.3536*\dx},{1.0000*\dy})
	-- ({4.3636*\dx},{1.0000*\dy})
	-- ({4.3736*\dx},{1.0000*\dy})
	-- ({4.3837*\dx},{1.0000*\dy})
	-- ({4.3937*\dx},{1.0000*\dy})
	-- ({4.4037*\dx},{1.0000*\dy})
	-- ({4.4137*\dx},{1.0000*\dy})
	-- ({4.4237*\dx},{1.0000*\dy})
	-- ({4.4337*\dx},{1.0000*\dy})
	-- ({4.4437*\dx},{1.0000*\dy})
	-- ({4.4537*\dx},{1.0000*\dy})
	-- ({4.4637*\dx},{1.0000*\dy})
	-- ({4.4737*\dx},{1.0000*\dy})
	-- ({4.4837*\dx},{1.0000*\dy})
	-- ({4.4937*\dx},{1.0000*\dy})
	-- ({4.5038*\dx},{1.0000*\dy})
	-- ({4.5138*\dx},{1.0000*\dy})
	-- ({4.5238*\dx},{1.0000*\dy})
	-- ({4.5338*\dx},{1.0000*\dy})
	-- ({4.5438*\dx},{1.0000*\dy})
	-- ({4.5538*\dx},{1.0000*\dy})
	-- ({4.5638*\dx},{1.0000*\dy})
	-- ({4.5738*\dx},{1.0000*\dy})
	-- ({4.5838*\dx},{1.0000*\dy})
	-- ({4.5938*\dx},{1.0000*\dy})
	-- ({4.6038*\dx},{1.0000*\dy})
	-- ({4.6138*\dx},{1.0000*\dy})
	-- ({4.6239*\dx},{1.0000*\dy})
	-- ({4.6339*\dx},{1.0000*\dy})
	-- ({4.6439*\dx},{1.0000*\dy})
	-- ({4.6539*\dx},{1.0000*\dy})
	-- ({4.6639*\dx},{1.0000*\dy})
	-- ({4.6739*\dx},{1.0000*\dy})
	-- ({4.6839*\dx},{1.0000*\dy})
	-- ({4.6939*\dx},{1.0000*\dy})
	-- ({4.7039*\dx},{1.0000*\dy})
	-- ({4.7139*\dx},{1.0000*\dy})
	-- ({4.7239*\dx},{1.0000*\dy})
	-- ({4.7339*\dx},{1.0000*\dy})
	-- ({4.7440*\dx},{1.0000*\dy})
	-- ({4.7540*\dx},{1.0000*\dy})
	-- ({4.7640*\dx},{1.0000*\dy})
	-- ({4.7740*\dx},{1.0000*\dy})
	-- ({4.7840*\dx},{1.0000*\dy})
	-- ({4.7940*\dx},{1.0000*\dy})
	-- ({4.8040*\dx},{1.0000*\dy})
	-- ({4.8140*\dx},{1.0000*\dy})
	-- ({4.8240*\dx},{1.0000*\dy})
	-- ({4.8340*\dx},{1.0000*\dy})
	-- ({4.8440*\dx},{1.0000*\dy})
	-- ({4.8540*\dx},{1.0000*\dy})
	-- ({4.8641*\dx},{1.0000*\dy})
	-- ({4.8741*\dx},{1.0000*\dy})
	-- ({4.8841*\dx},{1.0000*\dy})
	-- ({4.8941*\dx},{1.0000*\dy})
	-- ({4.9041*\dx},{1.0000*\dy})
	-- ({4.9141*\dx},{1.0000*\dy})
	-- ({4.9241*\dx},{1.0000*\dy})
	-- ({4.9341*\dx},{1.0000*\dy})
	-- ({4.9441*\dx},{1.0000*\dy})
	-- ({4.9541*\dx},{1.0000*\dy})
	-- ({4.9641*\dx},{1.0000*\dy})
	-- ({4.9741*\dx},{1.0000*\dy})
	-- ({4.9842*\dx},{1.0000*\dy})
	-- ({4.9942*\dx},{1.0000*\dy})
	-- ({5.0042*\dx},{1.0000*\dy})
	-- ({5.0142*\dx},{1.0000*\dy})
	-- ({5.0242*\dx},{1.0000*\dy})
	-- ({5.0342*\dx},{1.0000*\dy})
	-- ({5.0442*\dx},{1.0000*\dy})
	-- ({5.0542*\dx},{1.0000*\dy})
	-- ({5.0642*\dx},{1.0000*\dy})
	-- ({5.0742*\dx},{1.0000*\dy})
	-- ({5.0842*\dx},{1.0000*\dy})
	-- ({5.0942*\dx},{1.0000*\dy})
	-- ({5.1043*\dx},{1.0000*\dy})
	-- ({5.1143*\dx},{1.0000*\dy})
	-- ({5.1243*\dx},{1.0000*\dy})
	-- ({5.1343*\dx},{1.0000*\dy})
	-- ({5.1443*\dx},{1.0000*\dy})
	-- ({5.1543*\dx},{1.0000*\dy})
	-- ({5.1643*\dx},{1.0000*\dy})
	-- ({5.1743*\dx},{1.0000*\dy})
	-- ({5.1843*\dx},{1.0000*\dy})
	-- ({5.1943*\dx},{1.0000*\dy})
	-- ({5.2043*\dx},{1.0000*\dy})
	-- ({5.2143*\dx},{1.0000*\dy})
	-- ({5.2244*\dx},{1.0000*\dy})
	-- ({5.2344*\dx},{1.0000*\dy})
	-- ({5.2444*\dx},{1.0000*\dy})
	-- ({5.2544*\dx},{1.0000*\dy})
	-- ({5.2644*\dx},{1.0000*\dy})
	-- ({5.2744*\dx},{1.0000*\dy})
	-- ({5.2844*\dx},{1.0000*\dy})
	-- ({5.2944*\dx},{1.0000*\dy})
	-- ({5.3044*\dx},{1.0000*\dy})
	-- ({5.3144*\dx},{1.0000*\dy})
	-- ({5.3244*\dx},{1.0000*\dy})
	-- ({5.3344*\dx},{1.0000*\dy})
	-- ({5.3445*\dx},{1.0000*\dy})
	-- ({5.3545*\dx},{1.0000*\dy})
	-- ({5.3645*\dx},{1.0000*\dy})
	-- ({5.3745*\dx},{1.0000*\dy})
	-- ({5.3845*\dx},{1.0000*\dy})
	-- ({5.3945*\dx},{1.0000*\dy})
	-- ({5.4045*\dx},{1.0000*\dy})
	-- ({5.4145*\dx},{1.0000*\dy})
	-- ({5.4245*\dx},{1.0000*\dy})
	-- ({5.4345*\dx},{1.0000*\dy})
	-- ({5.4445*\dx},{1.0000*\dy})
	-- ({5.4545*\dx},{1.0000*\dy})
	-- ({5.4646*\dx},{1.0000*\dy})
	-- ({5.4746*\dx},{1.0000*\dy})
	-- ({5.4846*\dx},{1.0000*\dy})
	-- ({5.4946*\dx},{1.0000*\dy})
	-- ({5.5046*\dx},{1.0000*\dy})
	-- ({5.5146*\dx},{1.0000*\dy})
	-- ({5.5246*\dx},{1.0000*\dy})
	-- ({5.5346*\dx},{1.0000*\dy})
	-- ({5.5446*\dx},{1.0000*\dy})
	-- ({5.5546*\dx},{1.0000*\dy})
	-- ({5.5646*\dx},{1.0000*\dy})
	-- ({5.5746*\dx},{1.0000*\dy})
	-- ({5.5847*\dx},{1.0000*\dy})
	-- ({5.5947*\dx},{1.0000*\dy})
	-- ({5.6047*\dx},{1.0000*\dy})
	-- ({5.6147*\dx},{1.0000*\dy})
	-- ({5.6247*\dx},{1.0000*\dy})
	-- ({5.6347*\dx},{1.0000*\dy})
	-- ({5.6447*\dx},{1.0000*\dy})
	-- ({5.6547*\dx},{1.0000*\dy})
	-- ({5.6647*\dx},{1.0000*\dy})
	-- ({5.6747*\dx},{1.0000*\dy})
	-- ({5.6847*\dx},{1.0000*\dy})
	-- ({5.6947*\dx},{1.0000*\dy})
	-- ({5.7048*\dx},{1.0000*\dy})
	-- ({5.7148*\dx},{1.0000*\dy})
	-- ({5.7248*\dx},{1.0000*\dy})
	-- ({5.7348*\dx},{1.0000*\dy})
	-- ({5.7448*\dx},{1.0000*\dy})
	-- ({5.7548*\dx},{1.0000*\dy})
	-- ({5.7648*\dx},{1.0000*\dy})
	-- ({5.7748*\dx},{1.0000*\dy})
	-- ({5.7848*\dx},{1.0000*\dy})
	-- ({5.7948*\dx},{1.0000*\dy})
	-- ({5.8048*\dx},{1.0000*\dy})
	-- ({5.8148*\dx},{1.0000*\dy})
	-- ({5.8249*\dx},{1.0000*\dy})
	-- ({5.8349*\dx},{1.0000*\dy})
	-- ({5.8449*\dx},{1.0000*\dy})
	-- ({5.8549*\dx},{1.0000*\dy})
	-- ({5.8649*\dx},{1.0000*\dy})
	-- ({5.8749*\dx},{1.0000*\dy})
	-- ({5.8849*\dx},{1.0000*\dy})
	-- ({5.8949*\dx},{1.0000*\dy})
	-- ({5.9049*\dx},{1.0000*\dy})
	-- ({5.9149*\dx},{1.0000*\dy})
	-- ({5.9249*\dx},{1.0000*\dy})
	-- ({5.9349*\dx},{1.0000*\dy})
	-- ({5.9450*\dx},{1.0000*\dy})
	-- ({5.9550*\dx},{1.0000*\dy})
	-- ({5.9650*\dx},{1.0000*\dy})
	-- ({5.9750*\dx},{1.0000*\dy})
	-- ({5.9850*\dx},{1.0000*\dy})
	-- ({5.9950*\dx},{1.0000*\dy})
	-- ({6.0050*\dx},{1.0000*\dy})
	-- ({6.0150*\dx},{1.0000*\dy})
	-- ({6.0250*\dx},{1.0000*\dy})
	-- ({6.0350*\dx},{1.0000*\dy})
	-- ({6.0450*\dx},{1.0000*\dy})
	-- ({6.0550*\dx},{1.0000*\dy})
	-- ({6.0651*\dx},{1.0000*\dy})
	-- ({6.0751*\dx},{1.0000*\dy})
	-- ({6.0851*\dx},{1.0000*\dy})
	-- ({6.0951*\dx},{1.0000*\dy})
	-- ({6.1051*\dx},{1.0000*\dy})
	-- ({6.1151*\dx},{1.0000*\dy})
	-- ({6.1251*\dx},{1.0000*\dy})
	-- ({6.1351*\dx},{1.0000*\dy})
	-- ({6.1451*\dx},{1.0000*\dy})
	-- ({6.1551*\dx},{1.0000*\dy})
	-- ({6.1651*\dx},{1.0000*\dy})
	-- ({6.1751*\dx},{1.0000*\dy})
	-- ({6.1852*\dx},{1.0000*\dy})
	-- ({6.1952*\dx},{1.0000*\dy})
	-- ({6.2052*\dx},{1.0000*\dy})
	-- ({6.2152*\dx},{1.0000*\dy})
	-- ({6.2252*\dx},{1.0000*\dy})
	-- ({6.2352*\dx},{1.0000*\dy})
	-- ({6.2452*\dx},{1.0000*\dy})
	-- ({6.2552*\dx},{1.0000*\dy})
	-- ({6.2652*\dx},{1.0000*\dy})
	-- ({6.2752*\dx},{1.0000*\dy})
	-- ({6.2852*\dx},{1.0000*\dy})
	-- ({6.2952*\dx},{1.0000*\dy})
	-- ({6.3053*\dx},{1.0000*\dy})
	-- ({6.3153*\dx},{1.0000*\dy})
	-- ({6.3253*\dx},{1.0000*\dy})
	-- ({6.3353*\dx},{1.0000*\dy})
	-- ({6.3453*\dx},{1.0000*\dy})
	-- ({6.3553*\dx},{1.0000*\dy})
	-- ({6.3653*\dx},{1.0000*\dy})
	-- ({6.3753*\dx},{1.0000*\dy})
	-- ({6.3853*\dx},{1.0000*\dy})
	-- ({6.3953*\dx},{1.0000*\dy})
	-- ({6.4053*\dx},{1.0000*\dy})
	-- ({6.4153*\dx},{1.0000*\dy})
	-- ({6.4254*\dx},{1.0000*\dy})
	-- ({6.4354*\dx},{1.0000*\dy})
	-- ({6.4454*\dx},{1.0000*\dy})
	-- ({6.4554*\dx},{1.0000*\dy})
	-- ({6.4654*\dx},{1.0000*\dy})
	-- ({6.4754*\dx},{1.0000*\dy})
	-- ({6.4854*\dx},{1.0000*\dy})
	-- ({6.4954*\dx},{1.0000*\dy})
	-- ({6.5054*\dx},{1.0000*\dy})
	-- ({6.5154*\dx},{1.0000*\dy})
	-- ({6.5254*\dx},{1.0000*\dy})
	-- ({6.5354*\dx},{1.0000*\dy})
	-- ({6.5455*\dx},{1.0000*\dy})
	-- ({6.5555*\dx},{1.0000*\dy})
	-- ({6.5655*\dx},{1.0000*\dy})
	-- ({6.5755*\dx},{1.0000*\dy})
	-- ({6.5855*\dx},{1.0000*\dy})
	-- ({6.5955*\dx},{1.0000*\dy})
	-- ({6.6055*\dx},{1.0000*\dy})
	-- ({6.6155*\dx},{1.0000*\dy})
	-- ({6.6255*\dx},{1.0000*\dy})
	-- ({6.6355*\dx},{1.0000*\dy})
	-- ({6.6455*\dx},{1.0000*\dy})
	-- ({6.6555*\dx},{1.0000*\dy})
	-- ({6.6656*\dx},{1.0000*\dy})
	-- ({6.6756*\dx},{1.0000*\dy})
	-- ({6.6856*\dx},{1.0000*\dy})
	-- ({6.6956*\dx},{1.0000*\dy})
	-- ({6.7056*\dx},{1.0000*\dy})
	-- ({6.7156*\dx},{1.0000*\dy})
	-- ({6.7256*\dx},{1.0000*\dy})
	-- ({6.7356*\dx},{1.0000*\dy})
	-- ({6.7456*\dx},{1.0000*\dy})
	-- ({6.7556*\dx},{1.0000*\dy})
	-- ({6.7656*\dx},{1.0000*\dy})
	-- ({6.7756*\dx},{1.0000*\dy})
	-- ({6.7857*\dx},{1.0000*\dy})
	-- ({6.7957*\dx},{1.0000*\dy})
	-- ({6.8057*\dx},{1.0000*\dy})
	-- ({6.8157*\dx},{1.0000*\dy})
	-- ({6.8257*\dx},{1.0000*\dy})
	-- ({6.8357*\dx},{1.0000*\dy})
	-- ({6.8457*\dx},{1.0000*\dy})
	-- ({6.8557*\dx},{1.0000*\dy})
	-- ({6.8657*\dx},{1.0000*\dy})
	-- ({6.8757*\dx},{1.0000*\dy})
	-- ({6.8857*\dx},{1.0000*\dy})
	-- ({6.8957*\dx},{1.0000*\dy})
	-- ({6.9058*\dx},{1.0000*\dy})
	-- ({6.9158*\dx},{1.0000*\dy})
	-- ({6.9258*\dx},{1.0000*\dy})
	-- ({6.9358*\dx},{1.0000*\dy})
	-- ({6.9458*\dx},{1.0000*\dy})
	-- ({6.9558*\dx},{1.0000*\dy})
	-- ({6.9658*\dx},{1.0000*\dy})
	-- ({6.9758*\dx},{1.0000*\dy})
	-- ({6.9858*\dx},{1.0000*\dy})
	-- ({6.9958*\dx},{1.0000*\dy})
	-- ({7.0058*\dx},{1.0000*\dy})
	-- ({7.0158*\dx},{1.0000*\dy})
	-- ({7.0259*\dx},{1.0000*\dy})
	-- ({7.0359*\dx},{1.0000*\dy})
	-- ({7.0459*\dx},{1.0000*\dy})
	-- ({7.0559*\dx},{1.0000*\dy})
	-- ({7.0659*\dx},{1.0000*\dy})
	-- ({7.0759*\dx},{1.0000*\dy})
	-- ({7.0859*\dx},{1.0000*\dy})
	-- ({7.0959*\dx},{1.0000*\dy})
	-- ({7.1059*\dx},{1.0000*\dy})
	-- ({7.1159*\dx},{1.0000*\dy})
	-- ({7.1259*\dx},{1.0000*\dy})
	-- ({7.1359*\dx},{1.0000*\dy})
	-- ({7.1460*\dx},{1.0000*\dy})
	-- ({7.1560*\dx},{1.0000*\dy})
	-- ({7.1660*\dx},{1.0000*\dy})
	-- ({7.1760*\dx},{1.0000*\dy})
	-- ({7.1860*\dx},{1.0000*\dy})
	-- ({7.1960*\dx},{1.0000*\dy})
	-- ({7.2060*\dx},{1.0000*\dy})
	-- ({7.2160*\dx},{1.0000*\dy})
	-- ({7.2260*\dx},{1.0000*\dy})
	-- ({7.2360*\dx},{1.0000*\dy})
	-- ({7.2460*\dx},{1.0000*\dy})
	-- ({7.2560*\dx},{1.0000*\dy})
	-- ({7.2661*\dx},{1.0000*\dy})
	-- ({7.2761*\dx},{1.0000*\dy})
	-- ({7.2861*\dx},{1.0000*\dy})
	-- ({7.2961*\dx},{1.0000*\dy})
	-- ({7.3061*\dx},{1.0000*\dy})
	-- ({7.3161*\dx},{1.0000*\dy})
	-- ({7.3261*\dx},{1.0000*\dy})
	-- ({7.3361*\dx},{1.0000*\dy})
	-- ({7.3461*\dx},{1.0000*\dy})
	-- ({7.3561*\dx},{1.0000*\dy})
	-- ({7.3661*\dx},{1.0000*\dy})
	-- ({7.3761*\dx},{1.0000*\dy})
	-- ({7.3862*\dx},{1.0000*\dy})
	-- ({7.3962*\dx},{1.0000*\dy})
	-- ({7.4062*\dx},{1.0000*\dy})
	-- ({7.4162*\dx},{1.0000*\dy})
	-- ({7.4262*\dx},{1.0000*\dy})
	-- ({7.4362*\dx},{1.0000*\dy})
	-- ({7.4462*\dx},{1.0000*\dy})
	-- ({7.4562*\dx},{1.0000*\dy})
	-- ({7.4662*\dx},{1.0000*\dy})
	-- ({7.4762*\dx},{1.0000*\dy})
	-- ({7.4862*\dx},{1.0000*\dy})
	-- ({7.4962*\dx},{1.0000*\dy})
	-- ({7.5063*\dx},{1.0000*\dy})
	-- ({7.5163*\dx},{1.0000*\dy})
	-- ({7.5263*\dx},{1.0000*\dy})
	-- ({7.5363*\dx},{1.0000*\dy})
	-- ({7.5463*\dx},{1.0000*\dy})
	-- ({7.5563*\dx},{1.0000*\dy})
	-- ({7.5663*\dx},{1.0000*\dy})
	-- ({7.5763*\dx},{1.0000*\dy})
	-- ({7.5863*\dx},{1.0000*\dy})
	-- ({7.5963*\dx},{1.0000*\dy})
	-- ({7.6063*\dx},{1.0000*\dy})
	-- ({7.6163*\dx},{1.0000*\dy})
	-- ({7.6264*\dx},{1.0000*\dy})
	-- ({7.6364*\dx},{1.0000*\dy})
	-- ({7.6464*\dx},{1.0000*\dy})
	-- ({7.6564*\dx},{1.0000*\dy})
	-- ({7.6664*\dx},{1.0000*\dy})
	-- ({7.6764*\dx},{1.0000*\dy})
	-- ({7.6864*\dx},{1.0000*\dy})
	-- ({7.6964*\dx},{1.0000*\dy})
	-- ({7.7064*\dx},{1.0000*\dy})
	-- ({7.7164*\dx},{1.0000*\dy})
	-- ({7.7264*\dx},{1.0000*\dy})
	-- ({7.7364*\dx},{1.0000*\dy})
	-- ({7.7465*\dx},{1.0000*\dy})
	-- ({7.7565*\dx},{1.0000*\dy})
	-- ({7.7665*\dx},{1.0000*\dy})
	-- ({7.7765*\dx},{1.0000*\dy})
	-- ({7.7865*\dx},{1.0000*\dy})
	-- ({7.7965*\dx},{1.0000*\dy})
	-- ({7.8065*\dx},{1.0000*\dy})
	-- ({7.8165*\dx},{1.0000*\dy})
	-- ({7.8265*\dx},{1.0000*\dy})
	-- ({7.8365*\dx},{1.0000*\dy})
	-- ({7.8465*\dx},{1.0000*\dy})
	-- ({7.8565*\dx},{1.0000*\dy})
	-- ({7.8666*\dx},{1.0000*\dy})
	-- ({7.8766*\dx},{1.0000*\dy})
	-- ({7.8866*\dx},{1.0000*\dy})
	-- ({7.8966*\dx},{1.0000*\dy})
	-- ({7.9066*\dx},{1.0000*\dy})
	-- ({7.9166*\dx},{1.0000*\dy})
	-- ({7.9266*\dx},{1.0000*\dy})
	-- ({7.9366*\dx},{1.0000*\dy})
	-- ({7.9466*\dx},{1.0000*\dy})
	-- ({7.9566*\dx},{1.0000*\dy})
	-- ({7.9666*\dx},{1.0000*\dy})
	-- ({7.9766*\dx},{1.0000*\dy})
	-- ({7.9867*\dx},{1.0000*\dy})
	-- ({7.9967*\dx},{1.0000*\dy})
	-- ({8.0067*\dx},{1.0000*\dy})
	-- ({8.0167*\dx},{0.9997*\dy})
	-- ({8.0267*\dx},{0.9993*\dy})
	-- ({8.0367*\dx},{0.9987*\dy})
	-- ({8.0467*\dx},{0.9979*\dy})
	-- ({8.0567*\dx},{0.9968*\dy})
	-- ({8.0667*\dx},{0.9956*\dy})
	-- ({8.0767*\dx},{0.9941*\dy})
	-- ({8.0867*\dx},{0.9925*\dy})
	-- ({8.0967*\dx},{0.9906*\dy})
	-- ({8.1068*\dx},{0.9884*\dy})
	-- ({8.1168*\dx},{0.9861*\dy})
	-- ({8.1268*\dx},{0.9836*\dy})
	-- ({8.1368*\dx},{0.9808*\dy})
	-- ({8.1468*\dx},{0.9778*\dy})
	-- ({8.1568*\dx},{0.9745*\dy})
	-- ({8.1668*\dx},{0.9710*\dy})
	-- ({8.1768*\dx},{0.9674*\dy})
	-- ({8.1868*\dx},{0.9634*\dy})
	-- ({8.1968*\dx},{0.9593*\dy})
	-- ({8.2068*\dx},{0.9549*\dy})
	-- ({8.2168*\dx},{0.9503*\dy})
	-- ({8.2269*\dx},{0.9455*\dy})
	-- ({8.2369*\dx},{0.9404*\dy})
	-- ({8.2469*\dx},{0.9352*\dy})
	-- ({8.2569*\dx},{0.9297*\dy})
	-- ({8.2669*\dx},{0.9240*\dy})
	-- ({8.2769*\dx},{0.9181*\dy})
	-- ({8.2869*\dx},{0.9120*\dy})
	-- ({8.2969*\dx},{0.9057*\dy})
	-- ({8.3069*\dx},{0.8992*\dy})
	-- ({8.3169*\dx},{0.8925*\dy})
	-- ({8.3269*\dx},{0.8856*\dy})
	-- ({8.3369*\dx},{0.8785*\dy})
	-- ({8.3470*\dx},{0.8713*\dy})
	-- ({8.3570*\dx},{0.8639*\dy})
	-- ({8.3670*\dx},{0.8563*\dy})
	-- ({8.3770*\dx},{0.8485*\dy})
	-- ({8.3870*\dx},{0.8406*\dy})
	-- ({8.3970*\dx},{0.8326*\dy})
	-- ({8.4070*\dx},{0.8244*\dy})
	-- ({8.4170*\dx},{0.8160*\dy})
	-- ({8.4270*\dx},{0.8076*\dy})
	-- ({8.4370*\dx},{0.7990*\dy})
	-- ({8.4470*\dx},{0.7903*\dy})
	-- ({8.4570*\dx},{0.7815*\dy})
	-- ({8.4671*\dx},{0.7726*\dy})
	-- ({8.4771*\dx},{0.7636*\dy})
	-- ({8.4871*\dx},{0.7545*\dy})
	-- ({8.4971*\dx},{0.7454*\dy})
	-- ({8.5071*\dx},{0.7361*\dy})
	-- ({8.5171*\dx},{0.7268*\dy})
	-- ({8.5271*\dx},{0.7175*\dy})
	-- ({8.5371*\dx},{0.7081*\dy})
	-- ({8.5471*\dx},{0.6986*\dy})
	-- ({8.5571*\dx},{0.6892*\dy})
	-- ({8.5671*\dx},{0.6796*\dy})
	-- ({8.5771*\dx},{0.6701*\dy})
	-- ({8.5872*\dx},{0.6606*\dy})
	-- ({8.5972*\dx},{0.6510*\dy})
	-- ({8.6072*\dx},{0.6415*\dy})
	-- ({8.6172*\dx},{0.6319*\dy})
	-- ({8.6272*\dx},{0.6224*\dy})
	-- ({8.6372*\dx},{0.6129*\dy})
	-- ({8.6472*\dx},{0.6034*\dy})
	-- ({8.6572*\dx},{0.5939*\dy})
	-- ({8.6672*\dx},{0.5845*\dy})
	-- ({8.6772*\dx},{0.5751*\dy})
	-- ({8.6872*\dx},{0.5657*\dy})
	-- ({8.6972*\dx},{0.5564*\dy})
	-- ({8.7073*\dx},{0.5472*\dy})
	-- ({8.7173*\dx},{0.5380*\dy})
	-- ({8.7273*\dx},{0.5289*\dy})
	-- ({8.7373*\dx},{0.5198*\dy})
	-- ({8.7473*\dx},{0.5108*\dy})
	-- ({8.7573*\dx},{0.5019*\dy})
	-- ({8.7673*\dx},{0.4930*\dy})
	-- ({8.7773*\dx},{0.4843*\dy})
	-- ({8.7873*\dx},{0.4756*\dy})
	-- ({8.7973*\dx},{0.4670*\dy})
	-- ({8.8073*\dx},{0.4585*\dy})
	-- ({8.8173*\dx},{0.4501*\dy})
	-- ({8.8274*\dx},{0.4418*\dy})
	-- ({8.8374*\dx},{0.4336*\dy})
	-- ({8.8474*\dx},{0.4255*\dy})
	-- ({8.8574*\dx},{0.4175*\dy})
	-- ({8.8674*\dx},{0.4096*\dy})
	-- ({8.8774*\dx},{0.4018*\dy})
	-- ({8.8874*\dx},{0.3941*\dy})
	-- ({8.8974*\dx},{0.3865*\dy})
	-- ({8.9074*\dx},{0.3791*\dy})
	-- ({8.9174*\dx},{0.3717*\dy})
	-- ({8.9274*\dx},{0.3645*\dy})
	-- ({8.9374*\dx},{0.3573*\dy})
	-- ({8.9475*\dx},{0.3503*\dy})
	-- ({8.9575*\dx},{0.3434*\dy})
	-- ({8.9675*\dx},{0.3366*\dy})
	-- ({8.9775*\dx},{0.3300*\dy})
	-- ({8.9875*\dx},{0.3234*\dy})
	-- ({8.9975*\dx},{0.3170*\dy})
	-- ({9.0075*\dx},{0.3106*\dy})
	-- ({9.0175*\dx},{0.3044*\dy})
	-- ({9.0275*\dx},{0.2984*\dy})
	-- ({9.0375*\dx},{0.2924*\dy})
	-- ({9.0475*\dx},{0.2865*\dy})
	-- ({9.0575*\dx},{0.2808*\dy})
	-- ({9.0676*\dx},{0.2752*\dy})
	-- ({9.0776*\dx},{0.2697*\dy})
	-- ({9.0876*\dx},{0.2643*\dy})
	-- ({9.0976*\dx},{0.2590*\dy})
	-- ({9.1076*\dx},{0.2539*\dy})
	-- ({9.1176*\dx},{0.2488*\dy})
	-- ({9.1276*\dx},{0.2439*\dy})
	-- ({9.1376*\dx},{0.2391*\dy})
	-- ({9.1476*\dx},{0.2344*\dy})
	-- ({9.1576*\dx},{0.2299*\dy})
	-- ({9.1676*\dx},{0.2254*\dy})
	-- ({9.1776*\dx},{0.2211*\dy})
	-- ({9.1877*\dx},{0.2168*\dy})
	-- ({9.1977*\dx},{0.2127*\dy})
	-- ({9.2077*\dx},{0.2087*\dy})
	-- ({9.2177*\dx},{0.2048*\dy})
	-- ({9.2277*\dx},{0.2010*\dy})
	-- ({9.2377*\dx},{0.1974*\dy})
	-- ({9.2477*\dx},{0.1938*\dy})
	-- ({9.2577*\dx},{0.1904*\dy})
	-- ({9.2677*\dx},{0.1870*\dy})
	-- ({9.2777*\dx},{0.1838*\dy})
	-- ({9.2877*\dx},{0.1807*\dy})
	-- ({9.2977*\dx},{0.1777*\dy})
	-- ({9.3078*\dx},{0.1748*\dy})
	-- ({9.3178*\dx},{0.1720*\dy})
	-- ({9.3278*\dx},{0.1693*\dy})
	-- ({9.3378*\dx},{0.1667*\dy})
	-- ({9.3478*\dx},{0.1642*\dy})
	-- ({9.3578*\dx},{0.1619*\dy})
	-- ({9.3678*\dx},{0.1596*\dy})
	-- ({9.3778*\dx},{0.1574*\dy})
	-- ({9.3878*\dx},{0.1554*\dy})
	-- ({9.3978*\dx},{0.1535*\dy})
	-- ({9.4078*\dx},{0.1516*\dy})
	-- ({9.4178*\dx},{0.1499*\dy})
	-- ({9.4279*\dx},{0.1483*\dy})
	-- ({9.4379*\dx},{0.1468*\dy})
	-- ({9.4479*\dx},{0.1454*\dy})
	-- ({9.4579*\dx},{0.1441*\dy})
	-- ({9.4679*\dx},{0.1429*\dy})
	-- ({9.4779*\dx},{0.1418*\dy})
	-- ({9.4879*\dx},{0.1408*\dy})
	-- ({9.4979*\dx},{0.1399*\dy})
	-- ({9.5079*\dx},{0.1392*\dy})
	-- ({9.5179*\dx},{0.1385*\dy})
	-- ({9.5279*\dx},{0.1380*\dy})
	-- ({9.5379*\dx},{0.1375*\dy})
	-- ({9.5480*\dx},{0.1372*\dy})
	-- ({9.5580*\dx},{0.1370*\dy})
	-- ({9.5680*\dx},{0.1369*\dy})
	-- ({9.5780*\dx},{0.1369*\dy})
	-- ({9.5880*\dx},{0.1370*\dy})
	-- ({9.5980*\dx},{0.1372*\dy})
	-- ({9.6080*\dx},{0.1375*\dy})
	-- ({9.6180*\dx},{0.1380*\dy})
	-- ({9.6280*\dx},{0.1385*\dy})
	-- ({9.6380*\dx},{0.1392*\dy})
	-- ({9.6480*\dx},{0.1400*\dy})
	-- ({9.6580*\dx},{0.1409*\dy})
	-- ({9.6681*\dx},{0.1419*\dy})
	-- ({9.6781*\dx},{0.1431*\dy})
	-- ({9.6881*\dx},{0.1443*\dy})
	-- ({9.6981*\dx},{0.1457*\dy})
	-- ({9.7081*\dx},{0.1472*\dy})
	-- ({9.7181*\dx},{0.1488*\dy})
	-- ({9.7281*\dx},{0.1506*\dy})
	-- ({9.7381*\dx},{0.1524*\dy})
	-- ({9.7481*\dx},{0.1544*\dy})
	-- ({9.7581*\dx},{0.1566*\dy})
	-- ({9.7681*\dx},{0.1588*\dy})
	-- ({9.7781*\dx},{0.1612*\dy})
	-- ({9.7882*\dx},{0.1637*\dy})
	-- ({9.7982*\dx},{0.1664*\dy})
	-- ({9.8082*\dx},{0.1691*\dy})
	-- ({9.8182*\dx},{0.1720*\dy})
	-- ({9.8282*\dx},{0.1751*\dy})
	-- ({9.8382*\dx},{0.1783*\dy})
	-- ({9.8482*\dx},{0.1816*\dy})
	-- ({9.8582*\dx},{0.1851*\dy})
	-- ({9.8682*\dx},{0.1887*\dy})
	-- ({9.8782*\dx},{0.1924*\dy})
	-- ({9.8882*\dx},{0.1963*\dy})
	-- ({9.8982*\dx},{0.2004*\dy})
	-- ({9.9083*\dx},{0.2045*\dy})
	-- ({9.9183*\dx},{0.2089*\dy})
	-- ({9.9283*\dx},{0.2134*\dy})
	-- ({9.9383*\dx},{0.2180*\dy})
	-- ({9.9483*\dx},{0.2228*\dy})
	-- ({9.9583*\dx},{0.2277*\dy})
	-- ({9.9683*\dx},{0.2328*\dy})
	-- ({9.9783*\dx},{0.2381*\dy})
	-- ({9.9883*\dx},{0.2435*\dy})
	-- ({9.9983*\dx},{0.2491*\dy})
	-- ({10.0083*\dx},{0.2548*\dy})
	-- ({10.0183*\dx},{0.2606*\dy})
	-- ({10.0284*\dx},{0.2665*\dy})
	-- ({10.0384*\dx},{0.2725*\dy})
	-- ({10.0484*\dx},{0.2786*\dy})
	-- ({10.0584*\dx},{0.2848*\dy})
	-- ({10.0684*\dx},{0.2912*\dy})
	-- ({10.0784*\dx},{0.2976*\dy})
	-- ({10.0884*\dx},{0.3041*\dy})
	-- ({10.0984*\dx},{0.3108*\dy})
	-- ({10.1084*\dx},{0.3176*\dy})
	-- ({10.1184*\dx},{0.3245*\dy})
	-- ({10.1284*\dx},{0.3315*\dy})
	-- ({10.1384*\dx},{0.3386*\dy})
	-- ({10.1485*\dx},{0.3458*\dy})
	-- ({10.1585*\dx},{0.3531*\dy})
	-- ({10.1685*\dx},{0.3606*\dy})
	-- ({10.1785*\dx},{0.3682*\dy})
	-- ({10.1885*\dx},{0.3759*\dy})
	-- ({10.1985*\dx},{0.3837*\dy})
	-- ({10.2085*\dx},{0.3916*\dy})
	-- ({10.2185*\dx},{0.3996*\dy})
	-- ({10.2285*\dx},{0.4078*\dy})
	-- ({10.2385*\dx},{0.4161*\dy})
	-- ({10.2485*\dx},{0.4244*\dy})
	-- ({10.2585*\dx},{0.4329*\dy})
	-- ({10.2686*\dx},{0.4416*\dy})
	-- ({10.2786*\dx},{0.4503*\dy})
	-- ({10.2886*\dx},{0.4591*\dy})
	-- ({10.2986*\dx},{0.4681*\dy})
	-- ({10.3086*\dx},{0.4772*\dy})
	-- ({10.3186*\dx},{0.4863*\dy})
	-- ({10.3286*\dx},{0.4956*\dy})
	-- ({10.3386*\dx},{0.5050*\dy})
	-- ({10.3486*\dx},{0.5144*\dy})
	-- ({10.3586*\dx},{0.5240*\dy})
	-- ({10.3686*\dx},{0.5336*\dy})
	-- ({10.3786*\dx},{0.5434*\dy})
	-- ({10.3887*\dx},{0.5532*\dy})
	-- ({10.3987*\dx},{0.5631*\dy})
	-- ({10.4087*\dx},{0.5731*\dy})
	-- ({10.4187*\dx},{0.5832*\dy})
	-- ({10.4287*\dx},{0.5933*\dy})
	-- ({10.4387*\dx},{0.6034*\dy})
	-- ({10.4487*\dx},{0.6137*\dy})
	-- ({10.4587*\dx},{0.6239*\dy})
	-- ({10.4687*\dx},{0.6343*\dy})
	-- ({10.4787*\dx},{0.6446*\dy})
	-- ({10.4887*\dx},{0.6550*\dy})
	-- ({10.4987*\dx},{0.6654*\dy})
	-- ({10.5088*\dx},{0.6758*\dy})
	-- ({10.5188*\dx},{0.6862*\dy})
	-- ({10.5288*\dx},{0.6966*\dy})
	-- ({10.5388*\dx},{0.7069*\dy})
	-- ({10.5488*\dx},{0.7173*\dy})
	-- ({10.5588*\dx},{0.7276*\dy})
	-- ({10.5688*\dx},{0.7379*\dy})
	-- ({10.5788*\dx},{0.7481*\dy})
	-- ({10.5888*\dx},{0.7582*\dy})
	-- ({10.5988*\dx},{0.7683*\dy})
	-- ({10.6088*\dx},{0.7783*\dy})
	-- ({10.6188*\dx},{0.7882*\dy})
	-- ({10.6289*\dx},{0.7980*\dy})
	-- ({10.6389*\dx},{0.8077*\dy})
	-- ({10.6489*\dx},{0.8172*\dy})
	-- ({10.6589*\dx},{0.8266*\dy})
	-- ({10.6689*\dx},{0.8358*\dy})
	-- ({10.6789*\dx},{0.8449*\dy})
	-- ({10.6889*\dx},{0.8538*\dy})
	-- ({10.6989*\dx},{0.8625*\dy})
	-- ({10.7089*\dx},{0.8711*\dy})
	-- ({10.7189*\dx},{0.8794*\dy})
	-- ({10.7289*\dx},{0.8875*\dy})
	-- ({10.7389*\dx},{0.8954*\dy})
	-- ({10.7490*\dx},{0.9030*\dy})
	-- ({10.7590*\dx},{0.9104*\dy})
	-- ({10.7690*\dx},{0.9176*\dy})
	-- ({10.7790*\dx},{0.9245*\dy})
	-- ({10.7890*\dx},{0.9311*\dy})
	-- ({10.7990*\dx},{0.9374*\dy})
	-- ({10.8090*\dx},{0.9435*\dy})
	-- ({10.8190*\dx},{0.9493*\dy})
	-- ({10.8290*\dx},{0.9547*\dy})
	-- ({10.8390*\dx},{0.9599*\dy})
	-- ({10.8490*\dx},{0.9648*\dy})
	-- ({10.8590*\dx},{0.9693*\dy})
	-- ({10.8691*\dx},{0.9736*\dy})
	-- ({10.8791*\dx},{0.9775*\dy})
	-- ({10.8891*\dx},{0.9811*\dy})
	-- ({10.8991*\dx},{0.9844*\dy})
	-- ({10.9091*\dx},{0.9874*\dy})
	-- ({10.9191*\dx},{0.9901*\dy})
	-- ({10.9291*\dx},{0.9924*\dy})
	-- ({10.9391*\dx},{0.9944*\dy})
	-- ({10.9491*\dx},{0.9961*\dy})
	-- ({10.9591*\dx},{0.9975*\dy})
	-- ({10.9691*\dx},{0.9986*\dy})
	-- ({10.9791*\dx},{0.9994*\dy})
	-- ({10.9892*\dx},{0.9998*\dy})
	-- ({10.9992*\dx},{1.0000*\dy})
	-- ({11.0092*\dx},{1.0000*\dy})
	-- ({11.0192*\dx},{1.0000*\dy})
	-- ({11.0292*\dx},{1.0000*\dy})
	-- ({11.0392*\dx},{1.0000*\dy})
	-- ({11.0492*\dx},{1.0000*\dy})
	-- ({11.0592*\dx},{1.0000*\dy})
	-- ({11.0692*\dx},{1.0000*\dy})
	-- ({11.0792*\dx},{1.0000*\dy})
	-- ({11.0892*\dx},{1.0000*\dy})
	-- ({11.0992*\dx},{1.0000*\dy})
	-- ({11.1093*\dx},{1.0000*\dy})
	-- ({11.1193*\dx},{1.0000*\dy})
	-- ({11.1293*\dx},{1.0000*\dy})
	-- ({11.1393*\dx},{1.0000*\dy})
	-- ({11.1493*\dx},{1.0000*\dy})
	-- ({11.1593*\dx},{1.0000*\dy})
	-- ({11.1693*\dx},{1.0000*\dy})
	-- ({11.1793*\dx},{1.0000*\dy})
	-- ({11.1893*\dx},{1.0000*\dy})
	-- ({11.1993*\dx},{1.0000*\dy})
	-- ({11.2093*\dx},{1.0000*\dy})
	-- ({11.2193*\dx},{1.0000*\dy})
	-- ({11.2294*\dx},{1.0000*\dy})
	-- ({11.2394*\dx},{1.0000*\dy})
	-- ({11.2494*\dx},{1.0000*\dy})
	-- ({11.2594*\dx},{1.0000*\dy})
	-- ({11.2694*\dx},{1.0000*\dy})
	-- ({11.2794*\dx},{1.0000*\dy})
	-- ({11.2894*\dx},{1.0000*\dy})
	-- ({11.2994*\dx},{1.0000*\dy})
	-- ({11.3094*\dx},{1.0000*\dy})
	-- ({11.3194*\dx},{1.0000*\dy})
	-- ({11.3294*\dx},{1.0000*\dy})
	-- ({11.3394*\dx},{1.0000*\dy})
	-- ({11.3495*\dx},{1.0000*\dy})
	-- ({11.3595*\dx},{1.0000*\dy})
	-- ({11.3695*\dx},{1.0000*\dy})
	-- ({11.3795*\dx},{1.0000*\dy})
	-- ({11.3895*\dx},{1.0000*\dy})
	-- ({11.3995*\dx},{1.0000*\dy})
	-- ({11.4095*\dx},{1.0000*\dy})
	-- ({11.4195*\dx},{1.0000*\dy})
	-- ({11.4295*\dx},{1.0000*\dy})
	-- ({11.4395*\dx},{1.0000*\dy})
	-- ({11.4495*\dx},{1.0000*\dy})
	-- ({11.4595*\dx},{1.0000*\dy})
	-- ({11.4696*\dx},{1.0000*\dy})
	-- ({11.4796*\dx},{1.0000*\dy})
	-- ({11.4896*\dx},{1.0000*\dy})
	-- ({11.4996*\dx},{1.0000*\dy})
	-- ({11.5096*\dx},{1.0000*\dy})
	-- ({11.5196*\dx},{1.0000*\dy})
	-- ({11.5296*\dx},{1.0000*\dy})
	-- ({11.5396*\dx},{1.0000*\dy})
	-- ({11.5496*\dx},{1.0000*\dy})
	-- ({11.5596*\dx},{1.0000*\dy})
	-- ({11.5696*\dx},{1.0000*\dy})
	-- ({11.5796*\dx},{1.0000*\dy})
	-- ({11.5897*\dx},{1.0000*\dy})
	-- ({11.5997*\dx},{1.0000*\dy})
	-- ({11.6097*\dx},{1.0000*\dy})
	-- ({11.6197*\dx},{1.0000*\dy})
	-- ({11.6297*\dx},{1.0000*\dy})
	-- ({11.6397*\dx},{1.0000*\dy})
	-- ({11.6497*\dx},{1.0000*\dy})
	-- ({11.6597*\dx},{1.0000*\dy})
	-- ({11.6697*\dx},{1.0000*\dy})
	-- ({11.6797*\dx},{1.0000*\dy})
	-- ({11.6897*\dx},{1.0000*\dy})
	-- ({11.6997*\dx},{1.0000*\dy})
	-- ({11.7098*\dx},{1.0000*\dy})
	-- ({11.7198*\dx},{1.0000*\dy})
	-- ({11.7298*\dx},{1.0000*\dy})
	-- ({11.7398*\dx},{1.0000*\dy})
	-- ({11.7498*\dx},{1.0000*\dy})
	-- ({11.7598*\dx},{1.0000*\dy})
	-- ({11.7698*\dx},{1.0000*\dy})
	-- ({11.7798*\dx},{1.0000*\dy})
	-- ({11.7898*\dx},{1.0000*\dy})
	-- ({11.7998*\dx},{1.0000*\dy})
	-- ({11.8098*\dx},{1.0000*\dy})
	-- ({11.8198*\dx},{1.0000*\dy})
	-- ({11.8299*\dx},{1.0000*\dy})
	-- ({11.8399*\dx},{1.0000*\dy})
	-- ({11.8499*\dx},{1.0000*\dy})
	-- ({11.8599*\dx},{1.0000*\dy})
	-- ({11.8699*\dx},{1.0000*\dy})
	-- ({11.8799*\dx},{1.0000*\dy})
	-- ({11.8899*\dx},{1.0000*\dy})
	-- ({11.8999*\dx},{1.0000*\dy})
	-- ({11.9099*\dx},{1.0000*\dy})
	-- ({11.9199*\dx},{1.0000*\dy})
	-- ({11.9299*\dx},{1.0000*\dy})
	-- ({11.9399*\dx},{1.0000*\dy})
	-- ({11.9500*\dx},{1.0000*\dy})
	-- ({11.9600*\dx},{1.0000*\dy})
	-- ({11.9700*\dx},{1.0000*\dy})
	-- ({11.9800*\dx},{1.0000*\dy})
	-- ({11.9900*\dx},{1.0000*\dy})
	-- ({12.0000*\dx},{1.0000*\dy})
}
\def\cpsisix{
	({0.0000*\dx},{1.0000*\dy})
	-- ({0.0100*\dx},{1.0000*\dy})
	-- ({0.0200*\dx},{1.0000*\dy})
	-- ({0.0300*\dx},{1.0000*\dy})
	-- ({0.0400*\dx},{1.0000*\dy})
	-- ({0.0500*\dx},{1.0000*\dy})
	-- ({0.0601*\dx},{1.0000*\dy})
	-- ({0.0701*\dx},{1.0000*\dy})
	-- ({0.0801*\dx},{1.0000*\dy})
	-- ({0.0901*\dx},{1.0000*\dy})
	-- ({0.1001*\dx},{1.0000*\dy})
	-- ({0.1101*\dx},{1.0000*\dy})
	-- ({0.1201*\dx},{1.0000*\dy})
	-- ({0.1301*\dx},{1.0000*\dy})
	-- ({0.1401*\dx},{1.0000*\dy})
	-- ({0.1501*\dx},{1.0000*\dy})
	-- ({0.1601*\dx},{1.0000*\dy})
	-- ({0.1701*\dx},{1.0000*\dy})
	-- ({0.1802*\dx},{1.0000*\dy})
	-- ({0.1902*\dx},{1.0000*\dy})
	-- ({0.2002*\dx},{1.0000*\dy})
	-- ({0.2102*\dx},{1.0000*\dy})
	-- ({0.2202*\dx},{1.0000*\dy})
	-- ({0.2302*\dx},{1.0000*\dy})
	-- ({0.2402*\dx},{1.0000*\dy})
	-- ({0.2502*\dx},{1.0000*\dy})
	-- ({0.2602*\dx},{1.0000*\dy})
	-- ({0.2702*\dx},{1.0000*\dy})
	-- ({0.2802*\dx},{1.0000*\dy})
	-- ({0.2902*\dx},{1.0000*\dy})
	-- ({0.3003*\dx},{1.0000*\dy})
	-- ({0.3103*\dx},{1.0000*\dy})
	-- ({0.3203*\dx},{1.0000*\dy})
	-- ({0.3303*\dx},{1.0000*\dy})
	-- ({0.3403*\dx},{1.0000*\dy})
	-- ({0.3503*\dx},{1.0000*\dy})
	-- ({0.3603*\dx},{1.0000*\dy})
	-- ({0.3703*\dx},{1.0000*\dy})
	-- ({0.3803*\dx},{1.0000*\dy})
	-- ({0.3903*\dx},{1.0000*\dy})
	-- ({0.4003*\dx},{1.0000*\dy})
	-- ({0.4103*\dx},{1.0000*\dy})
	-- ({0.4204*\dx},{1.0000*\dy})
	-- ({0.4304*\dx},{1.0000*\dy})
	-- ({0.4404*\dx},{1.0000*\dy})
	-- ({0.4504*\dx},{1.0000*\dy})
	-- ({0.4604*\dx},{1.0000*\dy})
	-- ({0.4704*\dx},{1.0000*\dy})
	-- ({0.4804*\dx},{1.0000*\dy})
	-- ({0.4904*\dx},{1.0000*\dy})
	-- ({0.5004*\dx},{1.0000*\dy})
	-- ({0.5104*\dx},{1.0000*\dy})
	-- ({0.5204*\dx},{1.0000*\dy})
	-- ({0.5304*\dx},{1.0000*\dy})
	-- ({0.5405*\dx},{1.0000*\dy})
	-- ({0.5505*\dx},{1.0000*\dy})
	-- ({0.5605*\dx},{1.0000*\dy})
	-- ({0.5705*\dx},{1.0000*\dy})
	-- ({0.5805*\dx},{1.0000*\dy})
	-- ({0.5905*\dx},{1.0000*\dy})
	-- ({0.6005*\dx},{1.0000*\dy})
	-- ({0.6105*\dx},{1.0000*\dy})
	-- ({0.6205*\dx},{1.0000*\dy})
	-- ({0.6305*\dx},{1.0000*\dy})
	-- ({0.6405*\dx},{1.0000*\dy})
	-- ({0.6505*\dx},{1.0000*\dy})
	-- ({0.6606*\dx},{1.0000*\dy})
	-- ({0.6706*\dx},{1.0000*\dy})
	-- ({0.6806*\dx},{1.0000*\dy})
	-- ({0.6906*\dx},{1.0000*\dy})
	-- ({0.7006*\dx},{1.0000*\dy})
	-- ({0.7106*\dx},{1.0000*\dy})
	-- ({0.7206*\dx},{1.0000*\dy})
	-- ({0.7306*\dx},{1.0000*\dy})
	-- ({0.7406*\dx},{1.0000*\dy})
	-- ({0.7506*\dx},{1.0000*\dy})
	-- ({0.7606*\dx},{1.0000*\dy})
	-- ({0.7706*\dx},{1.0000*\dy})
	-- ({0.7807*\dx},{1.0000*\dy})
	-- ({0.7907*\dx},{1.0000*\dy})
	-- ({0.8007*\dx},{1.0000*\dy})
	-- ({0.8107*\dx},{1.0000*\dy})
	-- ({0.8207*\dx},{1.0000*\dy})
	-- ({0.8307*\dx},{1.0000*\dy})
	-- ({0.8407*\dx},{1.0000*\dy})
	-- ({0.8507*\dx},{1.0000*\dy})
	-- ({0.8607*\dx},{1.0000*\dy})
	-- ({0.8707*\dx},{1.0000*\dy})
	-- ({0.8807*\dx},{1.0000*\dy})
	-- ({0.8907*\dx},{1.0000*\dy})
	-- ({0.9008*\dx},{1.0000*\dy})
	-- ({0.9108*\dx},{1.0000*\dy})
	-- ({0.9208*\dx},{1.0000*\dy})
	-- ({0.9308*\dx},{1.0000*\dy})
	-- ({0.9408*\dx},{1.0000*\dy})
	-- ({0.9508*\dx},{1.0000*\dy})
	-- ({0.9608*\dx},{1.0000*\dy})
	-- ({0.9708*\dx},{1.0000*\dy})
	-- ({0.9808*\dx},{1.0000*\dy})
	-- ({0.9908*\dx},{1.0000*\dy})
	-- ({1.0008*\dx},{1.0000*\dy})
	-- ({1.0108*\dx},{1.0000*\dy})
	-- ({1.0209*\dx},{1.0000*\dy})
	-- ({1.0309*\dx},{1.0000*\dy})
	-- ({1.0409*\dx},{1.0000*\dy})
	-- ({1.0509*\dx},{1.0000*\dy})
	-- ({1.0609*\dx},{1.0000*\dy})
	-- ({1.0709*\dx},{1.0000*\dy})
	-- ({1.0809*\dx},{1.0000*\dy})
	-- ({1.0909*\dx},{1.0000*\dy})
	-- ({1.1009*\dx},{1.0000*\dy})
	-- ({1.1109*\dx},{1.0000*\dy})
	-- ({1.1209*\dx},{1.0000*\dy})
	-- ({1.1309*\dx},{1.0000*\dy})
	-- ({1.1410*\dx},{1.0000*\dy})
	-- ({1.1510*\dx},{1.0000*\dy})
	-- ({1.1610*\dx},{1.0000*\dy})
	-- ({1.1710*\dx},{1.0000*\dy})
	-- ({1.1810*\dx},{1.0000*\dy})
	-- ({1.1910*\dx},{1.0000*\dy})
	-- ({1.2010*\dx},{1.0000*\dy})
	-- ({1.2110*\dx},{1.0000*\dy})
	-- ({1.2210*\dx},{1.0000*\dy})
	-- ({1.2310*\dx},{1.0000*\dy})
	-- ({1.2410*\dx},{1.0000*\dy})
	-- ({1.2510*\dx},{1.0000*\dy})
	-- ({1.2611*\dx},{1.0000*\dy})
	-- ({1.2711*\dx},{1.0000*\dy})
	-- ({1.2811*\dx},{1.0000*\dy})
	-- ({1.2911*\dx},{1.0000*\dy})
	-- ({1.3011*\dx},{1.0000*\dy})
	-- ({1.3111*\dx},{1.0000*\dy})
	-- ({1.3211*\dx},{1.0000*\dy})
	-- ({1.3311*\dx},{1.0000*\dy})
	-- ({1.3411*\dx},{1.0000*\dy})
	-- ({1.3511*\dx},{1.0000*\dy})
	-- ({1.3611*\dx},{1.0000*\dy})
	-- ({1.3711*\dx},{1.0000*\dy})
	-- ({1.3812*\dx},{1.0000*\dy})
	-- ({1.3912*\dx},{1.0000*\dy})
	-- ({1.4012*\dx},{1.0000*\dy})
	-- ({1.4112*\dx},{1.0000*\dy})
	-- ({1.4212*\dx},{1.0000*\dy})
	-- ({1.4312*\dx},{1.0000*\dy})
	-- ({1.4412*\dx},{1.0000*\dy})
	-- ({1.4512*\dx},{1.0000*\dy})
	-- ({1.4612*\dx},{1.0000*\dy})
	-- ({1.4712*\dx},{1.0000*\dy})
	-- ({1.4812*\dx},{1.0000*\dy})
	-- ({1.4912*\dx},{1.0000*\dy})
	-- ({1.5013*\dx},{1.0000*\dy})
	-- ({1.5113*\dx},{1.0000*\dy})
	-- ({1.5213*\dx},{1.0000*\dy})
	-- ({1.5313*\dx},{1.0000*\dy})
	-- ({1.5413*\dx},{1.0000*\dy})
	-- ({1.5513*\dx},{1.0000*\dy})
	-- ({1.5613*\dx},{1.0000*\dy})
	-- ({1.5713*\dx},{1.0000*\dy})
	-- ({1.5813*\dx},{1.0000*\dy})
	-- ({1.5913*\dx},{1.0000*\dy})
	-- ({1.6013*\dx},{1.0000*\dy})
	-- ({1.6113*\dx},{1.0000*\dy})
	-- ({1.6214*\dx},{1.0000*\dy})
	-- ({1.6314*\dx},{1.0000*\dy})
	-- ({1.6414*\dx},{1.0000*\dy})
	-- ({1.6514*\dx},{1.0000*\dy})
	-- ({1.6614*\dx},{1.0000*\dy})
	-- ({1.6714*\dx},{1.0000*\dy})
	-- ({1.6814*\dx},{1.0000*\dy})
	-- ({1.6914*\dx},{1.0000*\dy})
	-- ({1.7014*\dx},{1.0000*\dy})
	-- ({1.7114*\dx},{1.0000*\dy})
	-- ({1.7214*\dx},{1.0000*\dy})
	-- ({1.7314*\dx},{1.0000*\dy})
	-- ({1.7415*\dx},{1.0000*\dy})
	-- ({1.7515*\dx},{1.0000*\dy})
	-- ({1.7615*\dx},{1.0000*\dy})
	-- ({1.7715*\dx},{1.0000*\dy})
	-- ({1.7815*\dx},{1.0000*\dy})
	-- ({1.7915*\dx},{1.0000*\dy})
	-- ({1.8015*\dx},{1.0000*\dy})
	-- ({1.8115*\dx},{1.0000*\dy})
	-- ({1.8215*\dx},{1.0000*\dy})
	-- ({1.8315*\dx},{1.0000*\dy})
	-- ({1.8415*\dx},{1.0000*\dy})
	-- ({1.8515*\dx},{1.0000*\dy})
	-- ({1.8616*\dx},{1.0000*\dy})
	-- ({1.8716*\dx},{1.0000*\dy})
	-- ({1.8816*\dx},{1.0000*\dy})
	-- ({1.8916*\dx},{1.0000*\dy})
	-- ({1.9016*\dx},{1.0000*\dy})
	-- ({1.9116*\dx},{1.0000*\dy})
	-- ({1.9216*\dx},{1.0000*\dy})
	-- ({1.9316*\dx},{1.0000*\dy})
	-- ({1.9416*\dx},{1.0000*\dy})
	-- ({1.9516*\dx},{1.0000*\dy})
	-- ({1.9616*\dx},{1.0000*\dy})
	-- ({1.9716*\dx},{1.0000*\dy})
	-- ({1.9817*\dx},{1.0000*\dy})
	-- ({1.9917*\dx},{1.0000*\dy})
	-- ({2.0017*\dx},{1.0000*\dy})
	-- ({2.0117*\dx},{1.0000*\dy})
	-- ({2.0217*\dx},{1.0000*\dy})
	-- ({2.0317*\dx},{1.0000*\dy})
	-- ({2.0417*\dx},{1.0000*\dy})
	-- ({2.0517*\dx},{1.0000*\dy})
	-- ({2.0617*\dx},{1.0000*\dy})
	-- ({2.0717*\dx},{1.0000*\dy})
	-- ({2.0817*\dx},{1.0000*\dy})
	-- ({2.0917*\dx},{1.0000*\dy})
	-- ({2.1018*\dx},{1.0000*\dy})
	-- ({2.1118*\dx},{1.0000*\dy})
	-- ({2.1218*\dx},{1.0000*\dy})
	-- ({2.1318*\dx},{1.0000*\dy})
	-- ({2.1418*\dx},{1.0000*\dy})
	-- ({2.1518*\dx},{1.0000*\dy})
	-- ({2.1618*\dx},{1.0000*\dy})
	-- ({2.1718*\dx},{1.0000*\dy})
	-- ({2.1818*\dx},{1.0000*\dy})
	-- ({2.1918*\dx},{1.0000*\dy})
	-- ({2.2018*\dx},{1.0000*\dy})
	-- ({2.2118*\dx},{1.0000*\dy})
	-- ({2.2219*\dx},{1.0000*\dy})
	-- ({2.2319*\dx},{1.0000*\dy})
	-- ({2.2419*\dx},{1.0000*\dy})
	-- ({2.2519*\dx},{1.0000*\dy})
	-- ({2.2619*\dx},{1.0000*\dy})
	-- ({2.2719*\dx},{1.0000*\dy})
	-- ({2.2819*\dx},{1.0000*\dy})
	-- ({2.2919*\dx},{1.0000*\dy})
	-- ({2.3019*\dx},{1.0000*\dy})
	-- ({2.3119*\dx},{1.0000*\dy})
	-- ({2.3219*\dx},{1.0000*\dy})
	-- ({2.3319*\dx},{1.0000*\dy})
	-- ({2.3420*\dx},{1.0000*\dy})
	-- ({2.3520*\dx},{1.0000*\dy})
	-- ({2.3620*\dx},{1.0000*\dy})
	-- ({2.3720*\dx},{1.0000*\dy})
	-- ({2.3820*\dx},{1.0000*\dy})
	-- ({2.3920*\dx},{1.0000*\dy})
	-- ({2.4020*\dx},{1.0000*\dy})
	-- ({2.4120*\dx},{1.0000*\dy})
	-- ({2.4220*\dx},{1.0000*\dy})
	-- ({2.4320*\dx},{1.0000*\dy})
	-- ({2.4420*\dx},{1.0000*\dy})
	-- ({2.4520*\dx},{1.0000*\dy})
	-- ({2.4621*\dx},{1.0000*\dy})
	-- ({2.4721*\dx},{1.0000*\dy})
	-- ({2.4821*\dx},{1.0000*\dy})
	-- ({2.4921*\dx},{1.0000*\dy})
	-- ({2.5021*\dx},{1.0000*\dy})
	-- ({2.5121*\dx},{1.0000*\dy})
	-- ({2.5221*\dx},{1.0000*\dy})
	-- ({2.5321*\dx},{1.0000*\dy})
	-- ({2.5421*\dx},{1.0000*\dy})
	-- ({2.5521*\dx},{1.0000*\dy})
	-- ({2.5621*\dx},{1.0000*\dy})
	-- ({2.5721*\dx},{1.0000*\dy})
	-- ({2.5822*\dx},{1.0000*\dy})
	-- ({2.5922*\dx},{1.0000*\dy})
	-- ({2.6022*\dx},{1.0000*\dy})
	-- ({2.6122*\dx},{1.0000*\dy})
	-- ({2.6222*\dx},{1.0000*\dy})
	-- ({2.6322*\dx},{1.0000*\dy})
	-- ({2.6422*\dx},{1.0000*\dy})
	-- ({2.6522*\dx},{1.0000*\dy})
	-- ({2.6622*\dx},{1.0000*\dy})
	-- ({2.6722*\dx},{1.0000*\dy})
	-- ({2.6822*\dx},{1.0000*\dy})
	-- ({2.6922*\dx},{1.0000*\dy})
	-- ({2.7023*\dx},{1.0000*\dy})
	-- ({2.7123*\dx},{1.0000*\dy})
	-- ({2.7223*\dx},{1.0000*\dy})
	-- ({2.7323*\dx},{1.0000*\dy})
	-- ({2.7423*\dx},{1.0000*\dy})
	-- ({2.7523*\dx},{1.0000*\dy})
	-- ({2.7623*\dx},{1.0000*\dy})
	-- ({2.7723*\dx},{1.0000*\dy})
	-- ({2.7823*\dx},{1.0000*\dy})
	-- ({2.7923*\dx},{1.0000*\dy})
	-- ({2.8023*\dx},{1.0000*\dy})
	-- ({2.8123*\dx},{1.0000*\dy})
	-- ({2.8224*\dx},{1.0000*\dy})
	-- ({2.8324*\dx},{1.0000*\dy})
	-- ({2.8424*\dx},{1.0000*\dy})
	-- ({2.8524*\dx},{1.0000*\dy})
	-- ({2.8624*\dx},{1.0000*\dy})
	-- ({2.8724*\dx},{1.0000*\dy})
	-- ({2.8824*\dx},{1.0000*\dy})
	-- ({2.8924*\dx},{1.0000*\dy})
	-- ({2.9024*\dx},{1.0000*\dy})
	-- ({2.9124*\dx},{1.0000*\dy})
	-- ({2.9224*\dx},{1.0000*\dy})
	-- ({2.9324*\dx},{1.0000*\dy})
	-- ({2.9425*\dx},{1.0000*\dy})
	-- ({2.9525*\dx},{1.0000*\dy})
	-- ({2.9625*\dx},{1.0000*\dy})
	-- ({2.9725*\dx},{1.0000*\dy})
	-- ({2.9825*\dx},{1.0000*\dy})
	-- ({2.9925*\dx},{1.0000*\dy})
	-- ({3.0025*\dx},{1.0000*\dy})
	-- ({3.0125*\dx},{1.0000*\dy})
	-- ({3.0225*\dx},{1.0000*\dy})
	-- ({3.0325*\dx},{1.0000*\dy})
	-- ({3.0425*\dx},{1.0000*\dy})
	-- ({3.0525*\dx},{1.0000*\dy})
	-- ({3.0626*\dx},{1.0000*\dy})
	-- ({3.0726*\dx},{1.0000*\dy})
	-- ({3.0826*\dx},{1.0000*\dy})
	-- ({3.0926*\dx},{1.0000*\dy})
	-- ({3.1026*\dx},{1.0000*\dy})
	-- ({3.1126*\dx},{1.0000*\dy})
	-- ({3.1226*\dx},{1.0000*\dy})
	-- ({3.1326*\dx},{1.0000*\dy})
	-- ({3.1426*\dx},{1.0000*\dy})
	-- ({3.1526*\dx},{1.0000*\dy})
	-- ({3.1626*\dx},{1.0000*\dy})
	-- ({3.1726*\dx},{1.0000*\dy})
	-- ({3.1827*\dx},{1.0000*\dy})
	-- ({3.1927*\dx},{1.0000*\dy})
	-- ({3.2027*\dx},{1.0000*\dy})
	-- ({3.2127*\dx},{1.0000*\dy})
	-- ({3.2227*\dx},{1.0000*\dy})
	-- ({3.2327*\dx},{1.0000*\dy})
	-- ({3.2427*\dx},{1.0000*\dy})
	-- ({3.2527*\dx},{1.0000*\dy})
	-- ({3.2627*\dx},{1.0000*\dy})
	-- ({3.2727*\dx},{1.0000*\dy})
	-- ({3.2827*\dx},{1.0000*\dy})
	-- ({3.2927*\dx},{1.0000*\dy})
	-- ({3.3028*\dx},{1.0000*\dy})
	-- ({3.3128*\dx},{1.0000*\dy})
	-- ({3.3228*\dx},{1.0000*\dy})
	-- ({3.3328*\dx},{1.0000*\dy})
	-- ({3.3428*\dx},{1.0000*\dy})
	-- ({3.3528*\dx},{1.0000*\dy})
	-- ({3.3628*\dx},{1.0000*\dy})
	-- ({3.3728*\dx},{1.0000*\dy})
	-- ({3.3828*\dx},{1.0000*\dy})
	-- ({3.3928*\dx},{1.0000*\dy})
	-- ({3.4028*\dx},{1.0000*\dy})
	-- ({3.4128*\dx},{1.0000*\dy})
	-- ({3.4229*\dx},{1.0000*\dy})
	-- ({3.4329*\dx},{1.0000*\dy})
	-- ({3.4429*\dx},{1.0000*\dy})
	-- ({3.4529*\dx},{1.0000*\dy})
	-- ({3.4629*\dx},{1.0000*\dy})
	-- ({3.4729*\dx},{1.0000*\dy})
	-- ({3.4829*\dx},{1.0000*\dy})
	-- ({3.4929*\dx},{1.0000*\dy})
	-- ({3.5029*\dx},{1.0000*\dy})
	-- ({3.5129*\dx},{1.0000*\dy})
	-- ({3.5229*\dx},{1.0000*\dy})
	-- ({3.5329*\dx},{1.0000*\dy})
	-- ({3.5430*\dx},{1.0000*\dy})
	-- ({3.5530*\dx},{1.0000*\dy})
	-- ({3.5630*\dx},{1.0000*\dy})
	-- ({3.5730*\dx},{1.0000*\dy})
	-- ({3.5830*\dx},{1.0000*\dy})
	-- ({3.5930*\dx},{1.0000*\dy})
	-- ({3.6030*\dx},{1.0000*\dy})
	-- ({3.6130*\dx},{1.0000*\dy})
	-- ({3.6230*\dx},{1.0000*\dy})
	-- ({3.6330*\dx},{1.0000*\dy})
	-- ({3.6430*\dx},{1.0000*\dy})
	-- ({3.6530*\dx},{1.0000*\dy})
	-- ({3.6631*\dx},{1.0000*\dy})
	-- ({3.6731*\dx},{1.0000*\dy})
	-- ({3.6831*\dx},{1.0000*\dy})
	-- ({3.6931*\dx},{1.0000*\dy})
	-- ({3.7031*\dx},{1.0000*\dy})
	-- ({3.7131*\dx},{1.0000*\dy})
	-- ({3.7231*\dx},{1.0000*\dy})
	-- ({3.7331*\dx},{1.0000*\dy})
	-- ({3.7431*\dx},{1.0000*\dy})
	-- ({3.7531*\dx},{1.0000*\dy})
	-- ({3.7631*\dx},{1.0000*\dy})
	-- ({3.7731*\dx},{1.0000*\dy})
	-- ({3.7832*\dx},{1.0000*\dy})
	-- ({3.7932*\dx},{1.0000*\dy})
	-- ({3.8032*\dx},{1.0000*\dy})
	-- ({3.8132*\dx},{1.0000*\dy})
	-- ({3.8232*\dx},{1.0000*\dy})
	-- ({3.8332*\dx},{1.0000*\dy})
	-- ({3.8432*\dx},{1.0000*\dy})
	-- ({3.8532*\dx},{1.0000*\dy})
	-- ({3.8632*\dx},{1.0000*\dy})
	-- ({3.8732*\dx},{1.0000*\dy})
	-- ({3.8832*\dx},{1.0000*\dy})
	-- ({3.8932*\dx},{1.0000*\dy})
	-- ({3.9033*\dx},{1.0000*\dy})
	-- ({3.9133*\dx},{1.0000*\dy})
	-- ({3.9233*\dx},{1.0000*\dy})
	-- ({3.9333*\dx},{1.0000*\dy})
	-- ({3.9433*\dx},{1.0000*\dy})
	-- ({3.9533*\dx},{1.0000*\dy})
	-- ({3.9633*\dx},{1.0000*\dy})
	-- ({3.9733*\dx},{1.0000*\dy})
	-- ({3.9833*\dx},{1.0000*\dy})
	-- ({3.9933*\dx},{1.0000*\dy})
	-- ({4.0033*\dx},{1.0000*\dy})
	-- ({4.0133*\dx},{1.0000*\dy})
	-- ({4.0234*\dx},{1.0000*\dy})
	-- ({4.0334*\dx},{1.0000*\dy})
	-- ({4.0434*\dx},{1.0000*\dy})
	-- ({4.0534*\dx},{1.0000*\dy})
	-- ({4.0634*\dx},{1.0000*\dy})
	-- ({4.0734*\dx},{1.0000*\dy})
	-- ({4.0834*\dx},{1.0000*\dy})
	-- ({4.0934*\dx},{1.0000*\dy})
	-- ({4.1034*\dx},{1.0000*\dy})
	-- ({4.1134*\dx},{1.0000*\dy})
	-- ({4.1234*\dx},{1.0000*\dy})
	-- ({4.1334*\dx},{1.0000*\dy})
	-- ({4.1435*\dx},{1.0000*\dy})
	-- ({4.1535*\dx},{1.0000*\dy})
	-- ({4.1635*\dx},{1.0000*\dy})
	-- ({4.1735*\dx},{1.0000*\dy})
	-- ({4.1835*\dx},{1.0000*\dy})
	-- ({4.1935*\dx},{1.0000*\dy})
	-- ({4.2035*\dx},{1.0000*\dy})
	-- ({4.2135*\dx},{1.0000*\dy})
	-- ({4.2235*\dx},{1.0000*\dy})
	-- ({4.2335*\dx},{1.0000*\dy})
	-- ({4.2435*\dx},{1.0000*\dy})
	-- ({4.2535*\dx},{1.0000*\dy})
	-- ({4.2636*\dx},{1.0000*\dy})
	-- ({4.2736*\dx},{1.0000*\dy})
	-- ({4.2836*\dx},{1.0000*\dy})
	-- ({4.2936*\dx},{1.0000*\dy})
	-- ({4.3036*\dx},{1.0000*\dy})
	-- ({4.3136*\dx},{1.0000*\dy})
	-- ({4.3236*\dx},{1.0000*\dy})
	-- ({4.3336*\dx},{1.0000*\dy})
	-- ({4.3436*\dx},{1.0000*\dy})
	-- ({4.3536*\dx},{1.0000*\dy})
	-- ({4.3636*\dx},{1.0000*\dy})
	-- ({4.3736*\dx},{1.0000*\dy})
	-- ({4.3837*\dx},{1.0000*\dy})
	-- ({4.3937*\dx},{1.0000*\dy})
	-- ({4.4037*\dx},{1.0000*\dy})
	-- ({4.4137*\dx},{1.0000*\dy})
	-- ({4.4237*\dx},{1.0000*\dy})
	-- ({4.4337*\dx},{1.0000*\dy})
	-- ({4.4437*\dx},{1.0000*\dy})
	-- ({4.4537*\dx},{1.0000*\dy})
	-- ({4.4637*\dx},{1.0000*\dy})
	-- ({4.4737*\dx},{1.0000*\dy})
	-- ({4.4837*\dx},{1.0000*\dy})
	-- ({4.4937*\dx},{1.0000*\dy})
	-- ({4.5038*\dx},{1.0000*\dy})
	-- ({4.5138*\dx},{1.0000*\dy})
	-- ({4.5238*\dx},{1.0000*\dy})
	-- ({4.5338*\dx},{1.0000*\dy})
	-- ({4.5438*\dx},{1.0000*\dy})
	-- ({4.5538*\dx},{1.0000*\dy})
	-- ({4.5638*\dx},{1.0000*\dy})
	-- ({4.5738*\dx},{1.0000*\dy})
	-- ({4.5838*\dx},{1.0000*\dy})
	-- ({4.5938*\dx},{1.0000*\dy})
	-- ({4.6038*\dx},{1.0000*\dy})
	-- ({4.6138*\dx},{1.0000*\dy})
	-- ({4.6239*\dx},{1.0000*\dy})
	-- ({4.6339*\dx},{1.0000*\dy})
	-- ({4.6439*\dx},{1.0000*\dy})
	-- ({4.6539*\dx},{1.0000*\dy})
	-- ({4.6639*\dx},{1.0000*\dy})
	-- ({4.6739*\dx},{1.0000*\dy})
	-- ({4.6839*\dx},{1.0000*\dy})
	-- ({4.6939*\dx},{1.0000*\dy})
	-- ({4.7039*\dx},{1.0000*\dy})
	-- ({4.7139*\dx},{1.0000*\dy})
	-- ({4.7239*\dx},{1.0000*\dy})
	-- ({4.7339*\dx},{1.0000*\dy})
	-- ({4.7440*\dx},{1.0000*\dy})
	-- ({4.7540*\dx},{1.0000*\dy})
	-- ({4.7640*\dx},{1.0000*\dy})
	-- ({4.7740*\dx},{1.0000*\dy})
	-- ({4.7840*\dx},{1.0000*\dy})
	-- ({4.7940*\dx},{1.0000*\dy})
	-- ({4.8040*\dx},{1.0000*\dy})
	-- ({4.8140*\dx},{1.0000*\dy})
	-- ({4.8240*\dx},{1.0000*\dy})
	-- ({4.8340*\dx},{1.0000*\dy})
	-- ({4.8440*\dx},{1.0000*\dy})
	-- ({4.8540*\dx},{1.0000*\dy})
	-- ({4.8641*\dx},{1.0000*\dy})
	-- ({4.8741*\dx},{1.0000*\dy})
	-- ({4.8841*\dx},{1.0000*\dy})
	-- ({4.8941*\dx},{1.0000*\dy})
	-- ({4.9041*\dx},{1.0000*\dy})
	-- ({4.9141*\dx},{1.0000*\dy})
	-- ({4.9241*\dx},{1.0000*\dy})
	-- ({4.9341*\dx},{1.0000*\dy})
	-- ({4.9441*\dx},{1.0000*\dy})
	-- ({4.9541*\dx},{1.0000*\dy})
	-- ({4.9641*\dx},{1.0000*\dy})
	-- ({4.9741*\dx},{1.0000*\dy})
	-- ({4.9842*\dx},{1.0000*\dy})
	-- ({4.9942*\dx},{1.0000*\dy})
	-- ({5.0042*\dx},{1.0000*\dy})
	-- ({5.0142*\dx},{1.0000*\dy})
	-- ({5.0242*\dx},{1.0000*\dy})
	-- ({5.0342*\dx},{1.0000*\dy})
	-- ({5.0442*\dx},{1.0000*\dy})
	-- ({5.0542*\dx},{1.0000*\dy})
	-- ({5.0642*\dx},{1.0000*\dy})
	-- ({5.0742*\dx},{1.0000*\dy})
	-- ({5.0842*\dx},{1.0000*\dy})
	-- ({5.0942*\dx},{1.0000*\dy})
	-- ({5.1043*\dx},{1.0000*\dy})
	-- ({5.1143*\dx},{1.0000*\dy})
	-- ({5.1243*\dx},{1.0000*\dy})
	-- ({5.1343*\dx},{1.0000*\dy})
	-- ({5.1443*\dx},{1.0000*\dy})
	-- ({5.1543*\dx},{1.0000*\dy})
	-- ({5.1643*\dx},{1.0000*\dy})
	-- ({5.1743*\dx},{1.0000*\dy})
	-- ({5.1843*\dx},{1.0000*\dy})
	-- ({5.1943*\dx},{1.0000*\dy})
	-- ({5.2043*\dx},{1.0000*\dy})
	-- ({5.2143*\dx},{1.0000*\dy})
	-- ({5.2244*\dx},{1.0000*\dy})
	-- ({5.2344*\dx},{1.0000*\dy})
	-- ({5.2444*\dx},{1.0000*\dy})
	-- ({5.2544*\dx},{1.0000*\dy})
	-- ({5.2644*\dx},{1.0000*\dy})
	-- ({5.2744*\dx},{1.0000*\dy})
	-- ({5.2844*\dx},{1.0000*\dy})
	-- ({5.2944*\dx},{1.0000*\dy})
	-- ({5.3044*\dx},{1.0000*\dy})
	-- ({5.3144*\dx},{1.0000*\dy})
	-- ({5.3244*\dx},{1.0000*\dy})
	-- ({5.3344*\dx},{1.0000*\dy})
	-- ({5.3445*\dx},{1.0000*\dy})
	-- ({5.3545*\dx},{1.0000*\dy})
	-- ({5.3645*\dx},{1.0000*\dy})
	-- ({5.3745*\dx},{1.0000*\dy})
	-- ({5.3845*\dx},{1.0000*\dy})
	-- ({5.3945*\dx},{1.0000*\dy})
	-- ({5.4045*\dx},{1.0000*\dy})
	-- ({5.4145*\dx},{1.0000*\dy})
	-- ({5.4245*\dx},{1.0000*\dy})
	-- ({5.4345*\dx},{1.0000*\dy})
	-- ({5.4445*\dx},{1.0000*\dy})
	-- ({5.4545*\dx},{1.0000*\dy})
	-- ({5.4646*\dx},{1.0000*\dy})
	-- ({5.4746*\dx},{1.0000*\dy})
	-- ({5.4846*\dx},{1.0000*\dy})
	-- ({5.4946*\dx},{1.0000*\dy})
	-- ({5.5046*\dx},{1.0000*\dy})
	-- ({5.5146*\dx},{1.0000*\dy})
	-- ({5.5246*\dx},{1.0000*\dy})
	-- ({5.5346*\dx},{1.0000*\dy})
	-- ({5.5446*\dx},{1.0000*\dy})
	-- ({5.5546*\dx},{1.0000*\dy})
	-- ({5.5646*\dx},{1.0000*\dy})
	-- ({5.5746*\dx},{1.0000*\dy})
	-- ({5.5847*\dx},{1.0000*\dy})
	-- ({5.5947*\dx},{1.0000*\dy})
	-- ({5.6047*\dx},{1.0000*\dy})
	-- ({5.6147*\dx},{1.0000*\dy})
	-- ({5.6247*\dx},{1.0000*\dy})
	-- ({5.6347*\dx},{1.0000*\dy})
	-- ({5.6447*\dx},{1.0000*\dy})
	-- ({5.6547*\dx},{1.0000*\dy})
	-- ({5.6647*\dx},{1.0000*\dy})
	-- ({5.6747*\dx},{1.0000*\dy})
	-- ({5.6847*\dx},{1.0000*\dy})
	-- ({5.6947*\dx},{1.0000*\dy})
	-- ({5.7048*\dx},{1.0000*\dy})
	-- ({5.7148*\dx},{1.0000*\dy})
	-- ({5.7248*\dx},{1.0000*\dy})
	-- ({5.7348*\dx},{1.0000*\dy})
	-- ({5.7448*\dx},{1.0000*\dy})
	-- ({5.7548*\dx},{1.0000*\dy})
	-- ({5.7648*\dx},{1.0000*\dy})
	-- ({5.7748*\dx},{1.0000*\dy})
	-- ({5.7848*\dx},{1.0000*\dy})
	-- ({5.7948*\dx},{1.0000*\dy})
	-- ({5.8048*\dx},{1.0000*\dy})
	-- ({5.8148*\dx},{1.0000*\dy})
	-- ({5.8249*\dx},{1.0000*\dy})
	-- ({5.8349*\dx},{1.0000*\dy})
	-- ({5.8449*\dx},{1.0000*\dy})
	-- ({5.8549*\dx},{1.0000*\dy})
	-- ({5.8649*\dx},{1.0000*\dy})
	-- ({5.8749*\dx},{1.0000*\dy})
	-- ({5.8849*\dx},{1.0000*\dy})
	-- ({5.8949*\dx},{1.0000*\dy})
	-- ({5.9049*\dx},{1.0000*\dy})
	-- ({5.9149*\dx},{1.0000*\dy})
	-- ({5.9249*\dx},{1.0000*\dy})
	-- ({5.9349*\dx},{1.0000*\dy})
	-- ({5.9450*\dx},{1.0000*\dy})
	-- ({5.9550*\dx},{1.0000*\dy})
	-- ({5.9650*\dx},{1.0000*\dy})
	-- ({5.9750*\dx},{1.0000*\dy})
	-- ({5.9850*\dx},{1.0000*\dy})
	-- ({5.9950*\dx},{1.0000*\dy})
	-- ({6.0050*\dx},{1.0000*\dy})
	-- ({6.0150*\dx},{1.0000*\dy})
	-- ({6.0250*\dx},{1.0000*\dy})
	-- ({6.0350*\dx},{1.0000*\dy})
	-- ({6.0450*\dx},{1.0000*\dy})
	-- ({6.0550*\dx},{1.0000*\dy})
	-- ({6.0651*\dx},{1.0000*\dy})
	-- ({6.0751*\dx},{1.0000*\dy})
	-- ({6.0851*\dx},{1.0000*\dy})
	-- ({6.0951*\dx},{1.0000*\dy})
	-- ({6.1051*\dx},{1.0000*\dy})
	-- ({6.1151*\dx},{1.0000*\dy})
	-- ({6.1251*\dx},{1.0000*\dy})
	-- ({6.1351*\dx},{1.0000*\dy})
	-- ({6.1451*\dx},{1.0000*\dy})
	-- ({6.1551*\dx},{1.0000*\dy})
	-- ({6.1651*\dx},{1.0000*\dy})
	-- ({6.1751*\dx},{1.0000*\dy})
	-- ({6.1852*\dx},{1.0000*\dy})
	-- ({6.1952*\dx},{1.0000*\dy})
	-- ({6.2052*\dx},{1.0000*\dy})
	-- ({6.2152*\dx},{1.0000*\dy})
	-- ({6.2252*\dx},{1.0000*\dy})
	-- ({6.2352*\dx},{1.0000*\dy})
	-- ({6.2452*\dx},{1.0000*\dy})
	-- ({6.2552*\dx},{1.0000*\dy})
	-- ({6.2652*\dx},{1.0000*\dy})
	-- ({6.2752*\dx},{1.0000*\dy})
	-- ({6.2852*\dx},{1.0000*\dy})
	-- ({6.2952*\dx},{1.0000*\dy})
	-- ({6.3053*\dx},{1.0000*\dy})
	-- ({6.3153*\dx},{1.0000*\dy})
	-- ({6.3253*\dx},{1.0000*\dy})
	-- ({6.3353*\dx},{1.0000*\dy})
	-- ({6.3453*\dx},{1.0000*\dy})
	-- ({6.3553*\dx},{1.0000*\dy})
	-- ({6.3653*\dx},{1.0000*\dy})
	-- ({6.3753*\dx},{1.0000*\dy})
	-- ({6.3853*\dx},{1.0000*\dy})
	-- ({6.3953*\dx},{1.0000*\dy})
	-- ({6.4053*\dx},{1.0000*\dy})
	-- ({6.4153*\dx},{1.0000*\dy})
	-- ({6.4254*\dx},{1.0000*\dy})
	-- ({6.4354*\dx},{1.0000*\dy})
	-- ({6.4454*\dx},{1.0000*\dy})
	-- ({6.4554*\dx},{1.0000*\dy})
	-- ({6.4654*\dx},{1.0000*\dy})
	-- ({6.4754*\dx},{1.0000*\dy})
	-- ({6.4854*\dx},{1.0000*\dy})
	-- ({6.4954*\dx},{1.0000*\dy})
	-- ({6.5054*\dx},{1.0000*\dy})
	-- ({6.5154*\dx},{1.0000*\dy})
	-- ({6.5254*\dx},{1.0000*\dy})
	-- ({6.5354*\dx},{1.0000*\dy})
	-- ({6.5455*\dx},{1.0000*\dy})
	-- ({6.5555*\dx},{1.0000*\dy})
	-- ({6.5655*\dx},{1.0000*\dy})
	-- ({6.5755*\dx},{1.0000*\dy})
	-- ({6.5855*\dx},{1.0000*\dy})
	-- ({6.5955*\dx},{1.0000*\dy})
	-- ({6.6055*\dx},{1.0000*\dy})
	-- ({6.6155*\dx},{1.0000*\dy})
	-- ({6.6255*\dx},{1.0000*\dy})
	-- ({6.6355*\dx},{1.0000*\dy})
	-- ({6.6455*\dx},{1.0000*\dy})
	-- ({6.6555*\dx},{1.0000*\dy})
	-- ({6.6656*\dx},{1.0000*\dy})
	-- ({6.6756*\dx},{1.0000*\dy})
	-- ({6.6856*\dx},{1.0000*\dy})
	-- ({6.6956*\dx},{1.0000*\dy})
	-- ({6.7056*\dx},{1.0000*\dy})
	-- ({6.7156*\dx},{1.0000*\dy})
	-- ({6.7256*\dx},{1.0000*\dy})
	-- ({6.7356*\dx},{1.0000*\dy})
	-- ({6.7456*\dx},{1.0000*\dy})
	-- ({6.7556*\dx},{1.0000*\dy})
	-- ({6.7656*\dx},{1.0000*\dy})
	-- ({6.7756*\dx},{1.0000*\dy})
	-- ({6.7857*\dx},{1.0000*\dy})
	-- ({6.7957*\dx},{1.0000*\dy})
	-- ({6.8057*\dx},{1.0000*\dy})
	-- ({6.8157*\dx},{1.0000*\dy})
	-- ({6.8257*\dx},{1.0000*\dy})
	-- ({6.8357*\dx},{1.0000*\dy})
	-- ({6.8457*\dx},{1.0000*\dy})
	-- ({6.8557*\dx},{1.0000*\dy})
	-- ({6.8657*\dx},{1.0000*\dy})
	-- ({6.8757*\dx},{1.0000*\dy})
	-- ({6.8857*\dx},{1.0000*\dy})
	-- ({6.8957*\dx},{1.0000*\dy})
	-- ({6.9058*\dx},{1.0000*\dy})
	-- ({6.9158*\dx},{1.0000*\dy})
	-- ({6.9258*\dx},{1.0000*\dy})
	-- ({6.9358*\dx},{1.0000*\dy})
	-- ({6.9458*\dx},{1.0000*\dy})
	-- ({6.9558*\dx},{1.0000*\dy})
	-- ({6.9658*\dx},{1.0000*\dy})
	-- ({6.9758*\dx},{1.0000*\dy})
	-- ({6.9858*\dx},{1.0000*\dy})
	-- ({6.9958*\dx},{1.0000*\dy})
	-- ({7.0058*\dx},{1.0000*\dy})
	-- ({7.0158*\dx},{1.0000*\dy})
	-- ({7.0259*\dx},{1.0000*\dy})
	-- ({7.0359*\dx},{1.0000*\dy})
	-- ({7.0459*\dx},{1.0000*\dy})
	-- ({7.0559*\dx},{1.0000*\dy})
	-- ({7.0659*\dx},{1.0000*\dy})
	-- ({7.0759*\dx},{1.0000*\dy})
	-- ({7.0859*\dx},{1.0000*\dy})
	-- ({7.0959*\dx},{1.0000*\dy})
	-- ({7.1059*\dx},{1.0000*\dy})
	-- ({7.1159*\dx},{1.0000*\dy})
	-- ({7.1259*\dx},{1.0000*\dy})
	-- ({7.1359*\dx},{1.0000*\dy})
	-- ({7.1460*\dx},{1.0000*\dy})
	-- ({7.1560*\dx},{1.0000*\dy})
	-- ({7.1660*\dx},{1.0000*\dy})
	-- ({7.1760*\dx},{1.0000*\dy})
	-- ({7.1860*\dx},{1.0000*\dy})
	-- ({7.1960*\dx},{1.0000*\dy})
	-- ({7.2060*\dx},{1.0000*\dy})
	-- ({7.2160*\dx},{1.0000*\dy})
	-- ({7.2260*\dx},{1.0000*\dy})
	-- ({7.2360*\dx},{1.0000*\dy})
	-- ({7.2460*\dx},{1.0000*\dy})
	-- ({7.2560*\dx},{1.0000*\dy})
	-- ({7.2661*\dx},{1.0000*\dy})
	-- ({7.2761*\dx},{1.0000*\dy})
	-- ({7.2861*\dx},{1.0000*\dy})
	-- ({7.2961*\dx},{1.0000*\dy})
	-- ({7.3061*\dx},{1.0000*\dy})
	-- ({7.3161*\dx},{1.0000*\dy})
	-- ({7.3261*\dx},{1.0000*\dy})
	-- ({7.3361*\dx},{1.0000*\dy})
	-- ({7.3461*\dx},{1.0000*\dy})
	-- ({7.3561*\dx},{1.0000*\dy})
	-- ({7.3661*\dx},{1.0000*\dy})
	-- ({7.3761*\dx},{1.0000*\dy})
	-- ({7.3862*\dx},{1.0000*\dy})
	-- ({7.3962*\dx},{1.0000*\dy})
	-- ({7.4062*\dx},{1.0000*\dy})
	-- ({7.4162*\dx},{1.0000*\dy})
	-- ({7.4262*\dx},{1.0000*\dy})
	-- ({7.4362*\dx},{1.0000*\dy})
	-- ({7.4462*\dx},{1.0000*\dy})
	-- ({7.4562*\dx},{1.0000*\dy})
	-- ({7.4662*\dx},{1.0000*\dy})
	-- ({7.4762*\dx},{1.0000*\dy})
	-- ({7.4862*\dx},{1.0000*\dy})
	-- ({7.4962*\dx},{1.0000*\dy})
	-- ({7.5063*\dx},{1.0000*\dy})
	-- ({7.5163*\dx},{1.0000*\dy})
	-- ({7.5263*\dx},{1.0000*\dy})
	-- ({7.5363*\dx},{1.0000*\dy})
	-- ({7.5463*\dx},{1.0000*\dy})
	-- ({7.5563*\dx},{1.0000*\dy})
	-- ({7.5663*\dx},{1.0000*\dy})
	-- ({7.5763*\dx},{1.0000*\dy})
	-- ({7.5863*\dx},{1.0000*\dy})
	-- ({7.5963*\dx},{1.0000*\dy})
	-- ({7.6063*\dx},{1.0000*\dy})
	-- ({7.6163*\dx},{1.0000*\dy})
	-- ({7.6264*\dx},{1.0000*\dy})
	-- ({7.6364*\dx},{1.0000*\dy})
	-- ({7.6464*\dx},{1.0000*\dy})
	-- ({7.6564*\dx},{1.0000*\dy})
	-- ({7.6664*\dx},{1.0000*\dy})
	-- ({7.6764*\dx},{1.0000*\dy})
	-- ({7.6864*\dx},{1.0000*\dy})
	-- ({7.6964*\dx},{1.0000*\dy})
	-- ({7.7064*\dx},{1.0000*\dy})
	-- ({7.7164*\dx},{1.0000*\dy})
	-- ({7.7264*\dx},{1.0000*\dy})
	-- ({7.7364*\dx},{1.0000*\dy})
	-- ({7.7465*\dx},{1.0000*\dy})
	-- ({7.7565*\dx},{1.0000*\dy})
	-- ({7.7665*\dx},{1.0000*\dy})
	-- ({7.7765*\dx},{1.0000*\dy})
	-- ({7.7865*\dx},{1.0000*\dy})
	-- ({7.7965*\dx},{1.0000*\dy})
	-- ({7.8065*\dx},{1.0000*\dy})
	-- ({7.8165*\dx},{1.0000*\dy})
	-- ({7.8265*\dx},{1.0000*\dy})
	-- ({7.8365*\dx},{1.0000*\dy})
	-- ({7.8465*\dx},{1.0000*\dy})
	-- ({7.8565*\dx},{1.0000*\dy})
	-- ({7.8666*\dx},{1.0000*\dy})
	-- ({7.8766*\dx},{1.0000*\dy})
	-- ({7.8866*\dx},{1.0000*\dy})
	-- ({7.8966*\dx},{1.0000*\dy})
	-- ({7.9066*\dx},{1.0000*\dy})
	-- ({7.9166*\dx},{1.0000*\dy})
	-- ({7.9266*\dx},{1.0000*\dy})
	-- ({7.9366*\dx},{1.0000*\dy})
	-- ({7.9466*\dx},{1.0000*\dy})
	-- ({7.9566*\dx},{1.0000*\dy})
	-- ({7.9666*\dx},{1.0000*\dy})
	-- ({7.9766*\dx},{1.0000*\dy})
	-- ({7.9867*\dx},{1.0000*\dy})
	-- ({7.9967*\dx},{1.0000*\dy})
	-- ({8.0067*\dx},{1.0000*\dy})
	-- ({8.0167*\dx},{1.0000*\dy})
	-- ({8.0267*\dx},{1.0000*\dy})
	-- ({8.0367*\dx},{1.0000*\dy})
	-- ({8.0467*\dx},{1.0000*\dy})
	-- ({8.0567*\dx},{1.0000*\dy})
	-- ({8.0667*\dx},{1.0000*\dy})
	-- ({8.0767*\dx},{1.0000*\dy})
	-- ({8.0867*\dx},{1.0000*\dy})
	-- ({8.0967*\dx},{1.0000*\dy})
	-- ({8.1068*\dx},{1.0000*\dy})
	-- ({8.1168*\dx},{1.0000*\dy})
	-- ({8.1268*\dx},{1.0000*\dy})
	-- ({8.1368*\dx},{1.0000*\dy})
	-- ({8.1468*\dx},{1.0000*\dy})
	-- ({8.1568*\dx},{1.0000*\dy})
	-- ({8.1668*\dx},{1.0000*\dy})
	-- ({8.1768*\dx},{1.0000*\dy})
	-- ({8.1868*\dx},{1.0000*\dy})
	-- ({8.1968*\dx},{1.0000*\dy})
	-- ({8.2068*\dx},{1.0000*\dy})
	-- ({8.2168*\dx},{1.0000*\dy})
	-- ({8.2269*\dx},{1.0000*\dy})
	-- ({8.2369*\dx},{1.0000*\dy})
	-- ({8.2469*\dx},{1.0000*\dy})
	-- ({8.2569*\dx},{1.0000*\dy})
	-- ({8.2669*\dx},{1.0000*\dy})
	-- ({8.2769*\dx},{1.0000*\dy})
	-- ({8.2869*\dx},{1.0000*\dy})
	-- ({8.2969*\dx},{1.0000*\dy})
	-- ({8.3069*\dx},{1.0000*\dy})
	-- ({8.3169*\dx},{1.0000*\dy})
	-- ({8.3269*\dx},{1.0000*\dy})
	-- ({8.3369*\dx},{1.0000*\dy})
	-- ({8.3470*\dx},{1.0000*\dy})
	-- ({8.3570*\dx},{1.0000*\dy})
	-- ({8.3670*\dx},{1.0000*\dy})
	-- ({8.3770*\dx},{1.0000*\dy})
	-- ({8.3870*\dx},{1.0000*\dy})
	-- ({8.3970*\dx},{1.0000*\dy})
	-- ({8.4070*\dx},{1.0000*\dy})
	-- ({8.4170*\dx},{1.0000*\dy})
	-- ({8.4270*\dx},{1.0000*\dy})
	-- ({8.4370*\dx},{1.0000*\dy})
	-- ({8.4470*\dx},{1.0000*\dy})
	-- ({8.4570*\dx},{1.0000*\dy})
	-- ({8.4671*\dx},{1.0000*\dy})
	-- ({8.4771*\dx},{1.0000*\dy})
	-- ({8.4871*\dx},{1.0000*\dy})
	-- ({8.4971*\dx},{1.0000*\dy})
	-- ({8.5071*\dx},{1.0000*\dy})
	-- ({8.5171*\dx},{1.0000*\dy})
	-- ({8.5271*\dx},{1.0000*\dy})
	-- ({8.5371*\dx},{1.0000*\dy})
	-- ({8.5471*\dx},{1.0000*\dy})
	-- ({8.5571*\dx},{1.0000*\dy})
	-- ({8.5671*\dx},{1.0000*\dy})
	-- ({8.5771*\dx},{1.0000*\dy})
	-- ({8.5872*\dx},{1.0000*\dy})
	-- ({8.5972*\dx},{1.0000*\dy})
	-- ({8.6072*\dx},{1.0000*\dy})
	-- ({8.6172*\dx},{1.0000*\dy})
	-- ({8.6272*\dx},{1.0000*\dy})
	-- ({8.6372*\dx},{1.0000*\dy})
	-- ({8.6472*\dx},{1.0000*\dy})
	-- ({8.6572*\dx},{1.0000*\dy})
	-- ({8.6672*\dx},{1.0000*\dy})
	-- ({8.6772*\dx},{1.0000*\dy})
	-- ({8.6872*\dx},{1.0000*\dy})
	-- ({8.6972*\dx},{1.0000*\dy})
	-- ({8.7073*\dx},{1.0000*\dy})
	-- ({8.7173*\dx},{1.0000*\dy})
	-- ({8.7273*\dx},{1.0000*\dy})
	-- ({8.7373*\dx},{1.0000*\dy})
	-- ({8.7473*\dx},{1.0000*\dy})
	-- ({8.7573*\dx},{1.0000*\dy})
	-- ({8.7673*\dx},{1.0000*\dy})
	-- ({8.7773*\dx},{1.0000*\dy})
	-- ({8.7873*\dx},{1.0000*\dy})
	-- ({8.7973*\dx},{1.0000*\dy})
	-- ({8.8073*\dx},{1.0000*\dy})
	-- ({8.8173*\dx},{1.0000*\dy})
	-- ({8.8274*\dx},{1.0000*\dy})
	-- ({8.8374*\dx},{1.0000*\dy})
	-- ({8.8474*\dx},{1.0000*\dy})
	-- ({8.8574*\dx},{1.0000*\dy})
	-- ({8.8674*\dx},{1.0000*\dy})
	-- ({8.8774*\dx},{1.0000*\dy})
	-- ({8.8874*\dx},{1.0000*\dy})
	-- ({8.8974*\dx},{1.0000*\dy})
	-- ({8.9074*\dx},{1.0000*\dy})
	-- ({8.9174*\dx},{1.0000*\dy})
	-- ({8.9274*\dx},{1.0000*\dy})
	-- ({8.9374*\dx},{1.0000*\dy})
	-- ({8.9475*\dx},{1.0000*\dy})
	-- ({8.9575*\dx},{1.0000*\dy})
	-- ({8.9675*\dx},{1.0000*\dy})
	-- ({8.9775*\dx},{1.0000*\dy})
	-- ({8.9875*\dx},{1.0000*\dy})
	-- ({8.9975*\dx},{1.0000*\dy})
	-- ({9.0075*\dx},{1.0000*\dy})
	-- ({9.0175*\dx},{1.0000*\dy})
	-- ({9.0275*\dx},{1.0000*\dy})
	-- ({9.0375*\dx},{1.0000*\dy})
	-- ({9.0475*\dx},{1.0000*\dy})
	-- ({9.0575*\dx},{1.0000*\dy})
	-- ({9.0676*\dx},{1.0000*\dy})
	-- ({9.0776*\dx},{1.0000*\dy})
	-- ({9.0876*\dx},{1.0000*\dy})
	-- ({9.0976*\dx},{1.0000*\dy})
	-- ({9.1076*\dx},{1.0000*\dy})
	-- ({9.1176*\dx},{1.0000*\dy})
	-- ({9.1276*\dx},{1.0000*\dy})
	-- ({9.1376*\dx},{1.0000*\dy})
	-- ({9.1476*\dx},{1.0000*\dy})
	-- ({9.1576*\dx},{1.0000*\dy})
	-- ({9.1676*\dx},{1.0000*\dy})
	-- ({9.1776*\dx},{1.0000*\dy})
	-- ({9.1877*\dx},{1.0000*\dy})
	-- ({9.1977*\dx},{1.0000*\dy})
	-- ({9.2077*\dx},{1.0000*\dy})
	-- ({9.2177*\dx},{1.0000*\dy})
	-- ({9.2277*\dx},{1.0000*\dy})
	-- ({9.2377*\dx},{1.0000*\dy})
	-- ({9.2477*\dx},{1.0000*\dy})
	-- ({9.2577*\dx},{1.0000*\dy})
	-- ({9.2677*\dx},{1.0000*\dy})
	-- ({9.2777*\dx},{1.0000*\dy})
	-- ({9.2877*\dx},{1.0000*\dy})
	-- ({9.2977*\dx},{1.0000*\dy})
	-- ({9.3078*\dx},{1.0000*\dy})
	-- ({9.3178*\dx},{1.0000*\dy})
	-- ({9.3278*\dx},{1.0000*\dy})
	-- ({9.3378*\dx},{1.0000*\dy})
	-- ({9.3478*\dx},{1.0000*\dy})
	-- ({9.3578*\dx},{1.0000*\dy})
	-- ({9.3678*\dx},{1.0000*\dy})
	-- ({9.3778*\dx},{1.0000*\dy})
	-- ({9.3878*\dx},{1.0000*\dy})
	-- ({9.3978*\dx},{1.0000*\dy})
	-- ({9.4078*\dx},{1.0000*\dy})
	-- ({9.4178*\dx},{1.0000*\dy})
	-- ({9.4279*\dx},{1.0000*\dy})
	-- ({9.4379*\dx},{1.0000*\dy})
	-- ({9.4479*\dx},{1.0000*\dy})
	-- ({9.4579*\dx},{1.0000*\dy})
	-- ({9.4679*\dx},{1.0000*\dy})
	-- ({9.4779*\dx},{1.0000*\dy})
	-- ({9.4879*\dx},{1.0000*\dy})
	-- ({9.4979*\dx},{1.0000*\dy})
	-- ({9.5079*\dx},{1.0000*\dy})
	-- ({9.5179*\dx},{1.0000*\dy})
	-- ({9.5279*\dx},{1.0000*\dy})
	-- ({9.5379*\dx},{1.0000*\dy})
	-- ({9.5480*\dx},{1.0000*\dy})
	-- ({9.5580*\dx},{1.0000*\dy})
	-- ({9.5680*\dx},{1.0000*\dy})
	-- ({9.5780*\dx},{1.0000*\dy})
	-- ({9.5880*\dx},{1.0000*\dy})
	-- ({9.5980*\dx},{1.0000*\dy})
	-- ({9.6080*\dx},{1.0000*\dy})
	-- ({9.6180*\dx},{1.0000*\dy})
	-- ({9.6280*\dx},{1.0000*\dy})
	-- ({9.6380*\dx},{1.0000*\dy})
	-- ({9.6480*\dx},{1.0000*\dy})
	-- ({9.6580*\dx},{1.0000*\dy})
	-- ({9.6681*\dx},{1.0000*\dy})
	-- ({9.6781*\dx},{1.0000*\dy})
	-- ({9.6881*\dx},{1.0000*\dy})
	-- ({9.6981*\dx},{1.0000*\dy})
	-- ({9.7081*\dx},{1.0000*\dy})
	-- ({9.7181*\dx},{1.0000*\dy})
	-- ({9.7281*\dx},{1.0000*\dy})
	-- ({9.7381*\dx},{1.0000*\dy})
	-- ({9.7481*\dx},{1.0000*\dy})
	-- ({9.7581*\dx},{1.0000*\dy})
	-- ({9.7681*\dx},{1.0000*\dy})
	-- ({9.7781*\dx},{1.0000*\dy})
	-- ({9.7882*\dx},{1.0000*\dy})
	-- ({9.7982*\dx},{1.0000*\dy})
	-- ({9.8082*\dx},{1.0000*\dy})
	-- ({9.8182*\dx},{1.0000*\dy})
	-- ({9.8282*\dx},{1.0000*\dy})
	-- ({9.8382*\dx},{1.0000*\dy})
	-- ({9.8482*\dx},{1.0000*\dy})
	-- ({9.8582*\dx},{1.0000*\dy})
	-- ({9.8682*\dx},{1.0000*\dy})
	-- ({9.8782*\dx},{1.0000*\dy})
	-- ({9.8882*\dx},{1.0000*\dy})
	-- ({9.8982*\dx},{1.0000*\dy})
	-- ({9.9083*\dx},{1.0000*\dy})
	-- ({9.9183*\dx},{1.0000*\dy})
	-- ({9.9283*\dx},{1.0000*\dy})
	-- ({9.9383*\dx},{1.0000*\dy})
	-- ({9.9483*\dx},{1.0000*\dy})
	-- ({9.9583*\dx},{1.0000*\dy})
	-- ({9.9683*\dx},{1.0000*\dy})
	-- ({9.9783*\dx},{1.0000*\dy})
	-- ({9.9883*\dx},{1.0000*\dy})
	-- ({9.9983*\dx},{1.0000*\dy})
	-- ({10.0083*\dx},{1.0000*\dy})
	-- ({10.0183*\dx},{1.0000*\dy})
	-- ({10.0284*\dx},{1.0000*\dy})
	-- ({10.0384*\dx},{1.0000*\dy})
	-- ({10.0484*\dx},{1.0000*\dy})
	-- ({10.0584*\dx},{1.0000*\dy})
	-- ({10.0684*\dx},{1.0000*\dy})
	-- ({10.0784*\dx},{1.0000*\dy})
	-- ({10.0884*\dx},{1.0000*\dy})
	-- ({10.0984*\dx},{1.0000*\dy})
	-- ({10.1084*\dx},{1.0000*\dy})
	-- ({10.1184*\dx},{1.0000*\dy})
	-- ({10.1284*\dx},{1.0000*\dy})
	-- ({10.1384*\dx},{1.0000*\dy})
	-- ({10.1485*\dx},{1.0000*\dy})
	-- ({10.1585*\dx},{1.0000*\dy})
	-- ({10.1685*\dx},{1.0000*\dy})
	-- ({10.1785*\dx},{1.0000*\dy})
	-- ({10.1885*\dx},{1.0000*\dy})
	-- ({10.1985*\dx},{1.0000*\dy})
	-- ({10.2085*\dx},{1.0000*\dy})
	-- ({10.2185*\dx},{1.0000*\dy})
	-- ({10.2285*\dx},{1.0000*\dy})
	-- ({10.2385*\dx},{1.0000*\dy})
	-- ({10.2485*\dx},{1.0000*\dy})
	-- ({10.2585*\dx},{1.0000*\dy})
	-- ({10.2686*\dx},{1.0000*\dy})
	-- ({10.2786*\dx},{1.0000*\dy})
	-- ({10.2886*\dx},{1.0000*\dy})
	-- ({10.2986*\dx},{1.0000*\dy})
	-- ({10.3086*\dx},{1.0000*\dy})
	-- ({10.3186*\dx},{1.0000*\dy})
	-- ({10.3286*\dx},{1.0000*\dy})
	-- ({10.3386*\dx},{1.0000*\dy})
	-- ({10.3486*\dx},{1.0000*\dy})
	-- ({10.3586*\dx},{1.0000*\dy})
	-- ({10.3686*\dx},{1.0000*\dy})
	-- ({10.3786*\dx},{1.0000*\dy})
	-- ({10.3887*\dx},{1.0000*\dy})
	-- ({10.3987*\dx},{1.0000*\dy})
	-- ({10.4087*\dx},{1.0000*\dy})
	-- ({10.4187*\dx},{1.0000*\dy})
	-- ({10.4287*\dx},{1.0000*\dy})
	-- ({10.4387*\dx},{1.0000*\dy})
	-- ({10.4487*\dx},{1.0000*\dy})
	-- ({10.4587*\dx},{1.0000*\dy})
	-- ({10.4687*\dx},{1.0000*\dy})
	-- ({10.4787*\dx},{1.0000*\dy})
	-- ({10.4887*\dx},{1.0000*\dy})
	-- ({10.4987*\dx},{1.0000*\dy})
	-- ({10.5088*\dx},{1.0000*\dy})
	-- ({10.5188*\dx},{1.0000*\dy})
	-- ({10.5288*\dx},{1.0000*\dy})
	-- ({10.5388*\dx},{1.0000*\dy})
	-- ({10.5488*\dx},{1.0000*\dy})
	-- ({10.5588*\dx},{1.0000*\dy})
	-- ({10.5688*\dx},{1.0000*\dy})
	-- ({10.5788*\dx},{1.0000*\dy})
	-- ({10.5888*\dx},{1.0000*\dy})
	-- ({10.5988*\dx},{1.0000*\dy})
	-- ({10.6088*\dx},{1.0000*\dy})
	-- ({10.6188*\dx},{1.0000*\dy})
	-- ({10.6289*\dx},{1.0000*\dy})
	-- ({10.6389*\dx},{1.0000*\dy})
	-- ({10.6489*\dx},{1.0000*\dy})
	-- ({10.6589*\dx},{1.0000*\dy})
	-- ({10.6689*\dx},{1.0000*\dy})
	-- ({10.6789*\dx},{1.0000*\dy})
	-- ({10.6889*\dx},{1.0000*\dy})
	-- ({10.6989*\dx},{1.0000*\dy})
	-- ({10.7089*\dx},{1.0000*\dy})
	-- ({10.7189*\dx},{1.0000*\dy})
	-- ({10.7289*\dx},{1.0000*\dy})
	-- ({10.7389*\dx},{1.0000*\dy})
	-- ({10.7490*\dx},{1.0000*\dy})
	-- ({10.7590*\dx},{1.0000*\dy})
	-- ({10.7690*\dx},{1.0000*\dy})
	-- ({10.7790*\dx},{1.0000*\dy})
	-- ({10.7890*\dx},{1.0000*\dy})
	-- ({10.7990*\dx},{1.0000*\dy})
	-- ({10.8090*\dx},{1.0000*\dy})
	-- ({10.8190*\dx},{1.0000*\dy})
	-- ({10.8290*\dx},{1.0000*\dy})
	-- ({10.8390*\dx},{1.0000*\dy})
	-- ({10.8490*\dx},{1.0000*\dy})
	-- ({10.8590*\dx},{1.0000*\dy})
	-- ({10.8691*\dx},{1.0000*\dy})
	-- ({10.8791*\dx},{1.0000*\dy})
	-- ({10.8891*\dx},{1.0000*\dy})
	-- ({10.8991*\dx},{1.0000*\dy})
	-- ({10.9091*\dx},{1.0000*\dy})
	-- ({10.9191*\dx},{1.0000*\dy})
	-- ({10.9291*\dx},{1.0000*\dy})
	-- ({10.9391*\dx},{1.0000*\dy})
	-- ({10.9491*\dx},{1.0000*\dy})
	-- ({10.9591*\dx},{1.0000*\dy})
	-- ({10.9691*\dx},{1.0000*\dy})
	-- ({10.9791*\dx},{1.0000*\dy})
	-- ({10.9892*\dx},{1.0000*\dy})
	-- ({10.9992*\dx},{1.0000*\dy})
	-- ({11.0092*\dx},{1.0000*\dy})
	-- ({11.0192*\dx},{1.0000*\dy})
	-- ({11.0292*\dx},{1.0000*\dy})
	-- ({11.0392*\dx},{1.0000*\dy})
	-- ({11.0492*\dx},{1.0000*\dy})
	-- ({11.0592*\dx},{1.0000*\dy})
	-- ({11.0692*\dx},{1.0000*\dy})
	-- ({11.0792*\dx},{1.0000*\dy})
	-- ({11.0892*\dx},{1.0000*\dy})
	-- ({11.0992*\dx},{1.0000*\dy})
	-- ({11.1093*\dx},{1.0000*\dy})
	-- ({11.1193*\dx},{1.0000*\dy})
	-- ({11.1293*\dx},{1.0000*\dy})
	-- ({11.1393*\dx},{1.0000*\dy})
	-- ({11.1493*\dx},{1.0000*\dy})
	-- ({11.1593*\dx},{1.0000*\dy})
	-- ({11.1693*\dx},{1.0000*\dy})
	-- ({11.1793*\dx},{1.0000*\dy})
	-- ({11.1893*\dx},{1.0000*\dy})
	-- ({11.1993*\dx},{1.0000*\dy})
	-- ({11.2093*\dx},{1.0000*\dy})
	-- ({11.2193*\dx},{1.0000*\dy})
	-- ({11.2294*\dx},{1.0000*\dy})
	-- ({11.2394*\dx},{1.0000*\dy})
	-- ({11.2494*\dx},{1.0000*\dy})
	-- ({11.2594*\dx},{1.0000*\dy})
	-- ({11.2694*\dx},{1.0000*\dy})
	-- ({11.2794*\dx},{1.0000*\dy})
	-- ({11.2894*\dx},{1.0000*\dy})
	-- ({11.2994*\dx},{1.0000*\dy})
	-- ({11.3094*\dx},{1.0000*\dy})
	-- ({11.3194*\dx},{1.0000*\dy})
	-- ({11.3294*\dx},{1.0000*\dy})
	-- ({11.3394*\dx},{1.0000*\dy})
	-- ({11.3495*\dx},{1.0000*\dy})
	-- ({11.3595*\dx},{1.0000*\dy})
	-- ({11.3695*\dx},{1.0000*\dy})
	-- ({11.3795*\dx},{1.0000*\dy})
	-- ({11.3895*\dx},{1.0000*\dy})
	-- ({11.3995*\dx},{1.0000*\dy})
	-- ({11.4095*\dx},{1.0000*\dy})
	-- ({11.4195*\dx},{1.0000*\dy})
	-- ({11.4295*\dx},{1.0000*\dy})
	-- ({11.4395*\dx},{1.0000*\dy})
	-- ({11.4495*\dx},{1.0000*\dy})
	-- ({11.4595*\dx},{1.0000*\dy})
	-- ({11.4696*\dx},{1.0000*\dy})
	-- ({11.4796*\dx},{1.0000*\dy})
	-- ({11.4896*\dx},{1.0000*\dy})
	-- ({11.4996*\dx},{1.0000*\dy})
	-- ({11.5096*\dx},{1.0000*\dy})
	-- ({11.5196*\dx},{1.0000*\dy})
	-- ({11.5296*\dx},{1.0000*\dy})
	-- ({11.5396*\dx},{1.0000*\dy})
	-- ({11.5496*\dx},{1.0000*\dy})
	-- ({11.5596*\dx},{1.0000*\dy})
	-- ({11.5696*\dx},{1.0000*\dy})
	-- ({11.5796*\dx},{1.0000*\dy})
	-- ({11.5897*\dx},{1.0000*\dy})
	-- ({11.5997*\dx},{1.0000*\dy})
	-- ({11.6097*\dx},{1.0000*\dy})
	-- ({11.6197*\dx},{1.0000*\dy})
	-- ({11.6297*\dx},{1.0000*\dy})
	-- ({11.6397*\dx},{1.0000*\dy})
	-- ({11.6497*\dx},{1.0000*\dy})
	-- ({11.6597*\dx},{1.0000*\dy})
	-- ({11.6697*\dx},{1.0000*\dy})
	-- ({11.6797*\dx},{1.0000*\dy})
	-- ({11.6897*\dx},{1.0000*\dy})
	-- ({11.6997*\dx},{1.0000*\dy})
	-- ({11.7098*\dx},{1.0000*\dy})
	-- ({11.7198*\dx},{1.0000*\dy})
	-- ({11.7298*\dx},{1.0000*\dy})
	-- ({11.7398*\dx},{1.0000*\dy})
	-- ({11.7498*\dx},{1.0000*\dy})
	-- ({11.7598*\dx},{1.0000*\dy})
	-- ({11.7698*\dx},{1.0000*\dy})
	-- ({11.7798*\dx},{1.0000*\dy})
	-- ({11.7898*\dx},{1.0000*\dy})
	-- ({11.7998*\dx},{1.0000*\dy})
	-- ({11.8098*\dx},{1.0000*\dy})
	-- ({11.8198*\dx},{1.0000*\dy})
	-- ({11.8299*\dx},{1.0000*\dy})
	-- ({11.8399*\dx},{1.0000*\dy})
	-- ({11.8499*\dx},{1.0000*\dy})
	-- ({11.8599*\dx},{1.0000*\dy})
	-- ({11.8699*\dx},{1.0000*\dy})
	-- ({11.8799*\dx},{1.0000*\dy})
	-- ({11.8899*\dx},{1.0000*\dy})
	-- ({11.8999*\dx},{1.0000*\dy})
	-- ({11.9099*\dx},{1.0000*\dy})
	-- ({11.9199*\dx},{1.0000*\dy})
	-- ({11.9299*\dx},{1.0000*\dy})
	-- ({11.9399*\dx},{1.0000*\dy})
	-- ({11.9500*\dx},{1.0000*\dy})
	-- ({11.9600*\dx},{1.0000*\dy})
	-- ({11.9700*\dx},{1.0000*\dy})
	-- ({11.9800*\dx},{1.0000*\dy})
	-- ({11.9900*\dx},{1.0000*\dy})
	-- ({12.0000*\dx},{1.0000*\dy})
}


\begin{scope}[yshift=12cm]
\fill[color=farbe1!20] \psione -- (12,0) -- (0,0) -- cycle;
\draw[color=farbe1,line width=1.2pt] \psione;
\draw[->] (-0.1,0) -- (12.3,0) coordinate[label={$x$}];
\draw[->] (0,-0.1) -- (0,1.5) coordinate[label={right:$h_1$}];
%\draw (\aone,-0.05) -- ++(0,0.1);
%\node at (\aone,0) [below] {$a_1$};
\draw (\bone,-0.05) -- ++(0,0.1);
\node at (\bone,0) [below] {$b_1$};
\node[color=farbe1] at (1.2,0.6) {$h_1(x)$};
\end{scope}

\begin{scope}[yshift=10cm]
\fill[color=farbe2!20] \psitwo -- (12,0) -- (0,0) -- cycle;
\draw[color=farbe2,line width=1.2pt] \psitwo;
\draw[->] (-0.1,0) -- (12.3,0) coordinate[label={$x$}];
\draw[->] (0,-0.1) -- (0,1.5) coordinate[label={right:$h_2$}];
\draw (\atwo,-0.05) -- ++(0,0.1);
\node at (\atwo,0) [below] {$a_2$};
%\draw (\btwo,-0.05) -- ++(0,0.1);
%\node at (\btwo,0) [below] {$b_2$};
\node[color=farbe2] at (11.2,0.6) {$h_2(x)$};
\end{scope}

\begin{scope}[yshift=8cm]
\fill[color=farbe3!20] \psithree -- (12,0) -- (0,0) -- cycle;
\draw[color=farbe3,line width=1.2pt] \psithree;
\draw[->] (-0.1,0) -- (12.3,0) coordinate[label={$x$}];
\draw[->] (0,-0.1) -- (0,1.5) coordinate[label={right:$h_3$}];
\draw (\athree,-0.05) -- ++(0,0.1);
\node at (\athree,0) [below] {$a_3$};
\draw (\bthree,-0.05) -- ++(0,0.1);
\node at (\bthree,0) [below] {$b_3$};
\node[color=farbe3] at (3.4,0.5) {$h_3(x)$};
\end{scope}

\begin{scope}[yshift=6cm]
\fill[color=farbe4!20] \psifour -- (12,0) -- (0,0) -- cycle;
\draw[color=farbe4,line width=1.2pt] \psifour;
\draw[->] (-0.1,0) -- (12.3,0) coordinate[label={$x$}];
\draw[->] (0,-0.1) -- (0,1.5) coordinate[label={right:$h_4$}];
\draw (\afour,-0.05) -- ++(0,0.1);
\node at (\afour,0) [below] {$a_4$};
\draw (\bfour,-0.05) -- ++(0,0.1);
\node at (\bfour,0) [below] {$b_4$};
\node[color=farbe4] at (5.6,0.4) {$h_4(x)$};
\end{scope}

\begin{scope}[yshift=4cm]
\fill[color=farbe5!20] \psifive -- (12,0) -- (0,0) -- cycle;
\draw[color=farbe5,line width=1.2pt] \psifive;
\draw[->] (-0.1,0) -- (12.3,0) coordinate[label={$x$}];
\draw[->] (0,-0.1) -- (0,1.5) coordinate[label={right:$h_5$}];
\draw (\afive,-0.05) -- ++(0,0.1);
\node at (\afive,0) [below] {$a_5$};
\draw (\bfive,-0.05) -- ++(0,0.1);
\node at (\bfive,0) [below] {$b_5$};
\node[color=farbe5] at (8,0.4) {$h_5(x)$};
\end{scope}

\begin{scope}[yshift=2cm]
\fill[color=farbe6!20] \psisix -- (12,0) -- (0,0) -- cycle;
\draw[color=farbe6,line width=1.2pt] \psisix;
\draw[->] (-0.1,0) -- (12.3,0) coordinate[label={$x$}];
\draw[->] (0,-0.1) -- (0,1.5) coordinate[label={right:$h_6$}];
\draw (\asix,-0.05) -- ++(0,0.1);
\node at (\asix,0) [below] {$a_6$};
\draw (\bsix,-0.05) -- ++(0,0.1);
\node at (\bsix,0) [below] {$b_6$};
\node[color=farbe6] at (9.6,0.5) {$h_6(x)$};
\end{scope}

\begin{scope}[yshift=-1cm]
\def\dy{2.0}
\fill[color=farbe6!40] \cpsisix -- (12,0) -- (0,0) -- cycle;
\fill[color=farbe5!40] \cpsifive -- (12,0) -- (0,0) -- cycle;
\fill[color=farbe4!40] \cpsifour -- (12,0) -- (0,0) -- cycle;
\fill[color=farbe3!40] \cpsithree -- (12,0) -- (0,0) -- cycle;
\fill[color=farbe2!40] \cpsitwo -- (12,0) -- (0,0) -- cycle;
\fill[color=farbe1!40] \cpsione -- (12,0) -- (0,0) -- cycle;
\draw (0,\dy) -- (12,\dy);
\draw[->] (-0.1,0) -- (12.3,0) coordinate[label={$x$}];
\draw[->] (0,-0.1) -- (0,2.5) coordinate[label={right:$y$}];
\end{scope}

\end{tikzpicture}
\end{document}

