%
% teil2.tex -- Beispiel-File für teil2
%
% (c) 2020 Prof Dr Andreas Müller, Hochschule Rapperswil
%
% !TEX root = ../../buch.tex
% !TEX encoding = UTF-8
%
\section{Kantorowitch-Formulierung%
\label{mongekant:section:teil2}}
\kopfrechts{Kantorowitch-Formulierung}

Im vorangegangenen Abschnitt haben wir das klassische Monge-Problem beschrieben.
Dieses Modell ist intuitiv,
aber die Forderung $T_{\#}\mu=\nu$ ist sehr streng.
Wie wir bereits in Abschnitt~\ref{mongekant:subsection:monge_inexistence} gesehen haben,
kann das Monge-Problem bereits für sehr einfache Masse keine Lösung besitzen.
Um dieses Existenzproblem zu umgehen,
hat Leonid Witaljewitch~Kantorowitch das Modell relaxiert.

\subsection{Transportpläne}
Statt nach einer Abbildung $T\colon X\to Y$ zu suchen,
führen wir einen \emph{Transportplan} $\gamma(x,y) \in \mathcal{P}(X \times Y)$ ein.
Hierbei ist $\mathcal{P}(X \times Y)$ die Menge aller Wahrscheinlichkeitsmasse
auf dem Produktraum $X \times Y$.
Der Plan $\gamma(x,y)$ gibt an,
wie viel Masse von $x \in X$ zu jedem Punkt $y \in Y$ transportiert wird.
Ein zulässiger Plan muss die Randbedingungen
\begin{align}
\begin{aligned}
\gamma(A \times Y)
&=
\mu(A)
,\quad \forall A \subset X
\\
\gamma(X \times B)
&=
\nu(B)
,\quad \forall B \subset Y
\end{aligned}
\label{mongekant:eq:kantorowitch_marginals}
\end{align}
erfüllen.
Die Menge aller zulässigen Transportpläne,
welche die Randbedingungen \eqref{mongekant:eq:kantorowitch_marginals} erfüllen,
bezeichnen wir mit $\Gamma(\mu, \nu)$.

\subsection{Optimierungsproblem}
Mit diesen Notationen lautet das Optimierungsproblem
\begin{align}
C_K(\mu, \nu)
&:=
\inf_{\gamma}
\int_{X \times Y} c(x,y)\, d\gamma(x,y)
,\quad
\forall \gamma \in \Gamma(\mu, \nu)
,
\label{mongekant:eq:kantorowitch_problem}
\end{align}
Diese Umformulierung hat drei wesentliche Vorteile:
\begin{enumerate}
\item \textbf{Linearität.}
Das Problem \eqref{mongekant:eq:kantorowitch_problem}
ist eine lineare Optimierung,
da sowohl die Zielfunktion
als auch die Randbedingungen \eqref{mongekant:eq:kantorowitch_marginals}
linear in $\gamma$ sind.
Hingegen ist das Monge-Problem \eqref{mongekant:eq:monge_problem}
stark nichtlinear in $T$.
\item \textbf{Existenz.}
Durch die Relaxation zu einer linearen Optimierung
über die konvexe Menge $\Gamma(\mu,\nu)$ kann man,
gemäss \cite{mongekant:ethlecture},
bereits unter sehr schwachen Annahmen
einen optimalen Plan $\gamma^{\ast}$ garantieren.
\item \textbf{Dualität.}
Das lineare Programm lässt sich in ein \emph{duales} Problem umformen,
was zu weiteren Einsichten in das Problem führt.
In Abschnitt~\ref{mongekant:subsection:kantorowitch_duality} werden
wir das duale Problem herleiten.
\end{enumerate}

\subsection{Zusammenhang zu Monge-Formulierung%
\label{mongekant:subsection:monge_kantorowitch_connection}}
Als nächstes wollen wir untersuchen,
wie die Monge- und Kantorowitch-Formulierung zusammenhängen und
ob sie ähnliche Ergebnisse liefern.
Dafür nehmen wir an,
dass eine optimale Abbildung $T\colon X \to Y$ für das Monge-Problem existiert,
für die \eqref{mongekant:eq:monge_problem} minimal wird.
Wählen wir jetzt
\begin{align}
d\gamma(x,y)
&=
d\mu(x) \delta_{y=T(x)}
\label{mongekant:eq:plan_from_map}
\end{align}
und setzen diesen Ausdruck
in die Randbedingungen \eqref{mongekant:eq:kantorowitch_marginals} ein,
\begin{align*}
\gamma(A \times Y)
&=
\int_A \delta_{T(x) \in Y}\, d\mu(x)
\\
&=
\mu(A)
\\
\gamma(X \times B)
&=
\int_X \delta_{T(x) \in B}\, d\mu(x)
\\
&=
\mu\left(T^{-1}[B]\right)
\\
&=
\nu(B)
,
\end{align*}
sehen wir,
dass $\gamma(x,y)$ diese Bedingungen erfüllt.
Jetzt können wir also \eqref{mongekant:eq:plan_from_map} in
\eqref{mongekant:eq:kantorowitch_problem} einsetzen:
\begin{align*}
\int_{X \times Y} c(x,y)\, d\gamma(x,y)
&=
\int_X \int_Y c(x,y) \delta_{y=T(x)}\, dy\, d\mu(x)
\\
&=
\int_X c(x, T(x))\, d\mu(x)
.
\end{align*}
Das entpricht genau den Kosten in \eqref{mongekant:eq:monge_transport_cost}.
Daraus folgt,
\begin{align*}
C_K(\mu, \nu)
&\leq
C_M(\mu, \nu)
.
\end{align*}
Somit liefert die Kantorowitch-Formulierung immer mindestens so gute Ergebnisse
wie die Monge-Formulierung,
da jede Abbildung $T$ einen zulässigen Plan $\gamma$ definiert.
Berücksichtigt man zudem,
dass die Kantorowitch-Formulierung weniger strenge Anforderungen an das Problem stellt,
ergibt sich,
ein klarer Vorteil für die Kantorowitch-Formulierung.

\subsection{Dualität im Kantorowitch-Problem%
\label{mongekant:subsection:kantorowitch_duality}}

Um das duale Problem herzuleiten,
starten wir mit dem Kantorowitch-Problem \eqref{mongekant:eq:kantorowitch_problem}
und versuchen die Randbedingungen \eqref{mongekant:eq:kantorowitch_marginals}
mit Lagrangemultiplikatoren $\xi \colon X \to \mathbb{R}$ und
$\eta \colon Y \to \mathbb{R}$ implizit in das Problem einzubauen,
so dass $\gamma$ nur noch eine Wahrscheinlichkeitsmasse im Gebiet $X\times Y$ sein muss.
\begin{align*}
\begin{aligned}
C_K(\mu, \nu)
&=
\inf_{\gamma \in \Gamma} \int_{X \times Y} c(x,y)\, d\gamma
\\
&=
\inf_{\gamma \geq 0} \int_{X \times Y} c(x,y)\, d\gamma
+ \sup_{\xi, \eta} L(\xi, \eta)
\\
&=
\inf_{\gamma \geq 0} \int_{X \times Y} c(x,y)\, d\gamma
\\
&\quad
+ \sup_{\xi, \eta} \Biggl(
\underbrace{
\int_X \xi(x)\, d\mu(x) - \int_{X \times Y} \xi(x)\, d\gamma
}_{\gamma(A \times Y) = \mu(A)}
+ \underbrace{
\int_Y \eta(x)\, d\nu(x) - \int_{X \times Y} \eta(x)\, d\gamma
}_{\gamma(X \times B) = \nu(B)}
\Biggr)
.
\end{aligned}
\end{align*}
Schauen wir uns den Term $\sup_{\xi, \eta} L(\xi, \eta)$ genauer an,
\begin{align*}
\sup_{\xi, \eta} L(\xi, \eta)
&=
\begin{cases}
0,
& \text{falls } \gamma \in \Gamma(\mu, \nu) \\
+\infty,
& \text{sonst}
\end{cases}
.
\end{align*}
sehen wir,
dass dieser Term implizit die Randbedingungen
\eqref{mongekant:eq:kantorowitch_marginals} wirklich erzwingt.
Verschieben wir jetzt das Supremum nach aussen,
was erlaubt ist,
da das Supremum über $\xi$ und $\eta$ nicht von $\gamma$ abhängt,
und fassen wir die Integrale zusammen,
ergibt sich
\begin{align*}
C_K(\mu, \nu)
&=
\inf_{\gamma \geq 0}
\sup_{\xi, \eta}
\int_X \xi(x)\, d\mu(x)
+ \int_Y \eta(y)\, d\nu(y)
+ \int_{X \times Y} \bigl(c(x,y) - \xi(x) - \eta(y)\bigr)\, d\gamma
.
\end{align*}
Vertauschen wir jetzt Infimum und Supremum,
wobei die Bedigungen für die Vertauschung gemäss \cite{mongekant:villani} erfüllt sind,
erhalten wir
\begin{align*}
C_K(\mu, \nu)
&=
\sup_{\xi, \eta}
\inf_{\gamma \geq 0} \int_X \xi(x)\, d\mu(x)
+ \int_Y \eta(y)\, d\nu(y)
+ \int_{X \times Y} \bigl(c(x,y) - \xi(x) - \eta(y)\bigr)\, d\gamma
\\
&=
\sup_{\xi, \eta} \int_X \xi(x)\, d\mu(x)
+ \int_Y \eta(y)\, d\nu(y)
+ \underbrace{
\inf_{\gamma \geq 0} \int_{X \times Y} \bigl(c(x,y) - \xi(x) - \eta(y)\bigr)\, d\gamma
}_{\Xi}
.
\end{align*}
Die Verschiebung des Infimums nach innen ist erlabut,
da die ersten beiden Integrale nicht von $\gamma$ abhängen.
Betrachten wir jetzt den Term
\begin{align*}
\Xi
&=
\begin{cases}
0,
& \text{falls } c(x,y) - \xi(x) - \eta(y) \geq 0
,\quad\forall (x,y) \in X \times Y
\\
-\infty,
& \text{sonst}
\end{cases}
,
\end{align*}
sehen wir,
dass dieser Term die Bedingung
\begin{align*}
\xi(x) + \eta(y) \leq c(x,y)
,\quad
\forall (x,y) \in X \times Y
\end{align*}
für eine optimale Lösung forciert.
Somit erhalten wir nun die duale Beziehung
\begin{align*}
C_K(\mu, \nu)
&=
\sup_{\xi, \eta}
\int_X \xi(x)\, d\mu(x)
+ \int_Y \eta(y)\, d\nu(y)
,\quad\text{wobei }
\xi(x) + \eta(y) \leq c(x,y)
.
\label{mongekant:eq:kantorowitch_dual}
\end{align*}
Die Funktionen $\xi(x)$ und $\eta(y)$ können dabei als
\emph{Abhol-} bzw. \emph{Lieferpreise} interpretiert werden.

\subsubsection{Interpretation des dualen Problems}
Um die Interpretation des dualen Problems zu verdeutlichen,
betrachten wir das folgende Beispiel:

Wir betrachten ein Unternehmen,
dass Kaffekapseln herstellt und in.
Insgesamt kostet es das Unternehmen $c(x,y)$ Franken,
um eine Schachtel der nötigen Kaffekapseln von Ort $x$ zu Ort $y$ zu transportieren,
also von den Lagern zu den Haushalten.
Das Unternehmen möchten diese teure Gewohnheit optimieren und
dafür das passende Kantorowitch-Problem formulieren.
Mathematiker kommen zum Unternehmen und schlagen ein neues Zahlungsmodell vor.
Für jede Schachtel,
die an Ort $x$ liegt,
verlangen sie $\xi(x)$ Franken für die Abholung,
und für die Lieferung an Ort $y$ verlangen sie $\eta(y)$ Franken.
Allerdings werden die Mathematiker die konkreten Versandrouten nicht preisgeben.
Damit das Unternehmen dieses Angebot akzeptiert,
muss selbstverständlich gelten $\xi(x)+\eta(y) \leq c(x,y)$.


Die Idee ist:
Wenn die Mathematiker klug genug sind,
können sie die Sendung günstiger machen.
Das wird durch den \emph{Kantorowitch-Dualitätssatz} garantiert.

\emph{Hinweis.}
In manchen Fällen können die Mathematiker sogar \emph{negative} Preise ansetzen.
Das bedeutet,
dass sie dem Unternehmen Geld bezahlen,
um die Sendung an einem bestimmten Ort abzuholen bzw. zu liefern.
Damit können die Gesamtkosten gesenkt werden.

\subsection{Lineare Programmierung%
\label{mongekant:subsection:linear_programming}}

Betrachten wir nun ein diskretes Transportproblem,
bei dem
\begin{align*}
\mu
&=
\sum_{i=1}^m \alpha_i \delta_{x_i}
,\quad\text{wobei }
\sum_{i=1}^m \alpha_i = 1
\text{ und }
\alpha_i \geq 0
\\
\nu
&=
\sum_{j=1}^n \beta_j \delta_{y_j}
,\quad\text{wobei }
\sum_{i=1}^m \beta = 1
\text{ und }
\beta_j \geq 0
.
\end{align*}
Zudem vereinfachen wir die Notation mit $c_{ij} = c(x_i, y_j)$ und
$\gamma_{ij} = \gamma(x_i, y_j)$.
Gehen wir davon aus,
dass ein optimaler Plan $\gamma_{ij}$ existiert,
dann können wir die Kantorowitch-Formulierung \eqref{mongekant:eq:kantorowitch_problem} schreiben als
\begin{align*}
C_K(\mu, \nu)
&=
\min_{\gamma_{ij}}
\sum_{i=1}^m \sum_{j=1}^n c_{ij} \gamma_{ij}
,\quad
\text{wobei }
\sum_{j=1}^n \gamma_{ij} = \alpha_i
\text{ und }
\sum_{i=1}^m \gamma_{ij} = \beta_j
.
\end{align*}
Dies entspricht einem linearen Programm,
wie man es aus der Optimierung kennt.
Tatsächlich wird Kantorowitch als der Begründer der Linearen Programmierung angesehen.
