%
% 2-koordinatenfrei.tex -- Koordinatenfreie Definition
%
% (c) 2025 Prof Dr Andreas Müller
%
\section{Koordinatenfreie Definition
\label{buch:hodge:section:koordinatenfrei}}
\kopfrechts{Koordinatenfreie Definition}
Der Hodge-Operators ist bis jetzt explizit mit Hilfe eines
Koordinatensystems definiert worden.
Es ist in keiner Weise offensichtlich, dass die Konstruktion auf
den Basis-$p$-Formen zum richtigen Koordinatentransverhalten führen
könnte.
In diesem Abschnitt wird daher eine weitere Definition gegeben,
die offensichtlich koordinatenunabhängig ist.
Sie zeigt auch, dass zur Definition des Hodge-Operators als
zusätzliche Eigenschaft der Mannigfaltigkeit ein Skalarprodukt
benötigt wird, die in den bisherigen Entwicklungen keine Rolle
gespielt hat.
Dies ist keine wesentliche Einschränkung der Nützlichkeit der
Definition, denn in allen praktischen Anwendungsfällen ist ein
Skalarprodukt automatisch gegeben oder kann ohne weitere Konsequenzen
verfügbar gemacht werden.

%
% Hodge-Operator und Skalarprodukt
%
\subsection{Hodge-Operator und Skalarprodukt
\label{buch:hodge:koordinatenfrei:subsection:skalarprodukt}}
Der Hodge-Operator ist eine lineare Abbildung, die 
$p$-Formen auf $n-p$-Formen abbildet.
Die Definition auf Basis-$p$-Formen in einem Koordinatensystem
hat verlangt, dass zu einer $p$-Form $\omega=dx^{i_1}\wedge\dots\wedge dx^{i_p}$
eine Basis-$n-p$-Form $\ast\omega$ so gefunden werden muss, so dass
\begin{equation}
\omega\wedge {\ast\omega}
=
dx^1\wedge\dots\wedge dx^n
\label{buch:hodge:skalarprodukt:eqn:def}
\end{equation}
ergibt.
Auf der rechten Seite von \eqref{buch:hodge:skalarprodukt:eqn:def}
wird die vom gewählten Koordinatensystem abhängige $n$-Form
$dx^1\wedge\dots\wedge dx^n$ verwendet.
Da der Raum der $n$-Formen eindimensional ist, wäre jedes Vielfache
dieser $n$-Form genauso als Basis für die Definition zulässig.
Eine Koordinatentransformation multipliziert die $n$-Form mit der
Funktionaldeterminante der Transformation.
Eine koordinatenunabhängige Definition des Hodge-Operators braucht
daher als Basis die Wahl einer $n$-Form.
Wir nehmen daher im Folgenden an, dass eine $n$-Form $\nu$ gegeben
ist.
Da das Integral einer $n$-Form das $n$-dimensionale Volumen misst,
nennen wir sie auch die {\em Volumenform} auf der Mannigfaltigkeit.

Auf der linken Seite von \eqref{buch:hodge:skalarprodukt:eqn:def}
treten in der Definition nur Basis-$p$-Formen auf.
Lässt man eine beliebige $p$-Formen $\alpha$ und $\beta$ zu, dann
entsteht die $n$-Form $\alpha\wedge{\ast\beta}$, die natürlich ein
Vielfaches der Volumenform $\nu$ sein muss.
Es gibt also eine Zahl $s(\alpha,\beta)$ derart, dass
\begin{equation}
\alpha \wedge {\ast\beta} = s(\alpha,\beta)\,\nu
\label{buch:hodge:skalarprodukt:eqn:salphabeta}
\end{equation}
gilt.
Das Wedge-Produkt auf der linken Seite erfüllt das Distributivgesetz,
die linke Seite ist also linear in $\alpha$.
Der Hodge-Operator wurde als linearer Operator definiert, so dass 
die linke Seite auch linear ist in $\beta$.
Die Funktion $s(\alpha,\beta)$ muss daher bilinear sein.

Für Basis-$p$-Formen $\alpha=dx^{i_1}\wedge\dots\wedge dx^{i_p}$
und $\beta=dx^{j_1}\wedge\dots\wedge dx^{j_p}$ ergibt die
Definition~\eqref{buch:hodge:skalarprodukt:eqn:salphabeta}
\[
s(\alpha,\beta)
=
s(
dx^{i_1}\wedge\dots\wedge dx^{i_p},
dx^{j_1}\wedge\dots\wedge dx^{j_p}
)
=
\begin{cases}
1&\qquad\text{falls }i_1=j_1,\dots,i_p=j_p\\
0&\qquad\text{sonst.}
\end{cases}
\]
Dies bedeutet, dass die Funktion $s(\alpha,\beta)$ ein Skalarprodukt
ist, in dem die Basis-$p$-Formen orthonormiert sind.
Wir schreiben die Funktion daher im folgenden als
$s(\alpha,\beta)=\langle\alpha,\beta\rangle$, die Eigenschaft
\eqref{buch:hodge:skalarprodukt:eqn:salphabeta}
wird daher zu
\begin{equation}
\alpha\wedge {\ast\beta}
=
\langle\alpha,\beta\rangle\,\nu.
\label{buch:hodge:skalarprodukt:eqn:skalar}
\end{equation}
In dieser Form kommt das Koordinatensystem nicht mehr vor.
Die Definition~\ref{buch:hodge:hodge:definition:hodge}
war so einfach, weil wir von einem kartesischen Koordinatensystem
ausgegangen sind, in dem es plausibel ist, dass die Basis-1-Formen
orthonormiert sind.

Es soll jetzt gezeigt werden, dass
\eqref{buch:hodge:skalarprodukt:eqn:skalar}
den Hodge-Operator eindeutig definiert.
Es muss also gezeigt werden, dass zu einer gegebenen $p$-Form
$\beta$ genau eine $n-p$-Form ${\color{darkred}\gamma}$ gibt, so dass
\begin{equation}
\alpha\wedge {\color{darkred}\gamma}
=
\langle\alpha,\beta\rangle\,\nu
\label{buch:hodge:skalarprodukt:def:gamma}
\end{equation}
für alle $p$-Formen $\alpha$.
Für ein Skalarprodukt verschwindet die rechte
Seite von \eqref{buch:hodge:skalarprodukt:def:gamma}
nicht identisch, es ist aber weniger klar, dass dies auch für die
linke Seite gilt.

\begin{satz}
\label{buch:hodge:skalarprodukt:satz:nullteilerfrei}
Das Wedge-Produkt
\[
\Omega^p \times \Omega^{n-p}
\to
\Omega^n
:
(\alpha,{\color{darkred}\gamma})
\mapsto
\alpha\wedge{\color{darkred}\gamma}
\]
ist nicht entartet, d.~h. wenn $\alpha\wedge{\color{darkred}\gamma}=0$
ist, dann ist einer der Faktoren 0.
\end{satz}

\begin{proof}
Die Behauptung des Satzes ist gleichbedeutend mit der Aussage, dass 
sich zu jeder $p$-Form $\alpha\ne 0$ eine $n-p$-Form ${\color{darkred}\gamma}$
finden lässt, so dass $\alpha\wedge{\color{darkred}\gamma}\ne  0$ ist.
Wir zeigen dies mit Hilfe einer Basis.

Ist $\lambda^1,\dots,\lambda^n$ eine Basis von 1-Formen, dann bilden
die
\[
\lambda^{i_1}\wedge\dots\wedge\lambda^{i_p},\quad i_1<\dots<i_p
\]
eine Basis der $p$-Formen und die
\[
\lambda^{j_1}\wedge\dots\wedge\lambda^{j_{n-p}},\quad j_1<\dots<j_{n-p}
\]
eine Basis der $n-p$-Formen.
Eine $p$-Form $\alpha$ ist eine Linearkombination
\[
\alpha
=
\sum_{i_1<\dots<i_p}
a_{i_1\dots i_p} \lambda^{i_1}\wedge\dots\wedge \lambda^{i_p}.
\]
Sei $i_1,\dots,i_p$ eine Wahl von Indizes so, dass
der $a_{i_1\dots i_p}\ne 0$ ist, und seien die aufsteigend geordneten
Indizes $j_1,\dots,j_{n-p}$ so gewählt, dass
\[
\{i_1,\dots,i_p\}\cup\{j_1,\dots,j_{n-p}\} = [n].
\]
Dann ist
\[
\alpha\wedge(\lambda^{j_1}\wedge\dots\wedge\lambda^{j_{n-p}})
=
\pm
a_{i_1\dots i_p} \lambda^1\wedge\dots\wedge\lambda^n
\ne
0.
\]
${\color{darkred}\gamma}=\lambda^{j_1}\wedge\dots\wedge\lambda^{j_{n-p}}$
erfüllt also die Forderungen.
\end{proof}

Aus Satz~\ref{buch:hodge:skalarprodukt:satz:nullteilerfrei} kann jetzt
abgeleitet werden, dass sich immer eine $n-p$-Form ${\color{darkred}\gamma}$
finden lässt, für die $\alpha\wedge{\color{darkred}\gamma}=
\langle\alpha,\beta\rangle\,\nu$ für alle $\alpha$ gilt.
Wir zeigen dies, indem wir ausnutzen, dass der Vektorraum der
$n-p$-Formen endlichdimensional ist und sich damit das Problem
auf eine Aufgabe über lineare Gleichungssysteme reduzieren lässt.
Zur Konstruktion des Gleichungssystems definieren zunächst die
Abbildung $a_\alpha$ wie folgt.

\begin{definition}
Sei $a_\alpha$ die lineare Abbildung
\[
a_\alpha
\colon
\Omega^{n-p}
\to
\mathbb{R}
:
{\color{darkred}\gamma}
\mapsto
a_\alpha({\color{darkred}\gamma})\,\nu
\]
für die
\[
\alpha\wedge{\color{darkred}\gamma}
=
a_\alpha({\color{darkred}\gamma})\,\nu
\]
ist.
\end{definition}

Die Funktion $a_\alpha({\color{darkred}\gamma})$ ist eine lineare
Funktion von ${\color{darkred}\gamma}$.
Wir möchten zeigen, dass sich damit ein bijektive Abbildung zwischen
endlichdimensionalen Vektorräumen konstruieren lässt.

\begin{satz}
\label{buch:hodge:skalarprodukt:satz:endlich}
Ist $\omega_i$, $i=1,\dots,N$ eine Basis von $p$-Formen, dann ist die
lineare Abbildung
\[
f
\colon
\Omega^{n-p}
\to
\mathbb{R}^N
:
{\color{darkred}\gamma}
\mapsto
\begin{pmatrix}
a_i({\color{darkred}\gamma})\\
\vdots\\
a_N({\color{darkred}\gamma})
\end{pmatrix}
\]
eine Bijektion.
\end{satz}

\begin{proof}
Gäbe es ein ${\color{darkred}\gamma}$ mit $f({\color{darkred}\gamma})=0$,
dann ist $\omega_i\wedge{\color{darkred}\gamma}=0$ für jedes $i$.
Da die $\omega_i$ eine Basis bilden, folgt
$\omega\wedge{\color{darkred}\gamma}=0$ für alle $p$-Formen $\omega$.
Dies widerspricht aber dem
Satz~\ref{buch:hodge:skalarprodukt:satz:nullteilerfrei}.
Somit kann es keine solches ${\color{darkred}\gamma}$ geben.
Da der Kern von $f$ nur aus der Nullform besteht, ist $f$ injektiv.
Da die beiden Vektorräume die gleiche Dimension haben, ist $f$ bijektiv.
\end{proof}

Um zu zeigen, dass mit \eqref{buch:hodge:skalarprodukt:eqn:skalar}
tatsächlich der Hodge-Operator definiert werden kann, müssen wir
nachweisen, dass das Skalarprodukt auf $p$-Formen den Hodge-Operator
vollständig festlegt.

\begin{satz}
\label{buch:hodge:satz:eindeutigkeit}
Ist $\langle\alpha,\beta\rangle$ eine nicht entartete Bilinearform
auf $p$-Formen, und $\nu$ eine $n$-Form, dann gibt es eine lineare
Abbildung $\beta\mapsto \ast\beta$ von $p$-Formen in $n-p$-Formen
derart, dass
\begin{equation}
\alpha\wedge {\ast\beta}
=
\langle\alpha,\beta\rangle\,\nu.
\label{buch:hodge:skalarprodukt:satz:eqn}
\end{equation}
Der Operator $\ast$ ist durch
\eqref{buch:hodge:skalarprodukt:satz:eqn}
eindeutig bestimmt.
\end{satz}

\begin{proof}
In einer Basis $\omega_i$ der $p$-Formen wie in
Satz~\ref{buch:hodge:skalarprodukt:satz:endlich}
ist ${\color{darkred}\gamma}$ durch 
\[
f({\color{darkred}\gamma})
=
\begin{pmatrix}
\langle \omega_1,\beta\rangle\\
\vdots\\
\langle \omega_N,\beta\rangle
\end{pmatrix}
\]
definiert.
Da $f$ bijektiv ist, ist ${\color{darkred}\gamma}$ eindeutig
bestimmt und kann durch Lösung eines linearen Gleichungssystems
gefunden werden.
\end{proof}

%
% Skalarprodukt auf Formen
%
\subsection{Skalarprodukt auf Formen}
Der Definitionsansatz~\eqref{buch:hodge:skalarprodukt:def:gamma}
für den Hodge-Operator verlangt, dass sich für beliebige $p$-Formen
ein Skalarprodukt definieren lässt.
Auf einer beliebigen Mannigfaltigkeit kann man nicht einmal für
Tangentialvektoren von der Existenz eines Skalarprodukts ausgehen.
Für die nachfolgende Konstruktion setzt daher voraus, dass auf
der Mannigfaltigkeit eine Metrik gegeben ist, die durch den 
metrischen Tensor $g_{ik}$ beschrieben wird.

%
% Volumenform
%
\subsubsection{Volumenform}
Für die Definition wird die Definition des Volumens benötigt.
Zu einem $n$-Vektor $X_1\wedge\dots\wedge X_n$ muss diese $n$-Form
das Volumen des von diesen Vektoren aufgespannten infinitesimalen
Parallelepipeds bestimmen.
Das Volumen wird durch die Gram-Determinante
(siehe \cite[Abschnitt 8.4]{buch:linalg}).
\begin{align*}
\operatorname{vol}(X_1,\dots,X_n)^2
&=
\operatorname{Gram}(X_1,\dots,X_n)
\\
&=
\left|
\begin{matrix}
\langle X_1,X_1\rangle
	&\langle X_1,X_2\rangle
	&\dots
	&\langle X_1,X_n\rangle
\\
\langle X_2,X_1\rangle
	&\langle X_2,X_2\rangle
	&\dots
	&\langle X_2,X_n\rangle
\\[-2pt]
\vdots
	&\vdots
	&\ddots
	&\vdots
\\
\langle X_n,X_1\rangle
	&\langle X_n,X_2\rangle
	&\dots
	&\langle X_n,X_n\rangle
\end{matrix}
\right|
\intertext{der Vektoren $X_1,\dots,X_n$ gegeben.
Wählt man für die Vektoren die Standardbasisvektoren
$X_k=\partial/\partial x^k$ sind die Skalarprodukt nach
Definition des metrischen Tensors durch die Matrixelement
$g_{ik}$ gegeben.
Somit ist das Volumen}
&=
\left|
\begin{matrix}
g_{11} & g_{12} & \dots  & g_{1n} \\
g_{21} & g_{22} & \dots  & g_{2n} \\
\vdots & \vdots & \ddots & \vdots \\
g_{n1} & g_{n2} & \dots  & g_{nn}
\end{matrix}
\right|
=
\det(g_{ik}).
\end{align*}
Die Determinante der Matrix des metrischen Tensors $g_{ik}$ wird auch
auch mit $g=\det(g_{ik})$ bezeichnet.
Die Volumenform ist daher
\[
\nu
=
\sqrt{g}\, dx^1\wedge\dots\wedge dx^n.
\]

\begin{beispiel}
In Polarkoordinaten $(r,\varphi)$ ist die Metrik durch die
Matrix
\[
(g_{ik})
=
\begin{pmatrix}
1 & 0 \\
0 & r^2
\end{pmatrix}
\]
gegeben.
Die Determinante ist
\[
g
=
\det(g_{ik})
=
r^2
\qquad
\Rightarrow
\qquad
\nu
=
\sqrt{g}\, dr\wedge d\varphi
=
r\,dr\wedge d\varphi.
\]
Dies ist das bekannte Flächenelement in Polarkoordinaten.
\end{beispiel}

\begin{beispiel}
In Kugelkoordinaten $(r,\vartheta,\varphi)$ ist die Metrik durch die
Matrix
\[
(g_{ik})
=
\begin{pmatrix}
1 &  0  & 0 \\
0 & r^2 & 0 \\
0 &  0  & r^2 \sin^2\varphi
\end{pmatrix}
\qquad
\text{mit der Determinanten}
\qquad
g = \det(g_{ik}) = r^4\sin^2\varphi
\]
gegeben.
Tatsächlich ist das Volumenelement 
\[
\nu 
=
r^2 \sin\varphi\,dr\wedge d\vartheta\wedge d\varphi
\]
in Kugelkoordinaten.
\end{beispiel}

%
% Ableitungen von $g^{ik}$
%
\subsubsection{Ableitungen von $g^{ik}$}
Für spätere Anwendungen studieren wir hier noch ein paar Identitäten
für die Ableitungen von $g$ und $^{ik}$.

Die Matrix mit Einträgen $g^{ik}$ ist invers zur Matrix mit
Einträgen $g_{ik}$.
Die Produktmatrix ist daher die Einheitsmatrix, in Komponenten gilt
\begin{equation*}
g^{ik}g_{kl}
=
\delta^i_l.
\end{equation*}
Die Ableitung nach der Koordinaten $x^m$ ist nach der Produktregel
\begin{equation*}
\frac{\partial g^{ik}}{\partial x^m} g_{kl}
+
g^{ik} \frac{\partial g_{kl}}{\partial x^m}
=
0
\qquad\Rightarrow\qquad
\frac{\partial g^{ik}}{\partial x^m} g_{kl}
=
-
g^{ik} \frac{\partial g_{kl}}{\partial x^m}.
\end{equation*}
Die Ableitung von $g^{ik}$ kann daraus bestimmt werden, indem auf
beiden Seiten mit der Inversen von $g_{kl}$ multipliziert wird.
Multiplikation mit der $g^{l\!j}$ auf beiden Seiten ergibt
\begin{align*}
\frac{\partial g^{ik}}{\partial x^m} g_{kl}  g^{lj}
&=
-
g^{ik} \frac{\partial g_{kl}}{\partial x^m} g^{lm}
\intertext{und wegen $g_{kl}g^{l\!j}=\delta^j_k$ folgt}
\frac{\partial g^{i\!j}}{\partial x^m}
&=
-
g^{ik}
g^{l\!j}
\frac{\partial g_{kl}}{\partial x^m}.
\end{align*}
Die Ableitung der kontravarianten metrischen Koeffizienten entstehen
also durch Hochziehen der Indizes der Ableitung der kovarianten
metrischen Koeffizienten und einen Vorzeichenwechsel.
Der Vorzeichenwechsel erinnert daran, dass Ableiten und Hochziehen
der Herunterziehen eines Index nicht vertauschen müssen.

%
% Ableitungen von $g$
%
\subsubsection{Ableitungen von $g$}
Die Determinanten $g$ ist eine algebraische Funktion allein der 
metrischen Koeffizienten $g_{ik}$.
Die Ableitung der Determinante $g$ nach einer Koordinate kann daher
mit der Kettenregel als
\begin{equation*}
\frac{\partial g}{\partial x^m}
=
\sum_{i,k=1}^n
\frac{\partial g}{\partial g_{ik}}
\frac{\partial g_{ik}}{\partial x^m}
\end{equation*}
geschrieben werden.
Die partiellen Ableitungen von $g$ nach den $g_{ik}$ lassen sich
mit dem laplaceschen Entwicklungssatz berechnen.
Wir bezeichnen mit $G_{ik}$ die Minormatrix von $g_{ik}$, in der
die Zeile $i$ und die Spalte $k$ weggelassen sind.
Dann ist 
\begin{align*}
g
&=
\sum_{l=1}^n (-1)^{l+k} g_{lk} \det G_{lk}
&&\text{die Entwicklung nach Zeile $i$ und}
\\
&=
\sum_{l=1}^n (-1)^{i+l} g_{il} \det G_{il}
&&\text{die Entwicklung nach Spalte $k$.}
\end{align*}
In der Minormatrix $G_{ik}$ kommen die $g_{il}$ und $g_{lk}$ für
$l=1,\dots,n$, nicht vor.
Insbesondere kommt $g_{ik}$ nur in einem einzigen Term vor und es
folgt
\[
\frac{\partial g}{\partial g_{ik}}
=
(-1)^{i+k}
\det G_{ik}.
\]
Andererseits lassen sich auch die Einträge $g^{ik}$ der inversen
Matrix als
\[
g^{ik} = \frac{1}{g} (-1)^{i+k} \det G_{ik}
\]
schreiben.
Es folgt, dass
\[
\frac{\partial g}{\partial g_{ik}}
=
g\, g^{ik}
\]
ist.
Die Ableitung nach $x^m$ ist daher
\[
\frac{\partial g}{\partial x^m}
=
\sum_{i,k=1}^n g\,g^{ik} \frac{\partial g_{ik}}{\partial x^m}.
\]
Diese Eigenschaft lässt sich auch als das Differential
\begin{equation}
dg
=
g\,g^{ik}\,dg_{ik}
\label{buch:hodge:koorinatenfrei:eqn:dg}
\end{equation}
schreiben.

Da $g^{ik}g_{kl}=\delta^i_l$ gilt, ist
\[
g^{ik}g_{ki} = \delta^i_i = n
\]
konstant.
Das Differential ist nach der Kettenregel
\[
g_{ki}\,dg^{ik}
+
g^{ik}\,dg_{ik}
=
0
\qquad\Rightarrow\qquad
g_{ki}\,dg^{ik}
=
-
g^{ik}\,dg_{ik}.
\]
Das Differential $dg$ kann daher ausser in der Form
\eqref{buch:hodge:koorinatenfrei:eqn:dg}
auch in der Form
\begin{equation}
dg
=
g\,g^{ik}\,dg_{ik}
=
-
g\,g_{ik}\,dg^{ik}
\label{buch:hodge:koorinatenfrei:eqn:dg2}
\end{equation}
geschrieben werden.

%
% Ableitung einer Determinanten
%
\subsubsection{Ableitung der Determinanten}
Die explizite Berechnung der Ableitung der Determinangen $g$ in
Komponenten kann auch in für eine beliebige matrixwertige Funktion
\[
A
\colon
\mathbb{R}\to M_{n\times n}(\mathbb{R})
:
t\mapsto A(t)
\]
durchgeführt werden.
Dazu berechnet man
\begin{align*}
\frac{d}{dt}
\det A(t)
&=
\lim_{h\to 0}
\frac{\det A(t+h) - \det A(t)}{h}
\\
&=
\det A(t)
\lim_{h\to 0}
\frac{\det A(t)^{-1}\bigl(\det A(t+h) - \det A(t)\bigr)}{h}
\\
&=
\det A(t)
\lim_{h\to 0}
\frac{\det(A(t)^{-1} A(t+h)) - 1)}{h}.
\intertext{Der Grenzwert auf der rechten Seite ist die Ableitung 
der Matrixfunktion $B(h)=A(t)^{-1}A(t+h)$ an der Stelle $h=0$
mit der Eigenschaft $B(0)=I$.}
&=
\det A(t)
\frac{d}{dh} \det(A(t)^{-1} A(t+h))\bigg|_{h=0}
\intertext{Eine solche Ableitung ist die Spur, es folgt daher, dass die
Ableitung von $\det A(t)$}
&=
\det A(t) \operatorname{Spur} \frac{d}{dh}A(t)^{-1}A(t+h)\bigg|_{h=0}
\\
&=
\det A(t) \operatorname{Spur}\biggl( A(t)\frac{dA(t)}{dt}\biggr).
\intertext{In Komponenten ausgeschrieben ist dies}
\frac{d}{dt}\det A(t)
&=
\det A(t)
\sum_{i,k=1}^n
a_{ik}(t) a_{ki}'(t),
\end{align*}
was für eine symmetrische Matrix mit den für $g$ gefundenen Formeln
übereinstimmt.


%
% Skalarprodukt der 1-Formen
%
\subsubsection{Skalarprodukt der 1-Formen}
Der metrische Tensor beschreibt das Skalarprodukt von Tangentialvektoren,
insbesondere ist
\[
\biggl\langle
\frac{\partial}{\partial x^i},\frac{\partial}{\partial x^k}
\biggr\rangle
=
g_{ik}.
\]
Benötigt wird jetzt das Skalarprodukt von 1-Formen.
Die Basisformen $dx^k$ sind dual zu den Basisvektoren, es gilt
\[
\biggl\langle dx^i,\frac{\partial}{\partial x^k}\biggr\rangle
=
\delta_{ik}.
\]
Das Skalarprodukt zweier beliebigen 1-Formen $a_i\,dx^i$ und
$b_k\,dx^k$ ist eine bilineare Funktion der Koeffizienten
$a_i$ und $b_i$, muss also durch einen kontravarianten 
Tensor $g^{ik}$ vermittelt werden.
Die Einträge $g^{ik}$ bilden eine Matrix, die zur Matrix $(g_{ik})$
invers ist.

%
% Skalarprodukt der p-Formen
%
\subsubsection{Skalarprodukt der $p$-Formen}
Um ein Skalarprodukt von $p$-Formen zu definieren, müssen in einem
Koordinatensystem die Skalarprodukt beliebiger Basis-$p$-Formen
\[
g^{i_1\dots i_pk_1\dots k_p}
=
\langle
dx^{i_1}\wedge\dots\wedge dx^{i_p}
,
dx^{k_1}\wedge\dots\wedge dx^{k_p}
\rangle
\]
bestimmt werden.
Falls zwei Indizes gleich sind, verschwindet das Skalarprodukt.
Vertauscht man zwei verschiedene Indizes in einer Basis-$p$-Form,
kehrt das Vorzeichen.
Eine bilineare Funktion mit diesen Eigenschaften ist ein Vielfaches
einer Bilinearform, die nach dem Muster der Gram-Determinante 
konstruiert wird.
Es muss
\[
g^{i_1\dots i_pk_1\dots k_p}
=
\left|
\begin{matrix}
\langle dx^{i_1}, dx^{k_1} \rangle
	&\langle dx^{i_1}, dx^{k_2} \rangle
	&\dots
	&\langle dx^{i_1}, dx^{k_p} \rangle
\\
\langle dx^{i_2}, dx^{k_1} \rangle
	&\langle dx^{i_2}, dx^{k_2} \rangle
	&\dots
	&\langle dx^{i_2}, dx^{k_p} \rangle
\\[-2pt]
\vdots
	&\vdots
	&\ddots
	&\vdots
\\
\langle dx^{i_p}, dx^{k_1} \rangle
	&\langle dx^{i_p}, dx^{k_2} \rangle
	&\dots
	&\langle dx^{i_p}, dx^{k_p} \rangle
\end{matrix}
\right|
\]
gesetzt werden.

%
% Definition des Hodge-Operators
%
\subsection{Definition des Hodge-Operators}
In den vorangegangenen Abschnitten wurde das Skalarprodukt auf
beliebigen $p$-Formen definiert.
Damit wird es jetzt möglich, den Hodge-Operator zu definieren.

\begin{definition}[Hodge-Operator]
\label{buch:hodge:koordinatenfrei:def:hodge-operator}
Der Hodge-Operator auf einer differenzierbaren Mannigfaltigkeit
mit einer Metrik ist eine lineare Abbildung, die $p$-Formen in
$n-p$-Formen abbildet.
Der Hodge-Operator $\ast\omega$ einer $p$-Form $\omega\in\Omega^p$
ist durch die Bedingung
\begin{equation}
\alpha \wedge (\ast\omega) = \langle \alpha,\omega\rangle \nu
\label{buch:hodge:koordinatenfrei:eqn:definition}
\end{equation}
für alle $p$-Formen $\alpha\in\Omega^p$ definiert.
\end{definition}

Die Existenz der $n-p$-Form $\ast\omega$ wird durch den
Satz~\ref{buch:hodge:skalarprodukt:satz:endlich} sichergestellt.
Satz~\ref{buch:hodge:satz:eindeutigkeit} garantiert, dass die
Bedingung~\eqref{buch:hodge:koordinatenfrei:eqn:definition}
die $n-p$-Form $\ast\omega$ eindeutig definiert ist.
Beide Sätze sind Koordinatenunabhängige Aussagen, sie verwenden
ein Koordinatensystem höchstens für den Beweis.
Damit legt die
Definition~\ref{buch:hodge:koordinatenfrei:def:hodge-operator}
den Hodge-Operator koordinatenfrei fest.

%
% Berechnung des Hodge-Operators
%
\subsection{Berechnung des Hodge-Operators}
Um den Hodge-Operator in einem beliebigen Koordinatensystem zu
bestimmen, ist wie folgt vorzugehen.
\begin{enumerate}
\item
Bestimmung der Metrik, die für Volumenform und Skalarprodukt
von $p$-Formen verwendet werden sollen.
\item
Bestimmung der Volumenform zu dieser Basis.
\item
Bestimmung des Skalarproduktes für 1-Formen.
\item
Erweiterung des Skalarproduktes auf $p$-Formen mit $p>1$.
\item
Berechnung des Hodge-Operators für Basis-$p$-Formen.
\end{enumerate}

%
% Hodge-Operator in Polarkoordinaten
%
\subsubsection{Hodge-Operator in Polarkoordinaten}
Wir betrachten Polarkoordinaten $(r,\varphi)$ auf der Ebene $\mathbb{R}^2$.
Die Umrechnung von Polarkoordinaten in kartesische Koordinaten ist durch
\begin{align*}
x^1 &= r\cos\varphi \\
x^2 &= r\sin\varphi
\end{align*}
gegeben.
Die Differentiale sind
\begin{align*}
dx^1 &= \cos\varphi\,dr - r\sin\varphi\,d\varphi \\
dx^2 &= \sin\varphi\,dr + r\cos\varphi\,d\varphi.
\end{align*}
Das Gleichungssystem kann man auch nach den Differentialen $dr$
und $d\varphi$ auflösen und erhalt
\begin{align*}
dr       &= \cos\varphi\, dx^1 + \sin\varphi\, dx^2 \\
d\varphi &= -\frac1r \sin\varphi\,dx^1 + \frac1r\cos\varphi\,dx^2.
\end{align*}
Zur Berechnung des Hodge-Operators gehen wir nach dem Rezept des 
vorangegangenen Abschnitts vor.
\begin{enumerate}
\item
{\bf Metrik:}
In kartesischen Koordinaten ist die Metrik in kartesichen Koordinaten
ist $g=(dx^1)^2+(dx^2)^2$, was in Polarkoordinaten zu
\begin{align*}
g
&=
(\cos\varphi\,dr - r\sin\varphi\,d\varphi)^2
+
(\sin\varphi\,dr + r\cos\varphi\,d\varphi)^2
\\
&=
\cos^2\varphi\,(dr)^2
-
2r\sin\varphi\cos\varphi\,dr\,d\varphi
+
r^2\sin^2\varphi\,(d\varphi)^2
\\
&+
\sin^2\varphi\,(dr)^2
+
2r\cos\varphi\sin\varphi\,dr\,d\varphi
+
r^2\cos^2\varphi\,(d\varphi)^2
\\
&=
(dr)^2 + r^2(d\varphi)^2.
\end{align*}
Der metrische Tensor in Matrixform ist
\[
g_{ik}
=
\begin{pmatrix}
1&0\\
0&r^2
\end{pmatrix}.
\]
\item {\bf Volumenform:}
Die Volumenform in Polarkoordinaten ist
\begin{align*}
\nu
=
dx^1\wedge dx^2
&=
(\cos\varphi\,dr - r\sin\varphi\,d\varphi)
\wedge
(\sin\varphi\,dr + r\cos\varphi\,d\varphi)
\\
&=
r \cos^2\varphi\,dr \wedge d\varphi
+
r \sin^2\varphi\,dr \wedge d\varphi
\\
&= r \,dr\wedge d\varphi.
\end{align*}
\item{\bf Skalarprodukt von $1$-Formen:}
Für das Skalarprodukt der 1-Formen muss die inverse Matrix
\[
g^{ik}
=
\begin{pmatrix}
1&0\\
0&\frac{1}{r^2}
\end{pmatrix}
\]
verwendet werden.
Die Determinante des metrischen Tensors ist $\det g = r^2$.
Die Basisformen $dr$ und $d\varphi$ sind orthogonal und haben die
Skalarprodukte
\begin{align*}
\langle dr,dr\rangle
&=
1
&
\langle d\varphi,d\varphi\rangle
&=
\frac{1}{r^2}.
\end{align*}
\item{\bf Skalarprodukt von $2$-Formen:}
Das Skalarprodukt von $dr\wedge d\varphi$ mit sich selbst wird durch die
Gram-Determinante
\[
\langle dr\wedge d\varphi,dr\wedge d\varphi\rangle
=
\left|
\begin{matrix}
\langle dr,dr\rangle         & \langle dr,d\varphi\rangle \\
\langle d\varphi, dr \rangle & \langle d\varphi,d\varphi\rangle
\end{matrix}
\right|
=
\left|
\begin{matrix}
1&0\\
0&\frac{1}{r^2}
\end{matrix}
\right|
=
\frac{1}{r^2}.
\]
der Skalarprodukte
\item{\bf Hodge-Operator:}
Für 1-Formen ergibt die Definition
\[
\renewcommand{\arraycolsep}{1.5pt}
\begin{array}{rcrcrclcrclclcrcl}
1\,\nu
&=&
\langle dr,dr\rangle\,\nu
&=&
dr &\wedge& {\ast\,dr}
&=&
dr &\wedge& a(r,\varphi)\,d\varphi
&=&
a(r,\varphi)
\frac1r
\, \nu
&\;\Rightarrow\;&
\ast\,dr
&=&
r\,d\varphi
\\
\frac{1}{r^2}\nu
&=&
\langle d\varphi,d\varphi\rangle\,\nu
&=&
d\varphi &\wedge& {\ast\,d\varphi}
&=&
d\varphi &\wedge& b(r,\varphi)\,dr
&=&
-b(r,\varphi)\frac{1}{r}\,\nu
&\;\Rightarrow\;&
\ast\,d\varphi
&=&
-\frac1r\, dr.
\end{array}
\]
Zur Kontrolle berecchnen wir den iterierten Hodge-Operator:
\[
\renewcommand{\arraycolsep}{1.5pt}
\begin{array}{rclclcl}
\ast{\ast\,dr}
&=&
\ast(r\,d\varphi)
&=&
r(-\frac1r\,dr)
&=&
-dr
\\
\ast{\ast\,d\varphi}
&=&
\ast(-\frac1{r}\,dr)
&=&
-\frac1r(r\, d\varphi)
&=&
-d\varphi.
\end{array}
\]
Der Hodge-Operator der 2-Form $dr\wedge d\varphi$ ist eine Funktion
$\ast(dr\wedge d\varphi)=c(r,\varphi)$, gemäss Definition muss gelten
\begin{align*}
dr\wedge d\varphi\wedge c(r,\varphi)
&=
c(r,\varphi)\,dr\wedge d\varphi
\\
&=
\langle dr\wedge d\varphi,dr\wedge d\varphi\rangle \nu
\\
&=
\frac{1}{r^2} r\,dr\wedge d\varphi
&&\Rightarrow&
c(r,\varphi)
&=
\ast(dr\wedge d\varphi)
=
\frac1r.
\end{align*}
Schliesslich ist $\ast 1$ die Zweiform $d(r,\varphi)\,dr\wedge d\varphi$,
nach Definition folgt
\begin{align*}
1\wedge{\ast 1}
&=
1\wedge d(r,\varphi)\,dr\wedge d\varphi
\\
&=
\langle 1,1\rangle \nu
=
r\,dr\wedge d\varphi
&&\Rightarrow&
d(r,\varphi)
&=
r
&&\Rightarrow&
\ast 1
&=
r\,dr\wedge d\varphi.
\end{align*}
Auch für diese Kombination berechnen wir den iterierten Hodge-Operator
\[
\renewcommand{\arraycolsep}{1.5pt}
\begin{array}{rclclclcl}
\ast{\ast 1}
&=&
\ast (r\,dr\wedge d\varphi)
&=&
r\,{\ast(dr\wedge d\varphi)}
&=&
r\cdot\frac1r
&=&
1
\\
\ast{\ast(dr\wedge d\varphi)}
&=&
\ast(\frac1r)
&=&
\frac1r\cdot({\ast\, 1})
&=&
\frac1r(r\,dr\wedge d\varphi)
&=&
dr\wedge d\varphi.
\end{array}
\]
In beiden Fällen ist der iterierte Hodge-Operator die identische
Abbildung.
\end{enumerate}

In kartesischen Koordinaten haben wir den Hodge-Operator schon früher
berechnet.
Durch Umrechnung der Formeln für Polarkoordinaten in kartesische Koordinaten
müssten sich die Formeln von Beispiel~\ref{buch:hodge:hodge:beispiel:r2} ergeben.
\begin{align*}
\ast dx^1
&=
\cos\varphi (\ast dr) - r\sin\varphi(\ast d\varphi)
\\
&=
\cos\varphi (r\,d\varphi) - r\sin\varphi(-{\textstyle\frac1r}\,dr)
\\
&=
r\cos\varphi d\varphi + \sin\varphi\,dr
\\
&=
r\cos\varphi(-{\textstyle\frac1r}\sin\varphi\,dx^1 + {\textstyle\frac1r}\cos\varphi\, dx^2)
+
\sin\varphi(\cos\varphi\,dx^1+\sin\varphi\,dx^2)
\\
&=
dx^2,
\\
\ast dx^2
&=
\sin\varphi\,(\ast dr) + r\cos\varphi\,(\ast d\varphi)
\\
&=
r\sin\varphi\,d\varphi + r\cos\varphi (-{\textstyle\frac1r})\,dr
\\
&=
r\sin\varphi (-{\textstyle\frac1r}\sin\varphi\,dx^1+{\textstyle\frac1r}\cos\varphi\,dx^2)
+
r\cos\varphi(-{\textstyle\frac1r})(\cos\varphi\,dx^1+\sin\varphi\,dx^2)
\\
&=
-dx^1,
\end{align*}
in Übereinstimmung mit den Resultaten von
Beispiel~\ref{buch:hodge:hodge:beispiel:r2}.

%
% Hodge-Operator in Kugelkoordinaten
%
\subsubsection{Hodge-Operator in Kugelkoordinaten}
Kugelkoordinaten $(r,\vartheta,\varphi)$ sind gegeben durch die
Umrechnungsformeln
\begin{equation}
\begin{aligned}
dx^1 &= r \sin\vartheta \cos\varphi \\
dx^2 &= r \sin\vartheta \sin\varphi \\
dx^3 &= r \cos\vartheta
\end{aligned}
\label{buch:hodge:skalarprodukt:eqn:kugelkoordinaten}
\end{equation}
mit den Differentialen
\begin{align*}
dx^1
&=
\sin\vartheta \cos\varphi \,dr
+
r \cos\vartheta \cos\varphi \,d\vartheta
-
r \sin\vartheta \sin\varphi \,d\varphi
\\
dx^2
&=
\sin\vartheta \sin\varphi \,dr
+
r \cos\vartheta \sin\varphi \,d\vartheta
+
r \sin\vartheta \cos\varphi \,d\varphi
\\
dx^3
&=
\cos\vartheta\,dr
-
r \sin\vartheta\,d\vartheta.
\end{align*}
Wie im vorangegangenen Abschnitt soll der Hodge-Operator in
Kugelkoordinaten berechnet werden.
\begin{enumerate}
\item {\bf Metrik:}
Wie bei den Plarkoordinaten kann die Metrik direkt aus 
\eqref{buch:hodge:skalarprodukt:eqn:kugelkoordinaten}
gewinnen.
\begin{align*}
g
&=
(dx^1)^2 + (dx^2)^2 + (dx^3)^2
\\
&=\phantom{+}
(
  \cos\varphi \sin\vartheta \,dr
+
r \cos\varphi \cos\vartheta \,d\vartheta
-
r \sin\varphi \sin\vartheta \,d\varphi
)^2
\\
&\phantom{=}\mathstrut+
(
  \sin\varphi \sin\vartheta \,dr
+
r \cos\varphi \sin\vartheta \,d\varphi
+
r \sin\varphi \cos\vartheta \,d\vartheta
)^2
\\
&\phantom{=}\mathstrut+
(
  \cos\vartheta \,dr
-
r \sin\vartheta \,d\vartheta
)^2
\\
&=\mathstrut\phantom{+}
\cos^2\varphi \sin^2\vartheta \, (dr)^2
+
r^2 \cos^2\varphi \cos^2\vartheta \, (d\vartheta)^2
+
r^2 \sin^2\varphi \sin^2\vartheta \, (d\varphi)^2
\\
&\phantom{=}\mathstrut-
2r \cos\varphi \sin\varphi \sin^2\vartheta \,dr\,d\varphi
+
2r \cos^2\varphi \sin\vartheta \cos\vartheta \,dr\,d\vartheta
\\
&\phantom{=}\mathstrut-
2r^2 \sin\varphi \cos\varphi \sin\vartheta \cos\vartheta \,d\varphi\,d\vartheta
\\
&\phantom{=}\mathstrut+
\sin^2\varphi\sin^2\vartheta \,(dr)^2
+
r^2\sin^2\varphi\cos^2\vartheta \,(d\vartheta)^2
+
r^2\cos^2\varphi\sin^2\vartheta \,(d\varphi)^2
\\
&\phantom{=}\mathstrut+
2r\sin\varphi\cos\varphi\sin^2\vartheta \,dr\,d\varphi
+
2r\sin^2\varphi\sin\vartheta\cos\vartheta \,dr\,d\vartheta
\\
&\phantom{=}\mathstrut+
2r^2\cos\varphi\sin\varphi\sin\vartheta\cos\vartheta \,d\varphi\,d\vartheta
\\
&\phantom{=}\mathstrut+
   \cos^2\vartheta \,(dr)^2
+
r^2 \sin^2\vartheta \,(d\vartheta)^2
-
2r \cos\vartheta \sin\vartheta \,dr\,d\vartheta
\\
&=
(dr)^2
+
r^2\, (d\vartheta)^2
+
r^2\sin^2\vartheta\,(d\varphi)^2
\end{align*}
Der metrische Tensor ist
\[
g
=
\begin{pmatrix}
1 &  0  &         0           \\
0 & r^2 &         0           \\
0 &  0  & r^2 \sin^2\vartheta
\end{pmatrix}
\]
der Determinante $\det g=r^4\sin^2\vartheta$ und der inversen Matrix
\[
g^{-1}
=
\begin{pmatrix}
1 & 0 & 0 \\
0 & \frac1{r^2} & 0 \\
0 & 0 & \frac{1}{r^2\sin^2\vartheta}
\end{pmatrix}.
\]
\item {\bf Volumenform:}
Die Volumenform ist
\begin{align*}
\nu
&=
dx^1\wedge dx^2 \wedge dx^3
\\
&=
\bigl(
r^2 \cos^2\vartheta \sin\vartheta (\sin^2\varphi+\cos^2\varphi)
+
r^2 \sin^3\vartheta (\cos^2\varphi +\sin^2\varphi)
\bigr)\,
dr\wedge d\vartheta \wedge d\varphi
\\
&=
r^2 \sin\vartheta
(
\cos^2\vartheta
+
\sin^2\vartheta
)
\,dr\wedge d\vartheta \wedge d\varphi
\\
&=
r^2 \sin\vartheta\, dr\wedge d\vartheta\wedge d\varphi.
\end{align*}
\item {\bf Skalarprodukt von 1-Formen:}
Für die Metrik der 1-Formen muss die inverse Matrix
des metrischen Tensors verwendet werden.
Es folgen die Skalarprodukte
\begin{align*}
\langle dr,dr\rangle &= 1
&
\langle d\vartheta,d\vartheta\rangle &= \frac{1}{r^2}
&
\langle d\varphi,d\varphi\rangle &= \frac{1}{r^2\sin^2\vartheta}
\end{align*}
von Basis-1-Formen.
Alle anderen Skalarprodukt von Basis-1-Formen verschwinden.

\item {\bf Skalarprodukt von $p$-Formen:}
Für 2-Formen muss die Gram-Determinante verwendet werden:
\begin{align*}
\langle dr\wedge d\vartheta , dr\wedge d\vartheta \rangle
&=
\left|\begin{matrix}
\langle dr, dr \rangle        & \langle dr,d\vartheta \rangle \\
\langle d\vartheta,dr \rangle & \langle d\vartheta, d\vartheta \rangle
\end{matrix}\right|
=
\langle dr, dr \rangle \langle d\vartheta, d\vartheta \rangle
=
\frac1{r^2}
\\
\langle dr\wedge d\varphi , dr\wedge d\varphi \rangle
&=
\left|\begin{matrix}
\langle dr, dr \rangle      & \langle dr,d\varphi \rangle \\
\langle d\varphi,dr \rangle & \langle d\varphi, d\varphi \rangle
\end{matrix}\right|
=
\langle dr, dr \rangle \langle d\varphi, d\varphi \rangle
=
\frac1{r^2\sin^2\vartheta}
\\
\langle d\vartheta\wedge d\varphi , d\vartheta\wedge d\varphi \rangle
&=
\left|\begin{matrix}
\langle d\vartheta, d\vartheta \rangle
	& \langle d\vartheta,d\varphi \rangle \\
\langle d\varphi,d\vartheta \rangle
	& \langle d\varphi, d\varphi \rangle
\end{matrix}\right|
=
\langle d\vartheta, d\vartheta \rangle \langle d\varphi, d\varphi \rangle
=
\frac1{r^4\sin^2\vartheta}
\end{align*}
Alle anderen Skalarprodukte von 2-Formen verschwinden.

Es gibt nur eine 3-Form, das Skalarprodukt ist wieder durch die 
Gram-Determinante
\begin{align*}
\langle
dr\wedge d\vartheta \wedge d\varphi,
dr\wedge d\vartheta \wedge d\varphi
\rangle
&=
\left|
\begin{matrix}
\langle dr, dr \rangle
&\langle dr, d\vartheta \rangle
&\langle dr, d\varphi \rangle
\\
\langle d\vartheta, dr \rangle
&\langle d\vartheta, d\vartheta \rangle
&\langle d\vartheta, d\varphi \rangle
\\
\langle d\varphi, dr \rangle
&\langle d\varphi, d\vartheta \rangle
&\langle d\varphi, d\varphi \rangle
\end{matrix}
\right|
\\
&=
\left|\begin{matrix}
1 & 0 & 0 \\
0 & \frac{1}{r^2} & 0 \\
0 & 0 & \frac{1}{r^2\sin^2\vartheta}
\end{matrix}\right|
=
\frac{1}{r^4\sin^2\vartheta}
\end{align*}
gegeben.
\item {\bf Hodge-Operator:}
Der Hodge-Operator angewendet auf die 1-Form $dv$ für
$v\in\{r,\vartheta,\varphi\}$  ergibt ein Vielfaches der Basis-2-Form,
die $dv$ nicht enthält.
Wir schreiben den Faktor $a_v$, also
z.~B.~$*dr = a_r\,d\vartheta\wedge d\varphi$.
Aus der Bedingung $\omega\wedge {\ast\omega}=\langle\omega,\omega\rangle\,\nu$
entstehen die Gleichungen
\begin{align*}
a_r\,
dr\wedge(d\vartheta\wedge d\varphi)
&=
\langle dr, dr\rangle
r^2\sin\vartheta\, dr\wedge d\vartheta\wedge d\varphi
&&\Rightarrow&
a_r &= r^2 \sin\vartheta
\\
a_\vartheta\,
d\vartheta \wedge(dr \wedge d\varphi)
&=
\langle d\vartheta,d\vartheta\rangle
r^2\sin\vartheta\,dr\wedge d\vartheta\wedge d\varphi
&&\Rightarrow&
a_\vartheta &= -\sin\vartheta
\\
a_\varphi\,
d\varphi \wedge(dr \wedge d\vartheta)
&=
\langle d\varphi,d\varphi\rangle
r^2\sin\vartheta\,dr\wedge d\vartheta\wedge d\varphi
&&\Rightarrow&
a_\varphi &= \frac{1}{\sin\vartheta}.
\end{align*}
Die Resultierenden Werte des Hodge-Operators auf 1-Formen sind in
Tabelle~\ref{buch:hodge:skalarprodukt:table:kugelhodge} zusammengestellt.

In der gleichen Art ist der Wert des Hodge-Operators auf einer
Basis-2-Form $\alpha$ ein Vielfaches von $dv$, wenn $v$ in $\alpha$
fehlt.
Wir schreiben den Faktor wieder als $b_v$ und stellen die Gleichungen
$\alpha\wedge {\ast\alpha} = \langle \alpha,\alpha\rangle\,\nu$ auf.
Es ergeben sich 
\begin{align*}
b_r(d\vartheta\wedge d\varphi)\wedge dr
&=
\langle d\vartheta\wedge d\varphi,d\vartheta\wedge d\varphi\rangle
r^2\sin\vartheta\, dr\wedge d\vartheta\wedge d\varphi
&&\Rightarrow&
b_r &= \frac{1}{r^2\sin \vartheta}
\\
b_\vartheta(dr\wedge d\varphi)\wedge d\vartheta
&=
\langle dr \wedge d\varphi,dr \wedge d\varphi\rangle
r^2\sin\vartheta\, dr\wedge d\vartheta\wedge d\varphi
&&\Rightarrow&
b_\vartheta &= -\frac{1}{\sin \vartheta}
\\
b_\varphi(dr\wedge d\vartheta)\wedge d\varphi
&=
\langle dr \wedge d\vartheta,dr \wedge d\vartheta\rangle
r^2\sin\vartheta\, dr\wedge d\vartheta\wedge d\varphi
&&\Rightarrow&
b_\varphi &= \sin \vartheta.
\end{align*}

Es bleibt noch den Hodge-Operator für die Basis-3-Form
$\omega=dr\wedge d\vartheta\wedge d\varphi$ zu bestimmen.
$\ast\omega$ ist eine Zahl $c$, für die
\begin{align*}
\omega\wedge c
&=
c
\,dr\wedge d\vartheta\wedge d\varphi
\\
&=
\langle
dr\wedge d\vartheta\wedge d\varphi,
dr\wedge d\vartheta\wedge d\varphi
\rangle
r^2\sin\vartheta
\,
dr\wedge d\vartheta\wedge d\varphi
\\
&=
\frac{1}{r^2\sin\vartheta}
dr\wedge d\vartheta\wedge d\varphi
\qquad\Rightarrow\qquad
c=\frac{1}{r^2\sin\vartheta}
\end{align*}
gelten muss.
Somit ist
\[
\ast(dr \wedge d\vartheta \wedge d\varphi)
=
\frac{1}{r^2\sin\vartheta}
\qquad\text{und}\qquad
{\ast}1
=
r^2\sin\vartheta\, dr \wedge d\vartheta \wedge d\varphi.
\]
Wie im Falle der Polarkoordinaten kann man nachprüfen, dass der
iterierte Hodge-operator $\ast{\ast}$ die identische Abbildung ist.
\end{enumerate}
%
% table-kugelhodge.tex
%
% (c) 2025 Prof Dr Andreas Müller
%
\begin{table}
\centering
\renewcommand\arraystretch{1.3}
\begin{tabular}{|>{$}c<{$}|>{$}c<{$}||>{$}c<{$}|>{$}c<{$}|}
\hline
\text{1-Form $\omega$} & \ast\omega & \text{2-Form $\alpha$}    & \ast\alpha\\
\hline
dr
& r^2\sin\vartheta\, d\vartheta\wedge d\varphi
& d\vartheta\wedge d\varphi
& \displaystyle \frac{1}{r^2\sin\vartheta}\,dr
\raisebox{9pt}{\mathstrut}
\\[7pt]
d\vartheta
& -\sin\vartheta\, dr\wedge d\varphi
& dr\wedge d\varphi
& \displaystyle -\frac{1}{\sin\vartheta}\,d\vartheta
\\[7pt]
d\varphi
& \displaystyle \frac{1}{\sin\vartheta}\, dr\wedge d\vartheta
& dr\wedge d\vartheta
& \sin\vartheta\, d\varphi
\\[7pt]
\hline
\end{tabular}
\caption{Tabelle der Wirkung des Hodge-Operators auf 1-Formen und 2-Formen
in Kugelkoordinaten.
\label{buch:hodge:skalarprodukt:table:kugelhodge}}
\end{table}


