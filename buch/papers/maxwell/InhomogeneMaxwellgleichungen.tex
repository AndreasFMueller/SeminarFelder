%
% InhomogeneMaxwellgleichungen.tex -- Beispiel-File für das Paper
%
% (c) 2020 Prof Dr Andreas Müller, Hochschule Rapperswil
%
% !TEX root = ../../buch.tex
% !TEX encoding = UTF-8
%
\section{Inhomogene Maxwellgleichungen
	\label{maxwell:section:InhomogeneMaxwellgleichungen}}
\kopfrechts{Inhomogene Maxwellgleichungen}
(Teil $dF = 0$ kommt dann vorher und auch die "Herleitung" davon)

Den Faraday-Tensor haben wir definiert als

\begin{equation}
	F = \begin{pmatrix}
		0 & -E_1 & -E_2 & -E_3 \\ E_1 & 0 & B_3 & -B_2 \\ E_2 & -B_3 & 0 & B_1 \\ E_3 & B_2 & -B_1 & 0 
	\end{pmatrix}.
%	\label{maxwell:section:teil1:metrik}
\end{equation}
Wir haben dabei festgestellt, dass die äussere Ableitung von $F$ verschwindet, da $F$ eine geschlossene 2-Form ist.
Um nun die inhomogenen Maxwellgleichungen bestimmen zu können müssen wir den Hodgeoperator auf $F$ anwenden.
Wir erinnern uns, dass $F$ ausgeschrieben mit Wedgeprodukten definiert ist als

\begin{align*}
	F = 
	&- E_1 \, dx^0 \wedge dx^1 - E_2 \, dx^0 \wedge dx^2 - E_3 \, dx^0 \wedge dx^3 \\
	&+ B_3 \, dx^1 \wedge dx^2 - B_2 \, dx^1 \wedge dx^3 + B_1 \, dx^2 \wedge dx^3.\\
\end{align*}

Der Hodgeoperator bildet eine $k$-Form auf eine $(n-k)$-Form ab und somit wird aus $\ast F$ wieder eine 2-Form.
Die Hodge-Duale können aus der Tabelle (WIE VERWEISEN??) entnommen werden.

\begin{align*}
	\ast F =
	& - E_{1} \, \ast(dx^0 \wedge dx^1) - E_{2} \, \ast(dx^0 \wedge dx^2) - E_{3} \, \ast(dx^0 \wedge dx^3) \\
	& + B_3 \, \ast(dx^1 \wedge dx^2) - B_2 \, \ast(dx^1 \wedge dx^3) + B_1 \, \ast(dx^2 \wedge dx^3)\\
	= 
	& \, E_{1} \, dx^2 \wedge dx^3 - E_{2} \, dx^1 \wedge dx^3 + E_{3} \, dx^1 \wedge dx^2 \\
	& + B_3 \, dx^0 \wedge dx^3 + B_2 \, dx^0 \wedge dx^2 + B_1 \, dx^0 \wedge dx^1.\\
\end{align*}
Die erhaltene 2-Form kann nun wieder als Matrix
\begin{equation}
	\ast F = \begin{pmatrix}
		0 & B_1 & B_2 & B_3 \\ -B_1 & 0 & E_3 & -E_2 \\ -B_2 & -E_3 & 0 & E_1 \\ -B_3 & E_2 & -E_1 & 0 
	\end{pmatrix}
	%	\label{maxwell:section:teil1:metrik}
\end{equation}
geschrieben werden.

Nun wenden wir die äussere Ableiung darauf an und erhalten
\begin{align*}
	d(\ast F) = \,
	& d (E_{1} \, dx^2 \wedge dx^3 - E_{2} \, dx^1 \wedge dx^3 + E_{3} \, dx^1 \wedge dx^2 + B_3 \, dx^0 \wedge dx^3 + B_2 \, dx^0 \wedge dx^2 + B_1 \, dx^0 \wedge dx^1)\\
	=
	& \frac{\partial E_1}{\partial x^0} dx^0 \wedge dx^2 \wedge dx^3 + \frac{\partial E_1}{\partial x^1} dx^1 \wedge dx^2 \wedge dx^3 -
	\frac{\partial E_2}{\partial x^0} dx^0 \wedge dx^1 \wedge dx^3 -
	\frac{\partial E_2}{\partial x^2} dx^2 \wedge dx^1 \wedge dx^3 +\\
	& \frac{\partial E_3}{\partial x^0} dx^0 \wedge dx^1 \wedge dx^2 +
	\frac{\partial E_3}{\partial x^3} dx^3 \wedge dx^1 \wedge dx^2 +
	\frac{\partial B_3}{\partial x^1} dx^1 \wedge dx^0 \wedge dx^3 +
	\frac{\partial B_3}{\partial x^2} dx^2 \wedge dx^0 \wedge dx^3 +\\
	& \frac{\partial B_2}{\partial x^1} dx^1 \wedge dx^0 \wedge dx^2 +
	\frac{\partial B_2}{\partial x^3} dx^3 \wedge dx^0 \wedge dx^2 +
	\frac{\partial B_1}{\partial x^2} dx^2 \wedge dx^0 \wedge dx^1 +
	\frac{\partial B_1}{\partial x^3} dx^3 \wedge dx^0 \wedge dx^1\\
	=
	&\left( \frac{\partial E_1}{\partial x^1} + \frac{\partial E_2}{\partial x^2} + \frac{\partial E_3}{\partial x^3} \right) dx^1 \wedge dx^2 \wedge dx^3 +
	\left(\frac{\partial E_1}{\partial x^0} + \frac{\partial B_2}{\partial x^3} - \frac{\partial B_3}{\partial x^2} \right) dx^0 \wedge dx^2 \wedge dx^3\\
	&\left( -\frac{\partial E_2}{\partial x^0} + \frac{\partial B_1}{\partial x^3} - \frac{\partial B_3}{\partial x^1} \right) dx^0 \wedge dx^1 \wedge dx^3 +
	\left( \frac{\partial E_3}{\partial x^0} + \frac{\partial B_1}{\partial x^2} - \frac{\partial B_2}{\partial x^1} \right) dx^0 \wedge dx^1 \wedge dx^2. 
\end{align*}
Wenden wir nun noch einmal den Hodge-Operator auf die erhaltene 3-Form an, erhalten wir
\begin{align*}
	\ast d \ast F = 
	&\left( \frac{\partial E_1}{\partial x^1} + \frac{\partial E_2}{\partial x^2} + \frac{\partial E_3}{\partial x^3} \right) (-dx^0) +
	\left(\frac{\partial E_1}{\partial x^0} + \frac{\partial B_2}{\partial x^3} - \frac{\partial B_3}{\partial x^2} \right) (-dx^1)\\
	&\left( -\frac{\partial E_2}{\partial x^0} + \frac{\partial B_1}{\partial x^3} - \frac{\partial B_3}{\partial x^1} \right) dx^2 +
	\left( \frac{\partial E_3}{\partial x^0} + \frac{\partial B_1}{\partial x^2} - \frac{\partial B_2}{\partial x^1} \right) (-dx^3).\\
	=
	&\left( -\frac{\partial E_1}{\partial x^1} -\frac{\partial E_2}{\partial x^2} - \frac{\partial E_3}{\partial x^3} \right) dx^0 +
	\left(-\frac{\partial E_1}{\partial x^0} - \frac{\partial B_2}{\partial x^3} + \frac{\partial B_3}{\partial x^2} \right) dx^1\\
	&\left( -\frac{\partial E_2}{\partial x^0} + \frac{\partial B_1}{\partial x^3} - \frac{\partial B_3}{\partial x^1} \right) dx^2 +
	\left( -\frac{\partial E_3}{\partial x^0} - \frac{\partial B_1}{\partial x^2} + \frac{\partial B_2}{\partial x^1} \right) dx^3.
\end{align*}

Weiter vereinfachen und nachvollziehen...

\subsection{Das ganze mit Si-Einheiten}
\begin{equation}
	F = \begin{pmatrix}
		0 & -E_1/c & -E_2/c & -E_3/c \\ E_1/c & 0 & B_3 & -B_2 \\ E_2/c & -B_3 & 0 & B_1 \\ E_3/c & B_2 & -B_1 & 0 
	\end{pmatrix}.
\end{equation}

\begin{align*}
	\ast F =
	& - \frac{E_{1}}{c} \, \ast(dx^0 \wedge dx^1) - \frac{E_{2}}{c} \, \ast(dx^0 \wedge dx^2) - \frac{E_{3}}{c} \, \ast(dx^0 \wedge dx^3) \\
	& + B_3 \, \ast(dx^1 \wedge dx^2) - B_2 \, \ast(dx^1 \wedge dx^3) + B_1 \, \ast(dx^2 \wedge dx^3)\\
	= 
	& \, \frac{E_{1}}{c} \, dx^2 \wedge dx^3 - \frac{E_{2}}{c} \, dx^1 \wedge dx^3 + \frac{E_{3}}{c} \, dx^1 \wedge dx^2 \\
	& + B_3 \, dx^0 \wedge dx^3 + B_2 \, dx^0 \wedge dx^2 + B_1 \, dx^0 \wedge dx^1.\\
\end{align*}
Die erhaltene 2-Form kann nun wieder als Matrix
\begin{equation}
	\ast F = \begin{pmatrix}
		0 & B_1 & B_2 & B_3 \\ -B_1 & 0 & E_3/c & -E_2/c \\ -B_2 & -E_3/c & 0 & E_1/c \\ -B_3 & E_2/c & -E_1/c & 0 
	\end{pmatrix}
	%	\label{maxwell:section:teil1:metrik}
\end{equation}
geschrieben werden.

\begin{align*}
	d(\ast F) = \,
	& d \left(\frac{E_{1}}{c} \, dx^2 \wedge dx^3 -\frac{E_{2}}{c} \, dx^1 \wedge dx^3 + \frac{E_{3}}{c} \, dx^1 \wedge dx^2 + B_3 \, dx^0 \wedge dx^3 + B_2 \, dx^0 \wedge dx^2 + B_1 \, dx^0 \wedge dx^1\right)\\
	=
	& \frac{1}{c}\frac{\partial E_1}{\partial x^0} dx^0 \wedge dx^2 \wedge dx^3 + \frac{1}{c}\frac{\partial E_1}{\partial x^1} dx^1 \wedge dx^2 \wedge dx^3 -
	\frac{1}{c}\frac{\partial E_2}{\partial x^0} dx^0 \wedge dx^1 \wedge dx^3 -
	\frac{1}{c}\frac{\partial E_2}{\partial x^2} dx^2 \wedge dx^1 \wedge dx^3 +\\
	& \frac{1}{c}\frac{\partial E_3}{\partial x^0} dx^0 \wedge dx^1 \wedge dx^2 +
	\frac{1}{c}\frac{\partial E_3}{\partial x^3} dx^3 \wedge dx^1 \wedge dx^2 +
	\frac{\partial B_3}{\partial x^1} dx^1 \wedge dx^0 \wedge dx^3 +
	\frac{\partial B_3}{\partial x^2} dx^2 \wedge dx^0 \wedge dx^3 +\\
	& \frac{\partial B_2}{\partial x^1} dx^1 \wedge dx^0 \wedge dx^2 +
	\frac{\partial B_2}{\partial x^3} dx^3 \wedge dx^0 \wedge dx^2 +
	\frac{\partial B_1}{\partial x^2} dx^2 \wedge dx^0 \wedge dx^1 +
	\frac{\partial B_1}{\partial x^3} dx^3 \wedge dx^0 \wedge dx^1\\
	=
	&\left( \frac{1}{c}\frac{\partial E_1}{\partial x^1} + \frac{1}{c}\frac{\partial E_2}{\partial x^2} + \frac{1}{c}\frac{\partial E_3}{\partial x^3} \right) dx^1 \wedge dx^2 \wedge dx^3 +
	\left(\frac{1}{c}\frac{\partial E_1}{\partial x^0} + \frac{\partial B_2}{\partial x^3} - \frac{\partial B_3}{\partial x^2} \right) dx^0 \wedge dx^2 \wedge dx^3\\
	&\left( -\frac{1}{c}\frac{\partial E_2}{\partial x^0} + \frac{\partial B_1}{\partial x^3} - \frac{\partial B_3}{\partial x^1} \right) dx^0 \wedge dx^1 \wedge dx^3 +
	\left( \frac{1}{c}\frac{\partial E_3}{\partial x^0} + \frac{\partial B_1}{\partial x^2} - \frac{\partial B_2}{\partial x^1} \right) dx^0 \wedge dx^1 \wedge dx^2. 
\end{align*}
Wenden wir nun noch einmal den Hodge-Operator auf die erhaltene 3-Form an, erhalten wir
\begin{align*}
	\ast d \ast F = 
	&\left( \frac{1}{c}\frac{\partial E_1}{\partial x^1} + \frac{1}{c}\frac{\partial E_2}{\partial x^2} + \frac{1}{c}\frac{\partial E_3}{\partial x^3} \right) (-dx^0) +
	\left(\frac{1}{c}\frac{\partial E_1}{\partial x^0} + \frac{\partial B_2}{\partial x^3} - \frac{\partial B_3}{\partial x^2} \right) (-dx^1)\\
	&\left( -\frac{1}{c}\frac{\partial E_2}{\partial x^0} + \frac{\partial B_1}{\partial x^3} - \frac{\partial B_3}{\partial x^1} \right) dx^2 +
	\left( \frac{1}{c}\frac{\partial E_3}{\partial x^0} + \frac{\partial B_1}{\partial x^2} - \frac{\partial B_2}{\partial x^1} \right) (-dx^3).\\
	=
	&\left( -\frac{1}{c}\frac{\partial E_1}{\partial x^1} -\frac{1}{c}\frac{\partial E_2}{\partial x^2} - \frac{1}{c}\frac{\partial E_3}{\partial x^3} \right) dx^0 +
	\left(-\frac{1}{c}\frac{\partial E_1}{\partial x^0} - \frac{\partial B_2}{\partial x^3} + \frac{\partial B_3}{\partial x^2} \right) dx^1\\
	&\left( -\frac{1}{c}\frac{\partial E_2}{\partial x^0} + \frac{\partial B_1}{\partial x^3} - \frac{\partial B_3}{\partial x^1} \right) dx^2 +
	\left( -\frac{1}{c}\frac{\partial E_3}{\partial x^0} - \frac{\partial B_1}{\partial x^2} + \frac{\partial B_2}{\partial x^1} \right) dx^3.
\end{align*}
Damit wir nun wieder auf etwas physikalisch Sinnvolles kommen, ersetzen wir alle $dx^0$ mit $cdt$, $dx^1$ mit $dx$, $dx^2$ mit $dy$ und $dx^3$ mit $dz$
\begin{align*}
	\ast d \ast F = 
	&\left( -\frac{1}{c}\frac{\partial E_1}{\partial x} -\frac{1}{c}\frac{\partial E_2}{\partial y} - \frac{1}{c}\frac{\partial E_3}{\partial z} \right) cdt +
	\left(-\frac{1}{c}\frac{\partial E_1}{\partial ct} - \frac{\partial B_2}{\partial z} + \frac{\partial B_3}{\partial y} \right) dx\\
	&\left( -\frac{1}{c}\frac{\partial E_2}{\partial ct} + \frac{\partial B_1}{\partial z} - \frac{\partial B_3}{\partial x} \right) dy +
	\left( -\frac{1}{c}\frac{\partial E_3}{\partial ct} - \frac{\partial B_1}{\partial y} + \frac{\partial B_2}{\partial x} \right) dz.
\end{align*}

\begin{align*}
	\ast d \ast F = 
	&\left( -\frac{\partial E_1}{\partial x} -\frac{\partial E_2}{\partial y} - \frac{\partial E_3}{\partial z} \right) dt +
	\left(-\frac{1}{c^2}\frac{\partial E_1}{\partial t} - \frac{\partial B_2}{\partial z} + \frac{\partial B_3}{\partial y} \right) dx\\
	&\left( -\frac{1}{c^2}\frac{\partial E_2}{\partial t} + \frac{\partial B_1}{\partial z} - \frac{\partial B_3}{\partial x} \right) dy +
	\left( -\frac{1}{c^2}\frac{\partial E_3}{\partial t} - \frac{\partial B_1}{\partial y} + \frac{\partial B_2}{\partial x} \right) dz.
\end{align*}
Damit wir nun den erhaltenen Ausdruck auch interpretieren können definieren wir die Viererstromdichte als
\begin{equation}
	J = -c\rho dx^0 + J_1 dx^1 + J_2 dx^2 +J_3 dx^3.
\end{equation}
Schreiben wir auch diese 1-Form in physikalisch sinnvollen Einheiten erhalten wir 
\begin{equation}
	J = -c^2\rho dt + J_1 dx + J_2 dy + J_3 dz.
\end{equation}
Multiplizieren wir zu $J$ noch ein $\mu_0$ erhalten wir den Ausdruck
\begin{equation}
	\mu_0 J = -\frac{\rho}{\varepsilon_0}dt + \mu_0 (J_1 dx +  J_2 dy +  J_3 dz),
\end{equation}
weil im Vakuum $c = \frac{1}{\sqrt{\varepsilon_0 \mu_0}}$ gilt. Setzen wir nun die beiden Teile der Gleichung zusammen ergibt sich
\begin{equation}
	\ast d \ast F = \mu_0 J
\end{equation}
Jetzt können wir die einzelnen Teile vergleichen und wissen bereits, dass es sich um vier unabhängige Gleichungen handeln muss, nämlich eine für jede Basis-1-Form.

Betrachten wir nur den Teil mit der Zeitkomponente $dt$, resultiert die Gleichung
\begin{equation}
	\frac{\partial E_1}{\partial x} +\frac{\partial E_2}{\partial y} + \frac{\partial E_3}{\partial z} = \frac{\rho}{\varepsilon_0},
\end{equation}
welche als das Gausssche Gesetz der Elektrostatik bekannt ist.
Weil es sich um Differentialformen handelt, wird hier nicht der Nabla-Operator verwendet für die Divergenz der elektrischen Feldstärke.

Vergleichen wir nun die anderen Komponenten $dx,dy,dz$ von beiden Seiten erhalten wir die drei Gleichungen
\begin{equation}
	-\frac{1}{c^2}\frac{\partial E_1}{\partial t} - \frac{\partial B_2}{\partial z} + \frac{\partial B_3}{\partial y} = \mu_0 J_1
\end{equation}
\begin{equation}
	-\frac{1}{c^2}\frac{\partial E_2}{\partial t} + \frac{\partial B_1}{\partial z} - \frac{\partial B_3}{\partial x} = \mu_0 J_2\\
\end{equation}
\begin{equation}
	 -\frac{1}{c^2}\frac{\partial E_3}{\partial t} - \frac{\partial B_1}{\partial y} + \frac{\partial B_2}{\partial x} = \mu_0 J_3.
\end{equation}
Schreiben wir erneut $c$ mit der Permittivität und Permeabilität des Vakuums erhalten wir die Gleichungen
\begin{equation}
	 - \frac{\partial B_2}{\partial z} + \frac{\partial B_3}{\partial y} = \mu_0 J_1 + \varepsilon_0\mu_0\frac{\partial E_1}{\partial t}
\end{equation}
\begin{equation}
 	\frac{\partial B_1}{\partial z} - \frac{\partial B_3}{\partial x} = \mu_0 J_2	+ \varepsilon_0\mu_0\frac{\partial E_2}{\partial t}
\end{equation}
\begin{equation}
	 - \frac{\partial B_1}{\partial y} + \frac{\partial B_2}{\partial x} = \mu_0 J_3 + \varepsilon_0\mu_0\frac{\partial E_3}{\partial t}.
\end{equation}