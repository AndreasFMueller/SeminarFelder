%
% teil2.tex -- Beispiel-File für teil2 
%
% (c) 2020 Prof Dr Andreas Müller, Hochschule Rapperswil
%
% !TEX root = ../../buch.tex
% !TEX encoding = UTF-8
%
\section{Einführung in die mathematische Formulierung der Elastizitätstheorie}
\subsection{Einleitung: Motivation und Ziel der Elastizitätstheorie}
Die Elastizitätstheorie beschreibt das Verhalten fester Körper unter mechanischer Belastung. 
Ziel ist es, aus bekannten Kräften oder Verschiebungen die Spannungs- und Dehnungsverteilung im Inneren eines Körpers zu berechnen. 
Diese Beschreibung basiert auf der Kontinuumsmechanik und bildet die Grundlage für viele Anwendungen im Bauingenieurwesen, Maschinenbau und der Materialwissenschaft.

\subsection{Spannungen und Gleichgewichtsbedingungen}
Die Spannungsverteilung beschreibt, welche inneren Kräfte im Material auftreten, um äussere Belastungen auszugleichen. 
Diese werden durch den Spannungstensor $\sigma_{ij}$ dargestellt. 
Um mechanisches Gleichgewicht sicherzustellen, muss die Impulsbilanz in jedem Punkt erfüllt sein:
\begin{equation}
	\frac{\partial \sigma_{ij}}{\partial x_j} + f_i = 
	0
\end{equation}
Dabei sind $f_i$ die Komponenten der Volumenkraftdichte.

\subsection{Kinematik: Deformation und Verschiebung}
Die Grundlage jeder elastischen Analyse bildet die Beschreibung der Geometrieänderung. 
Die Deformation wird über ein Verschiebungsfeld $u_i$ beschrieben, das angibt, wie sich Punkte im Inneren eines Körpers infolge einer Belastung verschieben. 
Daraus ergibt sich der Verzerrungstensor $\varepsilon_{ij}$, welcher lokale Längen- und Winkeländerungen erfasst.

Der Cauchy-Spannungstensor erlaubt es, zu jeder beliebigen Schnittfläche den zugehörigen Spannungsvektor $t_i$ zu bestimmen, der auf dieser Fläche wirkt.
Dieser Zusammenhang wird durch die cauchysche Formel beschrieben:
\begin{equation}
	t_i = 
	\sigma_{ij} \cdot n_j.
\end{equation}
Der Spannungstensor ist aufgrund des Momentengleichgewichts an einem infinitesimalen Volumenelement symmetrisch. 
Schubspannungen müssen sich dabei paarweise ausgleichen. 
Er beschreibt vollständig den inneren Spannungszustand eines elastischen Körpers.
Die Deformation des Körpers wird durch den Verzerrungstensor beschrieben, welcher die lokalen Längen- und Winkeländerungen in allen drei Raumrichtungen erfasst.
Zur Herleitung der linearen Verzerrungen im räumlichen Fall werden zunächst die Verschiebungsänderungen entlang der Koordinatenachsen untersucht. 
Aus diesen Änderungen wird beispielsweise in $x$-Richtung die relative Längenänderung berechnet, was direkt zur Definition der Dehnung $\varepsilon_{xx}$ führt. 
Dieses Vorgehen lässt sich analog auf die übrigen Normal- und Schubverzerrungen übertragen.
Die einzelnen Komponenten des Verzerrungstensors lauten \cite{elastomechanik:Technische_Mechanik_2:Elastostatik}:
\begin{align}
	\varepsilon_{xx} &=
	\frac{\partial u}{\partial x}
\end{align}
und
\begin{align}
	\varepsilon_{xy} &=
	\frac{1}{2} \left( \frac{\partial u}{\partial y} + \frac{\partial v}{\partial x} \right).
\end{align}

\subsection{Einstein’sche Summationskonvention}
In der Tensorrechnung wird häufig die sogenannte \textbf{Einstein’sche Summationskonvention} verwendet. Sie dient dazu, mathemische Ausdrücke kompakt und elegant zu schreiben.

\medskip
\textbf{Definition:} Tritt ein Index in einem Term zweimal auf, so ist über diesen Index zu summieren – ohne dass das Summenzeichen explizit geschrieben wird.
\begin{equation}
	a_i b_i = 
	\sum_{i=1}^3 a_i b_i
\end{equation}
Diese Schreibweise ist besonders hilfreich bei der Darstellung tensoranalytischer Gleichungen wie der Cauchy’schen Spannungsgleichung:
\begin{equation}
	t_i = 
	\sigma_{ij} n_j
\end{equation}
Hier wird implizit über den Index $j$ summiert, wobei $i$ frei bleibt. Das bedeutet:
\begin{equation}
	t_i = 
	\sum_{j=1}^{3} \sigma_{ij} n_j
\end{equation}


\textbf{Anmerkung zur Notation:} In der ursprünglichen Formulierung der Summationskonvention wird zwischen oberen und unteren Indizes unterschieden. Diese Unterscheidung ist insbesondere in gekrümmten Koordinatensystemen (z.\,B. Kugelkoordinaten, Relativitätstheorie) wesentlich, da dort kovariante und kontravariante Tensoren transformiert werden müssen.


\textbf{Vereinfachung im kartesischen Koordinatensystem:} In dieser Arbeit wird ausschliesslich im kartesischen Koordinatensystem gearbeitet. Dort fallen die Unterschiede zwischen oberen und unteren Indizes weg, da die Metrik konstant und orthonormal ist ($\delta_{ij}$). Daher schreiben wir konsequent alle Indizes tiefgestellt, ohne mathematische Korrektheit zu verlieren.

Diese Konvention erlaubt eine prägnante Darstellung komplexer Gleichungen und wird in der gesamten weiteren Herleitung verwendet.

\subsection{Konstitutives Gesetz: Materialverhalten}
Darüber hinaus gelten für ein ideal elastisches Material zusätzlich folgende mechanische Annahmen:
\begin{enumerate}
	\item Die Dehnung verschwindet vollständig bei Entlastung.
	\item Der Energieinhalt ist vollständig reversibel.
	\item Das Materialverhalten ist unabhängig von der Geschwindigkeit der Belastung.
	\item Es existiert ein eindeutiger Zusammenhang zwischen Spannung und Dehnung.
\end{enumerate}
Diese Annahmen ermöglichen eine vereinfachte mathematische Behandlung. 
Für komplexere Materialien (z. B. Faserverbundstoffe) sind jedoch anisotrope Modelle notwendig.

Die Beziehung zwischen Spannung und Dehnung wird durch ein Materialgesetz beschrieben. Für die klassische Elastizitätstheorie werden dabei folgende idealisierte Eigenschaften angenommen:
\begin{itemize}
	\item \textbf{Linearität:} Die Spannung ist proportional zur Dehnung (Hooke’sches Gesetz).
	\item \textbf{Reversibilität:} Nach Entlastung kehrt der Körper in seinen Ausgangszustand zurück.
	\item \textbf{Homogenität:} Die Materialeigenschaften sind an jedem Punkt des Körpers gleich.
	\item \textbf{Isotropie:} Die Materialeigenschaften sind in allen Raumrichtungen identisch.
\end{itemize}
Diese Annahmen ermöglichen eine vereinfachte mathematische Behandlung. Für komplexere Materialien (z. B. Faserverbundstoffe) sind jedoch anisotrope Modelle notwendig.

Das konstitutive Gesetz verknüpft die Verzerrungen mit den Spannungen. Für isotrope, linear elastische Materialien ergibt sich dieser Zusammenhang aus dem hookeschen Gesetz. Dieses wird meist in Form der Lamé-Konstanten $\lambda$ und $\mu$ oder alternativ über $E$ und $\nu$ beschrieben.

\subsection{Elastizitätsgesetz und Materialkonstanten}
Aus dem Hooke’schen Gesetz ist bekannt, dass im einachsigen, linearen Fall folgender Zusammenhang zwischen Spannung und Dehnung gilt:
\begin{equation}
	\sigma = 
	E \cdot \varepsilon.
\end{equation}
Im dreiachsigen Fall kann die lineare Beziehung zwischen den Spannungs- und Verzerrungskomponenten durch einen Elastizitätstensor 4. Stufe beschrieben werden:
\begin{equation}
	\sigma_{ij} = 
	E_{ijkl} \cdot \varepsilon_{kl}.
\end{equation}
Da wir den dreidimensionalen Fall unter der Annahme eines isotropen Materials betrachten, bei dem die mechanischen Eigenschaften in alle Richtungen gleich sind, reduziert sich die Form des Elastizitätstensors auf zwei unabhängige Konstanten: die sogenannten Lame’schen Konstanten $\lambda$ und $\mu$. 
Damit ergibt sich das Elastizitätsgesetz in kompakter Form:
\begin{equation}
	\sigma_{ij} = 
	\lambda \cdot \delta_{ij} \cdot \varepsilon_{kk} + 2\mu \cdot \varepsilon_{ij}.
\end{equation}
Die Lame’schen Konstanten lassen sich durch die technischen Materialkonstanten $E$ (Elastizitätsmodul), $\nu$ (Poissonzahl) und $G$ (Schubmodul) wie folgt ausdrücken:
\begin{equation}
	E = 
	\mu \cdot \frac{3\lambda + 2\mu}{\lambda + \mu}, \quad 
	\nu = 
	\frac{\lambda}{2(\lambda + \mu)}, \quad 
	G = 
	\mu.
\end{equation}
Durch Einsetzen in die Spannungs-Verzerrungs-Beziehung kann das Elastizitätsgesetz auch in Form der Deviatorzerlegung dargestellt werden. 
Dabei ergibt sich der Zusammenhang zwischen Spannungs- und Verzerrungsdeviator \cite{elastomechanik:Grundlagen_der_Elastizitaetstheorie}:
\begin{equation}
	s_{ij} = 
	\sigma_{ij} - \frac{1}{3} \sigma_{kk} \delta_{ij} = 
	2\mu e_{ij}.
\end{equation}
