%
% main.tex -- Paper zum Thema <geoalgebra>
%
% (c) 2020 Autor, OST Ostschweizer Fachhochschule
%
% !TEX root = ../../buch.tex
% !TEX encoding = UTF-8
%

\chapter{Geometrische Algebra\label{chapter:geoalgebra}}
\kopflinks{Geometrische Algebra}
\begin{refsection}
\chapterauthor{Damien Flury}
\index{Damien Flury}%
\index{Flury, Damien}%

{
\newcommand{\eone}{\boldsymbol{e}_1}
\newcommand{\etwo}{\boldsymbol{e}_2}
\newcommand{\ethree}{\boldsymbol{e}_3}

\renewcommand{\a}{\boldsymbol{a}}
\renewcommand{\b}{\boldsymbol{b}}
\renewcommand{\c}{\boldsymbol{c}}
\input{papers/geoalgebra/einleitung.tex}
\input{papers/geoalgebra/komplexe-zahlen.tex}
\input{papers/geoalgebra/wedge-produkt.tex}
\input{papers/geoalgebra/bivektoren.tex}
\input{papers/geoalgebra/trivektoren-n-vektoren.tex}
\input{papers/geoalgebra/geometrisches-produkt.tex}
\input{papers/geoalgebra/berechnungen-mit-der-ga.tex}
\input{papers/geoalgebra/spiegelungen.tex}
\input{papers/geoalgebra/rotationen.tex}
\input{papers/geoalgebra/rueckblick.tex}
}

\printbibliography[heading=subbibliography]
\end{refsection}
