%
% main.tex -- Paper zum Thema <rossby>
%
% (c) 2020 Autor, OST Ostschweizer Fachhochschule
%
% !TEX root = ../../buch.tex
% !TEX encoding = UTF-8
%
\chapter{Thema\label{chapter:rossby}}
\kopflinks{Thema}
\begin{refsection}
\chapterauthor{Michael Schmid}

\section{Einleitung}
Das Verständnis großskaliger atmosphärischer Muster ist ein zentrales Thema der Meteorologie. In dieser Arbeit wird das Konzept der Rossby-Wellen behandelt, wobei der Fokus auf den physikalischen Grundlagen liegt. Als Ingenieurstudent wird ein intuitiver Zugang gewählt, der auf komplizierte mathematische Formalismen weitgehend verzichtet.

\section{Grundlagen: Atmosphäre und Rotation}

\subsection{Die rotierende Erde}
Die Rotation der Erde beeinflusst Bewegungen in der Atmosphäre maßgeblich über die Corioliskraft, die bewegte Luftmassen ablenkt. In der \textit{f-Ebenen-Approximation} wird zunächst ein konstantes Coriolisparameter \( f \) angenommen.

\subsection{Vortizität und Zirkulation}
Die Vortizität beschreibt die lokale Drehung eines Luftpakets, während die Zirkulation die Rotation entlang eines geschlossenen Weges misst. Beide Konzepte sind wesentlich für das Verständnis der atmosphärischen Dynamik.

\section{Erhaltungsgesetze}

\subsection{Erhaltung der absoluten Vortizität}
Für Luftpakete, die sich langsam auf einer rotierenden Erde bewegen, bleibt die absolute Vortizität (Summe aus relativer und planetarer Vortizität) nahezu erhalten.

\subsection{Der Beta-Effekt}
Da das Coriolisparameter \( f \) mit der geographischen Breite variiert, erleben kleine Nord-Süd-Verschiebungen Änderungen in \( f \). Dies wird durch den \(\beta\)-Parameter beschrieben:
\[
\beta = \frac{df}{dy}.
\]

\section{Rossby-Wellen: Physikalische Idee}
Bei einer meridionalen Verschiebung eines Luftpakets verändert sich seine planetare Vortizität, was zu einer rücktreibenden Kraft führt. Dies resultiert in oszillatorischen Bewegungen, die sich als großskalige, westwärts wandernde Rossby-Wellen äußern. Längere Wellen bewegen sich dabei schneller als kürzere.

\section{Herleitung der Rossby-Wellen}

\subsection{Quasigeostrophisches Gleichgewicht}
In großskaligen Strömungen der Atmosphäre stehen Druckgradientenkraft und Corioliskraft im Gleichgewicht (geostrophisches Gleichgewicht). Kleine Störungen entwickeln sich dabei langsam.

\subsection{Vortizitätsgleichung}
Aus der Erhaltung der absoluten Vortizität und unter Annahme kleiner Störungen ergibt sich eine linearisierte Gleichung, die das Verhalten der Störung beschreibt.

\subsection{Dispersionsrelation}
Für wellenartige Lösungen der Form \( e^{i(kx + ly - \omega t)} \) ergibt sich die Dispersionsrelation der Rossby-Wellen:
\[
\omega = -\beta \frac{k}{k^2 + l^2}.
\]
Diese Relation zeigt, dass Rossby-Wellen westwärts wandern und ihre Ausbreitungsgeschwindigkeit von der Wellenlänge abhängt.

\section{Bedeutung in der Meteorologie}

\subsection{Wetterphänomene}
Rossby-Wellen prägen die Position und Struktur des Jetstreams und beeinflussen die Verlagerung von Hoch- und Tiefdruckgebieten.

\subsection{Blockierende Wetterlagen}
Unter bestimmten Bedingungen können Rossby-Wellen stationär werden und damit langanhaltende Wetterlagen wie Hitzewellen oder Dauerregen verursachen.

\section{Fazit}
Rossby-Wellen entstehen aus dem Zusammenspiel der Erddrehung und der Erhaltung der Vortizität. Sie sind entscheidend für das Verständnis der großräumigen atmosphärischen Dynamik und für die Wettervorhersage.


\printbibliography[heading=subbibliography]
\end{refsection}
