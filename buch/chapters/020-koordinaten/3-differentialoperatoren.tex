%
% 3-differentialoperatoren.tex -- Differentialoperatoren
%
% (c) 2024 Prof Dr Andreas Müller
%
\section{Differentialoperatoren
\label{buch:koordinaten:section:differentialoperatoren}}
\kopfrechts{Differentialoperatoren}
Der Tangentialvektor ist in
Definition~\ref{buch:koordinaten:tangentialvektoren:def:tangentialvektor}
sehr abstrakt definiert worden.
Tangentialvektoren sind das, was tangentialen Kurven gemeinsam ist,
also der Berührpunkt, die ``Richtung'' und ``Geschwindigkeit''.
Es ist aber nur indirekt mit Hilfe eines Koordinatensystems möglich, sich 
einen Tangentialvektor als einen ``Pfeil'' vorzustellen.
Richtung und Geschwindigkeit kann man aber auch dadurch detektieren, dass
man bestimmt, wie schnell sich eine Messgrösse, die von den Koordinaten
abhängit, ändert.
Die instantane Änderung einer Funktion ein einem Punkt ist so etwas
wie eine Ableitung.
In diesem Abschnitt soll daher gezeigt werden, dass man sich
Tagentialvektoren auch als Differentialoperatoren vorstellen
kann, die Funktionen ableiten können.

%
% Ableitung entlang einer Kurve
%
\subsection{Ableitung entlang einer Kurve}

%
% Tangentialvektoren als Differentialoperatoren
%
\subsection{Tangentialvektoren als Differentialoperatoren}

%
% Partielle Ableitungensoperatoren
%
\subsection{Partielle Ableitungsoperatoren}

