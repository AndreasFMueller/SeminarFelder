%
% teil1.tex -- Beispiel-File für das Paper
%
% (c) 2020 Prof Dr Andreas Müller, Hochschule Rapperswil
%
% !TEX root = ../../buch.tex
% !TEX encoding = UTF-8
%
\section{Geostrophischer Wind
\label{geostrophisch:section:geoWind}}
\kopfrechts{Problemstellung}


\subsection{Corioliskraft
\label{geostrophisch:subsection:coriolis}}
Die Corioliskraft, welche durch die Rotation der Erde entsteht, beeinflusst die Bewegung von Ludtmassen und anderen frei beweglichen Körpern auf der Erdoberfläche. Sie ist eine Scheinkraft, die nur in einem rotierendem Bezugssystem wie dem der Erde auftritt. Die Stärke hängt von der geografischen Breite ab und nimmt in Richtung der Pole zu.
Der Coriolisparameter $\boldsymbol{f}$ beschreibt diese Abhängigkeit und ist definiert als 
\begin{equation}
\boldsymbol{
f\, 
= 
2*\ohm*sin(\phi)
}
\label{geostrophisch:equation1}
\end{equation}
Wobei $\ohm$ die Winkelgeschwindigkeit der Erdrotation und $\phi$ die geografische Breite ist.
Die Corioliskraft wirkt immer senkrecht zur Bewegungsrichtung. Auf der Nordhalbkugel lenkt sie Bewegungen nach rechts ab, auf der Südhalbkugel nach links. Mathematisch lässt sie sich für eine Geschwindigkeit $\vec{v}_g $ mit Hilfe des Kreuzproduktes darstellen:
\begin{equation}
\boldsymbol{
\vec{F}_c 
= 
-f\, *\vec{k} \times \vec{v}_g 
}
\label{geostrophisch:equation2}
\end{equation}
Wobei $\vec{k}$ ein Einheitsvektor in Vertikalrichtung ist.
\begin{equation}
\vec{k} =
\left(
\begin{array}{c}
0 \\
0 \\
1
\end{array}
\right)
\label{geostrophisch:equation4}
\end{equation}
Das Kreuzprodukt von $\vec{k}$ mit  $\vec{v}_g $ sorgt dafür, dass die Corioliskraft immer genau senkrecht zur Bewegungsrichtung wirkt.
\subsection{Gradientenkraft
\label{geostrophisch:subsection:gradient}}
Die Gradientenkraft ist die treibende Kraft, die entsteht, wenn zwischen zwei Orten ein Druckunterschied besteht. Sie wirkt immer von Gebieten mit hohem Druck in Richtung niedrigen Drucks und ist die wichtigste antreibende Kraft für Luftbewegungen in der Atmosphäre. Je größer der Druckunterschied über eine bestimmte Entfernung, desto stärker ist die Gradientenkraft.
Mathematisch lässt sie sich wie folgt beschreiben:
\begin{equation}
\boldsymbol{
\vec{F}_p 
= 
-\frac{1}{\rho} \nabla p
}
\label{geostrophisch:equation3}
\end{equation}
wobei $\rho$ die Luftdichte und $\nabla p$ der Druckgradient ist. 
Der negative Vorfaktor zeigt, dass die Kraft immer in Richtung des abnehmenden Drucks wirkt.
Die Gradientenkraft setzt Luftmassen in Bewegung, da sie ein Ungleichgewicht im Druckfeld ausgleicht. Ohne weitere Kräfte wie die Corioliskraft würde die Luft somit direkt vom Hochdruckgebiet in das Tiefdruckgebiet strömen. Erst durch das Zusammenspiel mit der Corioliskraft stellt sich der geostrophische Wind ein, der nicht direkt ins Tief weht, sondern parallel zu den Isobaren verläuft.



