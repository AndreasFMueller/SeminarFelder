%
% teil4.tex -- Beispiel-File für Teil 4
%
% (c) 2020 Prof Dr Andreas Müller, Hochschule Rapperswil
%
% !TEX root = ../../buch.tex
% !TEX encoding = UTF-8
%
%\usepackage{siunitx}
%\usepackage{amsmath}
%\usepackage[ngerman]{babel}

%\begin{document}
	
	\section*{Einführung in das Praxisbeispiel}
	
	In diesem Kapitel wird ein praxisnahes Beispiel vorgestellt, das insbesondere im Bauingenieurwesen von Bedeutung ist. 
	Dazu betrachten wir ein einfaches statisches System, an dem grundlegende Konzepte der Elastizitätstheorie anschaulich demonstriert werden können.
	
	Im Mittelpunkt der Analyse stehen zwei unterschiedliche Materialien: \textbf{Holz} und \textbf{Stahl}. 
	Diese beiden Werkstoffe unterscheiden sich deutlich in ihren mechanischen Eigenschaften, insbesondere in Bezug auf ihre Steifigkeit, Festigkeit und ihr Verformungsverhalten unter Belastung.
	
	Ziel der Untersuchung ist es, die Unterschiede zwischen diesen Materialien herauszuarbeiten. 
	Wir zeigen, wie äussere Kräfte im Bauteil \textbf{Spannungen} erzeugen und wie sich daraus \textbf{Verformungen} ergeben.
	
	\subsection*{Berechnung des maximalen Biegemoments}
	
	Die Belastung von \SI{100}{\kilo\newton} erzeugt im Querschnitt des Trägers Schnittgrössen, insbesondere \textbf{Biegemomente} und \textbf{Querkraft}. Für dieses Beispiel sind vor allem die Biegemomente von Interesse, da sie massgeblich für die entstehenden Spannungen und Verformungen verantwortlich sind.
	
	Die Bestimmung des maximalen Moments erfolgt mithilfe einer einfachen Gleichgewichtsbedingung. 
	Bei einem symmetrisch belasteten Einfeldträger mit einer mittig aufgebrachten Einzellast kann das Moment wie folgt berechnet werden:
	
	\[
	M_\text{max} = \frac{F \cdot l}{4}
	\]
	
	Dabei ist:
	\begin{align}
		F = \SI{100}{\kilo\newton} && \text{(aufgebrachte Kraft)} \\
		l = \SI{5}{\meter} && \text{(Spannweite des Trägers)}
	\end{align}
	
	Einsetzen ergibt:
	\[
	M_\text{max} = \frac{100\,000 \cdot 5}{4} = \SI{125000}{\newton\meter} = \SI{125}{\kilo\newton\meter}
	\]
	
	Das maximale Biegemoment beträgt somit \(\boxed{\SI{125}{\kilo\newton\meter}}\).
	
	\section*{Berechnung der Biegespannung im Balkenquerschnitt}
	
	Die Kenntnis der Biegespannung ist im Bauingenieurwesen von zentraler Bedeutung, da sie Auskunft darüber gibt, wie stark ein Bauteil durch äussere Momente beansprucht wird. 
	Eine Überschreitung zulässiger Spannungen kann zur Schädigung oder gar zum Versagen des Bauteils führen. 
	Deshalb ist die Berechnung der maximalen Spannung ein wesentlicher Bestandteil der Tragwerksanalyse.
	
	\subsection*{Gegeben}
	
	Für das betrachtete Beispiel gelten die folgenden Angaben:
	
	\begin{itemize}
		\item Biegemoment: $M = \SI{125000}{\newton\meter} = \SI{125000000}{\newton\milli\meter}$
		\item Breite des Rechteckquerschnitts: $b = \SI{200}{\milli\meter}$
		\item Höhe des Rechteckquerschnitts: $h = \SI{300}{\milli\meter}$
	\end{itemize}
	
	\subsection*{1. Flächenträgheitsmoment $I$}
	
	Für einen rechteckigen Querschnitt berechnet sich das Flächenträgheitsmoment $I$ um die neutrale Achse mit:
	
	\[
	I = \frac{b \cdot h^3}{12}
	\]
	
	Einsetzen der Werte ergibt:
	
	\[
	I = \frac{200 \cdot 300^3}{12} = \frac{200 \cdot 27\,000\,000}{12} = \SI{450000000}{\milli\meter^4}
	\]
	
	\subsection*{2. Abstand $y$ zur äussersten Faser}
	
	Da die maximale Spannung am weitesten Punkt von der neutralen Faser auftritt, gilt:
	
	\[
	y = \frac{h}{2} = \frac{300}{2} = \SI{150}{\milli\meter}
	\]
	
	\subsection*{3. Biegespannung $\sigma$}
	
	Die Biegespannung wird mit der folgenden Formel berechnet:
	
	\[
	\sigma = \frac{M \cdot y}{I}
	\]
	
	Einsetzen der bekannten Grössen:
	
	\[
	\sigma = \frac{125\,000\,000 \cdot 150}{450\,000\,000} = \SI{41.67}{\newton\per\milli\meter\squared}
	\]
	
	\subsection*{Ergebnis}
	
	Die maximale Biegespannung im Balkenquerschnitt beträgt:
	
	\[
	\boxed{\sigma = \SI{41.67}{\mega\pascal}}
	\]
	
	
	\section*{Berechnung der Durchbiegung in der Balkenmitte}
	
	Die Durchbiegung eines Balkens unter Belastung ist ein entscheidendes Kriterium für die Gebrauchstauglichkeit eines Bauteils. 
	Sie beschreibt, wie stark sich der Balken unter der wirkenden Kraft verformt. 
	Eine zu grosse Durchbiegung kann die Funktion beeinträchtigen oder Schäden verursachen. 
	Daher ist die Berechnung der maximalen Durchbiegung von grosser Bedeutung.
	
	\subsection*{Gegebene Grössen}
	
	Für unser Einfeldträgersystem mit einer mittigen Einzellast gelten folgende Werte:
	
	\begin{itemize}
		\item Länge des Balkens: \( l = \SI{5}{\meter} = \SI{5000}{\milli\meter} \)
		\item Einzellast in der Mitte: \( F = \SI{100}{\kilo\newton} = \SI{100000}{\newton} \)
		\item Flächenträgheitsmoment des Querschnitts: \( I = \SI{450000000}{\milli\meter^4} \) (berechnet im vorherigen Kapitel)
		\item Elastizitätsmodul \( E \) (materialabhängig):
		\begin{itemize}
			\item Holz: \( E_{\text{Holz}} = \SI{12}{\giga\pascal} = \SI{12000}{\mega\pascal} \)
			\item Stahl: \( E_{\text{Stahl}} = \SI{210}{\giga\pascal} = \SI{210000}{\mega\pascal} \)
		\end{itemize}
	\end{itemize}
	
	\subsection*{Formel zur Berechnung der Durchbiegung}
	
	Für einen Einfeldträger mit einer mittigen Einzellast \( F \) lautet die Formel für die maximale Durchbiegung \( w_{\text{max}} \) in der Balkenmitte:
	
	\[
	w_{\text{max}} = \frac{F \cdot l^3}{48 \cdot E \cdot I}
	\]
	
	\begin{itemize}
		\item \( F \) ist die auf den Balken wirkende Kraft in Newton (N).
		\item \( l \) ist die Spannweite des Balkens in Millimetern (mm).
		\item \( E \) ist der Elastizitätsmodul des Materials in Megapascal (MPa).
		\item \( I \) ist das Flächenträgheitsmoment des Querschnitts in \(\mathrm{mm}^4\).
	\end{itemize}
	
	\subsection*{Schrittweise Berechnung}
	
	\paragraph{1. Berechnung der Durchbiegung für Holz:}
	
	\[
	w_{\text{max, Holz}} = \frac{F \cdot l^3}{48 \cdot E_{\text{Holz}} \cdot I}
	= \frac{100000 \cdot (5000)^3}{48 \cdot 12000 \cdot 450000000}
	\]
	
	Berechnen wir zunächst den Zähler:
	
	\[
	F \cdot l^3 = 100000 \times 125 \times 10^9 = 1.25 \times 10^{16}
	\]
	
	Dann den Nenner:
	
	\[
	48 \times 12000 \times 450000000 = 2.592 \times 10^{14}
	\]
	
	Setzen wir ein:
	
	\[
	w_{\text{max, Holz}} = \frac{1.25 \times 10^{16}}{2.592 \times 10^{14}} \approx 48.2\, \mathrm{mm}
	\]
	
	\paragraph{2. Berechnung der Durchbiegung für Stahl:}
	
	\[
	w_{\text{max, Stahl}} = \frac{F \cdot l^3}{48 \cdot E_{\text{Stahl}} \cdot I}
	= \frac{100000 \cdot (5000)^3}{48 \cdot 210000 \cdot 450000000}
	\]
	
	Zähler wie zuvor:
	
	\[
	1.25 \times 10^{16}
	\]
	
	Nenner:
	
	\[
	48 \times 210000 \times 450000000 = 4.536 \times 10^{15}
	\]
	
	Eingesetzt:
	
	\[
	w_{\text{max, Stahl}} = \frac{1.25 \times 10^{16}}{4.536 \times 10^{15}} \approx 2.76\, \mathrm{mm}
	\]
	
	\subsection*{Interpretation}
	
	Die Durchbiegung des Holzbalkens ist mit etwa \(\SI{48.2}{\milli\meter}\) deutlich grösser als die des Stahlbalkens mit \(\SI{2.76}{\milli\meter}\). 
	Dies liegt an dem wesentlich kleineren Elastizitätsmodul von Holz, das weniger steif ist als Stahl.
	
	Diese Grössenordnung zeigt, wie wichtig das Material für das Verformungsverhalten eines Bauteils ist. 
	Trotz gleicher Geometrie und Belastung führen unterschiedliche Werkstoffe zu sehr unterschiedlichen Durchbiegungen und somit zu unterschiedlicher Gebrauchstauglichkeit.
	
	\subsection*{Hinweis zur Materialanisotropie des Holzes}
	
	Holz ist ein anisotropes Material, das bedeutet, seine mechanischen Eigenschaften hängen von der Faserrichtung ab. 
	In unserer Berechnung wurde der Elastizitätsmodul \( E_{\text{Holz}} = \SI{12}{\giga\pascal} \) parallel zur Faser verwendet, da dies die Hauptrichtung der Belastung im Balken ist.
	
	Der Elastizitätsmodul quer zur Faser ist deutlich geringer, etwa nur \(10\%\) des Moduls parallel zur Faser, also ungefähr \( E_{\text{Holz, quer}} \approx \SI{1.2}{\giga\pascal} \).
	
	Würde man diesen kleineren Wert für \( E \) einsetzen, ergäbe sich eine deutlich grössere Durchbiegung:
	
	\[
	w_{\text{max, Holz quer}} \approx \frac{F \cdot l^3}{48 \cdot E_{\text{Holz, quer}} \cdot I} \approx 10 \times w_{\text{max, Holz parallel}}.
	\]
	
	Dies verdeutlicht die starke Abhängigkeit der Verformung vom Elastizitätsmodul und die Bedeutung der Faserrichtung bei der Bemessung von Holzbauteilen.
	
	\section*{Anwendung der Elastizitätstheorie im Oedometerversuch}
	
	Der Oedometerversuch dient zur Bestimmung der Verformungseigenschaften von Böden unter einaxialer Belastung. 
	Dabei wird eine zylindrische Bodenprobe in einer starren Form seitlich eingespannt, sodass keine laterale Dehnung stattfinden kann (\( \varepsilon_x = \varepsilon_y = 0 \)). 
	Auf die Probe wird schrittweise eine vertikale Spannung \( \sigma_z \) aufgebracht, während die vertikale Verformung \( \Delta h \) gemessen wird.
	
	\subsection*{Theorie}
	
	Unter der Annahme linear-elastischen Verhaltens lässt sich der Zusammenhang zwischen Spannung und Dehnung über den sogenannten Oedometer-Modul \( E_o \) ausdrücken:
	
	\[
	\varepsilon_z = \frac{\Delta h}{h_0} = \frac{\sigma_z}{E_o}
	\quad \Rightarrow \quad
	E_o = \frac{\sigma_z}{\varepsilon_z}
	\]
	
	Dabei ist:
	\begin{itemize}
		\item \( \sigma_z \) — vertikale Spannung [\si{\kilo\pascal}]
		\item \( \Delta h \) — gemessene Verkürzung der Probe [\si{\milli\meter}]
		\item \( h_0 \) — Ausgangshöhe der Probe [\si{\milli\meter}]
		\item \( \varepsilon_z \) — vertikale Dehnung [-]
		\item \( E_o \) — Oedometer-Modul [\si{\kilo\pascal}]
	\end{itemize}
	
	\subsection*{Beispielrechnung}
	
	Gegeben seien folgende Werte aus einem Oedometerversuch:
	
	\begin{itemize}
		\item Anfangshöhe der Probe: \( h_0 = \SI{20}{\milli\meter} \)
		\item Aufgebrachte Spannung: \( \sigma_z = \SI{100}{\kilo\pascal} \)
		\item Vertikale Verformung: \( \Delta h = \SI{0.5}{\milli\meter} \)
	\end{itemize}
	
	Berechnung der vertikalen Dehnung:
	
	\[
	\varepsilon_z = \frac{\Delta h}{h_0} = \frac{0.5}{20} = 0.025
	\]
	
	Berechnung des Oedometer-Moduls:
	
	\[
	E_o = \frac{\sigma_z}{\varepsilon_z} = \frac{100}{0.025} = \SI{4000}{\kilo\pascal}
	\]
	
	\subsection*{Interpretation}
	
	Der berechnete Oedometer-Modul beträgt:
	
	\[
	\boxed{E_o = \SI{4000}{\kilo\pascal}}
	\]
	
	Dieser Wert beschreibt die Steifigkeit des Bodens unter einaxialer Belastung. 
	In der Praxis liegt \( E_o \) für Kiese im Bereich von \( 10^4 \)–\( 10^5 \,\si{\kilo\pascal} \), für Sande bei \( 10^3 \)–\( 10^4 \,\si{\kilo\pascal} \), und für Tone oft deutlich darunter. 
	Das Ergebnis ist somit typisch für einen verdichteten sandigen Boden.
%\end{document}
