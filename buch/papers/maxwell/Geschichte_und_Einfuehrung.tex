%
% einleitung.tex -- Beispiel-File für die Einleitung
%
% (c) 2020 Prof Dr Andreas Müller, Hochschule Rapperswil
%
% !TEX root = ../../buch.tex
% !TEX encoding = UTF-8
\section{Geschichte der Elektrodynamik\label{maxwell:section:teil0}}
\kopfrechts{Geschichte der Elektrodynamik}

Die Geschichte der Elektrodynamik führt sehr weit zurück.
Das erste mal, das Elektrizität, wenn auch unbewusst, was es wirklich ist, verwendet wurde, war ca. 3100 Jahre v. Chr. im alten Ägypten.
Dort wurden die Elektroschocks des elektrischen Wels zur Behandlung von Gelenkschmerzen verwendet \cite{maxwell:History_of_bioelectricity}.

Ein wichtiger Meilenstein in der Entdeckung des Magnetismus war die Erfindung des Kompasses im Jahre 200 v. Chr.
Die Chinesen entdeckten, dass sich der natürlich vorkommende Magnetit auf wundersame Weise immer gleich ausrichtet, wenn er frei drehend aufgehängt wird.
Der Kompass wurde anfangs aber nur für das Wahrsagen verwendet, erst viel später, ca. 850 n. Chr. wurde er zur Navigation eingesetzt und fand seinen Weg im Jahre 1190 nach Europa \cite{maxwell:History_of_the_compass}.

Im Zeitraum von 1269 bis 1767 wurden die Phänomene der Elektrodynamik erstmals genauer beobachtet.
So beschrieb 1269 der Franzosen Petrus Peregrinus durch Experimente als erster die Polarität von Magneten.
Während des 17. Jahrhunderts wurde viel mit Reibungselektrizität experimentiert. Der Begriff der ``Elektrizität'' wurde erstmals verwendet und die Unterschiede magnetischer und elektrische Wirkung wurde erkannt.
Im Jahr 1767 erkannte Joseph Priestley, dass die elektrische Kraft quadratisch zum Abstand abnimmt, ein erster Hinweis auf das Coulomb-Gesetz \cite{maxwell:History_of_electromagnetic_theory}.

\subsection{erste wissenschaftliche Erkenntnisse}
Ab dem Jahr 1785 nimmt die Entwicklung der Elektrodynamik richtig Fahrt auf.
Charles Augustin de Coulomb entdeckte und bestätigte durch umfangreiche Experimente 1785, das die Kraft zwischen zwei Ladungen quadratisch mit dem Abstand ihres Mittelpunktes abnimmt \cite{maxwell:Charles-Augustin_de_Coulomb}:
\[
\vec{F}
=
\frac{1}{4 \pi \varepsilon}
\cdot
\frac{q_1 q_2}{r^2}
\cdot
\vec{r}_0.
\]
Ob er von der Entdeckung Priestleys wusste ist nicht bekannt.
%test
Als Hans Christian Ørsted an einem Tag im Jahre 1820 eine Vorlesung in Physik hielt, beobachtete er die Ablenkung einer Kompassnadel durch einen stromdurchflossenen Draht und endteckte somit zufällig die magnetische Wirkung des elektrischen Stromes.
André-Marie Ampère wurde auf Ørsteds Fund aufmerksam und verfolgte dieses Phenomen weiter.
Er erkannte eine sehr wichtige Eigenschaft: Die Magnetnadel des Kompasses richtet sich immer senkrecht zum stromdurchflossenen Leiter aus.
Er nahm an, dass die die Ursache, die die Magnetnadel zur Bewegung führt, ein vom Strom im Leiter selbst verursachtes Magnetfeld sei.
Dies konnte er durch viele Versuche ein wenig später sogar beweisen.
Ampère fand in diesen Versuchen heraus, dass sich zwei Leiter, in denen die elektrische Stromrichtung gleich ist, anziehen, und dass sie eine Abstossungskraft ausüben, wenn die Stromrichtung entgegengesetzt ist.
Er formulierte eine Gleichung, die besagt, dass elektrische Ströme magnetische Wirbelfelder hervorrufen, deren Stärke durch die Stromstärke gegeben ist.
Das Gesetz wird ampèrsches Gesetz oder auch Durchflutungssatz gennant und wird in Integralform
\[
\oint_{\d A}
\vec{H}
\cdot
d\vec{l}
=
\iint_{A}
\vec{J}
\cdot
d\vec{s}
=
I
\]
oder als differentielle Form
\[
\nabla
\times
\vec{H}
=
\vec{J}
\]
formuliert.
Dies sind sehr wichtige Gleichungen in der Elektrodynamik.
Jedoch sind sie so nicht ganz vollständig und werden später noch mal überarbeitet \cite{maxwell:Hans_Christian_Ørsted}\cite{maxwell:André-Marie_Ampère}.

\subsubsection{Herleitung der differentiellen Form der Ampère-Gleichnug}

Mit Hilfe des allgemeinen Satzes von Stokes(\ref{buch:green:green:satz:stokes})
\[
\int_{M} d\omega
=
\int_{\d M} \omega
\]
können wir die differentielle Form der Ampère-Gleichung herleiten.
Definieren wir eine 1-Form $\alpha$, die das $H$-Feld beschreibt:
\[
\alpha
=
H_x \, dx + H_y \, dy + H_z \, dz . 
\]
Berechnen wir nun als erstes die äussere Ableitung von $\alpha$
\begin{align*}
	d\alpha 
	&=
	\frac{\d H_x}{\d x} \, dx \wedge dx + \frac{\d H_x}{\d x} \, dy \wedge dx + \frac{\d H_x}{\d z} \, dz \wedge dx
	\\
	&+
	\frac{\d H_y}{\d x} \, dx \wedge dy + \frac{\d H_y}{\d y} \, dy \wedge dy + \frac{\d H_y}{\d z} \, dz \wedge dy
	\\
	&+
	\frac{\d H_z}{\d x} \, dx \wedge dz + \frac{\d H_z}{\d y} \, dy \wedge dz + \frac{\d H_z}{\d z} \, dz \wedge dz
	\\
	\\
	d\alpha
	=
	&\left(\frac{\d H_y}{\d x} - \frac{\d H_x}{\d y}\right) \, dx \wedge dy
	\\
	+
	&\left(\frac{\d H_z}{\d x} - \frac{\d H_x}{\d z}\right) \, dx \wedge dz
	\\
	+
	&\left(\frac{\d H_z}{\d y} - \frac{\d H_y}{\d z}\right) \, dy \wedge dz .
\end{align*}
Anschliessend können wir das Ergebnis in den allgemeinen Satz von Stokes einsetzen:
\[
\int_{\d M} H_x \, dx + H_y \, dy + H_z \, dz
=
\]
\[
\int_{M} \left(\frac{\d H_y}{\d x} - \frac{\d H_x}{\d y}\right) \, dx \wedge dy
+
\left(\frac{\d H_z}{\d x} - \frac{\d H_x}{\d z}\right) \, dx \wedge dz
+
\left(\frac{\d H_z}{\d y} - \frac{\d H_y}{\d z}\right) \, dy \wedge dz .
\]
Nun kann man die Gleichung in die Sprache der Vektoranalysis umformulieren, wobei die Klammerausdrücke rechts die Rotation und die Wedgeprodukte die orientierten Flächenstücke bilden: 
\begin{equation}
\label{maxwell:equation:ampere:integral}
\int_{\d A} \vec{H} \cdot d\vec{l}
=
\iint_{A} (\nabla \times \vec{H}) \cdot d\vec{s} .
\end{equation}
Damit ist gezeigt, dass die integrale Form der Gleichung
\begin{align}
	\int_{\d A}
	\vec{H} \cdot d\vec{l}
	&=
	\iint_{A}
	\vec{J} \cdot d\vec{s}
	\notag
	\\
	\intertext{in der Gleichung \refeq{maxwell:equation:ampere:integral} eingesetzt wird}
	\iint_{A}
	(
	\nabla \times \vec{H}
	)
	\cdot
	d\vec{s}
	&=
	\iint_{A}
	\vec{J} \cdot d\vec{s}
	\label{maxwell:equation:ampere:differential}
	\\
	\intertext{genau der differentiellen Form der Ampère-Gleichung}
	\nabla \times \vec{H}
	&=
	\vec{J} 
\end{align}
entspricht.

Dabei Stellt sich noch die Frage, warum die Integrale beim letzten Schritt weggelassen dürfen werden, welche wir im folgendem Lemma klären.
\begin{lemma}
	\label{maxwell:lemma:ampere}
	Sind $\vec{X}_1$ und $\vec{X}_2$ Vektorfelder, für die für jede Fläche $A$
	\[
	\int_A \vec{X}_1\cdot d\vec{s}
	=
	\int_A \vec{X}_2\cdot d\vec{s}
	\]
	gilt, dann ist $\vec{X}_1=\vec{X}_2$.
\end{lemma}

\begin{proof}
	\label{maxwell:proof:ampere}
	Mit einer Widerspruchsüberlegung lässt sich das lemma \ref{maxwell:lemma:ampere} beweisen.
	Der Einfachheit halber setzen wir $\vec{X} = \vec{X}_1-\vec{X}_2$.
	Nun wäre $\vec{X} \ne 0$ an einer bestimmten Stelle $P$, also $\vec{X}(P) \ne 0$.
	Wählen wir jetzt ein ganz kleines Ebenenstück um diesen Punkt $P$ mit Normale $\hat{n}$ parallel zu $\vec{X}$, können wir $\vec{X}$ als konstante betrachten. 
	Dann ist
	\begin{align*}
		\vec{X} \cdot d\vec{s} 
		&\approx
		\vec{X}(P) \cdot \hat{n} \cdot ds
		\\
		\int_{A} \vec{X} \cdot d\vec{s}
		&\approx
		\vec{X}(P) \cdot \hat{n} \cdot (\text{Flächeninhalt}) \overset{!}{\ne} 0.
	\end{align*}
	Dieser Widerspruch zeigt, dass $\vec{X} = 0$ sein muss.
\end{proof}
Das Lemma \ref{maxwell:lemma:ampere} kann nun auf Gleichung \refeq{maxwell:equation:ampere:differential} angewendent werden, und erhalten dadurch die Differentialform des ampèreschen Gesetzes ohne Integrale.


\subsection{Entdeckung des Induktionsgesetzes}

Im Jahre 1821 gelang Michael Faraday ein Experiment, bei dem sich ein stromdurchflossener Leiter unter Einfluss eines Dauermagneten um seine Achse drehte.
Er nannte dies die ``elektromagnetische Rotation'', was eine sehr wichtige Voraussetzung für die Entwicklung des Elektromotors war.
Die Idee, dass man Magnetismus in Elektrizität umwandeln könnte, hatte er bereits 1822.
Jedoch konnte er erst 1831, nach mehreren gescheiterten Versuchen und einer längeren Pause, in der er sich anderen Untersuchungen widmete, ein erfolgreiches Experiment durchführen.
Faraday entdeckte die elektromagnetische Induktion!

Fünf Jahre später konnte er erfolgreiche Versuche über den nach ihm benannten ``faradayschen Käfig'' erzielen.
Der faradayscher Käfig hat die Eigenschaft, das in seinem Inneren kein elektrisches Feld vorhanden ist.
Um das zu erreichen, muss der Körper aus elektrisch leitfähigem Material bestehen.

Faraday leistete in seinem Leben noch weitere Beiträge im Bereich der Elektrizität und im Magnetismus. 
Er war einer der Ersten, der nicht an ein Fernwirkungsgesetz glaubte.
Elektrische Kräfte oder auch die Gravitationskraft sollen durch Kraftlinien und Felder zu Stande kommen, was sich später auch bewahrheitete.
Aussergewöhnlich war ausserdem auch seine Arbeitsweise.
Seine veröffentlichten Errungenschaften beinhalteten wenig bis keine mathematischen Beweise, er konnte alles rein durch Experimente verifizieren \cite{maxwell:Michael_Faraday}.
\begin{quote}
	„Faraday sah im Geiste die den ganzen Raum durchdringenden Kraftlinien, wo die Mathematiker fernwirkende Kraftzentren sahen; Faraday sah ein Medium, wo sie nichts als Abstände sahen; Faraday suchte das Wesen der Vorgänge in den reellen Wirkungen, die sich in dem Medium abspielten, jene waren aber damit zufrieden, es in den fernwirkenden Kräften der elektrischen Fluida gefunden zu haben…“
	
	– James Clerk Maxwell: A Treatise on Electricity and Magnetism. Clarendon Press, 1873.
\end{quote}

\subsection{Entdeckung des gaussschen Gesetzes}

Carl Friedrich Gauss war ein sehr bedeutender Mathematiker und Theoretiker.
Bereits während seiner Lebenszeit wurde er als ``Fürst der Mathematiker'' bezeichnet.
Nach ihm wurden über 100 mathematische und wissenschaftliche Konzepte benannt, so auch in der Elektrodynamik.\cite{maxwell:Carl_Friedrich_Gauß}
1825 formulierte er das ``Gaussches Gesetz für elektrische Felder''.
Dieses Gesetz konnte er durch Coulombs Beobachtungen, dass das elektrische Feld einer Punktladung radial quadratisch abnimmt, folgern.
Durch diese radiale Abnahme musste auch der elektrisch Fluss durch eine Kugel, in der die Ladung eingeschlossen ist, konstant sein.
Somit lässt sich daraus schliessen, dass das Integral des elektrischen Flusses über diese Kugeloberfläche genau der darin enthaltenen Ladung entspricht\cite{maxwell:Gaußscher_Integralsatz}
\[
\oint_{A=\d V} \vec{D} \cdot d\vec{s}
=
\int_{V} \rho_{\text{frei}}\, dv
=
Q_{\text{frei}} ,
\]
wobei $\rho_{\text{frei}}$ die freie Ladungsdichte und $Q_{\text{frei}}$ die gesamten freien Ladungen in der Kugel sind. $\varepsilon_0$ enstpricht der Permitivität im Vakuum. Dies ist auch gerade eine Anwendung vom Gausschen Integralsatz aus der Mathematik.
Die Gleichung lässt sich natürlich auch mit der elektrischen Feldstärke ausdrücken. Da $\vec{D} = \varepsilon_0 \vec{E}$ gilt, folgt:
\[
\oint_{A=\d V} \vec{E} \cdot d\vec{s}
=
\frac{1}{\varepsilon_0}\int_{V} \rho\, dv
=
\frac{Q}{\varepsilon_0} .
\]
Natürlich lassen sich die Gleichungen auch in differentiellen Form darstellen als
\begin{align*}
	\nabla \cdot \vec{D}
	&=
	\rho
	\\
	\nabla \cdot \vec{E}
	&=
	\frac{\rho}{\varepsilon_0} .
\end{align*}

\subsubsection{Herleitung der differentiellen Form des gaussschen Gesetzes}

Der Integralsatz von Gauss ist ein Spezialfall des Satzes von Stokes.
Somit lässt sich das Gausssche Gesetz für elektrische Felder mittels des Satzes von Stokes herleiten.
Dafür integrieren wir bei
\[
\int_{M} d\omega
=
\int_{\d M} \omega
\]
über das Volumen, bzw. über den Rand des Volumens
\[
\int_{V} d\omega
=
\int_{\d V} \omega .
\]
Wie bereits bei der Umformulierung des Ampère-Gesetz, definieren wir wieder eine 1-Form
\[
\alpha
=
E_x \, dx + E_y \, dy + E_z \, dz ,
\]
welche dieses Mal dem elektrischen Feld entspricht.
Da wir einmal über den Rand eines Volumens und einmal über das Volumen selbst integrieren müssen, benötigen wir aber eine 2-Form und eine 3-Form, welche wir bekanntlich mit dem Hodgeoperator
\begin{align}
	\ast\alpha
	&=
	E_x \, dy \wedge dz + E_y \, dz \wedge dx + E_z \, dx \wedge dy
	\label{maxwell:equasion:gauss:zweiform}
	\\
	\intertext{und mit der äusseren Ableitung}
	d{\ast}\alpha
	&=
	\frac{\d E_x}{\d x} \, dx \wedge dy \wedge dz +
	\frac{\d E_y}{\d y} \, dy \wedge dz \wedge dx +
	\frac{\d E_z}{\d z} \, dz \wedge dx \wedge dy
	\notag
	\\ 
	&=
	\frac{\d E_x}{\d x} \, dx \wedge dy \wedge dz +
	\frac{\d E_y}{\d y} \, dx \wedge dy \wedge dz +
	\frac{\d E_z}{\d z} \, dx \wedge dy \wedge dz
	\notag
	\\
	&=
	\left(
	\frac{\d E_x}{\d x} + \frac{\d E_y}{\d y} + \frac{\d E_z}{\d z}
	\right)
	dx \wedge dy \wedge dz 
	\label{maxwell:equasion:gauss:dreiform}
\end{align}
bekommen.

Der allgemeine Satz von Stokes für $\ast \alpha$ ist
\[
\int_{V} d \ast \alpha 
=
\int_{\d V} \ast \alpha .
\]
Setzt man Gleichung \refeq{maxwell:equasion:gauss:zweiform} und \refeq{maxwell:equasion:gauss:dreiform} ein, erhalten wir
\[
\int_{\d V}
E_x \, dy \wedge dz + E_y \, dz \wedge dx + E_z \, dx \wedge dy
=
\int_{V}
\left(
\frac{\d E_x}{\d x} + \frac{\d E_y}{\d y} + \frac{\d E_z}{\d z}
\right)
dx \wedge dy \wedge dz .
\]
In Vektorform umgeformt, folgt nun der Integralsatz von Gauss:
\begin{equation}
	\label{maxwell:equasion:gauss:gauss}
	\int_{\d V}
	\vec{E} \cdot d\vec{s}
	=
	\int_{V}
	(\nabla \cdot \vec{E}) \, dV .
\end{equation}
Nun können wir in bei der integralen Form des Satzes von Gauss
\begin{align*}
	\oint_{A=\d V} \vec{E} \cdot d\vec{s}
	&=
	\frac{1}{\varepsilon_0}\int_{V} \rho \, dv
	\\
	\intertext{den Integralsatz \refeq{maxwell:equasion:gauss:gauss} einsetzen:}
	\int_{V}
	(\nabla \cdot \vec{E}) \, dv
	&=
	\frac{1}{\varepsilon_0}\int_{V} \rho \, dv .
	\\
	\intertext{Dadurch folgt mit Hilfe des Lemma \ref{maxwell:lemma:gauss} die differentielle Form}
	\nabla \cdot \vec{E}
	&=
	\frac{\rho}{\varepsilon_0}.
\end{align*} 

\begin{lemma}
	\label{maxwell:lemma:gauss}
	Für den allgemeinen Satz von Stokes gilt für jedes noch so kleine Volumen
	\[
	\lim_{\left| V \right| \to 0} \frac{1}{\left| V \right|} \int_{V} \rho \, dV = \rho(x),
	\]
	falls $x\in V$ während des Grenzübergangs.
\end{lemma}
\begin{proof}
	Die Funktion $\rho$ ist stetig in $x$. Das bedeutet, für jedes noch so kleine $\varepsilon > 0$ existiert ein $\delta > 0$.
	Dabei muss für alle Punkte $y$ mit
	\[
	\left|y-x\right| < \delta	
	\]
	gelten, dass 
	\[
	\left|\rho(y) - \rho(x)\right| < \varepsilon
	\] 
	gilt.
	Wir wählen ein Volumen $V$ mit Eigenschaften $x \in V$ und der Durchmesser von $V$ ist kleiner als $\delta$, was bedeutet, dass jeder Punkt $y \in V$ in einer Kugel um $x$ mit Radius $\delta$ liegt.
	Somit gilt auch für alle $y \in V$
	\[
	\left|\rho(y) - \rho(x)\right| < \varepsilon .
	\]
	Betrachten wir nun die Differenz zwischen dem Mittelwert über das volume $V$ und dem Wert $x$
	\[
	\left|
	\frac{1}{\left|V\right|}
	\int_{V}
	\rho(y)
	\,
	dV
	-
	\rho(x)
	\right|
	\]
	und nehmen das $\rho(x)$ ins Integral
	\[
	\left|
	\frac{1}{\left|V\right|}
	\int_{V}
	(\rho(y) - \rho(x)) \, dV
	\right|
	\leq
	\frac{1}{\left|V\right|}
	\int_{V}
	\left|
	\rho(y) - \rho(x)
	\right|
	\,
	dV .
	\]
	Da wir wissen, dass $\left|\rho(y) - \rho(x)\right| < \varepsilon \quad \forall \, y \in V$, folgt
	\[
	\frac{1}{\left|V\right|}
	\int_{V}
	\left|
	\rho(y) - \rho(x)
	\right|
	\,
	dV
	<
	\frac{1}{\left|V\right|}
	\int_{V}
	\varepsilon
	\,
	dV
	=
	\varepsilon .
	\]
	Somit ist 
	\[
	\left|
	\frac{1}{\left|V\right|}
	\int_{V}
	(\rho(y) - \rho(x)) \, dV
	\right|
	<
	\varepsilon
	\]
	für jedes $\varepsilon > 0$, wenn $V$ sehr klein ist, mit Durchmesser $< \delta$ und Lemma \ref{maxwell:lemma:gauss} ist korrekt.
\end{proof}


\subsection{Maxwells Beiträge zur Elektrodynamik}
Der wohl prägendste Namen in der Geschichte der Elektrodynamik ist wohl der von James Clerk Maxwell.
Mit seiner Publikation ``A dynamical Theory of the Electromagnetic Field'' im Jahre 1865 beschrieb Maxwell, dass sich das magnetische und elektrische Feld in Form von Wellen mit Lichtgeschwindigkeit im Vakuum ausbreiten.
Im selben Paper veröffentlichte er auch seine Finale Version seiner berühmten Gleichungen, doch wie kam es dazu?

Schon im jungen Alter war Maxwell sehr fasziniert an der Geometrie. Bereits mit 13 Jahren, im Jahre 1844, gewann er den Mathematikwettbewerb seiner Schule, der angesehenen Edinburgh Academy.
Sein Interesse ging weit über den gewöhnlichen Schulstoff hinweg.
Er beschäftigte sich hauptsächlich mit Verfahren über das Zeichnen mathematischer Kurven und Eigenschaften von Ellipsen, worüber er mit 14 Jahren seine erste wissenschaftliche Arbeit schrieb.

Er führte sein Studium in der University of Edinburgh fort und wechselte später zur University of Cambridge.
1865 wurde er Professor für Mathematik am Marischal College in Aberdeen.
Mit seinen 25 war er mit Abstand der jüngste Professor am College.%https://en.wikipedia.org/wiki/James_Clerk_Maxwell

Sein Studium über den Elektromagnetismus begann er aber bereits 1855 und dauerte bis 1870. In dieser Zeit veröffentlichte er mehrere Papers.

 

%https://en.wikipedia.org/wiki/History_of_Maxwell%27s_equations

%axwell also introduced the concept of the electromagnetic field in comparison to force lines that Faraday described.[1

%His discoveries helped usher in the era of modern physics, laying the foundations for such fields as relativity, also being the one to introduce the term into physics,[10] and quantum mechanics.[18][19] Many physicists regard Maxwell as the 19th-century scientist having the greatest influence on 20th-century physics. His contributions to the science are considered by many to be of the same magnitude as those of Isaac Newton and Albert Einstein.[20] On the centenary of Maxwell's birthday, his work was described by Einstein as the "most profound and the most fruitful that physics has experienced since the time of Newton".[21] When Einstein visited the University of Cambridge in 1922, he was told by his host that he had done great things because he stood on Newton's shoulders; Einstein replied: "No I don't. I stand on the shoulders of Maxwell."[22] Tom Siegfried described Maxwell as "one of those once-in-a-century geniuses who perceived the physical world with sharper senses than those around him".[23]
folgt noch: Geschichte Maxwell, war der erste der die relativität in betracht zog!! wichtig
Heaviside,
relativität mit einstein?
differentailformen eher nicht
bei maxwell 20 gleichhungne zerst, evtl sagen was es alles war, war nähmlich glaubs mehr drin als jetzt in den vier heuteigen




