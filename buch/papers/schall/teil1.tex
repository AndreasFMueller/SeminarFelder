%
% teil1.tex -- Beispiel-File für das Paper
%
% (c) 2020 Prof Dr Andreas Müller, Hochschule Rapperswil
%
% !TEX root = ../../buch.tex
% !TEX encoding = UTF-8
%
\section{Herleitung der Mediumsabhängigkeit
\label{schall:section:teil1}}
\kopfrechts{Herleitung der Mediumsabhängigkeit}

In diesem Abschnitt leiten wir die eindimensionale Wellengleichung her
und zeigen, wie daraus die bekannten Ausdrücke für die Schallgeschwindigkeit
in Gasen, Flüssigkeiten und Festkörpern folgen
\cite{schall:kinsler,schall:landaulifschitz}.

\subsection{Grundannahmen}
Wir betrachten kleine Störungen um einen ruhenden Grundzustand $(p_0,\rho_0)$.
Es seien
\[
    p(x,t)=p_0+p'(x,t),\qquad \rho(x,t)=\rho_0+\rho'(x,t),\qquad v(x,t)=v'(x,t),
\]
wobei $|p'|\ll p_0$, $|\rho'|\ll \rho_0$ und $|v'|\ll 1$ (lineare Akustik),
und die Terme von erster Ordnung sind mit
$\rho',p',v' = \mathcal{O}(\varepsilon) \text{ und } 0<\varepsilon\ll 1$.
Wir nehmen ferner reibungsfreie, quellenfreie Strömung an und
vernachlässigen äussere Kräfte.


\subsection{Kontinuität und Impulsbilanz}
In einer Dimension lauten die Erhaltung der Masse und der Impulsbilanz
(ohne Körperkräfte)
\begin{equation}
\frac{\partial \rho}{\partial t}+\frac{\partial(\rho v)}{\partial x}=0,
\qquad\text{bzw.}\qquad
\rho\left(\frac{\partial v}{\partial t}+v\frac{\partial v}{\partial x}\right)=-\frac{\partial p}{\partial x}.
\label{eq:exact-cont-mom}
\end{equation}

\subsubsection*{Kontinuität}
Setzen wir die Grundannahmen in die Kontinuitätsgleichung ein und
wende die Produktregel an:
\[
    \frac{\partial(\rho_0+\rho')}{\partial t}
    +\frac{\partial\big[(\rho_0+\rho')\,v'\big]}{\partial x}=0.
\]
Da \(\rho_0\) konstant ist, \(\partial_t\rho_0=0\) und \(\partial_x\rho_0=0\).
Es folgt
\[
    \underbrace{\frac{\partial \rho'}{\partial t}}_{\mathcal{O}(\varepsilon)}
    +\underbrace{\rho_0\,\frac{\partial v'}{\partial x}}_{\mathcal{O}(\varepsilon)}
    +\underbrace{v'\,\frac{\partial \rho'}{\partial x}}_{\mathcal{O}(\varepsilon^2)}
    +\underbrace{\rho'\,\frac{\partial v'}{\partial x}}_{\mathcal{O}(\varepsilon^2)}=0.
\]
Weglassen der Fluktuationen durch weglassen der \(\mathcal{O}(\varepsilon^2)\)-Terme
liefert die \emph{linearisierten} Kontinuitätsgleichung
\begin{equation}
    \frac{\partial \rho'}{\partial t}+\rho_0\,\frac{\partial v'}{\partial x}=0.
    \label{eq:lin-cont}
\end{equation}
Dies ist die Taylor-Entwicklung erster Ordnung um den Grundzustand.

\subsubsection*{Impulsbilanz}
Analog für die Impulsbilanz folgt bei Einsatz der Grundannahmen:
\[
    (\rho_0+\rho')\!\left(\frac{\partial v'}{\partial t}+v'\frac{\partial v'}{\partial x}\right)
    =-\frac{\partial (p_0+p')}{\partial x}.
\]
Mit \(\partial_x p_0=0\) und Ordnungszählung erhält man
\[
    \underbrace{\rho_0\,\frac{\partial v'}{\partial t}}_{\mathcal{O}(\varepsilon)}
    +\underbrace{\rho'\,\frac{\partial v'}{\partial t}}_{\mathcal{O}(\varepsilon^2)}
    +\underbrace{(\rho_0+\rho')\,v'\frac{\partial v'}{\partial x}}_{\mathcal{O}(\varepsilon^2)}
    =-\,\underbrace{\frac{\partial p'}{\partial x}}_{\mathcal{O}(\varepsilon)}.
\]
Nach Verwerfen der \(\mathcal{O}(\varepsilon^2)\)-Terme bleibt
\begin{equation}
    \rho_0\,\frac{\partial v'}{\partial t}=-\frac{\partial p'}{\partial x}.
    \label{eq:lin-mom}
\end{equation}

\subsection{Thermodynamische Schliessung}

Ziel ist nun die Kombination der linearisierten Kontinuität \eqref{eq:lin-cont}
mit der Impulsbilanz \eqref{eq:lin-mom} im homogenen Ruhezustand.
Dazu lösen wir zunächst \eqref{eq:lin-mom} nach $\partial_t v'$ auf:
\begin{equation}
    \partial_t v' \;=\; -\,\frac{1}{\rho_0}\,\partial_x p'.
    \label{eq:dtv}
\end{equation}
Wenden wir $\partial_t$ auf die Kontinuitätsgleichung \eqref{eq:lin-cont}
an und setzen \eqref{eq:dtv} ein:
\begin{align*}
    \partial_{tt}\rho' + \rho_0\,\partial_x(\partial_t v') &= 0,\\
    \partial_{tt}\rho' + \rho_0\,\partial_x\!\left(-\frac{1}{\rho_0}\partial_x p'\right) &= 0,
\end{align*}
woraus die gekoppelte partielle Differentialgleichung resultiert:
\begin{equation}
    \frac{\partial^2 \rho'}{\partial t^2} - \frac{\partial^2 p'}{\partial x^2} = 0.
    \label{eq:lin-cont-mom}
\end{equation}
Dieses System ist jedoch unterbestimmt, da die Kontinuität reine Kinematik,
und die Impulsbilanz die Trägheit beschreibt.
Beide enthalten keine Materialeigenschaften darüber, wie Druck auf Dichteänderung reagiert.
Somit kann \eqref{eq:lin-cont-mom} keine punktweise lokale Zustandsgleichung beschreiben.
Um das System zu schliessen, benötigen wir einen weiteren Zusammenhang,
welcher aus der Thermodynamik folgt:

\begin{enumerate}
\item Zustandsgleichung in Ortsnähe: $p=p(\rho,s)$.
      Für kleine Störungen um $(\rho_0,s_0)$ gilt die lineare Totaldifferentiale
      \[
        p' \;=\; \left(\frac{\partial p}{\partial \rho}\right)_{s}\,\rho'
             \;+\; \left(\frac{\partial p}{\partial s}\right)_{\rho}\,s'.
      \]
\item Akustische Störungen sind schnell und nahezu ohne Wärmeaustausch
      und Reibung: Der Prozess ist \emph{isentrop}.
      Die Entropieänderung entlang einer Teilchenbahn verschwindet:
      \[
        \frac{Ds}{Dt}=0
        \Rightarrow s'=0.
      \]
      Damit fällt der zweite Term weg:
      \begin{equation}
        p' \;=\; \left(\frac{\partial p}{\partial \rho}\right)_{s}\,\rho' \;=:\; \beta\,\rho'.
        \label{eq:p-rho-relation}
      \end{equation}
\end{enumerate}
Vergleichen wir die allgemeine Wellengleichung \eqref{buch:feldgleichungen:wellengleichung:eqn:wellengleichung},
welche im Kapitel \ref{chapter:feldgleichungen} eingeführt wurde:
\[
    \frac{\partial^2 u}{\partial t^2} \;=\;a^2 \frac{\partial^2 u}{\partial x^2}
\]
mit \eqref{eq:p-rho-relation} und \eqref{eq:lin-cont-mom}, so erkennen wir, dass
\[    a^2 \;=\; \beta \;=\; \left(\frac{\partial p}{\partial \rho}\right)_{s}.\]
Damit folgt die Wellengleichung für Druck- und Dichteschwankungen
\begin{equation}
    \frac{\partial^2 p'}{\partial t^2} - \beta\,\frac{\partial^2 p'}{\partial x^2} = 0,
    \qquad
    \frac{\partial^2 \rho'}{\partial t^2} - \beta\,\frac{\partial^2 \rho'}{\partial x^2} = 0.
    \label{eq:wave-equation-p-rho}
\end{equation}
Somit resultiert die Schallgeschwindigkeit des Mediums:
\begin{equation}
    c \;=\; \sqrt{\beta} \;=\; \sqrt{\left(\frac{\partial p}{\partial \rho}\right)_{s}},
    \label{eq:speed-of-sound}
\end{equation}
sowie die klassische eindimensionale Wellengleichung, die sowohl
Druck- als auch Dichteschwankungen abdeckt:
\begin{equation}
    \;\;\frac{\partial^2 \phi}{\partial t^2} = c^2\,
    \frac{\partial^2 \phi}{\partial x^2}, \qquad \phi\in\{p',\rho'\}\; .
    \label{eq:1d-wave-equation}
\end{equation}
In den folgenden Abschnitten werden wir die in \eqref{eq:speed-of-sound}
angegebene Schallgeschwindigkeit für Gase, Flüssigkeiten und Festkörper
herleiten.

\subsection{Schallgeschwindigkeit in verschiedenen Medien}

\subsubsection*{Gase}
Für ein ideales Gas gilt die Zustandsgleichung
\(
    p=\rho R T
\)
und wir betrachten isentrope (adiabatische, reversible) Kleinstörungen,
also \(s=\mathrm{const}\).
Für ideale Gase gilt bei Isentropie die Poisson-Relation
\(
    p\,\rho^{-\kappa}=\mathrm{const},
\)
wobei \(\kappa=c_p/c_v\) der Adiabatenexponent ist.
Logarithmisch differenziert:
\[
    d(\ln p) - \kappa\, d(\ln \rho) \;=\; 0
    \;\;\Longrightarrow\;\;
    \frac{dp}{p} \;=\; \kappa\,\frac{d\rho}{\rho}.
\]
Damit folgt aus \eqref{eq:speed-of-sound} bei \(s=\mathrm{const}\):
\[
    \left(\frac{\partial p}{\partial \rho}\right)_{s}
    \;=\; \kappa\,\frac{p}{\rho}
\quad\Longrightarrow\quad
    \,c^2 \;=\; \kappa\,\frac{p}{\rho}.
\]
Setzt man \(p=\rho R T\) ein, folgt die übliche Form
\[
    c_g^2 \;=\; \kappa R T
\]
und somit schlussendlich
\begin{equation}
    \,c_g=\sqrt{\kappa R T}\,,
\end{equation}
was die in der Einleitung verwendete Formel \eqref{eq:c-ideal-gas} reproduziert.

\subsubsection*{Flüssigkeiten}

Der (adiabatische) Kompressionsmodul ist definiert als
\begin{equation}
  K \;:=\; -\,V\left(\frac{\partial p}{\partial V}\right)_{s},
  \label{eq:def-bulk}
\end{equation}
wobei $V$ das Volumen der betrachteten Masse ist \cite{schall:will}.
Dabei müssen wir zwei Fälle unterscheiden:

\begin{itemize}
    \item Spezifisches Volumen $v = \frac{1}{\rho}$:
    \[
    K_s := -\,v\left(\frac{\partial p}{\partial v}\right)_{s},\qquad v=\frac{1}{\rho}.
    \quad\Rightarrow\quad
    \left(\frac{\partial v}{\partial \rho}\right)_{s} = -\frac{1}{\rho^2},
    \]
    \[
    \left(\frac{\partial p}{\partial v}\right)_{s}
    = \left(\frac{\partial p}{\partial \rho}\right)_{s}\left(\frac{\partial \rho}{\partial v}\right)_{s}
    = -\rho^2\left(\frac{\partial p}{\partial \rho}\right)_{s},
    \]
    \[
    K_s = -\frac{1}{\rho}\left(-\rho^2\left(\frac{\partial p}{\partial \rho}\right)_{s}\right)
    = \rho\left(\frac{\partial p}{\partial \rho}\right)_{s}.
    \]

\item Gesamtvolumen $V$ bei fester Masse $m$:
    \[
    K_s := -\,V\left(\frac{\partial p}{\partial V}\right)_{s,m},\qquad \rho=\frac{m}{V}.
    \quad\Rightarrow\quad
    \left(\frac{\partial \rho}{\partial V}\right)_{s,m} = -\frac{\rho}{V},
    \]
    \[
    \left(\frac{\partial p}{\partial V}\right)_{s,m}
    = \left(\frac{\partial p}{\partial \rho}\right)_{s}\left(\frac{\partial \rho}{\partial V}\right)_{s,m}
    = -\frac{\rho}{V}\left(\frac{\partial p}{\partial \rho}\right)_{s},
    \]
    \[
    K_s = -V\left(-\frac{\rho}{V}\left(\frac{\partial p}{\partial \rho}\right)_{s}\right)
    = \rho\left(\frac{\partial p}{\partial \rho}\right)_{s}.
    \]
\end{itemize}
In beiden Fällen erhalten wir die gleiche Beziehung:
\begin{equation}
    K \;=\; \rho\left(\frac{\partial p}{\partial \rho}\right)_{s}.
\end{equation}
Mit der akustischen Isentropie-Schliessung $c^2=\big(\frac{\partial p}{\partial \rho}\big)_s$
erhält man sodann
\begin{equation}
    c_l^2 \;=\; \frac{K}{\rho} \quad \Leftrightarrow \quad c_l=\sqrt{\frac{K}{\rho}}.
\end{equation}
Damit ist die in \eqref{eq:c-liquid} verwendete Formel aus den Definitionen von $K$
und der isentropen Schallgeschwindigkeit hergeleitet.

\subsubsection*{Festkörper}
Die Herleitung der Schallgeschwindigkeit in Festkörpern ist ähnlich wie diese
in Flüssigkeiten. Durch die Gitterstruktur eines Festkörpers entstehen
zusätzliche Scherkräfte, die eine weitere Wellenart nämlich die
Transversalwelle ermöglichen. Daher wird für ausführliche Herleitung
auf eines der Fachbücher verwiesen \cite{schall:landaulifschitz,schall:gurtin}.
Zusammengefasst kann aus dem hookeschen Gesetz und der Impulsbilanz
die Navier-Cauchy-Gleichung hergeleitet werden, was in~\ref{openfoam:navierstokes}
hergeleitet wird.
Durch Anwendung von Divergenz und Rotation auf die Navier-Cauchy-Gleichung
kann die Bewegungsgleichung in eine Longitudinal- und eine Transversalwelle
zerlegt werden, woraus die folgenden zwei Gleichungen resultieren:
\[
\begin{aligned}
    \partial_{tt}\theta &= c_L^2\,\Delta\theta, & \qquad c_L^2 &= \frac{\lambda+2\mu}{\rho}\, ,\\
    \partial_{tt}\boldsymbol{\omega} &= c_T^2\,\Delta\boldsymbol{\omega}, & \qquad c_T^2 &= \frac{\mu}{\rho}\, .
\end{aligned}
\]
Im Bereich der Seismologie sind die Longitudinal- und Transversalwellen als
P- und S-Wellen bekannt.

\begin{comment}
In isotropen, homogenen Festkörpern beschreibt die lineare Elastizität (Hooke) die Beziehung zwischen Spannung $\sigma$ und Dehnung $\varepsilon$
Mit den Lamé-Konstanten $(\lambda,\mu)$, wobei $\mu=G$ das Schubmodul ist und $\mathbf{u}$ das Verschibungsfeld im Festkörper beschreibt, lautet im 3D-Fall:
\begin{equation}
    \sigma = \lambda\,(\nabla\!\cdot\! \mathbf{u})\,\mathbf{I} + 2\mu\,\varepsilon,
    \qquad \varepsilon=\tfrac{1}{2}\left(\nabla\mathbf{u}+(\nabla\mathbf{u})^{\!\top}\right).
\end{equation}
wobei $\mathbf{I}$ die Einheitsmatrix ist und $\nabla\mathbf{u}$ die Volumendehnung beschreibt.

\paragraph{Hooke (isotrop, linear)}
Mit Lamé-Konstanten $\lambda,\mu$ gilt
\begin{equation}
  \sigma_{ij} = \lambda\,\varepsilon_{kk}\,\delta_{ij} + 2\mu\,\varepsilon_{ij},
  \qquad
  \varepsilon_{ij} = \tfrac12(\partial_i u_j + \partial_j u_i).
  \label{eq:hooke-index}
\end{equation}

\paragraph{Impulsbilanz (ohne Körperkräfte)}
\begin{equation}
  \rho\,\partial_{tt} u_i = \partial_j \sigma_{ij}.
  \label{eq:momentum-index}
\end{equation}

\paragraph{Der „Einsetz–Schritt“ in Indexschreibweise}
Setze \eqref{eq:hooke-index} in \eqref{eq:momentum-index} ein:
\[
    \partial_j \sigma_{ij}
    = \partial_j\!\big(\lambda\,\varepsilon_{kk}\,\delta_{ij} + 2\mu\,\varepsilon_{ij}\big)
    = \lambda\,\partial_i \varepsilon_{kk} + 2\mu\,\partial_j \varepsilon_{ij},
\]
denn $\partial_j(\delta_{ij}X)=\partial_i X$.
Nun $\varepsilon_{kk} = \partial_k u_k = \nabla\!\cdot\!\mathbf{u}$ und
\[
    \partial_j \varepsilon_{ij}
    = \frac12\,\partial_j(\partial_i u_j + \partial_j u_i)
    = \frac12\big(\partial_i\partial_j u_j + \partial_j\partial_j u_i\big)
    = \frac12\big(\partial_i(\partial_k u_k) + \Delta u_i\big).
\]
Damit
\[
    \partial_j \sigma_{ij}
    = \lambda\,\partial_i(\partial_k u_k) + 2\mu\cdot \frac12\big(\partial_i(\partial_k u_k) + \Delta u_i\big)
    = (\lambda+\mu)\,\partial_i(\partial_k u_k) + \mu\,\Delta u_i.
\]
In Vektorschreibweise:
\begin{equation}
    \boxed{\,\rho\,\partial_{tt}\mathbf{u} \;=\; (\lambda+\mu)\,\nabla(\nabla\!\cdot\!\mathbf{u}) \;+\; \mu\,\Delta \mathbf{u}\,}
    \label{eq:navier}
\end{equation}
Dies ist die \emph{Navier–Cauchy–Gleichung}.

\paragraph{Wellenzerlegung per Divergenz und Rotation}
Wende $\nabla\!\cdot$ auf \eqref{eq:navier} an.
Da $\nabla\!\cdot(\Delta\mathbf{u})=\Delta(\nabla\!\cdot\!\mathbf{u})$ gilt:
\[
    \rho\,\partial_{tt}(\nabla\!\cdot\!\mathbf{u})
    = (\lambda+\mu)\,\Delta(\nabla\!\cdot\!\mathbf{u}) + \mu\,\Delta(\nabla\!\cdot\!\mathbf{u})
    = (\lambda+2\mu)\,\Delta(\nabla\!\cdot\!\mathbf{u}).
\]
Setze $\theta := \nabla\!\cdot\!\mathbf{u}$:
\begin{equation}
    \boxed{\,\partial_{tt}\theta = c_L^2\,\Delta\theta,\qquad c_L^2=\frac{\lambda+2\mu}{\rho}\,}
    \label{eq:longitudinal}
\end{equation}
Dies ist die \emph{Longitudinalwelle} (Druck-/Volumendehnung).
Wende nun den Rotations-Operator $\nabla\times$ auf \eqref{eq:navier} an.
Wegen $\nabla\times\nabla f=\mathbf{0}$ verschwindet der erste Term:
\[
    \rho\,\partial_{tt}(\nabla\times\mathbf{u}) = \mu\,\nabla\times(\Delta\mathbf{u})
    = \mu\,\Delta(\nabla\times\mathbf{u}).
\]
Mit $\boldsymbol{\omega}:=\nabla\times\mathbf{u}$:
\begin{equation}
    \boxed{\,\partial_{tt}\boldsymbol{\omega} = c_T^2\,\Delta\boldsymbol{\omega},\qquad c_T^2=\frac{\mu}{\rho}\,}
    \label{eq:transversal}
\end{equation}
Dies ist die \emph{Transversalwelle} (Scherung).

\paragraph{Interpretation}
\begin{itemize}
  \item $c_L$ hängt von \emph{Volumensteifigkeit} ab: $(\lambda+2\mu)$ skaliert den Widerstand gegen Volumenänderung (\emph{Druck-/Kompressionswelle}).
  \item $c_T$ hängt allein von der \emph{Schersteifigkeit} $\mu$ ab (\emph{Scherwelle}); in Fluiden ist $\mu\approx 0$, daher keine Transversalwellen.
\end{itemize}

Somit erhält man in isotropen Medien zwei Wellengeschwindigkeiten \cite{schall:landaulifschitz,schall:gurtin}:
\begin{align}
    \boxed{\,c_T=\sqrt{\frac{\mu}{\rho}}=\sqrt{\frac{G}{\rho}}\,} \quad &\text{(transversal, Scherwellen)},\\[2mm]
    \boxed{\,c_L=\sqrt{\frac{\lambda+2\mu}{\rho}}\,} \quad &\text{(longitudinal, Druckwellen)}.
\end{align}
In technischen Materialparametern, sprich Elastizitätsmodul $E$ und Poissonzahl $\nu$, mit
\[
    \mu=G=\frac{E}{2(1+\nu)},
    \qquad
    \lambda=\frac{E\,\nu}{(1+\nu)(1-2\nu)},
\]
ergeben sich die in der Einleitung verwendeten Ausdrücke
\begin{align}
    \boxed{\,c_{s,\mathrm{long}}
    = \sqrt{\frac{E(1-\nu)}{\rho(1+\nu)(1-2\nu)}}\,},\qquad
    \boxed{\,c_{s,\mathrm{trans}}
    = \sqrt{\frac{E}{2\rho(1+\nu)}}=\sqrt{\frac{G}{\rho}}\,}.
\end{align}
Typisch ist $c_L>c_T$, und beide liegen deutlich über Gas- und Flüssigkeitswerten (z.\,B. Stahl $c\approx 5{,}0\text{--}5{,}9\,\mathrm{km/s}$ für Longitudinalwellen, je nach Legierung).

\end{comment}

\subsection{Reduktion auf die in der Einleitung verwendeten Vereinfachungen}
Unter den in Abschnitt \ref{schall:section:teil0} genannten Annahmen (eindimensional,
kleine Störungen, keine Reflexionen/Hindernisse sowie konstante
Materialparameter) reduziert sich die Dynamik auf die lineare
Wellengleichung
\[
    \frac{\partial^2 \phi}{\partial t^2}=c^2\,\frac{\partial^2 \phi}{\partial x^2},
\]
wobei $c$ je nach Medium durch die obigen isentropen bzw.\ elastischen
Beziehungen gegeben ist:
\[
    c=\begin{cases}
    \sqrt{\kappa\,p/\rho}=\sqrt{\kappa R T}, & \text{Gase},\\[1mm]
    \sqrt{K/\rho}, & \text{Flüssigkeiten},\\[1mm]
    \sqrt{(\lambda+2\mu)/\rho}\;\text{ bzw. }\;\sqrt{\mu/\rho}, & \text{Festkörper (long./trans.)}.
\end{cases}
\]
Diese Formen entsprechen genau den in der Einleitung verwendeten
vereinfachten Geschwindigkeitsausdrücken und motivieren die
tabellarischen Zahlenwerte für typische Medien.
