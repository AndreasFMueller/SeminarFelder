\newcommand{\eot}{\boldsymbol{e}_{12}}
% \newcommand{\eott}{\boldsymbol{e}_{123}}
\section{Berechnungen mit der geometrischen Algebra}
\kopfrechts{Berechnungen mit der geometrischen Algebra}%
Da für die Standardbasisvektoren $\eone, \etwo$ das Skalarprodukt $\eone \cdot \etwo = 0$ ergibt,
folgt, dass das geometrische Produkt zwischen $\eone$ und $\etwo$
\begin{align}
  \eone \etwo &= \eone \cdot \etwo + \eone \wedge \etwo
\notag
\\
  &= \eone \wedge \etwo
  \label{geoalgebra:eq:eot}
\end{align}
ergibt.
Daraus folgt, dass
\begin{align*}
  \eone^2 &= \eone \cdot \eone + \eone \wedge \eone \\
  &= \eone \cdot \eone \\
  &= 1.
\end{align*}

Der Ausdruck $\eone \etwo$ wird im Folgenden zu $\eot$ verkürzt.
In \eqref{geoalgebra:eq:eot} wurde aber gezeigt, dass $\eone \etwo$ eben
genau $\eone \wedge \etwo$ entspricht. Es gilt
\begin{equation}
  \eot = \eone \wedge \etwo.
  \label{geoalgebra:eq:eot_wedge}
\end{equation}
Analog dazu bedeutet $\boldsymbol{e}_{121} = \eone \etwo \eone$.
Anders als in \eqref{geoalgebra:eq:eot_wedge} lässt sich durch Anwenden von \eqref{geoalgebra:eq:antikommutativ}
aber zeigen, dass
\begin{equation*}
  \boldsymbol{e}_{121} = - \boldsymbol{e}_{112} = -\etwo. 
\end{equation*}
Jedesmal, wenn ein Index ``getauscht'' wird ($\boldsymbol{e}_{121} = -\boldsymbol{e}_{112}$), wird auch das Vorzeichen
gewechselt.

In \autoref{geoalgebra:section:trivectors-n-vectors} wurde bereits auf Trivektoren und $n$-Vektoren eingegangen. 
In der geometrischen Algebra
spricht man auch von \emph{Graden}. Ein Skalar ist von
\index{Grad}%
Grad $0$, ein Vektor ist von Grad $1$, ein Bivektor ist von Grad $2$, ein Trivektor von Grad $3$ und so weiter.

Desweiteren wird von \emph{Multivektoren} gesprochen, wenn eine Summe von verschiedenen Graden vorliegt, zum Beispiel
\begin{align*}
A &= 2 + 3 \eone + \eot \\
B &= 1 + \eone + 2 \eot.
\end{align*}

Mit der geometrischen Algebra ist ähnlich zu den komplexen Zahlen eine Algebra definiert,
mit welcher gerechnet werden kann. So können zum Beispiel Multivektoren addiert
\begin{align*}
A + B &= (2 + 3 \eone + \eot) + (1 + \eone + 2 \eot) \\
&= 3 + 4 \eone + 3 \eot,
\end{align*}
oder sogar multipliziert werden:
\begin{align*}
A  B
&=
(2 + 3 \eone + \eot) (1 + \eone + 2 \eot)
\notag
\\
&=
2 + 2 \eone + 4 \eot + 3 \eone + 3 \eone^2 + 6 \eone \eot + \eot + \eot \eone + 2 \eot^2
\notag
\\
&=
2 + 2 \eone + 4 \eot + 3 \eone + 3 + 6 \etwo + \eot - \etwo - 2
\notag
\\
&=
3 + 5 \eone + 5 \etwo + 5 \eot.
\end{align*}

Ausserdem ist das geometrische Produkt assoziativ
\index{assoziativ}%
\begin{equation*}
A (B C) = (A B) C
\end{equation*}
und es gilt das Distributivgesetz
\index{Distributivgesetz}%
\begin{equation*}
  A (B + C) = AB + AC.
\end{equation*}


\newcommand\equalhat{\mathrel{\stackon[1.5pt]{=}{\stretchto{%
    \scalerel*[\widthof{=}]{\wedge}{\rule{1ex}{3ex}}}{0.5ex}}}}

\subsection{Zusammenhang zwischen Multivektoren und komplexen Zahlen}
Wir betrachten den Bivektor $\eone \wedge \etwo = \eot$ und quadrieren ihn:
\begin{align*}
    \eot^2 &= \boldsymbol{e}_{1212} \\
    &= -\boldsymbol{e}_{1122} \\
    &= -\boldsymbol{e}_{11} \boldsymbol{e}_{22} \\
    &= -1.
\end{align*}
Das Quadrat ergibt also $(\eone \wedge \etwo)^2 = -1$. Der Bivektor
$(\eone \wedge \etwo)$ entspricht also effektiv der imaginären Zahl $i$.

