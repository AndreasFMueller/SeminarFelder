\section{Überschallknall: Entstehung und Low-Boom-Strategien}
\label{schall:section:boom}

Ein Flugzeug mit Machzahl $\textit{Ma}=V/c_0>1$ erzeugt entlang seiner Länge diskrete
Schockbeiträge an Bug, Tragflächen, Rumpfänderungen, Leitwerke und Auspuff,
die sich bei der Ausbreitung in der Atmosphäre je nach Grösse des Flugzeugs
überlagern und am Boden als impulsförmiger Knall wahrgenommen werden.
Die \emph{Nahfeld}-Geometrie, definiert durch die Flugzeugform sowie die  Last- oder
Auftriebsverteilung, bestimmt die anfängliche Drucksignatur;
die \emph{Fernfeld}-Ausbreitung formt diese durch Nichtlinearität und atmosphärische Dämpfung.

\subsection{Nahfeld: $F$-Funktion, Äquivalentfläche und Initialsignatur}
Im Überschallflug überlagern sich alle vom Flugzeug erzeugten Druckänderungen zu
einer mitwandernden Wellenstruktur im Nahfeld.
Entscheidend ist dabei nicht die absolute Grösse, sondern die \emph{Längsverteilung} der Querschnittsfläche: Sanfte, stetige Änderungen entlang des Rumpfs führen zu weichen
Druckanstiegen, abrupte Geometriesprünge zu schärferen Schockanteilen.
Whithams Ansatz fasst die reale Geometrie (Rumpf, Flügel, Einläufe) zu einer \emph{Äquivalentfläche} zusammen und liefert daraus eine kompakte Kenngrösse, die wir hier als \(F\)-Funktion nutzen, um
die Form der Nahfeldsignatur qualitativ zu lesen: Wo die Geometrie „sanft“ verläuft, bleibt die
Signatur glatt; wo sie „kantig“ wird, entstehen lokale Spitzen.
Diese Nahfeldsignatur dient als Startpunkt für die weitere Ausbreitung
in der Atmosphäre — Ziel von Low-Boom-Entwürfen ist daher eine möglichst glatt
verlaufende Signatur bereits im Nahfeld.

Für schlanke Überschallkörper liefert Whithams lineare Wellentheorie eine Kopplung
zwischen der Rumpf- oder Flügel-Geometrie (Äquivalentfläche $A(x)$) und der
Nahfeldsignatur über die sogenannte \emph{F-Funktion}~\cite{schall:whitham}.
Whithams Theorie definiert die zugehörige \emph{F-Funktion}
als Abel-Transformierte von $A_e''$:
\begin{equation}
    F(x_e)=\frac{1}{2\pi}\int_{0}^{x_e}\frac{A_e''(t)}{\sqrt{x_e-t}}\,dt,
\end{equation}
wobei $x_e$ eine entlang der Flugrichtung skalierte Koordinate ist.
Die dimensionslose Nahfeld-Überdrucksignatur ist, bis auf bekannte
Mach-/Gasfaktoren, proportional zu $F(x_e)$.
Glatte, wiederstandsarme $F$-Verläufe führen zu \emph{Low-Boom}-Signaturen.
Intuitiv gilt: stark gekrümmte Änderungen in $A'(x)$ erzeugen starke lokale
Schockanteile; eine „sanft“ verteilte Geometrie verteilt die Druckanstiege
längs der Rumpflänge.
Die klassische Low-Boom-Minimierung nach Seebass-George-Darden entwirft
Zielverteilungen für $A(x)$, die eine geringe Schockstärke am Eintritt in
die Atmosphäre bewirken \cite{schall:whitham,schall:seebassgeorge,schall:darden75}.

\subsection{Ausbreitung: (Augmentierte) Burgers-Gleichung}
Die Fernfeldausbreitung eines Überschalldrucksignals lässt sich bildhaft als
„fahrende Welle“ verstehen, die sich mit der Schallgeschwindigkeit vom
Flugzeug wegbewegt.
Zwei Prozesse formen diese Welle unterwegs entscheidend um:
\emph{Nichtlinearität} schiebt energiereiche Signalanteile (hoher Druck) nach
vorn und lässt die Flanke zunehmend steiler werden;\emph{Dämpfung} in der
Atmosphäre nimmt den hohen Frequenzen die Spitze und glättet die Form.
Zusätzlich nimmt die Amplitude mit wachsender Entfernung durch reine Ausbreitung ab.
Die augmentierte Burgers-Gleichung fasst genau diese drei
Wirkungen in einem Modell zusammen — ohne dass wir an dieser Stelle rechnen
müssen: Steilung nach vorn, Glättung durch Verluste, Abschwächung durch
Ausbreitung.

Bei der Fernfeld-Übertragung durch eine ruhende, schichtweise homogene
Atmosphäre (ohne Wind) wird die skalierte Schaltdrucksignatur $p(\tau,x)$
(hier $\tau$ = „retardierte Zeit“) häufig mit einer \emph{augmentierten Burgers-Gleichung},
dessen ursprüngliche Form im Kapitel \ref{chapter:neuronal} vertieft
behandelt wird, modelliert:
\begin{equation}
  \frac{\partial p}{\partial x}
  \;+\;\frac{\beta}{\rho_0 c_0^{3}}\,p\,\frac{\partial p}{\partial \tau}
  \;=\;
  -\,\frac{m}{x}\,p
  \;+\;\delta\,\frac{\partial^2 p}{\partial \tau^2},
  \label{eq:aug-burgers}
\end{equation}
wobei $\frac{m}{x}$ geometrische Abschwächung beschreibt, der nichtlineare
Term, welcher mit $\beta$ multipliziert wird, zur Intensität der Schockanteile
führt und $\delta$ frequenzabhängige atmosphärische Verluste aggregiert.
Grobe Faustregel: Nichtlinearität „zieht“ die Front zu einem N-Profil.
Unter einem N-Profil versteht man einen zeitlichen Verlauf mit einem kurzen,
steilen Druckanstieg (Kompressionsstoss) und einem anschliessenden, deutlich
längeren, nahezu linearen Abfall bis unter das Ausgangsniveau, bevor der
Druck langsam wieder auf den Umgebungswert zurückkehrt — die Kurve erinnert
damit an den Buchstaben „N“.
Physikalisch liegt das daran, dass Kompressionsanteile sich in nichtlinearen Medien
schneller ausbreiten als Verdünnungsanteile: Der vordere, hochamplitudige Teil
des Signals holt den hinteren ein, die Front steilt auf und es bildet sich ein Stoss.
Dämpfung in der Atmosphäre entfernt bevorzugt hohe Frequenzen, sodass scharfe Kanten
abgerundet werden und der Spitzenüberdruck \(\Delta p_\mathrm{max}\) sinkt;
gleichzeitig schwächt die geometrische Ausbreitung die Gesamtsamplitude.
\cite{schall:rallabhandi2023,schall:rallabhandiAIAA2023,schall:burgersJASA}.

\subsection{Boden-Signatur und Metriken}
Die für Wahrnehmung massgebliche Metrik ist häufig der \emph{Perceived Level},
der die zeitliche Impulsform und psychoakustische Bewertungen kombiniert.
Ziel moderner Low-Boom-Demonstratoren ist ein \emph{„sonic thump“} statt
eines klassischen Booms, mit einem Zielwerte um
${\sim}75$\ PLdB in Standardatmosphäre \cite{schall:x59pldb}.
\todo[inline]{PLdB erklären}

\subsection{Hebel zur Boom-Reduktion}
\paragraph{Aerodynamische Formgebung (Low-Boom Shaping).}
Gestaltung von $A(x)$ und der Auftriebsverteilung entlang der Länge,
um grosse Einzel-Schocks zu vermeiden und verteilte, kleinere Sprünge zu generieren.
Dies kann durch eine lange, schlanke Nase, durch Flügel- oder Canard-Anordnung,
verdeckte Triebwerksintegration oder durch Vermeidung starker
Querschnittssprünge realisiert werden.
Formal: Minimierung geeigneter Funktionale der F-Funktion bzw.\ der Nahfeldsignatur
gemäss Seebass-George-Darden-Ideen \cite{schall:seebassgeorge,schall:darden75}.

\paragraph{Flughöhe und Machzahl.}
Höhere Flughöhe erhöht die geometrische Abschwächung, nachvollziehbar durch
grösseres $x$ in \eqref{eq:aug-burgers}, und die atmosphärische Dämpfung durch
grössere effektive $\delta$; moderate Machzahl reduziert die erzeugten
Schockstärken bereits im Nahfeld.

\paragraph{Flugweg- und Operationsführung.}
Überlandflug mit Unterschall, Kurswahl, Steigprofile, und Wetter-Routing
durch Vermeidung fokussierender Schichtungen, welche in Kapitel
\ref{schall:subsection:atmos-scenarios} behandelt wurden, können die
Boom-Teppich-Geometrie signifikant beeinflussen.

\paragraph{Atmosphärische Effekte berücksichtigen.}
Temperatur- und Wind-Schichtung $c(z)$, Windschubvektoren und Turbulenz
modulieren Divergenz und effektive Dämpfung in \eqref{eq:aug-burgers};
in Inversionslagen kann es zu \emph{Fokussierung} kommen (lokal erhöhte
$\Delta p$), in Normal-Lapsen eher zu \emph{Defokussierung}.