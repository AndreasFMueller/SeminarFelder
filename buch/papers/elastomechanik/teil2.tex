%
% teil2.tex -- Beispiel-File für teil2 
%
% (c) 2020 Prof Dr Andreas Müller, Hochschule Rapperswil
%
% !TEX root = ../../buch.tex
% !TEX encoding = UTF-8
%
\section{Einführung in die mathematische Formulierung der Elastizitätstheorie}
\label{elastomechanik:section:teil2}
\subsection{Annahmen und Grundprinzipien}
Wird ein Festkörper mechanisch belastet, entstehen im Inneren Spannungen. Diese Spannungen führen zu Verformungen, die sich als Verschiebungen einzelner Punkte des Körpers äussern.
Aus dem Verschiebungsfeld lassen sich Verzerrungen berechnen, wodurch sich ein sogenannter Verzerrungszustand ausbildet.

Um das mechanische Verhalten eines Körpers unter Belastung vollständig beschreiben zu können, werden Verschiebungen, Verzerrungen und Spannungen durch die folgenden drei Grundgleichungen der linearen Elastizitätstheorie erfasst \cite{elastomechanik:Technische_Mechanik_2:Elastostatik}:

\begin{description}
	\item [\textbf{Kinematischen Gleichungen:}] Stellen den Zusammenhang zwischen den Verschiebungen $u_i$ und den daraus resultierenden Verzerrungen $\varepsilon_{ij}$ her. 
	Sie bilden die geometrische Grundlage der Deformation.
	
	\item [\textbf{Konstitutiven Gleichungen:}] Auch bekannt als Materialgesetze. 
	Sie beschreiben den materialabhängigen Zusammenhang zwischen den Spannungen $\sigma_{ij}$ und den Verzerrungen $\varepsilon_{ij}$. 
	Im Fall linear-elastischer, isotroper Materialien erfolgt dies über das Hooke’sche Gesetz.
	
	\item [\textbf{Gleichgewichtsbedingungen:}] Gewährleisten das mechanische Gleichgewicht im Festkörper, indem sie fordern, dass die Summe aller inneren Spannungen und äusseren Kräfte in jedem Punkt null ergibt.
\end{description}

\subsection{Spannungstensor und Verzerrungstensor}
Durch die Cauchy-Spannungstensor werden Spannungsvektooren in jedem beliebige Schnittrichtung zugeornet, der auf der betreffenden Schnittfläche wirkt.
Dieser Zusammenhang wird durch die Cauchy’sche Formel beschrieben:
	\begin{equation}
		t_i = 
		\sigma_{ij} \cdot n_j
	\end{equation}
Der Spannungstensor ist aufgrund des Momentengleichgewichts an einem infinitesimalen Volumenelement symmetrisch. Schubspannungen müssen sich dabei paarweise ausgleichen. 
Der Tensor enthält sowohl die Normal- als auch die Schubspannungen und beschreibt somit vollständig den inneren Spannungszustand im Material.

Die Deformation des Körpers wird durch den Verzerrungstensor beschrieben, welcher die lokalen Längen- und Winkeländerungen in allen drei Raumrichtungen erfasst.

Zur Herleitung der linearen Verzerrungen im räumlichen Fall werden zunächst die Verschiebungsänderungen entlang der Koordinatenachsen untersucht. 
Aus diesen Änderungen wird beispielsweise in $x$-Richtung die relative Längenänderung berechnet, was direkt zur Definition der Dehnung $\varepsilon_{xx}$ führt. 
Dieses Vorgehen lässt sich analog auf die übrigen Normal- und Schubverzerrungen übertragen.

Die einzelnen Komponenten des Verzerrungstensors lauten \cite{elastomechanik:Technische_Mechanik_2:Elastostatik}:
	\begin{equation}
		\varepsilon_{xx} =
		\frac{\partial u}{\partial x}
	\end{equation}
	und
	\begin{equation}
		\varepsilon_{xy} =
		\frac{1}{2} \left( \frac{\partial u}{\partial y} + \frac{\partial v}{\partial x} \right)
	\end{equation}

\subsection{Elastizitätsgesetz und Materialkonstanten}
Aus dem Hooke’schen Gesetz ist bekannt, dass im einachsigen, linearen Fall folgender Zusammenhang zwischen Spannung und Dehnung gilt:
	\begin{equation}
		\sigma = 
		E \cdot \varepsilon
	\end{equation}
Im dreiachsigen Fall kann die lineare Beziehung zwischen den Spannungs- und Verzerrungskomponenten durch einen Elastizitätstensor 4. Stufe beschrieben werden:
	\begin{equation}
		\sigma_{ij} = 
		E_{ijkl} \cdot \varepsilon_{kl}
	\end{equation}
Da wir den dreidimensionalen Fall unter der Annahme eines isotropen Materials betrachten, bei dem die mechanischen Eigenschaften in alle Richtungen gleich sind, reduziert sich die Form des Elastizitätstensors auf zwei unabhängige Konstanten: die sogenannten Lame’schen Konstanten $\lambda$ und $\mu$. 
Damit ergibt sich das Elastizitätsgesetz in kompakter Form:
	\begin{equation}
		\sigma_{ij} = 
		\lambda \cdot \delta_{ij} \cdot \varepsilon_{kk} + 2\mu \cdot \varepsilon_{ij}
	\end{equation}
Die Lame’schen Konstanten lassen sich durch die technischen Materialkonstanten $E$ (Elastizitätsmodul), $\nu$ (Poissonzahl) und $G$ (Schubmodul) wie folgt ausdrücken:
	\begin{equation}
		E = 
		\mu \cdot \frac{3\lambda + 2\mu}{\lambda + \mu}, \quad 
		\nu = 
		\frac{\lambda}{2(\lambda + \mu)}, \quad 
		G = 
		\mu
	\end{equation}
Durch Einsetzen in die Spannungs-Verzerrungs-Beziehung kann das Elastizitätsgesetz auch in Form der Deviatorzerlegung dargestellt werden. 
Dabei ergibt sich der Zusammenhang zwischen Spannungs- und Verzerrungsdeviator \cite{elastomechanik:Grundlagen_der_Elastizitaetstheorie}:
	\begin{equation}
		s_{ij} = 
		\sigma_{ij} - \frac{1}{3} \sigma_{kk} \delta_{ij} = 
		2\mu e_{ij}
	\end{equation}
