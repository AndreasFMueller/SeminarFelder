%
% einleitung.tex -- Beispiel-File für die Einleitung
%
% (c) 2020 Prof Dr Andreas Müller, Hochschule Rapperswil
%
% !TEX root = ../../buch.tex
% !TEX encoding = UTF-8
%
\section{Einleitung Zu Open Foam\label{openfoam:section:Einleitung}}
\kopfrechts{Einleitung Zu Open Foam}
\subsection{Ziel und Inhalt unserer Arbeit}

\subsection{Was ist eigentlich OpenFOAM?\label{openfoam:section:WasIstOpenFoam}}
OpenFOAM ist, wie der Name sagt, eine Open Source Software die unter der GNU
General Public License Kostenlos zugänglich ist und dadurch auch für Kommerzielle Zwecke 
verwendet werden kann. 
Bei OpenFOAM handelt es sich um eine \emph{CFD} Software, 
dies steht für \emph{C}omputational \emph{F}luid \emph{D}ynamics also Numerische Strömungsdynamik.
OpenFOAM kann dementsprechend verwendet werden um näherungsweise Lösungen für 
Strömungsmechanische Probleme zu finden.
Es können zum Beispiel Widerstandsbeiwerte von Autos oder die Wirbelschleppen einer Tragfläche simuliert werden.

OpenFOAM wird seit 2004 hauptsächlich von OpenCFD Ltd entwickelt, es wird jedoch auch aktiv durch die Community weiterentwickelt.
Die getesteten Releases werden alle 6 Monate herausgegeben.~\cite{openfoam:greenshieldsweller2022}
%TODO: add OpenFoam reference
\subsection{Was macht OpenFOAM?}
Wie in~\ref{openfoam:section:WasIstOpenFoam} erwähnt ist OpenFOAM eine Software für das lösen von Strömungsmechanischen Problemen.
OpenFOAM kann also die Bewegung und Kräfte die durch Flüssigkeiten ausgehen berechnen, 
dies ist heute aber auch noch mit Wärmetransfer also Thermodynamik, Chemische Prozesse wie Verbrennung
und die Wechselwirkung von Festkörpern und Flüssigkeiten zu erweitern. 
All diese Prozesse kann eine CFD Software mittels Numerischen Methoden näherungsweise simulieren. 
~\cite{openfoam:greenshieldsweller2022} 