\section{Geometrisches Produkt}
Im Rahmen der geometrischen Algebra ist das \emph{geometrische Produkt} eine zentrale Operation. 
\index{geometrisches Produkt}%
\index{Produkt, geometrisch}%
Es kombiniert das \emph{innere Produkt} (Skalarprodukt) und das
\index{inneres Produkt}%
\index{Skalarprodukt}%
\emph{äussere Produkt} (Wedgeprodukt) in einer Operation
\index{äusseres Produkt}%
\index{Produkt, äusseres}%
\begin{equation*}
\a \b = \a \cdot \b + \a \wedge \b
\end{equation*}
und kombiniert somit einen Skalar (0-Vektor) mit einem Bivektor (2-Vektor).

Das folgende Lemma ermöglicht, zwischen dem geometrischen Produkt und dem altbekannten
Skalarprodukt und dem Wedgeprodukt umzurechnen.

\begin{lemma}[Rechenregeln für das geometrische Produkt]
  Für beliebige Vektoren $\a$ und $\b$ gilt:
  \begin{enumerate}
    \item Vertauschungsregel:
\index{Vertauschungsregel}%
\begin{align}
        \a \b &= \a \cdot \b + \a \wedge \b,\label{geoalgebra:eq:ab} \\
        \b \a &= \a \cdot \b - \a \wedge \b.\label{geoalgebra:eq:ba}
      \end{align}
    \item Skalar- und Wedgeprodukt können durch das geometrische Produkt ausgedrückt werden: \begin{align}
        \a \cdot \b = \frac{1}{2} (\a \b + \b \a),\\
        \a \wedge \b = \frac{1}{2} (\a \b - \b \a).
        \label{geoalgebra:eq:geo-product-wedge}
      \end{align}
    \item Norm: \begin{equation*}
      \a^2 = |\a|^2.
    \end{equation*}
  \end{enumerate}
  \label{geoalgebra:lemma:rechenregeln-geom-produkt}
\end{lemma}
\noindent Das Lemma \ref{geoalgebra:lemma:rechenregeln-geom-produkt} lässt sich folgenderweise beweisen:
\begin{proof}
  1. Da das Wedgeprodukt antikommutativ ist (Lemma~\ref{geoalgebra:eq:antikommutativ}), folgt
  \begin{equation*}
    \b \a = \b \cdot \a + \b \wedge \a = \a \cdot \b - \a \wedge \b.
  \end{equation*}
  2. Durch Addieren von \eqref{geoalgebra:eq:ab} und \eqref{geoalgebra:eq:ba} erhalten wir
  \begin{align*}
      \a \b + \b \a &= 2 (\a \cdot \b) \\
      \a \cdot \b &= \frac{1}{2} (\a \b + \b \a).
  \end{align*}
  Durch Subtrahieren von \eqref{geoalgebra:eq:ab} und \eqref{geoalgebra:eq:ba} erhalten wir
  \begin{align*}
      \a \b - \b \a &= 2 (\a \wedge \b) \\
      \a \wedge \b &= \frac{1}{2} (\a \b - \b \a).
  \end{align*}
  3. Das Quadrat eines Vektors ist
  \begin{equation*}
    \a^2 = \a \cdot \a + \underbrace{\a \wedge \a}_{\displaystyle=0} = |\a|^2.
  \end{equation*}
  Damit sind alle Formeln bewiesen.
\end{proof}


