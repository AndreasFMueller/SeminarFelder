%
% teil3.tex -- Beispiel-File für Teil 3
%
% (c) 2020 Prof Dr Andreas Müller, Hochschule Rapperswil
%
% !TEX root = ../../buch.tex
% !TEX encoding = UTF-8
%
\section{Mathematische Herleitung der Feldgleichungen der
linearen Elastizitätstheorie}
\label{elastomechanik:section:herleitung2}
\kopfrechts{Feldgleichungen der linearen Elastizitätstheorie}%

\subsection{Kinematische Beziehungen und Verzerrungstensor}
Wir beschreiben die Verformung eines Punktes durch seinen
Verschiebungsvektor $u_i(x)$ mit $i$ = 1,2,3 ($x$, $y$, $z$).
$u_1$ entspricht der Verschiebung in $x$-Richtung, $u_2$ beschreibt
die Verschiebung in $y$-Richtung und $u_3$ beschreibt die Verschiebung
in $z$-Richtung.
Ausgehend vom Verschiebungsfeld $u_i(x)$ definieren wir den
Verzerrungstensor $\varepsilon_{ij}$ als symmetrisierten Gradienten
des Verschiebungsfeldes:
\begin{equation*}
	\varepsilon_{ij} = 
	\frac{1}{2} \left( \frac{\partial u_i}{\partial x_j} + \frac{\partial u_j}{\partial x_i} \right).
\end{equation*}
Diese Definition gewährleistet die Symmetrie des Verzerrungstensors
($\varepsilon_{ij}$ = $\varepsilon_{ji}$) und erfasst sowohl Dehnungen
($i = j$) als auch Scherungen ($i \neq j$).
In tensorieller Schreibweise sieht die Formel folgendermassen aus:
\begin{equation*}
	\boldsymbol{\varepsilon} = 
	\frac{1}{2} \left( \nabla \vec{u} + (\nabla \vec{u})^T \right).
\end{equation*}

\subsection{Konstitutives Gesetz für isotrope Materialien
\label{elastomechanik:math:subsection:isotrop}}
Das hookesche Gesetz in tensorieller Form verknüpft Spannungen
$\sigma_{ij}$ mit Verzerrungen $\varepsilon_{kl}$
\begin{equation}
	\sigma_{ij} = 
	\lambda \delta_{ij} \varepsilon_{kk} + 2\mu \varepsilon_{ij}.
\label{elastomechanik:math:eqn:materialgesetz}
\end{equation}
Hierbei sind $\lambda$ und $\mu$ (Lamé-Konstanten) materialabhängige
Parameter.
Für den Zusammenhang mit technischen Konstanten gilt:
\begin{equation*}
	\mu = 
	G = 
	\frac{E}{2(1+\nu)}, \quad \lambda = 
	\frac{E \nu}{(1+\nu)(1-2\nu)}.
\end{equation*}	
Als deviatorische Darstellung ergibt das:
\begin{equation*}
	s_{ij} =
	2\mu e_{ij}
\end{equation*}
mit
\begin{equation*}
	e_{ij} = 
	\varepsilon_{ij} - \frac{1}{3} \varepsilon_{kk} \delta_{ij}.
\end{equation*}

\subsection{Navier-Cauchy-Gleichungen (Feldgleichungen)}
Durch Einsetzen des Materialgesetzes
\eqref{elastomechanik:math:eqn:materialgesetz}
in die Gleichgewichtsbedingungen
\eqref{elastomechanik:math:eqn:gleichgewicht}
erhalten wir die fundamentalen Feldgleichungen:
\begin{equation*}
	\mu \nabla^2 u_i + (\lambda + \mu) \frac{\partial}{\partial x_i} (\nabla \cdot \vec{u}) + f_i =
	0.
\end{equation*}
In Vektorform ergibt das:
\begin{equation*}
	\mu \nabla^2 \vec{u} + (\lambda + \mu) \nabla (\nabla \cdot \vec{u}) + \vec{f} = 
	0.
\end{equation*}
Die Herleitung gestaltet sich wie folgt.

\begin{proof}[Herleitung]
Jetzt setzen wir die kinematischen, konstitutiven und Gleichgewichtsbeziehungen zusammen, um die vollständigen Feldgleichungen der linearen Elastizitätstheorie herzuleiten.
Wir ersetzen $\sigma_{ij}$ in die Gleichgewichtsbedingung:
\begin{equation*}
	\frac{\partial}{\partial x_j}\left( \lambda \delta_{ij} \varepsilon_{kk} + 2\mu \varepsilon_{ij} \right) + f_i = 
	0.
\end{equation*}
Einzeln ausrechnen:
\begin{equation*}
	\lambda \frac{\partial \varepsilon_{kk}}{\partial x_i} + 2\mu \frac{\partial \varepsilon_{ij}}{\partial x_j} + f_i = 
	0.
\end{equation*}
Jetzt drücken wir die Verzerrungen durch die Verschiebungen aus:
\begin{align*}
	\varepsilon_{kk} &= 
	\frac{\partial u_1}{\partial x_1} + \frac{\partial u_2}{\partial x_2} + \frac{\partial u_3}{\partial x_3} = 
	\nabla \cdot \vec{u}
	\\
	\varepsilon_{ij} &= 
	\frac{1}{2}\left( \frac{\partial u_i}{\partial x_j} + \frac{\partial 	u_j}{\partial x_i} \right).
\end{align*}
Zwischenschritt: Ableitung von $\varepsilon_{ij}$ einsetzen:
\begin{equation*}
	\frac{\partial \varepsilon_{ij}}{\partial x_j} = 
	\frac{1}{2}\left( \frac{\partial^2 u_i}{\partial x_j \partial x_j} + \frac{\partial^2 u_j}{\partial x_i \partial x_j} \right).
\end{equation*}
und wegen Vertauschbarkeit der Ableitungen:
\begin{equation*}
	\frac{\partial^2 u_j}{\partial x_i \partial x_j} = 
	\frac{\partial}{\partial x_i}\left( \frac{\partial u_j}{\partial x_j} \right) = 	\frac{\partial}{\partial x_i}(\nabla \cdot \vec{u}).
\end{equation*}
Somit:
\begin{equation*}
	2\mu \frac{\partial \varepsilon_{ij}}{\partial x_j} = 
	\mu \nabla^2 u_i + \mu 	\frac{\partial}{\partial x_i}(\nabla \cdot \vec{u}).
\end{equation*}
Zusammen ergibt sich schliesslich:
\begin{equation*}
	\mu \nabla^2 u_i + (\lambda+\mu) \frac{\partial}{\partial x_i}(\nabla \cdot \vec{u}) + f_i = 
	0.
\end{equation*}
In kompakter Vektorschreibweise lautet die Gleichung:
\begin{equation*}
	\mu \nabla^2 \vec{u} + (\lambda+\mu) \nabla (\nabla \cdot \vec{u}) + \vec{f} = 
	0.
\end{equation*}
Dies sind die Navier-Cauchy-Gleichungen.
\end{proof}

\subsection{Variationsformulierung der Elastizität}
Zusätzlich kann das Problem über ein Energieprinzip formuliert werden:

\begin{description}
	\item[\textbf{Potenzielle Energie:}]
	\begin{equation*}
		\Pi = \int_V \left( \frac{1}{2} \sigma_{ij} \varepsilon_{ij} - f_i u_i \right) \, dV
	\end{equation*}
	
	\item[\textbf{Interpretation:}] 
	Das System minimiert seine potentielle Energie im Gleichgewicht. 
	Dies bedeutet, dass kleine virtuelle Änderungen der
	Verschiebungen $u_i$ die Gesamtenergie nicht verringern
	\index{Gesamtenergie}%
	können.
	
	\item[\textbf{Bemerkung:}] 
	Durch die Bedingung $\delta \Pi = 0$ (erste Variation
	verschwindet) erhalten wir wieder die Navier-Cauchy-Gleichungen
	(Euler-Lagrange-Prinzip).
	\index{Euler-Lagrange-Prinzip}%
	
	\item[\textbf{Mathematische Herleitung:}] 
	Die potentielle Gesamtenergie $\Pi$ setzt sich zusammen aus
	der im Körper gespeicherten elastischen Deformationsenergie
	\index{Deformationsenergie}%
	$U$ und der von den äusseren Kräften geleisteten potentiellen
	Arbeit $W$:
	\begin{equation*}
		\Pi = U - W
	\end{equation*}
	mit
	\begin{align*}
		U &= \frac{1}{2} \int_V \sigma_{ij} \, \varepsilon_{ij} \, dV, \\
		W &= \int_V f_i \, u_i \, dV + \int_{\partial V} t_i \, u_i \, dS.
	\end{align*}
	Hierbei beschreibt $U$ die im Material gespeicherte Energie
	infolge elastischer Verformung, während $W$ die Arbeit der
	Volumenkräfte $f_i$ und der Oberflächenkräfte $t_i$ darstellt.
	
	\item[\textbf{Stationaritätsbedingung:}]
	Das Prinzip der minimalen potentiellen Energie besagt, dass
	sich die Gleichgewichtslösung dadurch auszeichnet, dass
	$\Pi$ stationär ist:
	\begin{equation*}
		\delta \Pi = 0
	\end{equation*}
	für alle zulässigen virtuellen Verschiebungen $\delta u_i$.
	
	\item[\textbf{Schrittweise Herleitung der schwachen Form:}]
	\begin{enumerate}
		\item Variation der inneren Energie:
		\begin{equation*}
			\delta U = 
			\int_V \sigma_{ij} \, \delta \varepsilon_{ij} \, dV
		\end{equation*}
		mit
		\begin{equation*}
			\delta \varepsilon_{ij} = 
			\frac{1}{2} \left( \delta u_{i,j} + \delta u_{j,i} \right).
		\end{equation*}
		Da $\sigma_{ij} = \sigma_{ji}$ gilt, vereinfacht sich der Ausdruck zu
		\begin{equation*}
			\delta U = 
			\int_V \sigma_{ij} \, \delta u_{i,j} \, dV.
		\end{equation*}
		
		\item Variation der äusseren Arbeit:
		\begin{equation*}
			\delta W = 
			\int_V f_i \, \delta u_i \, dV + \int_{\Gamma_t} t_i \, \delta u_i \, dS.
		\end{equation*}
		
		\item Einsetzen in $\delta\Pi = \delta U - \delta W = 0$:
		\begin{equation}
			\int_V \sigma_{ij} \, \delta u_{i,j} \, dV - \int_V f_i \, \delta u_i \, dV - \int_{\Gamma_t} t_i \, \delta u_i \, dS = 
			0.
			\label{eq:pvw}
		\end{equation}
		
		\item Anwendung der partiellen Integration
		(Divergenzsatz) auf den ersten Term:
		\index{Divergenzsatz}%
		\begin{equation*}
			\int_V \sigma_{ij} \, \delta u_{i,j} \, dV 
			= -\int_V \sigma_{ij,j} \, \delta u_i \, dV + \int_{\Gamma} \sigma_{ij} n_j \, \delta u_i \, dS.
		\end{equation*}
		Auf $\Gamma_t$ gilt $\sigma_{ij} n_j = t_i$, auf $\Gamma_u$ gilt $\delta u_i = 0$. 
		Damit reduziert sich \eqref{eq:pvw} auf
		\begin{equation*}
			\int_V \left( -\sigma_{ij,j} - f_i \right) \delta u_i \, dV = 
			0.
		\end{equation*}
		
		\item Da $\delta u_i$ im Volumen beliebig ist, folgt
		die starke Form
		\begin{equation*}
			\sigma_{ij,j} + f_i = 
			0 \quad \text{in } V
		\end{equation*}
		mit den Randbedingungen
		\begin{equation*}
			u_i = \bar u_i \ \text{auf } \Gamma_u, \quad \sigma_{ij} n_j =
			t_i \ \text{auf } \Gamma_t.
		\end{equation*}
	\end{enumerate}
\end{description}

\subsection{Energieerhaltung und Dissipation}
Die zuvor hergeleitete Variationsformulierung basiert auf der
Annahme, dass die im System gespeicherte Energie und die von den
äusseren Kräften geleistete Arbeit im Gleichgewicht zueinander
stehen.

Es ist daher sinnvoll, die physikalische Bedeutung dieser
Energiebetrachtung zu verdeutlichen und zu zeigen, wie sich
Energieerhaltung und eventuelle Dissipation in diesem Rahmen
\index{Energieerhaltung}%
\index{Dissipation}%
darstellen.

Für einen rein elastischen Körper ohne Dämpfungsmechanismen gilt:
\index{Dampfungsmechanismus@Dämpfungsmechanismus}%
\begin{equation*}
	\frac{d}{dt} \int_V \left( \frac{1}{2} \rho \dot{u}_i \dot{u}_i + \frac{1}{2} \sigma_{ij} \varepsilon_{ij} \right) \, dV = 
	\int_V f_i \dot{u}_i \, dV + \int_{\partial V} t_i \dot{u}_i \, dS.
\end{equation*}
Dies bedeutet, dass die von den äusseren Kräften zugeführte Arbeit
vollständig als kinetische oder elastische Deformationsenergie
gespeichert wird.
Es tritt keine Dissipation auf ($D=0$). 
Bei Anwesenheit von plastischen oder viskoelastischen Effekten würde
\index{viskoelastisch}%
ein Teil der Energie als Wärme dissipiert, was in dieser Herleitung
jedoch nicht betrachtet wird.

\subsection{Anwendungsbeispiel: Ebener Verzerrungszustand}
Die allgemeinen Gleichungen der linearen Elastizitätstheorie und
ihre Variationsformulierung gelten für dreidimensionale Probleme.
In vielen praktischen Anwendungen, beispielsweise bei dünnen Platten
\index{Platte}%
oder Bauteilen mit grosser Ausdehnung in zwei Richtungen, kann
jedoch eine Vereinfachung vorgenommen werden.
Ausgehend von den zuvor hergeleiteten allgemeinen Formeln ergibt
sich durch die Annahmen  \(\varepsilon_{33} = \gamma_{13} = \gamma_{23}
= 0\) der ebene Verzerrungszustand.

Hierbei bedeutet:

$\varepsilon_{33}$: Dehnung senkrecht zur betrachteten Ebene (in $x_3$-Richtung),

$\gamma_{13}$: Schubverzerrung in der $x_1$-$x_3$-Ebene (Verkippung dieser Ebene),

$\gamma_{23}$: Schubverzerrung in der $x_2$-$x_3$-Ebene (Verkippung dieser Ebene).

Diese Grössen werden zu null gesetzt, da im ebenen Verzerrungszustand
keine Formänderung in $x_3$-Richtung und keine Schubverformungen
senkrecht zur betrachteten Ebene auftreten.

Die verbleibenden Spannungs-Dehnungs-Beziehungen lauten:
\begin{align*}
	\sigma_{11} &=	\frac{E (1-\nu)}{(1+\nu)(1-2\nu)} \varepsilon_{11} + \frac{E \nu}{(1+\nu)(1-2\nu)} \varepsilon_{22}
	\\
	\sigma_{22} &=	\frac{E \nu}{(1+\nu)(1-2\nu)} \varepsilon_{11} + \frac{E (1-\nu)}{(1+\nu)(1-2\nu)} \varepsilon_{22}
	\\
	\sigma_{12} &=	\frac{E}{2(1+\nu)} \gamma_{12}
\end{align*}
Dieses vereinfachte Modell lässt sich direkt aus den allgemeinen
Materialgleichungen ableiten und zeigt, wie sich die theoretische
Herleitung auf konkrete technische Anwendungsfälle übertragen lässt.
