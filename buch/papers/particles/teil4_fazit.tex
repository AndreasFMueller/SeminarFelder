%
% teil3.tex -- Beispiel-File für Teil 3
%
% (c) 2020 Prof Dr Andreas Müller, Hochschule Rapperswil
%
% !TEX root = ../../buch.tex
% !TEX encoding = UTF-8
%
\section{Fazit\label{particles:section:fazit}}
\kopfrechts{Fazit}

Anhand der numerischen Simulation konnte gezeigt werden, dass nichtlineare Medien bei hohen Feldstärken bedeutende Einflüsse auf das Verhalten von Feldern haben können.
So ist es möglich, Energie örtlich zu fangen, was in linearen Feldern nicht möglich ist.

Es ist natürlich nun verlockend die Behauptung zu stellen, man könne damit die Wellen-Teilchen-Dualität erklären.
Phänomene wie deren ``verdampfen'' decken sich jedoch eher mit dem Verhalten von schwarzen Löchern als mit dem Verhalten von Teilchen, welche eine konstante Masse und so im Stillstand eine konstante Energie aufweisen.

Zudem sollte einem bewusst sein, dass es sich bei solchen Teilchen- und Wellendarstellungen von Materie lediglich um Modelle handelt, die von uns Menschen aufgestellt wurden.
Diese Modelle sind, als menschliche Erfindung, grösstenteils einleuchtend und beschreiben die Realität unter gewissen Bedingungen hervorragend.
Es kann aber durchaus der Fall sein, dass diese Modelle die Realität zwar beschreiben, jedoch mit der tatsächlichen Beschaffenheit der Materie nur begrenzt übereinstimmen.
Schlussendlich scheint es etwas eingebildet, wenn wir uns die Fähigkeit zuschreiben, in unseren weiterentwickelten Echsengehirnen den grundlegenden Aufbau des Universums vollständig verstehen zu können.

Dennoch ist es von Vorteil, sich der Existenz solcher Effekte bewusst zu sein, sollte zum Beispiel ein Resultat einer Simulation oder eine analytische Lösung nicht mit der Realität übereinstimmen.
