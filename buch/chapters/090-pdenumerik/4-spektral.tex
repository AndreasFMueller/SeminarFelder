%
% Spektrale Methoden
%
\section{Spektrale Methoden
\label{buch:pdenumerik:section:spektral}}
\kopfrechts{Spektrale Methoden}%
Die Transformationsmethode zur Lösung partieller Differentialgleichungen
verwendet die Tatsache, dass Lösungsfunktionen immer als Reihe von
Eigenfunktionen des Laplace-Operators entwickelt werden können.
Zum Beispiel sind die Funktionen
\(
e_k(x) = e^{ikx}
\)
Eigenfunktionen des Ableitungsoperators auf dem Intervall $[0,2\pi]$,
denn es gilt
\[
\frac{d^2}{dx^2} e_k(x)
=
\frac{d^2}{dx^2} e^{ikx}
=
-k^2 e^{ikx}
=
-k^2 e_k(x).
\]
Der Eigenwert ist also $-k^2$.
Die Fourier-Theorie besagt, dass nicht allzu irreguläre periodische
Funktionen $f(x)$ als Reihe
\[
f(x)
=
\sum_{k\in\mathbb{Z}} c_ke^{ikx}
\]
geschrieben werden können.

Funktionen $f(\vartheta,\varphi)$ auf einer Kugeloberfläche können
ganz ähnlich als Funktionenreihe der sogenannten Kugelfunktionen
$Y^m_l(\vartheta,\varphi)$ geschrieben werden, die Eigenfunktionen
des Laplace-Operators sind:
\[
\Delta Y^m_l
=
-l(l+1) Y^m_l.
\]
Die ganzzahligen Werte von $l$ und $m$ erfüllen $|m|\le l$ und $l\ge 0$.

Als Beispiel für die Anwendung dieser Eigenschaften, nehmen wir an,
dass $f_n$ eine vollständige Familie von Eigenfunktionen des Laplace-Operators
mit Eigenwert $\lambda_n$ auf dem Gebiet $G$ ist. 
Jede Funktion $f$ auf dem Gebiet lässt sich also schreiben als
\begin{equation}
f = \sum_{k=1}^\infty a_k f_k.
\label{buch:pdenumerik:spektral:ansatz}
\end{equation}
Damit kann jetzt z.~B.~die Wärmeleitungsgleichung
\[
\frac{\partial f}{\partial t}
=
\kappa
\Delta f
\]
gelöst werden.
Setzt man den Ansatz~\eqref{buch:pdenumerik:spektral:ansatz} in die 
Differentialgleichung ein, ergibt sich
\[
\sum_{k=1}^\infty
\frac{d a_k}{d t}
f_k
=
\kappa \sum_{k=1}^n a_k \lambda_k f.
\]
Durch Koeffizientenvergleich für jedes $k$ folgt die gewöhnliche
Differentialgleichung
\[
\frac{d a_k}{d t}
=
\kappa \lambda_k a_k.
\]
Ihre Lösung ist $a_k(t) = a_k(0) e^{\lambda_kt}$.
Das schwierige Problem der Lösung der partiellen Differentialgleichung
wird durch den Ansatz~\eqref{buch:pdenumerik:spektral:ansatz} auf die
Lösung einer sehr einfachen gewöhnliche Differentialgleichung reduziert.

Die Idee lässt sich auch auf komplizierte, sogar nichtlineare
Feldgleichungen übertragen.
Es entstehen möglicherweise nichtlinear Differentialgleichungssysteme
für die Koeffizienten $a_k$.
In vielen Anwendungsfällen reicht eine vergleichsweise kleine Zahl
von Koeffizienten für eine gute Approximation der Lösung.
Diese sogenannten \emph{spektralen Methoden}
\index{spektrale Methode}%
werden zum Beispiel für die Berechnung der Strömungsfelder in der 
Erdatmosphäre vom Europäischen Zentrum für Mittelfristprognose
(ECMWF) verwendet.

Die Arbeit~\ref{chapter:fourier} von Martina Knobel und Gian Kraus
zeigt eine weitere Anwendung dieser Methoden.
Sie erklärt, wie spektrale Methoden eine Feldgleichung in eine Menge
unabhängiger quantenmechanischer Oszillatoren zerlegt, die als Quanten
eines Feldes betrachtet werden können.
In diesem Sinn sind die spektralen Methode die Grundlage für die
moderne Quantenfeldtheorie.

