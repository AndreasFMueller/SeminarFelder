%
% teil1.tex -- Beispiel-File für das Paper
%
% (c) 2020 Prof Dr Andreas Müller, Hochschule Rapperswil
%
% !TEX root = ../../buch.tex
% !TEX encoding = UTF-8
%
\section{Strömungssimulation
\label{openfoam:section:teil1}}
\kopfrechts{Strömungssimulation}
Zum Simulieren von Strömungen benötigen wir zum einen ein Modell das das verhalten des Mediums, im Fall einer Strömungssimulation eine Flüssigkeit, dass das Verhalten des Mediums möglichst umfassend und genau beschreibt. Hierbei werden meist die Navier Stokes Gleichungen benutzt welche noch mit der Kontinuitätsgleichung und Energiegleichung erweitert jedoch werden auch in einigen Fällen die Euler Gleichungen oder die Potentialgleichungen verwendet da diese zwar ein weniger vollständiges Modell beschreiben jedoch weniger Rechenintensiv sind.

Zudem benötigen wir ein Lösungsverfahren das Einen Anfangszustand und das Mathematische Modell zu einer Numerischen Lösung verarbeiten kann dies ist nötig da für die Navier Stokes Gleichungen keine Analytische Lösung existiert und die der Euler Gleichungen sehr aufwendig sind zu berechnen. Hierbei kommen meist die Finite-Differenzen-Methode (FDM), die Finite-Volumen-Methode (FVM)und die Finite-Elemente-Methode (FEM) zum Einsatz. 

OpenFoam nutzt dabei meist die Finite-Volume-Methode dies ist jedoch eines der Gebiete in dem sich OpenFOAM von anderen Simulations Programmen unterscheidet da es sich Besonders dafür eignet Neue Lösungsverfahren zu implementieren und zu testen.

\subsection{Modellbildung}
\subsubsection{Stationär oder Instationär?}
Bei der Strömungsmechanik wird zuerst unterschieden ob es sich um eine Stationäre oder Instationäre Strömung handelt. Das heißt es ist entweder der Fall dass an jedem Ort im Raum die Geschwindigkeit der sich dort befindenden Flüssigkeit in Bezug auf Betrag und Richtung konstant ist (Stationär) oder dass sich diese Größe ändert (Instationär).
Bei einer Simulation wird meist Simuliert bis der Stationäre zustand, bis auf einen kleinen Fehler, erreicht wird.
Dadurch kann man sich Simulationsaufwand ersparen da sobald der Stationäre zustand erreicht wird das Resultat nicht mehr ändert und so keine neuen Ergebnisse gefunden werden können.


