%
% packages.tex -- packages required by the paper parallelisierung
%
% (c) 2019 Prof Dr Andreas Müller, Hochschule Rapperswil
%

% if your paper needs special packages, add package commands as in the
% following example
%\usepackage{packagename}

\usepackage{listings}
\usepackage{xcolor}
\usepackage{pgfplots}
\pgfplotsset{compat=1.18}
\usepackage{pgfplotstable} % falls du Daten aus einer Tabelle einlesen willst

\lstset{
	language=C++,
	basicstyle=\ttfamily\footnotesize,
	keywordstyle=\color{blue}\bfseries,
	stringstyle=\color{red},
	commentstyle=\color{green!50!black}\itshape,
	numbers=left,
	numberstyle=\tiny\color{gray},
	stepnumber=1,
	showstringspaces=false,
	tabsize=4,
	breaklines=true%,
	%frame=none
}
\newfloat{lstfloat}{htbp}{}[chapter]
\floatname{lstfloat}{Listing\strut}
\def\lstfloatautorefname{Listing}
