%
% table-schwarzschildradius.tex
%
% (c) 2025 Prof Dr Andreas Müller
%
\begin{table}
\centering
\begin{tabular}{|l|>{$}r<{$}|>{$}r<{$}|>{$}r<{$}|}
\hline
Objekt & R\,\textrm[m]& M\,\textrm{[kg]}&r_s\raisebox{3pt}{\strut}\\[2pt]
\hline
Fussball
	& 0.011\phantom{\mathstrut\cdot 10^{22}}
	& 0.450\phantom{\mathstrut\cdot 10^{22}}
	& 6.684\cdot{10^{-28}}\rlap{$\,\mathrm{m}$}\hspace*{6mm}
\raisebox{5pt}{\mathstrut}
\\
Mond
	& 1.737\cdot 10^{6\phantom{0}}
	& 7.346\cdot 10^{22}
	& 0.011\rlap{$\,\mathrm{mm}$}\hspace*{6mm}
\\
Erde 
	& 6.378\cdot 10^{6\phantom{0}}
	& 5.972\cdot 10^{24}
	& 8.870\rlap{$\,\mathrm{mm}$}\hspace*{6mm}
\\
Jupiter
	& 1.429\cdot 10^{8\phantom{0}}
	& 1.898\cdot 10^{27}
	& 2.829\rlap{$\,\mathrm{m}$}\hspace*{6mm}
\\
Sonne
	& R_{\sun}=6.963\cdot 10^{8\phantom{0}}
	& M_{\sun}=1.988\cdot 10^{30}
	&2.952\rlap{$\,\mathrm{km}$}\hspace*{6mm}
\\
Polarstern
	& 46.27\,R_{\sun}
	& 5.13\,M_{\sun}
	& 15.1\phantom{00}\rlap{$\,\mathrm{km}$}\hspace*{6mm}
\\
VY Canis Maioris
	& 1420\phantom{.00}\,R_{\sun}
	& 17\,M_{\sun}
	& 40.2\phantom{00}\rlap{$\,\mathrm{km}$}\hspace*{6mm}
\\
PSR J1748-2446ad
	& 16\cdot10^{3\phantom{0}}
	& 2\phantom{.00}\,M_{\sun}
	&  6\phantom{.000}\rlap{$\,\mathrm{km}$}\hspace*{6mm}
\\
Stellares SL
	&
	& 10\phantom{.00}\,M_{\sun}
	&  29.52\phantom{0}\rlap{$\,\mathrm{km}$}\hspace*{6mm}
\\
Supermassives SL
	&
	& 10^5\text{--}10\rlap{$\mathstrut^{11}$}\phantom{.00}\,M_{\sun}
	&   0.002\text{--}2000\rlap{$\,\mathrm{AU}$}\hspace*{6mm}
\\[2pt]
\hline
\end{tabular}
\caption{Schwarzschild-Radien einer Objekte des Sonnensystems und
einiger schwarzer Löcher.
VY Canis Maioris ist der grösste bekannte Stern. PSR J1748-2446d ist
der Pulsar mit der schnellsten bekannten Rotation.
Stellare schwarze Löcher (SL) entstehen beim Kollaps eines Sterns
am Ende seines Lebens, während supermassive SL im Zentrum von
Galaxien zu finden sind.
Eine Astronomische Einheit (AU) ist die mittlere Entfernung zwischen
Sonne und Erde und beträgt $1.495978707\cdot 10^8\,\text{km}$.
\label{buch:kruemmung:schwarzesloch:table:schwarzschildradien}}
\end{table}
