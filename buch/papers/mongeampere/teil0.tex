%
% einleitung.tex -- Beispiel-File für die Einleitung
%
% (c) 2020 Prof Dr Andreas Müller, Hochschule Rapperswil
%
% !TEX root = ../../buch.tex
% !TEX encoding = UTF-8
%
\section{Die Differentialgleichung\label{mongeampere:section:teil0}}
\kopfrechts{Teil 0}
Die Monge-Ampèresche Gleichung
\begin{equation}
  \det D^2 u = f(x, u, \nabla u)
  \label{mongeampere:eq:mongeampere}
\end{equation}
wobei $D$ die Hessche Matrix
\begin{equation}
  D^2 u =
  \begin{pmatrix}
    \pdv[2]{u}{x_1} & \dots &\pdv{u}{x_1}{x_n} \\
  \vdots & \ddots & \vdots \\
  \pdv{u}{x_n}{x_1} & \dots &\pdv[2]{u}{x_n} \\
  \end{pmatrix}
  \label{mongeampere:eq:hess}
\end{equation}
ist.
Wenn man diese Forlumlierung ansieht, erkennt man, dass es sich bei der Monge-Ampèreschen gleichung um eine partielle
nicht-lineare Gleichung handelt.
Das lösen einer solchen Differnetialgelichung ist in generellen ein anspruchvolles Problem.
Anhand von Bedinugnen für $u$ und $f$ kann man \eqref{mongeampere:eq:mongeampere} 
in einen bestimmten Typ Deffernetialgleichung bringen.
Setzt man voraus, dass $u$ eine glatte konvexe Function ist, und $f$ positiv, dann ist \eqref{mongeampere:eq:mongeampere}
eine elliptische partielle Differnetialgleichung.
Gemäss \cite{figalli2018mongeampereequation} gibt eine solche Form hoffnung auf die existenz einer regulären Lösung.

