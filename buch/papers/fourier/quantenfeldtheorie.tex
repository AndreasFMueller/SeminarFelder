%
% quantenfeldtheorie.tex -- Quantenfeldtheorie:  Quantisierung, Schrödinger-Gleichung, der harmonische Oszillator, Photonen als Oszillatoren
%
% (c) 2020 Prof Dr Andreas Müller, Hochschule Rapperswil
%
% !TEX root = ../../buch.tex
% !TEX encoding = UTF-8
%
\section{Quantenfeldtheorie
\label{fourier:section:quantenfeldtheorie}}
\kopfrechts{Quantenfeldtheorie}
Zunächst erfolgt ein Exkurs in die Quantentheorie, um schliesslich zum Laser zu gelangen. % Todo überarbeiten; evtl. weglassen

...
Energie wird in Form von Quanten gespeichert.
Vielfache der Grundfrequenz sind möglich.

Elektromagnetisches Feld $\rightarrow$ Fourier $\rightarrow$ Feldgleichung wie Gleichung vom Federpendel

\subsection{Der harmonische Oszillator
\label{fourier:subsection:derHarmonischeOszillator}}
Warum genau schauen wir uns den harmonischen Oszillator an?
In der Quantenelektrodynamik werden die elektromagnetischen Felder als quantisierte harmonische Oszillatoren modelliert, deren Energie ebenfalls in Vielfachen von $\hbar\cdot\omega$ vorliegt.
Die Zustände des Feldes (Photonenzahlenzustände) werden durch die Schwingungszustände des harmonischen Oszillators beschrieben.

Diese Quantisierung erklärt, warum Licht in Lasern nicht kontinuierlich, sondern in diskreten Paketen (Photonen) emittiert wird. % todo: diesen Satz an anderer passenden Stelle einfügen

Die Wellengleichung lautet bekanntlich
\begin{equation}
    \frac{\partial^2 u}{\partial t^2} = c^2 \left( \frac{\partial^2 u}{\partial x^2} + \frac{\partial^2 u}{\partial y^2} \right).
\end{equation}
Wenn nun der y-Anteil als konstant betrachtet wird, sind alle partiellen Ableitungen nach y gleich Null.
Dies führt zu der vereinfachten Gleichung
\begin{equation}
    \frac{\partial^2 u}{\partial t^2} = c^2 \frac{\partial^2 u}{\partial x^2}.
\end{equation}
Daraus lässt sich die Differentialgleichung
\begin{equation}
    \ddot{a}(t) = -k^2 a(t)
\end{equation}
aufstellen.
$a(t)$ ist hierbei ein Fourier-Koeffizient des elektromagnetischen Wellenfeldes.
Die Lösung dieser Differentialgleichung lautet
\begin{equation}
    u(t,x) = a_k(t) \cos(kx)
\end{equation}

% Besser sin verwenden --> besser e^ikx verwenden; Überlegen, ob wir komplex arbeiten möchten
% In Präsentation mit cos und im Paper mit e^ikx


% Wichtiger Schritt: Der harmonische Oszillator in der Quantenmechanik --> Unterlagen anschauen. 
% Evtl. Termin mit ihm, um Unklarheiten zu klären.

Diese Gleichung ist analog zur Gleichung eines Federpendels.
Es handelt sich hier um ein "Quanten-Federpendel".
% \begin{tikzpicture}

%     % Zeichne die horizontale Linie (Boden)
%     \draw[thick] (-2,0) -- (2,0);
    
%     % Zeichne die Wand
%     \draw[thick] (0,0) -- (0,2);
    
%     % Zeichne die Feder
%     \draw[thick] (0,2) -- (0,4);
%     \draw[thick] (0,4) -- (0.5,4.5);
%     \draw[thick] (0.5,4.5) -- (0,5);
    
%     % Zeichne die Masse
%     \filldraw[fill=gray] (0,5) circle (0.2);
    
%     % Zeichne die Schnur
%     \draw[thick] (0,5) -- (0,6);
    
%     \end{tikzpicture} % unschönes bild, sollte ein Federpendel werden; todo
% todo: weiterschreiben
