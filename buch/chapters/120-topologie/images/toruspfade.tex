%
% toruspfade.tex -- Pfadintegrale in einem Torus
%
% (c) 2021 Prof Dr Andreas Müller, OST Ostschweizer Fachhochschule
%
\documentclass[tikz]{standalone}
\usepackage{amsmath}
\usepackage{times}
\usepackage{txfonts}
\usepackage{pgfplots}
\usepackage{csvsimple}
\usetikzlibrary{arrows,intersections,math,calc}
\definecolor{darkred}{rgb}{0.8,0,0}
\definecolor{darkgreen}{rgb}{0,0.6,0}
\begin{document}
\def\skala{1}
\def\h{3}
\begin{tikzpicture}[>=latex,thick,scale=\skala,
declare function = {
	fx(\T) = \T;
	fy(\T) = 0.5*(1+\T)*\T*\T-0.25*sin(720*\T);
	Fx(\T) = (2+0.8)*\h*fx(\T);
	Fy(\T) = (1+0.7)*\h*fy(\x);
}]

\fill[color=gray!10] (0,0) rectangle ++(\h,\h);

\foreach \x in {-1,...,3}{
	\draw[line width=0.3pt] ({\x*\h},{-\h-0.1}) -- ({\x*\h},{2*\h+0.3});
}
\foreach \y in {-1,...,2}{
	\draw[line width=0.3pt] ({-\h-0.1},{\y*\h}) -- ({3*\h+0.3},{\y*\h});
}
\draw[->] ({-\h-0.1},0) -- ({3*\h+0.3},0) coordinate[label={$x$}];
\draw[->] (0,{-\h-0.1}) -- (0,{2*\h+0.3}) coordinate[label={right:$y$}];

%\draw[color=darkred,line width=1.4pt] (0,0) -- (\h,0);
%\draw[color=blue,line width=1.4pt] (0,0) -- (0,\h);

\coordinate (B) at ({0.8*\h},{0.7*\h});

\draw[color=orange,line width=1.4pt] (0,0) -- (B);
%\draw[color=darkgreen,line width=1.4pt] (0,0) -- ($(B)+({2*\h},\h)$);

\foreach \s in {0.1,0.2,...,0.9}{
	\draw[color=darkgreen,line width=0.3pt]
		plot[domain=0:0.5,samples=100]
		({\s*Fx(\x)+(1-\s)*(4*\h*\x)},{\s*Fy(\x)})
		--
		plot[domain=0.5:0.75,samples=100]
		({\s*Fx(\x)+(1-\s)*(2*\h))},{\s*Fy(\x)+(1-\s)*(4*\h*(\x-0.5))})
		--
		plot[domain=0.75:1,samples=100]
	({\s*Fx(\x)+(1-\s)*(2*\h+4*(\x-0.75)*0.8*\h)},{\s*Fy(\x)+(1-\s)*(\h+4*(\x-0.75)*0.7*\h)});
}
\draw[color=darkred,line width=1.4pt] (0,0) -- ++({2*\h},0);
\draw[color=blue,line width=1.4pt] ({2*\h},0) -- ++(0,\h);
\draw[color=orange,line width=1.4pt] ({2*\h},\h) -- ($(B)+({2*\h},\h)$);

\draw[color=darkgreen,line width=1.4pt] plot[domain=0:1,smooth]
	({(2+0.8)*\h*fx(\x)},{(1+0.7)*\h*fy(\x)});

\foreach \x in {-1,...,2}{
	\foreach \y in {-1,...,1}{
		\fill[color=gray!50] ($(B)+({\x*\h},{\y*\h})$)
			circle[radius=0.07];
		\node[color=gray!50] at ($(B)+({\x*\h},{\y*\h})$)
			[above right] {$B$};
	}
}

\fill (B) circle[radius=0.07];
\node at (B) [above right] {$B$};

\fill (0,0) circle[radius=0.07];
\node at (0,0) [above left] {$A$};

\end{tikzpicture}
\end{document}

