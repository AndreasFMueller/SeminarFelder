\section{Homogene Maxwellgleichungen}
Die homogenen Maxwellgleichungen bekommen wir durch die äussere Ableitung des Feldstärketensors, den wir in (Abschnitt vom Herr widmer, ref) durch $F = dA$ erhalten haben
\[
F_{\mu\nu}
= 
\begin{pmatrix}
	0 & -E_1/c & -E_2/c & -E_3/c \\
	E_1/c &  0 &  B_3 & -B_2 \\
	E_2/c & -B_3 &  0 &  B_1 \\
	E_3/c &  B_2 & -B_1 &  0 \\
\end{pmatrix}.
\]
Wir wissen, aus (kap 8.3.4 Operatorrelationen, kein label vorhanden), das
\[
ddA = dF = 0
\]
gelten muss.
Für die Berechnung der äusseren Ableitung muss der Tensor zuerst als Differentialform geschrieben werden.
Es resultiert die Zweiform
\[
F
= \frac{1}{2} F_{\mu\nu} \, dx^\mu \wedge dx^\nu,
\]
man beachte dabei die einsteinsche Summationskonvention(ref Buch?).
Da der Feldstärketensor antisymmetrisch $F_{\mu\nu} = -F_{\nu\mu}$ ist, benötigt man den Faktor $\frac{1}{2}$ um die doppelte Zählung der Komponenten zu vermeiden:
\begin{align*}
	F
	= 
	\phantom{+} &\frac{1}{2} F_{\mu\nu} \, dx^\mu \wedge dx^\nu
	\\
	=
	\phantom{+} &\frac{1}{2} F_{00} \, dx^0 \wedge dx^0 + \frac{1}{2} F_{01} \, dx^0 \wedge dx^1 + \frac{1}{2} F_{02} \, dx^0 \wedge dx^2 + \frac{1}{2} F_{03} \, dx^0 \wedge dx^3
	\\
	+ &\frac{1}{2} F_{10} \, dx^1 \wedge dx^0 + \frac{1}{2} F_{11} \, dx^1 \wedge dx^1 + \frac{1}{2} F_{12} \, dx^1 \wedge dx^2 + \frac{1}{2} F_{13} \, dx^1 \wedge dx^3
	\\
	+ &\frac{1}{2} F_{20} \, dx^2 \wedge dx^0 + \frac{1}{2} F_{21} \, dx^2 \wedge dx^1 + \frac{1}{2} F_{22} \, dx^2 \wedge dx^2 + \frac{1}{2} F_{23} \, dx^2 \wedge dx^3
	\\
	+ &\frac{1}{2} F_{30} \, dx^3 \wedge dx^0 + \frac{1}{2} F_{31} \, dx^3 \wedge dx^1 + \frac{1}{2} F_{32} \, dx^3 \wedge dx^2 + \frac{1}{2} F_{33} \, dx^3 \wedge dx^3
\end{align*} 
%-------------------------------------------ab da mit dx^îjk arbeiten !!-------------------------------------------------------------------
Mit der Definition (ref zu $dx^{ik}$ schreibweise def undso) gilt $dx^{ik} = 0$ wenn $i=k$ und $dx^{ik} = -dx^{ki}$ wenn $i \neq k $ . Somit folgt nach aufsteigenden Indizes sortiert
\begin{align*}
	F
	=
	&\phantom{+ \frac{1}{2} F_{00} \, dx^{00}} + \frac{1}{2} F_{01} \, dx^{01} + \frac{1}{2} F_{02} \, dx^{02} + \frac{1}{2} F_{03} \, dx^{03}
	\\
	&- \frac{1}{2} F_{10} \, dx^{01} \phantom{+ \frac{1}{2} F_{11} \, dx^{11}} + \frac{1}{2} F_{12} \, dx^{12} + \frac{1}{2} F_{13} \, dx^{13}
	\\
	&- \frac{1}{2} F_{20} \, dx^{02} - \frac{1}{2} F_{21} \, dx^{12} \phantom{+ \frac{1}{2} F_{22} \, dx^{22}} + \frac{1}{2} F_{23} \, dx^{23}
	\\
	&- \frac{1}{2} F_{30} \, dx^{03} - \frac{1}{2} F_{31} \, dx^{13} - \frac{1}{2} F_{32} \, dx^{23} \phantom{+ \frac{1}{2} F_{33} \, dx^{33}}
\end{align*}
und wegen $F_{\mu\nu} = -F_{\nu\mu}$
\begin{align*}
	F
	=
	&\phantom{+ \frac{1}{2} F_{00} \, dx^{00}} + \frac{1}{2} F_{01} \, dx^{01} + \frac{1}{2} F_{02} \, dx^{02} + \frac{1}{2} F_{03} \, dx^{03}
	\\
	&+ \frac{1}{2} F_{01} \, dx^{01} \phantom{+ \frac{1}{2} F_{11} \, dx^{11}} + \frac{1}{2} F_{12} \, dx^{12} + \frac{1}{2} F_{13} \, dx^{13}
	\\
	&+ \frac{1}{2} F_{02} \, dx^{02} + \frac{1}{2} F_{12} \, dx^{12} \phantom{+ \frac{1}{2} F_{22} \, dx^{22}} + \frac{1}{2} F_{23} \, dx^{23}
	\\
	&+ \frac{1}{2} F_{03} \, dx^{03} + \frac{1}{2} F_{13} \, dx^{13} + \frac{1}{2} F_{21} \, dx^{23} \phantom{+ \frac{1}{2} F_{33} \, dx^{33}}
\end{align*}
folgt anschliessend
\begin{align*}
	F
	=
	\phantom{+} F_{01} \, dx^{01} + F_{02} \, dx^{02} &+ F_{03} \, dx^{03}
	\\
	\phantom{+ F_{11} \, dx^{11}} + F_{12} \, dx^{12} &+ F_{13} \, dx^{13}
	\\
	\phantom{+ F_{12} \, dx^{12}} \phantom{+ F_{22} \, dx^{22}} &+ F_{23} \, dx^{23} \, .
\end{align*}
Für besseres Verständnis setzen wir nun konkret die Komponenten von $F_{\mu\nu}$ ein und erhalten
\begin{align*}
	F
	=
	- \frac{1}{c} E_1 \, dx^{01} - \frac{1}{c} E_2 \, dx^{02} &- \frac{1}{c} E_3 \, dx^{03}
	\\
	\phantom{+ \frac{1}{c} F_{11} \, dx^{11}} + \phantom{\frac{1}{c}} B_3 \, dx^{12} &- \phantom{\frac{1}{c}} B_2 \, dx^{13}
	\\
	\phantom{+ \frac{1}{c} F_{12} \, dx^{12}} \phantom{+ \frac{1}{c} F_{22} \, dx^{22}} &+ \phantom{\frac{1}{c}} B_1 \, dx^{23} \, ,
\end{align*}
was schön den Komponenten oberhalb der Diagonalen von $F_{\mu\nu}$ entspricht.
Nun wenden wir die äussere Ableitung auf die 2-Form $F$ an, man beachte dabei Definition(index wedges)$dx^i \wedge dx^j \wedge dx^k = dx^{ijk}$
 \begin{align*}
 	dF =
 	&\cancel{\, - \, \frac{\d E_1}{\d x^0} \, dx^{001}} \cancel{\, - \, \frac{\d E_1}{\d x^1} \, dx^{101}}
 	 - \frac{\d E_1}{\d x^2} \, dx^{201} - \frac{\d E_1}{\d x^3} \, dx^{301}
 	\\
 	&\cancel{\, - \, \frac{\d E_2}{\d x^0} \, dx^{002}} - \frac{\d E_2}{\d x^1} \, dx^{102}
 	 \cancel{\, - \, \frac{\d E_2}{\d x^2} \, dx^{202}} - \frac{\d E_2}{\d x^3} \, dx^{302}
 	\\
 	&\cancel{\, - \, \frac{\d E_3}{\d x^0} \, dx^{003}} - \frac{\d E_3}{\d x^1} \, dx^{103}
 	 - \frac{\d E_3}{\d x^2} \, dx^{203} \cancel{\, - \, \frac{\d E_3}{\d x^3} \, dx^{303}}
 	\\
 	&+ \frac{\d B_3}{\d x^0} \, dx^{012} \cancel{\, + \, \frac{\d B_3}{\d x^1} \, dx^{112}}
 	 \cancel{\, + \, \frac{\d B_3}{\d x^2} \, dx^{212}} + \frac{\d B_3}{\d x^3} \, dx^{312}
 	\\
 	&- \frac{\d B_2}{\d x^0} \, dx^{013} \cancel{\, - \, \frac{\d B_2}{\d x^1} \, dx^{113}}
 	 - \frac{\d B_2}{\d x^2} \, dx^{213} \cancel{\, - \, \frac{\d B_2}{\d x^3} \, dx^{313}}
 	\\
 	&+ \frac{\d B_1}{\d x^0} \, dx^{023} + \frac{\d B_1}{\d x^1} \, dx^{123}
 	 \cancel{\, + \, \frac{\d B_1}{\d x^2} \, dx^{223}} \cancel{\, + \, \frac{\d B_1}{\d x^3} \, dx^{323}} %\,
 	\\
 	%\text{die gestrichenen Komponenten sind $=0$, da $dx^{ii}=0}$ gilt:}
 	\\
 	=
 	&- \frac{\d E_1}{\d x^2} \, dx^{201} - \frac{\d E_1}{\d x^3} \, dx^{301}
 	 - \frac{\d E_2}{\d x^1} \, dx^{102} - \frac{\d E_2}{\d x^3} \, dx^{302}
 	\\
 	&- \frac{\d E_3}{\d x^1} \, dx^{103} - \frac{\d E_3}{\d x^2} \, dx^{203}
 	 + \frac{\d B_3}{\d x^0} \, dx^{012} + \frac{\d B_3}{\d x^3} \, dx^{312}
 	\\
 	&- \frac{\d B_2}{\d x^0} \, dx^{013} - \frac{\d B_2}{\d x^2} \, dx^{213}
 	 + \frac{\d B_1}{\d x^0} \, dx^{023} + \frac{\d B_1}{\d x^1} \, dx^{123} \, , 
 \end{align*}
sortieren nach aufsteigenden Indizes
 \begin{align*}
	dF =
	&- \frac{\d E_1}{\d x^2} \, dx^{012} - \frac{\d E_1}{\d x^3} \, dx^{013}
	 + \frac{\d E_2}{\d x^1} \, dx^{012} - \frac{\d E_2}{\d x^3} \, dx^{023}
	\\
	&+ \frac{\d E_3}{\d x^1} \, dx^{013} + \frac{\d E_3}{\d x^2} \, dx^{023}
	 + \frac{\d B_3}{\d x^0} \, dx^{012} + \frac{\d B_3}{\d x^3} \, dx^{123}
	\\
	&- \frac{\d B_2}{\d x^0} \, dx^{013} + \frac{\d B_2}{\d x^2} \, dx^{123}
	 + \frac{\d B_1}{\d x^0} \, dx^{023} + \frac{\d B_1}{\d x^1} \, dx^{123}
\end{align*}
und fassen anschliessend alles zusammen
\begin{align*}
	dF =
	&\left( - \frac{\d E_1}{\d x^2} + \frac{\d E_2}{\d x^1} + \frac{\d B_3}{\d x^0} \right) dx^{012}
	\\
	+&\left( - \frac{\d E_1}{\d x^3} + \frac{\d E_3}{\d x^1} - \frac{\d B_2}{\d x^0} \right) dx^{013}
	\\
	+&\left( - \frac{\d E_2}{\d x^3} + \frac{\d E_3}{\d x^2} + \frac{\d B_1}{\d x^0} \right) dx^{023}
	\\
	+&\left( + \frac{\d B_3}{\d x^3} + \frac{\d B_2}{\d x^2} + \frac{\d B_1}{\d x^1} \right) dx^{123} \, .
\end{align*}
 Um nun besser zu verstehen, was das nun bedeutet, ändern wir die Vorzeichen innerhalb der Klammer in der zweiten Zeile, indem wir die Reihenfolge der Wedgeprodukte verändern.
\begin{align*}
	dF =
	&\left( - \frac{\d E_1}{\d x^2} + \frac{\d E_2}{\d x^1} + \frac{\d B_3}{\d x^0} \right) dx^0 \wedge dx^1 \wedge dx^2
	\\
	+&\left( + \frac{\d E_1}{\d x^3} - \frac{\d E_3}{\d x^1} + \frac{\d B_2}{\d x^0} \right) dx^0 \wedge dx^3 \wedge dx^1
	\\
	+&\left( - \frac{\d E_2}{\d x^3} + \frac{\d E_3}{\d x^2} + \frac{\d B_1}{\d x^0} \right) dx^0 \wedge dx^2 \wedge dx^3
	\\
	+&\left( + \frac{\d B_3}{\d x^3} + \frac{\d B_2}{\d x^2} + \frac{\d B_1}{\d x^1} \right) dx^1 \wedge dx^2 \wedge dx^3 \, .
\end{align*}
und erkennen nun, wenn $dx^0 = dt, dx^1 = dx, dx^2 = dy, dx^3 = dz$:
\begin{align*}
	dF =
	&\Bigg( \underbrace{ - \frac{\d E_x}{\d y} + \frac{\d E_y}{\d x}}_{(\nabla \times E)_z} + \frac{\d B_z}{\d t} \Bigg) dt \wedge dx \wedge dy
	\\
	+&\Bigg( \underbrace{ + \frac{\d E_x}{\d z} - \frac{\d E_z}{\d x}}_{(\nabla \times E)_y} + \frac{\d B_y}{\d t} \Bigg) dt \wedge dz \wedge dx
	\\
	+&\Bigg( \underbrace{ - \frac{\d E_y}{\d z} + \frac{\d E_z}{\d y}}_{(\nabla \times E)_x} + \frac{\d B_x}{\d t} \Bigg) dt \wedge dy \wedge dz
	\\
	+&\Bigg( \underbrace{ + \frac{\d B_z}{\d z} + \frac{\d B_y}{\d y} + \frac{\d B_x}{\d x}}_{\nabla \cdot \vec{B}} \Bigg) dx \wedge dy \wedge dz \, .
\end{align*}
Das sind genau die homogenen Maxwellgleichungen
\[
\nabla \times \vec{E} + \frac{\d \vec{B}}{\d t} = 0
\]
und
\[
\nabla \cdot \vec{B} = 0 \, .
\]
