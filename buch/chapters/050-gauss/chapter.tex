%
% chapter.tex -- Kapitel 5: Der Satz von Gauss
%
% (c) 2024 Prof Dr Andreas Müller
%
\chapter{Der Satz von Gauss
\label{chapter:gauss}}
\kopflinks{Der Satz von Gauss}
In den vorangegangenen Kapiteln wurden $p$-Formen betrachtet mit
$p\in\{0,1,2\}$, es wurde ein Ableitungsoperator $d$ definiert
und die Integration über $p$-dimensionale Untermannigfaltigkeiten $M$
und über den Rand $\partial M$.
Die Sätze von Green und Stokes haben gezeigt, dass den Zusammenhang
\begin{equation}
\int_M d\omega = \int_{\partial M} \omega
\label{buch:gauss:eqn:stokes}
\end{equation}
zwischen dem Integral der Ableitung einer $p-1$-Form über eine
$p$-dimensionale Mannigfaltigkeit und dem Integral der ursprünglichen
Form über den $p-1$-dimensionalen Rand gibt.
Der nächste Schritt ist daher, einen Zusammenhang zwischen dem Volumenintegral
einer 3-Form über einen Körper und einem Oberflächenintegral
einer 2-Form über die geschlossen Fläche, die den Rand des Körpers bildet,
herstellt.
Dieser Zusammenhang ist bekannt als der Satz von Gauss und der
Differentialoperator ist die Divergenz.

Die Divergenz der Strömung eines Fluids beschreibt, in welchem Ausmass
in einem gegebenen Raumvolumen neues Fluid entsteht.
Eine $n-1$-Form beschreibt dann den Fluss des Fluids durch die Oberfläche.
Die Erhaltung der Menge des Fluids muss sich also als Gleichung
von Integralen wie in \eqref{buch:gauss:eqn:stokes} oder als
Differentialgleichung mit der Divergenz schreiben lassen.
Solche Feldgleichungen gehören zu den wichtigsten und in fast allen
Anwendungen der Feldtheorie auftretenden Gleichungen.
Sie werden in Abschnitt~\ref{buch:gauss:section:erhaltungssatz}
genauer untersucht.

%
% n-1-Formen und n-Formen
%
\section{$n-1$-Formen und $n$-Formen
\label{buch:gauss:section:1formenundnformen}}
\kopfrechts{$n-1$-Formen und $n$-Formen}

\subsection{Die äussere Ableitung}

%
% Der Satz von Gauss
%
\section{Der Satz von Gauss
\label{buch:gauss:section:satzvongauss}}
\kopfrechts{Der Satz von Gauss}

\subsection{Integration von $n$-Formen}

\subsection{Integration über den Rand}

\subsection{Der Satz von Gauss}

%
% Erhaltungssätze
%
\section{Erhaltungssätze
\label{buch:gauss:section:erhaltungssatz}}
\kopfrechts{Erhaltungssätze}

\subsection{Kontinuitätsgleichung}

\subsection{Wärmeleitung}

