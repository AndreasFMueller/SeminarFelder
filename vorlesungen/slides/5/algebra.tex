%
% algebra.tex -- slide template
%
% (c) 2021 Prof Dr Andreas Müller, OST Ostschweizer Fachhochschule
%
\bgroup
\begin{frame}[t]
\setlength{\abovedisplayskip}{5pt}
\setlength{\belowdisplayskip}{5pt}
\frametitle{Äussere Algebra und Ableitung}
\vspace{-20pt}
\begin{columns}[t,onlytextwidth]
\begin{column}{0.48\textwidth}
\begin{block}{Äussere Algebra mit $\wedge$-Produkt}
\begin{itemize}
\item $\alpha\in\Omega^p(M)$, $\beta\in\Omega^q(M)$
\begin{align*}
\uncover<2->{
\alpha\wedge \beta &= (-1)^{pq}\beta\wedge\alpha
}
\\
\uncover<3->{
(\alpha_1+\alpha_2)\wedge\beta &= \alpha_1\wedge \beta + \alpha_2\wedge\beta
}
\\
\uncover<4->{
\alpha\wedge(\beta_1+\beta_2) &= \alpha\wedge\beta_1 + \alpha\wedge\beta_2
}
\end{align*}
\item<5->
Basis der $p$-Formen:
$dx^{i_1}\wedge\dots\wedge dx^{i_p}$,
$i_1<\dots <i_p$
\item<6->
Komponenten $a_{i_1\dots i_p}$ (kovariant)
\item<7-> $p$-Formen auf $M$:
\begin{align}
\alpha
&=
\sum_{i_1<\dots<i_p}
a_{i_1\dots i_p} dx^{i_1}\wedge\dots\wedge dx^{i_p}
\label{slide:algebra:1}
\\[-10pt]
&\in \Omega^p (M)
\notag
\end{align}
\end{itemize}
\end{block}
\end{column}
\begin{column}{0.48\textwidth}
\uncover<8->{%
\begin{block}{Äussere Ableitung}
$\alpha\in\Omega^p(M)$ wie in \eqref{slide:algebra:1}
hat die äussere Ableitung
\[
d\alpha
=
\sum_{i=1}^n
\sum_{i_1<\dots<i_p}
\hspace*{-3pt}
\frac{\partial a_{i_1\dots i_p}}{\partial x^i} dx^i \wedge  dx^{i_1}\wedge\dots\wedge dx^{i_p}
\]
\uncover<9->{%
Für $\alpha\in\Omega^p(M)$ und $\beta\in\Omega^q(M)$:}
\begin{itemize}
\item<10-> linear:
\begin{align*}
d(\alpha+\beta)
&=
d\alpha + d\beta
\\
d(c\alpha) &= c\, d\alpha \qquad c\in\mathbb{R}
\end{align*}
\item<11-> Produktregel:
\[
d(\alpha\wedge\beta)
=
d\alpha \wedge \beta
+
(-1)^{p}
\alpha\wedge d\beta
\]
\end{itemize}
\end{block}}
\end{column}
\end{columns}
\end{frame}
\egroup
