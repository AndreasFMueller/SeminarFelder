%
% teil2.tex -- Beispiel-File für teil2 
%
% (c) 2020 Prof Dr Andreas Müller, Hochschule Rapperswil
%
% !TEX root = ../../buch.tex
% !TEX encoding = UTF-8
%
\section{Nichtlineares Medium
\label{particles:section:nichtlinear}}
\kopfrechts{Nichtlineares Medium}
Entgegen des linearen Mediums wobei der Proportionalitätsfaktor nicht von der wirkenden Kraft abhängig war, 
so wird dies bei der Nichtlinearität nun zum Problem.
Dies bedeutet also, dass sich das Hookesche Gesetz von 
\[
    \Delta l
    = 
    \frac{F}{D}
\]
zu
\[
    \Delta l
    = 
    \frac{F}{D(F)}
\]
abändert. 
Wie dieser Proportionalitätsfaktor genau von der Kraft abhängt, 
ist je nach Medium unterschiedlich.
Ein Beispiel davon, wird im Video \mytodo{Referenz Einfügen} von Huygens Optics gezeigt,
wobei für $D$ die Formel 
\[
    D(F)
    =
    D_0
    \cdot
    (1 + \alpha |F|^n)
\]
eingesetzt wird.
Wiederholen wir nun wieder die Simulation aus Abschnitt~\ref{particles:section:lin-medium:superposition}, 
so sieht diese in aus wie in Abbildung~\mytodo{Abbildung mit nichtlinearem Medium und Referenz dazu einfügen}.
% TODO: Grafik mit gleichem Aufbau wie in "lineares Medium", aber mit nichtlinearem medium (Zwei Lichtwellen)


\subsection{Schwinger-Limit in der Praxis}
Wie bereits in Abschnitt~\ref{particles:section:lin-medium:schwinger} erwähnt, 
ist das praktische Schwinger-Limit noch nicht genug erforscht um eine eindeutige Aussage darüber zu machen.
Es gibt jedoch reichlich andere Medien---beispielsweise Glasfasern---welche bereits bei geringen Feldstärken nichtlineare Eigenschaften aufweisen.
\mynote{Hier fehlt noch ein wenig Inhalt. Bin mir aber nicht ganz sicher, was genau\ldots}


\subsection{Selbstfokussierung -- Confined Energy}
% TODO: Was für Bedingungen müssen erfüllt sein, damit dies geschieht? 
%       Kann man das überhaupt so einfach klassifizieren?
%       Allenfalls Analogie zu Schwingkreisen herstellen?
Die Interaktion von den beiden Lichtstrahlen in Abbildung~\mytodo{Referenz auf Abbildung nichtlinearer Lichtstrahlen}
ist bereits interessant, jedoch wird es noch viel spannender, 
wenn statt zwei Strahlenquellen, eine punktuelle Quelle eingefügt wird.
Passt man nun die Parameter des nichtlinearen Teils, sowie die der Quelle an, 
so verschwinden die emittierten Wellen nicht einfach, 
sondern sie interagieren so, dass sie sich selbst fokussieren.
Betrachtet man nun die Kräfte, die im Feld wirken, 
so scheint es, als ob sich ein Teil vom Rest abgrenzt.
\mytodo{Grafik von "Partikel" einfügen.}
% TODO: Grafik von "Partikel" einfügen

Doch was geschieht hier genau? 
\mynote{Analogie zu Schwingkreis herstellen.}
\mynote{Hat das Medium eine Eigenfrequenz? Falls ja, ist diese ausschlaggebend, ob sich die Energie fängt?}
\mynote{Kann man dies als \emph{Soliton} bezeichnen?}

