%
% anwendungenFelder.tex -- 3.	Anwendungen auf Felder, PDEs in ODE umwandeln (Grundlage, um Feldgleichungen als eine Familie von Oszillator-Gleichungen zu interpretieren)
%
% (c) 2020 Prof Dr Andreas Müller, Hochschule Rapperswil
%
% !TEX root = ../../buch.tex
% !TEX encoding = UTF-8
%

%    Anwendungen auf Felder, PDEs in ODE umwandeln (Grundlage, um Feldgleichungen als eine Familie von Oszillator-Gleichungen zu interpretieren) 


\section{Anwendung auf Feld\label{fourier:section:teil0}}

\kopfrechts{Anwendung auf Feld}

Gewöhnliche Differentialgleichungen sind berühmt berüchtigt schwierig zu lösen.
Schlimmer geht es immer.
Partielle Differenzialgleichungen enthaltet Ableitungen nach mehreren Variabeln.
In diesem Abschnitt vereinfachen wir eine partielle in eine Gewöhnliche, mithilfe der Fourierreihe.
Alle Ableitungen sollten schlussendlich den identischen Nenner besitzen.
Als partielle Differentialgleichung verwenden wir die eindimensionale Wellengleichung mit der Lichtgeschwindigkeit als Ausbreitungsgeschwindigkeit. 

\begin{equation}
	\frac{\partial^2 u(x, t)}{\partial t^2} = c^2 \cdot \frac{\partial^2 u(x, t)}{\partial x^2}
\end{equation}

Wir nehmen an $u(x, t)$ ist eine sich periodisch wiederholende Funktion. 
Wir führen eine Fourierreihen-Entwicklung durch. 

\begin{equation}
	u(x,t) = \frac{a_0}{2} + \sum_{n=1}^{\infty} \left( a_n(t) \cos(n \omega x) + b_n(t) \sin(n \omega x) \right)
\end{equation}

Die Fourier Koeffizienten $a_n(t)$ und $b_n(t)$ sind neu von der Zeit abhängig. 
$a_0$ ist konstant.
Nun wird $u(x,t)$ in seiner Summenform in die Wellengleichung eingebaut. 
Dazu muss fleissig abgeleitet werden, je zwei Mal nach $t$ und $x$.

\begin{equation}
	\frac{\partial^2 u(x,t)}{\partial t^2} = \sum_{n=1}^{\infty} \left( \frac{d^2}{dt^2} a_n(t) \cos(n \omega x) + \frac{d^2}{dt^2} b_n(t) \sin(n \omega x) \right)
\end{equation}

\begin{equation}
	\frac{\partial^2 u(x,t)}{\partial x^2} = \sum_{n=1}^{\infty} \left( -a_n(t) n^2 \omega^2 \cos(n \omega t) - b_n(t) n^2 \omega^2 \sin(n \omega t) \right)
\end{equation}

Die Resultate in die Wellengleichung einsetzen.

\begin{equation}
	 \sum_{n=1}^{\infty} \left( \frac{d^2}{dt^2} a_n(t) \cos(n \omega x) + \frac{d^2}{dt^2} b_n(t) \sin(n \omega x) \right) = c^2 \sum_{n=1}^{\infty} \left( -a_n(t) n^2 \omega^2 \cos(n \omega t) - b_n(t) n^2 \omega^2 \sin(n \omega t) \right) 
\end{equation}

Die Beiden Summenzeichen kürzen sich raus. Mithilfe des Koeffizientenvergleichs, findet man zwei lösungen:

\begin{equation}
	\frac{d^2}{dt^2} a_n(t) + a_n(t) n^2 \omega^2 c^2 = 0
	  \quad   \text{und} \quad  \frac{d^2}{dt^2} b_n(t) + b_n(t) n^2 \omega^2 c^2 = 0
\end{equation}

Wenn man im physik Unterricht aufgepasst hat, sollte einem diese Form bekannt vorkommen. Die Differentialgleichung vom ungedämpften Federpendel besitz dieselbe Form. 

\begin{equation}
	\frac{d^2}{dt^2} x(t) + \omega^2 x(t) = 0
\end{equation}

Die lösung von dieser DGL ist einfach eine ungedämpfte schwingung der Form 

\begin{equation}
x(t) = A \cos(\omega t) + B \sin(\omega t) \quad \text{oder} \quad x(t) = C \cos(\omega t - \varphi)
\end{equation}

Dieses Resultat beschreib mit welcher Frequenz und Amplitude eine Masse an einer Feder schwingt. 

Könnte man nun dasselbe mit der resultierenden Lösung der Wellengleichung machen?
Das Resultat wäre eine unendliche Anzahl von Felderpendeln, da $n$ nach unendlich läuft. 
Die Kreisfrequenz wäre  $n \omega c$. 
Dies macht mathematisch Sinn, physikalisch müsste man sich jedoch fragen, was genau hin und her pendelt und ob eine unendliche Frequenz Sinn macht.


In diesem Kapitel wurde die Thematik stark vereinfacht dargestellt. 
Anstelle eines vollständigen Feldes $u(x,y,z,t)$ wurde lediglich eine eindimensionale Funktion $u(x,t)$ betrachtet. Die Fourierreihe lässt sich zwar auch auf mehrdimensionale Felder anwenden, der damit verbundene mathematische Aufwand ist jedoch deutlich höher. 








