%
% teil1.tex -- Beispiel-File für das Paper
%
% (c) 2020 Prof Dr Andreas Müller, Hochschule Rapperswil
%
% !TEX root = ../../buch.tex
% !TEX encoding = UTF-8
%
\section{Herleitung der Mediumsabhängigkeit
\label{schall:section:teil1}}
\kopfrechts{Problemstellung}

\kopfrechts{Teil 1}

In diesem Abschnitt leiten wir die eindimensionale Wellengleichung her und zeigen, wie daraus die bekannten Ausdrücke für die Schallgeschwindigkeit in Gasen, Flüssigkeiten und Festkörpern folgen \cite{schall:kinsler,schall:landaulifschitz}.

\subsection{Grundannahmen}
Wir betrachten kleine Störungen um einen ruhenden Grundzustand $(p_0,\rho_0)$.
Es seien
\[
    p(x,t)=p_0+p'(x,t),\qquad \rho(x,t)=\rho_0+\rho'(x,t),\qquad v(x,t)=v'(x,t),
\]
wobei $|p'|\ll p_0$, $|\rho'|\ll \rho_0$ und $|v'|\ll 1$ (lineare Akustik).
Wir nehmen ferner reibungsfreie, quellenfreie Strömung an und vernachlässigen äußere Kräfte.

\subsection{Kontinuität und Impulsbilanz (linearisiert)}
Die Massenbilanz (Kontinuitätsgleichung) in 1D lautet
\begin{equation}
    \frac{\partial \rho}{\partial t}+\frac{\partial(\rho v)}{\partial x}=0.
\end{equation}
Linearisiert um $(\rho_0, v=0)$ ergibt sich
\begin{equation}
    \frac{\partial \rho'}{\partial t}+\rho_0\,\frac{\partial v'}{\partial x}=0.
    \label{eq:lin-cont}
\end{equation}
Die Impulsbilanz (Euler-Gleichung) in 1D ist
\begin{equation}
    \rho\left(\frac{\partial v}{\partial t}+v\frac{\partial v}{\partial x}\right)=-\frac{\partial p}{\partial x}.
\end{equation}
Linearisiert (und ohne $v\partial_x v$) folgt
\begin{equation}
    \rho_0\,\frac{\partial v'}{\partial t}=-\frac{\partial p'}{\partial x}.
    \label{eq:lin-mom}
\end{equation}

\subsection{Thermodynamische Schließung}
Zur Schließung benötigt man einen Zusammenhang zwischen $p'$ und $\rho'$.
Für schnelle, kleine Schalldehnungen ist der Prozess näherungsweise \emph{isentrop} (Entropie $s=\mathrm{const.}$), daher
\begin{equation}
    p'= \left(\frac{\partial p}{\partial \rho}\right)_{s}\,\rho' \;\;=: \;\; c^2\,\rho'.
    \label{eq:closure}
\end{equation}
Die Größe $c$ ist \emph{per Definition} die Schallgeschwindigkeit des Mediums.

\subsection{Wellengleichung für Druck- und Dichteschwankungen}
Wenden wir $\partial_t$ auf \eqref{eq:lin-cont} an und $\partial_x$ auf \eqref{eq:lin-mom}, eliminieren $v'$ und nutzen \eqref{eq:closure}, so erhalten wir
\begin{align}
    \frac{\partial^2 \rho'}{\partial t^2} - c^2\,\frac{\partial^2 \rho'}{\partial x^2} &= 0, \\
    \frac{\partial^2 p'}{\partial t^2} - c^2\,\frac{\partial^2 p'}{\partial x^2} &= 0.
\end{align}
Damit genügen sowohl Druck- als auch Dichteschwankungen der klassischen 1D-Wellengleichung
\begin{equation}
    \boxed{\;\;\frac{\partial^2 \phi}{\partial t^2} = c^2\,\frac{\partial^2 \phi}{\partial x^2}, \qquad \phi\in\{p',\rho'\}\; .\;\;}
\end{equation}

\subsection{Alternative Form: Wellengleichung für die Teilchenverschiebung}
Führt man die \emph{Teilchenverschiebung} $\xi(x,t)$ ein mit $v'=\partial_t \xi$, so liefern \eqref{eq:lin-cont} und \eqref{eq:closure} nach Eliminierung von $p'$ und $\rho'$ ebenfalls
\begin{equation}
    \frac{\partial^2 \xi}{\partial t^2} = c^2\,\frac{\partial^2 \xi}{\partial x^2}.
\end{equation}
Diese Form ist besonders in der Elastodynamik von Festkörpern nützlich.

\subsection{Schallgeschwindigkeit in verschiedenen Medien}

\subsubsection*{Gase}
Für ein ideales Gas mit Zustandsgleichung $p=\rho R T$ und adiabatischem (isentropem) Prozess gilt $p\rho^{-\kappa}=\mathrm{const.}$, woraus aus \eqref{eq:closure} folgt
\begin{equation}
    \left(\frac{\partial p}{\partial \rho}\right)_{s}=\kappa\,\frac{p}{\rho}
    \quad\Longrightarrow\quad
    \boxed{\,c_g=\sqrt{\kappa\,\frac{p}{\rho}}\,}.
\end{equation}
Mit $p=\rho R T$ folgt die gebräuchliche Darstellung
\begin{equation}
    \boxed{\,c_g=\sqrt{\kappa R T}\,}.
\end{equation}
Dies reproduziert die in der Einleitung verwendete Formel.

\subsubsection*{Flüssigkeiten}
Für (nahezu) inkompressible Flüssigkeiten ist die isentrope Kompressibilität durch das \emph{Kompressionsmodul} (Bulk-Modul) $K$ gegeben:
\begin{equation}
    K := \rho\left(\frac{\partial p}{\partial \rho}\right)_{s}.
\end{equation}
Damit folgt aus \eqref{eq:closure}
\begin{equation}
    \left(\frac{\partial p}{\partial \rho}\right)_{s}=\frac{K}{\rho}
    \quad\Longrightarrow\quad
    \boxed{\,c_l=\sqrt{\frac{K}{\rho}}\,},
\end{equation}
wie in der Einleitung angegeben.

\subsubsection*{Festkörper}
In isotropen, homogenen Festkörpern beschreibt die lineare Elastizität (Hooke) die Beziehung zwischen Spannung $\sigma$ und Dehnung $\varepsilon$
Mit den Lamé-Konstanten $(\lambda,\mu)$, wobei $\mu=G$ das Schubmodul ist und $\mathbf{u}$ das Verschibungsfeld im Festkörper beschreibt, lautet im 3D-Fall:
\begin{equation}
    \sigma = \lambda\,(\nabla\!\cdot\! \mathbf{u})\,\mathbf{I} + 2\mu\,\varepsilon,
    \qquad \varepsilon=\tfrac{1}{2}\left(\nabla\mathbf{u}+(\nabla\mathbf{u})^{\!\top}\right).
\end{equation}
wobei $\mathbf{I}$ die Einheitsmatrix ist und $\nabla\mathbf{u}$ die Volumendehnung beschreibt.

\paragraph{Hooke (isotrop, linear)}
Mit Lamé-Konstanten $\lambda,\mu$ gilt
\begin{equation}
  \sigma_{ij} = \lambda\,\varepsilon_{kk}\,\delta_{ij} + 2\mu\,\varepsilon_{ij},
  \qquad
  \varepsilon_{ij} = \tfrac12(\partial_i u_j + \partial_j u_i).
  \label{eq:hooke-index}
\end{equation}

\paragraph{Impulsbilanz (ohne Körperkräfte)}
\begin{equation}
  \rho\,\partial_{tt} u_i = \partial_j \sigma_{ij}.
  \label{eq:momentum-index}
\end{equation}

\paragraph{Der „Einsetz–Schritt“ in Indexschreibweise}
Setze \eqref{eq:hooke-index} in \eqref{eq:momentum-index} ein:
\[
    \partial_j \sigma_{ij}
    = \partial_j\!\big(\lambda\,\varepsilon_{kk}\,\delta_{ij} + 2\mu\,\varepsilon_{ij}\big)
    = \lambda\,\partial_i \varepsilon_{kk} + 2\mu\,\partial_j \varepsilon_{ij},
\]
denn $\partial_j(\delta_{ij}X)=\partial_i X$.
Nun $\varepsilon_{kk} = \partial_k u_k = \nabla\!\cdot\!\mathbf{u}$ und
\[
    \partial_j \varepsilon_{ij}
    = \frac12\,\partial_j(\partial_i u_j + \partial_j u_i)
    = \frac12\big(\partial_i\partial_j u_j + \partial_j\partial_j u_i\big)
    = \frac12\big(\partial_i(\partial_k u_k) + \Delta u_i\big).
\]
Damit
\[
    \partial_j \sigma_{ij}
    = \lambda\,\partial_i(\partial_k u_k) + 2\mu\cdot \frac12\big(\partial_i(\partial_k u_k) + \Delta u_i\big)
    = (\lambda+\mu)\,\partial_i(\partial_k u_k) + \mu\,\Delta u_i.
\]
In Vektorschreibweise:
\begin{equation}
    \boxed{\,\rho\,\partial_{tt}\mathbf{u} \;=\; (\lambda+\mu)\,\nabla(\nabla\!\cdot\!\mathbf{u}) \;+\; \mu\,\Delta \mathbf{u}\,}
    \label{eq:navier}
\end{equation}
Dies ist die \emph{Navier–Cauchy–Gleichung}.

\paragraph{Wellenzerlegung per Divergenz und Rotation}
Wende $\nabla\!\cdot$ auf \eqref{eq:navier} an.
Da $\nabla\!\cdot(\Delta\mathbf{u})=\Delta(\nabla\!\cdot\!\mathbf{u})$ gilt:
\[
    \rho\,\partial_{tt}(\nabla\!\cdot\!\mathbf{u})
    = (\lambda+\mu)\,\Delta(\nabla\!\cdot\!\mathbf{u}) + \mu\,\Delta(\nabla\!\cdot\!\mathbf{u})
    = (\lambda+2\mu)\,\Delta(\nabla\!\cdot\!\mathbf{u}).
\]
Setze $\theta := \nabla\!\cdot\!\mathbf{u}$:
\begin{equation}
    \boxed{\,\partial_{tt}\theta = c_L^2\,\Delta\theta,\qquad c_L^2=\frac{\lambda+2\mu}{\rho}\,}
    \label{eq:longitudinal}
\end{equation}
Dies ist die \emph{Longitudinalwelle} (Druck-/Volumendehnung).
Wende nun den Rotations-Operator $\nabla\times$ auf \eqref{eq:navier} an.
Wegen $\nabla\times\nabla f=\mathbf{0}$ verschwindet der erste Term:
\[
    \rho\,\partial_{tt}(\nabla\times\mathbf{u}) = \mu\,\nabla\times(\Delta\mathbf{u})
    = \mu\,\Delta(\nabla\times\mathbf{u}).
\]
Mit $\boldsymbol{\omega}:=\nabla\times\mathbf{u}$:
\begin{equation}
    \boxed{\,\partial_{tt}\boldsymbol{\omega} = c_T^2\,\Delta\boldsymbol{\omega},\qquad c_T^2=\frac{\mu}{\rho}\,}
    \label{eq:transversal}
\end{equation}
Dies ist die \emph{Transversalwelle} (Scherung).

\paragraph{Interpretation}
\begin{itemize}
  \item $c_L$ hängt von \emph{Volumensteifigkeit} ab: $(\lambda+2\mu)$ skaliert den Widerstand gegen Volumenänderung (\emph{Druck-/Kompressionswelle}).
  \item $c_T$ hängt allein von der \emph{Schersteifigkeit} $\mu$ ab (\emph{Scherwelle}); in Fluiden ist $\mu\approx 0$, daher keine Transversalwellen.
\end{itemize}

Somit erhält man in isotropen Medien zwei Wellengeschwindigkeiten \cite{schall:landaulifschitz,schall:gurtin}:
\begin{align}
    \boxed{\,c_T=\sqrt{\frac{\mu}{\rho}}=\sqrt{\frac{G}{\rho}}\,} \quad &\text{(transversal, Scherwellen)},\\[2mm]
    \boxed{\,c_L=\sqrt{\frac{\lambda+2\mu}{\rho}}\,} \quad &\text{(longitudinal, Druckwellen)}.
\end{align}
In technischen Materialparametern, sprich Elastizitätsmodul $E$ und Poissonzahl $\nu$, mit
\[
    \mu=G=\frac{E}{2(1+\nu)},
    \qquad
    \lambda=\frac{E\,\nu}{(1+\nu)(1-2\nu)},
\]
ergeben sich die in der Einleitung verwendeten Ausdrücke
\begin{align}
    \boxed{\,c_{s,\mathrm{long}}
    = \sqrt{\frac{E(1-\nu)}{\rho(1+\nu)(1-2\nu)}}\,},\qquad
    \boxed{\,c_{s,\mathrm{trans}}
    = \sqrt{\frac{E}{2\rho(1+\nu)}}=\sqrt{\frac{G}{\rho}}\,}.
\end{align}
Typisch ist $c_L>c_T$, und beide liegen deutlich über Gas- und Flüssigkeitswerten (z.\,B. Stahl $c\approx 5{,}0\text{--}5{,}9\,\mathrm{km/s}$ für Longitudinalwellen, je nach Legierung).

\subsection{Reduktion auf die in der Einleitung verwendeten Vereinfachungen}
Unter den in \S\ref{schall:section:teil0} genannten Annahmen, sprich 1D, kleine Störungen, keine Reflexionen/Hindernisse sowie konstante Materialparameter, reduziert sich die Dynamik auf die lineare Wellengleichung
\[
    \frac{\partial^2 \phi}{\partial t^2}=c^2\,\frac{\partial^2 \phi}{\partial x^2},
\]
wobei $c$ je nach Medium durch die obigen isentropen bzw.\ elastischen Beziehungen gegeben ist:
\[
    c=\begin{cases}
    \sqrt{\kappa\,p/\rho}=\sqrt{\kappa R T}, & \text{Gase},\\[1mm]
    \sqrt{K/\rho}, & \text{Flüssigkeiten},\\[1mm]
    \sqrt{(\lambda+2\mu)/\rho}\;\text{ bzw. }\;\sqrt{\mu/\rho}, & \text{Festkörper (long./trans.)}.
\end{cases}
\]
Diese Formen entsprechen genau den in der Einleitung verwendeten vereinfachten Geschwindigkeitsausdrücken und motivieren die tabellarischen Zahlenwerte für typische Medien.


