%
% teil2.tex -- Beispiel-File für teil2 
%
% (c) 2020 Prof Dr Andreas Müller, Hochschule Rapperswil
%
% !TEX root = ../../buch.tex
% !TEX encoding = UTF-8
%
\section{Potentialströmung\label{ueberschall:section:potentialstroemung}}
\kopfrechts{Potentialströmung}
Bei der Potentialströmung handelt sich um eine quellenfreie 
und wirbelfreie Strömung
\begin{align*}
    \mathrm{div}\,\vec{v} &= 0 \\
    \mathrm{rot}\,\vec{v} &= 0,
\end{align*}
wobei $\vec{v}$ die Geschwindigkeit der Strömung ist.
Diese Idealisierung ist ein Spezialfall der laminaren Strömungen.
Unter Einhaltung der genannten Bedingung besitzt dieses
Geschwindigkeitsfeld auch ein Potential
\begin{align}
    \vec{v} = \mathrm{grad}\,\Phi,\label{eq:potential}
\end{align}
wobei $\vec{v}$ auch als Gradient des Potentials bezeichnet wird
und bedeutet nichts geringeres als 
die Ableitung in jede Koordinatenrichtung
\begin{align}
    \mathrm{grad}\,\Phi
    =
    \nabla\Phi
    =
    \begin{pmatrix}
        \frac{\partial\,\Phi_x}{\partial\,x}\\
        \frac{\partial\,\Phi_y}{\partial\,y}\\
        \frac{\partial\,\Phi_z}{\partial\,z}
    \end{pmatrix}.
\end{align} 
Dies kann auch als normale Multiplikation mit dem Nabla-Operator $\nabla$
dargestellt werden da das Potential $\Phi$ skalar ist.

Zudem nehmen wir eine stationäre Strömung an.
Natürlich muss auch die Kontinuitätsgleichung
\begin{align*}
    \frac{\partial\,\rho}{\partial\,t} + 
    \mathrm{div}\,\vec{v} &= 0
\end{align*}
erfüllt sein.
Bei inkompressiblen Medien wird der erste Term sowieso zu null
\begin{align*}
    \frac{\partial\,\rho}{\partial\,t} &= 0,
\end{align*}
wonach nur noch die Quellen- und Senkenfreiheit übrig bleibt.
Die Divergenz der Geschwindigkeit kann auch als Skalarprodukt 
des Nabla-Operators und der Geschwindigkeit dargestellt werden
\begin{align*}
    \mathrm{div}\,\vec{v}
    =
    \nabla \cdot \vec{v}
    =
    \frac{\partial\,v_x}{\partial\,x} +
    \frac{\partial\,v_y}{\partial\,y} +
    \frac{\partial\,v_z}{\partial\,z}.
\end{align*}
Das ist ein Mass dafür wie sehr sich ein Feld wie eine Quelle 
oder eine Senke verhält.
Bei $\mathrm{div}\,\vec{v} = 0$ ist sie weder Quelle noch Senke.
Das Kreuzprodukt mit dem Operator entspricht
der Rotation der Geschwindigkeit
\begin{align*}
    \mathrm{rot}\,\vec{v} 
    =
    \nabla \times \vec{v}
    =
    \begin{pmatrix}
        \frac{\partial\,v_z}{\partial\,y} - \frac{\partial\,v_y}{\partial\,z} \\
        \frac{\partial\,v_x}{\partial\,z} - \frac{\partial\,v_z}{\partial\,x} \\
        \frac{\partial\,v_y}{\partial\,x} - \frac{\partial\,v_x}{\partial\,y}
    \end{pmatrix},
\end{align*}
welcher angibt wie sich ein Körper der diesem Vektorfeld 
ausgesetzt rotieren würde.
Was für ein wirbelfreies Feld $\mathrm{rot}\,\vec{v} = 0$ bedeutet,
wodurch das Vektorfeld genau dann der Gradient einer Funktion ist
wie in~\ref{eq:potential}.

Wenn nun in die Kontinuitätsgleichung des inkompressiblen
Falls $ \nabla \cdot \vec{v} = 0$ die Definition 
des Geschwindigkeitsfeld~\ref{eq:potential} eingesetzt wird, dann folgt daraus
\begin{align}
    \nabla \cdot \nabla \Phi
    = 
    \nabla\,^2 \Phi
    =
    \Delta\Phi
    =
    0,\label{eq:laplace}
\end{align}
auch bezeichnet als Laplace-Gleichung mit dem Laplace-Operator $\Delta$.


