%
% 2-Formen
%
\section{Tensoren zweiter Stufe, 2-Vektoren und 2-Formen auf Mannigfaltigkeiten
\label{buch:green:section:2formen}}
\kopfrechts{2-Formen}
Die Algebra der 2-Vektoren ist nicht auf einen Vektorraum beschränkt,
sie kann auch auf die Tangentialvektoren einer $n$-dimensionalen
Mannigfaltigkeit angewendet werden.
Ein 2-Vektor auf einer Mannigfaltigkeit ist dann ein Tensor zweiter
Stufe vom Typ $(2,0)$.
In einer Karte wird er gegeben durch eine Grösse $a^{ik}$ mit zwei
kontravarianten Indizes.
Gegenüber einem beliebigen Tensor zweiter Stufe zeichnet sich der 
2-Vektor dadurch aus, dass $a^{ik}$ antisymmetrisch ist, dass also
$a^{ik}=-a^{ki}$ gilt.

%
% Notation für Tensoren
%
\subsection{Notation für Tensoren}
In Kapitel~\ref{chapter:kurvenintegral} haben wir zwar den Begriff des
Tensors eingeführt, aber nur eine Notation für Vektoren eingeführt.
Wenn mit Tensoren gerechnet worden ist, kam immer nur die Komponenten
in einer Karte zur Anwendung.
Das Wedge-Produkt von Vektoren hat 2-Vektoren ergeben, kann aber
nicht alle Tensoren zweiter Stufe darstellen.
Wir erweitern daher die Notation, indem wir wie bei der Einführung der
Basis für die 2-Vektoren, neue Basisvektoren für die Tensoren wie 
folgt definieren.

\begin{definition}
\label{buch:green:tensoren:definition:tensorprodukt}
Sei $V$ ein $n$-dimensionaler reeller Vektorraum mit Basis
$\vec{e}_1,\dots,\vec{e}_n$.
Der Vektorraum
\[
V^{\otimes p}
=
\underbrace{
V\otimes\dots\otimes V
}_{\displaystyle \text{$p$ Faktoren}}
=
\langle \vec{e}_{i_1}\otimes\dots\otimes \vec{e}_{i_p}
\mid
1\le 
i_1,\dots,i_p
\le n
\rangle
\]
heisst das $p$-fache Tensorprodukt von $V$.
\index{Tensorprodukt}%
Ein Tensor $p$-ter Stufe ist also von der Form
\[
a_{i_1\dots i_p}\vec{e}_{i_1}\otimes\dots\otimes\vec{e}_{i_p}
\]
mit Koeffizienten $a_{i_1\dots i_p}\in\mathbb{R}$.
\end{definition}

Ein 2-Vektor ist ein Tensor in $V^{\otimes 2}$, der antisymmetrisch ist.
Für die Basisvektoren der 2-Vektoren können daher als Tensoren
\[
\vec{e}_i\wedge\vec{e}_k
=
\vec{e}_i\otimes\vec{e}_k
-
\vec{e}_k\otimes\vec{e}_i
\]
geschrieben werden.
Ein 2-Vektor ist also allgemein eine Linearkombination
\begin{equation}
a^{ik}\vec{e}_i\otimes\vec{e}_k
\label{buch:green:tensoren:eqn:2tensor}
\end{equation}
mit $a^{ik}=-a^{ki}$.

Die Definition~\ref{buch:green:tensoren:definition:tensorprodukt}
berücksichtigt die Transformationseigenschaften nicht,
die die Komponenten eines aus Tangentialvektoren gebildeten
Tensors zusätzlich erfüllen müssen.
Ein kontravarianter Tensor zweiter Stufe kann dann als
\[
a
=
a^{ik}(x) \frac{\partial}{\partial x^i}\otimes\frac{\partial}{\partial x^k}
\]
geschrieben werden, wobei die bei einem Koordinatenwechsel die
Indizes $i$ und $k$ kontravariant transformiert werden.
Ein 2-Vektor auf der Mannigfaltigkeit entsteht dann aus den Koeffizienten
$a^{ik}(x)$ durch
\[
\sum_{i<k}
a^{ik}
\frac{\partial}{\partial x^i}
\wedge
\frac{\partial}{\partial x^k}
=
a^{ik}(x)
\frac{\partial}{\partial x^i}
\otimes
\frac{\partial}{\partial x^k},
\]
wobei $a^{ik}(x)=-a^{ki}(x)$ sein muss.

%
% 2-Formen
%
\subsection{2-Formen}
Linearformen sind lineare Abbildungen von Vektoren auf reelle Zahlen.
Linearformen können aber auch auf Tensoren zweiter Stufe definiert werden.
Eine Linearform $f\colon V^{\otimes 2}\to\mathbb{R}$ muss den Basisvektoren
$\vec{e}_i\otimes\vec{e}_k$ einen Wert zuordnen.
Schreiben wir $f(\vec{e}_i\otimes\vec{e}_k)=f_{ik}$, dann ist der Wert
auf einem Tensor der Form
\eqref{buch:green:tensoren:eqn:2tensor}
gegeben durch
\[
f\bigl(
a^{ik}\vec{e}_i\otimes \vec{e}_k
\bigr)
=
a^{ik}
f_{ik}.
\]
Eine Linearform auf $V^{\otimes 2}$ kann auch aus zwei Linearformen
$f$ und $g$ auf $V$ konstruiert werden.
Solche Linearformen können durch die Werte
\[
f_i = f(\vec{e}_i) 
\qquad\text{und}\qquad
g_k = f(\vec{e}_k)
\]
auf den Basisvektoren konstruiert werden.
Auf dem Tensor $a$ ergibt sie den Wert
\[
(f\otimes g)
=
a^{ik}f(\vec{e}_i)g(\vec{e}_k)
=
a^{ik}f_ig_k.
\]

