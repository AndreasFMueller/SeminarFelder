%
% einleitung.tex -- Beispiel-File für die Einleitung
%
% (c) 2020 Prof Dr Andreas Müller, Hochschule Rapperswil
%
% !TEX root = ../../buch.tex
% !TEX encoding = UTF-8
%
\section{Die Differentialgleichung\label{mongeampere:section:teil0}}
\kopfrechts{Teil 0}
Die monge-ampèresche Gleichung ist
\begin{equation}
  \det D^2 u = f(x, u, \nabla u),
  \label{mongeampere:eq:mongeampere}
\end{equation}
wobei $D$ die hessesche Matrix
\begin{equation}
  D^2 u =
  \begin{pmatrix}
    \frac{\partial^2 u}{\partial x_1^2} & \dots &\frac{\partial^2 u}{\partial x_1 \,\partial x_n} \\
  \vdots & \ddots & \vdots \\
  \frac{\partial u}{\partial x_n \,\partial x_1} & \dots &\frac{\partial^2 u}{\partial x_n^2} \\
  \end{pmatrix}
  \label{mongeampere:eq:hess}
\end{equation}
ist.
Wenn man die Gleichung ansieht, erkennt man, dass es sich bei ihr um eine partielle
nicht-lineare Differentialgleichung handelt.
Das Lösen einer solchen Differentialgleichung ist im Allgemeinen ein anspruchvolles Problem.
Anhand von Bedingungen für $u$ und $f$ kann man \eqref{mongeampere:eq:mongeampere} 
in einen bestimmten Typ Differentialgleichung bringen, was unterschiedliche Lösungsansätze ermöglicht.
Setzt man voraus, dass $u$ eine glatte konvexe Function ist, und $f$ positiv, dann ist \eqref{mongeampere:eq:mongeampere}
eine elliptische partielle Differnetialgleichung.
Gemäss \cite{mongeampere:figalli2018mongeampereequation} gibt eine solche Form Hoffnung auf die Existenz einer regulären Lösung.
Ein anderes Beispiel ist in \cite{mongeampere:figalli2022prescribednegativegausscurvature} zu finden, wo die 
Gleichung linearisiert wird um lokale Lösungen zu finden.

\subsection{Minkowski-Problem}
Eine Anwendung, welche vom deutschen Mathematiker Hermann Minkwoski entdekt wurde, ist das Lösen des
\emph{Prescribed Gaussian Curvature Problem} mit der Monge-Ampere Gleichung.
Diese Anwendung befasst damit eine konvexe kompakte Fläche zu finden, deren Gaussche Krümmung 
an jedem Punkt gegeben ist.

