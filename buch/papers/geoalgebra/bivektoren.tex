
\section{Bivektoren
\label{geoalgebra:section:bivektoren}}
\kopfrechts{Bivektoren}
\subsection{Idee}
\label{geoalgebra:section:bivektoren:idee}
\newcommand{\vone}{\color{red}\mathbf{v_1}\color{black}}
\newcommand{\vtwo}{\color{blue}\mathbf{v_2}\color{black}}
Durch das Wedgeprodukt zwischen $\vone$ und $\vtwo$ erhalten wir
den
Bivektor $\vone \wedge \vtwo$. Einen Bivektor können wir uns als \emph{gerichtete Fläche}, die zwischen den beiden Vektoren aufgespannt
wird, vorstellen (siehe \autoref{geoalgebra:fig:bivektor-als-flaeche}).
\begin{figure}
  \begin{center}
\begin{tikzpicture}
  \draw[step=.5cm,gray,very thin] (-0.25,-0.25) grid (5.25,4.25);
  \coordinate (O) at (0,0);
  \coordinate (A) at (3,1);
  \coordinate (B) at (2,3);
  \coordinate (C) at ($(A)+(B)$);

  \fill[blue!30,opacity=0.7] (O) -- (A) -- (C) -- (B) -- cycle;

  \draw[thick,->,red] (O) -- (A) node[midway,below] {$\mathbf{v_1}$};
  \draw[thick,->,blue] (O) -- (B) node[midway,left] {$\mathbf{v_2}$};
  \draw[dashed] (A) -- (C);
  \draw[dashed] (B) -- (C);

  \node (K) at ($0.25*(O)+0.25*(A)+0.25*(B)+0.25*(C)$)
    {$\color{red}\vone \wedge \vtwo$};
  \draw[thick,->] (K) ++(0.6,0) arc (0:270:0.6);
\end{tikzpicture}
  \end{center}
  \caption{Bivektor als gerichtete Fläche}\label{geoalgebra:fig:bivektor-als-flaeche}
\end{figure}

\subsection{Bivektoren als gerichtete Fläche}
Es ist wichtig, dass wir uns die Idee einer \emph{gerichteten Fläche}
verinnerlichen. Wie wir bereits in \autoref{geoalgebra:section:bivektoren:idee} gesehen haben,
gibt das Wedgeprodukt $\vone \wedge \vtwo$
einen Bivektor, der die Fläche zwischen den beiden Vektoren als
eine \emph{im Gegenuhrzeigersinn rotierte Fläche} repräsentiert.
Wir können uns vorstellen, als ob wir den Vektor $\vone$
am Vektor $\vtwo$ ``aufspannen'' (siehe \autoref{geoalgebra:fig:aufspannende-flaeche-v1-v2})

Analog dazu stellen wir uns vor, dass wir beim Bivektor $\vtwo \wedge \vone$ den Vektor
$\vtwo$ am Vektor $\vone$ aufspannen (siehe \autoref{geoalgebra:fig:aufspannende-flaeche-v2-v1})

\begin{figure}
\begin{center}
\begin{tikzpicture}
  \draw[step=.5cm,gray,very thin] (-0.25,-0.25) grid (5.25,4.25);
  \coordinate (O) at (0,0);
  \coordinate (A) at (3,1);
  \coordinate (B) at (2,3);
  \coordinate (C) at ($(A)+(B)$);
  \coordinate (M_AC) at ($0.5*(A)+0.5*(C)$);
  \coordinate (M_OB) at ($0.5*(O)+0.5*(B)$);

  \fill[blue!30,opacity=0.7] (O) -- (A) -- (M_AC) -- (M_OB) -- cycle;

  \draw[thick,->,red] (O) -- (A) node[midway,below] {$\mathbf{v_1}$};
  \draw[thick,->,blue] (O) -- (B) node[midway,left] {$\mathbf{v_2}$};
  \draw[dashed] (A) -- (C);
  \draw[dashed] (B) -- (C);

  \node (K) at ($0.25*(O)+0.25*(A)+0.25*(B)+0.25*(C)$)
    {$\vone \wedge \vtwo$};
  \draw[thick,->] (K) ++(0.6,0) arc (0:270:0.6);
\end{tikzpicture}
\begin{tikzpicture}
  \draw[step=.5cm,gray,very thin] (-0.25,-0.25) grid (5.25,4.25);
  \coordinate (0) at (0, 0);
  \coordinate (A) at (3, 1);
  \coordinate (B) at (2, 3);
  \coordinate (C) at ($(A) + (B)$);
  \fill[blue!30, opacity=0.7] (0) -- (A) -- (C) -- (B) -- cycle;

  \draw[thick, ->, red] (0) -- (A) node[midway,below] {$\mathbf{v_1}$};
  \draw[thick, ->, blue] (0) -- (B) node[midway,left] {$\mathbf{v_2}$};
  \draw[dashed] (A) -- (C);
  \draw[dashed] (B) -- (C);
  \node (K) at ($0.25*(0)+0.25*(A)+0.25*(B)+0.25*($(C)$)$)
  {$\vone \wedge \vtwo$};
\draw[thick,->] (K) ++(0.6,0) arc (0:270:0.6);
\end{tikzpicture}
\end{center}
  \caption{Vektor $\vone$ wird an $\vtwo$ aufgespannt}\label{geoalgebra:fig:aufspannende-flaeche-v1-v2}
\end{figure}


\begin{figure}
\begin{center}
\begin{tikzpicture}
  \draw[step=.5cm,gray,very thin] (-0.25,-0.25) grid (5.25,4.25);
  \coordinate (O) at (0,0);
  \coordinate (A) at (3,1);
  \coordinate (B) at (2,3);
  \coordinate (C) at ($(A)+(B)$);
  \coordinate (M_BC) at ($0.5*(B)+0.5*(C)$);
  \coordinate (M_OA) at ($0.5*(O)+0.5*(A)$);

  \fill[red!30,opacity=0.7] (O) -- (A) -- (M_OA) -- (M_BC) -- (B) -- cycle;

  \draw[thick,->,red] (O) -- (A) node[midway,below] {$\mathbf{v_1}$};
  \draw[thick,->,blue] (O) -- (B) node[midway,left] {$\mathbf{v_2}$};
  \draw[dashed] (A) -- (C);
  \draw[dashed] (B) -- (C);

  \node (K) at ($0.25*(O)+0.25*(A)+0.25*(B)+0.25*(C)$)
    {$\vtwo \wedge \vone$};
  \draw[thick,->] (K) ++(0.6,0) arc (0:-270:0.6);
\end{tikzpicture}
\begin{tikzpicture}
  \draw[step=.5cm,gray,very thin] (-0.25,-0.25) grid (5.25,4.25);
  \coordinate (0) at (0, 0);
  \coordinate (A) at (3, 1);
  \coordinate (B) at (2, 3);
  \coordinate (C) at ($(A) + (B)$);
  \fill[red!30, opacity=0.7] (0) -- (A) -- (C) -- (B) -- cycle;

  \draw[thick, ->, red] (0) -- (A) node[midway,below] {$\mathbf{v_1}$};
  \draw[thick, ->, blue] (0) -- (B) node[midway,left] {$\mathbf{v_2}$};
  \draw[dashed] (A) -- (C);
  \draw[dashed] (B) -- (C);
  \node (K) at ($0.25*(O)+0.25*(A)+0.25*(B)+0.25*(C)$)
    {$\vtwo \wedge \vone$};
  \draw[thick,->] (K) ++(0.6,0) arc (0:-270:0.6);
\end{tikzpicture}
\end{center}
  \caption{$\vtwo$ wird an $\vone$ aufgespannt}\label{geoalgebra:fig:aufspannende-flaeche-v2-v1}
\end{figure}
