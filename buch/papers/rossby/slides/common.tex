%
% common.tex -- gemeinsame Definitionen
%
% (c) 2019 Michael Schmid, Hochschule Rapperswil
%
\usepackage[utf8]{inputenc}
\usepackage[T1]{fontenc}
\usepackage{epic}
\usepackage{color}
\usepackage{textpos}
\usepackage{algorithm}
\usepackage{array}
\usepackage{listings}
\usepackage{ifthen}
\usepackage{adjustbox}
\usepackage{neuralnetwork}
\usepackage{amsmath}
\usepackage{hyperref}
\usepackage{lmodern}
\usepackage{pgffor}
\usepackage{tikz}
\usetikzlibrary{arrows,matrix,positioning}
\usetikzlibrary{overlay-beamer-styles}
% \usetikzlibrary{matrix.skeleton}
\usepackage{pgfplots}
\usepackage{listings}
\usepackage{svg}
\usepackage{bm}
\definecolor{codegreen}{rgb}{0,0.6,0}
\definecolor{codegray}{rgb}{0.5,0.5,0.5}
\definecolor{codepurple}{rgb}{0.58,0,0.82}
\definecolor{backcolour}{rgb}{0.95,0.95,0.92}
\definecolor{ost}{rgb}{0.640625,0,0.53125} % UBC Blue (primary)

\usetikzlibrary{calc,fadings}
\tikzfading[name=fade l,left color=transparent!100,right color=transparent!0]
\tikzfading[name=fade r,right color=transparent!100,left color=transparent!0]
\tikzfading[name=fade d,bottom color=transparent!100,top color=transparent!0]
\tikzfading[name=fade u,top color=transparent!100,bottom color=transparent!0]

% this "frames" a rectangle node
\newcommand\framenode[2][10pt]{
    \fill[white,path fading=fade u] (#2.south west) rectangle ($(#2.south east)+(0, #1)$);
    \fill[white,path fading=fade d] (#2.north west) rectangle ($(#2.north east)+(0,-#1)$);
    \fill[white,path fading=fade l] (#2.south east) rectangle ($(#2.north east)+(-#1,0)$);
    \fill[white,path fading=fade r] (#2.south west) rectangle ($(#2.north west)+( #1,0)$);
}

\lstdefinestyle{Python}{
  numbers=left,
  belowcaptionskip=1\baselineskip,
  breaklines=true,
  frame=l,
  framerule=0pt,
  framesep=-1pt,
  xleftmargin=1em,
  language=Python,
  showstringspaces=false,
  basicstyle=\scriptsize\ttfamily,
  keywordstyle=\bfseries\color{green!40!black},
  commentstyle=\itshape\color{purple!40!black},
  identifierstyle=\color{blue},
  stringstyle=\color{red},
  numberstyle=\ttfamily\tiny,
  backgroundcolor=\color{backcolour}
}

\lstdefinestyle{C++}{
  numbers=left,
  belowcaptionskip=1\baselineskip,
  breaklines=true,
  frame=l,
  framerule=0pt,
  framesep=-1pt,
  xleftmargin=1em,
  language=C++,
  showstringspaces=false,
  basicstyle=\scriptsize\ttfamily,
  keywordstyle=\bfseries\color{green!40!black},
  commentstyle=\itshape\color{purple!40!black},
  identifierstyle=\color{blue},
  stringstyle=\color{red},
  numberstyle=\ttfamily\tiny,
  backgroundcolor=\color{backcolour}
}
\hypersetup{
    colorlinks=true,
    linkcolor=magenta,
    filecolor=magenta,      
    urlcolor=magenta,
    pdfpagemode=FullScreen,
    }

\usetikzlibrary{fit}
\tikzset{%
  highlight/.style={rectangle,rounded corners,fill=red!15,draw,fill opacity=0.5,inner sep=0pt}
}
\newcommand{\tikzmark}[2]{\tikz[overlay,remember picture,baseline=(#1.base)] \node (#1) {#2};}
%
\newcommand{\Highlight}[1][submatrix]{%
    \tikz[overlay,remember picture]{
    \node[highlight,fit=(left.north west) (right.south east)] (#1) {};}
}
\lstdefinestyle{C++}{
  numbers=left,
  belowcaptionskip=1\baselineskip,
  breaklines=true,
  frame=l,
  framerule=0pt,
  framesep=-1pt,
  xleftmargin=1em,
  language=C++,
  showstringspaces=false,
  basicstyle=\scriptsize\ttfamily,
  keywordstyle=\bfseries\color{green!40!black},
  commentstyle=\itshape\color{purple!40!black},
  % identifierstyle=\color{blue},
  stringstyle=\color{red},
  numberstyle=\ttfamily\tiny,
  backgroundcolor=\color{backcolour},
  morekeywords={restrict},
  emph={int,char,double,float,unsigned,RNG_CONTEXT, void,bool,int8_t, int64_t, int32_t, uint16_t, int16_t, inner_shake256_context,uint8_t, fpr, size_t, uint64_t, uint32_t, pragma},
  emphstyle={\color{blue}},
}

\lstset{style=Python}



\usetikzlibrary{shapes.geometric}
\mode<beamer>{%
\usetheme[]{Frankfurt}}
\beamertemplatenavigationsymbolsempty
\usecolortheme[named=ost]{structure}\title[]{Rossby Wellen}
% \institute[icai]{ICAI Interdisciplinary Center for Artificial Intelligence}
\author[The author]{Michael Schmid}\date[]{\today}
\newboolean{presentation}
