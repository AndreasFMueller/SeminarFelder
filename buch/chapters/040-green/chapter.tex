%
% chapter.tex -- Kapitel 4: 2-Formen und der Satz von Green
%
% (c) 2024 Prof Dr Andreas Müller
%
\chapter{2-Vektoren, 2-Formen und der Satz von Green
\label{chapter:green}}
\kopflinks{2-Vektoren, 2-Formen und der Satz von Green}
Ein Durchflussmessgerät misst die Menge eines Fluids, welches pro
Zeiteinheit durch ein Flächenstück strömt.
Ein Magnetfeldsensor verwendet einen Sensor, der die Wirkung des Magnetfeldes
auf eine Drahtschleife misst, die ein Flächenstück berandet.
Je kleiner das Flächenstück, desto genauer die Bestimmung des Magnetfeldes
am Ort des Flächenstücks.
Um ein lokales Naturgesetz zu formulieren, muss also ein mathematisches
Konzept für ``beliebig kleine Flächenstücke'' entwickelt werden.
Dies ist das Ziel des folgenden Kapitels.

%
% 2-Vektoren
%
\section{2-Vektoren}
\kopfrechts{2-Vektoren}

\section{2-Formen}
\kopfrechts{2-Formen}

\section{Flächenintegrale}
\kopfrechts{Flächenintegrale}

\section{Der Satz von Green}
\kopfrechts{Der Satz von Green}


