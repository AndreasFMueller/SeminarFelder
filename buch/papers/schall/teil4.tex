\section{Überschallknall: Entstehung und Low-Boom-Stra\-te\-gien
\label{schall:section:boom}}
\kopfrechts{Überschallknall: Entstehung und Low-Boom-Strategien}

In diesem Abschnitt betrachten wir, warum Überschallflugzeuge am Boden
einen lauten Knall verursachen, wie sich dieser ausbreitet, und welche
Strategien es gibt, um den Knall zu reduzieren.

Ein Flugzeug mit Mach-Zahl $\textit{Ma}\ge1$ erzeugt entlang seiner Länge diskrete
Schockbeiträge an Bug, Tragflächen, Rumpfänderungen, Leitwerke und Auspuff,
die sich bei der Ausbreitung in der Atmosphäre je nach Grösse des Flugzeugs
überlagern und am Boden als impulsförmiger Knall wahrgenommen werden.
Für grosse Geometrien und hohe Mach-Zahlen entstehen dabei
starke, einzelne Drucksprünge, die als \emph{Boom} bezeichnet werden.
\index{Boom}%
Moderne Überschallflugzeuge zielen darauf ab, den Knall in einen
weichen \emph{Thump} zu verwandeln, der als deutlich weniger störend
\index{Thump}%
empfunden wird.

Um die Ausbreitung und Formung des Überschallknalls zu verstehen, betrachten wir
zwei Stufen: die Nahfeld-Geometrie nahe am Flugzeug und die Fernfeld-Ausbreitung
durch die Atmosphäre.
Die Nahfeld-Geometrie, definiert durch die Flugzeugform sowie die  Last- oder
Auftriebsverteilung, bestimmt die anfängliche Drucksignatur;
die Fernfeld-Ausbreitung formt diese durch Nichtlinearität und atmosphärische Dämpfung.

\subsection{Nahfeld: $F$-Funktion, Äquivalentfläche und Initialsignatur}
Im Überschallflug überlagern sich alle vom Flugzeug erzeugten
Druckänderungen zu einer mitwandernden Wellenstruktur im Nahfeld.
Entscheidend ist dabei nicht die absolute Grösse, sondern die
Längsverteilung der Querschnittsfläche:
Sanfte, stetige Änderungen entlang des Rumpfs führen zu weichen
Druckanstiegen, abrupte Geometriesprünge zu schärferen Schockanteilen.
Whithams Ansatz fasst die reale Geometrie (Rumpf, Flügel, Einläufe) zu
einer \emph{Äquivalentfläche} zusammen und liefert daraus eine kompakte
\index{Aquivalentflache@Äquivalentfläche}%
Kenngrösse, die wir hier als \(F\)-Funktion nutzen, um
die Form der Nahfeldsignatur qualitativ zu lesen.
Ziel von Low-Boom-Entwürfen ist daher eine möglichst glatt
verlaufende Signatur bereits im Nahfeld.

Für schlanke Überschallkörper liefert Whithams lineare Wellentheorie eine
Kopplung zwischen der Flugzeug-Äquivalentfläche $A(x)$ und der
Nahfeldsignatur über die sogenannte $F$-Funktion~\cite{schall:whitham}.
Whithams Theorie definiert die zugehörige Funktion
als Abel-Transformierte von $A_e''$:
\begin{equation*}
    F(x_e)=\frac{1}{2\pi}\int_{0}^{x_e}\frac{A_e''(t)}{\!\sqrt{x_e-t}}\,dt.
\end{equation*}
Dabei ist $x_e$ eine entlang der Flugrichtung skalierte Koordinate der
Äquivalentgeometrie, die von der Mach-Zahl und den Gasfaktoren,
wie dem Wärmekapazitätsverhältnis der Luft oder den Freiströmungsgrößen,
abhängt.
Die dimensionslose Nahfeld-Überdrucksignatur ist, bis auf bekannte
Mach-Zahl und den Gasfaktoren, proportional zu $F(x_e)$.

Intuitiv gilt: stark gekrümmte Änderungen in $A'(x)$ erzeugen starke lokale
Schockanteile; eine ``sanft'' verteilte Geometrie verteilt die Druckanstiege
längs der Rumpflänge.
Die $F$-Funktion fasst diese Information zu einer kompakten Nahfeldsignatur zusammen.
Die klassische Low-Boom-Minimierung nach Seebass-George-Darden entwirft
Zielverteilungen für $A(x)$, die eine geringe Schockstärke am Eintritt in
die Atmosphäre bewirken \cite{schall:darden75, schall:seebassgeorge}.

\subsection{Ausbreitung: Augmentierte Burgers-Gleichung}
Die Fernfeldausbreitung eines Überschalldrucksignals lässt sich bildhaft als
``fahrende Welle'' verstehen, die sich mit der Schallgeschwindigkeit vom
Flugzeug wegbewegt.
Zwei Prozesse formen diese Welle unterwegs entscheidend um:
Nichtlinearität schiebt energiereiche Signalanteile durch hohen
Druck nach vorn und lässt die Flanke zunehmend steiler werden;
die Dämpfung in der Atmosphäre nimmt den hohen Frequenzen die Spitze
\index{Dampfung@Dämpfung}%
und glättet die Form.
Zusätzlich nimmt die Amplitude mit wachsender Entfernung durch reine
Ausbreitungsverluste ab.
Die augmentierte Burgers-Gleichung fasst genau diese drei
Wirkungen in einem Modell zusammen: Steilung nach vorn, Glättung durch
Verluste, Abschwächung durch Ausbreitung.

Bei der Fernfeld-Übertragung durch eine ruhende, schichtweise homogene,
windstillen Atmosphäre wird die skalierte Schaltdrucksignatur $p(\tau,x)$
(hier $\tau$ = ``retardierte Zeit'') häufig mit einer
\emph{augmentierten Burgers-Gleichung}
\begin{equation}
  \underbrace{\frac{\partial p}{\partial x}
  +\frac{\beta}{\rho_0 c_0^{3}}\,p\,\frac{\partial p}{\partial \tau}}_{\text{Burgers-Ansatz}}
  =
  \underbrace{-\,\frac{m}{x}\,p \vphantom{\frac{\beta}{\rho_0 c_0^{3}}}
  +\delta\,\frac{\partial^2 p}{\partial \tau^2}}_{\text{Augmentierung}}
  \label{eq:aug-burgers}
\end{equation}
modelliert,
wobei $\frac{m}{x}$ die geometrische Abschwächung beschreibt, der nichtlineare
Term, welcher mit $\beta$ multipliziert wird, zur Intensität der Schockanteile
führt und $\delta$ frequenzabhängige atmosphärische Verluste aggregiert~\cite{schall:burgersJASA}.
Die ursprüngliche Form der Burgers-Gleichung wird im
Kapitel~\ref{chapter:neuronal} vertieft behandelt.

Die Nichtlinearität ``zieht'' die Front zu einem N-Profil.
Unter einem N-Profil versteht man einen zeitlichen Verlauf mit einem kurzen,
steilen Druckanstieg, dem Kompressionsstoss, und einem anschliessenden, deutlich
längeren, nahezu linearen Abfall bis unter das Ausgangsniveau, bevor der
Druck langsam wieder auf den Umgebungswert zurückkehrt und somit eine
N-förmige Kurve bildet.
Physikalisch liegt das daran, dass Kompressionsanteile sich in nichtlinearen
Medien schneller ausbreiten als Verdünnungsanteile: Der vordere,
hochamplitudige Teil des Signals holt den hinteren ein, die Front steilt
auf und es bildet sich ein Stoss.
Die Dämpfung in der Atmosphäre schwächt besonders hohe Frequenzen,
sodass scharfe Kanten abgerundet werden und der Spitzenüberdruck
\(\Delta p_\mathrm{max}\) sinkt.
Die Gesamtsamplitude nimmt zudem durch die geometrische Ausbreitung ab, da
die Energie auf eine immer grössere Bodenfläche verteilt wird
\cite{schall:rallabhandi2023}.

\subsection{Boden-Signatur und Metriken}
Die für Wahrnehmung massgebliche Metrik ist häufig der \textit{Perceived Level},
\index{Perceived Level}%
der die zeitliche Impulsform und psychoakustische Bewertungen kombiniert,
gemessen in PLdB. Für weitere Details wird auf \cite{schall:x59pldb} verwiesen.
Er wird nicht wie dB direkt aus dem Schalldruckpegel berechnet,
sondern aus der Boom-Drucksignatur über ein psychoakustisches Rechenverfahren,
das Frequenz- und Zeitbewertung enthält und so die menschliche
Wahrnehmung besser abbildet.
Ziel moderner Low-Boom-Demonstratoren ist ein ``sonic thump'' statt
eines klassischen Booms, mit einem Zielwert von
${\sim}75\,\text{PLdB}$ in Standardatmosphäre.
\index{PLdB}%

\subsection{Hebel zur Boom-Reduktion}
\subsubsection{Aerodynamische Formgebung}
Beim \emph{Low-Boom Shaping} geht es darum, die Gestaltung von $A(x)$ und
\index{Low-Boom Shaping}%
der Auftriebsverteilung entlang der Länge zu gestalten, dass
grosse Einzelschocks vermeiden und und mehrere kleine Sprünge verteilt werden.
Dies kann durch eine lange, schlanke Nase, durch Flügel- oder Canard-Anordnung,
verdeckte Triebwerksintegration oder durch Vermeidung starker
Querschnittssprünge realisiert werden.

Formal gilt: Minimierung geeigneter Funktionale der F-Funktion bzw.~der Nahfeldsignatur
gemäss Seebass-George-Darden-Ideen \cite{schall:seebassgeorge,schall:darden75}.

\subsubsection{Flugweg- und Operationsführung.}
Durch eine gezielte Wahl von Flugweg und -parametern lässt sich die
Bodenausbreitung des Überschallknalls spürbar steuern.
Dazu zählen unter anderem die Wahl von Flughöhe und Mach-Zahl, Steig-
und Sinkprofile sowie Kursführungen.
Wo möglich, wird der Überlandanteil im Unterschall geflogen und der
Übergang in den Überschall über See oder in grosser Höhe gelegt,
um den Boom-Teppich zu verkleinern.
Solche Betriebsstrategien reduzieren Spitzenpegel, verschieben Hot-Spots
und können die Lärmbelastung am Boden deutlich verringern.

\subsubsection{Flughöhe und Mach-Zahl.}
Eine höhere Flughöhe erhöht die geometrische Abschwächung, nachvollziehbar durch
grösseres $x$ in \eqref{eq:aug-burgers}, und die atmosphärische Dämpfung durch
grössere effektive $\delta$; moderate Mach-Zahl reduziert die erzeugten
Schockstärken bereits im Nahfeld.

\subsubsection{Atmosphärische Effekte berücksichtigen.}
Temperatur- und Wind-Schichtung $c(z)$, Windschubvektoren und Turbulenz
modulieren Divergenz und effektive Dämpfung in \eqref{eq:aug-burgers}.
In Inversionslagen kann es zu Fokussierung kommen, was lokal
zu erhöhten Druckspitzen führt (grösseres $\Delta p$).
Normale Temperaturlagen führen eher zu Defokussierung,
da der Schalldruckpegel über eine grössere Fläche verteilt wird,
was in Abschnitt~\ref{schall:subsection:atmos-scenarios} behandelt wurden.

\subsection{Zusammenfassung}
In diesem Kapitel wurden die physikalischen Grundlagen der
Schallausbreitung in Gasen, Flüssigkeiten und Festkörpern, sowie die
Entstehung und Ausbreitung des Überschallknalls in geschichteten
Atmosphären hergeleitet.
Unter den getroffenen Annahmen (kleine Störungen, geometrische Akustik,
stationäre Schichtung) konnten die Bahn der Mach-Ränder bestimmt,
Bedingungen für Bodentreffer und Schattenzonen formuliert und die Wirkung
typischer Atmosphärenprofile (normale Lage vs.~Inversion) quantifiziert werden.
Zudem wurden Hebel zur Reduktion des Bodenschalls identifiziert
(Low-Boom-Shaping, Flughöhe/Mach, Flugweg- und Operationsführung) und mit
wahrnehmungsbasierten Kenngrössen (PLdB) verknüpft.
Für Anwendungen in der Zivilluftfahrt wird auf die Bedeutung
realistischer Atmosphären (inkl.~Windschichtung/Turbulenz) und
nichtlinear-dissipativer Ausbreitungsmodelle
(augmentierte Burgers-Gleichung) verwiesen,
sowie auf die Validierung mit Messdaten, um die Akzeptanz künftiger
Überschalloperationen zu fördern.
Wer sich gezielt vertiefen möchte, findet Information zur Nahfeldtheorie
und $F$-Funktion bei Whitham \cite{schall:whitham},
zur Low-Boom-Formgebung nach Seebass–George–Darden \cite{schall:seebassgeorge,schall:darden75},
zur augmentierten Burgers-Gleichung und Ausbreitung \cite{schall:burgersJASA},
sowie zu wahrnehmungsbasierten Metriken (PLdB) und aktuellen
Demonstratoren \cite{schall:x59pldb} weiterführende Darstellungen.
