\section{Rückblick}
Die Geometrische Algebra bietet ein einheitliches mathematisches Framework für Berechnungen
in diversen Dimensionen. Anstelle mit Pseudovektoren zu arbeiten, definiert sie konisistente
Bivektoren und $n$-Vektoren höheren Grades. Die geometrische Algebra zeichnet sich besonders
darin aus, dass sie diverse Transformationen wie Spiegelungen, Rotationen und Translationen
einheitlich definiert und ausserordendlich einfache Wege bietet, diese umzusetzen.

Besonders Rotationen sind mit den herkömmlichen Rotationsmatrizen nicht ganz trivial, besonders wenn
nicht nur über eine Rotationsebene rotiert werden kann. Um die Rotation um beliebige Winkel in beliebiger
Richtung in drei Dimensionen zu ermöglichen, gibt es in der herkömmlichen Geometrie zwei Möglichkeiten. Entweder
man berechnet die Rotationsmatrix, die eine direkte Rotation ermöglicht, oder man arbeitet mit den drei bereits bekannten
Rotationsmatrizen um die $x, y, z$-Achsen. Dabei müssen bis zu drei
Rotationen an den drei Rotationsachsen $x, y, z$ nacheinander ausgeführt werden, wobei ein
Gimbal-Lock\footnote{\url{https://de.wikipedia.org/wiki/Gimbal_Lock}} auftreten kann. Dies ist ein Zustand, wo sich zwei der drei Rotationachsen genau gleich ausrichten und somit ein Freiheitsgrad verloren geht.
Beide Wege führen zu Berechnungen, die um ein Vielfaches
schwieriger durchzuführen sind und sind insbesondere bei Computergrafiken von Nachteil, da sie sowohl in Laufzeit,
als auch in Speichereffizienz wesentlich komplexer sind.

Die geometrische Algebra ermöglicht Transformationen ohne Drehmatrizen, ohne Gimbal-Lock und kann an einer beliebigen
Ebene durchgeführt werden. Berechnungen sind sehr laufzeit- und speichereffizient.
Sie definiert Transformationen einheitlich mithilfe des geometrischen Produkts.
