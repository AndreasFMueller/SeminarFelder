%
% vorwort.tex -- Vorwort zum Buch zum Seminar
%
% (c) 2019 Prof Dr Andreas Mueller, Hochschule Rapperswil
%
\chapter*{Vorwort}


Dieses Buch entstand im Rahmen des Mathematischen Seminars
im Frühjahrssemester 2025 an der Ostschweizer Fachhochschule in Rapperswil.
Die Teilnehmer, Studierende der Studiengänge für Elektrotechnik, Informatik,
Erneuerbare Energien und Umwelttechnik und Bauingenieurwesen
der OST, erarbeiteten nach einer Einführung in das Themengebiet jeweils
einzelne Aspekte des Gebietes in Form einer Seminararbeit, über
deren Resultate sie auch in einem Vortrag informierten. 

Im Frühjahr 2025 waren Felder das Thema des Seminars.
Der Anfängerstudent lernt in der Physik, dass sich das Newtonsche
Gravitationsgesetz als Gradient einer Funktion schreiben lässt.
Diese Idee präsentiert sich so zunächst eher als ein mathematischer
Trick.
Tatsächlich steckt jedoch viel mehr dahinter.
Physikalische Prinzipien schränken stark ein, welche Arten von
Differentialoperatoren überhaupt sinnvoll sind.
Umgekehrt können Lösungen von lokalen Feldproblemen Informationen
über die globale Topologie des Definitionsraumes liefern.
Im ersten Teil des Buches wird daher die mathematische Feldtheorie
systematisch entwickelt und die Verbindung zur klassischen
Vektoranalysis und zu bekannten Feldgleichungen der Physik entwickelt.

Im zweiten Teil erarbeiten die Teilnehmer des Seminars in individuellen
Arbeiten ein breites Spektrum von weiterführenden Fragestellungen
und Anwendungen.
In einigen Arbeiten wurde auch Code zur Demonstration der 
besprochenen Methoden und Resultate geschrieben, soweit
möglich und sinnvoll wurde dieser Code im Github-Repository
\index{Github-Repository}%
dieses Kurses%
\footnote{\url{https://github.com/AndreasFMueller/SeminarFelder.git}}
\cite{buch:repo}
abgelegt.
Im genannten Repository findet sich auch der Source-Code dieses
Skriptes, es wird hier unter einer Creative Commons Lizenz
zur Verfügung gestellt.

