%
% slugModell.tex -- Erleutert das Slug-Modell
%
% !TEX root = ../../buch.tex
% !TEX encoding = UTF-8
%
\section{Slug-Modell}
\kopfrechts{Slug-Modell}%
In diesem Kapitel nehmen wir das \emph{Slug-Modell} genauer unter die Lupe, um die Entstehung eines Wirbelrings besser zu verstehen.
\index{Slug-Modell}%
Die genaue mathematische Beschreibung des Entstehungsprozesses ist sehr komplex und würde den Rahmen dieses Kapitels sprengen.
Dafür kann man sich diesen Prozess mit Hilfe des Slug-Modells besser vorstellen.
In diesem Kapitel stellen wir zusätzlich eine eigene Formel vor, welche die Zirkulation eines Wirbelrings beschreibt.

\subsection{Grundlegende Idee}
Im Slug-Modell wird der Impulsaustritt eines Fluidvolumens (eines Fluid-Slugs\footnote{Im Deutschen: Fluid-Pfropfen}) aus einer Öffnung beschrieben.
\index{Fluid-Slug}%
Man kann sich dies als Zylinder eines Fluids vorstellen, welcher innerhalb kürzester Zeit aus einer Öffnung geschoben wird wie in Abbildung \ref{Wirbelringe:fig:slug_formung} gezeigt.
Dabei definieren wir folgende Grössen:
\begin{itemize}
    \item Zirkulation \(\Gamma\)
    \item Wirbelstärke \(\vec{\omega}\)
    \item Slug-Geschwindigkeit \(u_p(t)\)
    \item Slug-Durchmesser \(D\)
    \item Slug-Länge \(L\)
    \item Wirbelring-Querschnittsradius \(r_w\)
    \item Ausstosszeit \(T\)
\index{Ausstosszeit}%
\end{itemize}

\begin{figure}
\centering
\def\svgwidth{0.9\columnwidth}
\import{papers/wirbelringe/fig/}{slug_formung.eps_tex}
\caption{Formung eines Wirbelrings mit Krenngrössen des Slug Modells eingezeichnet. \label{buch:papers:Wirbelringe:fig:slug_formung}}
\end{figure}

Tritt nun das Slug aus dem Zylinder aus, so trifft die Aussenseite des Slugs auf das stehende Fluid (bspw. Luft oder Wasser) ausserhalb.
Dies bewirkt aufgrund der Geschwindigkeitsdifferenz zwischen dem Inneren des Slugs und dem stehenden Fluid aussen eine Scherung.
Die Scherung hat zur Folge, dass sich das Slug „aufzurollen“ beginnt und somit einen Wirbel formt.
Die in Abschnitt \ref{Wirbelringe:Stokes} genannte Formel für die Zirkulation kann näherungsweise auf
\begin{equation*}
\Gamma 
\approx 
\frac{1}{2}u_pL
\end{equation*}
vereinfacht werden.

\subsection{Berechnungen zum Slug-Modell}
Alternativ kann man auch über den Impulsvergleich auf die Formel \eqref{Wirbelringe:eq:naeherungZirkulation} für die Zirkulation kommen.
Dies funktioniert aber nur unter folgenden Annahmen:
\begin{itemize}
    \item Das verwendete Fluid ist inkompressibel.
    \item Der linear ausgestossene Impuls des Slugs wird in den Drehimpuls des Wirbelrings überführt.
\index{Drehimpuls}%
    \item Die Öffnungsfläche ist mit \(A = \frac{\pi}{4} D^2\) angenähert.
    \item \(\vec{\omega}\) ist tangential zum Wirbelring.
    \item Der entstehende Wirbelring ist achsensymmetrisch.
\end{itemize} 

Grundsätzlich wird die Länge und das Volumen eines Slugs als
\begin{align}
    \label{Wirbelringe:eq:sluglaenge}
    L
    &=
    \int_{0}^{T}u_p(t)\,dt\\
    V
    &=
    LA
\end{align}
definiert.
Es kann aber angenommen werden, dass das Slug mit konstanter Geschwindigkeit \(u_0\) bewegt wird, weshalb \eqref{Wirbelringe:eq:sluglaenge} zu
\begin{equation}
    L
    =
    u_0T
\end{equation}
wird.
Nun kann noch der lineare Impuls eines zylindrischen Slugs als
\begin{align}
    I_{\text{slug}}
    &=
    m_{\text{slug}}u_0\\
    \intertext{definiert werden. Da die Masse}
    m_{\text{slug}}
    &=
    \rho AL\\
    \intertext{ist, folgt}
    I_{\text{slug}}
    &=
    \rho ALu_0
    \label{Wirbelringe:eq:slugImp}.
\end{align}
Der Drehimpuls
\begin{equation*}
    \vec{I}
    =
    \int_{V}\rho(\vec{x}\,)\vec{x}\times\vec{v}(\vec{x}\,)\,d\mathbf{V}
\end{equation*}
eines kontinuierlichen Mediums
kann durch Einsetzen der Vektor-Identität für Rotationstensoren \cite{Wirbelringe:batchelor1967}
\begin{equation*}
    \vec{x}\times\vec{v}
    =
    \frac{1}{2}\int_{V}\vec{x}\times(\nabla\times\vec{v}\,)\,d\mathbf{V}
    =
    \frac{1}{2}\vec{x}\times\vec{\omega}
\end{equation*}
als Drehimpuls eines rotierenden Fluidvolumens 
\begin{equation}
    \vec{I}
    =
    \int_{V}\rho\cdot\vec{x}\times\vec{v}\,d\mathbf{V}
    =
    \int_{V}\rho\cdot\biggl({\textstyle \frac{1}{2}}\vec{x}\times\vec{\omega}\biggr)\,d\mathbf{V}
    =
    \frac{\rho}{2}\int_{V}\vec{x}\times\vec{\omega}\,d\mathbf{V}
    \label{Wirbelringe:eq:Drehimpuls}
\end{equation}
hergeleitet werden. 

Wie bereits erwähnt, ist \(\vec{\omega}\) tangential zum Wirbelring und achsensymmetrisch.
Ausserdem verläuft die Drehimpulsrichtung entlang der Symmetrieachse.
Um der gewünschten Formel näherzukommen, wird \eqref{Wirbelringe:eq:Drehimpuls} in das Zylinderkoordinatensystem umgewandelt.
Dabei wird
\begin{equation*}
    \vec{x}
    =
    r\hat{e}_r + z\hat{e}_z
\end{equation*}
und
\begin{equation*}
    \vec{\omega}
    =
    \omega_\varphi(r,z)\hat{e}_\varphi
\end{equation*}
wobei das Kreuzprodukt der beiden 
\begin{equation*}
    \vec{x}\times\vec{\omega}
    =
    r\omega_\varphi\hat{e}_z - z\omega_\varphi\hat{e}_r
\end{equation*}
ergibt. 
Da der Wirbelring achsensymmetrisch ist, wird nur die \(\hat{e}_z\)-Komponente betrachtet.
Als Nächstes können die gefundenen Werte für \(\vec{x}\) und \(\vec{\omega}\) in \eqref{Wirbelringe:eq:Drehimpuls} eingesetzt werden.
\(I_z\) wird dann zu
\begin{equation*}
    I_z
    =
    \frac{\rho}{2}\int_{0}^{2\pi}\int_{A}r\omega_\varphi(r,z)\cdot r\,d\varphi \,dr\,dz.
\end{equation*}
Da keine Integrationsvariable von \(\varphi\) abhängt, kann das erste Integral sofort gelöst werden. 
Nach Einsetzen der Integrationsgrenzen ergibt sich
\begin{equation*}
    I_z
    =
    \frac{\rho}{2}2\pi\int_{A}r^2\omega_\varphi(r,z)\,dr\,dz.
\end{equation*}
Die 2 kürzt sich heraus und die beiden \(r\) können zu \(r^2\) zusammengefasst werden, was zu
\begin{equation}
    I_z
    =
    \pi\rho\int_{A}r^2\vec{\omega}(r,z)\,dr\,dz
    \label{Wirbelringe:eq:achssymImp}
\end{equation}
führt, womit sich der Impuls eines Wirbelrings ermitteln lässt.

Da die Impulserhaltung gilt, ist \(I_z\) von \eqref{Wirbelringe:eq:achssymImp} ungefähr gleich gross wie \(I_{\text{slug}}\) von \eqref{Wirbelringe:eq:slugImp}, also
\begin{equation}
    \rho\pi\int_{A}r^2\vec{\omega}(r,z)\,dr\,dz
    \approx
    \rho ALu_0
    \label{Wirbelringe:eq:Igleichi}.
\end{equation}
Da die Zirkulation gemäss \eqref{Wirbelringe:eq:Zirkulation} und \ref{Wirbelringe:Stokes} in
\begin{equation*}
    \Gamma
    =
    \int_{A}\vec{\omega}(r,z)\,dr\,dz
\end{equation*}
umgewandelt werden kann, ergibt sich mittels Einsetzen in Gleichung \eqref{Wirbelringe:eq:Igleichi} die kompakte Formel
\begin{equation*}
    \Gamma
    =
    \frac{2u_0AL}{\pi D^2}.
\end{equation*}
Setzt man dann noch
\begin{equation*}
    A
    =
    \frac{\pi}{4} D^2
\end{equation*}
ein, erhält man
\begin{equation}
    \label{Wirbelringe:eq:naeherungZirkulation}
    \Gamma
    =
    u_0L\frac{r_w^2}{2}
\end{equation}
zur Berechnung der Zirkulation.

\textit{Hinweis:}
Die Formel ist dimensionsrichtig und physikalisch motiviert, jedoch nicht durch experimentelle oder numerische Studien bestätigt.
Daher ist sie als \emph{heuristische Näherung} zu interpretieren, insbesondere für qualitative Abschätzungen oder geometrisch skalierte Vergleiche.
