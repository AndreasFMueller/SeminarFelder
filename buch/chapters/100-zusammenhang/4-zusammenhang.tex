%
% 4-zusammenhang.tex
%
% (c) 2025 Prof Dr Andreas Müller
%

%
% Zusammenhang
%
\section{Zusammenhang
\label{buch:zusammenhang:section:zusammenhang}}
\kopfrechts{Zusammenhang}
In der bisherigen Entwicklung wurde die kovariante Ableitung vor allem
aus der Darstellung~\eqref{buch:zusammenhang:paralleltransport:eqn:kovabl}
durch Berechnung der Christoffel-Symbole und das Studium ihrer
Eigenschaften entwickelt.
Damit lassen sich weitere Abstraktionen vermeiden.
Es ist aber auch möglich, die kovariante Ableitung als ein Operator
mit geeigneten Eigenschaften axiomatisch zu beschreiben und dann zu
zeigen, dass es genau einen solchen Operator gibt.
Dies verlangt etwas abstraktere Konstruktionen, ermöglicht aber
die einfachere Verallgmeinerung auf beliebige Tensorfelder.
Diese Vorgehensweise soll in diesem Abschnitt durchgeführt werden.

Die Verallgemeinerung ermöglicht ausserdem, beliebige Arten von
vektorwertigen Funktionen abzuleiten, nicht nur Vektoren, die sich
unmittelbar aus Tangentialvektoren konstruieren lassen.
Die Theorie der $p$-Formen ging jeweils davon aus, dass die
Werte der $p$-Formen reelle Zahlen sind.
Setzt man einen $p$-Vektor in eine solche Form ein, entsteht ein
reellwertige Funktion, die sich beliebig ableiten lässt, weil
sich Werte an nahe beeinander liegenden Punkten problemlos
vergleichen lassen.
Damit dies auch möglich wird, wenn die Werte in einem Vektorraum
liegen, der von Punkt zu Punkt der Mannigfaltigkeit ändert, muss
die kovariante Ableitung zum Vergleich hinzugezogen werden.

%
% Vektorraumbündel
%
\subsection{Vektorraumbündel}
Das Tangentialbündel $TM$ einer differenzierbaren Mannigfaltigkeit
bestand für jeden Punkt $p\in M$ aus einem Vektorraum $T_pM$, dessen
Elemente wir mit Differentialoperatoren identifiziert haben, die
auf Funktionen auf der Mannigfaltigkeit wirken.
Diese letzte Eigenschaft ist nur für Tangentialvektoren sinnvoll.
Um zum Beispiel ein Kraftfeld auf einer Mannigfaltigkeit zu beschreiben,
möchten wir ebenfalls jedem Punkt der Mannigfaltigkeit Vektoren
zuordnen können, aber diese Vektoren haben nichts mit den Funktionen
auf der Mannigfaltigkeit zu tun.
Wir müssen also das Konzept des Tangentialbündels noch etwas
erweitern, so dass in jedem Punkt Vektoren aus einem beliebigen
Vektorraum ausgewählt werden können.

Auch aus rein mathematischer Motivation ist es nötig, die Idee des
Tangentialbündels zu erweitern.
Zum Beispiel bilden die $k$-Formen in einem Punkt einen
$\binom{n}{k}$-dimensionalen, reellen Vektorraum.
Eine Verallgemeinerung des Konzepts des Tangentialbündels ist also
bereits aufgetreten, wir haben es nur nicht abstrakt definiert.
Dies soll jetzt nachgeholt werden.

Das Tangentialbündel ist nicht einfach nur eine Menge von Vektoren.
In jedem Punkt $p\in M$ gibt es den Vektorraum $T_pM$ der Tangentialvektoren.
Es gibt aber auch ein Konzept der Stetigkeit für Tangentialvektorfelder.
Ein solches ist eine Funktion $X\colon M\mapsto TM$ derart, dass 
$X(p)\in T_pM$ ist.
Ausserdem sollen Vektoren $X(p)$ und $X(q)$ in zwei Punkten $p,q\in M$,
die nahe beeinander liegen, ebenfalls nahe beeinander sein.
Weiter möchten wir mit Koordinaten arbeiten können, so wie das für $M$
selbst auch möglich ist.
Schliesslich möchten wir die Vektorraumstruktur in verschiedenen
Punkten miteinander vergleichen können.
Für das Tangentialbündel ist jeder Tangentialraum $T_pM$ ein
$n$-dimensionaler reeller Vektorraum, also $T_pM \cong \mathbb{R}^n$.
Alle diese Forderungen führen auf die folgende Definition.

\begin{definition}[Vektorraumbündel]
Sei $M$ eine differenzierbare Mannigfaltigkeit und $V$ ein reeller
Vektorraum.
Ein \emph{differenzierbares Vektorraumbündel} über $M$ mit Faser $V$
ist eine differenzierbare Mannigfaltigkeit $E$ mit einer Abbildung
$\pi\colon E\to M$ mit folgenden Eigenschaften.
Jeder Punkt $p\in M$ hat eine offene Umgebung $U\subset M$ mit einem
Diffeomorphismus
\[
\varphi\colon \pi^{-1}(U) \to U \times V
\]
derart, dass $\operatorname{pr}_1\circ\varphi =\pi$, d.~h.
$\varphi(e)$ kann in der Form $\varphi(e) = (\pi(e), v)$ mit $v\in V$
geschrieben werden.
Eine solche Abbildung $\varphi$ heisst eine \emph{Bündelkarte}.
Für zwei Bündelkarten $\varphi_\alpha$ und $\varphi_\beta$ mit
Kartengebieten $U_\alpha$ und $U_\beta$ ist die Kartenwechselabbildung
\[
\varphi_\beta
\circ
\varphi_\alpha^{-1}
\colon
U_\alpha\cap U_\beta \times V
\to
U_\alpha\cap U_\beta \times V
.
\]
$E$ heisst \emph{Totalraum} des Bündels, $M$ heisst die \emph{Basis}
und $\pi$ ist die \emph{Bündelprojektion}.
Die Punkte von $E$, die auf den Punkt $p\in M$ abgebildet werden,
heisst die \emph{Faser} $E_p=\pi^{-1}(p)\subset E$ über dem Punkt $p$.
\end{definition}

\begin{beispiel}
Zu jeder Mannigfaltigkeit $M$ und jedem endlichdimensionalen
reellen Vektorraum $V$ gibt es das Bündel $M\times V$ mit
der Bündelprojektion
\[
\operatorname{pr}_1
\colon
M\times V \to M
:
(p,v) \mapsto p.
\]
Dieses Vektorraumbündel über $M$ heisst das \emph{triviale Bündel}
mit Faser $V$.
\end{beispiel}

\begin{beispiel}
\label{buch:zusammenhang:kovabl:beispiel:0d}
Die identische Abbildung $M\to M:p\mapsto p$ ist ein Vektorraumbündel
mit dem 0-dimensionalen Vektorraum $\{0\}$ als Faser.
\end{beispiel}

Wenn man Mannigfaltigkeiten miteinander vergleichen will, konstruiert
man Abbildungen zwischen diesen Mannigfaltigkeiten.
Da Vektorraumbündel aus zwei Mannigfaltigkeiten bestehen, die durch
die Bündelprojektion $\pi$ miteinander verbunden sind, müssen
Bündelabbildungen aus Abbildungen der Totalräume und der
Basismannigfaltigkeiten zusammengesetzt sein.
Ausserdem muss innerhalb der Fasern die Vektorraumstruktur zum
Ausdruck kommen.
Die folgende Definition erreicht dies.

\begin{definition}[Bündelabbildung]
Seien $\pi\colon E \to M$ und $\psi\colon F\to N$ zwei Vektorraumbündel
mit Fasern $V$ und $W$.
Eine \emph{Bündelabbildung} ist eine Abbildung $f\colon E\to F$ und
eine Abbildung $g\colon M\to N$ derart, dass
dass es eine Abbildung $g\colon M\to N$ gibt, derart dass 
$g\circ\pi=\psi\circ f$.
Ausserdem ist für jeden Punkt $p\in M$ die Einschränkung von $f$ auf
$\pi^{-1}(p)$ eine lineare Abbildung
\[
f_{|\pi^{-1}(p)}
\colon
\pi^{-1}(p)
\to
\pi^{-1}(g(p)).
\]
Sie heisst die \emph{Faserabbildung}.
\end{definition}

\begin{beispiel}
Beispiel~\ref{buch:zusammenhang:kovabl:beispiel:0d} hat gezeigt, dass
man die Mannigfaltigkeit $M$ als ein Vektorraumbündel mit Faser $\{0\}$
betrachten kann.
Die Definition einer Bündelabbildung $M\to E$ besagt dann, dass 
$f$ jedem Punkt $p$ genau ein Vektor in der Faser $E_p$ zuordnet.
\end{beispiel}

\begin{definition}
Sei $\pi\colon E\to M$ ein Vektorraumbündel.
Eine Bündelabbildung $f\colon M\to E$ hat die Eigenschaft
$\pi\circ f=\operatorname{id}_M$ und heisst ein \emph{Schnitt}
des Bündels.
Die Menge der Schnitte eines Vektorraumbündels wird mit $\Gamma(E)$
bezeichnet.
\end{definition}

In einer Bündelkarte $\varphi_\alpha:\pi^{-1}(U_\alpha) \to U_\alpha\times V$
ist ein Schnitt der Graph einer Funktion $U_\alpha\to V$.
Die Abbildung
\[
0
\colon
M\to E
:
x \mapsto 0_x\in E_x=\pi^{-1}(x),
\]
die einem Punkt der Mannigfaltigkeit den Nullvektor in der Faser
$\pi^{-1}(x)=E_x \subset E$
an dieser Stelle zuordnet, ist ein Schnitt von $E$, er heisst
der \emph{Nullschnitt}.
\index{Nullschnitt}%
Dieser Schnitt existiert immer.
Es lassen sich auch immer innerhalb eines Kartengebietes 
Schnitte definieren, die man dann mit einer Abschneidefunktion
$s\colon M\to\mathbb{R}$, deren Träger im Kartengebiet enthalten ist,
durch $0$ auf die ganze Mannigfaltigkeit fortsetzen kann.

Es ist aber nicht selbstverständlich, dass es in einem Vektorraumbündel
einen Schnitt gibt, der in \emph{keinem} Punkt von $M$ verschwindet.
Als Beispiel kann das Bündel $\bigwedge^mTM$ auf einer $m$-dimensionalen
Mannigfaltigkeit dienen.
Es ist eindimensional.
Wenn es einen nicht verschwindenden Schnitt hat, dann lässt sich daraus
eine Orientierung der Mannigfaltigkeit konstruieren
(Definition~\ref{buch:green:def:orientierung}).
Auf einer nicht orientierten Mannigfaltigkeit hat also das Vektorraumbündel
$E=\bigwedge^m TM$ keinen nichtverschwindenden Schnitt.


%
% Zusammenhang
%
\subsection{Zusammenhang}
Der Begriff des Zusammenhangs soll die Eigenschaften einer Ableitung 
eines Vektors entlang einer Kurve in einer Mannigfaltigkeit beschreiben.
Wir brauchen daher ein Vektorraumbündel $\pi\colon E\to M$, welches
die Vektoren enthält, die abgeleitet werden sollen.
Ein differenzierbarer Schnitt von $E$ ist eine vektorwertige
Funktion, die Werte in der jeweils ``richtigen'' Faser annimmt.
Es müssen also Schnitte abgeleitet werden.
Die Richtung der Ableitung wird durch ein Feld von Tangentialvektoren
an die Mannigfaltigkeit gegeben.

\begin{definition}
Ein \emph{Zusammenhang} ist eine Abbildung
\index{Zusammenhang}%
\[
\nabla 
\colon
\Gamma(TM)\times \Gamma(E) \to \Gamma(E)
:
(X,Y) \mapsto \nabla_X Y
\]
mit folgenden Eigenschaften:
\begin{enumerate}
\item $\nabla_X Y$ ist $C^\infty(M)$-linear in $X$, d.~h. für 
$f_1,f_2\in C^{\infty}(M)$, $X_1,X_2\in \Gamma(TM)$ und $Y\in\Gamma(E)$ gilt
\[
\nabla_{f_1X_1+f_2X_2}Y
=
f_1\nabla_{X_1}Y + f_2\nabla_{X_2}F.
\]
\item $\nabla_X Y$ ist linear in $Y$, d.~h. für $a_1,a_2\in\mathbb{R}$
und $Y_1,Y_2\in \Gamma(E)$ gilt
\[
\nabla_X(a_1Y_1+a_2Y_2)
=
a_1\nabla_XY_1
+
a_2\nabla_XY_2
\]
für jedes $X\in\Gamma(TM)$.
\item
$\nabla$ erfüllt die Produktregel: für $f\in C^\infty(M)$ gilt
\index{Produktregel fur Zusammenhang@Produktregel für Zusammenhang}%
\[
\nabla_X (fY)
=
(X\cdot f)Y + f\nabla_X Y
\]
für $X\in\Gamma(TM)$ und $Y\in\Gamma(E)$.
\end{enumerate}
Der Wert $\nabla_XY$ heisst auch die kovariante Ableitung von $Y$
in Richtung von $X$.
\end{definition}


Die Produktregel-Bedingung hat die interessante Konsequenz, dass der
Wert von $\nabla_XY$ in einem Punkt $p\in M$ nur vom Vektor $X(p)$ und
von den Werten von $Y$ in einer beliebig kleinen Umgebung des Punktes
$p$ in $M$ abhängt.
Wir beweisen dies nicht vollständig, sondern überlegen uns nur die
folgende Eigenschaft, mit deren Hilfe ein volständiger Beweis geführt
werden kann \cite[Proposition 4.5]{buch:leerm}.

\begin{lemma}
Sei $\nabla$ ein Zusammenhang im Vektorraumbündel $\pi\colon E\to M$,
$X\in \Gamma(TM)$, $Y\in \Gamma(E)$ und $p\in M$.
Dann hängt die kovariante Ableitung $\nabla_XY$ im Punkt $p$ nur von
den Werten von $X$ und $Y$ in einer beliebig kleinen Umgebung von $p$
ab.
Genauer: wenn $X=\tilde{X}$ und $Y=\tilde{Y}$ in einer Umgebung von
$p$ ist, dann ist $\nabla_{\tilde{X}}\tilde{Y} = \nabla_XY$ im Punkt $p$.
\end{lemma}

\begin{proof}
Wir müssen zeigen, dass $\nabla_X(Y-\tilde{Y})=0$ ist im Punkt $p$.
Die Differenz $Y-\tilde{Y}$ ist $=0$ in einer Umgebung von $p$.
Es genügt also zu zeigen, dass $\nabla_XY=0$ im Punkt $p$, wenn $Y=0$
ist in einer Umgebung $U$ des Punktes $p$.
Sei $\varphi\in C^\infty(M)$ derart, dass $\varphi(p)=1$ und 
$\varphi$ ist verschieden von $0$ nur innerhalb von $U$.
Dann ist das Produkt $\varphi Y=0$, innerhalb von $U$ weil $Y$ verschwindet
und ausserhalb von $U$ weil dort $\varphi$ verschwindet.
Wendet man $\nabla_X$ darauf an, folgt aus den Rechenregeln:
\[
0
=
\nabla X(\varphi Y)
=
(X\cdot \varphi)Y
+
\varphi\nabla_X Y
\qquad\Rightarrow\qquad
\varphi\nabla_X Y
=
-(X\cdot\varphi)Y.
\]
An der Stelle $p$ wird die linke Seite zu $\nabla_XY$ und die rechte
Seite verschwindet.
Es folgt, dass $\nabla_XY=0$ ist im Punkt $p$.

Ähnlich müssen wir jetzt auch noch zeigen, dass $\nabla_XY=0$ ist,
wenn $X=0$ ist in einer Umgebung von $p$.
Für eine Funktion $\varphi\in C^\infty(M)$ wie im ersten Teil des Beweises
folgt diesmal
\[
0
=
\nabla_{\varphi X}Y
=
\varphi\nabla_XY.
\]
Im Punkt $p$ ist $\varphi(p)=1$ und damit folgt für die rechte Seite
$\nabla_XY=0$.
\end{proof}

%
% Tensorbündel
%
\subsection{Tensorbündel}
Aus einem Vektorraumbündel $\pi\colon E\to M$ lässt sich ein neues
Vektorraumbündel konstruieren, dessen Faser beliebige Tensorprodukte
der Fasern des ursprünglichen Bündels sind.
Dazu muss nur gezeigt werden, wie lokal im Bündel Tensorprodukte
konstruiert werden sollen.
Wir bezeichnen daher den Faserraum wieder mit $V$ und betrachten
lokal das Bündel als triviales Bündel mit Hilfe einer Bündelkarte
\[
\varphi_\alpha
\colon
\pi^{-1}(U_\alpha)
\to
U_\alpha\times V 
\]
beschreiben können.

Zur Beschreibung des Tensorproduktes verwenden wir die Darstellung
mit Hilfe einer Basis.
Sei also $v_1,\dots,v_m\in V$ eine Basis des Faservektorraums.
Dann ist das $p$-fache Tensorprodukt von $V$ der Vektorraum aufgespannt
von der Basis aus Tensoren der Form
\[
v_{i_1}\otimes \cdots\otimes v_{i_p}
\]
mit $i_1,\dots,i_p\in \{1,\dots,m\}$.
Der Vektorraum des $p$-fachen Tensorproduktes ist also
\[
V^{p\otimes}
=
\underbrace{ V\otimes\cdots\otimes V }_{\displaystyle p}
=
\langle
v_{i_1}\otimes\cdots\otimes v_{i_p}
\mid
1\le
i_1,\dots,i_p 
\le m
\rangle.
\]
Durch Auswahl nur der antisymmetrischen Tensoren lässt sich auch
der Vektorraum der $k$-Vektoren aufgespannt von den Vektoren
\[
v_{i_1}\wedge\cdots\wedge v_{i_p}
\]
mit $1\le i_1 < \dots < i_p\le m$.
So entsteht der Vektorraum 
\[
V^{\wedge p}
=
\underbrace{V\wedge\dots\wedge V}_{\displaystyle p}
=
\langle
v_{i_1}\wedge\cdots\wedge v_{i_p}
\mid
1\le i_1 < \dots < i_p\le m
\rangle
\]
der $k$-Vektoren.

Diese Konstruktion kann jetzt auch für ein Vektorraumbündel
durchgeführt werden.
Dazu wählt man wie vorhin eine Basis in $V$. 
Die Abbildungen
\[
V_i
\colon
U_\alpha \to U_\alpha\times V
:
p\mapsto (p,v_i)
\]
sind $C^\infty(U_\alpha)$-linear unabhängige Schnitte des trivialen
Bündels $U_\alpha\times V$.
Zusammensetzung mit der Abbildung $\varphi_\alpha^{-1}$ macht aus den
Schnitten $V_i$ die $C^\infty(M)$-linear unabhängigen Schnitte
\[
E_i
\colon
U_\alpha \to \pi^{-1}(U_\alpha)
:
p \mapsto \varphi_\alpha^{-1}(p,v_i).
\]
Das $p$-fache Tensorprodukt $E^{p\otimes}=E\otimes\dots\otimes E$ von $E$ 
ist das Vektorraumbündel mit Faser $V^{p\otimes}$, welches lokal über der
offenen Menge $U_\alpha\subset M$ durch die Schnitte 
\[
E_{i_1}\otimes \dots \otimes E_{i_p}
\]
mit
$i_1,\dots,i_p\in\{1,\dots,m\}$
aufgespannt wird.

Nach dem gleichen Muster können auch Vektorraumbündel von
antisymmetrischen $k$-Vektoren konstruiert werden.
Ebenso ist es möglich, die verschiedenen Tensorpotenz oder
äusseren Potenzen mit Hilfe der direkten Summe zu einer Algebra
von Tensoren oder einer äusseren Algebra zusammenzubauen.

Zum Vektorraum $V$ ist $V^*$ der duale Vektorraum der Linearformen
auf $V$, also der linearen Abbildung von $V$ nach $\mathbb{R}$.
Zu einer Basis $\{v_i\mid i=1,\dots,m\}$ von $V$ lässt sich dann
die {\em duale Basis} aus den Linearformen $\beta_i$ konstruieren, die
\index{duale Basis}%
durch die Werte
\[
\langle \beta_i,v_k\rangle
=
\beta_i (v_k)
=
\delta_{ik}
\]
auf den Basisvektoren definiert ist.
Damit lässt sich auch das Vektorraumbünden $E^*$ der Linearformen
auf $E$ konstruieren.
Wendet man die Konstruktion des Tensorprodukts oder des Wedge-Produkts
auf das Bündel $E^*$ an, entsteht das Bündel der 
$p$-Formen.

Die Konstruktionen, die in früheren Kapiteln für Tangentialvektoren
und Differentialformen aufgebaut wurden, sind also auf 
Vektorraumbündel mit beliebigen Fasern verallgemeinerungsfähig.
Die früheren Konstruktionen sind Spezialfälle für das Tangentialbündel
$TM$.

%
% Kovariante Ableitung von Tensorfeldern
%
\subsection{Kovariante Ableitung von Tensorfeldern}
Ein Zusammenhang auf einem Vektorraumbündel $\pi\colon E\to M$ lässt
sich zu einem Zusammenhang auf dem Tensorbündel erweitern.
Da ein Zusammenhang linear ist, muss nur definiert werden, wie
der Zusammenhang auf Tensorprodukten von Vektoren wirkt.
Dies führt auf den folgenden Satz.

\begin{satz}
Sei $\pi\colon E\to M$ eine Vektorraumbündel und 
$\nabla \colon TM\times \Gamma(E)\to\Gamma(E)$
ein Zusammenhang.
Dann gibt es genau einen Zusammenhang
$\nabla \colon TM \times \Gamma(E^{\otimes p})\to\Gamma(E^{\otimes p})$
auf $E^{\otimes p}$, der
\[
\nabla_X(Y_1\otimes\dots\otimes Y_p)
=
(\nabla_XY_1) \otimes\dots\otimes Y_p
+
\dots
+
Y_1\otimes\dots\otimes (\nabla_XY_p)
\]
erfüllt.
Es gibt genau einen Zusammenhang auf $E^{\wedge p}$, der
\[
\nabla_X(Y_1\wedge\dots\wedge Y_p)
=
(\nabla_XY_1)\wedge\dots\wedge Y_p
+
\dots
+
Y_1\wedge\dots\wedge(\nabla_XY_p)
\]
erfüllt.
\end{satz}

Auch auf dem dualen Vektorraumbündel $E^*$ lässt sich ein Zusammenhang
definieren.
Dazu beachte man, dass sich aus einer Linearform $\beta\in\Gamma(E^*)$
und einem Vektorfeld $Y\in \Gamma(E)$ die Funktion
$\langle\beta,Y\rangle$ definieren lässt.
Für Funktionen stimmt die kovariante Ableitung mit der Wirkung des
Vektors $X$ überein.
Da die Abbildung $(\beta,Y)\to\langle\beta,Y\rangle$ bilinear ist,
muss die Ableitung die Form einer Produktregel annehmen, also
\[
\nabla_X\langle \beta,Y\rangle
=
X\cdot \langle \beta,Y\rangle
=
\langle \nabla_X\beta,Y\rangle
+
\langle \beta,\nabla_X Y\rangle.
\]
Aufgelöst nach $\nabla_X\beta$ wird dies zu
\begin{equation}
\langle \nabla_X\beta,Y\rangle
=
X\cdot \langle \beta,Y\rangle
-
\langle \beta,\nabla_X Y\rangle.
\label{buch:zusammenhang:kovarianteableitung:eqn:dual}
\end{equation}
Indem man für $Y$ die Schnitte einer Basis einsetzt, 
legt
\eqref{buch:zusammenhang:kovarianteableitung:eqn:dual}
die Werte der Linearform $\nabla_X\beta$ auf allen Basisvektoren
fest.

Wir verzichten für die Konstruktion der kovarianten Ableitung auf $E^*$
auf einen formellen Beweis und beschränken uns darauf, eine Schwierigkeit
der Definition \eqref{buch:zusammenhang:kovarianteableitung:eqn:dual}
zu adressieren.
Der Wert von $\nabla_X\beta$ im Punkt $p\in M$ ist festgelegt durch
die Werte auf den Vektoren $Y(p)$.
Multipliziert man $Y$ mit einer Funktion, die an der Stelle $p$ den Wert
$1$ hat, dann ändert sich der Wert auf der linken Seite nicht, aber
auf der rechten Seite treten zusätzlich Ableitungen von $f$ auf.
Sei also $f$ eine Funktion auf $M$ mit $f(p)=1$.
Setzt man $fY$ anstelle von $Y$ in die Definition
\eqref{buch:koordinaten:tangentialvektoren:def:einsteinschesummenkonvention}
ein, erhält man
\begin{align*}
\langle
\nabla_X \beta,fY
\rangle
&=
X\cdot\langle \beta,fY\rangle
-
\langle \beta,\nabla_X(fY)\rangle
\\
&=
X\cdot(f\langle \beta,Y\rangle)
-
\langle \beta,
(X\cdot f)Y
+
f\nabla_XY
\rangle
\\
&=
(X\cdot f)\langle \beta,Y\rangle
+
fX\cdot \langle \beta,Y\rangle
-
\langle \beta,
(X\cdot f)Y
\rangle
-
\langle \beta,
f\nabla_XY
\rangle
\\
&=
(X\cdot f)
\langle \beta,Y\rangle
+
f X\cdot \langle \beta,Y\rangle
-
(X\cdot f) \langle \beta, Y \rangle
-
f\langle \beta, \nabla_XY \rangle
\\
&=
f X\cdot \langle \beta,Y\rangle
-
f\langle \beta, \nabla_XY \rangle
\\
&=
f
\langle \nabla_X\beta,Y\rangle.
\end{align*}
Der Faktor $f$ hat also keinen Einfluss auf den Wert von
\eqref{buch:zusammenhang:kovarianteableitung:eqn:dual} an der Stelle $p$.

%
% Koordinatendarstellung
%
\subsection{Koordinatendarstellung}
Aus den Axiomen eines Zusammenhangs lässt sich auch herleiten, wie
der Zusammenhang in Koordinaten formuliert werden muss.
Um diese Formulierung zu finden, reduzieren wir wie üblich mit Hilfe
einer Karte auf eine offene Menge $U\subset\mathbb{R}^n$, in der das
$m$-dimensionale Vektorraumbündel $E$ die Form $U\times V$ erhält.
Für die Tangentialvektoren von $TU$ können wir die Operatoren
$\partial_i$, $i=1,\dots,n$, verwenden.
Ausserdem verwenden wir eine Basis $v_1,\dots,v_m$ von $E$.

Der Zusammenhang ist dann eine Abbildung
\[
\nabla
\colon
\Gamma(TM) \times \Gamma(E)
\to
\Gamma(E),
\]
daher sei $X(x)$ ein Vektorfeld von Tangentialvektoren mit den
Komponenten $\xi^i$, also $X(x)=\xi^i\partial_i$.
Weiter sei $Y(x)$ ein Vektorfeld mit Werten in $E$, in der Basis
$v_k$ ist es durch die Komponenten $y^k$, $k=1,\dots,m$, mit $Y(x)=y^k v_k$
festgelegt.
Die Linearität und Produktregel für den Zusammenhang liefern jetzt
\begin{align*}
\nabla_XY
&=
\nabla_{\xi^1\partial_1+\dots+\xi^n\partial_n} (y^1v_1+\dots+y^mv_m)
\\
&=
\xi^i\nabla_{\partial_i}(y^kv_k)
\\
&=
\xi^i\bigl((\partial_i y^k)v_k + y^k\nabla_{\partial_i} v_k\bigr).
\intertext{Der Vektor $\nabla_{\partial_i}v_k$ ist ein Vektor in $V$,
er lässt sich also als Linearkombination der Vektoren $v_k$ schreiben.
Wir bezeichnen die dazu nötigen Koeffizienten mit $\Gamma^l_{ik}$ und
erhalten}
&=
\xi^i\biggl(
\frac{\partial y^k}{\partial x^i} v_k
+
y^k\Gamma^l_{ik} v_l,
\biggr)
\intertext{oder durch Umbenennung der Summationsindizes}
&=
\xi^i
\biggl(
\frac{\partial y^l}{\partial x^i}
+
y^k\Gamma^l_{ik}
\biggr)v_l.
\end{align*}
Daraus können wir ablesen, dass die $l$-te Komponenten von
$\nabla_{\partial_i}Y$ in der Basis $v_1,\dots,v_m$ durch
\[
\frac{\partial y^l}{\partial x^i}
+
\Gamma^l_{ik}y^k
\]
gegeben ist.
Man beachte, dass der Index $i$ von $1$ bis $n$ läuft, während der
Index $k$ von $1$ bis $m$ läuft.

Nimmt man für $E$ das Tangentialbündel $TM$ der Mannigfaltigkeit, kann
als Basis von $V$ die Menge der Vektoren $\partial_i$ verwendet werden.
Die Koeffizienten $\Gamma^l_{ik}$ haben dann die gleiche Bedeutung
wie in
Definition~\ref{buch:zusammenhang:paralleltransport:kovabl:def:kovabl}.

%
% Kovariante äussere Ableitung
%
\subsection{Kovariante äussere Ableitung
\label{buch:kruemmung:zusammenhang:subsection:kovd}}
Ist $f$ eine reelle Funktion auf der Mannigfaltigkeit, dann ist
$df$ eine $1$-Form, mit der die Richtungsableitung in Richtung
des Vektorfeldes $X$ durch
\[
X\cdot f
=
\langle df,X\rangle
\]
berechnet werden kann.

Die Abbildung $X\mapsto \nabla_XY$ ist linear, man könnte sie
also als eine Linearform $\boldsymbol{\alpha}$ mit Werten im Vektorraumbündel $E$
betrachten, welche auf $X$ den Wert
\[
\langle \boldsymbol{\alpha},X\rangle
=
\nabla_X Y
\]
annimmt.
Offenbar ist diese $E$-wertige $1$-Form aus der $E$-wertigen
Funktion $Y$ dadurch Anwendung von $\nabla_X$ entstanden.
Anscheinend kann man also die äussere Ableitung von $p$-Formen
dadurch auf $E$-wertige $p$-Formen erweitern, in dem man
wo nötig die Richtungsableitung $X\cdot\mathstrut$ durch
die kovariante Ableitung $\nabla_X$ ersetzt.

Wir möchten diese Analogie dazu verwenden, die äussere Ableitung
für beliebige $E$-wertige $p$-Formen zu definieren.
Um Verwechslungen mit der äusseren Ableitung für 
reellwertige $p$-Formen zu verwechseln schreiben wir sie
mit einem fetten $\boldsymbol{d}$.

%
% Kovariante äussere Ableitung einer 1-Form
%
\subsubsection{Kovariante äussere Ableitung einer $1$-Form}
Wir illustrieren diese Idee für den Fall einer $E$-wertigen $1$-Form
$\boldsymbol{\alpha}$.
Für eine reellwertige $1$-Form $\alpha$ ist die äussere Ableitung
\[
\langle
d\alpha
,
X\wedge Y
\rangle
=
X\cdot\langle \alpha,Y\rangle
-
Y\cdot\langle \alpha,X\rangle
-
\langle \alpha,[X,Y]\rangle.
\]
Für eine $E$-wertige $1$-Form muss dies also ersetzt werden durch
\begin{equation}
\langle 
\boldsymbol{d}\boldsymbol{\alpha},
X\wedge Y
\rangle
=
\nabla_X \langle\boldsymbol{\alpha},Y\rangle
-
\nabla_Y \langle\boldsymbol{\alpha},X\rangle
-
\langle \boldsymbol{\alpha},[X,Y]\rangle.
\label{buch:zusammenhang:zusammenhang:eqn:d1form}
\end{equation}
Um zu zeigen, dass dies tatsächlich funktioniert, müssen wir nachweisen,
dass der Ausdruck \eqref{buch:zusammenhang:zusammenhang:eqn:d1form}
tatsächlich eine $2$-Form definiert, also bilinear und antisymmetrisch
in $X$ und $Y$ ist.
Dazu müssen wir $X$ bzw.~$Y$ mit einer glatten Funktion $f$ multiplizieren
und zeigen, dass die rechte Seite von
\eqref{buch:zusammenhang:zusammenhang:eqn:d1form}
ebenfalls nur mit $f$ multipliziert wird.
Dazu berechnen wird den zweiten Term
\begin{align*}
\nabla Y\langle\boldsymbol{\alpha}, fX\rangle
&=
\nabla Y(f\langle\boldsymbol{\alpha}, X\rangle)
=
(Y\cdot f)\langle\boldsymbol{\alpha}, X\rangle
+
f\nabla_X\langle \boldsymbol{\alpha},Y\rangle
\intertext{und den dritten}
\langle\boldsymbol{\alpha},[fX,Y]\rangle
&=
\langle\boldsymbol{\alpha},f[X,Y] - (Y\cdot f) X\rangle
=
f\langle\boldsymbol{\alpha},[X,Y]\rangle
-
(Y\cdot f) \langle\boldsymbol{\alpha}, X\rangle.
\intertext{In der Summe heben sich die Terme mit $Y\cdot f$ weg und ergeben}
\nabla Y\langle\boldsymbol{\alpha}, fY\rangle
+
\langle\boldsymbol{\alpha},[fX,Y]\rangle
&=
f\nabla_X\langle \boldsymbol{\alpha},Y\rangle
+
f\langle\boldsymbol{\alpha},[X,Y]\rangle.
\intertext{Aus \eqref{buch:zusammenhang:zusammenhang:eqn:d1form} wird daher}
\langle 
d\boldsymbol{\alpha},
fX\wedge Y
\rangle
&=
f\nabla_X \langle\boldsymbol{\alpha},Y\rangle
-
f\nabla_Y \langle\boldsymbol{\alpha},X\rangle
-
f\langle \boldsymbol{\alpha},[X,Y]\rangle
=
f
\langle 
d\boldsymbol{\alpha},
X\wedge Y
\rangle.
\end{align*}
Dies zeigt, dass die rechte Seite von
\eqref{buch:zusammenhang:zusammenhang:eqn:d1form}
tatsächlich eine 2-Form definiert.

%
% Kovariante äussere Ableitung von p-Formen
%
\subsubsection{Kovariante äussere Ableitung von $p$-Formen}
Für eine reellwertige $p$-Form $\omega$ war die äussere Ableitung
durch
\begin{align*}
\langle d\omega,X_0\wedge\dots\wedge X_p\rangle
&=
\sum_{k=0}^p
(-1)^k
X_k\cdot\langle \omega,X_0\wedge\dots\wedge \widehat{X_k}\wedge\dots\wedge X_p\rangle
\\
&\quad+
\sum_{0\le k<l\le p}
(-1)^{k+l}\langle\omega,
[X_k,X_l]\wedge X_0\wedge\dots\wedge\widehat{X_k}\wedge\dots\wedge\widehat{X_l}\wedge\dots\wedge X_p\rangle
\end{align*}
gegeben.
Die Konstruktion kann auf eine beliebige $E$-wertige $p$-Form $\omega$
erweitert werden.
Die $E$-wertige $(p+1)$-Form $d\omega$ wird definiert durch
\begin{align*}
\langle \boldsymbol{d}\boldsymbol{\omega}, X_0\wedge\dots\wedge X_{p}\rangle
&=
\sum_{k=0}^p
(-1)^k\nabla_{X_k}\langle\boldsymbol{\omega},X_0\wedge\dots\wedge\widehat{X_k}\wedge\dots\wedge X_p)
\\
&\quad+
\sum_{0\le k < l\le p}(-1)^{k+l}
\langle \boldsymbol{\omega},
[X_k,X_l]\wedge X_0\wedge\dots\wedge\widehat{X_k}\wedge\dots\wedge\widehat{X_l}\wedge\dots\wedge X_p\rangle.
\end{align*}
Der Beweis, dass dieser Ausdruck tatsächlich eine $(p+1)$-Form definiert
gestaltet sich analog zum Falle $p=1$.

%
% Die kovariante äussere Ableitung ist nicht nilpotent
%
\subsubsection{Die zweite kovariante äussere Ableitung verschwindet nicht}
Die Analogie der Konstruktion der kovarianten äusseren Ableitung
zur Konstruktion der äusseren Ableitungen kann einen dazu verführen
anzunehmen, dass auch die kovariante äussere Ableitung
$\boldsymbol{d} \boldsymbol{d}=0$ erfüllt.
Bei der äusseren Ableitung von reellwertigen Funktion hat diese
im Satz~\ref{buch:section:pformen:satz:zweiteableitung} bewiesene
Eigenschaft letztendlich darauf beruht, dass partielle Ableitungen,
also die Operatoren $\partial_i$, nach dem Satz von Schwarz vertauschen.
Wie wir im nächsten Kapitel~\ref{chapter:kruemmung} sehen werden, trifft
dies für die Operatoren $\nabla_i$ im Allgemeinen nicht zu und gibt
Anlass zum Begriff der Krümmung.



