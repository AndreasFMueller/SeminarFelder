%
% einleitung.tex -- Beispiel-File für die Einleitung
%
% (c) 2020 Prof Dr Andreas Müller, Hochschule Rapperswil
%
% !TEX root = ../../buch.tex
% !TEX encoding = UTF-8
%
\section{Simulation\label{particles:section:simulation}}
\kopfrechts{Simulation}

Für die konventionelle Wellengleichung~\ref{particles:eq:wellengleichung} können Lösungen besonders für einfache Fälle auch analytisch gefunden werden.
Sobald jedoch das Medium nichtlinear und der Proportionalitätsfaktor $c$ nicht mehr unabhängig von der Spannung deformation $u$ ist, wird ein analytisches Lösen beinahe unmöglich.
Entsprechend wurde in der hinsicht darauf, dass schliesslich auch nichtlineare Medien simuliert werden können, ein numerischer Ansatz gewählt.

So wurde für die Simulationen eine von \textbf{insert name} angepasste Version des Open-Source Wellensimulators WaveSimulator2D~\cite{repo:wavesim2d} verwendet. % TODO: Insert name and cite
Dieser simuliert eine zwidimensionale Szene, indem er die Wellengleichung für einen diskreten Zeitschritt simuliert und das resultierende Feld anzeigt.
Die Szene wird dabei ebenfalls örtlich diskretisiert, indem sie in Pixel aufgeteilt wird.

% TODO: Nähere erklärung der Berechnungen

\subsection{Szenendefinition}
Zur definition des Ausgangszustands des Feldes sowie der Eigenschaften des Mediums mimmt der Simulator ein RGB-Bild entgegen.
Den drei Farbkanälen wird jeweils eine Funktion zugeteilt.
Der rote Kanal bestimmt den Brechungsindex des jeweiligen Pixels, während über den blauen Kanal die Dämpfung bestimmt werden kann.
Über den grünen Kanal können im unabgeänderten Simulator Quellen dargestellt werden, wobei der Farbwert die Frequenz der Quelle darstellt.
Damit die in der Simulation enthaltene Energie nicht fortlaufend ansteigt, wurde die Funktion des grünen Kanals durch \textbf{insert name} angepasst. % TODO: Echter Name von Huygens einfügen
Der grüne Kanal diktiert nun das Spannungsfeld am Anfang der Simulation. % CHECK: ist es die Spannung oder die Deformation?
% TODO: Beispielszene

\subsection{Nichtlinearität}
Die Nichtlinearität des Mediums wird durch die Formel
\[
    \text{stress} = a \cdot \text{strain} \cdot e^{b \cdot \text{strain}} % TODO: Tatsächliche Formel
\]
gegeben, wobei das nichtlineare Verhalten durch die Parameter $a$ und $b$ angepasst werden kann.
Einige Beispiele der daraus resultierenden Spannungs-Deformations-Kurven sind in Abb.~\ref{particles:abb:nonlin} gezeigt. % TODO: Abbildung

\subsection{Simulation}
Die Simulation ist trotz relativ tiefer Auflösung nicht trivial. 
Der WaveSimulator2D zieht dabei nutzen aus \textbf{OpenGL} um die Simulation auf der Grafikkarte laufen zu lassen. % CHECK: Ist es OpenGL

Zum Lösen der Wellengleichung~\ref{particles:eq:wellen} wird... % TODO: Funktion der Simulation beschreiben?