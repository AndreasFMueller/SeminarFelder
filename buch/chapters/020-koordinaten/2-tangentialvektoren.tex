%
% 2-tangentialvektoren.tex -- Tangentialvektoren
%
% (c) 2024 Prof Dr Andreas Müller
%
\section{Tangentialvektoren
\label{buch:koordinaten:section:tangentialvektoren}}
\kopfrechts{Tangentialvektoren}
Das elektrische Feld übt auf eine Testladung eine Kraft aus, die
proportional zur Ladung ist.
Diese Kräfte stellt man sich gerne als ein Vektorfeld vor, doch
in welchem Raum sind diese Vektoren zu finden?
Die Kraft verursacht eine Beschleunigung und verändert damit 
die Geschwindigkeit.
Als erstes muss daher der Geschwindigkeitsvektor konstruiert werden,
der tangential an die Bahnkurve der Testladung verläuft.
An dieser naheliegenden und üblichen Darstellung ist aber eigentlich
falsch, dass der Geschwindigkeitsvektor gar nicht im gleichen Raum
dargestellt werden kann.
Die Masseinheit der Komponenten des Geschwindigkeitsvektors ist
[Länge/Zeit], während die Koordinaten die Masseinheit [Länge] haben.
Solange man das Koordinatensystem und damit die Masseinheiten nicht
wechselt, mag die Konfusion in Grenzen bleiben.
Da aber alle Gesetzmässigkeiten auf eine koordinatensystemunabhängige
Art formuliert werden müssen, bedarf auch das Konzept des Tangentialvektors
einer Reevaluation.

%
% Kurven
%
\subsection{Kurven}

%
% Koordinatenlinien
%
\subsection{Koordinatenlinien}

%
% Tangentialvektoren
%
\subsection{Tangentialvektoren}


