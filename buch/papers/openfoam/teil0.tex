%
% einleitung.tex -- Beispiel-File für die Einleitung
%
% (c) 2020 Prof Dr Andreas Müller, Hochschule Rapperswil
%
% !TEX root = ../../buch.tex
% !TEX encoding = UTF-8
%
\section{Einleitung Zu Open Foam\label{openfoam:section:Einleitung}}
\kopfrechts{Einleitung Zu Open Foam}

\subsection{Ziel und Inhalt unserer Arbeit}
OpenFOAM ist eine Open Source Software zur Berechnung von Strömungsfeldern. 

Das Ziel dieses Papers ist es dem Leser einen Überblick zu verschaffen was OpenFOAM ist und anhand eines 
Anwendungsbeispiel zeigen wie eine simple Simulation selbst durchgeführt werden kann.
%TODO
%XXXXXXXXXXXXXXXXXXXXXXXXXXXXXXXXXXXXXXXXXXXXXX

\subsection{Was ist eigentlich OpenFOAM?\label{openfoam:section:WasIstOpenFoam}}
OpenFOAM ist, wie der Name sagt, eine Open Source Software die unter der GNU
General Public License Kostenlos zugänglich ist und dadurch auch für Kommerzielle Zwecke 
verwendet werden kann. 

Bei OpenFOAM handelt es sich um eine \emph{CFD} Software, dies steht für \emph{C}omputational \emph{F}luid \emph{D}ynamics also Numerische Strömungsdynamik.
OpenFOAM kann dementsprechend verwendet werden um näherungsweise Lösungen für 
Strömungsmechanische Probleme zu finden.
Es können zum Beispiel Widerstandsbeiwerte von Autos oder die Wirbelschleppen einer Tragfläche simuliert werden.
Dabei liegen für OpenFOAM im Vergleich zu anderer CFD Software die Stärken nicht bei Nutzerfreundlichkeit oder Geschwindigkeit sondern bei der Anpassbarkeit, zum einen da es sich um offene Software aber auch dadurch das OpenFOAM entwickelt wurde mit dem ziel für die Erweiterbarkeit durch den Nutzer zu ermöglichen. Dies macht es weniger praktikabel die Software als Simulator zum lösen von Standardproblemen anzuwenden sondern für das Entwickeln von Solver für neue Problemstellungen für die noch keine herkömmliche Solver existieren.

OpenFOAM wird seit 2004 hauptsächlich von OpenCFD Ltd entwickelt, es wird jedoch auch aktiv durch die Community weiterentwickelt.
Die getesteten Releases werden alle 6 Monate herausgegeben. \cite{openfoam:greenshieldsweller2022}

\subsection{Was macht OpenFOAM?}
Zum Simulieren Strömungsmechanischer Probleme muss man die Partiellen Differenziellen Gleichungen (v.a Erhaltungssätze) die das Verhalten des Mediums beschreiben, ausgehend von Initial- und Grenzbedingungen, lösen.
Dabei beschreiben die Differenziellen Gleichungen Felder für die Simulationsgrößen wie Geschwindigkeit, Druck, Temperatur, Dichte und je nach Anwendung noch viele mehr. OpenFOAM nutzt die Manipulation von solchen Feldern um eine Numerische Lösung zu finden. Daher kommt auch das Akronym FOAM dies steht für Field Operation And  Manipulation.

