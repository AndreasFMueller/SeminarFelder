%
% sl.tex -- Schwarzes Loch
%
% (c) 2021 Prof Dr Andreas Müller, OST Ostschweizer Fachhochschule
%
\documentclass[tikz]{standalone}
\usepackage{amsmath}
\usepackage{times}
\usepackage{txfonts}
\usepackage{pgfplots}
\usepackage{csvsimple}
\usetikzlibrary{arrows,intersections,math}
\begin{document}
\def\skala{3}
\definecolor{darkred}{rgb}{0.8,0,0}
\def\kegel#1#2{
	\pgfmathparse{90-atan(1-1/#1)}
	\xdef\w{\pgfmathresult}
	\fill[color=blue!20] (#1,#2) -- ++(\w:0.2) -- ++({-2*0.2*cos(\w)},0)
		-- cycle;
	\fill[color=blue!10] (#1,#2) -- ++(-\w:0.2) -- ++({-2*0.2*cos(\w)},0)
		-- cycle;
	\draw (#1,#2) -- ++(\w:0.2);
	\draw (#1,#2) -- ++(-\w:0.2);
	\draw (#1,#2) -- ++({180-\w}:0.2);
	\draw (#1,#2) -- ++({-180+\w}:0.2);
}
\def\legek#1#2{
	\pgfmathparse{90-atan(1-1/#1)}
	\xdef\w{\pgfmathresult}
	\fill[color=blue!20] (#1,#2) -- ++(\w:0.2) -- ++(0,{-2*0.2*sin(\w)})
		-- cycle;
	\fill[color=blue!10] (#1,#2) -- ++({\w-180}:0.2)
		-- ++(0,{2*0.2*sin(\w)}) -- cycle;
	\draw (#1,#2) -- ++(\w:0.2);
	\draw (#1,#2) -- ++(-\w:0.2);
	\draw (#1,#2) -- ++({180-\w}:0.2);
	\draw (#1,#2) -- ++({-180+\w}:0.2);
}
\begin{tikzpicture}[>=latex,thick,scale=\skala]

\draw[color=darkred] (1,-0.1) -- (1,1);

\draw[->] ({-0.1/\skala},0) -- (3.9,0) coordinate[label={$r$}];
\draw[->] (0,{-0.1/\skala}) -- (0,1.1) coordinate[label={right:$ct$}];

\node[color=darkred] at (1,0) [below right] {$r_s$};

\kegel{1.1}{0.75}
\kegel{1.25}{0.75}
\kegel{1.5}{0.75}
\kegel{2.00}{0.75}
\kegel{2.50}{0.75}
\kegel{3.00}{0.75}
\kegel{3.50}{0.75}

\legek{0.9}{0.75}
\legek{0.75}{0.75}
\legek{0.5}{0.75}
\legek{0.10}{0.75}

\kegel{1.1}{0.25}
\kegel{1.25}{0.25}
\kegel{1.5}{0.25}
\kegel{2.00}{0.25}
\kegel{2.50}{0.25}
\kegel{3.00}{0.25}
\kegel{3.50}{0.25}

\legek{0.9}{0.25}
\legek{0.75}{0.25}
\legek{0.5}{0.25}
\legek{0.10}{0.25}

\end{tikzpicture}
\end{document}

