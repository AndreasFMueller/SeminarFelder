\section{Der Satz von Poincaré-Bendixson} \label{poinbendix:section:poinbendix}

\begin{satz}[Poincaré-Bendixson]
\label{poinbendix:satz:poinbendix}
$\Phi_t(p) \in \Xi^r(\mathbb{S}^2)$ sei ein $r$-Fach differenzierbares, zweidimensionales dynamisches System mit Startpunkt $p \in \mathbb{S}^2$.
Dann gilt für das Omega-Limit Set $\omega(p)$ eine der folgenden Optionen:
\begin{enumerate}
\item $\omega(p)$ ist eine Nullstelle
\item $\omega(p)$ ist ein geschlossener Orbit
\item $\omega(p)$ ist ein geschlossener Orbit welcher Singularitäten verbindet
\end{enumerate}
\end{satz}

Dies gilt auf der Kugeloberfläche $\mathbb{S}^2$.\cite{poinbendix:melo}
Bis anhin haben wir aber mehrheitlich über planare Fälle gesprochen, weshalb der nächste Abschnitt darauf eingeht wieso wir hier die Kugeloberfläche nehmen.
In den darauffolgenden Abschnitten wird für dann jeden Fall jeweils ein Beispiel gezeigt.

\subsection{Kugeloberfläche und Ebene} \label{poinbendix:subsection:kugeloberflaeche}

Im Beispiel \ref{buch:koordinaten:diffmannig:beispiel:stereographisch} wird die stereographische Projektion eingeführt, welche ein planares Polarkoordinatensystem auf die Kugeloberfläche projiziert.
In Abbildung \ref{buch:koordinaten:diffmannig:fig:stereographisch} sieht man, dass dabei der Punkt auf dem Nordpol der Kugeloberfläche ins unendliche auf der Ebene abgebildet wird.
Somit ergibt sich für den Fall einer Nullstelle auf dem Nordpol eine divergierende Bahnkurve auf der Ebene.

Auch wenn die Formulierung auf der Kugeloberfläche etwas eleganter ist, gilt der Satz von Poincaré-Bendixson sowohl auf der Ebene $\mathbb{R}^2$ wie auch auf der Kugeloberfläche $\mathbb{S}^2$.
Allerdings gilt er nicht auf allen zweidimensionalen Abbildungen, so können zum Beispiel auf dem Torus Bahnkurven auftreten welche nicht wirklich periodisch sind, aber auch nicht auf eine Singularität fallen.\cite{poinbendix:wiki}


\subsection{Fall 1: $\omega(p)$ ist eine Nullstelle} \label{poinbendix:subsection:fall1}

Der erste Fall ist intuitiv am einfachsten zu verstehen.
Sobald auf einer Bahnkurve eine Nullstelle auftritt, bleibt die Kurve für alle Zeiten stehen.
Die eindimensionale Version von Fall 1  wurde bereits im Beispiel \ref{poinbendix:beispiel:1dlimesmengen} über die Limesmengen gezeigt.

\begin{beispiel}
Wir betrachten das folgende Differentialgleichungssystem:
\begin{align*}
    \dot{x} &= -x + y(1-x^2-y^2) \\
    \dot{y} &= -y - x(1-x^2-y^2).
\end{align*}
Nullstellen sind Punkte wo beide Ableitungen Null sind.
Zunächst substituieren wir den Klammerausdruck durch $a = (1-x^2-y^2)$ und setzen die Ableitung nach $x$ gleich Null
\begin{align*}
    -x + ya &= 0 \\
    ya &= x.
\end{align*}
Nun setzen wir dieses Resultat in die Ableitung nach $y$ und setzen es gleich Null
\begin{align*}
    -y - ya^2 &= 0 \\
    y(a^2+1) &= 0.
\end{align*}
Nach dieser Untersuchung sehen wir, dass es nur eine reelle Nullstelle $x=y=0$ geben kann, da $a^2 + 1$ im reellen nie Null werden kann.

Dieses Beispiel wurde gewählt da sehr schöne Bahnkurven entstehen.
Wie diese Bahnkurven aussehen können sieht man in Abbildung \ref{poinbendix:fig:fixed_point_omega_set}.
\end{beispiel}

Zu beachten gilt es, dass Satz von Poincaré-Bendixson nur eine Aussage zu möglichen Lösungen ab einem Startpunkt $p$ macht.
Dies würde unterschiedliche Fälle erlauben für ein einzelnes Differentialgleichungssystem.
Das obige Beispiel zeigt aber einen häufigen Fall, dass nämlich diverse Startpunkte $p$ denselben Nullpunkt als Omega Limesmenge haben.

\begin{figure}
    \centering
    %% Creator: Matplotlib, PGF backend
%%
%% To include the figure in your LaTeX document, write
%%   \input{<filename>.pgf}
%%
%% Make sure the required packages are loaded in your preamble
%%   \usepackage{pgf}
%%
%% Also ensure that all the required font packages are loaded; for instance,
%% the lmodern package is sometimes necessary when using math font.
%%   \usepackage{lmodern}
%%
%% Figures using additional raster images can only be included by \input if
%% they are in the same directory as the main LaTeX file. For loading figures
%% from other directories you can use the `import` package
%%   \usepackage{import}
%%
%% and then include the figures with
%%   \import{<path to file>}{<filename>.pgf}
%%
%% Matplotlib used the following preamble
%%   \usepackage{bm}
%%   \usepackage{amsmath}
%%   \usepackage{xcolor}
%%   \usepackage{tgtermes}
%%   \makeatletter\@ifpackageloaded{underscore}{}{\usepackage[strings]{underscore}}\makeatother
%%
\begingroup%
\makeatletter%
\begin{pgfpicture}%
\pgfpathrectangle{\pgfpointorigin}{\pgfqpoint{4.500000in}{2.500000in}}%
\pgfusepath{use as bounding box, clip}%
\begin{pgfscope}%
\pgfsetbuttcap%
\pgfsetmiterjoin%
\definecolor{currentfill}{rgb}{1.000000,1.000000,1.000000}%
\pgfsetfillcolor{currentfill}%
\pgfsetlinewidth{0.000000pt}%
\definecolor{currentstroke}{rgb}{1.000000,1.000000,1.000000}%
\pgfsetstrokecolor{currentstroke}%
\pgfsetdash{}{0pt}%
\pgfpathmoveto{\pgfqpoint{0.000000in}{0.000000in}}%
\pgfpathlineto{\pgfqpoint{4.500000in}{0.000000in}}%
\pgfpathlineto{\pgfqpoint{4.500000in}{2.500000in}}%
\pgfpathlineto{\pgfqpoint{0.000000in}{2.500000in}}%
\pgfpathlineto{\pgfqpoint{0.000000in}{0.000000in}}%
\pgfpathclose%
\pgfusepath{fill}%
\end{pgfscope}%
\begin{pgfscope}%
\pgfsetbuttcap%
\pgfsetmiterjoin%
\definecolor{currentfill}{rgb}{1.000000,1.000000,1.000000}%
\pgfsetfillcolor{currentfill}%
\pgfsetlinewidth{0.000000pt}%
\definecolor{currentstroke}{rgb}{0.000000,0.000000,0.000000}%
\pgfsetstrokecolor{currentstroke}%
\pgfsetstrokeopacity{0.000000}%
\pgfsetdash{}{0pt}%
\pgfpathmoveto{\pgfqpoint{0.562500in}{0.275000in}}%
\pgfpathlineto{\pgfqpoint{4.050000in}{0.275000in}}%
\pgfpathlineto{\pgfqpoint{4.050000in}{2.200000in}}%
\pgfpathlineto{\pgfqpoint{0.562500in}{2.200000in}}%
\pgfpathlineto{\pgfqpoint{0.562500in}{0.275000in}}%
\pgfpathclose%
\pgfusepath{fill}%
\end{pgfscope}%
\begin{pgfscope}%
\pgfpathrectangle{\pgfqpoint{0.562500in}{0.275000in}}{\pgfqpoint{3.487500in}{1.925000in}}%
\pgfusepath{clip}%
\pgfsetrectcap%
\pgfsetroundjoin%
\pgfsetlinewidth{0.803000pt}%
\definecolor{currentstroke}{rgb}{0.690196,0.690196,0.690196}%
\pgfsetstrokecolor{currentstroke}%
\pgfsetdash{}{0pt}%
\pgfpathmoveto{\pgfqpoint{0.732791in}{0.275000in}}%
\pgfpathlineto{\pgfqpoint{0.732791in}{2.200000in}}%
\pgfusepath{stroke}%
\end{pgfscope}%
\begin{pgfscope}%
\pgfsetbuttcap%
\pgfsetroundjoin%
\definecolor{currentfill}{rgb}{0.000000,0.000000,0.000000}%
\pgfsetfillcolor{currentfill}%
\pgfsetlinewidth{0.803000pt}%
\definecolor{currentstroke}{rgb}{0.000000,0.000000,0.000000}%
\pgfsetstrokecolor{currentstroke}%
\pgfsetdash{}{0pt}%
\pgfsys@defobject{currentmarker}{\pgfqpoint{0.000000in}{-0.048611in}}{\pgfqpoint{0.000000in}{0.000000in}}{%
\pgfpathmoveto{\pgfqpoint{0.000000in}{0.000000in}}%
\pgfpathlineto{\pgfqpoint{0.000000in}{-0.048611in}}%
\pgfusepath{stroke,fill}%
}%
\begin{pgfscope}%
\pgfsys@transformshift{0.732791in}{0.275000in}%
\pgfsys@useobject{currentmarker}{}%
\end{pgfscope}%
\end{pgfscope}%
\begin{pgfscope}%
\definecolor{textcolor}{rgb}{0.000000,0.000000,0.000000}%
\pgfsetstrokecolor{textcolor}%
\pgfsetfillcolor{textcolor}%
\pgftext[x=0.732791in,y=0.177778in,,top]{\color{textcolor}\rmfamily\fontsize{10.000000}{12.000000}\selectfont \(\displaystyle {-1.0}\)}%
\end{pgfscope}%
\begin{pgfscope}%
\pgfpathrectangle{\pgfqpoint{0.562500in}{0.275000in}}{\pgfqpoint{3.487500in}{1.925000in}}%
\pgfusepath{clip}%
\pgfsetrectcap%
\pgfsetroundjoin%
\pgfsetlinewidth{0.803000pt}%
\definecolor{currentstroke}{rgb}{0.690196,0.690196,0.690196}%
\pgfsetstrokecolor{currentstroke}%
\pgfsetdash{}{0pt}%
\pgfpathmoveto{\pgfqpoint{1.522463in}{0.275000in}}%
\pgfpathlineto{\pgfqpoint{1.522463in}{2.200000in}}%
\pgfusepath{stroke}%
\end{pgfscope}%
\begin{pgfscope}%
\pgfsetbuttcap%
\pgfsetroundjoin%
\definecolor{currentfill}{rgb}{0.000000,0.000000,0.000000}%
\pgfsetfillcolor{currentfill}%
\pgfsetlinewidth{0.803000pt}%
\definecolor{currentstroke}{rgb}{0.000000,0.000000,0.000000}%
\pgfsetstrokecolor{currentstroke}%
\pgfsetdash{}{0pt}%
\pgfsys@defobject{currentmarker}{\pgfqpoint{0.000000in}{-0.048611in}}{\pgfqpoint{0.000000in}{0.000000in}}{%
\pgfpathmoveto{\pgfqpoint{0.000000in}{0.000000in}}%
\pgfpathlineto{\pgfqpoint{0.000000in}{-0.048611in}}%
\pgfusepath{stroke,fill}%
}%
\begin{pgfscope}%
\pgfsys@transformshift{1.522463in}{0.275000in}%
\pgfsys@useobject{currentmarker}{}%
\end{pgfscope}%
\end{pgfscope}%
\begin{pgfscope}%
\definecolor{textcolor}{rgb}{0.000000,0.000000,0.000000}%
\pgfsetstrokecolor{textcolor}%
\pgfsetfillcolor{textcolor}%
\pgftext[x=1.522463in,y=0.177778in,,top]{\color{textcolor}\rmfamily\fontsize{10.000000}{12.000000}\selectfont \(\displaystyle {-0.5}\)}%
\end{pgfscope}%
\begin{pgfscope}%
\pgfpathrectangle{\pgfqpoint{0.562500in}{0.275000in}}{\pgfqpoint{3.487500in}{1.925000in}}%
\pgfusepath{clip}%
\pgfsetrectcap%
\pgfsetroundjoin%
\pgfsetlinewidth{0.803000pt}%
\definecolor{currentstroke}{rgb}{0.690196,0.690196,0.690196}%
\pgfsetstrokecolor{currentstroke}%
\pgfsetdash{}{0pt}%
\pgfpathmoveto{\pgfqpoint{2.312134in}{0.275000in}}%
\pgfpathlineto{\pgfqpoint{2.312134in}{2.200000in}}%
\pgfusepath{stroke}%
\end{pgfscope}%
\begin{pgfscope}%
\pgfsetbuttcap%
\pgfsetroundjoin%
\definecolor{currentfill}{rgb}{0.000000,0.000000,0.000000}%
\pgfsetfillcolor{currentfill}%
\pgfsetlinewidth{0.803000pt}%
\definecolor{currentstroke}{rgb}{0.000000,0.000000,0.000000}%
\pgfsetstrokecolor{currentstroke}%
\pgfsetdash{}{0pt}%
\pgfsys@defobject{currentmarker}{\pgfqpoint{0.000000in}{-0.048611in}}{\pgfqpoint{0.000000in}{0.000000in}}{%
\pgfpathmoveto{\pgfqpoint{0.000000in}{0.000000in}}%
\pgfpathlineto{\pgfqpoint{0.000000in}{-0.048611in}}%
\pgfusepath{stroke,fill}%
}%
\begin{pgfscope}%
\pgfsys@transformshift{2.312134in}{0.275000in}%
\pgfsys@useobject{currentmarker}{}%
\end{pgfscope}%
\end{pgfscope}%
\begin{pgfscope}%
\definecolor{textcolor}{rgb}{0.000000,0.000000,0.000000}%
\pgfsetstrokecolor{textcolor}%
\pgfsetfillcolor{textcolor}%
\pgftext[x=2.312134in,y=0.177778in,,top]{\color{textcolor}\rmfamily\fontsize{10.000000}{12.000000}\selectfont \(\displaystyle {0.0}\)}%
\end{pgfscope}%
\begin{pgfscope}%
\pgfpathrectangle{\pgfqpoint{0.562500in}{0.275000in}}{\pgfqpoint{3.487500in}{1.925000in}}%
\pgfusepath{clip}%
\pgfsetrectcap%
\pgfsetroundjoin%
\pgfsetlinewidth{0.803000pt}%
\definecolor{currentstroke}{rgb}{0.690196,0.690196,0.690196}%
\pgfsetstrokecolor{currentstroke}%
\pgfsetdash{}{0pt}%
\pgfpathmoveto{\pgfqpoint{3.101806in}{0.275000in}}%
\pgfpathlineto{\pgfqpoint{3.101806in}{2.200000in}}%
\pgfusepath{stroke}%
\end{pgfscope}%
\begin{pgfscope}%
\pgfsetbuttcap%
\pgfsetroundjoin%
\definecolor{currentfill}{rgb}{0.000000,0.000000,0.000000}%
\pgfsetfillcolor{currentfill}%
\pgfsetlinewidth{0.803000pt}%
\definecolor{currentstroke}{rgb}{0.000000,0.000000,0.000000}%
\pgfsetstrokecolor{currentstroke}%
\pgfsetdash{}{0pt}%
\pgfsys@defobject{currentmarker}{\pgfqpoint{0.000000in}{-0.048611in}}{\pgfqpoint{0.000000in}{0.000000in}}{%
\pgfpathmoveto{\pgfqpoint{0.000000in}{0.000000in}}%
\pgfpathlineto{\pgfqpoint{0.000000in}{-0.048611in}}%
\pgfusepath{stroke,fill}%
}%
\begin{pgfscope}%
\pgfsys@transformshift{3.101806in}{0.275000in}%
\pgfsys@useobject{currentmarker}{}%
\end{pgfscope}%
\end{pgfscope}%
\begin{pgfscope}%
\definecolor{textcolor}{rgb}{0.000000,0.000000,0.000000}%
\pgfsetstrokecolor{textcolor}%
\pgfsetfillcolor{textcolor}%
\pgftext[x=3.101806in,y=0.177778in,,top]{\color{textcolor}\rmfamily\fontsize{10.000000}{12.000000}\selectfont \(\displaystyle {0.5}\)}%
\end{pgfscope}%
\begin{pgfscope}%
\pgfpathrectangle{\pgfqpoint{0.562500in}{0.275000in}}{\pgfqpoint{3.487500in}{1.925000in}}%
\pgfusepath{clip}%
\pgfsetrectcap%
\pgfsetroundjoin%
\pgfsetlinewidth{0.803000pt}%
\definecolor{currentstroke}{rgb}{0.690196,0.690196,0.690196}%
\pgfsetstrokecolor{currentstroke}%
\pgfsetdash{}{0pt}%
\pgfpathmoveto{\pgfqpoint{3.891477in}{0.275000in}}%
\pgfpathlineto{\pgfqpoint{3.891477in}{2.200000in}}%
\pgfusepath{stroke}%
\end{pgfscope}%
\begin{pgfscope}%
\pgfsetbuttcap%
\pgfsetroundjoin%
\definecolor{currentfill}{rgb}{0.000000,0.000000,0.000000}%
\pgfsetfillcolor{currentfill}%
\pgfsetlinewidth{0.803000pt}%
\definecolor{currentstroke}{rgb}{0.000000,0.000000,0.000000}%
\pgfsetstrokecolor{currentstroke}%
\pgfsetdash{}{0pt}%
\pgfsys@defobject{currentmarker}{\pgfqpoint{0.000000in}{-0.048611in}}{\pgfqpoint{0.000000in}{0.000000in}}{%
\pgfpathmoveto{\pgfqpoint{0.000000in}{0.000000in}}%
\pgfpathlineto{\pgfqpoint{0.000000in}{-0.048611in}}%
\pgfusepath{stroke,fill}%
}%
\begin{pgfscope}%
\pgfsys@transformshift{3.891477in}{0.275000in}%
\pgfsys@useobject{currentmarker}{}%
\end{pgfscope}%
\end{pgfscope}%
\begin{pgfscope}%
\definecolor{textcolor}{rgb}{0.000000,0.000000,0.000000}%
\pgfsetstrokecolor{textcolor}%
\pgfsetfillcolor{textcolor}%
\pgftext[x=3.891477in,y=0.177778in,,top]{\color{textcolor}\rmfamily\fontsize{10.000000}{12.000000}\selectfont \(\displaystyle {1.0}\)}%
\end{pgfscope}%
\begin{pgfscope}%
\pgfpathrectangle{\pgfqpoint{0.562500in}{0.275000in}}{\pgfqpoint{3.487500in}{1.925000in}}%
\pgfusepath{clip}%
\pgfsetrectcap%
\pgfsetroundjoin%
\pgfsetlinewidth{0.803000pt}%
\definecolor{currentstroke}{rgb}{0.690196,0.690196,0.690196}%
\pgfsetstrokecolor{currentstroke}%
\pgfsetdash{}{0pt}%
\pgfpathmoveto{\pgfqpoint{0.562500in}{0.362500in}}%
\pgfpathlineto{\pgfqpoint{4.050000in}{0.362500in}}%
\pgfusepath{stroke}%
\end{pgfscope}%
\begin{pgfscope}%
\pgfsetbuttcap%
\pgfsetroundjoin%
\definecolor{currentfill}{rgb}{0.000000,0.000000,0.000000}%
\pgfsetfillcolor{currentfill}%
\pgfsetlinewidth{0.803000pt}%
\definecolor{currentstroke}{rgb}{0.000000,0.000000,0.000000}%
\pgfsetstrokecolor{currentstroke}%
\pgfsetdash{}{0pt}%
\pgfsys@defobject{currentmarker}{\pgfqpoint{-0.048611in}{0.000000in}}{\pgfqpoint{-0.000000in}{0.000000in}}{%
\pgfpathmoveto{\pgfqpoint{-0.000000in}{0.000000in}}%
\pgfpathlineto{\pgfqpoint{-0.048611in}{0.000000in}}%
\pgfusepath{stroke,fill}%
}%
\begin{pgfscope}%
\pgfsys@transformshift{0.562500in}{0.362500in}%
\pgfsys@useobject{currentmarker}{}%
\end{pgfscope}%
\end{pgfscope}%
\begin{pgfscope}%
\definecolor{textcolor}{rgb}{0.000000,0.000000,0.000000}%
\pgfsetstrokecolor{textcolor}%
\pgfsetfillcolor{textcolor}%
\pgftext[x=0.287808in, y=0.315799in, left, base]{\color{textcolor}\rmfamily\fontsize{10.000000}{12.000000}\selectfont \(\displaystyle {-1}\)}%
\end{pgfscope}%
\begin{pgfscope}%
\pgfpathrectangle{\pgfqpoint{0.562500in}{0.275000in}}{\pgfqpoint{3.487500in}{1.925000in}}%
\pgfusepath{clip}%
\pgfsetrectcap%
\pgfsetroundjoin%
\pgfsetlinewidth{0.803000pt}%
\definecolor{currentstroke}{rgb}{0.690196,0.690196,0.690196}%
\pgfsetstrokecolor{currentstroke}%
\pgfsetdash{}{0pt}%
\pgfpathmoveto{\pgfqpoint{0.562500in}{0.943734in}}%
\pgfpathlineto{\pgfqpoint{4.050000in}{0.943734in}}%
\pgfusepath{stroke}%
\end{pgfscope}%
\begin{pgfscope}%
\pgfsetbuttcap%
\pgfsetroundjoin%
\definecolor{currentfill}{rgb}{0.000000,0.000000,0.000000}%
\pgfsetfillcolor{currentfill}%
\pgfsetlinewidth{0.803000pt}%
\definecolor{currentstroke}{rgb}{0.000000,0.000000,0.000000}%
\pgfsetstrokecolor{currentstroke}%
\pgfsetdash{}{0pt}%
\pgfsys@defobject{currentmarker}{\pgfqpoint{-0.048611in}{0.000000in}}{\pgfqpoint{-0.000000in}{0.000000in}}{%
\pgfpathmoveto{\pgfqpoint{-0.000000in}{0.000000in}}%
\pgfpathlineto{\pgfqpoint{-0.048611in}{0.000000in}}%
\pgfusepath{stroke,fill}%
}%
\begin{pgfscope}%
\pgfsys@transformshift{0.562500in}{0.943734in}%
\pgfsys@useobject{currentmarker}{}%
\end{pgfscope}%
\end{pgfscope}%
\begin{pgfscope}%
\definecolor{textcolor}{rgb}{0.000000,0.000000,0.000000}%
\pgfsetstrokecolor{textcolor}%
\pgfsetfillcolor{textcolor}%
\pgftext[x=0.395833in, y=0.897032in, left, base]{\color{textcolor}\rmfamily\fontsize{10.000000}{12.000000}\selectfont \(\displaystyle {0}\)}%
\end{pgfscope}%
\begin{pgfscope}%
\pgfpathrectangle{\pgfqpoint{0.562500in}{0.275000in}}{\pgfqpoint{3.487500in}{1.925000in}}%
\pgfusepath{clip}%
\pgfsetrectcap%
\pgfsetroundjoin%
\pgfsetlinewidth{0.803000pt}%
\definecolor{currentstroke}{rgb}{0.690196,0.690196,0.690196}%
\pgfsetstrokecolor{currentstroke}%
\pgfsetdash{}{0pt}%
\pgfpathmoveto{\pgfqpoint{0.562500in}{1.524967in}}%
\pgfpathlineto{\pgfqpoint{4.050000in}{1.524967in}}%
\pgfusepath{stroke}%
\end{pgfscope}%
\begin{pgfscope}%
\pgfsetbuttcap%
\pgfsetroundjoin%
\definecolor{currentfill}{rgb}{0.000000,0.000000,0.000000}%
\pgfsetfillcolor{currentfill}%
\pgfsetlinewidth{0.803000pt}%
\definecolor{currentstroke}{rgb}{0.000000,0.000000,0.000000}%
\pgfsetstrokecolor{currentstroke}%
\pgfsetdash{}{0pt}%
\pgfsys@defobject{currentmarker}{\pgfqpoint{-0.048611in}{0.000000in}}{\pgfqpoint{-0.000000in}{0.000000in}}{%
\pgfpathmoveto{\pgfqpoint{-0.000000in}{0.000000in}}%
\pgfpathlineto{\pgfqpoint{-0.048611in}{0.000000in}}%
\pgfusepath{stroke,fill}%
}%
\begin{pgfscope}%
\pgfsys@transformshift{0.562500in}{1.524967in}%
\pgfsys@useobject{currentmarker}{}%
\end{pgfscope}%
\end{pgfscope}%
\begin{pgfscope}%
\definecolor{textcolor}{rgb}{0.000000,0.000000,0.000000}%
\pgfsetstrokecolor{textcolor}%
\pgfsetfillcolor{textcolor}%
\pgftext[x=0.395833in, y=1.478266in, left, base]{\color{textcolor}\rmfamily\fontsize{10.000000}{12.000000}\selectfont \(\displaystyle {1}\)}%
\end{pgfscope}%
\begin{pgfscope}%
\pgfpathrectangle{\pgfqpoint{0.562500in}{0.275000in}}{\pgfqpoint{3.487500in}{1.925000in}}%
\pgfusepath{clip}%
\pgfsetrectcap%
\pgfsetroundjoin%
\pgfsetlinewidth{0.803000pt}%
\definecolor{currentstroke}{rgb}{0.690196,0.690196,0.690196}%
\pgfsetstrokecolor{currentstroke}%
\pgfsetdash{}{0pt}%
\pgfpathmoveto{\pgfqpoint{0.562500in}{2.106201in}}%
\pgfpathlineto{\pgfqpoint{4.050000in}{2.106201in}}%
\pgfusepath{stroke}%
\end{pgfscope}%
\begin{pgfscope}%
\pgfsetbuttcap%
\pgfsetroundjoin%
\definecolor{currentfill}{rgb}{0.000000,0.000000,0.000000}%
\pgfsetfillcolor{currentfill}%
\pgfsetlinewidth{0.803000pt}%
\definecolor{currentstroke}{rgb}{0.000000,0.000000,0.000000}%
\pgfsetstrokecolor{currentstroke}%
\pgfsetdash{}{0pt}%
\pgfsys@defobject{currentmarker}{\pgfqpoint{-0.048611in}{0.000000in}}{\pgfqpoint{-0.000000in}{0.000000in}}{%
\pgfpathmoveto{\pgfqpoint{-0.000000in}{0.000000in}}%
\pgfpathlineto{\pgfqpoint{-0.048611in}{0.000000in}}%
\pgfusepath{stroke,fill}%
}%
\begin{pgfscope}%
\pgfsys@transformshift{0.562500in}{2.106201in}%
\pgfsys@useobject{currentmarker}{}%
\end{pgfscope}%
\end{pgfscope}%
\begin{pgfscope}%
\definecolor{textcolor}{rgb}{0.000000,0.000000,0.000000}%
\pgfsetstrokecolor{textcolor}%
\pgfsetfillcolor{textcolor}%
\pgftext[x=0.395833in, y=2.059500in, left, base]{\color{textcolor}\rmfamily\fontsize{10.000000}{12.000000}\selectfont \(\displaystyle {2}\)}%
\end{pgfscope}%
\begin{pgfscope}%
\pgfpathrectangle{\pgfqpoint{0.562500in}{0.275000in}}{\pgfqpoint{3.487500in}{1.925000in}}%
\pgfusepath{clip}%
\pgfsetbuttcap%
\pgfsetroundjoin%
\definecolor{currentfill}{rgb}{0.121569,0.466667,0.705882}%
\pgfsetfillcolor{currentfill}%
\pgfsetlinewidth{1.003750pt}%
\definecolor{currentstroke}{rgb}{0.121569,0.466667,0.705882}%
\pgfsetstrokecolor{currentstroke}%
\pgfsetdash{}{0pt}%
\pgfsys@defobject{currentmarker}{\pgfqpoint{-0.020833in}{-0.020833in}}{\pgfqpoint{0.020833in}{0.020833in}}{%
\pgfpathmoveto{\pgfqpoint{0.000000in}{-0.020833in}}%
\pgfpathcurveto{\pgfqpoint{0.005525in}{-0.020833in}}{\pgfqpoint{0.010825in}{-0.018638in}}{\pgfqpoint{0.014731in}{-0.014731in}}%
\pgfpathcurveto{\pgfqpoint{0.018638in}{-0.010825in}}{\pgfqpoint{0.020833in}{-0.005525in}}{\pgfqpoint{0.020833in}{0.000000in}}%
\pgfpathcurveto{\pgfqpoint{0.020833in}{0.005525in}}{\pgfqpoint{0.018638in}{0.010825in}}{\pgfqpoint{0.014731in}{0.014731in}}%
\pgfpathcurveto{\pgfqpoint{0.010825in}{0.018638in}}{\pgfqpoint{0.005525in}{0.020833in}}{\pgfqpoint{0.000000in}{0.020833in}}%
\pgfpathcurveto{\pgfqpoint{-0.005525in}{0.020833in}}{\pgfqpoint{-0.010825in}{0.018638in}}{\pgfqpoint{-0.014731in}{0.014731in}}%
\pgfpathcurveto{\pgfqpoint{-0.018638in}{0.010825in}}{\pgfqpoint{-0.020833in}{0.005525in}}{\pgfqpoint{-0.020833in}{0.000000in}}%
\pgfpathcurveto{\pgfqpoint{-0.020833in}{-0.005525in}}{\pgfqpoint{-0.018638in}{-0.010825in}}{\pgfqpoint{-0.014731in}{-0.014731in}}%
\pgfpathcurveto{\pgfqpoint{-0.010825in}{-0.018638in}}{\pgfqpoint{-0.005525in}{-0.020833in}}{\pgfqpoint{0.000000in}{-0.020833in}}%
\pgfpathlineto{\pgfqpoint{0.000000in}{-0.020833in}}%
\pgfpathclose%
\pgfusepath{stroke,fill}%
}%
\begin{pgfscope}%
\pgfsys@transformshift{2.312134in}{0.943734in}%
\pgfsys@useobject{currentmarker}{}%
\end{pgfscope}%
\begin{pgfscope}%
\pgfsys@transformshift{2.312134in}{0.943734in}%
\pgfsys@useobject{currentmarker}{}%
\end{pgfscope}%
\begin{pgfscope}%
\pgfsys@transformshift{2.312134in}{0.943734in}%
\pgfsys@useobject{currentmarker}{}%
\end{pgfscope}%
\begin{pgfscope}%
\pgfsys@transformshift{2.312134in}{0.943734in}%
\pgfsys@useobject{currentmarker}{}%
\end{pgfscope}%
\begin{pgfscope}%
\pgfsys@transformshift{2.312134in}{0.943734in}%
\pgfsys@useobject{currentmarker}{}%
\end{pgfscope}%
\begin{pgfscope}%
\pgfsys@transformshift{2.312134in}{0.943734in}%
\pgfsys@useobject{currentmarker}{}%
\end{pgfscope}%
\begin{pgfscope}%
\pgfsys@transformshift{2.312134in}{0.943734in}%
\pgfsys@useobject{currentmarker}{}%
\end{pgfscope}%
\begin{pgfscope}%
\pgfsys@transformshift{2.312134in}{0.943734in}%
\pgfsys@useobject{currentmarker}{}%
\end{pgfscope}%
\begin{pgfscope}%
\pgfsys@transformshift{2.312134in}{0.943734in}%
\pgfsys@useobject{currentmarker}{}%
\end{pgfscope}%
\begin{pgfscope}%
\pgfsys@transformshift{2.312134in}{0.943734in}%
\pgfsys@useobject{currentmarker}{}%
\end{pgfscope}%
\begin{pgfscope}%
\pgfsys@transformshift{2.312134in}{0.943734in}%
\pgfsys@useobject{currentmarker}{}%
\end{pgfscope}%
\begin{pgfscope}%
\pgfsys@transformshift{2.312134in}{0.943734in}%
\pgfsys@useobject{currentmarker}{}%
\end{pgfscope}%
\begin{pgfscope}%
\pgfsys@transformshift{2.312134in}{0.943734in}%
\pgfsys@useobject{currentmarker}{}%
\end{pgfscope}%
\begin{pgfscope}%
\pgfsys@transformshift{2.312134in}{0.943734in}%
\pgfsys@useobject{currentmarker}{}%
\end{pgfscope}%
\begin{pgfscope}%
\pgfsys@transformshift{2.312134in}{0.943734in}%
\pgfsys@useobject{currentmarker}{}%
\end{pgfscope}%
\begin{pgfscope}%
\pgfsys@transformshift{2.312134in}{0.943734in}%
\pgfsys@useobject{currentmarker}{}%
\end{pgfscope}%
\begin{pgfscope}%
\pgfsys@transformshift{2.312134in}{0.943734in}%
\pgfsys@useobject{currentmarker}{}%
\end{pgfscope}%
\begin{pgfscope}%
\pgfsys@transformshift{2.312134in}{0.943734in}%
\pgfsys@useobject{currentmarker}{}%
\end{pgfscope}%
\begin{pgfscope}%
\pgfsys@transformshift{2.312134in}{0.943734in}%
\pgfsys@useobject{currentmarker}{}%
\end{pgfscope}%
\begin{pgfscope}%
\pgfsys@transformshift{2.312134in}{0.943734in}%
\pgfsys@useobject{currentmarker}{}%
\end{pgfscope}%
\begin{pgfscope}%
\pgfsys@transformshift{2.312134in}{0.943734in}%
\pgfsys@useobject{currentmarker}{}%
\end{pgfscope}%
\begin{pgfscope}%
\pgfsys@transformshift{2.312134in}{0.943734in}%
\pgfsys@useobject{currentmarker}{}%
\end{pgfscope}%
\begin{pgfscope}%
\pgfsys@transformshift{2.312134in}{0.943734in}%
\pgfsys@useobject{currentmarker}{}%
\end{pgfscope}%
\begin{pgfscope}%
\pgfsys@transformshift{2.312134in}{0.943734in}%
\pgfsys@useobject{currentmarker}{}%
\end{pgfscope}%
\begin{pgfscope}%
\pgfsys@transformshift{2.312134in}{0.943734in}%
\pgfsys@useobject{currentmarker}{}%
\end{pgfscope}%
\begin{pgfscope}%
\pgfsys@transformshift{2.312134in}{0.943734in}%
\pgfsys@useobject{currentmarker}{}%
\end{pgfscope}%
\begin{pgfscope}%
\pgfsys@transformshift{2.312134in}{0.943734in}%
\pgfsys@useobject{currentmarker}{}%
\end{pgfscope}%
\begin{pgfscope}%
\pgfsys@transformshift{2.312134in}{0.943734in}%
\pgfsys@useobject{currentmarker}{}%
\end{pgfscope}%
\begin{pgfscope}%
\pgfsys@transformshift{2.312134in}{0.943734in}%
\pgfsys@useobject{currentmarker}{}%
\end{pgfscope}%
\begin{pgfscope}%
\pgfsys@transformshift{2.312134in}{0.943734in}%
\pgfsys@useobject{currentmarker}{}%
\end{pgfscope}%
\begin{pgfscope}%
\pgfsys@transformshift{2.312134in}{0.943734in}%
\pgfsys@useobject{currentmarker}{}%
\end{pgfscope}%
\begin{pgfscope}%
\pgfsys@transformshift{2.312134in}{0.943734in}%
\pgfsys@useobject{currentmarker}{}%
\end{pgfscope}%
\begin{pgfscope}%
\pgfsys@transformshift{2.312134in}{0.943734in}%
\pgfsys@useobject{currentmarker}{}%
\end{pgfscope}%
\begin{pgfscope}%
\pgfsys@transformshift{2.312134in}{0.943734in}%
\pgfsys@useobject{currentmarker}{}%
\end{pgfscope}%
\begin{pgfscope}%
\pgfsys@transformshift{2.312134in}{0.943734in}%
\pgfsys@useobject{currentmarker}{}%
\end{pgfscope}%
\begin{pgfscope}%
\pgfsys@transformshift{2.312134in}{0.943734in}%
\pgfsys@useobject{currentmarker}{}%
\end{pgfscope}%
\begin{pgfscope}%
\pgfsys@transformshift{2.312134in}{0.943734in}%
\pgfsys@useobject{currentmarker}{}%
\end{pgfscope}%
\begin{pgfscope}%
\pgfsys@transformshift{2.312134in}{0.943734in}%
\pgfsys@useobject{currentmarker}{}%
\end{pgfscope}%
\begin{pgfscope}%
\pgfsys@transformshift{2.312134in}{0.943734in}%
\pgfsys@useobject{currentmarker}{}%
\end{pgfscope}%
\begin{pgfscope}%
\pgfsys@transformshift{2.312134in}{0.943734in}%
\pgfsys@useobject{currentmarker}{}%
\end{pgfscope}%
\begin{pgfscope}%
\pgfsys@transformshift{2.312134in}{0.943734in}%
\pgfsys@useobject{currentmarker}{}%
\end{pgfscope}%
\begin{pgfscope}%
\pgfsys@transformshift{2.312134in}{0.943734in}%
\pgfsys@useobject{currentmarker}{}%
\end{pgfscope}%
\begin{pgfscope}%
\pgfsys@transformshift{2.312134in}{0.943734in}%
\pgfsys@useobject{currentmarker}{}%
\end{pgfscope}%
\begin{pgfscope}%
\pgfsys@transformshift{2.312134in}{0.943734in}%
\pgfsys@useobject{currentmarker}{}%
\end{pgfscope}%
\begin{pgfscope}%
\pgfsys@transformshift{2.312134in}{0.943734in}%
\pgfsys@useobject{currentmarker}{}%
\end{pgfscope}%
\begin{pgfscope}%
\pgfsys@transformshift{2.312134in}{0.943734in}%
\pgfsys@useobject{currentmarker}{}%
\end{pgfscope}%
\begin{pgfscope}%
\pgfsys@transformshift{2.312134in}{0.943734in}%
\pgfsys@useobject{currentmarker}{}%
\end{pgfscope}%
\begin{pgfscope}%
\pgfsys@transformshift{2.312134in}{0.943734in}%
\pgfsys@useobject{currentmarker}{}%
\end{pgfscope}%
\begin{pgfscope}%
\pgfsys@transformshift{2.312134in}{0.943734in}%
\pgfsys@useobject{currentmarker}{}%
\end{pgfscope}%
\begin{pgfscope}%
\pgfsys@transformshift{2.312134in}{0.943734in}%
\pgfsys@useobject{currentmarker}{}%
\end{pgfscope}%
\begin{pgfscope}%
\pgfsys@transformshift{2.312134in}{0.943734in}%
\pgfsys@useobject{currentmarker}{}%
\end{pgfscope}%
\begin{pgfscope}%
\pgfsys@transformshift{2.312134in}{0.943734in}%
\pgfsys@useobject{currentmarker}{}%
\end{pgfscope}%
\begin{pgfscope}%
\pgfsys@transformshift{2.312134in}{0.943734in}%
\pgfsys@useobject{currentmarker}{}%
\end{pgfscope}%
\begin{pgfscope}%
\pgfsys@transformshift{2.312134in}{0.943734in}%
\pgfsys@useobject{currentmarker}{}%
\end{pgfscope}%
\begin{pgfscope}%
\pgfsys@transformshift{2.312134in}{0.943734in}%
\pgfsys@useobject{currentmarker}{}%
\end{pgfscope}%
\begin{pgfscope}%
\pgfsys@transformshift{2.312134in}{0.943734in}%
\pgfsys@useobject{currentmarker}{}%
\end{pgfscope}%
\begin{pgfscope}%
\pgfsys@transformshift{2.312134in}{0.943734in}%
\pgfsys@useobject{currentmarker}{}%
\end{pgfscope}%
\begin{pgfscope}%
\pgfsys@transformshift{2.312134in}{0.943734in}%
\pgfsys@useobject{currentmarker}{}%
\end{pgfscope}%
\begin{pgfscope}%
\pgfsys@transformshift{2.312134in}{0.943734in}%
\pgfsys@useobject{currentmarker}{}%
\end{pgfscope}%
\begin{pgfscope}%
\pgfsys@transformshift{2.312134in}{0.943734in}%
\pgfsys@useobject{currentmarker}{}%
\end{pgfscope}%
\begin{pgfscope}%
\pgfsys@transformshift{2.312134in}{0.943734in}%
\pgfsys@useobject{currentmarker}{}%
\end{pgfscope}%
\begin{pgfscope}%
\pgfsys@transformshift{2.312134in}{0.943734in}%
\pgfsys@useobject{currentmarker}{}%
\end{pgfscope}%
\begin{pgfscope}%
\pgfsys@transformshift{2.312134in}{0.943734in}%
\pgfsys@useobject{currentmarker}{}%
\end{pgfscope}%
\begin{pgfscope}%
\pgfsys@transformshift{2.312134in}{0.943734in}%
\pgfsys@useobject{currentmarker}{}%
\end{pgfscope}%
\begin{pgfscope}%
\pgfsys@transformshift{2.312134in}{0.943734in}%
\pgfsys@useobject{currentmarker}{}%
\end{pgfscope}%
\begin{pgfscope}%
\pgfsys@transformshift{2.312134in}{0.943734in}%
\pgfsys@useobject{currentmarker}{}%
\end{pgfscope}%
\begin{pgfscope}%
\pgfsys@transformshift{2.312134in}{0.943734in}%
\pgfsys@useobject{currentmarker}{}%
\end{pgfscope}%
\begin{pgfscope}%
\pgfsys@transformshift{2.312134in}{0.943734in}%
\pgfsys@useobject{currentmarker}{}%
\end{pgfscope}%
\begin{pgfscope}%
\pgfsys@transformshift{2.312134in}{0.943734in}%
\pgfsys@useobject{currentmarker}{}%
\end{pgfscope}%
\begin{pgfscope}%
\pgfsys@transformshift{2.312134in}{0.943734in}%
\pgfsys@useobject{currentmarker}{}%
\end{pgfscope}%
\begin{pgfscope}%
\pgfsys@transformshift{2.312134in}{0.943734in}%
\pgfsys@useobject{currentmarker}{}%
\end{pgfscope}%
\begin{pgfscope}%
\pgfsys@transformshift{2.312134in}{0.943734in}%
\pgfsys@useobject{currentmarker}{}%
\end{pgfscope}%
\begin{pgfscope}%
\pgfsys@transformshift{2.312134in}{0.943734in}%
\pgfsys@useobject{currentmarker}{}%
\end{pgfscope}%
\begin{pgfscope}%
\pgfsys@transformshift{2.312134in}{0.943734in}%
\pgfsys@useobject{currentmarker}{}%
\end{pgfscope}%
\begin{pgfscope}%
\pgfsys@transformshift{2.312134in}{0.943734in}%
\pgfsys@useobject{currentmarker}{}%
\end{pgfscope}%
\begin{pgfscope}%
\pgfsys@transformshift{2.312134in}{0.943734in}%
\pgfsys@useobject{currentmarker}{}%
\end{pgfscope}%
\begin{pgfscope}%
\pgfsys@transformshift{2.312134in}{0.943734in}%
\pgfsys@useobject{currentmarker}{}%
\end{pgfscope}%
\begin{pgfscope}%
\pgfsys@transformshift{2.312134in}{0.943734in}%
\pgfsys@useobject{currentmarker}{}%
\end{pgfscope}%
\begin{pgfscope}%
\pgfsys@transformshift{2.312134in}{0.943734in}%
\pgfsys@useobject{currentmarker}{}%
\end{pgfscope}%
\begin{pgfscope}%
\pgfsys@transformshift{2.312134in}{0.943734in}%
\pgfsys@useobject{currentmarker}{}%
\end{pgfscope}%
\begin{pgfscope}%
\pgfsys@transformshift{2.312134in}{0.943734in}%
\pgfsys@useobject{currentmarker}{}%
\end{pgfscope}%
\begin{pgfscope}%
\pgfsys@transformshift{2.312134in}{0.943734in}%
\pgfsys@useobject{currentmarker}{}%
\end{pgfscope}%
\begin{pgfscope}%
\pgfsys@transformshift{2.312134in}{0.943734in}%
\pgfsys@useobject{currentmarker}{}%
\end{pgfscope}%
\begin{pgfscope}%
\pgfsys@transformshift{2.312134in}{0.943734in}%
\pgfsys@useobject{currentmarker}{}%
\end{pgfscope}%
\begin{pgfscope}%
\pgfsys@transformshift{2.312134in}{0.943734in}%
\pgfsys@useobject{currentmarker}{}%
\end{pgfscope}%
\begin{pgfscope}%
\pgfsys@transformshift{2.312134in}{0.943734in}%
\pgfsys@useobject{currentmarker}{}%
\end{pgfscope}%
\begin{pgfscope}%
\pgfsys@transformshift{2.312134in}{0.943734in}%
\pgfsys@useobject{currentmarker}{}%
\end{pgfscope}%
\begin{pgfscope}%
\pgfsys@transformshift{2.312134in}{0.943734in}%
\pgfsys@useobject{currentmarker}{}%
\end{pgfscope}%
\begin{pgfscope}%
\pgfsys@transformshift{2.312134in}{0.943734in}%
\pgfsys@useobject{currentmarker}{}%
\end{pgfscope}%
\begin{pgfscope}%
\pgfsys@transformshift{2.312134in}{0.943734in}%
\pgfsys@useobject{currentmarker}{}%
\end{pgfscope}%
\begin{pgfscope}%
\pgfsys@transformshift{2.312134in}{0.943734in}%
\pgfsys@useobject{currentmarker}{}%
\end{pgfscope}%
\begin{pgfscope}%
\pgfsys@transformshift{2.312134in}{0.943734in}%
\pgfsys@useobject{currentmarker}{}%
\end{pgfscope}%
\begin{pgfscope}%
\pgfsys@transformshift{2.312134in}{0.943734in}%
\pgfsys@useobject{currentmarker}{}%
\end{pgfscope}%
\begin{pgfscope}%
\pgfsys@transformshift{2.312134in}{0.943734in}%
\pgfsys@useobject{currentmarker}{}%
\end{pgfscope}%
\begin{pgfscope}%
\pgfsys@transformshift{2.312134in}{0.943734in}%
\pgfsys@useobject{currentmarker}{}%
\end{pgfscope}%
\begin{pgfscope}%
\pgfsys@transformshift{2.312134in}{0.943734in}%
\pgfsys@useobject{currentmarker}{}%
\end{pgfscope}%
\begin{pgfscope}%
\pgfsys@transformshift{2.312134in}{0.943734in}%
\pgfsys@useobject{currentmarker}{}%
\end{pgfscope}%
\begin{pgfscope}%
\pgfsys@transformshift{2.312134in}{0.943734in}%
\pgfsys@useobject{currentmarker}{}%
\end{pgfscope}%
\begin{pgfscope}%
\pgfsys@transformshift{2.312134in}{0.943734in}%
\pgfsys@useobject{currentmarker}{}%
\end{pgfscope}%
\begin{pgfscope}%
\pgfsys@transformshift{2.312134in}{0.943734in}%
\pgfsys@useobject{currentmarker}{}%
\end{pgfscope}%
\begin{pgfscope}%
\pgfsys@transformshift{2.312134in}{0.943734in}%
\pgfsys@useobject{currentmarker}{}%
\end{pgfscope}%
\begin{pgfscope}%
\pgfsys@transformshift{2.312134in}{0.943734in}%
\pgfsys@useobject{currentmarker}{}%
\end{pgfscope}%
\begin{pgfscope}%
\pgfsys@transformshift{2.312134in}{0.943734in}%
\pgfsys@useobject{currentmarker}{}%
\end{pgfscope}%
\begin{pgfscope}%
\pgfsys@transformshift{2.312134in}{0.943734in}%
\pgfsys@useobject{currentmarker}{}%
\end{pgfscope}%
\begin{pgfscope}%
\pgfsys@transformshift{2.312134in}{0.943734in}%
\pgfsys@useobject{currentmarker}{}%
\end{pgfscope}%
\begin{pgfscope}%
\pgfsys@transformshift{2.312134in}{0.943734in}%
\pgfsys@useobject{currentmarker}{}%
\end{pgfscope}%
\begin{pgfscope}%
\pgfsys@transformshift{2.312134in}{0.943734in}%
\pgfsys@useobject{currentmarker}{}%
\end{pgfscope}%
\begin{pgfscope}%
\pgfsys@transformshift{2.312134in}{0.943734in}%
\pgfsys@useobject{currentmarker}{}%
\end{pgfscope}%
\begin{pgfscope}%
\pgfsys@transformshift{2.312134in}{0.943734in}%
\pgfsys@useobject{currentmarker}{}%
\end{pgfscope}%
\begin{pgfscope}%
\pgfsys@transformshift{2.312134in}{0.943734in}%
\pgfsys@useobject{currentmarker}{}%
\end{pgfscope}%
\begin{pgfscope}%
\pgfsys@transformshift{2.312134in}{0.943734in}%
\pgfsys@useobject{currentmarker}{}%
\end{pgfscope}%
\begin{pgfscope}%
\pgfsys@transformshift{2.312134in}{0.943734in}%
\pgfsys@useobject{currentmarker}{}%
\end{pgfscope}%
\begin{pgfscope}%
\pgfsys@transformshift{2.312134in}{0.943734in}%
\pgfsys@useobject{currentmarker}{}%
\end{pgfscope}%
\begin{pgfscope}%
\pgfsys@transformshift{2.312134in}{0.943734in}%
\pgfsys@useobject{currentmarker}{}%
\end{pgfscope}%
\begin{pgfscope}%
\pgfsys@transformshift{2.312134in}{0.943734in}%
\pgfsys@useobject{currentmarker}{}%
\end{pgfscope}%
\begin{pgfscope}%
\pgfsys@transformshift{2.312134in}{0.943734in}%
\pgfsys@useobject{currentmarker}{}%
\end{pgfscope}%
\begin{pgfscope}%
\pgfsys@transformshift{2.312134in}{0.943734in}%
\pgfsys@useobject{currentmarker}{}%
\end{pgfscope}%
\begin{pgfscope}%
\pgfsys@transformshift{2.312134in}{0.943734in}%
\pgfsys@useobject{currentmarker}{}%
\end{pgfscope}%
\begin{pgfscope}%
\pgfsys@transformshift{2.312134in}{0.943734in}%
\pgfsys@useobject{currentmarker}{}%
\end{pgfscope}%
\begin{pgfscope}%
\pgfsys@transformshift{2.312134in}{0.943734in}%
\pgfsys@useobject{currentmarker}{}%
\end{pgfscope}%
\begin{pgfscope}%
\pgfsys@transformshift{2.312134in}{0.943734in}%
\pgfsys@useobject{currentmarker}{}%
\end{pgfscope}%
\begin{pgfscope}%
\pgfsys@transformshift{2.312134in}{0.943734in}%
\pgfsys@useobject{currentmarker}{}%
\end{pgfscope}%
\begin{pgfscope}%
\pgfsys@transformshift{2.312134in}{0.943734in}%
\pgfsys@useobject{currentmarker}{}%
\end{pgfscope}%
\begin{pgfscope}%
\pgfsys@transformshift{2.312134in}{0.943734in}%
\pgfsys@useobject{currentmarker}{}%
\end{pgfscope}%
\begin{pgfscope}%
\pgfsys@transformshift{2.312134in}{0.943734in}%
\pgfsys@useobject{currentmarker}{}%
\end{pgfscope}%
\begin{pgfscope}%
\pgfsys@transformshift{2.312134in}{0.943734in}%
\pgfsys@useobject{currentmarker}{}%
\end{pgfscope}%
\begin{pgfscope}%
\pgfsys@transformshift{2.312134in}{0.943734in}%
\pgfsys@useobject{currentmarker}{}%
\end{pgfscope}%
\begin{pgfscope}%
\pgfsys@transformshift{2.312134in}{0.943734in}%
\pgfsys@useobject{currentmarker}{}%
\end{pgfscope}%
\begin{pgfscope}%
\pgfsys@transformshift{2.312134in}{0.943734in}%
\pgfsys@useobject{currentmarker}{}%
\end{pgfscope}%
\begin{pgfscope}%
\pgfsys@transformshift{2.312134in}{0.943734in}%
\pgfsys@useobject{currentmarker}{}%
\end{pgfscope}%
\begin{pgfscope}%
\pgfsys@transformshift{2.312134in}{0.943734in}%
\pgfsys@useobject{currentmarker}{}%
\end{pgfscope}%
\begin{pgfscope}%
\pgfsys@transformshift{2.312134in}{0.943734in}%
\pgfsys@useobject{currentmarker}{}%
\end{pgfscope}%
\begin{pgfscope}%
\pgfsys@transformshift{2.312134in}{0.943734in}%
\pgfsys@useobject{currentmarker}{}%
\end{pgfscope}%
\begin{pgfscope}%
\pgfsys@transformshift{2.312134in}{0.943734in}%
\pgfsys@useobject{currentmarker}{}%
\end{pgfscope}%
\begin{pgfscope}%
\pgfsys@transformshift{2.312134in}{0.943734in}%
\pgfsys@useobject{currentmarker}{}%
\end{pgfscope}%
\begin{pgfscope}%
\pgfsys@transformshift{2.312134in}{0.943734in}%
\pgfsys@useobject{currentmarker}{}%
\end{pgfscope}%
\begin{pgfscope}%
\pgfsys@transformshift{2.312134in}{0.943734in}%
\pgfsys@useobject{currentmarker}{}%
\end{pgfscope}%
\begin{pgfscope}%
\pgfsys@transformshift{2.312134in}{0.943734in}%
\pgfsys@useobject{currentmarker}{}%
\end{pgfscope}%
\begin{pgfscope}%
\pgfsys@transformshift{2.312134in}{0.943734in}%
\pgfsys@useobject{currentmarker}{}%
\end{pgfscope}%
\begin{pgfscope}%
\pgfsys@transformshift{2.312134in}{0.943734in}%
\pgfsys@useobject{currentmarker}{}%
\end{pgfscope}%
\begin{pgfscope}%
\pgfsys@transformshift{2.312134in}{0.943734in}%
\pgfsys@useobject{currentmarker}{}%
\end{pgfscope}%
\begin{pgfscope}%
\pgfsys@transformshift{2.312134in}{0.943734in}%
\pgfsys@useobject{currentmarker}{}%
\end{pgfscope}%
\begin{pgfscope}%
\pgfsys@transformshift{2.312134in}{0.943734in}%
\pgfsys@useobject{currentmarker}{}%
\end{pgfscope}%
\begin{pgfscope}%
\pgfsys@transformshift{2.312134in}{0.943734in}%
\pgfsys@useobject{currentmarker}{}%
\end{pgfscope}%
\begin{pgfscope}%
\pgfsys@transformshift{2.312134in}{0.943734in}%
\pgfsys@useobject{currentmarker}{}%
\end{pgfscope}%
\begin{pgfscope}%
\pgfsys@transformshift{2.312134in}{0.943734in}%
\pgfsys@useobject{currentmarker}{}%
\end{pgfscope}%
\begin{pgfscope}%
\pgfsys@transformshift{2.312134in}{0.943734in}%
\pgfsys@useobject{currentmarker}{}%
\end{pgfscope}%
\begin{pgfscope}%
\pgfsys@transformshift{2.312134in}{0.943734in}%
\pgfsys@useobject{currentmarker}{}%
\end{pgfscope}%
\begin{pgfscope}%
\pgfsys@transformshift{2.312134in}{0.943734in}%
\pgfsys@useobject{currentmarker}{}%
\end{pgfscope}%
\begin{pgfscope}%
\pgfsys@transformshift{2.312134in}{0.943734in}%
\pgfsys@useobject{currentmarker}{}%
\end{pgfscope}%
\begin{pgfscope}%
\pgfsys@transformshift{2.312134in}{0.943734in}%
\pgfsys@useobject{currentmarker}{}%
\end{pgfscope}%
\begin{pgfscope}%
\pgfsys@transformshift{2.312134in}{0.943734in}%
\pgfsys@useobject{currentmarker}{}%
\end{pgfscope}%
\begin{pgfscope}%
\pgfsys@transformshift{2.312134in}{0.943734in}%
\pgfsys@useobject{currentmarker}{}%
\end{pgfscope}%
\begin{pgfscope}%
\pgfsys@transformshift{2.312134in}{0.943734in}%
\pgfsys@useobject{currentmarker}{}%
\end{pgfscope}%
\begin{pgfscope}%
\pgfsys@transformshift{2.312134in}{0.943734in}%
\pgfsys@useobject{currentmarker}{}%
\end{pgfscope}%
\begin{pgfscope}%
\pgfsys@transformshift{2.312134in}{0.943734in}%
\pgfsys@useobject{currentmarker}{}%
\end{pgfscope}%
\begin{pgfscope}%
\pgfsys@transformshift{2.312134in}{0.943734in}%
\pgfsys@useobject{currentmarker}{}%
\end{pgfscope}%
\begin{pgfscope}%
\pgfsys@transformshift{2.312134in}{0.943734in}%
\pgfsys@useobject{currentmarker}{}%
\end{pgfscope}%
\begin{pgfscope}%
\pgfsys@transformshift{2.312134in}{0.943734in}%
\pgfsys@useobject{currentmarker}{}%
\end{pgfscope}%
\begin{pgfscope}%
\pgfsys@transformshift{2.312134in}{0.943734in}%
\pgfsys@useobject{currentmarker}{}%
\end{pgfscope}%
\begin{pgfscope}%
\pgfsys@transformshift{2.312134in}{0.943734in}%
\pgfsys@useobject{currentmarker}{}%
\end{pgfscope}%
\begin{pgfscope}%
\pgfsys@transformshift{2.312134in}{0.943734in}%
\pgfsys@useobject{currentmarker}{}%
\end{pgfscope}%
\begin{pgfscope}%
\pgfsys@transformshift{2.312134in}{0.943734in}%
\pgfsys@useobject{currentmarker}{}%
\end{pgfscope}%
\begin{pgfscope}%
\pgfsys@transformshift{2.312134in}{0.943734in}%
\pgfsys@useobject{currentmarker}{}%
\end{pgfscope}%
\begin{pgfscope}%
\pgfsys@transformshift{2.312134in}{0.943734in}%
\pgfsys@useobject{currentmarker}{}%
\end{pgfscope}%
\begin{pgfscope}%
\pgfsys@transformshift{2.312134in}{0.943734in}%
\pgfsys@useobject{currentmarker}{}%
\end{pgfscope}%
\begin{pgfscope}%
\pgfsys@transformshift{2.312134in}{0.943734in}%
\pgfsys@useobject{currentmarker}{}%
\end{pgfscope}%
\begin{pgfscope}%
\pgfsys@transformshift{2.312134in}{0.943734in}%
\pgfsys@useobject{currentmarker}{}%
\end{pgfscope}%
\begin{pgfscope}%
\pgfsys@transformshift{2.312134in}{0.943734in}%
\pgfsys@useobject{currentmarker}{}%
\end{pgfscope}%
\begin{pgfscope}%
\pgfsys@transformshift{2.312134in}{0.943734in}%
\pgfsys@useobject{currentmarker}{}%
\end{pgfscope}%
\begin{pgfscope}%
\pgfsys@transformshift{2.312134in}{0.943734in}%
\pgfsys@useobject{currentmarker}{}%
\end{pgfscope}%
\begin{pgfscope}%
\pgfsys@transformshift{2.312134in}{0.943734in}%
\pgfsys@useobject{currentmarker}{}%
\end{pgfscope}%
\begin{pgfscope}%
\pgfsys@transformshift{2.312134in}{0.943734in}%
\pgfsys@useobject{currentmarker}{}%
\end{pgfscope}%
\begin{pgfscope}%
\pgfsys@transformshift{2.312134in}{0.943734in}%
\pgfsys@useobject{currentmarker}{}%
\end{pgfscope}%
\begin{pgfscope}%
\pgfsys@transformshift{2.312134in}{0.943734in}%
\pgfsys@useobject{currentmarker}{}%
\end{pgfscope}%
\begin{pgfscope}%
\pgfsys@transformshift{2.312134in}{0.943734in}%
\pgfsys@useobject{currentmarker}{}%
\end{pgfscope}%
\begin{pgfscope}%
\pgfsys@transformshift{2.312134in}{0.943734in}%
\pgfsys@useobject{currentmarker}{}%
\end{pgfscope}%
\begin{pgfscope}%
\pgfsys@transformshift{2.312134in}{0.943734in}%
\pgfsys@useobject{currentmarker}{}%
\end{pgfscope}%
\begin{pgfscope}%
\pgfsys@transformshift{2.312134in}{0.943734in}%
\pgfsys@useobject{currentmarker}{}%
\end{pgfscope}%
\begin{pgfscope}%
\pgfsys@transformshift{2.312134in}{0.943734in}%
\pgfsys@useobject{currentmarker}{}%
\end{pgfscope}%
\begin{pgfscope}%
\pgfsys@transformshift{2.312134in}{0.943734in}%
\pgfsys@useobject{currentmarker}{}%
\end{pgfscope}%
\begin{pgfscope}%
\pgfsys@transformshift{2.312134in}{0.943734in}%
\pgfsys@useobject{currentmarker}{}%
\end{pgfscope}%
\begin{pgfscope}%
\pgfsys@transformshift{2.312134in}{0.943734in}%
\pgfsys@useobject{currentmarker}{}%
\end{pgfscope}%
\begin{pgfscope}%
\pgfsys@transformshift{2.312134in}{0.943734in}%
\pgfsys@useobject{currentmarker}{}%
\end{pgfscope}%
\begin{pgfscope}%
\pgfsys@transformshift{2.312134in}{0.943734in}%
\pgfsys@useobject{currentmarker}{}%
\end{pgfscope}%
\begin{pgfscope}%
\pgfsys@transformshift{2.312134in}{0.943734in}%
\pgfsys@useobject{currentmarker}{}%
\end{pgfscope}%
\begin{pgfscope}%
\pgfsys@transformshift{2.312134in}{0.943734in}%
\pgfsys@useobject{currentmarker}{}%
\end{pgfscope}%
\begin{pgfscope}%
\pgfsys@transformshift{2.312134in}{0.943734in}%
\pgfsys@useobject{currentmarker}{}%
\end{pgfscope}%
\begin{pgfscope}%
\pgfsys@transformshift{2.312134in}{0.943734in}%
\pgfsys@useobject{currentmarker}{}%
\end{pgfscope}%
\begin{pgfscope}%
\pgfsys@transformshift{2.312134in}{0.943734in}%
\pgfsys@useobject{currentmarker}{}%
\end{pgfscope}%
\begin{pgfscope}%
\pgfsys@transformshift{2.312134in}{0.943734in}%
\pgfsys@useobject{currentmarker}{}%
\end{pgfscope}%
\begin{pgfscope}%
\pgfsys@transformshift{2.312134in}{0.943734in}%
\pgfsys@useobject{currentmarker}{}%
\end{pgfscope}%
\begin{pgfscope}%
\pgfsys@transformshift{2.312134in}{0.943734in}%
\pgfsys@useobject{currentmarker}{}%
\end{pgfscope}%
\begin{pgfscope}%
\pgfsys@transformshift{2.312134in}{0.943734in}%
\pgfsys@useobject{currentmarker}{}%
\end{pgfscope}%
\begin{pgfscope}%
\pgfsys@transformshift{2.312134in}{0.943734in}%
\pgfsys@useobject{currentmarker}{}%
\end{pgfscope}%
\begin{pgfscope}%
\pgfsys@transformshift{2.312134in}{0.943734in}%
\pgfsys@useobject{currentmarker}{}%
\end{pgfscope}%
\begin{pgfscope}%
\pgfsys@transformshift{2.312134in}{0.943734in}%
\pgfsys@useobject{currentmarker}{}%
\end{pgfscope}%
\begin{pgfscope}%
\pgfsys@transformshift{2.312134in}{0.943734in}%
\pgfsys@useobject{currentmarker}{}%
\end{pgfscope}%
\begin{pgfscope}%
\pgfsys@transformshift{2.312134in}{0.943734in}%
\pgfsys@useobject{currentmarker}{}%
\end{pgfscope}%
\begin{pgfscope}%
\pgfsys@transformshift{2.312134in}{0.943734in}%
\pgfsys@useobject{currentmarker}{}%
\end{pgfscope}%
\begin{pgfscope}%
\pgfsys@transformshift{2.312134in}{0.943734in}%
\pgfsys@useobject{currentmarker}{}%
\end{pgfscope}%
\begin{pgfscope}%
\pgfsys@transformshift{2.312134in}{0.943734in}%
\pgfsys@useobject{currentmarker}{}%
\end{pgfscope}%
\begin{pgfscope}%
\pgfsys@transformshift{2.312134in}{0.943734in}%
\pgfsys@useobject{currentmarker}{}%
\end{pgfscope}%
\begin{pgfscope}%
\pgfsys@transformshift{2.312134in}{0.943734in}%
\pgfsys@useobject{currentmarker}{}%
\end{pgfscope}%
\begin{pgfscope}%
\pgfsys@transformshift{2.312134in}{0.943734in}%
\pgfsys@useobject{currentmarker}{}%
\end{pgfscope}%
\begin{pgfscope}%
\pgfsys@transformshift{2.312134in}{0.943734in}%
\pgfsys@useobject{currentmarker}{}%
\end{pgfscope}%
\begin{pgfscope}%
\pgfsys@transformshift{2.312134in}{0.943734in}%
\pgfsys@useobject{currentmarker}{}%
\end{pgfscope}%
\begin{pgfscope}%
\pgfsys@transformshift{2.312134in}{0.943734in}%
\pgfsys@useobject{currentmarker}{}%
\end{pgfscope}%
\begin{pgfscope}%
\pgfsys@transformshift{2.312134in}{0.943734in}%
\pgfsys@useobject{currentmarker}{}%
\end{pgfscope}%
\begin{pgfscope}%
\pgfsys@transformshift{2.312134in}{0.943734in}%
\pgfsys@useobject{currentmarker}{}%
\end{pgfscope}%
\begin{pgfscope}%
\pgfsys@transformshift{2.312134in}{0.943734in}%
\pgfsys@useobject{currentmarker}{}%
\end{pgfscope}%
\begin{pgfscope}%
\pgfsys@transformshift{2.312134in}{0.943734in}%
\pgfsys@useobject{currentmarker}{}%
\end{pgfscope}%
\begin{pgfscope}%
\pgfsys@transformshift{2.312134in}{0.943734in}%
\pgfsys@useobject{currentmarker}{}%
\end{pgfscope}%
\begin{pgfscope}%
\pgfsys@transformshift{2.312134in}{0.943734in}%
\pgfsys@useobject{currentmarker}{}%
\end{pgfscope}%
\begin{pgfscope}%
\pgfsys@transformshift{2.312134in}{0.943734in}%
\pgfsys@useobject{currentmarker}{}%
\end{pgfscope}%
\begin{pgfscope}%
\pgfsys@transformshift{2.312134in}{0.943734in}%
\pgfsys@useobject{currentmarker}{}%
\end{pgfscope}%
\begin{pgfscope}%
\pgfsys@transformshift{2.312134in}{0.943734in}%
\pgfsys@useobject{currentmarker}{}%
\end{pgfscope}%
\begin{pgfscope}%
\pgfsys@transformshift{2.312134in}{0.943734in}%
\pgfsys@useobject{currentmarker}{}%
\end{pgfscope}%
\begin{pgfscope}%
\pgfsys@transformshift{2.312134in}{0.943734in}%
\pgfsys@useobject{currentmarker}{}%
\end{pgfscope}%
\begin{pgfscope}%
\pgfsys@transformshift{2.312134in}{0.943734in}%
\pgfsys@useobject{currentmarker}{}%
\end{pgfscope}%
\begin{pgfscope}%
\pgfsys@transformshift{2.312134in}{0.943734in}%
\pgfsys@useobject{currentmarker}{}%
\end{pgfscope}%
\begin{pgfscope}%
\pgfsys@transformshift{2.312134in}{0.943734in}%
\pgfsys@useobject{currentmarker}{}%
\end{pgfscope}%
\begin{pgfscope}%
\pgfsys@transformshift{2.312134in}{0.943734in}%
\pgfsys@useobject{currentmarker}{}%
\end{pgfscope}%
\begin{pgfscope}%
\pgfsys@transformshift{2.312134in}{0.943734in}%
\pgfsys@useobject{currentmarker}{}%
\end{pgfscope}%
\begin{pgfscope}%
\pgfsys@transformshift{2.312134in}{0.943734in}%
\pgfsys@useobject{currentmarker}{}%
\end{pgfscope}%
\begin{pgfscope}%
\pgfsys@transformshift{2.312134in}{0.943734in}%
\pgfsys@useobject{currentmarker}{}%
\end{pgfscope}%
\begin{pgfscope}%
\pgfsys@transformshift{2.312134in}{0.943734in}%
\pgfsys@useobject{currentmarker}{}%
\end{pgfscope}%
\begin{pgfscope}%
\pgfsys@transformshift{2.312134in}{0.943734in}%
\pgfsys@useobject{currentmarker}{}%
\end{pgfscope}%
\begin{pgfscope}%
\pgfsys@transformshift{2.312134in}{0.943734in}%
\pgfsys@useobject{currentmarker}{}%
\end{pgfscope}%
\begin{pgfscope}%
\pgfsys@transformshift{2.312134in}{0.943734in}%
\pgfsys@useobject{currentmarker}{}%
\end{pgfscope}%
\begin{pgfscope}%
\pgfsys@transformshift{2.312134in}{0.943734in}%
\pgfsys@useobject{currentmarker}{}%
\end{pgfscope}%
\begin{pgfscope}%
\pgfsys@transformshift{2.312134in}{0.943734in}%
\pgfsys@useobject{currentmarker}{}%
\end{pgfscope}%
\begin{pgfscope}%
\pgfsys@transformshift{2.312134in}{0.943734in}%
\pgfsys@useobject{currentmarker}{}%
\end{pgfscope}%
\begin{pgfscope}%
\pgfsys@transformshift{2.312134in}{0.943734in}%
\pgfsys@useobject{currentmarker}{}%
\end{pgfscope}%
\begin{pgfscope}%
\pgfsys@transformshift{2.312134in}{0.943734in}%
\pgfsys@useobject{currentmarker}{}%
\end{pgfscope}%
\begin{pgfscope}%
\pgfsys@transformshift{2.312134in}{0.943734in}%
\pgfsys@useobject{currentmarker}{}%
\end{pgfscope}%
\begin{pgfscope}%
\pgfsys@transformshift{2.312134in}{0.943734in}%
\pgfsys@useobject{currentmarker}{}%
\end{pgfscope}%
\begin{pgfscope}%
\pgfsys@transformshift{2.312134in}{0.943734in}%
\pgfsys@useobject{currentmarker}{}%
\end{pgfscope}%
\begin{pgfscope}%
\pgfsys@transformshift{2.312134in}{0.943734in}%
\pgfsys@useobject{currentmarker}{}%
\end{pgfscope}%
\begin{pgfscope}%
\pgfsys@transformshift{2.312134in}{0.943734in}%
\pgfsys@useobject{currentmarker}{}%
\end{pgfscope}%
\begin{pgfscope}%
\pgfsys@transformshift{2.312134in}{0.943734in}%
\pgfsys@useobject{currentmarker}{}%
\end{pgfscope}%
\begin{pgfscope}%
\pgfsys@transformshift{2.312134in}{0.943734in}%
\pgfsys@useobject{currentmarker}{}%
\end{pgfscope}%
\begin{pgfscope}%
\pgfsys@transformshift{2.312134in}{0.943734in}%
\pgfsys@useobject{currentmarker}{}%
\end{pgfscope}%
\begin{pgfscope}%
\pgfsys@transformshift{2.312134in}{0.943734in}%
\pgfsys@useobject{currentmarker}{}%
\end{pgfscope}%
\begin{pgfscope}%
\pgfsys@transformshift{2.312134in}{0.943734in}%
\pgfsys@useobject{currentmarker}{}%
\end{pgfscope}%
\begin{pgfscope}%
\pgfsys@transformshift{2.312134in}{0.943734in}%
\pgfsys@useobject{currentmarker}{}%
\end{pgfscope}%
\begin{pgfscope}%
\pgfsys@transformshift{2.312134in}{0.943734in}%
\pgfsys@useobject{currentmarker}{}%
\end{pgfscope}%
\begin{pgfscope}%
\pgfsys@transformshift{2.312134in}{0.943734in}%
\pgfsys@useobject{currentmarker}{}%
\end{pgfscope}%
\begin{pgfscope}%
\pgfsys@transformshift{2.312134in}{0.943734in}%
\pgfsys@useobject{currentmarker}{}%
\end{pgfscope}%
\begin{pgfscope}%
\pgfsys@transformshift{2.312134in}{0.943734in}%
\pgfsys@useobject{currentmarker}{}%
\end{pgfscope}%
\begin{pgfscope}%
\pgfsys@transformshift{2.312134in}{0.943734in}%
\pgfsys@useobject{currentmarker}{}%
\end{pgfscope}%
\begin{pgfscope}%
\pgfsys@transformshift{2.312134in}{0.943734in}%
\pgfsys@useobject{currentmarker}{}%
\end{pgfscope}%
\begin{pgfscope}%
\pgfsys@transformshift{2.312134in}{0.943734in}%
\pgfsys@useobject{currentmarker}{}%
\end{pgfscope}%
\begin{pgfscope}%
\pgfsys@transformshift{2.312134in}{0.943734in}%
\pgfsys@useobject{currentmarker}{}%
\end{pgfscope}%
\begin{pgfscope}%
\pgfsys@transformshift{2.312134in}{0.943734in}%
\pgfsys@useobject{currentmarker}{}%
\end{pgfscope}%
\begin{pgfscope}%
\pgfsys@transformshift{2.312134in}{0.943734in}%
\pgfsys@useobject{currentmarker}{}%
\end{pgfscope}%
\begin{pgfscope}%
\pgfsys@transformshift{2.312134in}{0.943734in}%
\pgfsys@useobject{currentmarker}{}%
\end{pgfscope}%
\begin{pgfscope}%
\pgfsys@transformshift{2.312134in}{0.943734in}%
\pgfsys@useobject{currentmarker}{}%
\end{pgfscope}%
\begin{pgfscope}%
\pgfsys@transformshift{2.312134in}{0.943734in}%
\pgfsys@useobject{currentmarker}{}%
\end{pgfscope}%
\begin{pgfscope}%
\pgfsys@transformshift{2.312134in}{0.943734in}%
\pgfsys@useobject{currentmarker}{}%
\end{pgfscope}%
\begin{pgfscope}%
\pgfsys@transformshift{2.312134in}{0.943734in}%
\pgfsys@useobject{currentmarker}{}%
\end{pgfscope}%
\begin{pgfscope}%
\pgfsys@transformshift{2.312134in}{0.943734in}%
\pgfsys@useobject{currentmarker}{}%
\end{pgfscope}%
\begin{pgfscope}%
\pgfsys@transformshift{2.312134in}{0.943734in}%
\pgfsys@useobject{currentmarker}{}%
\end{pgfscope}%
\begin{pgfscope}%
\pgfsys@transformshift{2.312134in}{0.943734in}%
\pgfsys@useobject{currentmarker}{}%
\end{pgfscope}%
\begin{pgfscope}%
\pgfsys@transformshift{2.312134in}{0.943734in}%
\pgfsys@useobject{currentmarker}{}%
\end{pgfscope}%
\begin{pgfscope}%
\pgfsys@transformshift{2.312134in}{0.943734in}%
\pgfsys@useobject{currentmarker}{}%
\end{pgfscope}%
\begin{pgfscope}%
\pgfsys@transformshift{2.312134in}{0.943734in}%
\pgfsys@useobject{currentmarker}{}%
\end{pgfscope}%
\begin{pgfscope}%
\pgfsys@transformshift{2.312134in}{0.943734in}%
\pgfsys@useobject{currentmarker}{}%
\end{pgfscope}%
\begin{pgfscope}%
\pgfsys@transformshift{2.312134in}{0.943734in}%
\pgfsys@useobject{currentmarker}{}%
\end{pgfscope}%
\begin{pgfscope}%
\pgfsys@transformshift{2.312134in}{0.943734in}%
\pgfsys@useobject{currentmarker}{}%
\end{pgfscope}%
\begin{pgfscope}%
\pgfsys@transformshift{2.312134in}{0.943734in}%
\pgfsys@useobject{currentmarker}{}%
\end{pgfscope}%
\begin{pgfscope}%
\pgfsys@transformshift{2.312134in}{0.943734in}%
\pgfsys@useobject{currentmarker}{}%
\end{pgfscope}%
\begin{pgfscope}%
\pgfsys@transformshift{2.312134in}{0.943734in}%
\pgfsys@useobject{currentmarker}{}%
\end{pgfscope}%
\begin{pgfscope}%
\pgfsys@transformshift{2.312134in}{0.943734in}%
\pgfsys@useobject{currentmarker}{}%
\end{pgfscope}%
\begin{pgfscope}%
\pgfsys@transformshift{2.312134in}{0.943734in}%
\pgfsys@useobject{currentmarker}{}%
\end{pgfscope}%
\begin{pgfscope}%
\pgfsys@transformshift{2.312134in}{0.943734in}%
\pgfsys@useobject{currentmarker}{}%
\end{pgfscope}%
\begin{pgfscope}%
\pgfsys@transformshift{2.312134in}{0.943734in}%
\pgfsys@useobject{currentmarker}{}%
\end{pgfscope}%
\begin{pgfscope}%
\pgfsys@transformshift{2.312134in}{0.943734in}%
\pgfsys@useobject{currentmarker}{}%
\end{pgfscope}%
\begin{pgfscope}%
\pgfsys@transformshift{2.312134in}{0.943734in}%
\pgfsys@useobject{currentmarker}{}%
\end{pgfscope}%
\begin{pgfscope}%
\pgfsys@transformshift{2.312134in}{0.943734in}%
\pgfsys@useobject{currentmarker}{}%
\end{pgfscope}%
\begin{pgfscope}%
\pgfsys@transformshift{2.312134in}{0.943734in}%
\pgfsys@useobject{currentmarker}{}%
\end{pgfscope}%
\begin{pgfscope}%
\pgfsys@transformshift{2.312134in}{0.943734in}%
\pgfsys@useobject{currentmarker}{}%
\end{pgfscope}%
\begin{pgfscope}%
\pgfsys@transformshift{2.312134in}{0.943734in}%
\pgfsys@useobject{currentmarker}{}%
\end{pgfscope}%
\begin{pgfscope}%
\pgfsys@transformshift{2.312134in}{0.943734in}%
\pgfsys@useobject{currentmarker}{}%
\end{pgfscope}%
\begin{pgfscope}%
\pgfsys@transformshift{2.312134in}{0.943734in}%
\pgfsys@useobject{currentmarker}{}%
\end{pgfscope}%
\begin{pgfscope}%
\pgfsys@transformshift{2.312134in}{0.943734in}%
\pgfsys@useobject{currentmarker}{}%
\end{pgfscope}%
\begin{pgfscope}%
\pgfsys@transformshift{2.312134in}{0.943734in}%
\pgfsys@useobject{currentmarker}{}%
\end{pgfscope}%
\begin{pgfscope}%
\pgfsys@transformshift{2.312134in}{0.943734in}%
\pgfsys@useobject{currentmarker}{}%
\end{pgfscope}%
\begin{pgfscope}%
\pgfsys@transformshift{2.312134in}{0.943734in}%
\pgfsys@useobject{currentmarker}{}%
\end{pgfscope}%
\begin{pgfscope}%
\pgfsys@transformshift{2.312134in}{0.943734in}%
\pgfsys@useobject{currentmarker}{}%
\end{pgfscope}%
\begin{pgfscope}%
\pgfsys@transformshift{2.312134in}{0.943734in}%
\pgfsys@useobject{currentmarker}{}%
\end{pgfscope}%
\begin{pgfscope}%
\pgfsys@transformshift{2.312134in}{0.943734in}%
\pgfsys@useobject{currentmarker}{}%
\end{pgfscope}%
\begin{pgfscope}%
\pgfsys@transformshift{2.312134in}{0.943734in}%
\pgfsys@useobject{currentmarker}{}%
\end{pgfscope}%
\begin{pgfscope}%
\pgfsys@transformshift{2.312134in}{0.943734in}%
\pgfsys@useobject{currentmarker}{}%
\end{pgfscope}%
\begin{pgfscope}%
\pgfsys@transformshift{2.312134in}{0.943734in}%
\pgfsys@useobject{currentmarker}{}%
\end{pgfscope}%
\begin{pgfscope}%
\pgfsys@transformshift{2.312134in}{0.943734in}%
\pgfsys@useobject{currentmarker}{}%
\end{pgfscope}%
\begin{pgfscope}%
\pgfsys@transformshift{2.312134in}{0.943734in}%
\pgfsys@useobject{currentmarker}{}%
\end{pgfscope}%
\begin{pgfscope}%
\pgfsys@transformshift{2.312134in}{0.943734in}%
\pgfsys@useobject{currentmarker}{}%
\end{pgfscope}%
\begin{pgfscope}%
\pgfsys@transformshift{2.312134in}{0.943734in}%
\pgfsys@useobject{currentmarker}{}%
\end{pgfscope}%
\begin{pgfscope}%
\pgfsys@transformshift{2.312134in}{0.943734in}%
\pgfsys@useobject{currentmarker}{}%
\end{pgfscope}%
\begin{pgfscope}%
\pgfsys@transformshift{2.312134in}{0.943734in}%
\pgfsys@useobject{currentmarker}{}%
\end{pgfscope}%
\begin{pgfscope}%
\pgfsys@transformshift{2.312134in}{0.943734in}%
\pgfsys@useobject{currentmarker}{}%
\end{pgfscope}%
\begin{pgfscope}%
\pgfsys@transformshift{2.312134in}{0.943734in}%
\pgfsys@useobject{currentmarker}{}%
\end{pgfscope}%
\begin{pgfscope}%
\pgfsys@transformshift{2.312134in}{0.943734in}%
\pgfsys@useobject{currentmarker}{}%
\end{pgfscope}%
\begin{pgfscope}%
\pgfsys@transformshift{2.312134in}{0.943734in}%
\pgfsys@useobject{currentmarker}{}%
\end{pgfscope}%
\begin{pgfscope}%
\pgfsys@transformshift{2.312134in}{0.943734in}%
\pgfsys@useobject{currentmarker}{}%
\end{pgfscope}%
\begin{pgfscope}%
\pgfsys@transformshift{2.312134in}{0.943734in}%
\pgfsys@useobject{currentmarker}{}%
\end{pgfscope}%
\begin{pgfscope}%
\pgfsys@transformshift{2.312134in}{0.943734in}%
\pgfsys@useobject{currentmarker}{}%
\end{pgfscope}%
\begin{pgfscope}%
\pgfsys@transformshift{2.312134in}{0.943734in}%
\pgfsys@useobject{currentmarker}{}%
\end{pgfscope}%
\begin{pgfscope}%
\pgfsys@transformshift{2.312134in}{0.943734in}%
\pgfsys@useobject{currentmarker}{}%
\end{pgfscope}%
\begin{pgfscope}%
\pgfsys@transformshift{2.312134in}{0.943734in}%
\pgfsys@useobject{currentmarker}{}%
\end{pgfscope}%
\begin{pgfscope}%
\pgfsys@transformshift{2.312134in}{0.943734in}%
\pgfsys@useobject{currentmarker}{}%
\end{pgfscope}%
\begin{pgfscope}%
\pgfsys@transformshift{2.312134in}{0.943734in}%
\pgfsys@useobject{currentmarker}{}%
\end{pgfscope}%
\begin{pgfscope}%
\pgfsys@transformshift{2.312134in}{0.943734in}%
\pgfsys@useobject{currentmarker}{}%
\end{pgfscope}%
\begin{pgfscope}%
\pgfsys@transformshift{2.312134in}{0.943734in}%
\pgfsys@useobject{currentmarker}{}%
\end{pgfscope}%
\begin{pgfscope}%
\pgfsys@transformshift{2.312134in}{0.943734in}%
\pgfsys@useobject{currentmarker}{}%
\end{pgfscope}%
\begin{pgfscope}%
\pgfsys@transformshift{2.312134in}{0.943734in}%
\pgfsys@useobject{currentmarker}{}%
\end{pgfscope}%
\begin{pgfscope}%
\pgfsys@transformshift{2.312134in}{0.943734in}%
\pgfsys@useobject{currentmarker}{}%
\end{pgfscope}%
\begin{pgfscope}%
\pgfsys@transformshift{2.312134in}{0.943734in}%
\pgfsys@useobject{currentmarker}{}%
\end{pgfscope}%
\begin{pgfscope}%
\pgfsys@transformshift{2.312134in}{0.943734in}%
\pgfsys@useobject{currentmarker}{}%
\end{pgfscope}%
\begin{pgfscope}%
\pgfsys@transformshift{2.312134in}{0.943734in}%
\pgfsys@useobject{currentmarker}{}%
\end{pgfscope}%
\begin{pgfscope}%
\pgfsys@transformshift{2.312134in}{0.943734in}%
\pgfsys@useobject{currentmarker}{}%
\end{pgfscope}%
\begin{pgfscope}%
\pgfsys@transformshift{2.312134in}{0.943734in}%
\pgfsys@useobject{currentmarker}{}%
\end{pgfscope}%
\begin{pgfscope}%
\pgfsys@transformshift{2.312134in}{0.943734in}%
\pgfsys@useobject{currentmarker}{}%
\end{pgfscope}%
\begin{pgfscope}%
\pgfsys@transformshift{2.312134in}{0.943734in}%
\pgfsys@useobject{currentmarker}{}%
\end{pgfscope}%
\begin{pgfscope}%
\pgfsys@transformshift{2.312134in}{0.943734in}%
\pgfsys@useobject{currentmarker}{}%
\end{pgfscope}%
\begin{pgfscope}%
\pgfsys@transformshift{2.312134in}{0.943734in}%
\pgfsys@useobject{currentmarker}{}%
\end{pgfscope}%
\begin{pgfscope}%
\pgfsys@transformshift{2.312134in}{0.943734in}%
\pgfsys@useobject{currentmarker}{}%
\end{pgfscope}%
\begin{pgfscope}%
\pgfsys@transformshift{2.312134in}{0.943734in}%
\pgfsys@useobject{currentmarker}{}%
\end{pgfscope}%
\begin{pgfscope}%
\pgfsys@transformshift{2.312134in}{0.943734in}%
\pgfsys@useobject{currentmarker}{}%
\end{pgfscope}%
\begin{pgfscope}%
\pgfsys@transformshift{2.312134in}{0.943734in}%
\pgfsys@useobject{currentmarker}{}%
\end{pgfscope}%
\begin{pgfscope}%
\pgfsys@transformshift{2.312134in}{0.943734in}%
\pgfsys@useobject{currentmarker}{}%
\end{pgfscope}%
\begin{pgfscope}%
\pgfsys@transformshift{2.312134in}{0.943734in}%
\pgfsys@useobject{currentmarker}{}%
\end{pgfscope}%
\begin{pgfscope}%
\pgfsys@transformshift{2.312134in}{0.943734in}%
\pgfsys@useobject{currentmarker}{}%
\end{pgfscope}%
\begin{pgfscope}%
\pgfsys@transformshift{2.312134in}{0.943734in}%
\pgfsys@useobject{currentmarker}{}%
\end{pgfscope}%
\begin{pgfscope}%
\pgfsys@transformshift{2.312134in}{0.943734in}%
\pgfsys@useobject{currentmarker}{}%
\end{pgfscope}%
\begin{pgfscope}%
\pgfsys@transformshift{2.312134in}{0.943734in}%
\pgfsys@useobject{currentmarker}{}%
\end{pgfscope}%
\begin{pgfscope}%
\pgfsys@transformshift{2.312134in}{0.943734in}%
\pgfsys@useobject{currentmarker}{}%
\end{pgfscope}%
\begin{pgfscope}%
\pgfsys@transformshift{2.312134in}{0.943734in}%
\pgfsys@useobject{currentmarker}{}%
\end{pgfscope}%
\begin{pgfscope}%
\pgfsys@transformshift{2.312134in}{0.943734in}%
\pgfsys@useobject{currentmarker}{}%
\end{pgfscope}%
\begin{pgfscope}%
\pgfsys@transformshift{2.312134in}{0.943734in}%
\pgfsys@useobject{currentmarker}{}%
\end{pgfscope}%
\begin{pgfscope}%
\pgfsys@transformshift{2.312134in}{0.943734in}%
\pgfsys@useobject{currentmarker}{}%
\end{pgfscope}%
\begin{pgfscope}%
\pgfsys@transformshift{2.312134in}{0.943734in}%
\pgfsys@useobject{currentmarker}{}%
\end{pgfscope}%
\begin{pgfscope}%
\pgfsys@transformshift{2.312134in}{0.943734in}%
\pgfsys@useobject{currentmarker}{}%
\end{pgfscope}%
\begin{pgfscope}%
\pgfsys@transformshift{2.312134in}{0.943734in}%
\pgfsys@useobject{currentmarker}{}%
\end{pgfscope}%
\begin{pgfscope}%
\pgfsys@transformshift{2.312134in}{0.943734in}%
\pgfsys@useobject{currentmarker}{}%
\end{pgfscope}%
\begin{pgfscope}%
\pgfsys@transformshift{2.312134in}{0.943734in}%
\pgfsys@useobject{currentmarker}{}%
\end{pgfscope}%
\begin{pgfscope}%
\pgfsys@transformshift{2.312134in}{0.943734in}%
\pgfsys@useobject{currentmarker}{}%
\end{pgfscope}%
\begin{pgfscope}%
\pgfsys@transformshift{2.312134in}{0.943734in}%
\pgfsys@useobject{currentmarker}{}%
\end{pgfscope}%
\begin{pgfscope}%
\pgfsys@transformshift{2.312134in}{0.943734in}%
\pgfsys@useobject{currentmarker}{}%
\end{pgfscope}%
\begin{pgfscope}%
\pgfsys@transformshift{2.312134in}{0.943734in}%
\pgfsys@useobject{currentmarker}{}%
\end{pgfscope}%
\begin{pgfscope}%
\pgfsys@transformshift{2.312134in}{0.943734in}%
\pgfsys@useobject{currentmarker}{}%
\end{pgfscope}%
\begin{pgfscope}%
\pgfsys@transformshift{2.312134in}{0.943734in}%
\pgfsys@useobject{currentmarker}{}%
\end{pgfscope}%
\begin{pgfscope}%
\pgfsys@transformshift{2.312134in}{0.943734in}%
\pgfsys@useobject{currentmarker}{}%
\end{pgfscope}%
\begin{pgfscope}%
\pgfsys@transformshift{2.312134in}{0.943734in}%
\pgfsys@useobject{currentmarker}{}%
\end{pgfscope}%
\begin{pgfscope}%
\pgfsys@transformshift{2.312134in}{0.943734in}%
\pgfsys@useobject{currentmarker}{}%
\end{pgfscope}%
\begin{pgfscope}%
\pgfsys@transformshift{2.312134in}{0.943734in}%
\pgfsys@useobject{currentmarker}{}%
\end{pgfscope}%
\begin{pgfscope}%
\pgfsys@transformshift{2.312134in}{0.943734in}%
\pgfsys@useobject{currentmarker}{}%
\end{pgfscope}%
\begin{pgfscope}%
\pgfsys@transformshift{2.312134in}{0.943734in}%
\pgfsys@useobject{currentmarker}{}%
\end{pgfscope}%
\begin{pgfscope}%
\pgfsys@transformshift{2.312134in}{0.943734in}%
\pgfsys@useobject{currentmarker}{}%
\end{pgfscope}%
\begin{pgfscope}%
\pgfsys@transformshift{2.312134in}{0.943734in}%
\pgfsys@useobject{currentmarker}{}%
\end{pgfscope}%
\begin{pgfscope}%
\pgfsys@transformshift{2.312134in}{0.943734in}%
\pgfsys@useobject{currentmarker}{}%
\end{pgfscope}%
\begin{pgfscope}%
\pgfsys@transformshift{2.312134in}{0.943734in}%
\pgfsys@useobject{currentmarker}{}%
\end{pgfscope}%
\begin{pgfscope}%
\pgfsys@transformshift{2.312134in}{0.943734in}%
\pgfsys@useobject{currentmarker}{}%
\end{pgfscope}%
\begin{pgfscope}%
\pgfsys@transformshift{2.312134in}{0.943734in}%
\pgfsys@useobject{currentmarker}{}%
\end{pgfscope}%
\begin{pgfscope}%
\pgfsys@transformshift{2.312134in}{0.943734in}%
\pgfsys@useobject{currentmarker}{}%
\end{pgfscope}%
\begin{pgfscope}%
\pgfsys@transformshift{2.312134in}{0.943734in}%
\pgfsys@useobject{currentmarker}{}%
\end{pgfscope}%
\begin{pgfscope}%
\pgfsys@transformshift{2.312134in}{0.943734in}%
\pgfsys@useobject{currentmarker}{}%
\end{pgfscope}%
\begin{pgfscope}%
\pgfsys@transformshift{2.312134in}{0.943734in}%
\pgfsys@useobject{currentmarker}{}%
\end{pgfscope}%
\begin{pgfscope}%
\pgfsys@transformshift{2.312134in}{0.943734in}%
\pgfsys@useobject{currentmarker}{}%
\end{pgfscope}%
\begin{pgfscope}%
\pgfsys@transformshift{2.312134in}{0.943734in}%
\pgfsys@useobject{currentmarker}{}%
\end{pgfscope}%
\begin{pgfscope}%
\pgfsys@transformshift{2.312134in}{0.943734in}%
\pgfsys@useobject{currentmarker}{}%
\end{pgfscope}%
\begin{pgfscope}%
\pgfsys@transformshift{2.312134in}{0.943734in}%
\pgfsys@useobject{currentmarker}{}%
\end{pgfscope}%
\begin{pgfscope}%
\pgfsys@transformshift{2.312134in}{0.943734in}%
\pgfsys@useobject{currentmarker}{}%
\end{pgfscope}%
\begin{pgfscope}%
\pgfsys@transformshift{2.312134in}{0.943734in}%
\pgfsys@useobject{currentmarker}{}%
\end{pgfscope}%
\begin{pgfscope}%
\pgfsys@transformshift{2.312134in}{0.943734in}%
\pgfsys@useobject{currentmarker}{}%
\end{pgfscope}%
\begin{pgfscope}%
\pgfsys@transformshift{2.312134in}{0.943734in}%
\pgfsys@useobject{currentmarker}{}%
\end{pgfscope}%
\begin{pgfscope}%
\pgfsys@transformshift{2.312134in}{0.943734in}%
\pgfsys@useobject{currentmarker}{}%
\end{pgfscope}%
\begin{pgfscope}%
\pgfsys@transformshift{2.312134in}{0.943734in}%
\pgfsys@useobject{currentmarker}{}%
\end{pgfscope}%
\begin{pgfscope}%
\pgfsys@transformshift{2.312134in}{0.943734in}%
\pgfsys@useobject{currentmarker}{}%
\end{pgfscope}%
\begin{pgfscope}%
\pgfsys@transformshift{2.312134in}{0.943734in}%
\pgfsys@useobject{currentmarker}{}%
\end{pgfscope}%
\begin{pgfscope}%
\pgfsys@transformshift{2.312134in}{0.943734in}%
\pgfsys@useobject{currentmarker}{}%
\end{pgfscope}%
\begin{pgfscope}%
\pgfsys@transformshift{2.312134in}{0.943734in}%
\pgfsys@useobject{currentmarker}{}%
\end{pgfscope}%
\begin{pgfscope}%
\pgfsys@transformshift{2.312134in}{0.943734in}%
\pgfsys@useobject{currentmarker}{}%
\end{pgfscope}%
\begin{pgfscope}%
\pgfsys@transformshift{2.312134in}{0.943734in}%
\pgfsys@useobject{currentmarker}{}%
\end{pgfscope}%
\begin{pgfscope}%
\pgfsys@transformshift{2.312134in}{0.943734in}%
\pgfsys@useobject{currentmarker}{}%
\end{pgfscope}%
\begin{pgfscope}%
\pgfsys@transformshift{2.312134in}{0.943734in}%
\pgfsys@useobject{currentmarker}{}%
\end{pgfscope}%
\begin{pgfscope}%
\pgfsys@transformshift{2.312134in}{0.943734in}%
\pgfsys@useobject{currentmarker}{}%
\end{pgfscope}%
\begin{pgfscope}%
\pgfsys@transformshift{2.312134in}{0.943734in}%
\pgfsys@useobject{currentmarker}{}%
\end{pgfscope}%
\begin{pgfscope}%
\pgfsys@transformshift{2.312134in}{0.943734in}%
\pgfsys@useobject{currentmarker}{}%
\end{pgfscope}%
\begin{pgfscope}%
\pgfsys@transformshift{2.312134in}{0.943734in}%
\pgfsys@useobject{currentmarker}{}%
\end{pgfscope}%
\begin{pgfscope}%
\pgfsys@transformshift{2.312134in}{0.943734in}%
\pgfsys@useobject{currentmarker}{}%
\end{pgfscope}%
\begin{pgfscope}%
\pgfsys@transformshift{2.312134in}{0.943734in}%
\pgfsys@useobject{currentmarker}{}%
\end{pgfscope}%
\begin{pgfscope}%
\pgfsys@transformshift{2.312134in}{0.943734in}%
\pgfsys@useobject{currentmarker}{}%
\end{pgfscope}%
\begin{pgfscope}%
\pgfsys@transformshift{2.312134in}{0.943734in}%
\pgfsys@useobject{currentmarker}{}%
\end{pgfscope}%
\begin{pgfscope}%
\pgfsys@transformshift{2.312134in}{0.943734in}%
\pgfsys@useobject{currentmarker}{}%
\end{pgfscope}%
\begin{pgfscope}%
\pgfsys@transformshift{2.312134in}{0.943734in}%
\pgfsys@useobject{currentmarker}{}%
\end{pgfscope}%
\begin{pgfscope}%
\pgfsys@transformshift{2.312134in}{0.943734in}%
\pgfsys@useobject{currentmarker}{}%
\end{pgfscope}%
\begin{pgfscope}%
\pgfsys@transformshift{2.312134in}{0.943734in}%
\pgfsys@useobject{currentmarker}{}%
\end{pgfscope}%
\begin{pgfscope}%
\pgfsys@transformshift{2.312134in}{0.943734in}%
\pgfsys@useobject{currentmarker}{}%
\end{pgfscope}%
\begin{pgfscope}%
\pgfsys@transformshift{2.312134in}{0.943734in}%
\pgfsys@useobject{currentmarker}{}%
\end{pgfscope}%
\begin{pgfscope}%
\pgfsys@transformshift{2.312134in}{0.943734in}%
\pgfsys@useobject{currentmarker}{}%
\end{pgfscope}%
\begin{pgfscope}%
\pgfsys@transformshift{2.312134in}{0.943734in}%
\pgfsys@useobject{currentmarker}{}%
\end{pgfscope}%
\begin{pgfscope}%
\pgfsys@transformshift{2.312134in}{0.943734in}%
\pgfsys@useobject{currentmarker}{}%
\end{pgfscope}%
\begin{pgfscope}%
\pgfsys@transformshift{2.312134in}{0.943734in}%
\pgfsys@useobject{currentmarker}{}%
\end{pgfscope}%
\begin{pgfscope}%
\pgfsys@transformshift{2.312134in}{0.943734in}%
\pgfsys@useobject{currentmarker}{}%
\end{pgfscope}%
\begin{pgfscope}%
\pgfsys@transformshift{2.312134in}{0.943734in}%
\pgfsys@useobject{currentmarker}{}%
\end{pgfscope}%
\begin{pgfscope}%
\pgfsys@transformshift{2.312134in}{0.943734in}%
\pgfsys@useobject{currentmarker}{}%
\end{pgfscope}%
\begin{pgfscope}%
\pgfsys@transformshift{2.312134in}{0.943734in}%
\pgfsys@useobject{currentmarker}{}%
\end{pgfscope}%
\begin{pgfscope}%
\pgfsys@transformshift{2.312134in}{0.943734in}%
\pgfsys@useobject{currentmarker}{}%
\end{pgfscope}%
\begin{pgfscope}%
\pgfsys@transformshift{2.312134in}{0.943734in}%
\pgfsys@useobject{currentmarker}{}%
\end{pgfscope}%
\begin{pgfscope}%
\pgfsys@transformshift{2.312134in}{0.943734in}%
\pgfsys@useobject{currentmarker}{}%
\end{pgfscope}%
\begin{pgfscope}%
\pgfsys@transformshift{2.312134in}{0.943734in}%
\pgfsys@useobject{currentmarker}{}%
\end{pgfscope}%
\begin{pgfscope}%
\pgfsys@transformshift{2.312134in}{0.943734in}%
\pgfsys@useobject{currentmarker}{}%
\end{pgfscope}%
\begin{pgfscope}%
\pgfsys@transformshift{2.312134in}{0.943734in}%
\pgfsys@useobject{currentmarker}{}%
\end{pgfscope}%
\begin{pgfscope}%
\pgfsys@transformshift{2.312134in}{0.943734in}%
\pgfsys@useobject{currentmarker}{}%
\end{pgfscope}%
\begin{pgfscope}%
\pgfsys@transformshift{2.312134in}{0.943734in}%
\pgfsys@useobject{currentmarker}{}%
\end{pgfscope}%
\begin{pgfscope}%
\pgfsys@transformshift{2.312134in}{0.943734in}%
\pgfsys@useobject{currentmarker}{}%
\end{pgfscope}%
\begin{pgfscope}%
\pgfsys@transformshift{2.312134in}{0.943734in}%
\pgfsys@useobject{currentmarker}{}%
\end{pgfscope}%
\begin{pgfscope}%
\pgfsys@transformshift{2.312134in}{0.943734in}%
\pgfsys@useobject{currentmarker}{}%
\end{pgfscope}%
\begin{pgfscope}%
\pgfsys@transformshift{2.312134in}{0.943734in}%
\pgfsys@useobject{currentmarker}{}%
\end{pgfscope}%
\begin{pgfscope}%
\pgfsys@transformshift{2.312134in}{0.943734in}%
\pgfsys@useobject{currentmarker}{}%
\end{pgfscope}%
\begin{pgfscope}%
\pgfsys@transformshift{2.312134in}{0.943734in}%
\pgfsys@useobject{currentmarker}{}%
\end{pgfscope}%
\begin{pgfscope}%
\pgfsys@transformshift{2.312134in}{0.943734in}%
\pgfsys@useobject{currentmarker}{}%
\end{pgfscope}%
\begin{pgfscope}%
\pgfsys@transformshift{2.312134in}{0.943734in}%
\pgfsys@useobject{currentmarker}{}%
\end{pgfscope}%
\begin{pgfscope}%
\pgfsys@transformshift{2.312134in}{0.943734in}%
\pgfsys@useobject{currentmarker}{}%
\end{pgfscope}%
\begin{pgfscope}%
\pgfsys@transformshift{2.312134in}{0.943734in}%
\pgfsys@useobject{currentmarker}{}%
\end{pgfscope}%
\begin{pgfscope}%
\pgfsys@transformshift{2.312134in}{0.943734in}%
\pgfsys@useobject{currentmarker}{}%
\end{pgfscope}%
\begin{pgfscope}%
\pgfsys@transformshift{2.312134in}{0.943734in}%
\pgfsys@useobject{currentmarker}{}%
\end{pgfscope}%
\begin{pgfscope}%
\pgfsys@transformshift{2.312134in}{0.943734in}%
\pgfsys@useobject{currentmarker}{}%
\end{pgfscope}%
\begin{pgfscope}%
\pgfsys@transformshift{2.312134in}{0.943734in}%
\pgfsys@useobject{currentmarker}{}%
\end{pgfscope}%
\begin{pgfscope}%
\pgfsys@transformshift{2.312134in}{0.943734in}%
\pgfsys@useobject{currentmarker}{}%
\end{pgfscope}%
\begin{pgfscope}%
\pgfsys@transformshift{2.312134in}{0.943734in}%
\pgfsys@useobject{currentmarker}{}%
\end{pgfscope}%
\begin{pgfscope}%
\pgfsys@transformshift{2.312134in}{0.943734in}%
\pgfsys@useobject{currentmarker}{}%
\end{pgfscope}%
\begin{pgfscope}%
\pgfsys@transformshift{2.312134in}{0.943734in}%
\pgfsys@useobject{currentmarker}{}%
\end{pgfscope}%
\begin{pgfscope}%
\pgfsys@transformshift{2.312134in}{0.943734in}%
\pgfsys@useobject{currentmarker}{}%
\end{pgfscope}%
\begin{pgfscope}%
\pgfsys@transformshift{2.312134in}{0.943734in}%
\pgfsys@useobject{currentmarker}{}%
\end{pgfscope}%
\begin{pgfscope}%
\pgfsys@transformshift{2.312134in}{0.943734in}%
\pgfsys@useobject{currentmarker}{}%
\end{pgfscope}%
\begin{pgfscope}%
\pgfsys@transformshift{2.312134in}{0.943734in}%
\pgfsys@useobject{currentmarker}{}%
\end{pgfscope}%
\begin{pgfscope}%
\pgfsys@transformshift{2.312134in}{0.943734in}%
\pgfsys@useobject{currentmarker}{}%
\end{pgfscope}%
\begin{pgfscope}%
\pgfsys@transformshift{2.312134in}{0.943734in}%
\pgfsys@useobject{currentmarker}{}%
\end{pgfscope}%
\begin{pgfscope}%
\pgfsys@transformshift{2.312134in}{0.943734in}%
\pgfsys@useobject{currentmarker}{}%
\end{pgfscope}%
\begin{pgfscope}%
\pgfsys@transformshift{2.312134in}{0.943734in}%
\pgfsys@useobject{currentmarker}{}%
\end{pgfscope}%
\begin{pgfscope}%
\pgfsys@transformshift{2.312134in}{0.943734in}%
\pgfsys@useobject{currentmarker}{}%
\end{pgfscope}%
\begin{pgfscope}%
\pgfsys@transformshift{2.312134in}{0.943734in}%
\pgfsys@useobject{currentmarker}{}%
\end{pgfscope}%
\begin{pgfscope}%
\pgfsys@transformshift{2.312134in}{0.943734in}%
\pgfsys@useobject{currentmarker}{}%
\end{pgfscope}%
\begin{pgfscope}%
\pgfsys@transformshift{2.312134in}{0.943734in}%
\pgfsys@useobject{currentmarker}{}%
\end{pgfscope}%
\begin{pgfscope}%
\pgfsys@transformshift{2.312134in}{0.943734in}%
\pgfsys@useobject{currentmarker}{}%
\end{pgfscope}%
\begin{pgfscope}%
\pgfsys@transformshift{2.312134in}{0.943734in}%
\pgfsys@useobject{currentmarker}{}%
\end{pgfscope}%
\begin{pgfscope}%
\pgfsys@transformshift{2.312134in}{0.943734in}%
\pgfsys@useobject{currentmarker}{}%
\end{pgfscope}%
\begin{pgfscope}%
\pgfsys@transformshift{2.312134in}{0.943734in}%
\pgfsys@useobject{currentmarker}{}%
\end{pgfscope}%
\begin{pgfscope}%
\pgfsys@transformshift{2.312134in}{0.943734in}%
\pgfsys@useobject{currentmarker}{}%
\end{pgfscope}%
\begin{pgfscope}%
\pgfsys@transformshift{2.312134in}{0.943734in}%
\pgfsys@useobject{currentmarker}{}%
\end{pgfscope}%
\begin{pgfscope}%
\pgfsys@transformshift{2.312134in}{0.943734in}%
\pgfsys@useobject{currentmarker}{}%
\end{pgfscope}%
\begin{pgfscope}%
\pgfsys@transformshift{2.312134in}{0.943734in}%
\pgfsys@useobject{currentmarker}{}%
\end{pgfscope}%
\begin{pgfscope}%
\pgfsys@transformshift{2.312134in}{0.943734in}%
\pgfsys@useobject{currentmarker}{}%
\end{pgfscope}%
\begin{pgfscope}%
\pgfsys@transformshift{2.312134in}{0.943734in}%
\pgfsys@useobject{currentmarker}{}%
\end{pgfscope}%
\begin{pgfscope}%
\pgfsys@transformshift{2.312134in}{0.943734in}%
\pgfsys@useobject{currentmarker}{}%
\end{pgfscope}%
\begin{pgfscope}%
\pgfsys@transformshift{2.312134in}{0.943734in}%
\pgfsys@useobject{currentmarker}{}%
\end{pgfscope}%
\begin{pgfscope}%
\pgfsys@transformshift{2.312134in}{0.943734in}%
\pgfsys@useobject{currentmarker}{}%
\end{pgfscope}%
\begin{pgfscope}%
\pgfsys@transformshift{2.312134in}{0.943734in}%
\pgfsys@useobject{currentmarker}{}%
\end{pgfscope}%
\begin{pgfscope}%
\pgfsys@transformshift{2.312134in}{0.943734in}%
\pgfsys@useobject{currentmarker}{}%
\end{pgfscope}%
\begin{pgfscope}%
\pgfsys@transformshift{2.312134in}{0.943734in}%
\pgfsys@useobject{currentmarker}{}%
\end{pgfscope}%
\begin{pgfscope}%
\pgfsys@transformshift{2.312134in}{0.943734in}%
\pgfsys@useobject{currentmarker}{}%
\end{pgfscope}%
\begin{pgfscope}%
\pgfsys@transformshift{2.312134in}{0.943734in}%
\pgfsys@useobject{currentmarker}{}%
\end{pgfscope}%
\begin{pgfscope}%
\pgfsys@transformshift{2.312134in}{0.943734in}%
\pgfsys@useobject{currentmarker}{}%
\end{pgfscope}%
\begin{pgfscope}%
\pgfsys@transformshift{2.312134in}{0.943734in}%
\pgfsys@useobject{currentmarker}{}%
\end{pgfscope}%
\begin{pgfscope}%
\pgfsys@transformshift{2.312134in}{0.943734in}%
\pgfsys@useobject{currentmarker}{}%
\end{pgfscope}%
\begin{pgfscope}%
\pgfsys@transformshift{2.312134in}{0.943734in}%
\pgfsys@useobject{currentmarker}{}%
\end{pgfscope}%
\begin{pgfscope}%
\pgfsys@transformshift{2.312134in}{0.943734in}%
\pgfsys@useobject{currentmarker}{}%
\end{pgfscope}%
\begin{pgfscope}%
\pgfsys@transformshift{2.312134in}{0.943734in}%
\pgfsys@useobject{currentmarker}{}%
\end{pgfscope}%
\begin{pgfscope}%
\pgfsys@transformshift{2.312134in}{0.943734in}%
\pgfsys@useobject{currentmarker}{}%
\end{pgfscope}%
\begin{pgfscope}%
\pgfsys@transformshift{2.312134in}{0.943734in}%
\pgfsys@useobject{currentmarker}{}%
\end{pgfscope}%
\begin{pgfscope}%
\pgfsys@transformshift{2.312134in}{0.943734in}%
\pgfsys@useobject{currentmarker}{}%
\end{pgfscope}%
\begin{pgfscope}%
\pgfsys@transformshift{2.312134in}{0.943734in}%
\pgfsys@useobject{currentmarker}{}%
\end{pgfscope}%
\begin{pgfscope}%
\pgfsys@transformshift{2.312134in}{0.943734in}%
\pgfsys@useobject{currentmarker}{}%
\end{pgfscope}%
\begin{pgfscope}%
\pgfsys@transformshift{2.312134in}{0.943734in}%
\pgfsys@useobject{currentmarker}{}%
\end{pgfscope}%
\begin{pgfscope}%
\pgfsys@transformshift{2.312134in}{0.943734in}%
\pgfsys@useobject{currentmarker}{}%
\end{pgfscope}%
\begin{pgfscope}%
\pgfsys@transformshift{2.312134in}{0.943734in}%
\pgfsys@useobject{currentmarker}{}%
\end{pgfscope}%
\begin{pgfscope}%
\pgfsys@transformshift{2.312134in}{0.943734in}%
\pgfsys@useobject{currentmarker}{}%
\end{pgfscope}%
\begin{pgfscope}%
\pgfsys@transformshift{2.312134in}{0.943734in}%
\pgfsys@useobject{currentmarker}{}%
\end{pgfscope}%
\begin{pgfscope}%
\pgfsys@transformshift{2.312134in}{0.943734in}%
\pgfsys@useobject{currentmarker}{}%
\end{pgfscope}%
\begin{pgfscope}%
\pgfsys@transformshift{2.312134in}{0.943734in}%
\pgfsys@useobject{currentmarker}{}%
\end{pgfscope}%
\begin{pgfscope}%
\pgfsys@transformshift{2.312134in}{0.943734in}%
\pgfsys@useobject{currentmarker}{}%
\end{pgfscope}%
\begin{pgfscope}%
\pgfsys@transformshift{2.312134in}{0.943734in}%
\pgfsys@useobject{currentmarker}{}%
\end{pgfscope}%
\begin{pgfscope}%
\pgfsys@transformshift{2.312134in}{0.943734in}%
\pgfsys@useobject{currentmarker}{}%
\end{pgfscope}%
\begin{pgfscope}%
\pgfsys@transformshift{2.312134in}{0.943734in}%
\pgfsys@useobject{currentmarker}{}%
\end{pgfscope}%
\begin{pgfscope}%
\pgfsys@transformshift{2.312134in}{0.943734in}%
\pgfsys@useobject{currentmarker}{}%
\end{pgfscope}%
\begin{pgfscope}%
\pgfsys@transformshift{2.312134in}{0.943734in}%
\pgfsys@useobject{currentmarker}{}%
\end{pgfscope}%
\begin{pgfscope}%
\pgfsys@transformshift{2.312134in}{0.943734in}%
\pgfsys@useobject{currentmarker}{}%
\end{pgfscope}%
\begin{pgfscope}%
\pgfsys@transformshift{2.312134in}{0.943734in}%
\pgfsys@useobject{currentmarker}{}%
\end{pgfscope}%
\begin{pgfscope}%
\pgfsys@transformshift{2.312134in}{0.943734in}%
\pgfsys@useobject{currentmarker}{}%
\end{pgfscope}%
\begin{pgfscope}%
\pgfsys@transformshift{2.312134in}{0.943734in}%
\pgfsys@useobject{currentmarker}{}%
\end{pgfscope}%
\begin{pgfscope}%
\pgfsys@transformshift{2.312134in}{0.943734in}%
\pgfsys@useobject{currentmarker}{}%
\end{pgfscope}%
\begin{pgfscope}%
\pgfsys@transformshift{2.312134in}{0.943734in}%
\pgfsys@useobject{currentmarker}{}%
\end{pgfscope}%
\begin{pgfscope}%
\pgfsys@transformshift{2.312134in}{0.943734in}%
\pgfsys@useobject{currentmarker}{}%
\end{pgfscope}%
\begin{pgfscope}%
\pgfsys@transformshift{2.312134in}{0.943734in}%
\pgfsys@useobject{currentmarker}{}%
\end{pgfscope}%
\begin{pgfscope}%
\pgfsys@transformshift{2.312134in}{0.943734in}%
\pgfsys@useobject{currentmarker}{}%
\end{pgfscope}%
\begin{pgfscope}%
\pgfsys@transformshift{2.312134in}{0.943734in}%
\pgfsys@useobject{currentmarker}{}%
\end{pgfscope}%
\begin{pgfscope}%
\pgfsys@transformshift{2.312134in}{0.943734in}%
\pgfsys@useobject{currentmarker}{}%
\end{pgfscope}%
\begin{pgfscope}%
\pgfsys@transformshift{2.312134in}{0.943734in}%
\pgfsys@useobject{currentmarker}{}%
\end{pgfscope}%
\begin{pgfscope}%
\pgfsys@transformshift{2.312134in}{0.943734in}%
\pgfsys@useobject{currentmarker}{}%
\end{pgfscope}%
\begin{pgfscope}%
\pgfsys@transformshift{2.312134in}{0.943734in}%
\pgfsys@useobject{currentmarker}{}%
\end{pgfscope}%
\begin{pgfscope}%
\pgfsys@transformshift{2.312134in}{0.943734in}%
\pgfsys@useobject{currentmarker}{}%
\end{pgfscope}%
\begin{pgfscope}%
\pgfsys@transformshift{2.312134in}{0.943734in}%
\pgfsys@useobject{currentmarker}{}%
\end{pgfscope}%
\begin{pgfscope}%
\pgfsys@transformshift{2.312134in}{0.943734in}%
\pgfsys@useobject{currentmarker}{}%
\end{pgfscope}%
\begin{pgfscope}%
\pgfsys@transformshift{2.312134in}{0.943734in}%
\pgfsys@useobject{currentmarker}{}%
\end{pgfscope}%
\begin{pgfscope}%
\pgfsys@transformshift{2.312134in}{0.943734in}%
\pgfsys@useobject{currentmarker}{}%
\end{pgfscope}%
\begin{pgfscope}%
\pgfsys@transformshift{2.312134in}{0.943734in}%
\pgfsys@useobject{currentmarker}{}%
\end{pgfscope}%
\begin{pgfscope}%
\pgfsys@transformshift{2.312134in}{0.943734in}%
\pgfsys@useobject{currentmarker}{}%
\end{pgfscope}%
\begin{pgfscope}%
\pgfsys@transformshift{2.312134in}{0.943734in}%
\pgfsys@useobject{currentmarker}{}%
\end{pgfscope}%
\begin{pgfscope}%
\pgfsys@transformshift{2.312134in}{0.943734in}%
\pgfsys@useobject{currentmarker}{}%
\end{pgfscope}%
\begin{pgfscope}%
\pgfsys@transformshift{2.312134in}{0.943734in}%
\pgfsys@useobject{currentmarker}{}%
\end{pgfscope}%
\begin{pgfscope}%
\pgfsys@transformshift{2.312134in}{0.943734in}%
\pgfsys@useobject{currentmarker}{}%
\end{pgfscope}%
\begin{pgfscope}%
\pgfsys@transformshift{2.312134in}{0.943734in}%
\pgfsys@useobject{currentmarker}{}%
\end{pgfscope}%
\begin{pgfscope}%
\pgfsys@transformshift{2.312134in}{0.943734in}%
\pgfsys@useobject{currentmarker}{}%
\end{pgfscope}%
\begin{pgfscope}%
\pgfsys@transformshift{2.312134in}{0.943734in}%
\pgfsys@useobject{currentmarker}{}%
\end{pgfscope}%
\begin{pgfscope}%
\pgfsys@transformshift{2.312134in}{0.943734in}%
\pgfsys@useobject{currentmarker}{}%
\end{pgfscope}%
\begin{pgfscope}%
\pgfsys@transformshift{2.312134in}{0.943734in}%
\pgfsys@useobject{currentmarker}{}%
\end{pgfscope}%
\begin{pgfscope}%
\pgfsys@transformshift{2.312134in}{0.943734in}%
\pgfsys@useobject{currentmarker}{}%
\end{pgfscope}%
\begin{pgfscope}%
\pgfsys@transformshift{2.312134in}{0.943734in}%
\pgfsys@useobject{currentmarker}{}%
\end{pgfscope}%
\begin{pgfscope}%
\pgfsys@transformshift{2.312134in}{0.943734in}%
\pgfsys@useobject{currentmarker}{}%
\end{pgfscope}%
\begin{pgfscope}%
\pgfsys@transformshift{2.312134in}{0.943734in}%
\pgfsys@useobject{currentmarker}{}%
\end{pgfscope}%
\begin{pgfscope}%
\pgfsys@transformshift{2.312134in}{0.943734in}%
\pgfsys@useobject{currentmarker}{}%
\end{pgfscope}%
\begin{pgfscope}%
\pgfsys@transformshift{2.312134in}{0.943734in}%
\pgfsys@useobject{currentmarker}{}%
\end{pgfscope}%
\begin{pgfscope}%
\pgfsys@transformshift{2.312134in}{0.943734in}%
\pgfsys@useobject{currentmarker}{}%
\end{pgfscope}%
\begin{pgfscope}%
\pgfsys@transformshift{2.312134in}{0.943734in}%
\pgfsys@useobject{currentmarker}{}%
\end{pgfscope}%
\begin{pgfscope}%
\pgfsys@transformshift{2.312134in}{0.943734in}%
\pgfsys@useobject{currentmarker}{}%
\end{pgfscope}%
\begin{pgfscope}%
\pgfsys@transformshift{2.312134in}{0.943734in}%
\pgfsys@useobject{currentmarker}{}%
\end{pgfscope}%
\begin{pgfscope}%
\pgfsys@transformshift{2.312134in}{0.943734in}%
\pgfsys@useobject{currentmarker}{}%
\end{pgfscope}%
\begin{pgfscope}%
\pgfsys@transformshift{2.312134in}{0.943734in}%
\pgfsys@useobject{currentmarker}{}%
\end{pgfscope}%
\begin{pgfscope}%
\pgfsys@transformshift{2.312134in}{0.943734in}%
\pgfsys@useobject{currentmarker}{}%
\end{pgfscope}%
\begin{pgfscope}%
\pgfsys@transformshift{2.312134in}{0.943734in}%
\pgfsys@useobject{currentmarker}{}%
\end{pgfscope}%
\begin{pgfscope}%
\pgfsys@transformshift{2.312134in}{0.943734in}%
\pgfsys@useobject{currentmarker}{}%
\end{pgfscope}%
\begin{pgfscope}%
\pgfsys@transformshift{2.312134in}{0.943734in}%
\pgfsys@useobject{currentmarker}{}%
\end{pgfscope}%
\begin{pgfscope}%
\pgfsys@transformshift{2.312134in}{0.943734in}%
\pgfsys@useobject{currentmarker}{}%
\end{pgfscope}%
\begin{pgfscope}%
\pgfsys@transformshift{2.312134in}{0.943734in}%
\pgfsys@useobject{currentmarker}{}%
\end{pgfscope}%
\begin{pgfscope}%
\pgfsys@transformshift{2.312134in}{0.943734in}%
\pgfsys@useobject{currentmarker}{}%
\end{pgfscope}%
\begin{pgfscope}%
\pgfsys@transformshift{2.312134in}{0.943734in}%
\pgfsys@useobject{currentmarker}{}%
\end{pgfscope}%
\begin{pgfscope}%
\pgfsys@transformshift{2.312134in}{0.943734in}%
\pgfsys@useobject{currentmarker}{}%
\end{pgfscope}%
\begin{pgfscope}%
\pgfsys@transformshift{2.312134in}{0.943734in}%
\pgfsys@useobject{currentmarker}{}%
\end{pgfscope}%
\begin{pgfscope}%
\pgfsys@transformshift{2.312134in}{0.943734in}%
\pgfsys@useobject{currentmarker}{}%
\end{pgfscope}%
\begin{pgfscope}%
\pgfsys@transformshift{2.312134in}{0.943734in}%
\pgfsys@useobject{currentmarker}{}%
\end{pgfscope}%
\begin{pgfscope}%
\pgfsys@transformshift{2.312134in}{0.943734in}%
\pgfsys@useobject{currentmarker}{}%
\end{pgfscope}%
\begin{pgfscope}%
\pgfsys@transformshift{2.312134in}{0.943734in}%
\pgfsys@useobject{currentmarker}{}%
\end{pgfscope}%
\begin{pgfscope}%
\pgfsys@transformshift{2.312134in}{0.943734in}%
\pgfsys@useobject{currentmarker}{}%
\end{pgfscope}%
\begin{pgfscope}%
\pgfsys@transformshift{2.312134in}{0.943734in}%
\pgfsys@useobject{currentmarker}{}%
\end{pgfscope}%
\begin{pgfscope}%
\pgfsys@transformshift{2.312134in}{0.943734in}%
\pgfsys@useobject{currentmarker}{}%
\end{pgfscope}%
\begin{pgfscope}%
\pgfsys@transformshift{2.312134in}{0.943734in}%
\pgfsys@useobject{currentmarker}{}%
\end{pgfscope}%
\begin{pgfscope}%
\pgfsys@transformshift{2.312134in}{0.943734in}%
\pgfsys@useobject{currentmarker}{}%
\end{pgfscope}%
\begin{pgfscope}%
\pgfsys@transformshift{2.312134in}{0.943734in}%
\pgfsys@useobject{currentmarker}{}%
\end{pgfscope}%
\begin{pgfscope}%
\pgfsys@transformshift{2.312134in}{0.943734in}%
\pgfsys@useobject{currentmarker}{}%
\end{pgfscope}%
\begin{pgfscope}%
\pgfsys@transformshift{2.312134in}{0.943734in}%
\pgfsys@useobject{currentmarker}{}%
\end{pgfscope}%
\begin{pgfscope}%
\pgfsys@transformshift{2.312134in}{0.943734in}%
\pgfsys@useobject{currentmarker}{}%
\end{pgfscope}%
\begin{pgfscope}%
\pgfsys@transformshift{2.312134in}{0.943734in}%
\pgfsys@useobject{currentmarker}{}%
\end{pgfscope}%
\begin{pgfscope}%
\pgfsys@transformshift{2.312134in}{0.943734in}%
\pgfsys@useobject{currentmarker}{}%
\end{pgfscope}%
\begin{pgfscope}%
\pgfsys@transformshift{2.312134in}{0.943734in}%
\pgfsys@useobject{currentmarker}{}%
\end{pgfscope}%
\begin{pgfscope}%
\pgfsys@transformshift{2.312134in}{0.943734in}%
\pgfsys@useobject{currentmarker}{}%
\end{pgfscope}%
\begin{pgfscope}%
\pgfsys@transformshift{2.312134in}{0.943734in}%
\pgfsys@useobject{currentmarker}{}%
\end{pgfscope}%
\begin{pgfscope}%
\pgfsys@transformshift{2.312134in}{0.943734in}%
\pgfsys@useobject{currentmarker}{}%
\end{pgfscope}%
\begin{pgfscope}%
\pgfsys@transformshift{2.312134in}{0.943734in}%
\pgfsys@useobject{currentmarker}{}%
\end{pgfscope}%
\begin{pgfscope}%
\pgfsys@transformshift{2.312134in}{0.943734in}%
\pgfsys@useobject{currentmarker}{}%
\end{pgfscope}%
\begin{pgfscope}%
\pgfsys@transformshift{2.312134in}{0.943734in}%
\pgfsys@useobject{currentmarker}{}%
\end{pgfscope}%
\begin{pgfscope}%
\pgfsys@transformshift{2.312134in}{0.943734in}%
\pgfsys@useobject{currentmarker}{}%
\end{pgfscope}%
\begin{pgfscope}%
\pgfsys@transformshift{2.312134in}{0.943734in}%
\pgfsys@useobject{currentmarker}{}%
\end{pgfscope}%
\begin{pgfscope}%
\pgfsys@transformshift{2.312134in}{0.943734in}%
\pgfsys@useobject{currentmarker}{}%
\end{pgfscope}%
\begin{pgfscope}%
\pgfsys@transformshift{2.312134in}{0.943734in}%
\pgfsys@useobject{currentmarker}{}%
\end{pgfscope}%
\begin{pgfscope}%
\pgfsys@transformshift{2.312134in}{0.943734in}%
\pgfsys@useobject{currentmarker}{}%
\end{pgfscope}%
\begin{pgfscope}%
\pgfsys@transformshift{2.312134in}{0.943734in}%
\pgfsys@useobject{currentmarker}{}%
\end{pgfscope}%
\begin{pgfscope}%
\pgfsys@transformshift{2.312134in}{0.943734in}%
\pgfsys@useobject{currentmarker}{}%
\end{pgfscope}%
\begin{pgfscope}%
\pgfsys@transformshift{2.312134in}{0.943734in}%
\pgfsys@useobject{currentmarker}{}%
\end{pgfscope}%
\begin{pgfscope}%
\pgfsys@transformshift{2.312134in}{0.943734in}%
\pgfsys@useobject{currentmarker}{}%
\end{pgfscope}%
\begin{pgfscope}%
\pgfsys@transformshift{2.312134in}{0.943734in}%
\pgfsys@useobject{currentmarker}{}%
\end{pgfscope}%
\begin{pgfscope}%
\pgfsys@transformshift{2.312134in}{0.943734in}%
\pgfsys@useobject{currentmarker}{}%
\end{pgfscope}%
\begin{pgfscope}%
\pgfsys@transformshift{2.312134in}{0.943734in}%
\pgfsys@useobject{currentmarker}{}%
\end{pgfscope}%
\begin{pgfscope}%
\pgfsys@transformshift{2.312134in}{0.943734in}%
\pgfsys@useobject{currentmarker}{}%
\end{pgfscope}%
\begin{pgfscope}%
\pgfsys@transformshift{2.312134in}{0.943734in}%
\pgfsys@useobject{currentmarker}{}%
\end{pgfscope}%
\begin{pgfscope}%
\pgfsys@transformshift{2.312134in}{0.943734in}%
\pgfsys@useobject{currentmarker}{}%
\end{pgfscope}%
\begin{pgfscope}%
\pgfsys@transformshift{2.312134in}{0.943734in}%
\pgfsys@useobject{currentmarker}{}%
\end{pgfscope}%
\begin{pgfscope}%
\pgfsys@transformshift{2.312134in}{0.943734in}%
\pgfsys@useobject{currentmarker}{}%
\end{pgfscope}%
\begin{pgfscope}%
\pgfsys@transformshift{2.312134in}{0.943734in}%
\pgfsys@useobject{currentmarker}{}%
\end{pgfscope}%
\begin{pgfscope}%
\pgfsys@transformshift{2.312134in}{0.943734in}%
\pgfsys@useobject{currentmarker}{}%
\end{pgfscope}%
\begin{pgfscope}%
\pgfsys@transformshift{2.312134in}{0.943734in}%
\pgfsys@useobject{currentmarker}{}%
\end{pgfscope}%
\begin{pgfscope}%
\pgfsys@transformshift{2.312134in}{0.943734in}%
\pgfsys@useobject{currentmarker}{}%
\end{pgfscope}%
\begin{pgfscope}%
\pgfsys@transformshift{2.312134in}{0.943734in}%
\pgfsys@useobject{currentmarker}{}%
\end{pgfscope}%
\begin{pgfscope}%
\pgfsys@transformshift{2.312134in}{0.943734in}%
\pgfsys@useobject{currentmarker}{}%
\end{pgfscope}%
\begin{pgfscope}%
\pgfsys@transformshift{2.312134in}{0.943734in}%
\pgfsys@useobject{currentmarker}{}%
\end{pgfscope}%
\begin{pgfscope}%
\pgfsys@transformshift{2.312134in}{0.943734in}%
\pgfsys@useobject{currentmarker}{}%
\end{pgfscope}%
\begin{pgfscope}%
\pgfsys@transformshift{2.312134in}{0.943734in}%
\pgfsys@useobject{currentmarker}{}%
\end{pgfscope}%
\begin{pgfscope}%
\pgfsys@transformshift{2.312134in}{0.943734in}%
\pgfsys@useobject{currentmarker}{}%
\end{pgfscope}%
\begin{pgfscope}%
\pgfsys@transformshift{2.312134in}{0.943734in}%
\pgfsys@useobject{currentmarker}{}%
\end{pgfscope}%
\begin{pgfscope}%
\pgfsys@transformshift{2.312134in}{0.943734in}%
\pgfsys@useobject{currentmarker}{}%
\end{pgfscope}%
\begin{pgfscope}%
\pgfsys@transformshift{2.312134in}{0.943734in}%
\pgfsys@useobject{currentmarker}{}%
\end{pgfscope}%
\begin{pgfscope}%
\pgfsys@transformshift{2.312134in}{0.943734in}%
\pgfsys@useobject{currentmarker}{}%
\end{pgfscope}%
\begin{pgfscope}%
\pgfsys@transformshift{2.312134in}{0.943734in}%
\pgfsys@useobject{currentmarker}{}%
\end{pgfscope}%
\begin{pgfscope}%
\pgfsys@transformshift{2.312134in}{0.943734in}%
\pgfsys@useobject{currentmarker}{}%
\end{pgfscope}%
\begin{pgfscope}%
\pgfsys@transformshift{2.312134in}{0.943734in}%
\pgfsys@useobject{currentmarker}{}%
\end{pgfscope}%
\begin{pgfscope}%
\pgfsys@transformshift{2.312134in}{0.943734in}%
\pgfsys@useobject{currentmarker}{}%
\end{pgfscope}%
\begin{pgfscope}%
\pgfsys@transformshift{2.312134in}{0.943734in}%
\pgfsys@useobject{currentmarker}{}%
\end{pgfscope}%
\begin{pgfscope}%
\pgfsys@transformshift{2.312134in}{0.943734in}%
\pgfsys@useobject{currentmarker}{}%
\end{pgfscope}%
\begin{pgfscope}%
\pgfsys@transformshift{2.312134in}{0.943734in}%
\pgfsys@useobject{currentmarker}{}%
\end{pgfscope}%
\begin{pgfscope}%
\pgfsys@transformshift{2.312134in}{0.943734in}%
\pgfsys@useobject{currentmarker}{}%
\end{pgfscope}%
\begin{pgfscope}%
\pgfsys@transformshift{2.312134in}{0.943734in}%
\pgfsys@useobject{currentmarker}{}%
\end{pgfscope}%
\begin{pgfscope}%
\pgfsys@transformshift{2.312134in}{0.943734in}%
\pgfsys@useobject{currentmarker}{}%
\end{pgfscope}%
\begin{pgfscope}%
\pgfsys@transformshift{2.312134in}{0.943734in}%
\pgfsys@useobject{currentmarker}{}%
\end{pgfscope}%
\begin{pgfscope}%
\pgfsys@transformshift{2.312134in}{0.943734in}%
\pgfsys@useobject{currentmarker}{}%
\end{pgfscope}%
\begin{pgfscope}%
\pgfsys@transformshift{2.312134in}{0.943734in}%
\pgfsys@useobject{currentmarker}{}%
\end{pgfscope}%
\begin{pgfscope}%
\pgfsys@transformshift{2.312134in}{0.943734in}%
\pgfsys@useobject{currentmarker}{}%
\end{pgfscope}%
\begin{pgfscope}%
\pgfsys@transformshift{2.312134in}{0.943734in}%
\pgfsys@useobject{currentmarker}{}%
\end{pgfscope}%
\begin{pgfscope}%
\pgfsys@transformshift{2.312134in}{0.943734in}%
\pgfsys@useobject{currentmarker}{}%
\end{pgfscope}%
\begin{pgfscope}%
\pgfsys@transformshift{2.312134in}{0.943734in}%
\pgfsys@useobject{currentmarker}{}%
\end{pgfscope}%
\begin{pgfscope}%
\pgfsys@transformshift{2.312134in}{0.943734in}%
\pgfsys@useobject{currentmarker}{}%
\end{pgfscope}%
\begin{pgfscope}%
\pgfsys@transformshift{2.312134in}{0.943734in}%
\pgfsys@useobject{currentmarker}{}%
\end{pgfscope}%
\begin{pgfscope}%
\pgfsys@transformshift{2.312134in}{0.943734in}%
\pgfsys@useobject{currentmarker}{}%
\end{pgfscope}%
\begin{pgfscope}%
\pgfsys@transformshift{2.312134in}{0.943734in}%
\pgfsys@useobject{currentmarker}{}%
\end{pgfscope}%
\begin{pgfscope}%
\pgfsys@transformshift{2.312134in}{0.943734in}%
\pgfsys@useobject{currentmarker}{}%
\end{pgfscope}%
\begin{pgfscope}%
\pgfsys@transformshift{2.312134in}{0.943734in}%
\pgfsys@useobject{currentmarker}{}%
\end{pgfscope}%
\begin{pgfscope}%
\pgfsys@transformshift{2.312134in}{0.943734in}%
\pgfsys@useobject{currentmarker}{}%
\end{pgfscope}%
\begin{pgfscope}%
\pgfsys@transformshift{2.312134in}{0.943734in}%
\pgfsys@useobject{currentmarker}{}%
\end{pgfscope}%
\begin{pgfscope}%
\pgfsys@transformshift{2.312134in}{0.943734in}%
\pgfsys@useobject{currentmarker}{}%
\end{pgfscope}%
\begin{pgfscope}%
\pgfsys@transformshift{2.312134in}{0.943734in}%
\pgfsys@useobject{currentmarker}{}%
\end{pgfscope}%
\begin{pgfscope}%
\pgfsys@transformshift{2.312134in}{0.943734in}%
\pgfsys@useobject{currentmarker}{}%
\end{pgfscope}%
\begin{pgfscope}%
\pgfsys@transformshift{2.312134in}{0.943734in}%
\pgfsys@useobject{currentmarker}{}%
\end{pgfscope}%
\begin{pgfscope}%
\pgfsys@transformshift{2.312134in}{0.943734in}%
\pgfsys@useobject{currentmarker}{}%
\end{pgfscope}%
\begin{pgfscope}%
\pgfsys@transformshift{2.312134in}{0.943734in}%
\pgfsys@useobject{currentmarker}{}%
\end{pgfscope}%
\begin{pgfscope}%
\pgfsys@transformshift{2.312134in}{0.943734in}%
\pgfsys@useobject{currentmarker}{}%
\end{pgfscope}%
\begin{pgfscope}%
\pgfsys@transformshift{2.312134in}{0.943734in}%
\pgfsys@useobject{currentmarker}{}%
\end{pgfscope}%
\begin{pgfscope}%
\pgfsys@transformshift{2.312134in}{0.943734in}%
\pgfsys@useobject{currentmarker}{}%
\end{pgfscope}%
\begin{pgfscope}%
\pgfsys@transformshift{2.312134in}{0.943734in}%
\pgfsys@useobject{currentmarker}{}%
\end{pgfscope}%
\begin{pgfscope}%
\pgfsys@transformshift{2.312134in}{0.943734in}%
\pgfsys@useobject{currentmarker}{}%
\end{pgfscope}%
\begin{pgfscope}%
\pgfsys@transformshift{2.312134in}{0.943734in}%
\pgfsys@useobject{currentmarker}{}%
\end{pgfscope}%
\begin{pgfscope}%
\pgfsys@transformshift{2.312134in}{0.943734in}%
\pgfsys@useobject{currentmarker}{}%
\end{pgfscope}%
\begin{pgfscope}%
\pgfsys@transformshift{2.312134in}{0.943734in}%
\pgfsys@useobject{currentmarker}{}%
\end{pgfscope}%
\begin{pgfscope}%
\pgfsys@transformshift{2.312134in}{0.943734in}%
\pgfsys@useobject{currentmarker}{}%
\end{pgfscope}%
\begin{pgfscope}%
\pgfsys@transformshift{2.312134in}{0.943734in}%
\pgfsys@useobject{currentmarker}{}%
\end{pgfscope}%
\begin{pgfscope}%
\pgfsys@transformshift{2.312134in}{0.943734in}%
\pgfsys@useobject{currentmarker}{}%
\end{pgfscope}%
\begin{pgfscope}%
\pgfsys@transformshift{2.312134in}{0.943734in}%
\pgfsys@useobject{currentmarker}{}%
\end{pgfscope}%
\begin{pgfscope}%
\pgfsys@transformshift{2.312134in}{0.943734in}%
\pgfsys@useobject{currentmarker}{}%
\end{pgfscope}%
\begin{pgfscope}%
\pgfsys@transformshift{2.312134in}{0.943734in}%
\pgfsys@useobject{currentmarker}{}%
\end{pgfscope}%
\begin{pgfscope}%
\pgfsys@transformshift{2.312134in}{0.943734in}%
\pgfsys@useobject{currentmarker}{}%
\end{pgfscope}%
\begin{pgfscope}%
\pgfsys@transformshift{2.312134in}{0.943734in}%
\pgfsys@useobject{currentmarker}{}%
\end{pgfscope}%
\begin{pgfscope}%
\pgfsys@transformshift{2.312134in}{0.943734in}%
\pgfsys@useobject{currentmarker}{}%
\end{pgfscope}%
\begin{pgfscope}%
\pgfsys@transformshift{2.312134in}{0.943734in}%
\pgfsys@useobject{currentmarker}{}%
\end{pgfscope}%
\begin{pgfscope}%
\pgfsys@transformshift{2.312134in}{0.943734in}%
\pgfsys@useobject{currentmarker}{}%
\end{pgfscope}%
\begin{pgfscope}%
\pgfsys@transformshift{2.312134in}{0.943734in}%
\pgfsys@useobject{currentmarker}{}%
\end{pgfscope}%
\begin{pgfscope}%
\pgfsys@transformshift{2.312134in}{0.943734in}%
\pgfsys@useobject{currentmarker}{}%
\end{pgfscope}%
\begin{pgfscope}%
\pgfsys@transformshift{2.312134in}{0.943734in}%
\pgfsys@useobject{currentmarker}{}%
\end{pgfscope}%
\begin{pgfscope}%
\pgfsys@transformshift{2.312134in}{0.943734in}%
\pgfsys@useobject{currentmarker}{}%
\end{pgfscope}%
\begin{pgfscope}%
\pgfsys@transformshift{2.312134in}{0.943734in}%
\pgfsys@useobject{currentmarker}{}%
\end{pgfscope}%
\begin{pgfscope}%
\pgfsys@transformshift{2.312134in}{0.943734in}%
\pgfsys@useobject{currentmarker}{}%
\end{pgfscope}%
\begin{pgfscope}%
\pgfsys@transformshift{2.312134in}{0.943734in}%
\pgfsys@useobject{currentmarker}{}%
\end{pgfscope}%
\begin{pgfscope}%
\pgfsys@transformshift{2.312134in}{0.943734in}%
\pgfsys@useobject{currentmarker}{}%
\end{pgfscope}%
\begin{pgfscope}%
\pgfsys@transformshift{2.312134in}{0.943734in}%
\pgfsys@useobject{currentmarker}{}%
\end{pgfscope}%
\begin{pgfscope}%
\pgfsys@transformshift{2.312134in}{0.943734in}%
\pgfsys@useobject{currentmarker}{}%
\end{pgfscope}%
\begin{pgfscope}%
\pgfsys@transformshift{2.312134in}{0.943734in}%
\pgfsys@useobject{currentmarker}{}%
\end{pgfscope}%
\begin{pgfscope}%
\pgfsys@transformshift{2.312134in}{0.943734in}%
\pgfsys@useobject{currentmarker}{}%
\end{pgfscope}%
\begin{pgfscope}%
\pgfsys@transformshift{2.312134in}{0.943734in}%
\pgfsys@useobject{currentmarker}{}%
\end{pgfscope}%
\begin{pgfscope}%
\pgfsys@transformshift{2.312134in}{0.943734in}%
\pgfsys@useobject{currentmarker}{}%
\end{pgfscope}%
\begin{pgfscope}%
\pgfsys@transformshift{2.312134in}{0.943734in}%
\pgfsys@useobject{currentmarker}{}%
\end{pgfscope}%
\begin{pgfscope}%
\pgfsys@transformshift{2.312134in}{0.943734in}%
\pgfsys@useobject{currentmarker}{}%
\end{pgfscope}%
\begin{pgfscope}%
\pgfsys@transformshift{2.312134in}{0.943734in}%
\pgfsys@useobject{currentmarker}{}%
\end{pgfscope}%
\begin{pgfscope}%
\pgfsys@transformshift{2.312134in}{0.943734in}%
\pgfsys@useobject{currentmarker}{}%
\end{pgfscope}%
\begin{pgfscope}%
\pgfsys@transformshift{2.312134in}{0.943734in}%
\pgfsys@useobject{currentmarker}{}%
\end{pgfscope}%
\begin{pgfscope}%
\pgfsys@transformshift{2.312134in}{0.943734in}%
\pgfsys@useobject{currentmarker}{}%
\end{pgfscope}%
\begin{pgfscope}%
\pgfsys@transformshift{2.312134in}{0.943734in}%
\pgfsys@useobject{currentmarker}{}%
\end{pgfscope}%
\begin{pgfscope}%
\pgfsys@transformshift{2.312134in}{0.943734in}%
\pgfsys@useobject{currentmarker}{}%
\end{pgfscope}%
\begin{pgfscope}%
\pgfsys@transformshift{2.312134in}{0.943734in}%
\pgfsys@useobject{currentmarker}{}%
\end{pgfscope}%
\begin{pgfscope}%
\pgfsys@transformshift{2.312134in}{0.943734in}%
\pgfsys@useobject{currentmarker}{}%
\end{pgfscope}%
\begin{pgfscope}%
\pgfsys@transformshift{2.312134in}{0.943734in}%
\pgfsys@useobject{currentmarker}{}%
\end{pgfscope}%
\begin{pgfscope}%
\pgfsys@transformshift{2.312134in}{0.943734in}%
\pgfsys@useobject{currentmarker}{}%
\end{pgfscope}%
\begin{pgfscope}%
\pgfsys@transformshift{2.312134in}{0.943734in}%
\pgfsys@useobject{currentmarker}{}%
\end{pgfscope}%
\begin{pgfscope}%
\pgfsys@transformshift{2.312134in}{0.943734in}%
\pgfsys@useobject{currentmarker}{}%
\end{pgfscope}%
\begin{pgfscope}%
\pgfsys@transformshift{2.312134in}{0.943734in}%
\pgfsys@useobject{currentmarker}{}%
\end{pgfscope}%
\begin{pgfscope}%
\pgfsys@transformshift{2.312134in}{0.943734in}%
\pgfsys@useobject{currentmarker}{}%
\end{pgfscope}%
\begin{pgfscope}%
\pgfsys@transformshift{2.312134in}{0.943734in}%
\pgfsys@useobject{currentmarker}{}%
\end{pgfscope}%
\begin{pgfscope}%
\pgfsys@transformshift{2.312134in}{0.943734in}%
\pgfsys@useobject{currentmarker}{}%
\end{pgfscope}%
\begin{pgfscope}%
\pgfsys@transformshift{2.312134in}{0.943734in}%
\pgfsys@useobject{currentmarker}{}%
\end{pgfscope}%
\begin{pgfscope}%
\pgfsys@transformshift{2.312134in}{0.943734in}%
\pgfsys@useobject{currentmarker}{}%
\end{pgfscope}%
\begin{pgfscope}%
\pgfsys@transformshift{2.312134in}{0.943734in}%
\pgfsys@useobject{currentmarker}{}%
\end{pgfscope}%
\begin{pgfscope}%
\pgfsys@transformshift{2.312134in}{0.943734in}%
\pgfsys@useobject{currentmarker}{}%
\end{pgfscope}%
\begin{pgfscope}%
\pgfsys@transformshift{2.312134in}{0.943734in}%
\pgfsys@useobject{currentmarker}{}%
\end{pgfscope}%
\begin{pgfscope}%
\pgfsys@transformshift{2.312134in}{0.943734in}%
\pgfsys@useobject{currentmarker}{}%
\end{pgfscope}%
\begin{pgfscope}%
\pgfsys@transformshift{2.312134in}{0.943734in}%
\pgfsys@useobject{currentmarker}{}%
\end{pgfscope}%
\begin{pgfscope}%
\pgfsys@transformshift{2.312134in}{0.943734in}%
\pgfsys@useobject{currentmarker}{}%
\end{pgfscope}%
\begin{pgfscope}%
\pgfsys@transformshift{2.312134in}{0.943734in}%
\pgfsys@useobject{currentmarker}{}%
\end{pgfscope}%
\begin{pgfscope}%
\pgfsys@transformshift{2.312134in}{0.943734in}%
\pgfsys@useobject{currentmarker}{}%
\end{pgfscope}%
\begin{pgfscope}%
\pgfsys@transformshift{2.312134in}{0.943734in}%
\pgfsys@useobject{currentmarker}{}%
\end{pgfscope}%
\begin{pgfscope}%
\pgfsys@transformshift{2.312134in}{0.943734in}%
\pgfsys@useobject{currentmarker}{}%
\end{pgfscope}%
\begin{pgfscope}%
\pgfsys@transformshift{2.312134in}{0.943734in}%
\pgfsys@useobject{currentmarker}{}%
\end{pgfscope}%
\begin{pgfscope}%
\pgfsys@transformshift{2.312134in}{0.943734in}%
\pgfsys@useobject{currentmarker}{}%
\end{pgfscope}%
\begin{pgfscope}%
\pgfsys@transformshift{2.312134in}{0.943734in}%
\pgfsys@useobject{currentmarker}{}%
\end{pgfscope}%
\begin{pgfscope}%
\pgfsys@transformshift{2.312134in}{0.943734in}%
\pgfsys@useobject{currentmarker}{}%
\end{pgfscope}%
\begin{pgfscope}%
\pgfsys@transformshift{2.312134in}{0.943734in}%
\pgfsys@useobject{currentmarker}{}%
\end{pgfscope}%
\begin{pgfscope}%
\pgfsys@transformshift{2.312134in}{0.943734in}%
\pgfsys@useobject{currentmarker}{}%
\end{pgfscope}%
\begin{pgfscope}%
\pgfsys@transformshift{2.312134in}{0.943734in}%
\pgfsys@useobject{currentmarker}{}%
\end{pgfscope}%
\begin{pgfscope}%
\pgfsys@transformshift{2.312134in}{0.943734in}%
\pgfsys@useobject{currentmarker}{}%
\end{pgfscope}%
\begin{pgfscope}%
\pgfsys@transformshift{2.312134in}{0.943734in}%
\pgfsys@useobject{currentmarker}{}%
\end{pgfscope}%
\begin{pgfscope}%
\pgfsys@transformshift{2.312134in}{0.943734in}%
\pgfsys@useobject{currentmarker}{}%
\end{pgfscope}%
\begin{pgfscope}%
\pgfsys@transformshift{2.312134in}{0.943734in}%
\pgfsys@useobject{currentmarker}{}%
\end{pgfscope}%
\begin{pgfscope}%
\pgfsys@transformshift{2.312134in}{0.943734in}%
\pgfsys@useobject{currentmarker}{}%
\end{pgfscope}%
\begin{pgfscope}%
\pgfsys@transformshift{2.312134in}{0.943734in}%
\pgfsys@useobject{currentmarker}{}%
\end{pgfscope}%
\begin{pgfscope}%
\pgfsys@transformshift{2.312134in}{0.943734in}%
\pgfsys@useobject{currentmarker}{}%
\end{pgfscope}%
\begin{pgfscope}%
\pgfsys@transformshift{2.312134in}{0.943734in}%
\pgfsys@useobject{currentmarker}{}%
\end{pgfscope}%
\begin{pgfscope}%
\pgfsys@transformshift{2.312134in}{0.943734in}%
\pgfsys@useobject{currentmarker}{}%
\end{pgfscope}%
\begin{pgfscope}%
\pgfsys@transformshift{2.312134in}{0.943734in}%
\pgfsys@useobject{currentmarker}{}%
\end{pgfscope}%
\begin{pgfscope}%
\pgfsys@transformshift{2.312134in}{0.943734in}%
\pgfsys@useobject{currentmarker}{}%
\end{pgfscope}%
\begin{pgfscope}%
\pgfsys@transformshift{2.312134in}{0.943734in}%
\pgfsys@useobject{currentmarker}{}%
\end{pgfscope}%
\begin{pgfscope}%
\pgfsys@transformshift{2.312134in}{0.943734in}%
\pgfsys@useobject{currentmarker}{}%
\end{pgfscope}%
\begin{pgfscope}%
\pgfsys@transformshift{2.312134in}{0.943734in}%
\pgfsys@useobject{currentmarker}{}%
\end{pgfscope}%
\begin{pgfscope}%
\pgfsys@transformshift{2.312134in}{0.943734in}%
\pgfsys@useobject{currentmarker}{}%
\end{pgfscope}%
\begin{pgfscope}%
\pgfsys@transformshift{2.312134in}{0.943734in}%
\pgfsys@useobject{currentmarker}{}%
\end{pgfscope}%
\begin{pgfscope}%
\pgfsys@transformshift{2.312134in}{0.943734in}%
\pgfsys@useobject{currentmarker}{}%
\end{pgfscope}%
\begin{pgfscope}%
\pgfsys@transformshift{2.312134in}{0.943734in}%
\pgfsys@useobject{currentmarker}{}%
\end{pgfscope}%
\begin{pgfscope}%
\pgfsys@transformshift{2.312134in}{0.943734in}%
\pgfsys@useobject{currentmarker}{}%
\end{pgfscope}%
\begin{pgfscope}%
\pgfsys@transformshift{2.312134in}{0.943734in}%
\pgfsys@useobject{currentmarker}{}%
\end{pgfscope}%
\begin{pgfscope}%
\pgfsys@transformshift{2.312134in}{0.943734in}%
\pgfsys@useobject{currentmarker}{}%
\end{pgfscope}%
\begin{pgfscope}%
\pgfsys@transformshift{2.312134in}{0.943734in}%
\pgfsys@useobject{currentmarker}{}%
\end{pgfscope}%
\begin{pgfscope}%
\pgfsys@transformshift{2.312134in}{0.943734in}%
\pgfsys@useobject{currentmarker}{}%
\end{pgfscope}%
\begin{pgfscope}%
\pgfsys@transformshift{2.312134in}{0.943734in}%
\pgfsys@useobject{currentmarker}{}%
\end{pgfscope}%
\begin{pgfscope}%
\pgfsys@transformshift{2.312134in}{0.943734in}%
\pgfsys@useobject{currentmarker}{}%
\end{pgfscope}%
\begin{pgfscope}%
\pgfsys@transformshift{2.312134in}{0.943734in}%
\pgfsys@useobject{currentmarker}{}%
\end{pgfscope}%
\begin{pgfscope}%
\pgfsys@transformshift{2.312134in}{0.943734in}%
\pgfsys@useobject{currentmarker}{}%
\end{pgfscope}%
\begin{pgfscope}%
\pgfsys@transformshift{2.312134in}{0.943734in}%
\pgfsys@useobject{currentmarker}{}%
\end{pgfscope}%
\begin{pgfscope}%
\pgfsys@transformshift{2.312134in}{0.943734in}%
\pgfsys@useobject{currentmarker}{}%
\end{pgfscope}%
\begin{pgfscope}%
\pgfsys@transformshift{2.312134in}{0.943734in}%
\pgfsys@useobject{currentmarker}{}%
\end{pgfscope}%
\begin{pgfscope}%
\pgfsys@transformshift{2.312134in}{0.943734in}%
\pgfsys@useobject{currentmarker}{}%
\end{pgfscope}%
\begin{pgfscope}%
\pgfsys@transformshift{2.312134in}{0.943734in}%
\pgfsys@useobject{currentmarker}{}%
\end{pgfscope}%
\begin{pgfscope}%
\pgfsys@transformshift{2.312134in}{0.943734in}%
\pgfsys@useobject{currentmarker}{}%
\end{pgfscope}%
\begin{pgfscope}%
\pgfsys@transformshift{2.312134in}{0.943734in}%
\pgfsys@useobject{currentmarker}{}%
\end{pgfscope}%
\begin{pgfscope}%
\pgfsys@transformshift{2.312134in}{0.943734in}%
\pgfsys@useobject{currentmarker}{}%
\end{pgfscope}%
\begin{pgfscope}%
\pgfsys@transformshift{2.312134in}{0.943734in}%
\pgfsys@useobject{currentmarker}{}%
\end{pgfscope}%
\begin{pgfscope}%
\pgfsys@transformshift{2.312134in}{0.943734in}%
\pgfsys@useobject{currentmarker}{}%
\end{pgfscope}%
\begin{pgfscope}%
\pgfsys@transformshift{2.312134in}{0.943734in}%
\pgfsys@useobject{currentmarker}{}%
\end{pgfscope}%
\begin{pgfscope}%
\pgfsys@transformshift{2.312134in}{0.943734in}%
\pgfsys@useobject{currentmarker}{}%
\end{pgfscope}%
\begin{pgfscope}%
\pgfsys@transformshift{2.312134in}{0.943734in}%
\pgfsys@useobject{currentmarker}{}%
\end{pgfscope}%
\begin{pgfscope}%
\pgfsys@transformshift{2.312134in}{0.943734in}%
\pgfsys@useobject{currentmarker}{}%
\end{pgfscope}%
\begin{pgfscope}%
\pgfsys@transformshift{2.312134in}{0.943734in}%
\pgfsys@useobject{currentmarker}{}%
\end{pgfscope}%
\begin{pgfscope}%
\pgfsys@transformshift{2.312134in}{0.943734in}%
\pgfsys@useobject{currentmarker}{}%
\end{pgfscope}%
\begin{pgfscope}%
\pgfsys@transformshift{2.312134in}{0.943734in}%
\pgfsys@useobject{currentmarker}{}%
\end{pgfscope}%
\begin{pgfscope}%
\pgfsys@transformshift{2.312134in}{0.943734in}%
\pgfsys@useobject{currentmarker}{}%
\end{pgfscope}%
\begin{pgfscope}%
\pgfsys@transformshift{2.312134in}{0.943734in}%
\pgfsys@useobject{currentmarker}{}%
\end{pgfscope}%
\begin{pgfscope}%
\pgfsys@transformshift{2.312134in}{0.943734in}%
\pgfsys@useobject{currentmarker}{}%
\end{pgfscope}%
\begin{pgfscope}%
\pgfsys@transformshift{2.312134in}{0.943734in}%
\pgfsys@useobject{currentmarker}{}%
\end{pgfscope}%
\begin{pgfscope}%
\pgfsys@transformshift{2.312134in}{0.943734in}%
\pgfsys@useobject{currentmarker}{}%
\end{pgfscope}%
\begin{pgfscope}%
\pgfsys@transformshift{2.312134in}{0.943734in}%
\pgfsys@useobject{currentmarker}{}%
\end{pgfscope}%
\begin{pgfscope}%
\pgfsys@transformshift{2.312134in}{0.943734in}%
\pgfsys@useobject{currentmarker}{}%
\end{pgfscope}%
\begin{pgfscope}%
\pgfsys@transformshift{2.312134in}{0.943734in}%
\pgfsys@useobject{currentmarker}{}%
\end{pgfscope}%
\begin{pgfscope}%
\pgfsys@transformshift{2.312134in}{0.943734in}%
\pgfsys@useobject{currentmarker}{}%
\end{pgfscope}%
\begin{pgfscope}%
\pgfsys@transformshift{2.312134in}{0.943734in}%
\pgfsys@useobject{currentmarker}{}%
\end{pgfscope}%
\begin{pgfscope}%
\pgfsys@transformshift{2.312134in}{0.943734in}%
\pgfsys@useobject{currentmarker}{}%
\end{pgfscope}%
\begin{pgfscope}%
\pgfsys@transformshift{2.312134in}{0.943734in}%
\pgfsys@useobject{currentmarker}{}%
\end{pgfscope}%
\begin{pgfscope}%
\pgfsys@transformshift{2.312134in}{0.943734in}%
\pgfsys@useobject{currentmarker}{}%
\end{pgfscope}%
\begin{pgfscope}%
\pgfsys@transformshift{2.312134in}{0.943734in}%
\pgfsys@useobject{currentmarker}{}%
\end{pgfscope}%
\begin{pgfscope}%
\pgfsys@transformshift{2.312134in}{0.943734in}%
\pgfsys@useobject{currentmarker}{}%
\end{pgfscope}%
\begin{pgfscope}%
\pgfsys@transformshift{2.312134in}{0.943734in}%
\pgfsys@useobject{currentmarker}{}%
\end{pgfscope}%
\begin{pgfscope}%
\pgfsys@transformshift{2.312134in}{0.943734in}%
\pgfsys@useobject{currentmarker}{}%
\end{pgfscope}%
\begin{pgfscope}%
\pgfsys@transformshift{2.312134in}{0.943734in}%
\pgfsys@useobject{currentmarker}{}%
\end{pgfscope}%
\begin{pgfscope}%
\pgfsys@transformshift{2.312134in}{0.943734in}%
\pgfsys@useobject{currentmarker}{}%
\end{pgfscope}%
\begin{pgfscope}%
\pgfsys@transformshift{2.312134in}{0.943734in}%
\pgfsys@useobject{currentmarker}{}%
\end{pgfscope}%
\begin{pgfscope}%
\pgfsys@transformshift{2.312134in}{0.943734in}%
\pgfsys@useobject{currentmarker}{}%
\end{pgfscope}%
\begin{pgfscope}%
\pgfsys@transformshift{2.312134in}{0.943734in}%
\pgfsys@useobject{currentmarker}{}%
\end{pgfscope}%
\begin{pgfscope}%
\pgfsys@transformshift{2.312134in}{0.943734in}%
\pgfsys@useobject{currentmarker}{}%
\end{pgfscope}%
\begin{pgfscope}%
\pgfsys@transformshift{2.312134in}{0.943734in}%
\pgfsys@useobject{currentmarker}{}%
\end{pgfscope}%
\begin{pgfscope}%
\pgfsys@transformshift{2.312134in}{0.943734in}%
\pgfsys@useobject{currentmarker}{}%
\end{pgfscope}%
\begin{pgfscope}%
\pgfsys@transformshift{2.312134in}{0.943734in}%
\pgfsys@useobject{currentmarker}{}%
\end{pgfscope}%
\begin{pgfscope}%
\pgfsys@transformshift{2.312134in}{0.943734in}%
\pgfsys@useobject{currentmarker}{}%
\end{pgfscope}%
\begin{pgfscope}%
\pgfsys@transformshift{2.312134in}{0.943734in}%
\pgfsys@useobject{currentmarker}{}%
\end{pgfscope}%
\begin{pgfscope}%
\pgfsys@transformshift{2.312134in}{0.943734in}%
\pgfsys@useobject{currentmarker}{}%
\end{pgfscope}%
\begin{pgfscope}%
\pgfsys@transformshift{2.312134in}{0.943734in}%
\pgfsys@useobject{currentmarker}{}%
\end{pgfscope}%
\begin{pgfscope}%
\pgfsys@transformshift{2.312134in}{0.943734in}%
\pgfsys@useobject{currentmarker}{}%
\end{pgfscope}%
\begin{pgfscope}%
\pgfsys@transformshift{2.312134in}{0.943734in}%
\pgfsys@useobject{currentmarker}{}%
\end{pgfscope}%
\begin{pgfscope}%
\pgfsys@transformshift{2.312134in}{0.943734in}%
\pgfsys@useobject{currentmarker}{}%
\end{pgfscope}%
\begin{pgfscope}%
\pgfsys@transformshift{2.312134in}{0.943734in}%
\pgfsys@useobject{currentmarker}{}%
\end{pgfscope}%
\begin{pgfscope}%
\pgfsys@transformshift{2.312134in}{0.943734in}%
\pgfsys@useobject{currentmarker}{}%
\end{pgfscope}%
\begin{pgfscope}%
\pgfsys@transformshift{2.312134in}{0.943734in}%
\pgfsys@useobject{currentmarker}{}%
\end{pgfscope}%
\begin{pgfscope}%
\pgfsys@transformshift{2.312134in}{0.943734in}%
\pgfsys@useobject{currentmarker}{}%
\end{pgfscope}%
\begin{pgfscope}%
\pgfsys@transformshift{2.312134in}{0.943734in}%
\pgfsys@useobject{currentmarker}{}%
\end{pgfscope}%
\begin{pgfscope}%
\pgfsys@transformshift{2.312134in}{0.943734in}%
\pgfsys@useobject{currentmarker}{}%
\end{pgfscope}%
\begin{pgfscope}%
\pgfsys@transformshift{2.312134in}{0.943734in}%
\pgfsys@useobject{currentmarker}{}%
\end{pgfscope}%
\begin{pgfscope}%
\pgfsys@transformshift{2.312134in}{0.943734in}%
\pgfsys@useobject{currentmarker}{}%
\end{pgfscope}%
\begin{pgfscope}%
\pgfsys@transformshift{2.312134in}{0.943734in}%
\pgfsys@useobject{currentmarker}{}%
\end{pgfscope}%
\begin{pgfscope}%
\pgfsys@transformshift{2.312134in}{0.943734in}%
\pgfsys@useobject{currentmarker}{}%
\end{pgfscope}%
\begin{pgfscope}%
\pgfsys@transformshift{2.312134in}{0.943734in}%
\pgfsys@useobject{currentmarker}{}%
\end{pgfscope}%
\begin{pgfscope}%
\pgfsys@transformshift{2.312134in}{0.943734in}%
\pgfsys@useobject{currentmarker}{}%
\end{pgfscope}%
\begin{pgfscope}%
\pgfsys@transformshift{2.312134in}{0.943734in}%
\pgfsys@useobject{currentmarker}{}%
\end{pgfscope}%
\begin{pgfscope}%
\pgfsys@transformshift{2.312134in}{0.943734in}%
\pgfsys@useobject{currentmarker}{}%
\end{pgfscope}%
\begin{pgfscope}%
\pgfsys@transformshift{2.312134in}{0.943734in}%
\pgfsys@useobject{currentmarker}{}%
\end{pgfscope}%
\begin{pgfscope}%
\pgfsys@transformshift{2.312134in}{0.943734in}%
\pgfsys@useobject{currentmarker}{}%
\end{pgfscope}%
\begin{pgfscope}%
\pgfsys@transformshift{2.312134in}{0.943734in}%
\pgfsys@useobject{currentmarker}{}%
\end{pgfscope}%
\begin{pgfscope}%
\pgfsys@transformshift{2.312134in}{0.943734in}%
\pgfsys@useobject{currentmarker}{}%
\end{pgfscope}%
\begin{pgfscope}%
\pgfsys@transformshift{2.312134in}{0.943734in}%
\pgfsys@useobject{currentmarker}{}%
\end{pgfscope}%
\begin{pgfscope}%
\pgfsys@transformshift{2.312134in}{0.943734in}%
\pgfsys@useobject{currentmarker}{}%
\end{pgfscope}%
\begin{pgfscope}%
\pgfsys@transformshift{2.312134in}{0.943734in}%
\pgfsys@useobject{currentmarker}{}%
\end{pgfscope}%
\begin{pgfscope}%
\pgfsys@transformshift{2.312134in}{0.943734in}%
\pgfsys@useobject{currentmarker}{}%
\end{pgfscope}%
\begin{pgfscope}%
\pgfsys@transformshift{2.312134in}{0.943734in}%
\pgfsys@useobject{currentmarker}{}%
\end{pgfscope}%
\begin{pgfscope}%
\pgfsys@transformshift{2.312134in}{0.943734in}%
\pgfsys@useobject{currentmarker}{}%
\end{pgfscope}%
\begin{pgfscope}%
\pgfsys@transformshift{2.312134in}{0.943734in}%
\pgfsys@useobject{currentmarker}{}%
\end{pgfscope}%
\begin{pgfscope}%
\pgfsys@transformshift{2.312134in}{0.943734in}%
\pgfsys@useobject{currentmarker}{}%
\end{pgfscope}%
\begin{pgfscope}%
\pgfsys@transformshift{2.312134in}{0.943734in}%
\pgfsys@useobject{currentmarker}{}%
\end{pgfscope}%
\begin{pgfscope}%
\pgfsys@transformshift{2.312134in}{0.943734in}%
\pgfsys@useobject{currentmarker}{}%
\end{pgfscope}%
\begin{pgfscope}%
\pgfsys@transformshift{2.312134in}{0.943734in}%
\pgfsys@useobject{currentmarker}{}%
\end{pgfscope}%
\begin{pgfscope}%
\pgfsys@transformshift{2.312134in}{0.943734in}%
\pgfsys@useobject{currentmarker}{}%
\end{pgfscope}%
\begin{pgfscope}%
\pgfsys@transformshift{2.312134in}{0.943734in}%
\pgfsys@useobject{currentmarker}{}%
\end{pgfscope}%
\begin{pgfscope}%
\pgfsys@transformshift{2.312134in}{0.943734in}%
\pgfsys@useobject{currentmarker}{}%
\end{pgfscope}%
\begin{pgfscope}%
\pgfsys@transformshift{2.312134in}{0.943734in}%
\pgfsys@useobject{currentmarker}{}%
\end{pgfscope}%
\begin{pgfscope}%
\pgfsys@transformshift{2.312134in}{0.943734in}%
\pgfsys@useobject{currentmarker}{}%
\end{pgfscope}%
\begin{pgfscope}%
\pgfsys@transformshift{2.312134in}{0.943734in}%
\pgfsys@useobject{currentmarker}{}%
\end{pgfscope}%
\begin{pgfscope}%
\pgfsys@transformshift{2.312134in}{0.943734in}%
\pgfsys@useobject{currentmarker}{}%
\end{pgfscope}%
\begin{pgfscope}%
\pgfsys@transformshift{2.312134in}{0.943734in}%
\pgfsys@useobject{currentmarker}{}%
\end{pgfscope}%
\begin{pgfscope}%
\pgfsys@transformshift{2.312134in}{0.943734in}%
\pgfsys@useobject{currentmarker}{}%
\end{pgfscope}%
\begin{pgfscope}%
\pgfsys@transformshift{2.312134in}{0.943734in}%
\pgfsys@useobject{currentmarker}{}%
\end{pgfscope}%
\begin{pgfscope}%
\pgfsys@transformshift{2.312134in}{0.943734in}%
\pgfsys@useobject{currentmarker}{}%
\end{pgfscope}%
\begin{pgfscope}%
\pgfsys@transformshift{2.312134in}{0.943734in}%
\pgfsys@useobject{currentmarker}{}%
\end{pgfscope}%
\begin{pgfscope}%
\pgfsys@transformshift{2.312134in}{0.943734in}%
\pgfsys@useobject{currentmarker}{}%
\end{pgfscope}%
\begin{pgfscope}%
\pgfsys@transformshift{2.312134in}{0.943734in}%
\pgfsys@useobject{currentmarker}{}%
\end{pgfscope}%
\begin{pgfscope}%
\pgfsys@transformshift{2.312134in}{0.943734in}%
\pgfsys@useobject{currentmarker}{}%
\end{pgfscope}%
\begin{pgfscope}%
\pgfsys@transformshift{2.312134in}{0.943734in}%
\pgfsys@useobject{currentmarker}{}%
\end{pgfscope}%
\begin{pgfscope}%
\pgfsys@transformshift{2.312134in}{0.943734in}%
\pgfsys@useobject{currentmarker}{}%
\end{pgfscope}%
\begin{pgfscope}%
\pgfsys@transformshift{2.312134in}{0.943734in}%
\pgfsys@useobject{currentmarker}{}%
\end{pgfscope}%
\begin{pgfscope}%
\pgfsys@transformshift{2.312134in}{0.943734in}%
\pgfsys@useobject{currentmarker}{}%
\end{pgfscope}%
\begin{pgfscope}%
\pgfsys@transformshift{2.312134in}{0.943734in}%
\pgfsys@useobject{currentmarker}{}%
\end{pgfscope}%
\begin{pgfscope}%
\pgfsys@transformshift{2.312134in}{0.943734in}%
\pgfsys@useobject{currentmarker}{}%
\end{pgfscope}%
\begin{pgfscope}%
\pgfsys@transformshift{2.312134in}{0.943734in}%
\pgfsys@useobject{currentmarker}{}%
\end{pgfscope}%
\begin{pgfscope}%
\pgfsys@transformshift{2.312134in}{0.943734in}%
\pgfsys@useobject{currentmarker}{}%
\end{pgfscope}%
\begin{pgfscope}%
\pgfsys@transformshift{2.312134in}{0.943734in}%
\pgfsys@useobject{currentmarker}{}%
\end{pgfscope}%
\begin{pgfscope}%
\pgfsys@transformshift{2.312134in}{0.943734in}%
\pgfsys@useobject{currentmarker}{}%
\end{pgfscope}%
\begin{pgfscope}%
\pgfsys@transformshift{2.312134in}{0.943734in}%
\pgfsys@useobject{currentmarker}{}%
\end{pgfscope}%
\begin{pgfscope}%
\pgfsys@transformshift{2.312134in}{0.943734in}%
\pgfsys@useobject{currentmarker}{}%
\end{pgfscope}%
\begin{pgfscope}%
\pgfsys@transformshift{2.312134in}{0.943734in}%
\pgfsys@useobject{currentmarker}{}%
\end{pgfscope}%
\begin{pgfscope}%
\pgfsys@transformshift{2.312134in}{0.943734in}%
\pgfsys@useobject{currentmarker}{}%
\end{pgfscope}%
\begin{pgfscope}%
\pgfsys@transformshift{2.312134in}{0.943734in}%
\pgfsys@useobject{currentmarker}{}%
\end{pgfscope}%
\begin{pgfscope}%
\pgfsys@transformshift{2.312134in}{0.943734in}%
\pgfsys@useobject{currentmarker}{}%
\end{pgfscope}%
\begin{pgfscope}%
\pgfsys@transformshift{2.312134in}{0.943734in}%
\pgfsys@useobject{currentmarker}{}%
\end{pgfscope}%
\begin{pgfscope}%
\pgfsys@transformshift{2.312134in}{0.943734in}%
\pgfsys@useobject{currentmarker}{}%
\end{pgfscope}%
\begin{pgfscope}%
\pgfsys@transformshift{2.312134in}{0.943734in}%
\pgfsys@useobject{currentmarker}{}%
\end{pgfscope}%
\begin{pgfscope}%
\pgfsys@transformshift{2.312134in}{0.943734in}%
\pgfsys@useobject{currentmarker}{}%
\end{pgfscope}%
\begin{pgfscope}%
\pgfsys@transformshift{2.312134in}{0.943734in}%
\pgfsys@useobject{currentmarker}{}%
\end{pgfscope}%
\begin{pgfscope}%
\pgfsys@transformshift{2.312134in}{0.943734in}%
\pgfsys@useobject{currentmarker}{}%
\end{pgfscope}%
\begin{pgfscope}%
\pgfsys@transformshift{2.312134in}{0.943734in}%
\pgfsys@useobject{currentmarker}{}%
\end{pgfscope}%
\begin{pgfscope}%
\pgfsys@transformshift{2.312134in}{0.943734in}%
\pgfsys@useobject{currentmarker}{}%
\end{pgfscope}%
\begin{pgfscope}%
\pgfsys@transformshift{2.312134in}{0.943734in}%
\pgfsys@useobject{currentmarker}{}%
\end{pgfscope}%
\begin{pgfscope}%
\pgfsys@transformshift{2.312134in}{0.943734in}%
\pgfsys@useobject{currentmarker}{}%
\end{pgfscope}%
\begin{pgfscope}%
\pgfsys@transformshift{2.312134in}{0.943734in}%
\pgfsys@useobject{currentmarker}{}%
\end{pgfscope}%
\begin{pgfscope}%
\pgfsys@transformshift{2.312134in}{0.943734in}%
\pgfsys@useobject{currentmarker}{}%
\end{pgfscope}%
\begin{pgfscope}%
\pgfsys@transformshift{2.312134in}{0.943734in}%
\pgfsys@useobject{currentmarker}{}%
\end{pgfscope}%
\begin{pgfscope}%
\pgfsys@transformshift{2.312134in}{0.943734in}%
\pgfsys@useobject{currentmarker}{}%
\end{pgfscope}%
\begin{pgfscope}%
\pgfsys@transformshift{2.312134in}{0.943734in}%
\pgfsys@useobject{currentmarker}{}%
\end{pgfscope}%
\begin{pgfscope}%
\pgfsys@transformshift{2.312134in}{0.943734in}%
\pgfsys@useobject{currentmarker}{}%
\end{pgfscope}%
\begin{pgfscope}%
\pgfsys@transformshift{2.312134in}{0.943734in}%
\pgfsys@useobject{currentmarker}{}%
\end{pgfscope}%
\begin{pgfscope}%
\pgfsys@transformshift{2.312134in}{0.943734in}%
\pgfsys@useobject{currentmarker}{}%
\end{pgfscope}%
\begin{pgfscope}%
\pgfsys@transformshift{2.312134in}{0.943734in}%
\pgfsys@useobject{currentmarker}{}%
\end{pgfscope}%
\begin{pgfscope}%
\pgfsys@transformshift{2.312134in}{0.943734in}%
\pgfsys@useobject{currentmarker}{}%
\end{pgfscope}%
\begin{pgfscope}%
\pgfsys@transformshift{2.312134in}{0.943734in}%
\pgfsys@useobject{currentmarker}{}%
\end{pgfscope}%
\begin{pgfscope}%
\pgfsys@transformshift{2.312134in}{0.943734in}%
\pgfsys@useobject{currentmarker}{}%
\end{pgfscope}%
\begin{pgfscope}%
\pgfsys@transformshift{2.312134in}{0.943734in}%
\pgfsys@useobject{currentmarker}{}%
\end{pgfscope}%
\begin{pgfscope}%
\pgfsys@transformshift{2.312134in}{0.943734in}%
\pgfsys@useobject{currentmarker}{}%
\end{pgfscope}%
\begin{pgfscope}%
\pgfsys@transformshift{2.312134in}{0.943734in}%
\pgfsys@useobject{currentmarker}{}%
\end{pgfscope}%
\begin{pgfscope}%
\pgfsys@transformshift{2.312134in}{0.943734in}%
\pgfsys@useobject{currentmarker}{}%
\end{pgfscope}%
\begin{pgfscope}%
\pgfsys@transformshift{2.312134in}{0.943734in}%
\pgfsys@useobject{currentmarker}{}%
\end{pgfscope}%
\begin{pgfscope}%
\pgfsys@transformshift{2.312134in}{0.943734in}%
\pgfsys@useobject{currentmarker}{}%
\end{pgfscope}%
\begin{pgfscope}%
\pgfsys@transformshift{2.312134in}{0.943734in}%
\pgfsys@useobject{currentmarker}{}%
\end{pgfscope}%
\begin{pgfscope}%
\pgfsys@transformshift{2.312134in}{0.943734in}%
\pgfsys@useobject{currentmarker}{}%
\end{pgfscope}%
\begin{pgfscope}%
\pgfsys@transformshift{2.312134in}{0.943734in}%
\pgfsys@useobject{currentmarker}{}%
\end{pgfscope}%
\begin{pgfscope}%
\pgfsys@transformshift{2.312134in}{0.943734in}%
\pgfsys@useobject{currentmarker}{}%
\end{pgfscope}%
\begin{pgfscope}%
\pgfsys@transformshift{2.312134in}{0.943734in}%
\pgfsys@useobject{currentmarker}{}%
\end{pgfscope}%
\begin{pgfscope}%
\pgfsys@transformshift{2.312134in}{0.943734in}%
\pgfsys@useobject{currentmarker}{}%
\end{pgfscope}%
\begin{pgfscope}%
\pgfsys@transformshift{2.312134in}{0.943734in}%
\pgfsys@useobject{currentmarker}{}%
\end{pgfscope}%
\begin{pgfscope}%
\pgfsys@transformshift{2.312134in}{0.943734in}%
\pgfsys@useobject{currentmarker}{}%
\end{pgfscope}%
\begin{pgfscope}%
\pgfsys@transformshift{2.312134in}{0.943734in}%
\pgfsys@useobject{currentmarker}{}%
\end{pgfscope}%
\begin{pgfscope}%
\pgfsys@transformshift{2.312134in}{0.943734in}%
\pgfsys@useobject{currentmarker}{}%
\end{pgfscope}%
\begin{pgfscope}%
\pgfsys@transformshift{2.312134in}{0.943734in}%
\pgfsys@useobject{currentmarker}{}%
\end{pgfscope}%
\begin{pgfscope}%
\pgfsys@transformshift{2.312134in}{0.943734in}%
\pgfsys@useobject{currentmarker}{}%
\end{pgfscope}%
\begin{pgfscope}%
\pgfsys@transformshift{2.312134in}{0.943734in}%
\pgfsys@useobject{currentmarker}{}%
\end{pgfscope}%
\begin{pgfscope}%
\pgfsys@transformshift{2.312134in}{0.943734in}%
\pgfsys@useobject{currentmarker}{}%
\end{pgfscope}%
\begin{pgfscope}%
\pgfsys@transformshift{2.312134in}{0.943734in}%
\pgfsys@useobject{currentmarker}{}%
\end{pgfscope}%
\begin{pgfscope}%
\pgfsys@transformshift{2.312134in}{0.943734in}%
\pgfsys@useobject{currentmarker}{}%
\end{pgfscope}%
\begin{pgfscope}%
\pgfsys@transformshift{2.312134in}{0.943734in}%
\pgfsys@useobject{currentmarker}{}%
\end{pgfscope}%
\begin{pgfscope}%
\pgfsys@transformshift{2.312134in}{0.943734in}%
\pgfsys@useobject{currentmarker}{}%
\end{pgfscope}%
\begin{pgfscope}%
\pgfsys@transformshift{2.312134in}{0.943734in}%
\pgfsys@useobject{currentmarker}{}%
\end{pgfscope}%
\begin{pgfscope}%
\pgfsys@transformshift{2.312134in}{0.943734in}%
\pgfsys@useobject{currentmarker}{}%
\end{pgfscope}%
\begin{pgfscope}%
\pgfsys@transformshift{2.312134in}{0.943734in}%
\pgfsys@useobject{currentmarker}{}%
\end{pgfscope}%
\begin{pgfscope}%
\pgfsys@transformshift{2.312134in}{0.943734in}%
\pgfsys@useobject{currentmarker}{}%
\end{pgfscope}%
\begin{pgfscope}%
\pgfsys@transformshift{2.312134in}{0.943734in}%
\pgfsys@useobject{currentmarker}{}%
\end{pgfscope}%
\begin{pgfscope}%
\pgfsys@transformshift{2.312134in}{0.943734in}%
\pgfsys@useobject{currentmarker}{}%
\end{pgfscope}%
\begin{pgfscope}%
\pgfsys@transformshift{2.312134in}{0.943734in}%
\pgfsys@useobject{currentmarker}{}%
\end{pgfscope}%
\begin{pgfscope}%
\pgfsys@transformshift{2.312134in}{0.943734in}%
\pgfsys@useobject{currentmarker}{}%
\end{pgfscope}%
\begin{pgfscope}%
\pgfsys@transformshift{2.312134in}{0.943734in}%
\pgfsys@useobject{currentmarker}{}%
\end{pgfscope}%
\begin{pgfscope}%
\pgfsys@transformshift{2.312134in}{0.943734in}%
\pgfsys@useobject{currentmarker}{}%
\end{pgfscope}%
\begin{pgfscope}%
\pgfsys@transformshift{2.312134in}{0.943734in}%
\pgfsys@useobject{currentmarker}{}%
\end{pgfscope}%
\begin{pgfscope}%
\pgfsys@transformshift{2.312134in}{0.943734in}%
\pgfsys@useobject{currentmarker}{}%
\end{pgfscope}%
\begin{pgfscope}%
\pgfsys@transformshift{2.312134in}{0.943734in}%
\pgfsys@useobject{currentmarker}{}%
\end{pgfscope}%
\begin{pgfscope}%
\pgfsys@transformshift{2.312134in}{0.943734in}%
\pgfsys@useobject{currentmarker}{}%
\end{pgfscope}%
\begin{pgfscope}%
\pgfsys@transformshift{2.312134in}{0.943734in}%
\pgfsys@useobject{currentmarker}{}%
\end{pgfscope}%
\begin{pgfscope}%
\pgfsys@transformshift{2.312134in}{0.943734in}%
\pgfsys@useobject{currentmarker}{}%
\end{pgfscope}%
\begin{pgfscope}%
\pgfsys@transformshift{2.312134in}{0.943734in}%
\pgfsys@useobject{currentmarker}{}%
\end{pgfscope}%
\begin{pgfscope}%
\pgfsys@transformshift{2.312134in}{0.943734in}%
\pgfsys@useobject{currentmarker}{}%
\end{pgfscope}%
\begin{pgfscope}%
\pgfsys@transformshift{2.312134in}{0.943734in}%
\pgfsys@useobject{currentmarker}{}%
\end{pgfscope}%
\begin{pgfscope}%
\pgfsys@transformshift{2.312134in}{0.943734in}%
\pgfsys@useobject{currentmarker}{}%
\end{pgfscope}%
\begin{pgfscope}%
\pgfsys@transformshift{2.312134in}{0.943734in}%
\pgfsys@useobject{currentmarker}{}%
\end{pgfscope}%
\begin{pgfscope}%
\pgfsys@transformshift{2.312134in}{0.943734in}%
\pgfsys@useobject{currentmarker}{}%
\end{pgfscope}%
\begin{pgfscope}%
\pgfsys@transformshift{2.312134in}{0.943734in}%
\pgfsys@useobject{currentmarker}{}%
\end{pgfscope}%
\begin{pgfscope}%
\pgfsys@transformshift{2.312134in}{0.943734in}%
\pgfsys@useobject{currentmarker}{}%
\end{pgfscope}%
\begin{pgfscope}%
\pgfsys@transformshift{2.312134in}{0.943734in}%
\pgfsys@useobject{currentmarker}{}%
\end{pgfscope}%
\begin{pgfscope}%
\pgfsys@transformshift{2.312134in}{0.943734in}%
\pgfsys@useobject{currentmarker}{}%
\end{pgfscope}%
\begin{pgfscope}%
\pgfsys@transformshift{2.312134in}{0.943734in}%
\pgfsys@useobject{currentmarker}{}%
\end{pgfscope}%
\begin{pgfscope}%
\pgfsys@transformshift{2.312134in}{0.943734in}%
\pgfsys@useobject{currentmarker}{}%
\end{pgfscope}%
\begin{pgfscope}%
\pgfsys@transformshift{2.312134in}{0.943734in}%
\pgfsys@useobject{currentmarker}{}%
\end{pgfscope}%
\begin{pgfscope}%
\pgfsys@transformshift{2.312134in}{0.943734in}%
\pgfsys@useobject{currentmarker}{}%
\end{pgfscope}%
\begin{pgfscope}%
\pgfsys@transformshift{2.312134in}{0.943734in}%
\pgfsys@useobject{currentmarker}{}%
\end{pgfscope}%
\begin{pgfscope}%
\pgfsys@transformshift{2.312134in}{0.943734in}%
\pgfsys@useobject{currentmarker}{}%
\end{pgfscope}%
\begin{pgfscope}%
\pgfsys@transformshift{2.312134in}{0.943734in}%
\pgfsys@useobject{currentmarker}{}%
\end{pgfscope}%
\begin{pgfscope}%
\pgfsys@transformshift{2.312134in}{0.943734in}%
\pgfsys@useobject{currentmarker}{}%
\end{pgfscope}%
\begin{pgfscope}%
\pgfsys@transformshift{2.312134in}{0.943734in}%
\pgfsys@useobject{currentmarker}{}%
\end{pgfscope}%
\begin{pgfscope}%
\pgfsys@transformshift{2.312134in}{0.943734in}%
\pgfsys@useobject{currentmarker}{}%
\end{pgfscope}%
\begin{pgfscope}%
\pgfsys@transformshift{2.312134in}{0.943734in}%
\pgfsys@useobject{currentmarker}{}%
\end{pgfscope}%
\begin{pgfscope}%
\pgfsys@transformshift{2.312134in}{0.943734in}%
\pgfsys@useobject{currentmarker}{}%
\end{pgfscope}%
\begin{pgfscope}%
\pgfsys@transformshift{2.312134in}{0.943734in}%
\pgfsys@useobject{currentmarker}{}%
\end{pgfscope}%
\begin{pgfscope}%
\pgfsys@transformshift{2.312134in}{0.943734in}%
\pgfsys@useobject{currentmarker}{}%
\end{pgfscope}%
\begin{pgfscope}%
\pgfsys@transformshift{2.312134in}{0.943734in}%
\pgfsys@useobject{currentmarker}{}%
\end{pgfscope}%
\begin{pgfscope}%
\pgfsys@transformshift{2.312134in}{0.943734in}%
\pgfsys@useobject{currentmarker}{}%
\end{pgfscope}%
\begin{pgfscope}%
\pgfsys@transformshift{2.312134in}{0.943734in}%
\pgfsys@useobject{currentmarker}{}%
\end{pgfscope}%
\begin{pgfscope}%
\pgfsys@transformshift{2.312134in}{0.943734in}%
\pgfsys@useobject{currentmarker}{}%
\end{pgfscope}%
\begin{pgfscope}%
\pgfsys@transformshift{2.312134in}{0.943734in}%
\pgfsys@useobject{currentmarker}{}%
\end{pgfscope}%
\begin{pgfscope}%
\pgfsys@transformshift{2.312134in}{0.943734in}%
\pgfsys@useobject{currentmarker}{}%
\end{pgfscope}%
\begin{pgfscope}%
\pgfsys@transformshift{2.312134in}{0.943734in}%
\pgfsys@useobject{currentmarker}{}%
\end{pgfscope}%
\begin{pgfscope}%
\pgfsys@transformshift{2.312134in}{0.943734in}%
\pgfsys@useobject{currentmarker}{}%
\end{pgfscope}%
\begin{pgfscope}%
\pgfsys@transformshift{2.312134in}{0.943734in}%
\pgfsys@useobject{currentmarker}{}%
\end{pgfscope}%
\begin{pgfscope}%
\pgfsys@transformshift{2.312134in}{0.943734in}%
\pgfsys@useobject{currentmarker}{}%
\end{pgfscope}%
\begin{pgfscope}%
\pgfsys@transformshift{2.312134in}{0.943734in}%
\pgfsys@useobject{currentmarker}{}%
\end{pgfscope}%
\begin{pgfscope}%
\pgfsys@transformshift{2.312134in}{0.943734in}%
\pgfsys@useobject{currentmarker}{}%
\end{pgfscope}%
\begin{pgfscope}%
\pgfsys@transformshift{2.312134in}{0.943734in}%
\pgfsys@useobject{currentmarker}{}%
\end{pgfscope}%
\begin{pgfscope}%
\pgfsys@transformshift{2.312134in}{0.943734in}%
\pgfsys@useobject{currentmarker}{}%
\end{pgfscope}%
\begin{pgfscope}%
\pgfsys@transformshift{2.312134in}{0.943734in}%
\pgfsys@useobject{currentmarker}{}%
\end{pgfscope}%
\begin{pgfscope}%
\pgfsys@transformshift{2.312134in}{0.943734in}%
\pgfsys@useobject{currentmarker}{}%
\end{pgfscope}%
\begin{pgfscope}%
\pgfsys@transformshift{2.312134in}{0.943734in}%
\pgfsys@useobject{currentmarker}{}%
\end{pgfscope}%
\begin{pgfscope}%
\pgfsys@transformshift{2.312134in}{0.943734in}%
\pgfsys@useobject{currentmarker}{}%
\end{pgfscope}%
\begin{pgfscope}%
\pgfsys@transformshift{2.312134in}{0.943734in}%
\pgfsys@useobject{currentmarker}{}%
\end{pgfscope}%
\begin{pgfscope}%
\pgfsys@transformshift{2.312134in}{0.943734in}%
\pgfsys@useobject{currentmarker}{}%
\end{pgfscope}%
\begin{pgfscope}%
\pgfsys@transformshift{2.312134in}{0.943734in}%
\pgfsys@useobject{currentmarker}{}%
\end{pgfscope}%
\begin{pgfscope}%
\pgfsys@transformshift{2.312134in}{0.943734in}%
\pgfsys@useobject{currentmarker}{}%
\end{pgfscope}%
\begin{pgfscope}%
\pgfsys@transformshift{2.312134in}{0.943734in}%
\pgfsys@useobject{currentmarker}{}%
\end{pgfscope}%
\begin{pgfscope}%
\pgfsys@transformshift{2.312134in}{0.943734in}%
\pgfsys@useobject{currentmarker}{}%
\end{pgfscope}%
\begin{pgfscope}%
\pgfsys@transformshift{2.312134in}{0.943734in}%
\pgfsys@useobject{currentmarker}{}%
\end{pgfscope}%
\begin{pgfscope}%
\pgfsys@transformshift{2.312134in}{0.943734in}%
\pgfsys@useobject{currentmarker}{}%
\end{pgfscope}%
\begin{pgfscope}%
\pgfsys@transformshift{2.312134in}{0.943734in}%
\pgfsys@useobject{currentmarker}{}%
\end{pgfscope}%
\begin{pgfscope}%
\pgfsys@transformshift{2.312134in}{0.943734in}%
\pgfsys@useobject{currentmarker}{}%
\end{pgfscope}%
\begin{pgfscope}%
\pgfsys@transformshift{2.312134in}{0.943734in}%
\pgfsys@useobject{currentmarker}{}%
\end{pgfscope}%
\begin{pgfscope}%
\pgfsys@transformshift{2.312134in}{0.943734in}%
\pgfsys@useobject{currentmarker}{}%
\end{pgfscope}%
\begin{pgfscope}%
\pgfsys@transformshift{2.312134in}{0.943734in}%
\pgfsys@useobject{currentmarker}{}%
\end{pgfscope}%
\begin{pgfscope}%
\pgfsys@transformshift{2.312134in}{0.943734in}%
\pgfsys@useobject{currentmarker}{}%
\end{pgfscope}%
\begin{pgfscope}%
\pgfsys@transformshift{2.312134in}{0.943734in}%
\pgfsys@useobject{currentmarker}{}%
\end{pgfscope}%
\begin{pgfscope}%
\pgfsys@transformshift{2.312134in}{0.943734in}%
\pgfsys@useobject{currentmarker}{}%
\end{pgfscope}%
\begin{pgfscope}%
\pgfsys@transformshift{2.312134in}{0.943734in}%
\pgfsys@useobject{currentmarker}{}%
\end{pgfscope}%
\begin{pgfscope}%
\pgfsys@transformshift{2.312134in}{0.943734in}%
\pgfsys@useobject{currentmarker}{}%
\end{pgfscope}%
\begin{pgfscope}%
\pgfsys@transformshift{2.312134in}{0.943734in}%
\pgfsys@useobject{currentmarker}{}%
\end{pgfscope}%
\begin{pgfscope}%
\pgfsys@transformshift{2.312134in}{0.943734in}%
\pgfsys@useobject{currentmarker}{}%
\end{pgfscope}%
\begin{pgfscope}%
\pgfsys@transformshift{2.312134in}{0.943734in}%
\pgfsys@useobject{currentmarker}{}%
\end{pgfscope}%
\begin{pgfscope}%
\pgfsys@transformshift{2.312134in}{0.943734in}%
\pgfsys@useobject{currentmarker}{}%
\end{pgfscope}%
\begin{pgfscope}%
\pgfsys@transformshift{2.312134in}{0.943734in}%
\pgfsys@useobject{currentmarker}{}%
\end{pgfscope}%
\begin{pgfscope}%
\pgfsys@transformshift{2.312134in}{0.943734in}%
\pgfsys@useobject{currentmarker}{}%
\end{pgfscope}%
\begin{pgfscope}%
\pgfsys@transformshift{2.312134in}{0.943734in}%
\pgfsys@useobject{currentmarker}{}%
\end{pgfscope}%
\begin{pgfscope}%
\pgfsys@transformshift{2.312134in}{0.943734in}%
\pgfsys@useobject{currentmarker}{}%
\end{pgfscope}%
\begin{pgfscope}%
\pgfsys@transformshift{2.312134in}{0.943734in}%
\pgfsys@useobject{currentmarker}{}%
\end{pgfscope}%
\begin{pgfscope}%
\pgfsys@transformshift{2.312134in}{0.943734in}%
\pgfsys@useobject{currentmarker}{}%
\end{pgfscope}%
\begin{pgfscope}%
\pgfsys@transformshift{2.312134in}{0.943734in}%
\pgfsys@useobject{currentmarker}{}%
\end{pgfscope}%
\begin{pgfscope}%
\pgfsys@transformshift{2.312134in}{0.943734in}%
\pgfsys@useobject{currentmarker}{}%
\end{pgfscope}%
\begin{pgfscope}%
\pgfsys@transformshift{2.312134in}{0.943734in}%
\pgfsys@useobject{currentmarker}{}%
\end{pgfscope}%
\begin{pgfscope}%
\pgfsys@transformshift{2.312134in}{0.943734in}%
\pgfsys@useobject{currentmarker}{}%
\end{pgfscope}%
\begin{pgfscope}%
\pgfsys@transformshift{2.312134in}{0.943734in}%
\pgfsys@useobject{currentmarker}{}%
\end{pgfscope}%
\begin{pgfscope}%
\pgfsys@transformshift{2.312134in}{0.943734in}%
\pgfsys@useobject{currentmarker}{}%
\end{pgfscope}%
\begin{pgfscope}%
\pgfsys@transformshift{2.312134in}{0.943734in}%
\pgfsys@useobject{currentmarker}{}%
\end{pgfscope}%
\begin{pgfscope}%
\pgfsys@transformshift{2.312134in}{0.943734in}%
\pgfsys@useobject{currentmarker}{}%
\end{pgfscope}%
\begin{pgfscope}%
\pgfsys@transformshift{2.312134in}{0.943734in}%
\pgfsys@useobject{currentmarker}{}%
\end{pgfscope}%
\begin{pgfscope}%
\pgfsys@transformshift{2.312134in}{0.943734in}%
\pgfsys@useobject{currentmarker}{}%
\end{pgfscope}%
\begin{pgfscope}%
\pgfsys@transformshift{2.312134in}{0.943734in}%
\pgfsys@useobject{currentmarker}{}%
\end{pgfscope}%
\begin{pgfscope}%
\pgfsys@transformshift{2.312134in}{0.943734in}%
\pgfsys@useobject{currentmarker}{}%
\end{pgfscope}%
\begin{pgfscope}%
\pgfsys@transformshift{2.312134in}{0.943734in}%
\pgfsys@useobject{currentmarker}{}%
\end{pgfscope}%
\begin{pgfscope}%
\pgfsys@transformshift{2.312134in}{0.943734in}%
\pgfsys@useobject{currentmarker}{}%
\end{pgfscope}%
\begin{pgfscope}%
\pgfsys@transformshift{2.312134in}{0.943734in}%
\pgfsys@useobject{currentmarker}{}%
\end{pgfscope}%
\begin{pgfscope}%
\pgfsys@transformshift{2.312134in}{0.943734in}%
\pgfsys@useobject{currentmarker}{}%
\end{pgfscope}%
\begin{pgfscope}%
\pgfsys@transformshift{2.312134in}{0.943734in}%
\pgfsys@useobject{currentmarker}{}%
\end{pgfscope}%
\begin{pgfscope}%
\pgfsys@transformshift{2.312134in}{0.943734in}%
\pgfsys@useobject{currentmarker}{}%
\end{pgfscope}%
\begin{pgfscope}%
\pgfsys@transformshift{2.312134in}{0.943734in}%
\pgfsys@useobject{currentmarker}{}%
\end{pgfscope}%
\begin{pgfscope}%
\pgfsys@transformshift{2.312134in}{0.943734in}%
\pgfsys@useobject{currentmarker}{}%
\end{pgfscope}%
\begin{pgfscope}%
\pgfsys@transformshift{2.312134in}{0.943734in}%
\pgfsys@useobject{currentmarker}{}%
\end{pgfscope}%
\begin{pgfscope}%
\pgfsys@transformshift{2.312134in}{0.943734in}%
\pgfsys@useobject{currentmarker}{}%
\end{pgfscope}%
\begin{pgfscope}%
\pgfsys@transformshift{2.312134in}{0.943734in}%
\pgfsys@useobject{currentmarker}{}%
\end{pgfscope}%
\begin{pgfscope}%
\pgfsys@transformshift{2.312134in}{0.943734in}%
\pgfsys@useobject{currentmarker}{}%
\end{pgfscope}%
\begin{pgfscope}%
\pgfsys@transformshift{2.312134in}{0.943734in}%
\pgfsys@useobject{currentmarker}{}%
\end{pgfscope}%
\begin{pgfscope}%
\pgfsys@transformshift{2.312134in}{0.943734in}%
\pgfsys@useobject{currentmarker}{}%
\end{pgfscope}%
\begin{pgfscope}%
\pgfsys@transformshift{2.312134in}{0.943734in}%
\pgfsys@useobject{currentmarker}{}%
\end{pgfscope}%
\begin{pgfscope}%
\pgfsys@transformshift{2.312134in}{0.943734in}%
\pgfsys@useobject{currentmarker}{}%
\end{pgfscope}%
\begin{pgfscope}%
\pgfsys@transformshift{2.312134in}{0.943734in}%
\pgfsys@useobject{currentmarker}{}%
\end{pgfscope}%
\begin{pgfscope}%
\pgfsys@transformshift{2.312134in}{0.943734in}%
\pgfsys@useobject{currentmarker}{}%
\end{pgfscope}%
\begin{pgfscope}%
\pgfsys@transformshift{2.312134in}{0.943734in}%
\pgfsys@useobject{currentmarker}{}%
\end{pgfscope}%
\begin{pgfscope}%
\pgfsys@transformshift{2.312134in}{0.943734in}%
\pgfsys@useobject{currentmarker}{}%
\end{pgfscope}%
\begin{pgfscope}%
\pgfsys@transformshift{2.312134in}{0.943734in}%
\pgfsys@useobject{currentmarker}{}%
\end{pgfscope}%
\begin{pgfscope}%
\pgfsys@transformshift{2.312134in}{0.943734in}%
\pgfsys@useobject{currentmarker}{}%
\end{pgfscope}%
\begin{pgfscope}%
\pgfsys@transformshift{2.312134in}{0.943734in}%
\pgfsys@useobject{currentmarker}{}%
\end{pgfscope}%
\begin{pgfscope}%
\pgfsys@transformshift{2.312134in}{0.943734in}%
\pgfsys@useobject{currentmarker}{}%
\end{pgfscope}%
\end{pgfscope}%
\begin{pgfscope}%
\pgfpathrectangle{\pgfqpoint{0.562500in}{0.275000in}}{\pgfqpoint{3.487500in}{1.925000in}}%
\pgfusepath{clip}%
\pgfsetrectcap%
\pgfsetroundjoin%
\pgfsetlinewidth{1.505625pt}%
\definecolor{currentstroke}{rgb}{1.000000,0.498039,0.054902}%
\pgfsetstrokecolor{currentstroke}%
\pgfsetdash{}{0pt}%
\pgfpathmoveto{\pgfqpoint{2.312134in}{0.943734in}}%
\pgfpathlineto{\pgfqpoint{2.312134in}{0.943734in}}%
\pgfusepath{stroke}%
\end{pgfscope}%
\begin{pgfscope}%
\pgfpathrectangle{\pgfqpoint{0.562500in}{0.275000in}}{\pgfqpoint{3.487500in}{1.925000in}}%
\pgfusepath{clip}%
\pgfsetrectcap%
\pgfsetroundjoin%
\pgfsetlinewidth{1.505625pt}%
\definecolor{currentstroke}{rgb}{0.172549,0.627451,0.172549}%
\pgfsetstrokecolor{currentstroke}%
\pgfsetdash{}{0pt}%
\pgfpathmoveto{\pgfqpoint{3.891477in}{0.362500in}}%
\pgfpathlineto{\pgfqpoint{3.852525in}{0.462371in}}%
\pgfpathlineto{\pgfqpoint{3.761952in}{0.533896in}}%
\pgfpathlineto{\pgfqpoint{3.644763in}{0.583099in}}%
\pgfpathlineto{\pgfqpoint{3.516977in}{0.616087in}}%
\pgfpathlineto{\pgfqpoint{3.387283in}{0.639333in}}%
\pgfpathlineto{\pgfqpoint{3.259617in}{0.656646in}}%
\pgfpathlineto{\pgfqpoint{3.136438in}{0.669953in}}%
\pgfpathlineto{\pgfqpoint{2.808048in}{0.702394in}}%
\pgfpathlineto{\pgfqpoint{2.714816in}{0.713564in}}%
\pgfpathlineto{\pgfqpoint{2.630104in}{0.725426in}}%
\pgfpathlineto{\pgfqpoint{2.553950in}{0.738038in}}%
\pgfpathlineto{\pgfqpoint{2.486290in}{0.751356in}}%
\pgfpathlineto{\pgfqpoint{2.426956in}{0.765236in}}%
\pgfpathlineto{\pgfqpoint{2.375675in}{0.779443in}}%
\pgfpathlineto{\pgfqpoint{2.332026in}{0.793769in}}%
\pgfpathlineto{\pgfqpoint{2.295450in}{0.808028in}}%
\pgfpathlineto{\pgfqpoint{2.265386in}{0.822039in}}%
\pgfpathlineto{\pgfqpoint{2.241288in}{0.835639in}}%
\pgfpathlineto{\pgfqpoint{2.222620in}{0.848679in}}%
\pgfpathlineto{\pgfqpoint{2.208821in}{0.861035in}}%
\pgfpathlineto{\pgfqpoint{2.199259in}{0.872627in}}%
\pgfpathlineto{\pgfqpoint{2.193352in}{0.883399in}}%
\pgfpathlineto{\pgfqpoint{2.190568in}{0.893311in}}%
\pgfpathlineto{\pgfqpoint{2.190425in}{0.902339in}}%
\pgfpathlineto{\pgfqpoint{2.192489in}{0.910473in}}%
\pgfpathlineto{\pgfqpoint{2.196326in}{0.917728in}}%
\pgfpathlineto{\pgfqpoint{2.201541in}{0.924141in}}%
\pgfpathlineto{\pgfqpoint{2.214764in}{0.934617in}}%
\pgfpathlineto{\pgfqpoint{2.229969in}{0.942249in}}%
\pgfpathlineto{\pgfqpoint{2.245555in}{0.947450in}}%
\pgfpathlineto{\pgfqpoint{2.267239in}{0.951729in}}%
\pgfpathlineto{\pgfqpoint{2.285122in}{0.953006in}}%
\pgfpathlineto{\pgfqpoint{2.302044in}{0.952052in}}%
\pgfpathlineto{\pgfqpoint{2.313350in}{0.949240in}}%
\pgfpathlineto{\pgfqpoint{2.317133in}{0.946533in}}%
\pgfpathlineto{\pgfqpoint{2.317006in}{0.944687in}}%
\pgfpathlineto{\pgfqpoint{2.314493in}{0.943453in}}%
\pgfpathlineto{\pgfqpoint{2.311907in}{0.943643in}}%
\pgfpathlineto{\pgfqpoint{2.312135in}{0.943734in}}%
\pgfusepath{stroke}%
\end{pgfscope}%
\begin{pgfscope}%
\pgfpathrectangle{\pgfqpoint{0.562500in}{0.275000in}}{\pgfqpoint{3.487500in}{1.925000in}}%
\pgfusepath{clip}%
\pgfsetrectcap%
\pgfsetroundjoin%
\pgfsetlinewidth{1.505625pt}%
\definecolor{currentstroke}{rgb}{0.839216,0.152941,0.156863}%
\pgfsetstrokecolor{currentstroke}%
\pgfsetdash{}{0pt}%
\pgfpathmoveto{\pgfqpoint{3.891477in}{2.106201in}}%
\pgfpathlineto{\pgfqpoint{2.663098in}{2.112500in}}%
\pgfpathlineto{\pgfqpoint{1.850286in}{1.993740in}}%
\pgfpathlineto{\pgfqpoint{1.380991in}{1.842951in}}%
\pgfpathlineto{\pgfqpoint{1.151678in}{1.702442in}}%
\pgfpathlineto{\pgfqpoint{1.075929in}{1.586669in}}%
\pgfpathlineto{\pgfqpoint{1.090800in}{1.496720in}}%
\pgfpathlineto{\pgfqpoint{1.158731in}{1.428976in}}%
\pgfpathlineto{\pgfqpoint{1.255509in}{1.378783in}}%
\pgfpathlineto{\pgfqpoint{1.365520in}{1.340665in}}%
\pgfpathlineto{\pgfqpoint{1.480280in}{1.310630in}}%
\pgfpathlineto{\pgfqpoint{1.595185in}{1.286174in}}%
\pgfpathlineto{\pgfqpoint{2.000430in}{1.209557in}}%
\pgfpathlineto{\pgfqpoint{2.081617in}{1.191568in}}%
\pgfpathlineto{\pgfqpoint{2.154012in}{1.173503in}}%
\pgfpathlineto{\pgfqpoint{2.217599in}{1.155346in}}%
\pgfpathlineto{\pgfqpoint{2.272546in}{1.137189in}}%
\pgfpathlineto{\pgfqpoint{2.319199in}{1.119226in}}%
\pgfpathlineto{\pgfqpoint{2.358011in}{1.101632in}}%
\pgfpathlineto{\pgfqpoint{2.389605in}{1.084553in}}%
\pgfpathlineto{\pgfqpoint{2.414624in}{1.068132in}}%
\pgfpathlineto{\pgfqpoint{2.433688in}{1.052499in}}%
\pgfpathlineto{\pgfqpoint{2.447396in}{1.037771in}}%
\pgfpathlineto{\pgfqpoint{2.456342in}{1.024049in}}%
\pgfpathlineto{\pgfqpoint{2.461205in}{1.011383in}}%
\pgfpathlineto{\pgfqpoint{2.462628in}{0.999790in}}%
\pgfpathlineto{\pgfqpoint{2.461186in}{0.989275in}}%
\pgfpathlineto{\pgfqpoint{2.457395in}{0.979832in}}%
\pgfpathlineto{\pgfqpoint{2.451707in}{0.971445in}}%
\pgfpathlineto{\pgfqpoint{2.444537in}{0.964084in}}%
\pgfpathlineto{\pgfqpoint{2.436293in}{0.957689in}}%
\pgfpathlineto{\pgfqpoint{2.417896in}{0.947532in}}%
\pgfpathlineto{\pgfqpoint{2.398673in}{0.940456in}}%
\pgfpathlineto{\pgfqpoint{2.380126in}{0.935938in}}%
\pgfpathlineto{\pgfqpoint{2.355602in}{0.932739in}}%
\pgfpathlineto{\pgfqpoint{2.336355in}{0.932453in}}%
\pgfpathlineto{\pgfqpoint{2.319069in}{0.934443in}}%
\pgfpathlineto{\pgfqpoint{2.308478in}{0.938049in}}%
\pgfpathlineto{\pgfqpoint{2.305696in}{0.941102in}}%
\pgfpathlineto{\pgfqpoint{2.306629in}{0.943023in}}%
\pgfpathlineto{\pgfqpoint{2.310130in}{0.944200in}}%
\pgfpathlineto{\pgfqpoint{2.312414in}{0.943835in}}%
\pgfpathlineto{\pgfqpoint{2.312134in}{0.943734in}}%
\pgfusepath{stroke}%
\end{pgfscope}%
\begin{pgfscope}%
\pgfpathrectangle{\pgfqpoint{0.562500in}{0.275000in}}{\pgfqpoint{3.487500in}{1.925000in}}%
\pgfusepath{clip}%
\pgfsetrectcap%
\pgfsetroundjoin%
\pgfsetlinewidth{1.505625pt}%
\definecolor{currentstroke}{rgb}{0.580392,0.403922,0.741176}%
\pgfsetstrokecolor{currentstroke}%
\pgfsetdash{}{0pt}%
\pgfpathmoveto{\pgfqpoint{1.917299in}{2.106201in}}%
\pgfpathlineto{\pgfqpoint{1.209532in}{1.922923in}}%
\pgfpathlineto{\pgfqpoint{0.854164in}{1.738293in}}%
\pgfpathlineto{\pgfqpoint{0.721672in}{1.583888in}}%
\pgfpathlineto{\pgfqpoint{0.721023in}{1.466126in}}%
\pgfpathlineto{\pgfqpoint{0.791563in}{1.380657in}}%
\pgfpathlineto{\pgfqpoint{0.899796in}{1.321178in}}%
\pgfpathlineto{\pgfqpoint{1.024616in}{1.280808in}}%
\pgfpathlineto{\pgfqpoint{1.154807in}{1.252698in}}%
\pgfpathlineto{\pgfqpoint{1.284797in}{1.233036in}}%
\pgfpathlineto{\pgfqpoint{1.411661in}{1.218766in}}%
\pgfpathlineto{\pgfqpoint{1.851782in}{1.177168in}}%
\pgfpathlineto{\pgfqpoint{1.941478in}{1.166226in}}%
\pgfpathlineto{\pgfqpoint{2.022620in}{1.154465in}}%
\pgfpathlineto{\pgfqpoint{2.095203in}{1.141886in}}%
\pgfpathlineto{\pgfqpoint{2.159358in}{1.128644in}}%
\pgfpathlineto{\pgfqpoint{2.215310in}{1.114929in}}%
\pgfpathlineto{\pgfqpoint{2.263488in}{1.100930in}}%
\pgfpathlineto{\pgfqpoint{2.304373in}{1.086849in}}%
\pgfpathlineto{\pgfqpoint{2.338451in}{1.072875in}}%
\pgfpathlineto{\pgfqpoint{2.366209in}{1.059195in}}%
\pgfpathlineto{\pgfqpoint{2.388155in}{1.045979in}}%
\pgfpathlineto{\pgfqpoint{2.404925in}{1.033357in}}%
\pgfpathlineto{\pgfqpoint{2.417151in}{1.021425in}}%
\pgfpathlineto{\pgfqpoint{2.425411in}{1.010258in}}%
\pgfpathlineto{\pgfqpoint{2.430236in}{0.999913in}}%
\pgfpathlineto{\pgfqpoint{2.432106in}{0.990432in}}%
\pgfpathlineto{\pgfqpoint{2.431485in}{0.981835in}}%
\pgfpathlineto{\pgfqpoint{2.428846in}{0.974111in}}%
\pgfpathlineto{\pgfqpoint{2.424608in}{0.967229in}}%
\pgfpathlineto{\pgfqpoint{2.419135in}{0.961161in}}%
\pgfpathlineto{\pgfqpoint{2.405692in}{0.951317in}}%
\pgfpathlineto{\pgfqpoint{2.390544in}{0.944233in}}%
\pgfpathlineto{\pgfqpoint{2.375260in}{0.939444in}}%
\pgfpathlineto{\pgfqpoint{2.354250in}{0.935615in}}%
\pgfpathlineto{\pgfqpoint{2.337098in}{0.934595in}}%
\pgfpathlineto{\pgfqpoint{2.321064in}{0.935697in}}%
\pgfpathlineto{\pgfqpoint{2.310525in}{0.938502in}}%
\pgfpathlineto{\pgfqpoint{2.306998in}{0.941550in}}%
\pgfpathlineto{\pgfqpoint{2.307971in}{0.943331in}}%
\pgfpathlineto{\pgfqpoint{2.311545in}{0.944107in}}%
\pgfpathlineto{\pgfqpoint{2.312358in}{0.943818in}}%
\pgfpathlineto{\pgfqpoint{2.312134in}{0.943734in}}%
\pgfusepath{stroke}%
\end{pgfscope}%
\begin{pgfscope}%
\pgfpathrectangle{\pgfqpoint{0.562500in}{0.275000in}}{\pgfqpoint{3.487500in}{1.925000in}}%
\pgfusepath{clip}%
\pgfsetrectcap%
\pgfsetroundjoin%
\pgfsetlinewidth{1.505625pt}%
\definecolor{currentstroke}{rgb}{0.549020,0.337255,0.294118}%
\pgfsetstrokecolor{currentstroke}%
\pgfsetdash{}{0pt}%
\pgfpathmoveto{\pgfqpoint{1.127627in}{1.524967in}}%
\pgfpathlineto{\pgfqpoint{1.181776in}{1.452700in}}%
\pgfpathlineto{\pgfqpoint{1.268559in}{1.397830in}}%
\pgfpathlineto{\pgfqpoint{1.373825in}{1.356265in}}%
\pgfpathlineto{\pgfqpoint{1.487694in}{1.324410in}}%
\pgfpathlineto{\pgfqpoint{1.602599in}{1.298848in}}%
\pgfpathlineto{\pgfqpoint{2.005993in}{1.216938in}}%
\pgfpathlineto{\pgfqpoint{2.087403in}{1.197993in}}%
\pgfpathlineto{\pgfqpoint{2.159973in}{1.179057in}}%
\pgfpathlineto{\pgfqpoint{2.223655in}{1.160155in}}%
\pgfpathlineto{\pgfqpoint{2.278701in}{1.141371in}}%
\pgfpathlineto{\pgfqpoint{2.325501in}{1.122830in}}%
\pgfpathlineto{\pgfqpoint{2.364490in}{1.104677in}}%
\pgfpathlineto{\pgfqpoint{2.396154in}{1.087074in}}%
\pgfpathlineto{\pgfqpoint{2.421027in}{1.070200in}}%
\pgfpathlineto{\pgfqpoint{2.439775in}{1.054192in}}%
\pgfpathlineto{\pgfqpoint{2.453143in}{1.039134in}}%
\pgfpathlineto{\pgfqpoint{2.461819in}{1.025094in}}%
\pgfpathlineto{\pgfqpoint{2.466436in}{1.012127in}}%
\pgfpathlineto{\pgfqpoint{2.467571in}{1.000272in}}%
\pgfpathlineto{\pgfqpoint{2.465749in}{0.989553in}}%
\pgfpathlineto{\pgfqpoint{2.461524in}{0.979961in}}%
\pgfpathlineto{\pgfqpoint{2.455414in}{0.971448in}}%
\pgfpathlineto{\pgfqpoint{2.447868in}{0.963962in}}%
\pgfpathlineto{\pgfqpoint{2.439276in}{0.957452in}}%
\pgfpathlineto{\pgfqpoint{2.420219in}{0.947144in}}%
\pgfpathlineto{\pgfqpoint{2.400362in}{0.940025in}}%
\pgfpathlineto{\pgfqpoint{2.381280in}{0.935483in}}%
\pgfpathlineto{\pgfqpoint{2.356159in}{0.932328in}}%
\pgfpathlineto{\pgfqpoint{2.336493in}{0.932116in}}%
\pgfpathlineto{\pgfqpoint{2.318937in}{0.934225in}}%
\pgfpathlineto{\pgfqpoint{2.309571in}{0.937211in}}%
\pgfpathlineto{\pgfqpoint{2.305611in}{0.940539in}}%
\pgfpathlineto{\pgfqpoint{2.306139in}{0.942733in}}%
\pgfpathlineto{\pgfqpoint{2.309383in}{0.944133in}}%
\pgfpathlineto{\pgfqpoint{2.312422in}{0.943857in}}%
\pgfpathlineto{\pgfqpoint{2.312133in}{0.943734in}}%
\pgfusepath{stroke}%
\end{pgfscope}%
\begin{pgfscope}%
\pgfpathrectangle{\pgfqpoint{0.562500in}{0.275000in}}{\pgfqpoint{3.487500in}{1.925000in}}%
\pgfusepath{clip}%
\pgfsetrectcap%
\pgfsetroundjoin%
\pgfsetlinewidth{1.505625pt}%
\definecolor{currentstroke}{rgb}{0.890196,0.466667,0.760784}%
\pgfsetstrokecolor{currentstroke}%
\pgfsetdash{}{0pt}%
\pgfpathmoveto{\pgfqpoint{0.732791in}{0.362500in}}%
\pgfpathlineto{\pgfqpoint{1.004162in}{0.376835in}}%
\pgfpathlineto{\pgfqpoint{1.198512in}{0.410168in}}%
\pgfpathlineto{\pgfqpoint{1.332209in}{0.453296in}}%
\pgfpathlineto{\pgfqpoint{1.421844in}{0.500325in}}%
\pgfpathlineto{\pgfqpoint{1.485009in}{0.548055in}}%
\pgfpathlineto{\pgfqpoint{1.532052in}{0.595039in}}%
\pgfpathlineto{\pgfqpoint{1.568210in}{0.640372in}}%
\pgfpathlineto{\pgfqpoint{1.627436in}{0.723830in}}%
\pgfpathlineto{\pgfqpoint{1.656360in}{0.761226in}}%
\pgfpathlineto{\pgfqpoint{1.686710in}{0.795538in}}%
\pgfpathlineto{\pgfqpoint{1.718944in}{0.826714in}}%
\pgfpathlineto{\pgfqpoint{1.753214in}{0.854740in}}%
\pgfpathlineto{\pgfqpoint{1.789401in}{0.879640in}}%
\pgfpathlineto{\pgfqpoint{1.827117in}{0.901477in}}%
\pgfpathlineto{\pgfqpoint{1.865719in}{0.920349in}}%
\pgfpathlineto{\pgfqpoint{1.904645in}{0.936413in}}%
\pgfpathlineto{\pgfqpoint{1.943391in}{0.949874in}}%
\pgfpathlineto{\pgfqpoint{1.981463in}{0.960938in}}%
\pgfpathlineto{\pgfqpoint{2.018416in}{0.969807in}}%
\pgfpathlineto{\pgfqpoint{2.053849in}{0.976677in}}%
\pgfpathlineto{\pgfqpoint{2.087423in}{0.981755in}}%
\pgfpathlineto{\pgfqpoint{2.148190in}{0.987448in}}%
\pgfpathlineto{\pgfqpoint{2.199655in}{0.988525in}}%
\pgfpathlineto{\pgfqpoint{2.241470in}{0.986354in}}%
\pgfpathlineto{\pgfqpoint{2.274155in}{0.982136in}}%
\pgfpathlineto{\pgfqpoint{2.298637in}{0.976822in}}%
\pgfpathlineto{\pgfqpoint{2.315907in}{0.971091in}}%
\pgfpathlineto{\pgfqpoint{2.327235in}{0.965456in}}%
\pgfpathlineto{\pgfqpoint{2.333860in}{0.960256in}}%
\pgfpathlineto{\pgfqpoint{2.336862in}{0.955690in}}%
\pgfpathlineto{\pgfqpoint{2.337258in}{0.951844in}}%
\pgfpathlineto{\pgfqpoint{2.334738in}{0.947447in}}%
\pgfpathlineto{\pgfqpoint{2.328719in}{0.943852in}}%
\pgfpathlineto{\pgfqpoint{2.319739in}{0.941894in}}%
\pgfpathlineto{\pgfqpoint{2.312096in}{0.942502in}}%
\pgfpathlineto{\pgfqpoint{2.311058in}{0.943428in}}%
\pgfpathlineto{\pgfqpoint{2.311784in}{0.943812in}}%
\pgfpathlineto{\pgfqpoint{2.312182in}{0.943752in}}%
\pgfpathlineto{\pgfqpoint{2.312134in}{0.943734in}}%
\pgfusepath{stroke}%
\end{pgfscope}%
\begin{pgfscope}%
\pgfsetrectcap%
\pgfsetmiterjoin%
\pgfsetlinewidth{0.803000pt}%
\definecolor{currentstroke}{rgb}{0.000000,0.000000,0.000000}%
\pgfsetstrokecolor{currentstroke}%
\pgfsetdash{}{0pt}%
\pgfpathmoveto{\pgfqpoint{0.562500in}{0.275000in}}%
\pgfpathlineto{\pgfqpoint{0.562500in}{2.200000in}}%
\pgfusepath{stroke}%
\end{pgfscope}%
\begin{pgfscope}%
\pgfsetrectcap%
\pgfsetmiterjoin%
\pgfsetlinewidth{0.803000pt}%
\definecolor{currentstroke}{rgb}{0.000000,0.000000,0.000000}%
\pgfsetstrokecolor{currentstroke}%
\pgfsetdash{}{0pt}%
\pgfpathmoveto{\pgfqpoint{4.050000in}{0.275000in}}%
\pgfpathlineto{\pgfqpoint{4.050000in}{2.200000in}}%
\pgfusepath{stroke}%
\end{pgfscope}%
\begin{pgfscope}%
\pgfsetrectcap%
\pgfsetmiterjoin%
\pgfsetlinewidth{0.803000pt}%
\definecolor{currentstroke}{rgb}{0.000000,0.000000,0.000000}%
\pgfsetstrokecolor{currentstroke}%
\pgfsetdash{}{0pt}%
\pgfpathmoveto{\pgfqpoint{0.562500in}{0.275000in}}%
\pgfpathlineto{\pgfqpoint{4.050000in}{0.275000in}}%
\pgfusepath{stroke}%
\end{pgfscope}%
\begin{pgfscope}%
\pgfsetrectcap%
\pgfsetmiterjoin%
\pgfsetlinewidth{0.803000pt}%
\definecolor{currentstroke}{rgb}{0.000000,0.000000,0.000000}%
\pgfsetstrokecolor{currentstroke}%
\pgfsetdash{}{0pt}%
\pgfpathmoveto{\pgfqpoint{0.562500in}{2.200000in}}%
\pgfpathlineto{\pgfqpoint{4.050000in}{2.200000in}}%
\pgfusepath{stroke}%
\end{pgfscope}%
\end{pgfpicture}%
\makeatother%
\endgroup%

    \caption{Lösungen des Differentialgleichungssystems %\ref{TODO}
    Diverse Anfangspunkte haben die selbe Omega-Limesmenge auf der Nullstelle $(0,0)$.}
    \label{poinbendix:fig:fixed_point_omega_set}
\end{figure}

\subsection{Fall 2: $\omega(p)$ ist ein Geschlossener Orbit} \label{poinbendix:subsection:fall2}

\begin{figure}
\centering
    %% Creator: Matplotlib, PGF backend
%%
%% To include the figure in your LaTeX document, write
%%   \input{<filename>.pgf}
%%
%% Make sure the required packages are loaded in your preamble
%%   \usepackage{pgf}
%%
%% Also ensure that all the required font packages are loaded; for instance,
%% the lmodern package is sometimes necessary when using math font.
%%   \usepackage{lmodern}
%%
%% Figures using additional raster images can only be included by \input if
%% they are in the same directory as the main LaTeX file. For loading figures
%% from other directories you can use the `import` package
%%   \usepackage{import}
%%
%% and then include the figures with
%%   \import{<path to file>}{<filename>.pgf}
%%
%% Matplotlib used the following preamble
%%   \usepackage{bm}
%%   \usepackage{amsmath}
%%   \usepackage{xcolor}
%%   \usepackage{tgtermes}
%%   \makeatletter\@ifpackageloaded{underscore}{}{\usepackage[strings]{underscore}}\makeatother
%%
\begingroup%
\makeatletter%
\begin{pgfpicture}%
\pgfpathrectangle{\pgfpointorigin}{\pgfqpoint{4.500000in}{2.500000in}}%
\pgfusepath{use as bounding box, clip}%
\begin{pgfscope}%
\pgfsetbuttcap%
\pgfsetmiterjoin%
\definecolor{currentfill}{rgb}{1.000000,1.000000,1.000000}%
\pgfsetfillcolor{currentfill}%
\pgfsetlinewidth{0.000000pt}%
\definecolor{currentstroke}{rgb}{1.000000,1.000000,1.000000}%
\pgfsetstrokecolor{currentstroke}%
\pgfsetdash{}{0pt}%
\pgfpathmoveto{\pgfqpoint{0.000000in}{0.000000in}}%
\pgfpathlineto{\pgfqpoint{4.500000in}{0.000000in}}%
\pgfpathlineto{\pgfqpoint{4.500000in}{2.500000in}}%
\pgfpathlineto{\pgfqpoint{0.000000in}{2.500000in}}%
\pgfpathlineto{\pgfqpoint{0.000000in}{0.000000in}}%
\pgfpathclose%
\pgfusepath{fill}%
\end{pgfscope}%
\begin{pgfscope}%
\pgfsetbuttcap%
\pgfsetmiterjoin%
\definecolor{currentfill}{rgb}{1.000000,1.000000,1.000000}%
\pgfsetfillcolor{currentfill}%
\pgfsetlinewidth{0.000000pt}%
\definecolor{currentstroke}{rgb}{0.000000,0.000000,0.000000}%
\pgfsetstrokecolor{currentstroke}%
\pgfsetstrokeopacity{0.000000}%
\pgfsetdash{}{0pt}%
\pgfpathmoveto{\pgfqpoint{0.562500in}{0.275000in}}%
\pgfpathlineto{\pgfqpoint{4.050000in}{0.275000in}}%
\pgfpathlineto{\pgfqpoint{4.050000in}{2.200000in}}%
\pgfpathlineto{\pgfqpoint{0.562500in}{2.200000in}}%
\pgfpathlineto{\pgfqpoint{0.562500in}{0.275000in}}%
\pgfpathclose%
\pgfusepath{fill}%
\end{pgfscope}%
\begin{pgfscope}%
\pgfpathrectangle{\pgfqpoint{0.562500in}{0.275000in}}{\pgfqpoint{3.487500in}{1.925000in}}%
\pgfusepath{clip}%
\pgfsetrectcap%
\pgfsetroundjoin%
\pgfsetlinewidth{0.803000pt}%
\definecolor{currentstroke}{rgb}{0.690196,0.690196,0.690196}%
\pgfsetstrokecolor{currentstroke}%
\pgfsetdash{}{0pt}%
\pgfpathmoveto{\pgfqpoint{0.721023in}{0.275000in}}%
\pgfpathlineto{\pgfqpoint{0.721023in}{2.200000in}}%
\pgfusepath{stroke}%
\end{pgfscope}%
\begin{pgfscope}%
\pgfsetbuttcap%
\pgfsetroundjoin%
\definecolor{currentfill}{rgb}{0.000000,0.000000,0.000000}%
\pgfsetfillcolor{currentfill}%
\pgfsetlinewidth{0.803000pt}%
\definecolor{currentstroke}{rgb}{0.000000,0.000000,0.000000}%
\pgfsetstrokecolor{currentstroke}%
\pgfsetdash{}{0pt}%
\pgfsys@defobject{currentmarker}{\pgfqpoint{0.000000in}{-0.048611in}}{\pgfqpoint{0.000000in}{0.000000in}}{%
\pgfpathmoveto{\pgfqpoint{0.000000in}{0.000000in}}%
\pgfpathlineto{\pgfqpoint{0.000000in}{-0.048611in}}%
\pgfusepath{stroke,fill}%
}%
\begin{pgfscope}%
\pgfsys@transformshift{0.721023in}{0.275000in}%
\pgfsys@useobject{currentmarker}{}%
\end{pgfscope}%
\end{pgfscope}%
\begin{pgfscope}%
\definecolor{textcolor}{rgb}{0.000000,0.000000,0.000000}%
\pgfsetstrokecolor{textcolor}%
\pgfsetfillcolor{textcolor}%
\pgftext[x=0.721023in,y=0.177778in,,top]{\color{textcolor}\rmfamily\fontsize{10.000000}{12.000000}\selectfont \(\displaystyle {-1.0}\)}%
\end{pgfscope}%
\begin{pgfscope}%
\pgfpathrectangle{\pgfqpoint{0.562500in}{0.275000in}}{\pgfqpoint{3.487500in}{1.925000in}}%
\pgfusepath{clip}%
\pgfsetrectcap%
\pgfsetroundjoin%
\pgfsetlinewidth{0.803000pt}%
\definecolor{currentstroke}{rgb}{0.690196,0.690196,0.690196}%
\pgfsetstrokecolor{currentstroke}%
\pgfsetdash{}{0pt}%
\pgfpathmoveto{\pgfqpoint{1.513636in}{0.275000in}}%
\pgfpathlineto{\pgfqpoint{1.513636in}{2.200000in}}%
\pgfusepath{stroke}%
\end{pgfscope}%
\begin{pgfscope}%
\pgfsetbuttcap%
\pgfsetroundjoin%
\definecolor{currentfill}{rgb}{0.000000,0.000000,0.000000}%
\pgfsetfillcolor{currentfill}%
\pgfsetlinewidth{0.803000pt}%
\definecolor{currentstroke}{rgb}{0.000000,0.000000,0.000000}%
\pgfsetstrokecolor{currentstroke}%
\pgfsetdash{}{0pt}%
\pgfsys@defobject{currentmarker}{\pgfqpoint{0.000000in}{-0.048611in}}{\pgfqpoint{0.000000in}{0.000000in}}{%
\pgfpathmoveto{\pgfqpoint{0.000000in}{0.000000in}}%
\pgfpathlineto{\pgfqpoint{0.000000in}{-0.048611in}}%
\pgfusepath{stroke,fill}%
}%
\begin{pgfscope}%
\pgfsys@transformshift{1.513636in}{0.275000in}%
\pgfsys@useobject{currentmarker}{}%
\end{pgfscope}%
\end{pgfscope}%
\begin{pgfscope}%
\definecolor{textcolor}{rgb}{0.000000,0.000000,0.000000}%
\pgfsetstrokecolor{textcolor}%
\pgfsetfillcolor{textcolor}%
\pgftext[x=1.513636in,y=0.177778in,,top]{\color{textcolor}\rmfamily\fontsize{10.000000}{12.000000}\selectfont \(\displaystyle {-0.5}\)}%
\end{pgfscope}%
\begin{pgfscope}%
\pgfpathrectangle{\pgfqpoint{0.562500in}{0.275000in}}{\pgfqpoint{3.487500in}{1.925000in}}%
\pgfusepath{clip}%
\pgfsetrectcap%
\pgfsetroundjoin%
\pgfsetlinewidth{0.803000pt}%
\definecolor{currentstroke}{rgb}{0.690196,0.690196,0.690196}%
\pgfsetstrokecolor{currentstroke}%
\pgfsetdash{}{0pt}%
\pgfpathmoveto{\pgfqpoint{2.306250in}{0.275000in}}%
\pgfpathlineto{\pgfqpoint{2.306250in}{2.200000in}}%
\pgfusepath{stroke}%
\end{pgfscope}%
\begin{pgfscope}%
\pgfsetbuttcap%
\pgfsetroundjoin%
\definecolor{currentfill}{rgb}{0.000000,0.000000,0.000000}%
\pgfsetfillcolor{currentfill}%
\pgfsetlinewidth{0.803000pt}%
\definecolor{currentstroke}{rgb}{0.000000,0.000000,0.000000}%
\pgfsetstrokecolor{currentstroke}%
\pgfsetdash{}{0pt}%
\pgfsys@defobject{currentmarker}{\pgfqpoint{0.000000in}{-0.048611in}}{\pgfqpoint{0.000000in}{0.000000in}}{%
\pgfpathmoveto{\pgfqpoint{0.000000in}{0.000000in}}%
\pgfpathlineto{\pgfqpoint{0.000000in}{-0.048611in}}%
\pgfusepath{stroke,fill}%
}%
\begin{pgfscope}%
\pgfsys@transformshift{2.306250in}{0.275000in}%
\pgfsys@useobject{currentmarker}{}%
\end{pgfscope}%
\end{pgfscope}%
\begin{pgfscope}%
\definecolor{textcolor}{rgb}{0.000000,0.000000,0.000000}%
\pgfsetstrokecolor{textcolor}%
\pgfsetfillcolor{textcolor}%
\pgftext[x=2.306250in,y=0.177778in,,top]{\color{textcolor}\rmfamily\fontsize{10.000000}{12.000000}\selectfont \(\displaystyle {0.0}\)}%
\end{pgfscope}%
\begin{pgfscope}%
\pgfpathrectangle{\pgfqpoint{0.562500in}{0.275000in}}{\pgfqpoint{3.487500in}{1.925000in}}%
\pgfusepath{clip}%
\pgfsetrectcap%
\pgfsetroundjoin%
\pgfsetlinewidth{0.803000pt}%
\definecolor{currentstroke}{rgb}{0.690196,0.690196,0.690196}%
\pgfsetstrokecolor{currentstroke}%
\pgfsetdash{}{0pt}%
\pgfpathmoveto{\pgfqpoint{3.098864in}{0.275000in}}%
\pgfpathlineto{\pgfqpoint{3.098864in}{2.200000in}}%
\pgfusepath{stroke}%
\end{pgfscope}%
\begin{pgfscope}%
\pgfsetbuttcap%
\pgfsetroundjoin%
\definecolor{currentfill}{rgb}{0.000000,0.000000,0.000000}%
\pgfsetfillcolor{currentfill}%
\pgfsetlinewidth{0.803000pt}%
\definecolor{currentstroke}{rgb}{0.000000,0.000000,0.000000}%
\pgfsetstrokecolor{currentstroke}%
\pgfsetdash{}{0pt}%
\pgfsys@defobject{currentmarker}{\pgfqpoint{0.000000in}{-0.048611in}}{\pgfqpoint{0.000000in}{0.000000in}}{%
\pgfpathmoveto{\pgfqpoint{0.000000in}{0.000000in}}%
\pgfpathlineto{\pgfqpoint{0.000000in}{-0.048611in}}%
\pgfusepath{stroke,fill}%
}%
\begin{pgfscope}%
\pgfsys@transformshift{3.098864in}{0.275000in}%
\pgfsys@useobject{currentmarker}{}%
\end{pgfscope}%
\end{pgfscope}%
\begin{pgfscope}%
\definecolor{textcolor}{rgb}{0.000000,0.000000,0.000000}%
\pgfsetstrokecolor{textcolor}%
\pgfsetfillcolor{textcolor}%
\pgftext[x=3.098864in,y=0.177778in,,top]{\color{textcolor}\rmfamily\fontsize{10.000000}{12.000000}\selectfont \(\displaystyle {0.5}\)}%
\end{pgfscope}%
\begin{pgfscope}%
\pgfpathrectangle{\pgfqpoint{0.562500in}{0.275000in}}{\pgfqpoint{3.487500in}{1.925000in}}%
\pgfusepath{clip}%
\pgfsetrectcap%
\pgfsetroundjoin%
\pgfsetlinewidth{0.803000pt}%
\definecolor{currentstroke}{rgb}{0.690196,0.690196,0.690196}%
\pgfsetstrokecolor{currentstroke}%
\pgfsetdash{}{0pt}%
\pgfpathmoveto{\pgfqpoint{3.891477in}{0.275000in}}%
\pgfpathlineto{\pgfqpoint{3.891477in}{2.200000in}}%
\pgfusepath{stroke}%
\end{pgfscope}%
\begin{pgfscope}%
\pgfsetbuttcap%
\pgfsetroundjoin%
\definecolor{currentfill}{rgb}{0.000000,0.000000,0.000000}%
\pgfsetfillcolor{currentfill}%
\pgfsetlinewidth{0.803000pt}%
\definecolor{currentstroke}{rgb}{0.000000,0.000000,0.000000}%
\pgfsetstrokecolor{currentstroke}%
\pgfsetdash{}{0pt}%
\pgfsys@defobject{currentmarker}{\pgfqpoint{0.000000in}{-0.048611in}}{\pgfqpoint{0.000000in}{0.000000in}}{%
\pgfpathmoveto{\pgfqpoint{0.000000in}{0.000000in}}%
\pgfpathlineto{\pgfqpoint{0.000000in}{-0.048611in}}%
\pgfusepath{stroke,fill}%
}%
\begin{pgfscope}%
\pgfsys@transformshift{3.891477in}{0.275000in}%
\pgfsys@useobject{currentmarker}{}%
\end{pgfscope}%
\end{pgfscope}%
\begin{pgfscope}%
\definecolor{textcolor}{rgb}{0.000000,0.000000,0.000000}%
\pgfsetstrokecolor{textcolor}%
\pgfsetfillcolor{textcolor}%
\pgftext[x=3.891477in,y=0.177778in,,top]{\color{textcolor}\rmfamily\fontsize{10.000000}{12.000000}\selectfont \(\displaystyle {1.0}\)}%
\end{pgfscope}%
\begin{pgfscope}%
\pgfpathrectangle{\pgfqpoint{0.562500in}{0.275000in}}{\pgfqpoint{3.487500in}{1.925000in}}%
\pgfusepath{clip}%
\pgfsetrectcap%
\pgfsetroundjoin%
\pgfsetlinewidth{0.803000pt}%
\definecolor{currentstroke}{rgb}{0.690196,0.690196,0.690196}%
\pgfsetstrokecolor{currentstroke}%
\pgfsetdash{}{0pt}%
\pgfpathmoveto{\pgfqpoint{0.562500in}{0.362500in}}%
\pgfpathlineto{\pgfqpoint{4.050000in}{0.362500in}}%
\pgfusepath{stroke}%
\end{pgfscope}%
\begin{pgfscope}%
\pgfsetbuttcap%
\pgfsetroundjoin%
\definecolor{currentfill}{rgb}{0.000000,0.000000,0.000000}%
\pgfsetfillcolor{currentfill}%
\pgfsetlinewidth{0.803000pt}%
\definecolor{currentstroke}{rgb}{0.000000,0.000000,0.000000}%
\pgfsetstrokecolor{currentstroke}%
\pgfsetdash{}{0pt}%
\pgfsys@defobject{currentmarker}{\pgfqpoint{-0.048611in}{0.000000in}}{\pgfqpoint{-0.000000in}{0.000000in}}{%
\pgfpathmoveto{\pgfqpoint{-0.000000in}{0.000000in}}%
\pgfpathlineto{\pgfqpoint{-0.048611in}{0.000000in}}%
\pgfusepath{stroke,fill}%
}%
\begin{pgfscope}%
\pgfsys@transformshift{0.562500in}{0.362500in}%
\pgfsys@useobject{currentmarker}{}%
\end{pgfscope}%
\end{pgfscope}%
\begin{pgfscope}%
\definecolor{textcolor}{rgb}{0.000000,0.000000,0.000000}%
\pgfsetstrokecolor{textcolor}%
\pgfsetfillcolor{textcolor}%
\pgftext[x=0.287808in, y=0.315799in, left, base]{\color{textcolor}\rmfamily\fontsize{10.000000}{12.000000}\selectfont \(\displaystyle {-1}\)}%
\end{pgfscope}%
\begin{pgfscope}%
\pgfpathrectangle{\pgfqpoint{0.562500in}{0.275000in}}{\pgfqpoint{3.487500in}{1.925000in}}%
\pgfusepath{clip}%
\pgfsetrectcap%
\pgfsetroundjoin%
\pgfsetlinewidth{0.803000pt}%
\definecolor{currentstroke}{rgb}{0.690196,0.690196,0.690196}%
\pgfsetstrokecolor{currentstroke}%
\pgfsetdash{}{0pt}%
\pgfpathmoveto{\pgfqpoint{0.562500in}{0.945833in}}%
\pgfpathlineto{\pgfqpoint{4.050000in}{0.945833in}}%
\pgfusepath{stroke}%
\end{pgfscope}%
\begin{pgfscope}%
\pgfsetbuttcap%
\pgfsetroundjoin%
\definecolor{currentfill}{rgb}{0.000000,0.000000,0.000000}%
\pgfsetfillcolor{currentfill}%
\pgfsetlinewidth{0.803000pt}%
\definecolor{currentstroke}{rgb}{0.000000,0.000000,0.000000}%
\pgfsetstrokecolor{currentstroke}%
\pgfsetdash{}{0pt}%
\pgfsys@defobject{currentmarker}{\pgfqpoint{-0.048611in}{0.000000in}}{\pgfqpoint{-0.000000in}{0.000000in}}{%
\pgfpathmoveto{\pgfqpoint{-0.000000in}{0.000000in}}%
\pgfpathlineto{\pgfqpoint{-0.048611in}{0.000000in}}%
\pgfusepath{stroke,fill}%
}%
\begin{pgfscope}%
\pgfsys@transformshift{0.562500in}{0.945833in}%
\pgfsys@useobject{currentmarker}{}%
\end{pgfscope}%
\end{pgfscope}%
\begin{pgfscope}%
\definecolor{textcolor}{rgb}{0.000000,0.000000,0.000000}%
\pgfsetstrokecolor{textcolor}%
\pgfsetfillcolor{textcolor}%
\pgftext[x=0.395833in, y=0.899132in, left, base]{\color{textcolor}\rmfamily\fontsize{10.000000}{12.000000}\selectfont \(\displaystyle {0}\)}%
\end{pgfscope}%
\begin{pgfscope}%
\pgfpathrectangle{\pgfqpoint{0.562500in}{0.275000in}}{\pgfqpoint{3.487500in}{1.925000in}}%
\pgfusepath{clip}%
\pgfsetrectcap%
\pgfsetroundjoin%
\pgfsetlinewidth{0.803000pt}%
\definecolor{currentstroke}{rgb}{0.690196,0.690196,0.690196}%
\pgfsetstrokecolor{currentstroke}%
\pgfsetdash{}{0pt}%
\pgfpathmoveto{\pgfqpoint{0.562500in}{1.529167in}}%
\pgfpathlineto{\pgfqpoint{4.050000in}{1.529167in}}%
\pgfusepath{stroke}%
\end{pgfscope}%
\begin{pgfscope}%
\pgfsetbuttcap%
\pgfsetroundjoin%
\definecolor{currentfill}{rgb}{0.000000,0.000000,0.000000}%
\pgfsetfillcolor{currentfill}%
\pgfsetlinewidth{0.803000pt}%
\definecolor{currentstroke}{rgb}{0.000000,0.000000,0.000000}%
\pgfsetstrokecolor{currentstroke}%
\pgfsetdash{}{0pt}%
\pgfsys@defobject{currentmarker}{\pgfqpoint{-0.048611in}{0.000000in}}{\pgfqpoint{-0.000000in}{0.000000in}}{%
\pgfpathmoveto{\pgfqpoint{-0.000000in}{0.000000in}}%
\pgfpathlineto{\pgfqpoint{-0.048611in}{0.000000in}}%
\pgfusepath{stroke,fill}%
}%
\begin{pgfscope}%
\pgfsys@transformshift{0.562500in}{1.529167in}%
\pgfsys@useobject{currentmarker}{}%
\end{pgfscope}%
\end{pgfscope}%
\begin{pgfscope}%
\definecolor{textcolor}{rgb}{0.000000,0.000000,0.000000}%
\pgfsetstrokecolor{textcolor}%
\pgfsetfillcolor{textcolor}%
\pgftext[x=0.395833in, y=1.482465in, left, base]{\color{textcolor}\rmfamily\fontsize{10.000000}{12.000000}\selectfont \(\displaystyle {1}\)}%
\end{pgfscope}%
\begin{pgfscope}%
\pgfpathrectangle{\pgfqpoint{0.562500in}{0.275000in}}{\pgfqpoint{3.487500in}{1.925000in}}%
\pgfusepath{clip}%
\pgfsetrectcap%
\pgfsetroundjoin%
\pgfsetlinewidth{0.803000pt}%
\definecolor{currentstroke}{rgb}{0.690196,0.690196,0.690196}%
\pgfsetstrokecolor{currentstroke}%
\pgfsetdash{}{0pt}%
\pgfpathmoveto{\pgfqpoint{0.562500in}{2.112500in}}%
\pgfpathlineto{\pgfqpoint{4.050000in}{2.112500in}}%
\pgfusepath{stroke}%
\end{pgfscope}%
\begin{pgfscope}%
\pgfsetbuttcap%
\pgfsetroundjoin%
\definecolor{currentfill}{rgb}{0.000000,0.000000,0.000000}%
\pgfsetfillcolor{currentfill}%
\pgfsetlinewidth{0.803000pt}%
\definecolor{currentstroke}{rgb}{0.000000,0.000000,0.000000}%
\pgfsetstrokecolor{currentstroke}%
\pgfsetdash{}{0pt}%
\pgfsys@defobject{currentmarker}{\pgfqpoint{-0.048611in}{0.000000in}}{\pgfqpoint{-0.000000in}{0.000000in}}{%
\pgfpathmoveto{\pgfqpoint{-0.000000in}{0.000000in}}%
\pgfpathlineto{\pgfqpoint{-0.048611in}{0.000000in}}%
\pgfusepath{stroke,fill}%
}%
\begin{pgfscope}%
\pgfsys@transformshift{0.562500in}{2.112500in}%
\pgfsys@useobject{currentmarker}{}%
\end{pgfscope}%
\end{pgfscope}%
\begin{pgfscope}%
\definecolor{textcolor}{rgb}{0.000000,0.000000,0.000000}%
\pgfsetstrokecolor{textcolor}%
\pgfsetfillcolor{textcolor}%
\pgftext[x=0.395833in, y=2.065799in, left, base]{\color{textcolor}\rmfamily\fontsize{10.000000}{12.000000}\selectfont \(\displaystyle {2}\)}%
\end{pgfscope}%
\begin{pgfscope}%
\pgfpathrectangle{\pgfqpoint{0.562500in}{0.275000in}}{\pgfqpoint{3.487500in}{1.925000in}}%
\pgfusepath{clip}%
\pgfsetrectcap%
\pgfsetroundjoin%
\pgfsetlinewidth{1.505625pt}%
\definecolor{currentstroke}{rgb}{0.121569,0.466667,0.705882}%
\pgfsetstrokecolor{currentstroke}%
\pgfsetdash{}{0pt}%
\pgfpathmoveto{\pgfqpoint{2.306250in}{0.945833in}}%
\pgfpathlineto{\pgfqpoint{2.306250in}{0.945833in}}%
\pgfusepath{stroke}%
\end{pgfscope}%
\begin{pgfscope}%
\pgfpathrectangle{\pgfqpoint{0.562500in}{0.275000in}}{\pgfqpoint{3.487500in}{1.925000in}}%
\pgfusepath{clip}%
\pgfsetrectcap%
\pgfsetroundjoin%
\pgfsetlinewidth{1.505625pt}%
\definecolor{currentstroke}{rgb}{1.000000,0.498039,0.054902}%
\pgfsetstrokecolor{currentstroke}%
\pgfsetdash{}{0pt}%
\pgfpathmoveto{\pgfqpoint{3.891477in}{0.362500in}}%
\pgfpathlineto{\pgfqpoint{3.781003in}{0.363501in}}%
\pgfpathlineto{\pgfqpoint{3.669965in}{0.366436in}}%
\pgfpathlineto{\pgfqpoint{3.556115in}{0.371251in}}%
\pgfpathlineto{\pgfqpoint{3.437544in}{0.377920in}}%
\pgfpathlineto{\pgfqpoint{3.247436in}{0.391416in}}%
\pgfpathlineto{\pgfqpoint{3.039148in}{0.409219in}}%
\pgfpathlineto{\pgfqpoint{2.806136in}{0.431656in}}%
\pgfpathlineto{\pgfqpoint{2.637728in}{0.449325in}}%
\pgfpathlineto{\pgfqpoint{2.465583in}{0.469110in}}%
\pgfpathlineto{\pgfqpoint{2.294772in}{0.490916in}}%
\pgfpathlineto{\pgfqpoint{2.129630in}{0.514610in}}%
\pgfpathlineto{\pgfqpoint{1.973751in}{0.540020in}}%
\pgfpathlineto{\pgfqpoint{1.829992in}{0.566932in}}%
\pgfpathlineto{\pgfqpoint{1.700471in}{0.595093in}}%
\pgfpathlineto{\pgfqpoint{1.641510in}{0.609553in}}%
\pgfpathlineto{\pgfqpoint{1.586568in}{0.624212in}}%
\pgfpathlineto{\pgfqpoint{1.535702in}{0.639027in}}%
\pgfpathlineto{\pgfqpoint{1.488923in}{0.653955in}}%
\pgfpathlineto{\pgfqpoint{1.446197in}{0.668946in}}%
\pgfpathlineto{\pgfqpoint{1.407440in}{0.683950in}}%
\pgfpathlineto{\pgfqpoint{1.372527in}{0.698915in}}%
\pgfpathlineto{\pgfqpoint{1.341283in}{0.713786in}}%
\pgfpathlineto{\pgfqpoint{1.313488in}{0.728504in}}%
\pgfpathlineto{\pgfqpoint{1.266531in}{0.757397in}}%
\pgfpathlineto{\pgfqpoint{1.228197in}{0.785632in}}%
\pgfpathlineto{\pgfqpoint{1.196762in}{0.813113in}}%
\pgfpathlineto{\pgfqpoint{1.170901in}{0.839765in}}%
\pgfpathlineto{\pgfqpoint{1.149599in}{0.865538in}}%
\pgfpathlineto{\pgfqpoint{1.132120in}{0.890388in}}%
\pgfpathlineto{\pgfqpoint{1.117843in}{0.914304in}}%
\pgfpathlineto{\pgfqpoint{1.106244in}{0.937294in}}%
\pgfpathlineto{\pgfqpoint{1.096959in}{0.959361in}}%
\pgfpathlineto{\pgfqpoint{1.089790in}{0.980508in}}%
\pgfpathlineto{\pgfqpoint{1.084698in}{1.000733in}}%
\pgfpathlineto{\pgfqpoint{1.081790in}{1.020031in}}%
\pgfpathlineto{\pgfqpoint{1.080813in}{1.038415in}}%
\pgfpathlineto{\pgfqpoint{1.081537in}{1.055903in}}%
\pgfpathlineto{\pgfqpoint{1.083881in}{1.072511in}}%
\pgfpathlineto{\pgfqpoint{1.087800in}{1.088254in}}%
\pgfpathlineto{\pgfqpoint{1.093279in}{1.103144in}}%
\pgfpathlineto{\pgfqpoint{1.100337in}{1.117194in}}%
\pgfpathlineto{\pgfqpoint{1.109031in}{1.130416in}}%
\pgfpathlineto{\pgfqpoint{1.119437in}{1.142824in}}%
\pgfpathlineto{\pgfqpoint{1.131486in}{1.154430in}}%
\pgfpathlineto{\pgfqpoint{1.145100in}{1.165241in}}%
\pgfpathlineto{\pgfqpoint{1.160239in}{1.175265in}}%
\pgfpathlineto{\pgfqpoint{1.185804in}{1.188845in}}%
\pgfpathlineto{\pgfqpoint{1.214939in}{1.200699in}}%
\pgfpathlineto{\pgfqpoint{1.247970in}{1.210852in}}%
\pgfpathlineto{\pgfqpoint{1.285409in}{1.219328in}}%
\pgfpathlineto{\pgfqpoint{1.327889in}{1.226149in}}%
\pgfpathlineto{\pgfqpoint{1.375411in}{1.231290in}}%
\pgfpathlineto{\pgfqpoint{1.428353in}{1.234725in}}%
\pgfpathlineto{\pgfqpoint{1.487312in}{1.236420in}}%
\pgfpathlineto{\pgfqpoint{1.552923in}{1.236331in}}%
\pgfpathlineto{\pgfqpoint{1.625852in}{1.234400in}}%
\pgfpathlineto{\pgfqpoint{1.735690in}{1.228839in}}%
\pgfpathlineto{\pgfqpoint{1.861545in}{1.219680in}}%
\pgfpathlineto{\pgfqpoint{2.005000in}{1.206681in}}%
\pgfpathlineto{\pgfqpoint{2.166482in}{1.189564in}}%
\pgfpathlineto{\pgfqpoint{2.343906in}{1.168128in}}%
\pgfpathlineto{\pgfqpoint{2.484391in}{1.149164in}}%
\pgfpathlineto{\pgfqpoint{2.626905in}{1.127795in}}%
\pgfpathlineto{\pgfqpoint{2.766529in}{1.104209in}}%
\pgfpathlineto{\pgfqpoint{2.855567in}{1.087397in}}%
\pgfpathlineto{\pgfqpoint{2.939661in}{1.069848in}}%
\pgfpathlineto{\pgfqpoint{3.017018in}{1.051693in}}%
\pgfpathlineto{\pgfqpoint{3.087013in}{1.033088in}}%
\pgfpathlineto{\pgfqpoint{3.149875in}{1.014196in}}%
\pgfpathlineto{\pgfqpoint{3.205851in}{0.995163in}}%
\pgfpathlineto{\pgfqpoint{3.255200in}{0.976124in}}%
\pgfpathlineto{\pgfqpoint{3.298201in}{0.957199in}}%
\pgfpathlineto{\pgfqpoint{3.335148in}{0.938496in}}%
\pgfpathlineto{\pgfqpoint{3.366348in}{0.920108in}}%
\pgfpathlineto{\pgfqpoint{3.392130in}{0.902115in}}%
\pgfpathlineto{\pgfqpoint{3.412833in}{0.884582in}}%
\pgfpathlineto{\pgfqpoint{3.428817in}{0.867564in}}%
\pgfpathlineto{\pgfqpoint{3.440458in}{0.851112in}}%
\pgfpathlineto{\pgfqpoint{3.448368in}{0.835290in}}%
\pgfpathlineto{\pgfqpoint{3.453279in}{0.820119in}}%
\pgfpathlineto{\pgfqpoint{3.455772in}{0.805612in}}%
\pgfpathlineto{\pgfqpoint{3.456269in}{0.791780in}}%
\pgfpathlineto{\pgfqpoint{3.455029in}{0.778632in}}%
\pgfpathlineto{\pgfqpoint{3.452154in}{0.766171in}}%
\pgfpathlineto{\pgfqpoint{3.447582in}{0.754402in}}%
\pgfpathlineto{\pgfqpoint{3.441095in}{0.743322in}}%
\pgfpathlineto{\pgfqpoint{3.432312in}{0.732927in}}%
\pgfpathlineto{\pgfqpoint{3.420694in}{0.723212in}}%
\pgfpathlineto{\pgfqpoint{3.406349in}{0.714186in}}%
\pgfpathlineto{\pgfqpoint{3.390104in}{0.705872in}}%
\pgfpathlineto{\pgfqpoint{3.362198in}{0.694739in}}%
\pgfpathlineto{\pgfqpoint{3.329982in}{0.685214in}}%
\pgfpathlineto{\pgfqpoint{3.293282in}{0.677304in}}%
\pgfpathlineto{\pgfqpoint{3.251812in}{0.671014in}}%
\pgfpathlineto{\pgfqpoint{3.205172in}{0.666354in}}%
\pgfpathlineto{\pgfqpoint{3.152945in}{0.663334in}}%
\pgfpathlineto{\pgfqpoint{3.094925in}{0.661989in}}%
\pgfpathlineto{\pgfqpoint{3.030182in}{0.662381in}}%
\pgfpathlineto{\pgfqpoint{2.931883in}{0.665727in}}%
\pgfpathlineto{\pgfqpoint{2.818410in}{0.672475in}}%
\pgfpathlineto{\pgfqpoint{2.688627in}{0.682832in}}%
\pgfpathlineto{\pgfqpoint{2.541921in}{0.697031in}}%
\pgfpathlineto{\pgfqpoint{2.378174in}{0.715326in}}%
\pgfpathlineto{\pgfqpoint{2.198222in}{0.737993in}}%
\pgfpathlineto{\pgfqpoint{2.058427in}{0.757810in}}%
\pgfpathlineto{\pgfqpoint{1.919952in}{0.779859in}}%
\pgfpathlineto{\pgfqpoint{1.787069in}{0.803901in}}%
\pgfpathlineto{\pgfqpoint{1.703430in}{0.820884in}}%
\pgfpathlineto{\pgfqpoint{1.624867in}{0.838506in}}%
\pgfpathlineto{\pgfqpoint{1.552208in}{0.856643in}}%
\pgfpathlineto{\pgfqpoint{1.486164in}{0.875158in}}%
\pgfpathlineto{\pgfqpoint{1.427326in}{0.893900in}}%
\pgfpathlineto{\pgfqpoint{1.376131in}{0.912704in}}%
\pgfpathlineto{\pgfqpoint{1.332106in}{0.931434in}}%
\pgfpathlineto{\pgfqpoint{1.294458in}{0.949984in}}%
\pgfpathlineto{\pgfqpoint{1.262565in}{0.968255in}}%
\pgfpathlineto{\pgfqpoint{1.235890in}{0.986161in}}%
\pgfpathlineto{\pgfqpoint{1.213991in}{1.003630in}}%
\pgfpathlineto{\pgfqpoint{1.196514in}{1.020596in}}%
\pgfpathlineto{\pgfqpoint{1.183100in}{1.037008in}}%
\pgfpathlineto{\pgfqpoint{1.173177in}{1.052827in}}%
\pgfpathlineto{\pgfqpoint{1.166351in}{1.068025in}}%
\pgfpathlineto{\pgfqpoint{1.162313in}{1.082579in}}%
\pgfpathlineto{\pgfqpoint{1.160838in}{1.096471in}}%
\pgfpathlineto{\pgfqpoint{1.161779in}{1.109688in}}%
\pgfpathlineto{\pgfqpoint{1.165070in}{1.122222in}}%
\pgfpathlineto{\pgfqpoint{1.170641in}{1.134066in}}%
\pgfpathlineto{\pgfqpoint{1.178329in}{1.145218in}}%
\pgfpathlineto{\pgfqpoint{1.187992in}{1.155671in}}%
\pgfpathlineto{\pgfqpoint{1.199535in}{1.165423in}}%
\pgfpathlineto{\pgfqpoint{1.212898in}{1.174472in}}%
\pgfpathlineto{\pgfqpoint{1.228067in}{1.182817in}}%
\pgfpathlineto{\pgfqpoint{1.254269in}{1.194017in}}%
\pgfpathlineto{\pgfqpoint{1.284853in}{1.203641in}}%
\pgfpathlineto{\pgfqpoint{1.320275in}{1.211700in}}%
\pgfpathlineto{\pgfqpoint{1.360663in}{1.218183in}}%
\pgfpathlineto{\pgfqpoint{1.406134in}{1.223059in}}%
\pgfpathlineto{\pgfqpoint{1.457132in}{1.226298in}}%
\pgfpathlineto{\pgfqpoint{1.514159in}{1.227861in}}%
\pgfpathlineto{\pgfqpoint{1.577773in}{1.227700in}}%
\pgfpathlineto{\pgfqpoint{1.648591in}{1.225756in}}%
\pgfpathlineto{\pgfqpoint{1.755392in}{1.220271in}}%
\pgfpathlineto{\pgfqpoint{1.877909in}{1.211302in}}%
\pgfpathlineto{\pgfqpoint{2.017583in}{1.198591in}}%
\pgfpathlineto{\pgfqpoint{2.174690in}{1.181882in}}%
\pgfpathlineto{\pgfqpoint{2.347380in}{1.160974in}}%
\pgfpathlineto{\pgfqpoint{2.484178in}{1.142485in}}%
\pgfpathlineto{\pgfqpoint{2.623199in}{1.121653in}}%
\pgfpathlineto{\pgfqpoint{2.760018in}{1.098647in}}%
\pgfpathlineto{\pgfqpoint{2.847510in}{1.082236in}}%
\pgfpathlineto{\pgfqpoint{2.930185in}{1.065066in}}%
\pgfpathlineto{\pgfqpoint{3.007086in}{1.047290in}}%
\pgfpathlineto{\pgfqpoint{3.077574in}{1.029069in}}%
\pgfpathlineto{\pgfqpoint{3.141215in}{1.010556in}}%
\pgfpathlineto{\pgfqpoint{3.197783in}{0.991889in}}%
\pgfpathlineto{\pgfqpoint{3.247260in}{0.973197in}}%
\pgfpathlineto{\pgfqpoint{3.289833in}{0.954597in}}%
\pgfpathlineto{\pgfqpoint{3.325899in}{0.936195in}}%
\pgfpathlineto{\pgfqpoint{3.356060in}{0.918086in}}%
\pgfpathlineto{\pgfqpoint{3.381115in}{0.900352in}}%
\pgfpathlineto{\pgfqpoint{3.401292in}{0.883085in}}%
\pgfpathlineto{\pgfqpoint{3.417120in}{0.866339in}}%
\pgfpathlineto{\pgfqpoint{3.429418in}{0.850150in}}%
\pgfpathlineto{\pgfqpoint{3.438802in}{0.834546in}}%
\pgfpathlineto{\pgfqpoint{3.445687in}{0.819553in}}%
\pgfpathlineto{\pgfqpoint{3.450282in}{0.805191in}}%
\pgfpathlineto{\pgfqpoint{3.452597in}{0.791478in}}%
\pgfpathlineto{\pgfqpoint{3.452436in}{0.778426in}}%
\pgfpathlineto{\pgfqpoint{3.449401in}{0.766043in}}%
\pgfpathlineto{\pgfqpoint{3.443236in}{0.754337in}}%
\pgfpathlineto{\pgfqpoint{3.434890in}{0.743330in}}%
\pgfpathlineto{\pgfqpoint{3.424574in}{0.733026in}}%
\pgfpathlineto{\pgfqpoint{3.412363in}{0.723428in}}%
\pgfpathlineto{\pgfqpoint{3.398301in}{0.714536in}}%
\pgfpathlineto{\pgfqpoint{3.382398in}{0.706350in}}%
\pgfpathlineto{\pgfqpoint{3.355031in}{0.695396in}}%
\pgfpathlineto{\pgfqpoint{3.323248in}{0.686028in}}%
\pgfpathlineto{\pgfqpoint{3.286697in}{0.678237in}}%
\pgfpathlineto{\pgfqpoint{3.245286in}{0.672038in}}%
\pgfpathlineto{\pgfqpoint{3.198719in}{0.667455in}}%
\pgfpathlineto{\pgfqpoint{3.146534in}{0.664519in}}%
\pgfpathlineto{\pgfqpoint{3.088215in}{0.663270in}}%
\pgfpathlineto{\pgfqpoint{3.023191in}{0.663759in}}%
\pgfpathlineto{\pgfqpoint{2.924969in}{0.667221in}}%
\pgfpathlineto{\pgfqpoint{2.812066in}{0.674051in}}%
\pgfpathlineto{\pgfqpoint{2.682805in}{0.684462in}}%
\pgfpathlineto{\pgfqpoint{2.536036in}{0.698726in}}%
\pgfpathlineto{\pgfqpoint{2.372254in}{0.717080in}}%
\pgfpathlineto{\pgfqpoint{2.194374in}{0.739691in}}%
\pgfpathlineto{\pgfqpoint{2.055716in}{0.759442in}}%
\pgfpathlineto{\pgfqpoint{1.917100in}{0.781464in}}%
\pgfpathlineto{\pgfqpoint{1.783280in}{0.805528in}}%
\pgfpathlineto{\pgfqpoint{1.699361in}{0.822558in}}%
\pgfpathlineto{\pgfqpoint{1.621117in}{0.840236in}}%
\pgfpathlineto{\pgfqpoint{1.549202in}{0.858394in}}%
\pgfpathlineto{\pgfqpoint{1.484071in}{0.876878in}}%
\pgfpathlineto{\pgfqpoint{1.425981in}{0.895543in}}%
\pgfpathlineto{\pgfqpoint{1.374987in}{0.914259in}}%
\pgfpathlineto{\pgfqpoint{1.330944in}{0.932905in}}%
\pgfpathlineto{\pgfqpoint{1.293511in}{0.951372in}}%
\pgfpathlineto{\pgfqpoint{1.262143in}{0.969564in}}%
\pgfpathlineto{\pgfqpoint{1.236099in}{0.987396in}}%
\pgfpathlineto{\pgfqpoint{1.214820in}{1.004785in}}%
\pgfpathlineto{\pgfqpoint{1.198134in}{1.021658in}}%
\pgfpathlineto{\pgfqpoint{1.185190in}{1.037979in}}%
\pgfpathlineto{\pgfqpoint{1.175309in}{1.053719in}}%
\pgfpathlineto{\pgfqpoint{1.168009in}{1.068853in}}%
\pgfpathlineto{\pgfqpoint{1.163009in}{1.083357in}}%
\pgfpathlineto{\pgfqpoint{1.160231in}{1.097216in}}%
\pgfpathlineto{\pgfqpoint{1.159795in}{1.110416in}}%
\pgfpathlineto{\pgfqpoint{1.162022in}{1.122948in}}%
\pgfpathlineto{\pgfqpoint{1.167396in}{1.134807in}}%
\pgfpathlineto{\pgfqpoint{1.175299in}{1.145974in}}%
\pgfpathlineto{\pgfqpoint{1.185200in}{1.156437in}}%
\pgfpathlineto{\pgfqpoint{1.197015in}{1.166196in}}%
\pgfpathlineto{\pgfqpoint{1.210692in}{1.175249in}}%
\pgfpathlineto{\pgfqpoint{1.226212in}{1.183594in}}%
\pgfpathlineto{\pgfqpoint{1.252981in}{1.194787in}}%
\pgfpathlineto{\pgfqpoint{1.284105in}{1.204391in}}%
\pgfpathlineto{\pgfqpoint{1.319919in}{1.212415in}}%
\pgfpathlineto{\pgfqpoint{1.360619in}{1.218851in}}%
\pgfpathlineto{\pgfqpoint{1.406398in}{1.223676in}}%
\pgfpathlineto{\pgfqpoint{1.457722in}{1.226859in}}%
\pgfpathlineto{\pgfqpoint{1.515106in}{1.228362in}}%
\pgfpathlineto{\pgfqpoint{1.579116in}{1.228135in}}%
\pgfpathlineto{\pgfqpoint{1.650370in}{1.226119in}}%
\pgfpathlineto{\pgfqpoint{1.757804in}{1.220526in}}%
\pgfpathlineto{\pgfqpoint{1.881010in}{1.211433in}}%
\pgfpathlineto{\pgfqpoint{2.021410in}{1.198589in}}%
\pgfpathlineto{\pgfqpoint{2.179287in}{1.181731in}}%
\pgfpathlineto{\pgfqpoint{2.352631in}{1.160665in}}%
\pgfpathlineto{\pgfqpoint{2.489834in}{1.142053in}}%
\pgfpathlineto{\pgfqpoint{2.629048in}{1.121098in}}%
\pgfpathlineto{\pgfqpoint{2.765799in}{1.097978in}}%
\pgfpathlineto{\pgfqpoint{2.853163in}{1.081498in}}%
\pgfpathlineto{\pgfqpoint{2.935626in}{1.064277in}}%
\pgfpathlineto{\pgfqpoint{3.012128in}{1.046449in}}%
\pgfpathlineto{\pgfqpoint{3.082124in}{1.028181in}}%
\pgfpathlineto{\pgfqpoint{3.145253in}{1.009627in}}%
\pgfpathlineto{\pgfqpoint{3.201338in}{0.990929in}}%
\pgfpathlineto{\pgfqpoint{3.250387in}{0.972217in}}%
\pgfpathlineto{\pgfqpoint{3.292593in}{0.953609in}}%
\pgfpathlineto{\pgfqpoint{3.328333in}{0.935209in}}%
\pgfpathlineto{\pgfqpoint{3.358167in}{0.917110in}}%
\pgfpathlineto{\pgfqpoint{3.382840in}{0.899394in}}%
\pgfpathlineto{\pgfqpoint{3.402867in}{0.882139in}}%
\pgfpathlineto{\pgfqpoint{3.418433in}{0.865414in}}%
\pgfpathlineto{\pgfqpoint{3.430374in}{0.849252in}}%
\pgfpathlineto{\pgfqpoint{3.439351in}{0.833681in}}%
\pgfpathlineto{\pgfqpoint{3.445830in}{0.818724in}}%
\pgfpathlineto{\pgfqpoint{3.450080in}{0.804402in}}%
\pgfpathlineto{\pgfqpoint{3.452176in}{0.790730in}}%
\pgfpathlineto{\pgfqpoint{3.451994in}{0.777720in}}%
\pgfpathlineto{\pgfqpoint{3.449216in}{0.765380in}}%
\pgfpathlineto{\pgfqpoint{3.443344in}{0.753712in}}%
\pgfpathlineto{\pgfqpoint{3.434887in}{0.742736in}}%
\pgfpathlineto{\pgfqpoint{3.424445in}{0.732463in}}%
\pgfpathlineto{\pgfqpoint{3.412093in}{0.722896in}}%
\pgfpathlineto{\pgfqpoint{3.397879in}{0.714034in}}%
\pgfpathlineto{\pgfqpoint{3.381819in}{0.705881in}}%
\pgfpathlineto{\pgfqpoint{3.354226in}{0.694976in}}%
\pgfpathlineto{\pgfqpoint{3.322262in}{0.685661in}}%
\pgfpathlineto{\pgfqpoint{3.285592in}{0.677930in}}%
\pgfpathlineto{\pgfqpoint{3.243975in}{0.671788in}}%
\pgfpathlineto{\pgfqpoint{3.197227in}{0.667262in}}%
\pgfpathlineto{\pgfqpoint{3.144847in}{0.664382in}}%
\pgfpathlineto{\pgfqpoint{3.086295in}{0.663190in}}%
\pgfpathlineto{\pgfqpoint{3.020989in}{0.663737in}}%
\pgfpathlineto{\pgfqpoint{2.922325in}{0.667282in}}%
\pgfpathlineto{\pgfqpoint{2.808954in}{0.674207in}}%
\pgfpathlineto{\pgfqpoint{2.679210in}{0.684723in}}%
\pgfpathlineto{\pgfqpoint{2.531963in}{0.699095in}}%
\pgfpathlineto{\pgfqpoint{2.367668in}{0.717569in}}%
\pgfpathlineto{\pgfqpoint{2.189500in}{0.740299in}}%
\pgfpathlineto{\pgfqpoint{2.050628in}{0.760140in}}%
\pgfpathlineto{\pgfqpoint{1.912108in}{0.782246in}}%
\pgfpathlineto{\pgfqpoint{1.778627in}{0.806381in}}%
\pgfpathlineto{\pgfqpoint{1.694774in}{0.823433in}}%
\pgfpathlineto{\pgfqpoint{1.616902in}{0.841117in}}%
\pgfpathlineto{\pgfqpoint{1.546154in}{0.859291in}}%
\pgfpathlineto{\pgfqpoint{1.482367in}{0.877795in}}%
\pgfpathlineto{\pgfqpoint{1.425321in}{0.896477in}}%
\pgfpathlineto{\pgfqpoint{1.374777in}{0.915203in}}%
\pgfpathlineto{\pgfqpoint{1.330481in}{0.933850in}}%
\pgfpathlineto{\pgfqpoint{1.292163in}{0.952309in}}%
\pgfpathlineto{\pgfqpoint{1.259535in}{0.970486in}}%
\pgfpathlineto{\pgfqpoint{1.232295in}{0.988298in}}%
\pgfpathlineto{\pgfqpoint{1.210123in}{1.005678in}}%
\pgfpathlineto{\pgfqpoint{1.192684in}{1.022571in}}%
\pgfpathlineto{\pgfqpoint{1.179624in}{1.038937in}}%
\pgfpathlineto{\pgfqpoint{1.170530in}{1.054715in}}%
\pgfpathlineto{\pgfqpoint{1.164752in}{1.069859in}}%
\pgfpathlineto{\pgfqpoint{1.161683in}{1.084350in}}%
\pgfpathlineto{\pgfqpoint{1.160860in}{1.098176in}}%
\pgfpathlineto{\pgfqpoint{1.161958in}{1.111327in}}%
\pgfpathlineto{\pgfqpoint{1.164792in}{1.123794in}}%
\pgfpathlineto{\pgfqpoint{1.169318in}{1.135574in}}%
\pgfpathlineto{\pgfqpoint{1.175632in}{1.146667in}}%
\pgfpathlineto{\pgfqpoint{1.183969in}{1.157073in}}%
\pgfpathlineto{\pgfqpoint{1.194705in}{1.166800in}}%
\pgfpathlineto{\pgfqpoint{1.208285in}{1.175854in}}%
\pgfpathlineto{\pgfqpoint{1.224087in}{1.184212in}}%
\pgfpathlineto{\pgfqpoint{1.251374in}{1.195422in}}%
\pgfpathlineto{\pgfqpoint{1.283007in}{1.205037in}}%
\pgfpathlineto{\pgfqpoint{1.319134in}{1.213049in}}%
\pgfpathlineto{\pgfqpoint{1.360011in}{1.219447in}}%
\pgfpathlineto{\pgfqpoint{1.406004in}{1.224221in}}%
\pgfpathlineto{\pgfqpoint{1.457587in}{1.227355in}}%
\pgfpathlineto{\pgfqpoint{1.514962in}{1.228832in}}%
\pgfpathlineto{\pgfqpoint{1.578780in}{1.228591in}}%
\pgfpathlineto{\pgfqpoint{1.650120in}{1.226555in}}%
\pgfpathlineto{\pgfqpoint{1.758364in}{1.220906in}}%
\pgfpathlineto{\pgfqpoint{1.882807in}{1.211714in}}%
\pgfpathlineto{\pgfqpoint{2.024064in}{1.198761in}}%
\pgfpathlineto{\pgfqpoint{2.182012in}{1.181819in}}%
\pgfpathlineto{\pgfqpoint{2.355890in}{1.160637in}}%
\pgfpathlineto{\pgfqpoint{2.494486in}{1.141843in}}%
\pgfpathlineto{\pgfqpoint{2.634352in}{1.120717in}}%
\pgfpathlineto{\pgfqpoint{2.770468in}{1.097499in}}%
\pgfpathlineto{\pgfqpoint{2.857013in}{1.081003in}}%
\pgfpathlineto{\pgfqpoint{2.938947in}{1.063808in}}%
\pgfpathlineto{\pgfqpoint{3.015372in}{1.046029in}}%
\pgfpathlineto{\pgfqpoint{3.085550in}{1.027788in}}%
\pgfpathlineto{\pgfqpoint{3.148899in}{1.009218in}}%
\pgfpathlineto{\pgfqpoint{3.204996in}{0.990467in}}%
\pgfpathlineto{\pgfqpoint{3.253581in}{0.971688in}}%
\pgfpathlineto{\pgfqpoint{3.295014in}{0.953032in}}%
\pgfpathlineto{\pgfqpoint{3.330221in}{0.934599in}}%
\pgfpathlineto{\pgfqpoint{3.359980in}{0.916478in}}%
\pgfpathlineto{\pgfqpoint{3.384907in}{0.898743in}}%
\pgfpathlineto{\pgfqpoint{3.405456in}{0.881465in}}%
\pgfpathlineto{\pgfqpoint{3.421918in}{0.864700in}}%
\pgfpathlineto{\pgfqpoint{3.434424in}{0.848498in}}%
\pgfpathlineto{\pgfqpoint{3.443112in}{0.832899in}}%
\pgfpathlineto{\pgfqpoint{3.448699in}{0.817936in}}%
\pgfpathlineto{\pgfqpoint{3.451560in}{0.803626in}}%
\pgfpathlineto{\pgfqpoint{3.451955in}{0.789985in}}%
\pgfpathlineto{\pgfqpoint{3.450071in}{0.777024in}}%
\pgfpathlineto{\pgfqpoint{3.446026in}{0.764751in}}%
\pgfpathlineto{\pgfqpoint{3.439866in}{0.753171in}}%
\pgfpathlineto{\pgfqpoint{3.431568in}{0.742285in}}%
\pgfpathlineto{\pgfqpoint{3.421167in}{0.732095in}}%
\pgfpathlineto{\pgfqpoint{3.408791in}{0.722603in}}%
\pgfpathlineto{\pgfqpoint{3.394507in}{0.713811in}}%
\pgfpathlineto{\pgfqpoint{3.378353in}{0.705723in}}%
\pgfpathlineto{\pgfqpoint{3.350624in}{0.694912in}}%
\pgfpathlineto{\pgfqpoint{3.318602in}{0.685691in}}%
\pgfpathlineto{\pgfqpoint{3.282044in}{0.678063in}}%
\pgfpathlineto{\pgfqpoint{3.240560in}{0.672028in}}%
\pgfpathlineto{\pgfqpoint{3.193693in}{0.667588in}}%
\pgfpathlineto{\pgfqpoint{3.141329in}{0.664780in}}%
\pgfpathlineto{\pgfqpoint{3.082814in}{0.663652in}}%
\pgfpathlineto{\pgfqpoint{3.017426in}{0.664259in}}%
\pgfpathlineto{\pgfqpoint{2.918391in}{0.667882in}}%
\pgfpathlineto{\pgfqpoint{2.804485in}{0.674891in}}%
\pgfpathlineto{\pgfqpoint{2.674406in}{0.685503in}}%
\pgfpathlineto{\pgfqpoint{2.526995in}{0.699966in}}%
\pgfpathlineto{\pgfqpoint{2.361669in}{0.718499in}}%
\pgfpathlineto{\pgfqpoint{2.228655in}{0.735209in}}%
\pgfpathlineto{\pgfqpoint{2.091308in}{0.754342in}}%
\pgfpathlineto{\pgfqpoint{1.953168in}{0.775796in}}%
\pgfpathlineto{\pgfqpoint{1.818225in}{0.799362in}}%
\pgfpathlineto{\pgfqpoint{1.732210in}{0.816094in}}%
\pgfpathlineto{\pgfqpoint{1.651003in}{0.833507in}}%
\pgfpathlineto{\pgfqpoint{1.576114in}{0.851459in}}%
\pgfpathlineto{\pgfqpoint{1.508650in}{0.869803in}}%
\pgfpathlineto{\pgfqpoint{1.448460in}{0.888397in}}%
\pgfpathlineto{\pgfqpoint{1.395198in}{0.907102in}}%
\pgfpathlineto{\pgfqpoint{1.348515in}{0.925793in}}%
\pgfpathlineto{\pgfqpoint{1.308069in}{0.944353in}}%
\pgfpathlineto{\pgfqpoint{1.273516in}{0.962679in}}%
\pgfpathlineto{\pgfqpoint{1.244514in}{0.980679in}}%
\pgfpathlineto{\pgfqpoint{1.220726in}{0.998272in}}%
\pgfpathlineto{\pgfqpoint{1.201814in}{1.015389in}}%
\pgfpathlineto{\pgfqpoint{1.187346in}{1.031967in}}%
\pgfpathlineto{\pgfqpoint{1.176621in}{1.047960in}}%
\pgfpathlineto{\pgfqpoint{1.169038in}{1.063342in}}%
\pgfpathlineto{\pgfqpoint{1.164135in}{1.078089in}}%
\pgfpathlineto{\pgfqpoint{1.161587in}{1.092185in}}%
\pgfpathlineto{\pgfqpoint{1.161211in}{1.105615in}}%
\pgfpathlineto{\pgfqpoint{1.162963in}{1.118370in}}%
\pgfpathlineto{\pgfqpoint{1.166937in}{1.130443in}}%
\pgfpathlineto{\pgfqpoint{1.173367in}{1.141833in}}%
\pgfpathlineto{\pgfqpoint{1.182380in}{1.152538in}}%
\pgfpathlineto{\pgfqpoint{1.193434in}{1.162543in}}%
\pgfpathlineto{\pgfqpoint{1.206413in}{1.171846in}}%
\pgfpathlineto{\pgfqpoint{1.221271in}{1.180445in}}%
\pgfpathlineto{\pgfqpoint{1.247052in}{1.192021in}}%
\pgfpathlineto{\pgfqpoint{1.277088in}{1.202006in}}%
\pgfpathlineto{\pgfqpoint{1.311585in}{1.210401in}}%
\pgfpathlineto{\pgfqpoint{1.350891in}{1.217207in}}%
\pgfpathlineto{\pgfqpoint{1.395345in}{1.222419in}}%
\pgfpathlineto{\pgfqpoint{1.445116in}{1.226006in}}%
\pgfpathlineto{\pgfqpoint{1.500805in}{1.227930in}}%
\pgfpathlineto{\pgfqpoint{1.563036in}{1.228142in}}%
\pgfpathlineto{\pgfqpoint{1.632439in}{1.226584in}}%
\pgfpathlineto{\pgfqpoint{1.737218in}{1.221629in}}%
\pgfpathlineto{\pgfqpoint{1.857360in}{1.213205in}}%
\pgfpathlineto{\pgfqpoint{1.994359in}{1.201080in}}%
\pgfpathlineto{\pgfqpoint{2.148923in}{1.184997in}}%
\pgfpathlineto{\pgfqpoint{2.319611in}{1.164727in}}%
\pgfpathlineto{\pgfqpoint{2.455516in}{1.146721in}}%
\pgfpathlineto{\pgfqpoint{2.501725in}{1.140189in}}%
\pgfpathlineto{\pgfqpoint{2.501725in}{1.140189in}}%
\pgfusepath{stroke}%
\end{pgfscope}%
\begin{pgfscope}%
\pgfpathrectangle{\pgfqpoint{0.562500in}{0.275000in}}{\pgfqpoint{3.487500in}{1.925000in}}%
\pgfusepath{clip}%
\pgfsetrectcap%
\pgfsetroundjoin%
\pgfsetlinewidth{1.505625pt}%
\definecolor{currentstroke}{rgb}{0.172549,0.627451,0.172549}%
\pgfsetstrokecolor{currentstroke}%
\pgfsetdash{}{0pt}%
\pgfpathmoveto{\pgfqpoint{3.891477in}{2.112500in}}%
\pgfpathlineto{\pgfqpoint{3.425176in}{2.071661in}}%
\pgfpathlineto{\pgfqpoint{3.203906in}{2.034784in}}%
\pgfpathlineto{\pgfqpoint{3.082401in}{2.000315in}}%
\pgfpathlineto{\pgfqpoint{3.014881in}{1.967535in}}%
\pgfpathlineto{\pgfqpoint{2.977831in}{1.936024in}}%
\pgfpathlineto{\pgfqpoint{2.960451in}{1.905538in}}%
\pgfpathlineto{\pgfqpoint{2.955053in}{1.875901in}}%
\pgfpathlineto{\pgfqpoint{2.957880in}{1.847005in}}%
\pgfpathlineto{\pgfqpoint{2.966493in}{1.818773in}}%
\pgfpathlineto{\pgfqpoint{2.978799in}{1.791142in}}%
\pgfpathlineto{\pgfqpoint{2.993505in}{1.764066in}}%
\pgfpathlineto{\pgfqpoint{3.010005in}{1.737518in}}%
\pgfpathlineto{\pgfqpoint{3.046163in}{1.685908in}}%
\pgfpathlineto{\pgfqpoint{3.102510in}{1.611927in}}%
\pgfpathlineto{\pgfqpoint{3.274511in}{1.392032in}}%
\pgfpathlineto{\pgfqpoint{3.331256in}{1.314558in}}%
\pgfpathlineto{\pgfqpoint{3.381306in}{1.242659in}}%
\pgfpathlineto{\pgfqpoint{3.413908in}{1.192152in}}%
\pgfpathlineto{\pgfqpoint{3.442136in}{1.144428in}}%
\pgfpathlineto{\pgfqpoint{3.466448in}{1.099435in}}%
\pgfpathlineto{\pgfqpoint{3.487533in}{1.057154in}}%
\pgfpathlineto{\pgfqpoint{3.505072in}{1.017438in}}%
\pgfpathlineto{\pgfqpoint{3.519075in}{0.980161in}}%
\pgfpathlineto{\pgfqpoint{3.529586in}{0.945235in}}%
\pgfpathlineto{\pgfqpoint{3.536612in}{0.912597in}}%
\pgfpathlineto{\pgfqpoint{3.540112in}{0.882202in}}%
\pgfpathlineto{\pgfqpoint{3.540208in}{0.853966in}}%
\pgfpathlineto{\pgfqpoint{3.537160in}{0.827806in}}%
\pgfpathlineto{\pgfqpoint{3.533427in}{0.811485in}}%
\pgfpathlineto{\pgfqpoint{3.528308in}{0.796039in}}%
\pgfpathlineto{\pgfqpoint{3.521738in}{0.781451in}}%
\pgfpathlineto{\pgfqpoint{3.513614in}{0.767709in}}%
\pgfpathlineto{\pgfqpoint{3.503797in}{0.754801in}}%
\pgfpathlineto{\pgfqpoint{3.492263in}{0.742712in}}%
\pgfpathlineto{\pgfqpoint{3.479122in}{0.731429in}}%
\pgfpathlineto{\pgfqpoint{3.464436in}{0.720945in}}%
\pgfpathlineto{\pgfqpoint{3.439560in}{0.706697in}}%
\pgfpathlineto{\pgfqpoint{3.411212in}{0.694199in}}%
\pgfpathlineto{\pgfqpoint{3.379171in}{0.683424in}}%
\pgfpathlineto{\pgfqpoint{3.343044in}{0.674351in}}%
\pgfpathlineto{\pgfqpoint{3.302267in}{0.666956in}}%
\pgfpathlineto{\pgfqpoint{3.256278in}{0.661228in}}%
\pgfpathlineto{\pgfqpoint{3.205073in}{0.657199in}}%
\pgfpathlineto{\pgfqpoint{3.148079in}{0.654901in}}%
\pgfpathlineto{\pgfqpoint{3.084629in}{0.654376in}}%
\pgfpathlineto{\pgfqpoint{3.014046in}{0.655675in}}%
\pgfpathlineto{\pgfqpoint{2.907646in}{0.660364in}}%
\pgfpathlineto{\pgfqpoint{2.785675in}{0.668608in}}%
\pgfpathlineto{\pgfqpoint{2.646457in}{0.680639in}}%
\pgfpathlineto{\pgfqpoint{2.489124in}{0.696722in}}%
\pgfpathlineto{\pgfqpoint{2.314887in}{0.717110in}}%
\pgfpathlineto{\pgfqpoint{2.175816in}{0.735285in}}%
\pgfpathlineto{\pgfqpoint{2.033208in}{0.755900in}}%
\pgfpathlineto{\pgfqpoint{1.891579in}{0.778836in}}%
\pgfpathlineto{\pgfqpoint{1.800463in}{0.795274in}}%
\pgfpathlineto{\pgfqpoint{1.713721in}{0.812498in}}%
\pgfpathlineto{\pgfqpoint{1.632549in}{0.830385in}}%
\pgfpathlineto{\pgfqpoint{1.557887in}{0.848801in}}%
\pgfpathlineto{\pgfqpoint{1.490415in}{0.867602in}}%
\pgfpathlineto{\pgfqpoint{1.430586in}{0.886628in}}%
\pgfpathlineto{\pgfqpoint{1.378490in}{0.905720in}}%
\pgfpathlineto{\pgfqpoint{1.333394in}{0.924751in}}%
\pgfpathlineto{\pgfqpoint{1.294582in}{0.943615in}}%
\pgfpathlineto{\pgfqpoint{1.261464in}{0.962211in}}%
\pgfpathlineto{\pgfqpoint{1.233580in}{0.980450in}}%
\pgfpathlineto{\pgfqpoint{1.210593in}{0.998255in}}%
\pgfpathlineto{\pgfqpoint{1.192298in}{1.015556in}}%
\pgfpathlineto{\pgfqpoint{1.178514in}{1.032294in}}%
\pgfpathlineto{\pgfqpoint{1.168414in}{1.048426in}}%
\pgfpathlineto{\pgfqpoint{1.161438in}{1.063925in}}%
\pgfpathlineto{\pgfqpoint{1.157206in}{1.078770in}}%
\pgfpathlineto{\pgfqpoint{1.155441in}{1.092945in}}%
\pgfpathlineto{\pgfqpoint{1.155970in}{1.106437in}}%
\pgfpathlineto{\pgfqpoint{1.158720in}{1.119237in}}%
\pgfpathlineto{\pgfqpoint{1.163720in}{1.131340in}}%
\pgfpathlineto{\pgfqpoint{1.171038in}{1.142746in}}%
\pgfpathlineto{\pgfqpoint{1.180437in}{1.153448in}}%
\pgfpathlineto{\pgfqpoint{1.191784in}{1.163444in}}%
\pgfpathlineto{\pgfqpoint{1.205005in}{1.172732in}}%
\pgfpathlineto{\pgfqpoint{1.220059in}{1.181311in}}%
\pgfpathlineto{\pgfqpoint{1.246071in}{1.192849in}}%
\pgfpathlineto{\pgfqpoint{1.276308in}{1.202792in}}%
\pgfpathlineto{\pgfqpoint{1.311046in}{1.211142in}}%
\pgfpathlineto{\pgfqpoint{1.350720in}{1.217908in}}%
\pgfpathlineto{\pgfqpoint{1.395482in}{1.223074in}}%
\pgfpathlineto{\pgfqpoint{1.445636in}{1.226605in}}%
\pgfpathlineto{\pgfqpoint{1.501745in}{1.228467in}}%
\pgfpathlineto{\pgfqpoint{1.564396in}{1.228609in}}%
\pgfpathlineto{\pgfqpoint{1.634204in}{1.226975in}}%
\pgfpathlineto{\pgfqpoint{1.739526in}{1.221909in}}%
\pgfpathlineto{\pgfqpoint{1.860301in}{1.213364in}}%
\pgfpathlineto{\pgfqpoint{1.998074in}{1.201110in}}%
\pgfpathlineto{\pgfqpoint{2.153445in}{1.184876in}}%
\pgfpathlineto{\pgfqpoint{2.324847in}{1.164446in}}%
\pgfpathlineto{\pgfqpoint{2.461144in}{1.146317in}}%
\pgfpathlineto{\pgfqpoint{2.600579in}{1.125812in}}%
\pgfpathlineto{\pgfqpoint{2.738374in}{1.103092in}}%
\pgfpathlineto{\pgfqpoint{2.869536in}{1.078437in}}%
\pgfpathlineto{\pgfqpoint{2.951163in}{1.061102in}}%
\pgfpathlineto{\pgfqpoint{3.026865in}{1.043195in}}%
\pgfpathlineto{\pgfqpoint{3.095689in}{1.024863in}}%
\pgfpathlineto{\pgfqpoint{3.156971in}{1.006264in}}%
\pgfpathlineto{\pgfqpoint{3.211078in}{0.987537in}}%
\pgfpathlineto{\pgfqpoint{3.258489in}{0.968811in}}%
\pgfpathlineto{\pgfqpoint{3.299644in}{0.950199in}}%
\pgfpathlineto{\pgfqpoint{3.334941in}{0.931808in}}%
\pgfpathlineto{\pgfqpoint{3.364737in}{0.913730in}}%
\pgfpathlineto{\pgfqpoint{3.389345in}{0.896048in}}%
\pgfpathlineto{\pgfqpoint{3.409041in}{0.878836in}}%
\pgfpathlineto{\pgfqpoint{3.424140in}{0.862154in}}%
\pgfpathlineto{\pgfqpoint{3.435390in}{0.846054in}}%
\pgfpathlineto{\pgfqpoint{3.443399in}{0.830564in}}%
\pgfpathlineto{\pgfqpoint{3.448618in}{0.815708in}}%
\pgfpathlineto{\pgfqpoint{3.451377in}{0.801504in}}%
\pgfpathlineto{\pgfqpoint{3.451879in}{0.787966in}}%
\pgfpathlineto{\pgfqpoint{3.450201in}{0.775106in}}%
\pgfpathlineto{\pgfqpoint{3.446296in}{0.762929in}}%
\pgfpathlineto{\pgfqpoint{3.439990in}{0.751439in}}%
\pgfpathlineto{\pgfqpoint{3.431240in}{0.740637in}}%
\pgfpathlineto{\pgfqpoint{3.420471in}{0.730535in}}%
\pgfpathlineto{\pgfqpoint{3.407774in}{0.721135in}}%
\pgfpathlineto{\pgfqpoint{3.393202in}{0.712439in}}%
\pgfpathlineto{\pgfqpoint{3.367863in}{0.700717in}}%
\pgfpathlineto{\pgfqpoint{3.338288in}{0.690584in}}%
\pgfpathlineto{\pgfqpoint{3.304271in}{0.682040in}}%
\pgfpathlineto{\pgfqpoint{3.265458in}{0.675083in}}%
\pgfpathlineto{\pgfqpoint{3.221511in}{0.669716in}}%
\pgfpathlineto{\pgfqpoint{3.172277in}{0.665972in}}%
\pgfpathlineto{\pgfqpoint{3.117176in}{0.663888in}}%
\pgfpathlineto{\pgfqpoint{3.055599in}{0.663509in}}%
\pgfpathlineto{\pgfqpoint{2.986925in}{0.664896in}}%
\pgfpathlineto{\pgfqpoint{2.883232in}{0.669609in}}%
\pgfpathlineto{\pgfqpoint{2.764285in}{0.677775in}}%
\pgfpathlineto{\pgfqpoint{2.628552in}{0.689623in}}%
\pgfpathlineto{\pgfqpoint{2.475257in}{0.705403in}}%
\pgfpathlineto{\pgfqpoint{2.305615in}{0.725370in}}%
\pgfpathlineto{\pgfqpoint{2.170223in}{0.743151in}}%
\pgfpathlineto{\pgfqpoint{2.031305in}{0.763294in}}%
\pgfpathlineto{\pgfqpoint{1.893143in}{0.785694in}}%
\pgfpathlineto{\pgfqpoint{1.760805in}{0.810094in}}%
\pgfpathlineto{\pgfqpoint{1.678182in}{0.827274in}}%
\pgfpathlineto{\pgfqpoint{1.601309in}{0.845038in}}%
\pgfpathlineto{\pgfqpoint{1.530968in}{0.863254in}}%
\pgfpathlineto{\pgfqpoint{1.467664in}{0.881786in}}%
\pgfpathlineto{\pgfqpoint{1.411633in}{0.900493in}}%
\pgfpathlineto{\pgfqpoint{1.362998in}{0.919219in}}%
\pgfpathlineto{\pgfqpoint{1.321590in}{0.937814in}}%
\pgfpathlineto{\pgfqpoint{1.286265in}{0.956192in}}%
\pgfpathlineto{\pgfqpoint{1.256058in}{0.974280in}}%
\pgfpathlineto{\pgfqpoint{1.230258in}{0.992009in}}%
\pgfpathlineto{\pgfqpoint{1.208405in}{1.009313in}}%
\pgfpathlineto{\pgfqpoint{1.190295in}{1.026134in}}%
\pgfpathlineto{\pgfqpoint{1.175975in}{1.042417in}}%
\pgfpathlineto{\pgfqpoint{1.165746in}{1.058114in}}%
\pgfpathlineto{\pgfqpoint{1.159971in}{1.073180in}}%
\pgfpathlineto{\pgfqpoint{1.157337in}{1.087582in}}%
\pgfpathlineto{\pgfqpoint{1.157214in}{1.101307in}}%
\pgfpathlineto{\pgfqpoint{1.159370in}{1.114345in}}%
\pgfpathlineto{\pgfqpoint{1.163636in}{1.126689in}}%
\pgfpathlineto{\pgfqpoint{1.169906in}{1.138333in}}%
\pgfpathlineto{\pgfqpoint{1.178137in}{1.149274in}}%
\pgfpathlineto{\pgfqpoint{1.188348in}{1.159514in}}%
\pgfpathlineto{\pgfqpoint{1.200622in}{1.169054in}}%
\pgfpathlineto{\pgfqpoint{1.214928in}{1.177895in}}%
\pgfpathlineto{\pgfqpoint{1.231162in}{1.186033in}}%
\pgfpathlineto{\pgfqpoint{1.259094in}{1.196921in}}%
\pgfpathlineto{\pgfqpoint{1.291372in}{1.206216in}}%
\pgfpathlineto{\pgfqpoint{1.328157in}{1.213909in}}%
\pgfpathlineto{\pgfqpoint{1.369729in}{1.219990in}}%
\pgfpathlineto{\pgfqpoint{1.416489in}{1.224448in}}%
\pgfpathlineto{\pgfqpoint{1.468957in}{1.227271in}}%
\pgfpathlineto{\pgfqpoint{1.527361in}{1.228438in}}%
\pgfpathlineto{\pgfqpoint{1.592305in}{1.227882in}}%
\pgfpathlineto{\pgfqpoint{1.690951in}{1.224319in}}%
\pgfpathlineto{\pgfqpoint{1.805081in}{1.217343in}}%
\pgfpathlineto{\pgfqpoint{1.935791in}{1.206744in}}%
\pgfpathlineto{\pgfqpoint{2.083359in}{1.192303in}}%
\pgfpathlineto{\pgfqpoint{2.247244in}{1.173793in}}%
\pgfpathlineto{\pgfqpoint{2.426070in}{1.150961in}}%
\pgfpathlineto{\pgfqpoint{2.565790in}{1.130977in}}%
\pgfpathlineto{\pgfqpoint{2.704276in}{1.108772in}}%
\pgfpathlineto{\pgfqpoint{2.836826in}{1.084622in}}%
\pgfpathlineto{\pgfqpoint{2.919950in}{1.067600in}}%
\pgfpathlineto{\pgfqpoint{2.997756in}{1.049967in}}%
\pgfpathlineto{\pgfqpoint{3.069466in}{1.031844in}}%
\pgfpathlineto{\pgfqpoint{3.134464in}{1.013361in}}%
\pgfpathlineto{\pgfqpoint{3.192298in}{0.994658in}}%
\pgfpathlineto{\pgfqpoint{3.242682in}{0.975887in}}%
\pgfpathlineto{\pgfqpoint{3.285698in}{0.957204in}}%
\pgfpathlineto{\pgfqpoint{3.322234in}{0.938726in}}%
\pgfpathlineto{\pgfqpoint{3.353156in}{0.920539in}}%
\pgfpathlineto{\pgfqpoint{3.379155in}{0.902721in}}%
\pgfpathlineto{\pgfqpoint{3.400743in}{0.885341in}}%
\pgfpathlineto{\pgfqpoint{3.418259in}{0.868460in}}%
\pgfpathlineto{\pgfqpoint{3.431866in}{0.852128in}}%
\pgfpathlineto{\pgfqpoint{3.441555in}{0.836388in}}%
\pgfpathlineto{\pgfqpoint{3.447804in}{0.821276in}}%
\pgfpathlineto{\pgfqpoint{3.451236in}{0.806815in}}%
\pgfpathlineto{\pgfqpoint{3.452137in}{0.793019in}}%
\pgfpathlineto{\pgfqpoint{3.450723in}{0.779902in}}%
\pgfpathlineto{\pgfqpoint{3.447143in}{0.767471in}}%
\pgfpathlineto{\pgfqpoint{3.441474in}{0.755733in}}%
\pgfpathlineto{\pgfqpoint{3.433727in}{0.744690in}}%
\pgfpathlineto{\pgfqpoint{3.423852in}{0.734342in}}%
\pgfpathlineto{\pgfqpoint{3.411936in}{0.724691in}}%
\pgfpathlineto{\pgfqpoint{3.398078in}{0.715739in}}%
\pgfpathlineto{\pgfqpoint{3.382325in}{0.707488in}}%
\pgfpathlineto{\pgfqpoint{3.355175in}{0.696432in}}%
\pgfpathlineto{\pgfqpoint{3.323742in}{0.686965in}}%
\pgfpathlineto{\pgfqpoint{3.287833in}{0.679093in}}%
\pgfpathlineto{\pgfqpoint{3.247118in}{0.672820in}}%
\pgfpathlineto{\pgfqpoint{3.201125in}{0.668147in}}%
\pgfpathlineto{\pgfqpoint{3.149579in}{0.665092in}}%
\pgfpathlineto{\pgfqpoint{3.092108in}{0.663706in}}%
\pgfpathlineto{\pgfqpoint{3.027880in}{0.664045in}}%
\pgfpathlineto{\pgfqpoint{2.930460in}{0.667295in}}%
\pgfpathlineto{\pgfqpoint{2.818202in}{0.673912in}}%
\pgfpathlineto{\pgfqpoint{2.689882in}{0.684107in}}%
\pgfpathlineto{\pgfqpoint{2.544561in}{0.698120in}}%
\pgfpathlineto{\pgfqpoint{2.380325in}{0.716289in}}%
\pgfpathlineto{\pgfqpoint{2.199059in}{0.738910in}}%
\pgfpathlineto{\pgfqpoint{2.059509in}{0.758638in}}%
\pgfpathlineto{\pgfqpoint{1.922195in}{0.780529in}}%
\pgfpathlineto{\pgfqpoint{1.791115in}{0.804341in}}%
\pgfpathlineto{\pgfqpoint{1.708834in}{0.821142in}}%
\pgfpathlineto{\pgfqpoint{1.631577in}{0.838569in}}%
\pgfpathlineto{\pgfqpoint{1.559986in}{0.856514in}}%
\pgfpathlineto{\pgfqpoint{1.494559in}{0.874856in}}%
\pgfpathlineto{\pgfqpoint{1.435649in}{0.893467in}}%
\pgfpathlineto{\pgfqpoint{1.383465in}{0.912207in}}%
\pgfpathlineto{\pgfqpoint{1.338070in}{0.930926in}}%
\pgfpathlineto{\pgfqpoint{1.299386in}{0.949465in}}%
\pgfpathlineto{\pgfqpoint{1.267048in}{0.967681in}}%
\pgfpathlineto{\pgfqpoint{1.240141in}{0.985532in}}%
\pgfpathlineto{\pgfqpoint{1.218037in}{1.002953in}}%
\pgfpathlineto{\pgfqpoint{1.200226in}{1.019882in}}%
\pgfpathlineto{\pgfqpoint{1.186291in}{1.036269in}}%
\pgfpathlineto{\pgfqpoint{1.175908in}{1.052072in}}%
\pgfpathlineto{\pgfqpoint{1.168740in}{1.067260in}}%
\pgfpathlineto{\pgfqpoint{1.164448in}{1.081810in}}%
\pgfpathlineto{\pgfqpoint{1.162786in}{1.095702in}}%
\pgfpathlineto{\pgfqpoint{1.163558in}{1.108924in}}%
\pgfpathlineto{\pgfqpoint{1.166625in}{1.121465in}}%
\pgfpathlineto{\pgfqpoint{1.171915in}{1.133320in}}%
\pgfpathlineto{\pgfqpoint{1.179387in}{1.144486in}}%
\pgfpathlineto{\pgfqpoint{1.188896in}{1.154959in}}%
\pgfpathlineto{\pgfqpoint{1.200324in}{1.164734in}}%
\pgfpathlineto{\pgfqpoint{1.213593in}{1.173809in}}%
\pgfpathlineto{\pgfqpoint{1.228663in}{1.182181in}}%
\pgfpathlineto{\pgfqpoint{1.254662in}{1.193422in}}%
\pgfpathlineto{\pgfqpoint{1.284903in}{1.203082in}}%
\pgfpathlineto{\pgfqpoint{1.319760in}{1.211171in}}%
\pgfpathlineto{\pgfqpoint{1.359770in}{1.217699in}}%
\pgfpathlineto{\pgfqpoint{1.404939in}{1.222634in}}%
\pgfpathlineto{\pgfqpoint{1.455583in}{1.225939in}}%
\pgfpathlineto{\pgfqpoint{1.512249in}{1.227575in}}%
\pgfpathlineto{\pgfqpoint{1.575516in}{1.227494in}}%
\pgfpathlineto{\pgfqpoint{1.645995in}{1.225635in}}%
\pgfpathlineto{\pgfqpoint{1.752304in}{1.220266in}}%
\pgfpathlineto{\pgfqpoint{1.874197in}{1.211411in}}%
\pgfpathlineto{\pgfqpoint{2.013151in}{1.198835in}}%
\pgfpathlineto{\pgfqpoint{2.169618in}{1.182267in}}%
\pgfpathlineto{\pgfqpoint{2.341721in}{1.161509in}}%
\pgfpathlineto{\pgfqpoint{2.478222in}{1.143138in}}%
\pgfpathlineto{\pgfqpoint{2.617302in}{1.122413in}}%
\pgfpathlineto{\pgfqpoint{2.754110in}{1.099512in}}%
\pgfpathlineto{\pgfqpoint{2.883964in}{1.074711in}}%
\pgfpathlineto{\pgfqpoint{2.964508in}{1.057303in}}%
\pgfpathlineto{\pgfqpoint{3.038697in}{1.039355in}}%
\pgfpathlineto{\pgfqpoint{3.105351in}{1.021022in}}%
\pgfpathlineto{\pgfqpoint{3.164829in}{1.002449in}}%
\pgfpathlineto{\pgfqpoint{3.217614in}{0.983771in}}%
\pgfpathlineto{\pgfqpoint{3.264137in}{0.965108in}}%
\pgfpathlineto{\pgfqpoint{3.304777in}{0.946574in}}%
\pgfpathlineto{\pgfqpoint{3.339859in}{0.928268in}}%
\pgfpathlineto{\pgfqpoint{3.369655in}{0.910280in}}%
\pgfpathlineto{\pgfqpoint{3.394387in}{0.892689in}}%
\pgfpathlineto{\pgfqpoint{3.414222in}{0.875562in}}%
\pgfpathlineto{\pgfqpoint{3.429274in}{0.858956in}}%
\pgfpathlineto{\pgfqpoint{3.439811in}{0.842925in}}%
\pgfpathlineto{\pgfqpoint{3.446823in}{0.827519in}}%
\pgfpathlineto{\pgfqpoint{3.450941in}{0.812760in}}%
\pgfpathlineto{\pgfqpoint{3.452626in}{0.798664in}}%
\pgfpathlineto{\pgfqpoint{3.452213in}{0.785242in}}%
\pgfpathlineto{\pgfqpoint{3.449911in}{0.772503in}}%
\pgfpathlineto{\pgfqpoint{3.445800in}{0.760452in}}%
\pgfpathlineto{\pgfqpoint{3.439830in}{0.749092in}}%
\pgfpathlineto{\pgfqpoint{3.431826in}{0.738422in}}%
\pgfpathlineto{\pgfqpoint{3.421484in}{0.728437in}}%
\pgfpathlineto{\pgfqpoint{3.408493in}{0.719134in}}%
\pgfpathlineto{\pgfqpoint{3.393419in}{0.710530in}}%
\pgfpathlineto{\pgfqpoint{3.367231in}{0.698948in}}%
\pgfpathlineto{\pgfqpoint{3.336731in}{0.688960in}}%
\pgfpathlineto{\pgfqpoint{3.301793in}{0.680572in}}%
\pgfpathlineto{\pgfqpoint{3.262175in}{0.673792in}}%
\pgfpathlineto{\pgfqpoint{3.217524in}{0.668630in}}%
\pgfpathlineto{\pgfqpoint{3.167372in}{0.665097in}}%
\pgfpathlineto{\pgfqpoint{3.111531in}{0.663214in}}%
\pgfpathlineto{\pgfqpoint{3.049349in}{0.663036in}}%
\pgfpathlineto{\pgfqpoint{2.979845in}{0.664635in}}%
\pgfpathlineto{\pgfqpoint{2.874419in}{0.669663in}}%
\pgfpathlineto{\pgfqpoint{2.753169in}{0.678186in}}%
\pgfpathlineto{\pgfqpoint{2.615265in}{0.690421in}}%
\pgfpathlineto{\pgfqpoint{2.460450in}{0.706607in}}%
\pgfpathlineto{\pgfqpoint{2.288814in}{0.727015in}}%
\pgfpathlineto{\pgfqpoint{2.151209in}{0.745208in}}%
\pgfpathlineto{\pgfqpoint{2.011584in}{0.765753in}}%
\pgfpathlineto{\pgfqpoint{1.874717in}{0.788450in}}%
\pgfpathlineto{\pgfqpoint{1.744746in}{0.813032in}}%
\pgfpathlineto{\pgfqpoint{1.663683in}{0.830306in}}%
\pgfpathlineto{\pgfqpoint{1.588130in}{0.848156in}}%
\pgfpathlineto{\pgfqpoint{1.518838in}{0.866453in}}%
\pgfpathlineto{\pgfqpoint{1.456432in}{0.885056in}}%
\pgfpathlineto{\pgfqpoint{1.401408in}{0.903810in}}%
\pgfpathlineto{\pgfqpoint{1.353928in}{0.922556in}}%
\pgfpathlineto{\pgfqpoint{1.313209in}{0.941173in}}%
\pgfpathlineto{\pgfqpoint{1.278536in}{0.959560in}}%
\pgfpathlineto{\pgfqpoint{1.249307in}{0.977626in}}%
\pgfpathlineto{\pgfqpoint{1.225026in}{0.995292in}}%
\pgfpathlineto{\pgfqpoint{1.205295in}{1.012490in}}%
\pgfpathlineto{\pgfqpoint{1.189819in}{1.029161in}}%
\pgfpathlineto{\pgfqpoint{1.178198in}{1.045258in}}%
\pgfpathlineto{\pgfqpoint{1.169875in}{1.060748in}}%
\pgfpathlineto{\pgfqpoint{1.164495in}{1.075605in}}%
\pgfpathlineto{\pgfqpoint{1.161778in}{1.089811in}}%
\pgfpathlineto{\pgfqpoint{1.161527in}{1.103348in}}%
\pgfpathlineto{\pgfqpoint{1.163619in}{1.116206in}}%
\pgfpathlineto{\pgfqpoint{1.168010in}{1.128378in}}%
\pgfpathlineto{\pgfqpoint{1.174631in}{1.139861in}}%
\pgfpathlineto{\pgfqpoint{1.183326in}{1.150649in}}%
\pgfpathlineto{\pgfqpoint{1.193973in}{1.160738in}}%
\pgfpathlineto{\pgfqpoint{1.206488in}{1.170127in}}%
\pgfpathlineto{\pgfqpoint{1.220826in}{1.178812in}}%
\pgfpathlineto{\pgfqpoint{1.245746in}{1.190521in}}%
\pgfpathlineto{\pgfqpoint{1.274896in}{1.200649in}}%
\pgfpathlineto{\pgfqpoint{1.308607in}{1.209202in}}%
\pgfpathlineto{\pgfqpoint{1.347338in}{1.216190in}}%
\pgfpathlineto{\pgfqpoint{1.391092in}{1.221584in}}%
\pgfpathlineto{\pgfqpoint{1.440208in}{1.225355in}}%
\pgfpathlineto{\pgfqpoint{1.495191in}{1.227467in}}%
\pgfpathlineto{\pgfqpoint{1.556595in}{1.227872in}}%
\pgfpathlineto{\pgfqpoint{1.625015in}{1.226516in}}%
\pgfpathlineto{\pgfqpoint{1.728275in}{1.221849in}}%
\pgfpathlineto{\pgfqpoint{1.846806in}{1.213743in}}%
\pgfpathlineto{\pgfqpoint{1.982172in}{1.201968in}}%
\pgfpathlineto{\pgfqpoint{2.135107in}{1.186252in}}%
\pgfpathlineto{\pgfqpoint{2.304277in}{1.166381in}}%
\pgfpathlineto{\pgfqpoint{2.439496in}{1.148672in}}%
\pgfpathlineto{\pgfqpoint{2.578274in}{1.128588in}}%
\pgfpathlineto{\pgfqpoint{2.716250in}{1.106265in}}%
\pgfpathlineto{\pgfqpoint{2.848711in}{1.081948in}}%
\pgfpathlineto{\pgfqpoint{2.931437in}{1.064794in}}%
\pgfpathlineto{\pgfqpoint{3.008129in}{1.047021in}}%
\pgfpathlineto{\pgfqpoint{3.078287in}{1.028793in}}%
\pgfpathlineto{\pgfqpoint{3.141618in}{1.010269in}}%
\pgfpathlineto{\pgfqpoint{3.197987in}{0.991593in}}%
\pgfpathlineto{\pgfqpoint{3.247413in}{0.972897in}}%
\pgfpathlineto{\pgfqpoint{3.290068in}{0.954299in}}%
\pgfpathlineto{\pgfqpoint{3.326283in}{0.935907in}}%
\pgfpathlineto{\pgfqpoint{3.356539in}{0.917812in}}%
\pgfpathlineto{\pgfqpoint{3.381478in}{0.900095in}}%
\pgfpathlineto{\pgfqpoint{3.401831in}{0.882826in}}%
\pgfpathlineto{\pgfqpoint{3.417691in}{0.866082in}}%
\pgfpathlineto{\pgfqpoint{3.429697in}{0.849903in}}%
\pgfpathlineto{\pgfqpoint{3.438579in}{0.834316in}}%
\pgfpathlineto{\pgfqpoint{3.444882in}{0.819346in}}%
\pgfpathlineto{\pgfqpoint{3.448962in}{0.805010in}}%
\pgfpathlineto{\pgfqpoint{3.450988in}{0.791325in}}%
\pgfpathlineto{\pgfqpoint{3.450942in}{0.778301in}}%
\pgfpathlineto{\pgfqpoint{3.448620in}{0.765946in}}%
\pgfpathlineto{\pgfqpoint{3.443627in}{0.754264in}}%
\pgfpathlineto{\pgfqpoint{3.435542in}{0.743257in}}%
\pgfpathlineto{\pgfqpoint{3.425204in}{0.732948in}}%
\pgfpathlineto{\pgfqpoint{3.412930in}{0.723342in}}%
\pgfpathlineto{\pgfqpoint{3.398772in}{0.714443in}}%
\pgfpathlineto{\pgfqpoint{3.382755in}{0.706251in}}%
\pgfpathlineto{\pgfqpoint{3.355227in}{0.695289in}}%
\pgfpathlineto{\pgfqpoint{3.323369in}{0.685920in}}%
\pgfpathlineto{\pgfqpoint{3.286905in}{0.678142in}}%
\pgfpathlineto{\pgfqpoint{3.245458in}{0.671954in}}%
\pgfpathlineto{\pgfqpoint{3.198898in}{0.667376in}}%
\pgfpathlineto{\pgfqpoint{3.146761in}{0.664441in}}%
\pgfpathlineto{\pgfqpoint{3.088455in}{0.663190in}}%
\pgfpathlineto{\pgfqpoint{3.023374in}{0.663675in}}%
\pgfpathlineto{\pgfqpoint{2.924980in}{0.667133in}}%
\pgfpathlineto{\pgfqpoint{2.811911in}{0.673971in}}%
\pgfpathlineto{\pgfqpoint{2.682595in}{0.684402in}}%
\pgfpathlineto{\pgfqpoint{2.535808in}{0.698671in}}%
\pgfpathlineto{\pgfqpoint{2.371870in}{0.717047in}}%
\pgfpathlineto{\pgfqpoint{2.193955in}{0.739681in}}%
\pgfpathlineto{\pgfqpoint{2.055242in}{0.759440in}}%
\pgfpathlineto{\pgfqpoint{2.055242in}{0.759440in}}%
\pgfusepath{stroke}%
\end{pgfscope}%
\begin{pgfscope}%
\pgfpathrectangle{\pgfqpoint{0.562500in}{0.275000in}}{\pgfqpoint{3.487500in}{1.925000in}}%
\pgfusepath{clip}%
\pgfsetrectcap%
\pgfsetroundjoin%
\pgfsetlinewidth{1.505625pt}%
\definecolor{currentstroke}{rgb}{0.839216,0.152941,0.156863}%
\pgfsetstrokecolor{currentstroke}%
\pgfsetdash{}{0pt}%
\pgfpathmoveto{\pgfqpoint{1.909943in}{2.112500in}}%
\pgfpathlineto{\pgfqpoint{2.066924in}{2.086542in}}%
\pgfpathlineto{\pgfqpoint{2.207506in}{2.059874in}}%
\pgfpathlineto{\pgfqpoint{2.331111in}{2.032664in}}%
\pgfpathlineto{\pgfqpoint{2.438155in}{2.005075in}}%
\pgfpathlineto{\pgfqpoint{2.529953in}{1.977263in}}%
\pgfpathlineto{\pgfqpoint{2.608531in}{1.949374in}}%
\pgfpathlineto{\pgfqpoint{2.675901in}{1.921512in}}%
\pgfpathlineto{\pgfqpoint{2.733926in}{1.893761in}}%
\pgfpathlineto{\pgfqpoint{2.784312in}{1.866202in}}%
\pgfpathlineto{\pgfqpoint{2.828633in}{1.838904in}}%
\pgfpathlineto{\pgfqpoint{2.868131in}{1.811901in}}%
\pgfpathlineto{\pgfqpoint{2.903693in}{1.785221in}}%
\pgfpathlineto{\pgfqpoint{2.936093in}{1.758888in}}%
\pgfpathlineto{\pgfqpoint{2.993920in}{1.707347in}}%
\pgfpathlineto{\pgfqpoint{3.044856in}{1.657350in}}%
\pgfpathlineto{\pgfqpoint{3.090714in}{1.608918in}}%
\pgfpathlineto{\pgfqpoint{3.152914in}{1.539235in}}%
\pgfpathlineto{\pgfqpoint{3.208460in}{1.473017in}}%
\pgfpathlineto{\pgfqpoint{3.274360in}{1.389982in}}%
\pgfpathlineto{\pgfqpoint{3.318304in}{1.331500in}}%
\pgfpathlineto{\pgfqpoint{3.357562in}{1.276119in}}%
\pgfpathlineto{\pgfqpoint{3.403605in}{1.206968in}}%
\pgfpathlineto{\pgfqpoint{3.433639in}{1.158486in}}%
\pgfpathlineto{\pgfqpoint{3.459648in}{1.112743in}}%
\pgfpathlineto{\pgfqpoint{3.481728in}{1.069641in}}%
\pgfpathlineto{\pgfqpoint{3.500190in}{1.029130in}}%
\pgfpathlineto{\pgfqpoint{3.515137in}{0.991139in}}%
\pgfpathlineto{\pgfqpoint{3.526647in}{0.955542in}}%
\pgfpathlineto{\pgfqpoint{3.534824in}{0.922240in}}%
\pgfpathlineto{\pgfqpoint{3.539661in}{0.891155in}}%
\pgfpathlineto{\pgfqpoint{3.540975in}{0.871638in}}%
\pgfpathlineto{\pgfqpoint{3.540692in}{0.853074in}}%
\pgfpathlineto{\pgfqpoint{3.538831in}{0.835443in}}%
\pgfpathlineto{\pgfqpoint{3.535490in}{0.818720in}}%
\pgfpathlineto{\pgfqpoint{3.530727in}{0.802885in}}%
\pgfpathlineto{\pgfqpoint{3.524569in}{0.787920in}}%
\pgfpathlineto{\pgfqpoint{3.517010in}{0.773808in}}%
\pgfpathlineto{\pgfqpoint{3.508014in}{0.760534in}}%
\pgfpathlineto{\pgfqpoint{3.497513in}{0.748084in}}%
\pgfpathlineto{\pgfqpoint{3.485405in}{0.736445in}}%
\pgfpathlineto{\pgfqpoint{3.471578in}{0.725606in}}%
\pgfpathlineto{\pgfqpoint{3.456047in}{0.715553in}}%
\pgfpathlineto{\pgfqpoint{3.438861in}{0.706280in}}%
\pgfpathlineto{\pgfqpoint{3.410009in}{0.693821in}}%
\pgfpathlineto{\pgfqpoint{3.377397in}{0.683089in}}%
\pgfpathlineto{\pgfqpoint{3.340815in}{0.674070in}}%
\pgfpathlineto{\pgfqpoint{3.299904in}{0.666752in}}%
\pgfpathlineto{\pgfqpoint{3.254160in}{0.661128in}}%
\pgfpathlineto{\pgfqpoint{3.202933in}{0.657192in}}%
\pgfpathlineto{\pgfqpoint{3.145818in}{0.654955in}}%
\pgfpathlineto{\pgfqpoint{3.082482in}{0.654479in}}%
\pgfpathlineto{\pgfqpoint{3.011906in}{0.655827in}}%
\pgfpathlineto{\pgfqpoint{2.905071in}{0.660591in}}%
\pgfpathlineto{\pgfqpoint{2.782260in}{0.668926in}}%
\pgfpathlineto{\pgfqpoint{2.642415in}{0.681056in}}%
\pgfpathlineto{\pgfqpoint{2.485009in}{0.697231in}}%
\pgfpathlineto{\pgfqpoint{2.309638in}{0.717743in}}%
\pgfpathlineto{\pgfqpoint{2.168984in}{0.736077in}}%
\pgfpathlineto{\pgfqpoint{2.026264in}{0.756829in}}%
\pgfpathlineto{\pgfqpoint{1.886244in}{0.779809in}}%
\pgfpathlineto{\pgfqpoint{1.753116in}{0.804754in}}%
\pgfpathlineto{\pgfqpoint{1.670000in}{0.822309in}}%
\pgfpathlineto{\pgfqpoint{1.592493in}{0.840467in}}%
\pgfpathlineto{\pgfqpoint{1.521410in}{0.859093in}}%
\pgfpathlineto{\pgfqpoint{1.457452in}{0.878036in}}%
\pgfpathlineto{\pgfqpoint{1.401205in}{0.897130in}}%
\pgfpathlineto{\pgfqpoint{1.352753in}{0.916214in}}%
\pgfpathlineto{\pgfqpoint{1.311190in}{0.935172in}}%
\pgfpathlineto{\pgfqpoint{1.275852in}{0.953896in}}%
\pgfpathlineto{\pgfqpoint{1.246156in}{0.972291in}}%
\pgfpathlineto{\pgfqpoint{1.221594in}{0.990276in}}%
\pgfpathlineto{\pgfqpoint{1.201730in}{1.007780in}}%
\pgfpathlineto{\pgfqpoint{1.186135in}{1.024744in}}%
\pgfpathlineto{\pgfqpoint{1.174244in}{1.041125in}}%
\pgfpathlineto{\pgfqpoint{1.165632in}{1.056892in}}%
\pgfpathlineto{\pgfqpoint{1.159965in}{1.072017in}}%
\pgfpathlineto{\pgfqpoint{1.156995in}{1.086481in}}%
\pgfpathlineto{\pgfqpoint{1.156559in}{1.100267in}}%
\pgfpathlineto{\pgfqpoint{1.158572in}{1.113367in}}%
\pgfpathlineto{\pgfqpoint{1.162898in}{1.125774in}}%
\pgfpathlineto{\pgfqpoint{1.169334in}{1.137480in}}%
\pgfpathlineto{\pgfqpoint{1.177728in}{1.148484in}}%
\pgfpathlineto{\pgfqpoint{1.187973in}{1.158781in}}%
\pgfpathlineto{\pgfqpoint{1.200013in}{1.168371in}}%
\pgfpathlineto{\pgfqpoint{1.213839in}{1.177254in}}%
\pgfpathlineto{\pgfqpoint{1.229493in}{1.185432in}}%
\pgfpathlineto{\pgfqpoint{1.256611in}{1.196383in}}%
\pgfpathlineto{\pgfqpoint{1.288526in}{1.205771in}}%
\pgfpathlineto{\pgfqpoint{1.325176in}{1.213580in}}%
\pgfpathlineto{\pgfqpoint{1.366654in}{1.219788in}}%
\pgfpathlineto{\pgfqpoint{1.413303in}{1.224373in}}%
\pgfpathlineto{\pgfqpoint{1.465543in}{1.227305in}}%
\pgfpathlineto{\pgfqpoint{1.523875in}{1.228544in}}%
\pgfpathlineto{\pgfqpoint{1.588877in}{1.228047in}}%
\pgfpathlineto{\pgfqpoint{1.687071in}{1.224586in}}%
\pgfpathlineto{\pgfqpoint{1.799934in}{1.217775in}}%
\pgfpathlineto{\pgfqpoint{1.929314in}{1.207361in}}%
\pgfpathlineto{\pgfqpoint{2.076254in}{1.193087in}}%
\pgfpathlineto{\pgfqpoint{2.240034in}{1.174730in}}%
\pgfpathlineto{\pgfqpoint{2.418135in}{1.152107in}}%
\pgfpathlineto{\pgfqpoint{2.557038in}{1.132338in}}%
\pgfpathlineto{\pgfqpoint{2.695475in}{1.110307in}}%
\pgfpathlineto{\pgfqpoint{2.828974in}{1.086244in}}%
\pgfpathlineto{\pgfqpoint{2.913082in}{1.069232in}}%
\pgfpathlineto{\pgfqpoint{2.991807in}{1.051585in}}%
\pgfpathlineto{\pgfqpoint{3.063931in}{1.033442in}}%
\pgfpathlineto{\pgfqpoint{3.128763in}{1.014956in}}%
\pgfpathlineto{\pgfqpoint{3.186431in}{0.996273in}}%
\pgfpathlineto{\pgfqpoint{3.237177in}{0.977530in}}%
\pgfpathlineto{\pgfqpoint{3.281289in}{0.958851in}}%
\pgfpathlineto{\pgfqpoint{3.319105in}{0.940349in}}%
\pgfpathlineto{\pgfqpoint{3.351009in}{0.922124in}}%
\pgfpathlineto{\pgfqpoint{3.377434in}{0.904266in}}%
\pgfpathlineto{\pgfqpoint{3.398859in}{0.886851in}}%
\pgfpathlineto{\pgfqpoint{3.415804in}{0.869948in}}%
\pgfpathlineto{\pgfqpoint{3.428905in}{0.853600in}}%
\pgfpathlineto{\pgfqpoint{3.438705in}{0.837840in}}%
\pgfpathlineto{\pgfqpoint{3.445612in}{0.822697in}}%
\pgfpathlineto{\pgfqpoint{3.449897in}{0.808193in}}%
\pgfpathlineto{\pgfqpoint{3.451694in}{0.794347in}}%
\pgfpathlineto{\pgfqpoint{3.451002in}{0.781172in}}%
\pgfpathlineto{\pgfqpoint{3.447691in}{0.768674in}}%
\pgfpathlineto{\pgfqpoint{3.441975in}{0.756864in}}%
\pgfpathlineto{\pgfqpoint{3.434167in}{0.745750in}}%
\pgfpathlineto{\pgfqpoint{3.424388in}{0.735335in}}%
\pgfpathlineto{\pgfqpoint{3.412724in}{0.725621in}}%
\pgfpathlineto{\pgfqpoint{3.399220in}{0.716610in}}%
\pgfpathlineto{\pgfqpoint{3.383884in}{0.708304in}}%
\pgfpathlineto{\pgfqpoint{3.357371in}{0.697162in}}%
\pgfpathlineto{\pgfqpoint{3.326404in}{0.687595in}}%
\pgfpathlineto{\pgfqpoint{3.290610in}{0.679595in}}%
\pgfpathlineto{\pgfqpoint{3.249993in}{0.673182in}}%
\pgfpathlineto{\pgfqpoint{3.204256in}{0.668379in}}%
\pgfpathlineto{\pgfqpoint{3.152972in}{0.665217in}}%
\pgfpathlineto{\pgfqpoint{3.095645in}{0.663734in}}%
\pgfpathlineto{\pgfqpoint{3.031716in}{0.663980in}}%
\pgfpathlineto{\pgfqpoint{2.935121in}{0.667095in}}%
\pgfpathlineto{\pgfqpoint{2.824016in}{0.673548in}}%
\pgfpathlineto{\pgfqpoint{2.696735in}{0.683553in}}%
\pgfpathlineto{\pgfqpoint{2.552009in}{0.697379in}}%
\pgfpathlineto{\pgfqpoint{2.390178in}{0.715265in}}%
\pgfpathlineto{\pgfqpoint{2.213749in}{0.737393in}}%
\pgfpathlineto{\pgfqpoint{2.075618in}{0.756786in}}%
\pgfpathlineto{\pgfqpoint{1.936899in}{0.778471in}}%
\pgfpathlineto{\pgfqpoint{1.802205in}{0.802234in}}%
\pgfpathlineto{\pgfqpoint{1.717211in}{0.819082in}}%
\pgfpathlineto{\pgfqpoint{1.637756in}{0.836614in}}%
\pgfpathlineto{\pgfqpoint{1.564536in}{0.854666in}}%
\pgfpathlineto{\pgfqpoint{1.498005in}{0.873079in}}%
\pgfpathlineto{\pgfqpoint{1.438435in}{0.891704in}}%
\pgfpathlineto{\pgfqpoint{1.385913in}{0.910406in}}%
\pgfpathlineto{\pgfqpoint{1.340344in}{0.929061in}}%
\pgfpathlineto{\pgfqpoint{1.301448in}{0.947558in}}%
\pgfpathlineto{\pgfqpoint{1.268762in}{0.965799in}}%
\pgfpathlineto{\pgfqpoint{1.241641in}{0.983698in}}%
\pgfpathlineto{\pgfqpoint{1.219287in}{1.001179in}}%
\pgfpathlineto{\pgfqpoint{1.201564in}{1.018160in}}%
\pgfpathlineto{\pgfqpoint{1.187904in}{1.034593in}}%
\pgfpathlineto{\pgfqpoint{1.177547in}{1.050449in}}%
\pgfpathlineto{\pgfqpoint{1.169922in}{1.065702in}}%
\pgfpathlineto{\pgfqpoint{1.164651in}{1.080329in}}%
\pgfpathlineto{\pgfqpoint{1.161547in}{1.094314in}}%
\pgfpathlineto{\pgfqpoint{1.160616in}{1.107641in}}%
\pgfpathlineto{\pgfqpoint{1.162057in}{1.120303in}}%
\pgfpathlineto{\pgfqpoint{1.166259in}{1.132292in}}%
\pgfpathlineto{\pgfqpoint{1.173537in}{1.143605in}}%
\pgfpathlineto{\pgfqpoint{1.182995in}{1.154219in}}%
\pgfpathlineto{\pgfqpoint{1.194402in}{1.164129in}}%
\pgfpathlineto{\pgfqpoint{1.207695in}{1.173333in}}%
\pgfpathlineto{\pgfqpoint{1.222841in}{1.181832in}}%
\pgfpathlineto{\pgfqpoint{1.249040in}{1.193253in}}%
\pgfpathlineto{\pgfqpoint{1.279532in}{1.203085in}}%
\pgfpathlineto{\pgfqpoint{1.314592in}{1.211330in}}%
\pgfpathlineto{\pgfqpoint{1.354573in}{1.217991in}}%
\pgfpathlineto{\pgfqpoint{1.399573in}{1.223047in}}%
\pgfpathlineto{\pgfqpoint{1.450031in}{1.226469in}}%
\pgfpathlineto{\pgfqpoint{1.506482in}{1.228218in}}%
\pgfpathlineto{\pgfqpoint{1.569497in}{1.228247in}}%
\pgfpathlineto{\pgfqpoint{1.639686in}{1.226496in}}%
\pgfpathlineto{\pgfqpoint{1.745555in}{1.221272in}}%
\pgfpathlineto{\pgfqpoint{1.866967in}{1.212565in}}%
\pgfpathlineto{\pgfqpoint{2.005433in}{1.200141in}}%
\pgfpathlineto{\pgfqpoint{2.161472in}{1.183728in}}%
\pgfpathlineto{\pgfqpoint{2.333317in}{1.163122in}}%
\pgfpathlineto{\pgfqpoint{2.469854in}{1.144857in}}%
\pgfpathlineto{\pgfqpoint{2.609149in}{1.124230in}}%
\pgfpathlineto{\pgfqpoint{2.746460in}{1.101408in}}%
\pgfpathlineto{\pgfqpoint{2.877100in}{1.076661in}}%
\pgfpathlineto{\pgfqpoint{2.958232in}{1.059277in}}%
\pgfpathlineto{\pgfqpoint{3.032792in}{1.041337in}}%
\pgfpathlineto{\pgfqpoint{3.099994in}{1.022993in}}%
\pgfpathlineto{\pgfqpoint{3.160193in}{1.004396in}}%
\pgfpathlineto{\pgfqpoint{3.213740in}{0.985684in}}%
\pgfpathlineto{\pgfqpoint{3.260966in}{0.966986in}}%
\pgfpathlineto{\pgfqpoint{3.302186in}{0.948415in}}%
\pgfpathlineto{\pgfqpoint{3.337696in}{0.930072in}}%
\pgfpathlineto{\pgfqpoint{3.367775in}{0.912048in}}%
\pgfpathlineto{\pgfqpoint{3.392683in}{0.894420in}}%
\pgfpathlineto{\pgfqpoint{3.412664in}{0.877253in}}%
\pgfpathlineto{\pgfqpoint{3.427944in}{0.860600in}}%
\pgfpathlineto{\pgfqpoint{3.438807in}{0.844510in}}%
\pgfpathlineto{\pgfqpoint{3.446003in}{0.829045in}}%
\pgfpathlineto{\pgfqpoint{3.450230in}{0.814225in}}%
\pgfpathlineto{\pgfqpoint{3.452028in}{0.800065in}}%
\pgfpathlineto{\pgfqpoint{3.451795in}{0.786578in}}%
\pgfpathlineto{\pgfqpoint{3.449785in}{0.773772in}}%
\pgfpathlineto{\pgfqpoint{3.446107in}{0.761651in}}%
\pgfpathlineto{\pgfqpoint{3.440728in}{0.750219in}}%
\pgfpathlineto{\pgfqpoint{3.433471in}{0.739474in}}%
\pgfpathlineto{\pgfqpoint{3.424014in}{0.729412in}}%
\pgfpathlineto{\pgfqpoint{3.411893in}{0.720025in}}%
\pgfpathlineto{\pgfqpoint{3.397019in}{0.711318in}}%
\pgfpathlineto{\pgfqpoint{3.380199in}{0.703317in}}%
\pgfpathlineto{\pgfqpoint{3.351383in}{0.692645in}}%
\pgfpathlineto{\pgfqpoint{3.318186in}{0.683572in}}%
\pgfpathlineto{\pgfqpoint{3.280421in}{0.676107in}}%
\pgfpathlineto{\pgfqpoint{3.237794in}{0.670261in}}%
\pgfpathlineto{\pgfqpoint{3.189904in}{0.666045in}}%
\pgfpathlineto{\pgfqpoint{3.136287in}{0.663475in}}%
\pgfpathlineto{\pgfqpoint{3.076800in}{0.662579in}}%
\pgfpathlineto{\pgfqpoint{3.010477in}{0.663427in}}%
\pgfpathlineto{\pgfqpoint{2.909713in}{0.667409in}}%
\pgfpathlineto{\pgfqpoint{2.793310in}{0.674834in}}%
\pgfpathlineto{\pgfqpoint{2.660287in}{0.685913in}}%
\pgfpathlineto{\pgfqpoint{2.510414in}{0.700869in}}%
\pgfpathlineto{\pgfqpoint{2.344206in}{0.719936in}}%
\pgfpathlineto{\pgfqpoint{2.163199in}{0.743379in}}%
\pgfpathlineto{\pgfqpoint{2.023092in}{0.763785in}}%
\pgfpathlineto{\pgfqpoint{1.885194in}{0.786368in}}%
\pgfpathlineto{\pgfqpoint{1.753999in}{0.810848in}}%
\pgfpathlineto{\pgfqpoint{1.672115in}{0.828060in}}%
\pgfpathlineto{\pgfqpoint{1.595762in}{0.845854in}}%
\pgfpathlineto{\pgfqpoint{1.525675in}{0.864108in}}%
\pgfpathlineto{\pgfqpoint{1.462433in}{0.882688in}}%
\pgfpathlineto{\pgfqpoint{1.406454in}{0.901447in}}%
\pgfpathlineto{\pgfqpoint{1.357992in}{0.920231in}}%
\pgfpathlineto{\pgfqpoint{1.316679in}{0.938891in}}%
\pgfpathlineto{\pgfqpoint{1.281584in}{0.957324in}}%
\pgfpathlineto{\pgfqpoint{1.251924in}{0.975444in}}%
\pgfpathlineto{\pgfqpoint{1.227081in}{0.993176in}}%
\pgfpathlineto{\pgfqpoint{1.206602in}{1.010451in}}%
\pgfpathlineto{\pgfqpoint{1.190196in}{1.027212in}}%
\pgfpathlineto{\pgfqpoint{1.177739in}{1.043408in}}%
\pgfpathlineto{\pgfqpoint{1.169107in}{1.059001in}}%
\pgfpathlineto{\pgfqpoint{1.163578in}{1.073957in}}%
\pgfpathlineto{\pgfqpoint{1.160771in}{1.088259in}}%
\pgfpathlineto{\pgfqpoint{1.160426in}{1.101893in}}%
\pgfpathlineto{\pgfqpoint{1.162357in}{1.114845in}}%
\pgfpathlineto{\pgfqpoint{1.166443in}{1.127109in}}%
\pgfpathlineto{\pgfqpoint{1.172639in}{1.138680in}}%
\pgfpathlineto{\pgfqpoint{1.180967in}{1.149556in}}%
\pgfpathlineto{\pgfqpoint{1.191401in}{1.159736in}}%
\pgfpathlineto{\pgfqpoint{1.203811in}{1.169219in}}%
\pgfpathlineto{\pgfqpoint{1.218130in}{1.178001in}}%
\pgfpathlineto{\pgfqpoint{1.234322in}{1.186079in}}%
\pgfpathlineto{\pgfqpoint{1.262109in}{1.196875in}}%
\pgfpathlineto{\pgfqpoint{1.294191in}{1.206081in}}%
\pgfpathlineto{\pgfqpoint{1.330808in}{1.213693in}}%
\pgfpathlineto{\pgfqpoint{1.372348in}{1.219712in}}%
\pgfpathlineto{\pgfqpoint{1.419275in}{1.224134in}}%
\pgfpathlineto{\pgfqpoint{1.471704in}{1.226925in}}%
\pgfpathlineto{\pgfqpoint{1.530282in}{1.228035in}}%
\pgfpathlineto{\pgfqpoint{1.595744in}{1.227410in}}%
\pgfpathlineto{\pgfqpoint{1.694898in}{1.223761in}}%
\pgfpathlineto{\pgfqpoint{1.808948in}{1.216723in}}%
\pgfpathlineto{\pgfqpoint{1.939178in}{1.206081in}}%
\pgfpathlineto{\pgfqpoint{2.086716in}{1.191585in}}%
\pgfpathlineto{\pgfqpoint{2.252357in}{1.173018in}}%
\pgfpathlineto{\pgfqpoint{2.385610in}{1.156285in}}%
\pgfpathlineto{\pgfqpoint{2.523006in}{1.137129in}}%
\pgfpathlineto{\pgfqpoint{2.660961in}{1.115652in}}%
\pgfpathlineto{\pgfqpoint{2.795575in}{1.092061in}}%
\pgfpathlineto{\pgfqpoint{2.881388in}{1.075311in}}%
\pgfpathlineto{\pgfqpoint{2.962513in}{1.057880in}}%
\pgfpathlineto{\pgfqpoint{3.037556in}{1.039908in}}%
\pgfpathlineto{\pgfqpoint{3.105202in}{1.021553in}}%
\pgfpathlineto{\pgfqpoint{3.165353in}{1.002953in}}%
\pgfpathlineto{\pgfqpoint{3.218446in}{0.984244in}}%
\pgfpathlineto{\pgfqpoint{3.264888in}{0.965553in}}%
\pgfpathlineto{\pgfqpoint{3.305072in}{0.946995in}}%
\pgfpathlineto{\pgfqpoint{3.339377in}{0.928673in}}%
\pgfpathlineto{\pgfqpoint{3.368162in}{0.910680in}}%
\pgfpathlineto{\pgfqpoint{3.391775in}{0.893097in}}%
\pgfpathlineto{\pgfqpoint{3.410545in}{0.875993in}}%
\pgfpathlineto{\pgfqpoint{3.424988in}{0.859432in}}%
\pgfpathlineto{\pgfqpoint{3.435790in}{0.843453in}}%
\pgfpathlineto{\pgfqpoint{3.443477in}{0.828084in}}%
\pgfpathlineto{\pgfqpoint{3.448448in}{0.813349in}}%
\pgfpathlineto{\pgfqpoint{3.450977in}{0.799266in}}%
\pgfpathlineto{\pgfqpoint{3.451208in}{0.785850in}}%
\pgfpathlineto{\pgfqpoint{3.449160in}{0.773111in}}%
\pgfpathlineto{\pgfqpoint{3.444726in}{0.761055in}}%
\pgfpathlineto{\pgfqpoint{3.437775in}{0.749684in}}%
\pgfpathlineto{\pgfqpoint{3.428694in}{0.739008in}}%
\pgfpathlineto{\pgfqpoint{3.417643in}{0.729032in}}%
\pgfpathlineto{\pgfqpoint{3.404695in}{0.719758in}}%
\pgfpathlineto{\pgfqpoint{3.389892in}{0.711188in}}%
\pgfpathlineto{\pgfqpoint{3.364221in}{0.699653in}}%
\pgfpathlineto{\pgfqpoint{3.334285in}{0.689705in}}%
\pgfpathlineto{\pgfqpoint{3.299818in}{0.681339in}}%
\pgfpathlineto{\pgfqpoint{3.260415in}{0.674551in}}%
\pgfpathlineto{\pgfqpoint{3.215969in}{0.669360in}}%
\pgfpathlineto{\pgfqpoint{3.166123in}{0.665796in}}%
\pgfpathlineto{\pgfqpoint{3.110335in}{0.663897in}}%
\pgfpathlineto{\pgfqpoint{3.048030in}{0.663711in}}%
\pgfpathlineto{\pgfqpoint{2.978603in}{0.665295in}}%
\pgfpathlineto{\pgfqpoint{2.873847in}{0.670284in}}%
\pgfpathlineto{\pgfqpoint{2.753690in}{0.678738in}}%
\pgfpathlineto{\pgfqpoint{2.616586in}{0.690886in}}%
\pgfpathlineto{\pgfqpoint{2.461916in}{0.707002in}}%
\pgfpathlineto{\pgfqpoint{2.291215in}{0.727297in}}%
\pgfpathlineto{\pgfqpoint{2.155333in}{0.745320in}}%
\pgfpathlineto{\pgfqpoint{2.016192in}{0.765716in}}%
\pgfpathlineto{\pgfqpoint{1.878595in}{0.788326in}}%
\pgfpathlineto{\pgfqpoint{1.747323in}{0.812883in}}%
\pgfpathlineto{\pgfqpoint{1.665491in}{0.830161in}}%
\pgfpathlineto{\pgfqpoint{1.589605in}{0.848016in}}%
\pgfpathlineto{\pgfqpoint{1.521082in}{0.866290in}}%
\pgfpathlineto{\pgfqpoint{1.460009in}{0.884831in}}%
\pgfpathlineto{\pgfqpoint{1.405816in}{0.903506in}}%
\pgfpathlineto{\pgfqpoint{1.358001in}{0.922193in}}%
\pgfpathlineto{\pgfqpoint{1.316132in}{0.940778in}}%
\pgfpathlineto{\pgfqpoint{1.279847in}{0.959158in}}%
\pgfpathlineto{\pgfqpoint{1.248852in}{0.977242in}}%
\pgfpathlineto{\pgfqpoint{1.222923in}{0.994948in}}%
\pgfpathlineto{\pgfqpoint{1.201906in}{1.012204in}}%
\pgfpathlineto{\pgfqpoint{1.185715in}{1.028948in}}%
\pgfpathlineto{\pgfqpoint{1.174235in}{1.045128in}}%
\pgfpathlineto{\pgfqpoint{1.166459in}{1.060689in}}%
\pgfpathlineto{\pgfqpoint{1.161677in}{1.075606in}}%
\pgfpathlineto{\pgfqpoint{1.159436in}{1.089864in}}%
\pgfpathlineto{\pgfqpoint{1.159401in}{1.103448in}}%
\pgfpathlineto{\pgfqpoint{1.161357in}{1.116350in}}%
\pgfpathlineto{\pgfqpoint{1.165207in}{1.128562in}}%
\pgfpathlineto{\pgfqpoint{1.170974in}{1.140082in}}%
\pgfpathlineto{\pgfqpoint{1.178797in}{1.150911in}}%
\pgfpathlineto{\pgfqpoint{1.188937in}{1.161050in}}%
\pgfpathlineto{\pgfqpoint{1.201582in}{1.170503in}}%
\pgfpathlineto{\pgfqpoint{1.216223in}{1.179254in}}%
\pgfpathlineto{\pgfqpoint{1.232769in}{1.187300in}}%
\pgfpathlineto{\pgfqpoint{1.261151in}{1.198042in}}%
\pgfpathlineto{\pgfqpoint{1.293886in}{1.207188in}}%
\pgfpathlineto{\pgfqpoint{1.331168in}{1.214732in}}%
\pgfpathlineto{\pgfqpoint{1.373311in}{1.220664in}}%
\pgfpathlineto{\pgfqpoint{1.420746in}{1.224977in}}%
\pgfpathlineto{\pgfqpoint{1.473789in}{1.227658in}}%
\pgfpathlineto{\pgfqpoint{1.532810in}{1.228658in}}%
\pgfpathlineto{\pgfqpoint{1.598752in}{1.227916in}}%
\pgfpathlineto{\pgfqpoint{1.698829in}{1.224091in}}%
\pgfpathlineto{\pgfqpoint{1.814174in}{1.216851in}}%
\pgfpathlineto{\pgfqpoint{1.945860in}{1.205985in}}%
\pgfpathlineto{\pgfqpoint{2.094501in}{1.191254in}}%
\pgfpathlineto{\pgfqpoint{2.260473in}{1.172389in}}%
\pgfpathlineto{\pgfqpoint{2.395493in}{1.155372in}}%
\pgfpathlineto{\pgfqpoint{2.534538in}{1.135952in}}%
\pgfpathlineto{\pgfqpoint{2.672861in}{1.114279in}}%
\pgfpathlineto{\pgfqpoint{2.806251in}{1.090578in}}%
\pgfpathlineto{\pgfqpoint{2.890597in}{1.073794in}}%
\pgfpathlineto{\pgfqpoint{2.970144in}{1.056344in}}%
\pgfpathlineto{\pgfqpoint{3.044038in}{1.038350in}}%
\pgfpathlineto{\pgfqpoint{3.111529in}{1.019946in}}%
\pgfpathlineto{\pgfqpoint{3.171973in}{1.001283in}}%
\pgfpathlineto{\pgfqpoint{3.224837in}{0.982523in}}%
\pgfpathlineto{\pgfqpoint{3.270362in}{0.963811in}}%
\pgfpathlineto{\pgfqpoint{3.309381in}{0.945255in}}%
\pgfpathlineto{\pgfqpoint{3.342510in}{0.926954in}}%
\pgfpathlineto{\pgfqpoint{3.370291in}{0.908998in}}%
\pgfpathlineto{\pgfqpoint{3.393186in}{0.891464in}}%
\pgfpathlineto{\pgfqpoint{3.411578in}{0.874418in}}%
\pgfpathlineto{\pgfqpoint{3.425846in}{0.857914in}}%
\pgfpathlineto{\pgfqpoint{3.436540in}{0.841993in}}%
\pgfpathlineto{\pgfqpoint{3.444063in}{0.826685in}}%
\pgfpathlineto{\pgfqpoint{3.448731in}{0.812014in}}%
\pgfpathlineto{\pgfqpoint{3.450778in}{0.798000in}}%
\pgfpathlineto{\pgfqpoint{3.450360in}{0.784657in}}%
\pgfpathlineto{\pgfqpoint{3.447552in}{0.771994in}}%
\pgfpathlineto{\pgfqpoint{3.442443in}{0.760018in}}%
\pgfpathlineto{\pgfqpoint{3.435206in}{0.748734in}}%
\pgfpathlineto{\pgfqpoint{3.425990in}{0.738146in}}%
\pgfpathlineto{\pgfqpoint{3.414895in}{0.728259in}}%
\pgfpathlineto{\pgfqpoint{3.401982in}{0.719073in}}%
\pgfpathlineto{\pgfqpoint{3.387266in}{0.710590in}}%
\pgfpathlineto{\pgfqpoint{3.361743in}{0.699182in}}%
\pgfpathlineto{\pgfqpoint{3.331820in}{0.689346in}}%
\pgfpathlineto{\pgfqpoint{3.297026in}{0.681068in}}%
\pgfpathlineto{\pgfqpoint{3.257338in}{0.674367in}}%
\pgfpathlineto{\pgfqpoint{3.212603in}{0.669271in}}%
\pgfpathlineto{\pgfqpoint{3.162405in}{0.665807in}}%
\pgfpathlineto{\pgfqpoint{3.106259in}{0.664014in}}%
\pgfpathlineto{\pgfqpoint{3.043619in}{0.663938in}}%
\pgfpathlineto{\pgfqpoint{2.973875in}{0.665634in}}%
\pgfpathlineto{\pgfqpoint{2.868653in}{0.670769in}}%
\pgfpathlineto{\pgfqpoint{2.747946in}{0.679355in}}%
\pgfpathlineto{\pgfqpoint{2.610152in}{0.691657in}}%
\pgfpathlineto{\pgfqpoint{2.454676in}{0.707952in}}%
\pgfpathlineto{\pgfqpoint{2.283504in}{0.728419in}}%
\pgfpathlineto{\pgfqpoint{2.147512in}{0.746563in}}%
\pgfpathlineto{\pgfqpoint{2.147512in}{0.746563in}}%
\pgfusepath{stroke}%
\end{pgfscope}%
\begin{pgfscope}%
\pgfpathrectangle{\pgfqpoint{0.562500in}{0.275000in}}{\pgfqpoint{3.487500in}{1.925000in}}%
\pgfusepath{clip}%
\pgfsetrectcap%
\pgfsetroundjoin%
\pgfsetlinewidth{1.505625pt}%
\definecolor{currentstroke}{rgb}{0.580392,0.403922,0.741176}%
\pgfsetstrokecolor{currentstroke}%
\pgfsetdash{}{0pt}%
\pgfpathmoveto{\pgfqpoint{1.117330in}{1.529167in}}%
\pgfpathlineto{\pgfqpoint{1.309244in}{1.516074in}}%
\pgfpathlineto{\pgfqpoint{1.519296in}{1.498581in}}%
\pgfpathlineto{\pgfqpoint{1.750176in}{1.476466in}}%
\pgfpathlineto{\pgfqpoint{1.999677in}{1.449565in}}%
\pgfpathlineto{\pgfqpoint{2.173013in}{1.428917in}}%
\pgfpathlineto{\pgfqpoint{2.347627in}{1.406101in}}%
\pgfpathlineto{\pgfqpoint{2.517389in}{1.381311in}}%
\pgfpathlineto{\pgfqpoint{2.676934in}{1.354806in}}%
\pgfpathlineto{\pgfqpoint{2.822138in}{1.326879in}}%
\pgfpathlineto{\pgfqpoint{2.888416in}{1.312484in}}%
\pgfpathlineto{\pgfqpoint{2.950127in}{1.297858in}}%
\pgfpathlineto{\pgfqpoint{3.007107in}{1.283048in}}%
\pgfpathlineto{\pgfqpoint{3.059267in}{1.268101in}}%
\pgfpathlineto{\pgfqpoint{3.106655in}{1.253071in}}%
\pgfpathlineto{\pgfqpoint{3.149682in}{1.238008in}}%
\pgfpathlineto{\pgfqpoint{3.188720in}{1.222945in}}%
\pgfpathlineto{\pgfqpoint{3.224101in}{1.207912in}}%
\pgfpathlineto{\pgfqpoint{3.256130in}{1.192938in}}%
\pgfpathlineto{\pgfqpoint{3.285086in}{1.178051in}}%
\pgfpathlineto{\pgfqpoint{3.311225in}{1.163279in}}%
\pgfpathlineto{\pgfqpoint{3.334776in}{1.148646in}}%
\pgfpathlineto{\pgfqpoint{3.375014in}{1.119889in}}%
\pgfpathlineto{\pgfqpoint{3.407956in}{1.091892in}}%
\pgfpathlineto{\pgfqpoint{3.435123in}{1.064716in}}%
\pgfpathlineto{\pgfqpoint{3.457587in}{1.038416in}}%
\pgfpathlineto{\pgfqpoint{3.476062in}{1.013037in}}%
\pgfpathlineto{\pgfqpoint{3.491175in}{0.988609in}}%
\pgfpathlineto{\pgfqpoint{3.503473in}{0.965128in}}%
\pgfpathlineto{\pgfqpoint{3.513338in}{0.942587in}}%
\pgfpathlineto{\pgfqpoint{3.521030in}{0.920981in}}%
\pgfpathlineto{\pgfqpoint{3.526688in}{0.900308in}}%
\pgfpathlineto{\pgfqpoint{3.530331in}{0.880566in}}%
\pgfpathlineto{\pgfqpoint{3.532042in}{0.861746in}}%
\pgfpathlineto{\pgfqpoint{3.532026in}{0.843830in}}%
\pgfpathlineto{\pgfqpoint{3.530377in}{0.826804in}}%
\pgfpathlineto{\pgfqpoint{3.527151in}{0.810650in}}%
\pgfpathlineto{\pgfqpoint{3.522373in}{0.795357in}}%
\pgfpathlineto{\pgfqpoint{3.516030in}{0.780909in}}%
\pgfpathlineto{\pgfqpoint{3.508078in}{0.767297in}}%
\pgfpathlineto{\pgfqpoint{3.498431in}{0.754505in}}%
\pgfpathlineto{\pgfqpoint{3.487110in}{0.742522in}}%
\pgfpathlineto{\pgfqpoint{3.474204in}{0.731338in}}%
\pgfpathlineto{\pgfqpoint{3.459767in}{0.720945in}}%
\pgfpathlineto{\pgfqpoint{3.435272in}{0.706820in}}%
\pgfpathlineto{\pgfqpoint{3.407271in}{0.694430in}}%
\pgfpathlineto{\pgfqpoint{3.375484in}{0.683748in}}%
\pgfpathlineto{\pgfqpoint{3.339442in}{0.674749in}}%
\pgfpathlineto{\pgfqpoint{3.298504in}{0.667410in}}%
\pgfpathlineto{\pgfqpoint{3.252538in}{0.661743in}}%
\pgfpathlineto{\pgfqpoint{3.201321in}{0.657775in}}%
\pgfpathlineto{\pgfqpoint{3.144274in}{0.655536in}}%
\pgfpathlineto{\pgfqpoint{3.080781in}{0.655067in}}%
\pgfpathlineto{\pgfqpoint{3.010187in}{0.656421in}}%
\pgfpathlineto{\pgfqpoint{2.903811in}{0.661177in}}%
\pgfpathlineto{\pgfqpoint{2.781827in}{0.669481in}}%
\pgfpathlineto{\pgfqpoint{2.642535in}{0.681563in}}%
\pgfpathlineto{\pgfqpoint{2.485145in}{0.697703in}}%
\pgfpathlineto{\pgfqpoint{2.311061in}{0.718125in}}%
\pgfpathlineto{\pgfqpoint{2.172080in}{0.736326in}}%
\pgfpathlineto{\pgfqpoint{2.029695in}{0.756967in}}%
\pgfpathlineto{\pgfqpoint{1.888619in}{0.779896in}}%
\pgfpathlineto{\pgfqpoint{1.753815in}{0.804848in}}%
\pgfpathlineto{\pgfqpoint{1.670032in}{0.822427in}}%
\pgfpathlineto{\pgfqpoint{1.592807in}{0.840618in}}%
\pgfpathlineto{\pgfqpoint{1.522519in}{0.859250in}}%
\pgfpathlineto{\pgfqpoint{1.459325in}{0.878162in}}%
\pgfpathlineto{\pgfqpoint{1.403252in}{0.897206in}}%
\pgfpathlineto{\pgfqpoint{1.354206in}{0.916245in}}%
\pgfpathlineto{\pgfqpoint{1.311965in}{0.935161in}}%
\pgfpathlineto{\pgfqpoint{1.276183in}{0.953846in}}%
\pgfpathlineto{\pgfqpoint{1.246387in}{0.972208in}}%
\pgfpathlineto{\pgfqpoint{1.221980in}{0.990167in}}%
\pgfpathlineto{\pgfqpoint{1.202247in}{1.007659in}}%
\pgfpathlineto{\pgfqpoint{1.186968in}{1.024608in}}%
\pgfpathlineto{\pgfqpoint{1.175613in}{1.040966in}}%
\pgfpathlineto{\pgfqpoint{1.167387in}{1.056711in}}%
\pgfpathlineto{\pgfqpoint{1.161687in}{1.071824in}}%
\pgfpathlineto{\pgfqpoint{1.158106in}{1.086287in}}%
\pgfpathlineto{\pgfqpoint{1.156431in}{1.100090in}}%
\pgfpathlineto{\pgfqpoint{1.156643in}{1.113221in}}%
\pgfpathlineto{\pgfqpoint{1.158919in}{1.125675in}}%
\pgfpathlineto{\pgfqpoint{1.163631in}{1.137448in}}%
\pgfpathlineto{\pgfqpoint{1.171345in}{1.148541in}}%
\pgfpathlineto{\pgfqpoint{1.181896in}{1.158942in}}%
\pgfpathlineto{\pgfqpoint{1.194378in}{1.168630in}}%
\pgfpathlineto{\pgfqpoint{1.208736in}{1.177606in}}%
\pgfpathlineto{\pgfqpoint{1.224944in}{1.185868in}}%
\pgfpathlineto{\pgfqpoint{1.252736in}{1.196924in}}%
\pgfpathlineto{\pgfqpoint{1.284819in}{1.206374in}}%
\pgfpathlineto{\pgfqpoint{1.321445in}{1.214220in}}%
\pgfpathlineto{\pgfqpoint{1.363002in}{1.220462in}}%
\pgfpathlineto{\pgfqpoint{1.409702in}{1.225090in}}%
\pgfpathlineto{\pgfqpoint{1.461915in}{1.228070in}}%
\pgfpathlineto{\pgfqpoint{1.520284in}{1.229361in}}%
\pgfpathlineto{\pgfqpoint{1.585448in}{1.228910in}}%
\pgfpathlineto{\pgfqpoint{1.684005in}{1.225491in}}%
\pgfpathlineto{\pgfqpoint{1.797279in}{1.218684in}}%
\pgfpathlineto{\pgfqpoint{1.926762in}{1.208273in}}%
\pgfpathlineto{\pgfqpoint{2.073779in}{1.194018in}}%
\pgfpathlineto{\pgfqpoint{2.238016in}{1.175654in}}%
\pgfpathlineto{\pgfqpoint{2.416277in}{1.152994in}}%
\pgfpathlineto{\pgfqpoint{2.555294in}{1.133207in}}%
\pgfpathlineto{\pgfqpoint{2.694411in}{1.111152in}}%
\pgfpathlineto{\pgfqpoint{2.828537in}{1.087021in}}%
\pgfpathlineto{\pgfqpoint{2.912747in}{1.069950in}}%
\pgfpathlineto{\pgfqpoint{2.991438in}{1.052260in}}%
\pgfpathlineto{\pgfqpoint{3.063761in}{1.034096in}}%
\pgfpathlineto{\pgfqpoint{3.129145in}{1.015595in}}%
\pgfpathlineto{\pgfqpoint{3.187295in}{0.996893in}}%
\pgfpathlineto{\pgfqpoint{3.238194in}{0.978119in}}%
\pgfpathlineto{\pgfqpoint{3.282103in}{0.959401in}}%
\pgfpathlineto{\pgfqpoint{3.319443in}{0.940865in}}%
\pgfpathlineto{\pgfqpoint{3.350439in}{0.922643in}}%
\pgfpathlineto{\pgfqpoint{3.376161in}{0.904800in}}%
\pgfpathlineto{\pgfqpoint{3.397577in}{0.887390in}}%
\pgfpathlineto{\pgfqpoint{3.415390in}{0.870461in}}%
\pgfpathlineto{\pgfqpoint{3.430040in}{0.854057in}}%
\pgfpathlineto{\pgfqpoint{3.441706in}{0.838219in}}%
\pgfpathlineto{\pgfqpoint{3.450300in}{0.822981in}}%
\pgfpathlineto{\pgfqpoint{3.455472in}{0.808373in}}%
\pgfpathlineto{\pgfqpoint{3.456734in}{0.794423in}}%
\pgfpathlineto{\pgfqpoint{3.455151in}{0.781158in}}%
\pgfpathlineto{\pgfqpoint{3.451324in}{0.768587in}}%
\pgfpathlineto{\pgfqpoint{3.445418in}{0.756716in}}%
\pgfpathlineto{\pgfqpoint{3.437551in}{0.745550in}}%
\pgfpathlineto{\pgfqpoint{3.427794in}{0.735089in}}%
\pgfpathlineto{\pgfqpoint{3.416168in}{0.725335in}}%
\pgfpathlineto{\pgfqpoint{3.402650in}{0.716286in}}%
\pgfpathlineto{\pgfqpoint{3.387165in}{0.707939in}}%
\pgfpathlineto{\pgfqpoint{3.360124in}{0.696729in}}%
\pgfpathlineto{\pgfqpoint{3.328653in}{0.687101in}}%
\pgfpathlineto{\pgfqpoint{3.292646in}{0.679064in}}%
\pgfpathlineto{\pgfqpoint{3.251883in}{0.672636in}}%
\pgfpathlineto{\pgfqpoint{3.206044in}{0.667835in}}%
\pgfpathlineto{\pgfqpoint{3.154710in}{0.664686in}}%
\pgfpathlineto{\pgfqpoint{3.097364in}{0.663217in}}%
\pgfpathlineto{\pgfqpoint{3.033390in}{0.663460in}}%
\pgfpathlineto{\pgfqpoint{2.937324in}{0.666458in}}%
\pgfpathlineto{\pgfqpoint{2.826488in}{0.672776in}}%
\pgfpathlineto{\pgfqpoint{2.698243in}{0.682784in}}%
\pgfpathlineto{\pgfqpoint{2.551993in}{0.696748in}}%
\pgfpathlineto{\pgfqpoint{2.389187in}{0.714829in}}%
\pgfpathlineto{\pgfqpoint{2.213319in}{0.737083in}}%
\pgfpathlineto{\pgfqpoint{2.076102in}{0.756488in}}%
\pgfpathlineto{\pgfqpoint{1.937709in}{0.778159in}}%
\pgfpathlineto{\pgfqpoint{1.803011in}{0.801971in}}%
\pgfpathlineto{\pgfqpoint{1.717993in}{0.818847in}}%
\pgfpathlineto{\pgfqpoint{1.638176in}{0.836360in}}%
\pgfpathlineto{\pgfqpoint{1.564453in}{0.854374in}}%
\pgfpathlineto{\pgfqpoint{1.497464in}{0.872754in}}%
\pgfpathlineto{\pgfqpoint{1.437590in}{0.891370in}}%
\pgfpathlineto{\pgfqpoint{1.384960in}{0.910092in}}%
\pgfpathlineto{\pgfqpoint{1.339444in}{0.928793in}}%
\pgfpathlineto{\pgfqpoint{1.300670in}{0.947350in}}%
\pgfpathlineto{\pgfqpoint{1.268406in}{0.965620in}}%
\pgfpathlineto{\pgfqpoint{1.241738in}{0.983520in}}%
\pgfpathlineto{\pgfqpoint{1.219599in}{1.000996in}}%
\pgfpathlineto{\pgfqpoint{1.201187in}{1.018001in}}%
\pgfpathlineto{\pgfqpoint{1.185969in}{1.034492in}}%
\pgfpathlineto{\pgfqpoint{1.173676in}{1.050427in}}%
\pgfpathlineto{\pgfqpoint{1.164307in}{1.065773in}}%
\pgfpathlineto{\pgfqpoint{1.158127in}{1.080497in}}%
\pgfpathlineto{\pgfqpoint{1.155668in}{1.094574in}}%
\pgfpathlineto{\pgfqpoint{1.156761in}{1.107976in}}%
\pgfpathlineto{\pgfqpoint{1.160168in}{1.120684in}}%
\pgfpathlineto{\pgfqpoint{1.165696in}{1.132693in}}%
\pgfpathlineto{\pgfqpoint{1.173213in}{1.143998in}}%
\pgfpathlineto{\pgfqpoint{1.182635in}{1.154598in}}%
\pgfpathlineto{\pgfqpoint{1.193920in}{1.164490in}}%
\pgfpathlineto{\pgfqpoint{1.207075in}{1.173677in}}%
\pgfpathlineto{\pgfqpoint{1.222152in}{1.182160in}}%
\pgfpathlineto{\pgfqpoint{1.239248in}{1.189943in}}%
\pgfpathlineto{\pgfqpoint{1.268658in}{1.200304in}}%
\pgfpathlineto{\pgfqpoint{1.302542in}{1.209080in}}%
\pgfpathlineto{\pgfqpoint{1.341076in}{1.216258in}}%
\pgfpathlineto{\pgfqpoint{1.384533in}{1.221818in}}%
\pgfpathlineto{\pgfqpoint{1.433281in}{1.225739in}}%
\pgfpathlineto{\pgfqpoint{1.487786in}{1.227992in}}%
\pgfpathlineto{\pgfqpoint{1.548609in}{1.228544in}}%
\pgfpathlineto{\pgfqpoint{1.616374in}{1.227360in}}%
\pgfpathlineto{\pgfqpoint{1.718201in}{1.222994in}}%
\pgfpathlineto{\pgfqpoint{1.835767in}{1.215171in}}%
\pgfpathlineto{\pgfqpoint{1.970896in}{1.203610in}}%
\pgfpathlineto{\pgfqpoint{2.123556in}{1.188091in}}%
\pgfpathlineto{\pgfqpoint{2.291848in}{1.168455in}}%
\pgfpathlineto{\pgfqpoint{2.426138in}{1.150963in}}%
\pgfpathlineto{\pgfqpoint{2.564819in}{1.131074in}}%
\pgfpathlineto{\pgfqpoint{2.703520in}{1.108848in}}%
\pgfpathlineto{\pgfqpoint{2.836801in}{1.084620in}}%
\pgfpathlineto{\pgfqpoint{2.920425in}{1.067548in}}%
\pgfpathlineto{\pgfqpoint{2.998629in}{1.049882in}}%
\pgfpathlineto{\pgfqpoint{3.070597in}{1.031744in}}%
\pgfpathlineto{\pgfqpoint{3.135730in}{1.013263in}}%
\pgfpathlineto{\pgfqpoint{3.193647in}{0.994569in}}%
\pgfpathlineto{\pgfqpoint{3.244183in}{0.975798in}}%
\pgfpathlineto{\pgfqpoint{3.287389in}{0.957087in}}%
\pgfpathlineto{\pgfqpoint{3.323688in}{0.938590in}}%
\pgfpathlineto{\pgfqpoint{3.354122in}{0.920401in}}%
\pgfpathlineto{\pgfqpoint{3.379638in}{0.902587in}}%
\pgfpathlineto{\pgfqpoint{3.400954in}{0.885210in}}%
\pgfpathlineto{\pgfqpoint{3.418552in}{0.868324in}}%
\pgfpathlineto{\pgfqpoint{3.432688in}{0.851979in}}%
\pgfpathlineto{\pgfqpoint{3.443383in}{0.836216in}}%
\pgfpathlineto{\pgfqpoint{3.450426in}{0.821070in}}%
\pgfpathlineto{\pgfqpoint{3.453749in}{0.806572in}}%
\pgfpathlineto{\pgfqpoint{3.454381in}{0.792747in}}%
\pgfpathlineto{\pgfqpoint{3.452650in}{0.779605in}}%
\pgfpathlineto{\pgfqpoint{3.448753in}{0.767155in}}%
\pgfpathlineto{\pgfqpoint{3.442827in}{0.755402in}}%
\pgfpathlineto{\pgfqpoint{3.434953in}{0.744349in}}%
\pgfpathlineto{\pgfqpoint{3.425152in}{0.733999in}}%
\pgfpathlineto{\pgfqpoint{3.413388in}{0.724349in}}%
\pgfpathlineto{\pgfqpoint{3.399569in}{0.715398in}}%
\pgfpathlineto{\pgfqpoint{3.383736in}{0.707144in}}%
\pgfpathlineto{\pgfqpoint{3.356363in}{0.696078in}}%
\pgfpathlineto{\pgfqpoint{3.324613in}{0.686598in}}%
\pgfpathlineto{\pgfqpoint{3.288354in}{0.678715in}}%
\pgfpathlineto{\pgfqpoint{3.247341in}{0.672442in}}%
\pgfpathlineto{\pgfqpoint{3.201219in}{0.667793in}}%
\pgfpathlineto{\pgfqpoint{3.149520in}{0.664787in}}%
\pgfpathlineto{\pgfqpoint{3.091700in}{0.663442in}}%
\pgfpathlineto{\pgfqpoint{3.027666in}{0.663784in}}%
\pgfpathlineto{\pgfqpoint{2.930730in}{0.667027in}}%
\pgfpathlineto{\pgfqpoint{2.818264in}{0.673677in}}%
\pgfpathlineto{\pgfqpoint{2.688875in}{0.683959in}}%
\pgfpathlineto{\pgfqpoint{2.542383in}{0.698082in}}%
\pgfpathlineto{\pgfqpoint{2.379823in}{0.716237in}}%
\pgfpathlineto{\pgfqpoint{2.203442in}{0.738600in}}%
\pgfpathlineto{\pgfqpoint{2.064665in}{0.758220in}}%
\pgfpathlineto{\pgfqpoint{1.925651in}{0.780147in}}%
\pgfpathlineto{\pgfqpoint{1.791699in}{0.804088in}}%
\pgfpathlineto{\pgfqpoint{1.707393in}{0.821001in}}%
\pgfpathlineto{\pgfqpoint{1.628332in}{0.838547in}}%
\pgfpathlineto{\pgfqpoint{1.555373in}{0.856601in}}%
\pgfpathlineto{\pgfqpoint{1.489184in}{0.875035in}}%
\pgfpathlineto{\pgfqpoint{1.430240in}{0.893706in}}%
\pgfpathlineto{\pgfqpoint{1.378828in}{0.912463in}}%
\pgfpathlineto{\pgfqpoint{1.334755in}{0.931148in}}%
\pgfpathlineto{\pgfqpoint{1.297109in}{0.949653in}}%
\pgfpathlineto{\pgfqpoint{1.265048in}{0.967889in}}%
\pgfpathlineto{\pgfqpoint{1.237916in}{0.985776in}}%
\pgfpathlineto{\pgfqpoint{1.215239in}{1.003241in}}%
\pgfpathlineto{\pgfqpoint{1.196729in}{1.020223in}}%
\pgfpathlineto{\pgfqpoint{1.182279in}{1.036665in}}%
\pgfpathlineto{\pgfqpoint{1.171963in}{1.052522in}}%
\pgfpathlineto{\pgfqpoint{1.165240in}{1.067754in}}%
\pgfpathlineto{\pgfqpoint{1.161385in}{1.082338in}}%
\pgfpathlineto{\pgfqpoint{1.160085in}{1.096259in}}%
\pgfpathlineto{\pgfqpoint{1.161104in}{1.109502in}}%
\pgfpathlineto{\pgfqpoint{1.164282in}{1.122060in}}%
\pgfpathlineto{\pgfqpoint{1.169534in}{1.133926in}}%
\pgfpathlineto{\pgfqpoint{1.176849in}{1.145096in}}%
\pgfpathlineto{\pgfqpoint{1.186296in}{1.155571in}}%
\pgfpathlineto{\pgfqpoint{1.197866in}{1.165352in}}%
\pgfpathlineto{\pgfqpoint{1.211394in}{1.174433in}}%
\pgfpathlineto{\pgfqpoint{1.226829in}{1.182813in}}%
\pgfpathlineto{\pgfqpoint{1.253515in}{1.194064in}}%
\pgfpathlineto{\pgfqpoint{1.284483in}{1.203725in}}%
\pgfpathlineto{\pgfqpoint{1.319901in}{1.211792in}}%
\pgfpathlineto{\pgfqpoint{1.360073in}{1.218258in}}%
\pgfpathlineto{\pgfqpoint{1.405436in}{1.223121in}}%
\pgfpathlineto{\pgfqpoint{1.456356in}{1.226369in}}%
\pgfpathlineto{\pgfqpoint{1.513088in}{1.227956in}}%
\pgfpathlineto{\pgfqpoint{1.576474in}{1.227824in}}%
\pgfpathlineto{\pgfqpoint{1.647290in}{1.225908in}}%
\pgfpathlineto{\pgfqpoint{1.754418in}{1.220451in}}%
\pgfpathlineto{\pgfqpoint{1.877254in}{1.211489in}}%
\pgfpathlineto{\pgfqpoint{2.016791in}{1.198793in}}%
\pgfpathlineto{\pgfqpoint{2.173826in}{1.182101in}}%
\pgfpathlineto{\pgfqpoint{2.349153in}{1.161112in}}%
\pgfpathlineto{\pgfqpoint{2.487198in}{1.142546in}}%
\pgfpathlineto{\pgfqpoint{2.625655in}{1.121682in}}%
\pgfpathlineto{\pgfqpoint{2.760385in}{1.098710in}}%
\pgfpathlineto{\pgfqpoint{2.887757in}{1.073894in}}%
\pgfpathlineto{\pgfqpoint{2.967018in}{1.056488in}}%
\pgfpathlineto{\pgfqpoint{3.040807in}{1.038530in}}%
\pgfpathlineto{\pgfqpoint{3.108421in}{1.020151in}}%
\pgfpathlineto{\pgfqpoint{3.169258in}{1.001499in}}%
\pgfpathlineto{\pgfqpoint{3.222815in}{0.982735in}}%
\pgfpathlineto{\pgfqpoint{3.268783in}{0.964031in}}%
\pgfpathlineto{\pgfqpoint{3.308007in}{0.945488in}}%
\pgfpathlineto{\pgfqpoint{3.341243in}{0.927201in}}%
\pgfpathlineto{\pgfqpoint{3.369047in}{0.909257in}}%
\pgfpathlineto{\pgfqpoint{3.391926in}{0.891735in}}%
\pgfpathlineto{\pgfqpoint{3.410335in}{0.874699in}}%
\pgfpathlineto{\pgfqpoint{3.424745in}{0.858202in}}%
\pgfpathlineto{\pgfqpoint{3.435604in}{0.842284in}}%
\pgfpathlineto{\pgfqpoint{3.443272in}{0.826977in}}%
\pgfpathlineto{\pgfqpoint{3.448039in}{0.812306in}}%
\pgfpathlineto{\pgfqpoint{3.450121in}{0.798290in}}%
\pgfpathlineto{\pgfqpoint{3.449627in}{0.784944in}}%
\pgfpathlineto{\pgfqpoint{3.446720in}{0.772277in}}%
\pgfpathlineto{\pgfqpoint{3.441636in}{0.760297in}}%
\pgfpathlineto{\pgfqpoint{3.434556in}{0.749009in}}%
\pgfpathlineto{\pgfqpoint{3.425603in}{0.738419in}}%
\pgfpathlineto{\pgfqpoint{3.414844in}{0.728529in}}%
\pgfpathlineto{\pgfqpoint{3.402288in}{0.719338in}}%
\pgfpathlineto{\pgfqpoint{3.387887in}{0.710846in}}%
\pgfpathlineto{\pgfqpoint{3.371537in}{0.703048in}}%
\pgfpathlineto{\pgfqpoint{3.342990in}{0.692643in}}%
\pgfpathlineto{\pgfqpoint{3.309647in}{0.683797in}}%
\pgfpathlineto{\pgfqpoint{3.271636in}{0.676536in}}%
\pgfpathlineto{\pgfqpoint{3.228686in}{0.670877in}}%
\pgfpathlineto{\pgfqpoint{3.180441in}{0.666848in}}%
\pgfpathlineto{\pgfqpoint{3.126460in}{0.664478in}}%
\pgfpathlineto{\pgfqpoint{3.066215in}{0.663808in}}%
\pgfpathlineto{\pgfqpoint{2.999094in}{0.664882in}}%
\pgfpathlineto{\pgfqpoint{2.897800in}{0.669120in}}%
\pgfpathlineto{\pgfqpoint{2.781404in}{0.676764in}}%
\pgfpathlineto{\pgfqpoint{2.647943in}{0.688079in}}%
\pgfpathlineto{\pgfqpoint{2.497017in}{0.703305in}}%
\pgfpathlineto{\pgfqpoint{2.329920in}{0.722644in}}%
\pgfpathlineto{\pgfqpoint{2.149615in}{0.746270in}}%
\pgfpathlineto{\pgfqpoint{2.010232in}{0.766785in}}%
\pgfpathlineto{\pgfqpoint{1.873046in}{0.789476in}}%
\pgfpathlineto{\pgfqpoint{1.742632in}{0.814065in}}%
\pgfpathlineto{\pgfqpoint{1.661381in}{0.831343in}}%
\pgfpathlineto{\pgfqpoint{1.585823in}{0.849192in}}%
\pgfpathlineto{\pgfqpoint{1.516776in}{0.867479in}}%
\pgfpathlineto{\pgfqpoint{1.454910in}{0.886058in}}%
\pgfpathlineto{\pgfqpoint{1.400677in}{0.904769in}}%
\pgfpathlineto{\pgfqpoint{1.353668in}{0.923467in}}%
\pgfpathlineto{\pgfqpoint{1.313123in}{0.942040in}}%
\pgfpathlineto{\pgfqpoint{1.278393in}{0.960390in}}%
\pgfpathlineto{\pgfqpoint{1.248951in}{0.978425in}}%
\pgfpathlineto{\pgfqpoint{1.224392in}{0.996065in}}%
\pgfpathlineto{\pgfqpoint{1.204428in}{1.013241in}}%
\pgfpathlineto{\pgfqpoint{1.188895in}{1.029892in}}%
\pgfpathlineto{\pgfqpoint{1.177465in}{1.045968in}}%
\pgfpathlineto{\pgfqpoint{1.169368in}{1.061434in}}%
\pgfpathlineto{\pgfqpoint{1.164190in}{1.076265in}}%
\pgfpathlineto{\pgfqpoint{1.161617in}{1.090444in}}%
\pgfpathlineto{\pgfqpoint{1.161426in}{1.103955in}}%
\pgfpathlineto{\pgfqpoint{1.163484in}{1.116786in}}%
\pgfpathlineto{\pgfqpoint{1.167748in}{1.128931in}}%
\pgfpathlineto{\pgfqpoint{1.174268in}{1.140386in}}%
\pgfpathlineto{\pgfqpoint{1.183007in}{1.151149in}}%
\pgfpathlineto{\pgfqpoint{1.193759in}{1.161216in}}%
\pgfpathlineto{\pgfqpoint{1.206439in}{1.170582in}}%
\pgfpathlineto{\pgfqpoint{1.220992in}{1.179247in}}%
\pgfpathlineto{\pgfqpoint{1.246290in}{1.190924in}}%
\pgfpathlineto{\pgfqpoint{1.275811in}{1.201015in}}%
\pgfpathlineto{\pgfqpoint{1.309765in}{1.209519in}}%
\pgfpathlineto{\pgfqpoint{1.348521in}{1.216440in}}%
\pgfpathlineto{\pgfqpoint{1.392474in}{1.221776in}}%
\pgfpathlineto{\pgfqpoint{1.441723in}{1.225493in}}%
\pgfpathlineto{\pgfqpoint{1.496838in}{1.227552in}}%
\pgfpathlineto{\pgfqpoint{1.558438in}{1.227906in}}%
\pgfpathlineto{\pgfqpoint{1.627148in}{1.226496in}}%
\pgfpathlineto{\pgfqpoint{1.730909in}{1.221754in}}%
\pgfpathlineto{\pgfqpoint{1.849930in}{1.213558in}}%
\pgfpathlineto{\pgfqpoint{1.985720in}{1.201681in}}%
\pgfpathlineto{\pgfqpoint{2.139071in}{1.185875in}}%
\pgfpathlineto{\pgfqpoint{2.308717in}{1.165881in}}%
\pgfpathlineto{\pgfqpoint{2.444068in}{1.148079in}}%
\pgfpathlineto{\pgfqpoint{2.582906in}{1.127918in}}%
\pgfpathlineto{\pgfqpoint{2.720908in}{1.105516in}}%
\pgfpathlineto{\pgfqpoint{2.853061in}{1.081101in}}%
\pgfpathlineto{\pgfqpoint{2.935559in}{1.063911in}}%
\pgfpathlineto{\pgfqpoint{3.012299in}{1.046145in}}%
\pgfpathlineto{\pgfqpoint{3.082506in}{1.027936in}}%
\pgfpathlineto{\pgfqpoint{3.145684in}{1.009418in}}%
\pgfpathlineto{\pgfqpoint{3.201618in}{0.990727in}}%
\pgfpathlineto{\pgfqpoint{3.250377in}{0.972000in}}%
\pgfpathlineto{\pgfqpoint{3.291981in}{0.953393in}}%
\pgfpathlineto{\pgfqpoint{3.327039in}{0.935022in}}%
\pgfpathlineto{\pgfqpoint{3.356699in}{0.916958in}}%
\pgfpathlineto{\pgfqpoint{3.381849in}{0.899264in}}%
\pgfpathlineto{\pgfqpoint{3.403113in}{0.882000in}}%
\pgfpathlineto{\pgfqpoint{3.420851in}{0.865221in}}%
\pgfpathlineto{\pgfqpoint{3.435160in}{0.848975in}}%
\pgfpathlineto{\pgfqpoint{3.445872in}{0.833308in}}%
\pgfpathlineto{\pgfqpoint{3.452559in}{0.818260in}}%
\pgfpathlineto{\pgfqpoint{3.455251in}{0.803867in}}%
\pgfpathlineto{\pgfqpoint{3.455334in}{0.790152in}}%
\pgfpathlineto{\pgfqpoint{3.453100in}{0.777124in}}%
\pgfpathlineto{\pgfqpoint{3.448733in}{0.764792in}}%
\pgfpathlineto{\pgfqpoint{3.442358in}{0.753159in}}%
\pgfpathlineto{\pgfqpoint{3.434048in}{0.742228in}}%
\pgfpathlineto{\pgfqpoint{3.423814in}{0.732002in}}%
\pgfpathlineto{\pgfqpoint{3.411615in}{0.722477in}}%
\pgfpathlineto{\pgfqpoint{3.397362in}{0.713652in}}%
\pgfpathlineto{\pgfqpoint{3.381122in}{0.705526in}}%
\pgfpathlineto{\pgfqpoint{3.353142in}{0.694654in}}%
\pgfpathlineto{\pgfqpoint{3.320778in}{0.685371in}}%
\pgfpathlineto{\pgfqpoint{3.283884in}{0.677688in}}%
\pgfpathlineto{\pgfqpoint{3.242199in}{0.671618in}}%
\pgfpathlineto{\pgfqpoint{3.195354in}{0.667177in}}%
\pgfpathlineto{\pgfqpoint{3.142868in}{0.664382in}}%
\pgfpathlineto{\pgfqpoint{3.084211in}{0.663251in}}%
\pgfpathlineto{\pgfqpoint{3.019277in}{0.663816in}}%
\pgfpathlineto{\pgfqpoint{2.920913in}{0.667377in}}%
\pgfpathlineto{\pgfqpoint{2.806820in}{0.674370in}}%
\pgfpathlineto{\pgfqpoint{2.675697in}{0.685019in}}%
\pgfpathlineto{\pgfqpoint{2.527488in}{0.699532in}}%
\pgfpathlineto{\pgfqpoint{2.363378in}{0.718097in}}%
\pgfpathlineto{\pgfqpoint{2.185796in}{0.740883in}}%
\pgfpathlineto{\pgfqpoint{2.046962in}{0.760778in}}%
\pgfpathlineto{\pgfqpoint{1.908470in}{0.782936in}}%
\pgfpathlineto{\pgfqpoint{1.775170in}{0.807117in}}%
\pgfpathlineto{\pgfqpoint{1.775170in}{0.807117in}}%
\pgfusepath{stroke}%
\end{pgfscope}%
\begin{pgfscope}%
\pgfpathrectangle{\pgfqpoint{0.562500in}{0.275000in}}{\pgfqpoint{3.487500in}{1.925000in}}%
\pgfusepath{clip}%
\pgfsetrectcap%
\pgfsetroundjoin%
\pgfsetlinewidth{1.505625pt}%
\definecolor{currentstroke}{rgb}{0.549020,0.337255,0.294118}%
\pgfsetstrokecolor{currentstroke}%
\pgfsetdash{}{0pt}%
\pgfpathmoveto{\pgfqpoint{0.721023in}{0.362500in}}%
\pgfpathlineto{\pgfqpoint{0.900108in}{0.390449in}}%
\pgfpathlineto{\pgfqpoint{1.011075in}{0.416426in}}%
\pgfpathlineto{\pgfqpoint{1.080172in}{0.440941in}}%
\pgfpathlineto{\pgfqpoint{1.127471in}{0.464339in}}%
\pgfpathlineto{\pgfqpoint{1.159575in}{0.486806in}}%
\pgfpathlineto{\pgfqpoint{1.180224in}{0.508476in}}%
\pgfpathlineto{\pgfqpoint{1.193151in}{0.529458in}}%
\pgfpathlineto{\pgfqpoint{1.200871in}{0.549836in}}%
\pgfpathlineto{\pgfqpoint{1.204452in}{0.569661in}}%
\pgfpathlineto{\pgfqpoint{1.204842in}{0.588974in}}%
\pgfpathlineto{\pgfqpoint{1.202898in}{0.607814in}}%
\pgfpathlineto{\pgfqpoint{1.194903in}{0.644199in}}%
\pgfpathlineto{\pgfqpoint{1.183629in}{0.678993in}}%
\pgfpathlineto{\pgfqpoint{1.163659in}{0.728431in}}%
\pgfpathlineto{\pgfqpoint{1.113406in}{0.845958in}}%
\pgfpathlineto{\pgfqpoint{1.098728in}{0.885132in}}%
\pgfpathlineto{\pgfqpoint{1.087026in}{0.921822in}}%
\pgfpathlineto{\pgfqpoint{1.078643in}{0.956096in}}%
\pgfpathlineto{\pgfqpoint{1.073312in}{0.988071in}}%
\pgfpathlineto{\pgfqpoint{1.070786in}{1.017859in}}%
\pgfpathlineto{\pgfqpoint{1.071058in}{1.045550in}}%
\pgfpathlineto{\pgfqpoint{1.072896in}{1.062879in}}%
\pgfpathlineto{\pgfqpoint{1.076200in}{1.079318in}}%
\pgfpathlineto{\pgfqpoint{1.081142in}{1.094877in}}%
\pgfpathlineto{\pgfqpoint{1.087942in}{1.109558in}}%
\pgfpathlineto{\pgfqpoint{1.096542in}{1.123377in}}%
\pgfpathlineto{\pgfqpoint{1.106676in}{1.136356in}}%
\pgfpathlineto{\pgfqpoint{1.118279in}{1.148508in}}%
\pgfpathlineto{\pgfqpoint{1.131316in}{1.159845in}}%
\pgfpathlineto{\pgfqpoint{1.145788in}{1.170380in}}%
\pgfpathlineto{\pgfqpoint{1.170274in}{1.184701in}}%
\pgfpathlineto{\pgfqpoint{1.198338in}{1.197276in}}%
\pgfpathlineto{\pgfqpoint{1.230414in}{1.208138in}}%
\pgfpathlineto{\pgfqpoint{1.267040in}{1.217312in}}%
\pgfpathlineto{\pgfqpoint{1.308215in}{1.224799in}}%
\pgfpathlineto{\pgfqpoint{1.354192in}{1.230589in}}%
\pgfpathlineto{\pgfqpoint{1.405413in}{1.234664in}}%
\pgfpathlineto{\pgfqpoint{1.462395in}{1.236997in}}%
\pgfpathlineto{\pgfqpoint{1.525729in}{1.237552in}}%
\pgfpathlineto{\pgfqpoint{1.596082in}{1.236281in}}%
\pgfpathlineto{\pgfqpoint{1.702092in}{1.231643in}}%
\pgfpathlineto{\pgfqpoint{1.823743in}{1.223478in}}%
\pgfpathlineto{\pgfqpoint{1.962808in}{1.211516in}}%
\pgfpathlineto{\pgfqpoint{2.120000in}{1.195494in}}%
\pgfpathlineto{\pgfqpoint{2.293937in}{1.175191in}}%
\pgfpathlineto{\pgfqpoint{2.433165in}{1.157065in}}%
\pgfpathlineto{\pgfqpoint{2.575818in}{1.136490in}}%
\pgfpathlineto{\pgfqpoint{2.717265in}{1.113613in}}%
\pgfpathlineto{\pgfqpoint{2.852715in}{1.088697in}}%
\pgfpathlineto{\pgfqpoint{2.937242in}{1.071132in}}%
\pgfpathlineto{\pgfqpoint{3.015389in}{1.052959in}}%
\pgfpathlineto{\pgfqpoint{3.086300in}{1.034327in}}%
\pgfpathlineto{\pgfqpoint{3.149947in}{1.015398in}}%
\pgfpathlineto{\pgfqpoint{3.206414in}{0.996320in}}%
\pgfpathlineto{\pgfqpoint{3.255871in}{0.977229in}}%
\pgfpathlineto{\pgfqpoint{3.298577in}{0.958248in}}%
\pgfpathlineto{\pgfqpoint{3.334877in}{0.939488in}}%
\pgfpathlineto{\pgfqpoint{3.365204in}{0.921045in}}%
\pgfpathlineto{\pgfqpoint{3.390079in}{0.903004in}}%
\pgfpathlineto{\pgfqpoint{3.410068in}{0.885439in}}%
\pgfpathlineto{\pgfqpoint{3.425582in}{0.868422in}}%
\pgfpathlineto{\pgfqpoint{3.437361in}{0.851988in}}%
\pgfpathlineto{\pgfqpoint{3.446063in}{0.836164in}}%
\pgfpathlineto{\pgfqpoint{3.452166in}{0.820973in}}%
\pgfpathlineto{\pgfqpoint{3.455970in}{0.806434in}}%
\pgfpathlineto{\pgfqpoint{3.457596in}{0.792561in}}%
\pgfpathlineto{\pgfqpoint{3.456988in}{0.779366in}}%
\pgfpathlineto{\pgfqpoint{3.453911in}{0.766855in}}%
\pgfpathlineto{\pgfqpoint{3.447967in}{0.755031in}}%
\pgfpathlineto{\pgfqpoint{3.439566in}{0.743906in}}%
\pgfpathlineto{\pgfqpoint{3.429194in}{0.733492in}}%
\pgfpathlineto{\pgfqpoint{3.416928in}{0.723787in}}%
\pgfpathlineto{\pgfqpoint{3.402817in}{0.714795in}}%
\pgfpathlineto{\pgfqpoint{3.386878in}{0.706514in}}%
\pgfpathlineto{\pgfqpoint{3.359500in}{0.695426in}}%
\pgfpathlineto{\pgfqpoint{3.327787in}{0.685935in}}%
\pgfpathlineto{\pgfqpoint{3.291397in}{0.678034in}}%
\pgfpathlineto{\pgfqpoint{3.250078in}{0.671727in}}%
\pgfpathlineto{\pgfqpoint{3.203665in}{0.667038in}}%
\pgfpathlineto{\pgfqpoint{3.151669in}{0.663998in}}%
\pgfpathlineto{\pgfqpoint{3.093555in}{0.662646in}}%
\pgfpathlineto{\pgfqpoint{3.028743in}{0.663033in}}%
\pgfpathlineto{\pgfqpoint{2.930826in}{0.666360in}}%
\pgfpathlineto{\pgfqpoint{2.818290in}{0.673061in}}%
\pgfpathlineto{\pgfqpoint{2.689449in}{0.683343in}}%
\pgfpathlineto{\pgfqpoint{2.543106in}{0.697471in}}%
\pgfpathlineto{\pgfqpoint{2.379590in}{0.715694in}}%
\pgfpathlineto{\pgfqpoint{2.201845in}{0.738176in}}%
\pgfpathlineto{\pgfqpoint{2.062961in}{0.757841in}}%
\pgfpathlineto{\pgfqpoint{1.924009in}{0.779790in}}%
\pgfpathlineto{\pgfqpoint{1.789661in}{0.803797in}}%
\pgfpathlineto{\pgfqpoint{1.705072in}{0.820784in}}%
\pgfpathlineto{\pgfqpoint{1.626288in}{0.838430in}}%
\pgfpathlineto{\pgfqpoint{1.554074in}{0.856584in}}%
\pgfpathlineto{\pgfqpoint{1.488674in}{0.875085in}}%
\pgfpathlineto{\pgfqpoint{1.430203in}{0.893780in}}%
\pgfpathlineto{\pgfqpoint{1.378647in}{0.912533in}}%
\pgfpathlineto{\pgfqpoint{1.333865in}{0.931219in}}%
\pgfpathlineto{\pgfqpoint{1.295588in}{0.949728in}}%
\pgfpathlineto{\pgfqpoint{1.263418in}{0.967963in}}%
\pgfpathlineto{\pgfqpoint{1.236830in}{0.985840in}}%
\pgfpathlineto{\pgfqpoint{1.215171in}{1.003288in}}%
\pgfpathlineto{\pgfqpoint{1.197886in}{1.020242in}}%
\pgfpathlineto{\pgfqpoint{1.184840in}{1.036633in}}%
\pgfpathlineto{\pgfqpoint{1.175261in}{1.052434in}}%
\pgfpathlineto{\pgfqpoint{1.168489in}{1.067620in}}%
\pgfpathlineto{\pgfqpoint{1.164050in}{1.082173in}}%
\pgfpathlineto{\pgfqpoint{1.161651in}{1.096076in}}%
\pgfpathlineto{\pgfqpoint{1.161182in}{1.109319in}}%
\pgfpathlineto{\pgfqpoint{1.162716in}{1.121892in}}%
\pgfpathlineto{\pgfqpoint{1.166509in}{1.133792in}}%
\pgfpathlineto{\pgfqpoint{1.173001in}{1.145017in}}%
\pgfpathlineto{\pgfqpoint{1.182561in}{1.155566in}}%
\pgfpathlineto{\pgfqpoint{1.194264in}{1.165413in}}%
\pgfpathlineto{\pgfqpoint{1.207878in}{1.174555in}}%
\pgfpathlineto{\pgfqpoint{1.223368in}{1.182989in}}%
\pgfpathlineto{\pgfqpoint{1.250104in}{1.194311in}}%
\pgfpathlineto{\pgfqpoint{1.281122in}{1.204037in}}%
\pgfpathlineto{\pgfqpoint{1.316633in}{1.212166in}}%
\pgfpathlineto{\pgfqpoint{1.356983in}{1.218698in}}%
\pgfpathlineto{\pgfqpoint{1.402498in}{1.223626in}}%
\pgfpathlineto{\pgfqpoint{1.453393in}{1.226921in}}%
\pgfpathlineto{\pgfqpoint{1.510308in}{1.228541in}}%
\pgfpathlineto{\pgfqpoint{1.573897in}{1.228438in}}%
\pgfpathlineto{\pgfqpoint{1.644798in}{1.226548in}}%
\pgfpathlineto{\pgfqpoint{1.751798in}{1.221125in}}%
\pgfpathlineto{\pgfqpoint{1.874382in}{1.212196in}}%
\pgfpathlineto{\pgfqpoint{2.013973in}{1.199526in}}%
\pgfpathlineto{\pgfqpoint{2.171257in}{1.182876in}}%
\pgfpathlineto{\pgfqpoint{2.344149in}{1.161986in}}%
\pgfpathlineto{\pgfqpoint{2.480996in}{1.143492in}}%
\pgfpathlineto{\pgfqpoint{2.620283in}{1.122657in}}%
\pgfpathlineto{\pgfqpoint{2.757555in}{1.099651in}}%
\pgfpathlineto{\pgfqpoint{2.845302in}{1.083230in}}%
\pgfpathlineto{\pgfqpoint{2.928306in}{1.066051in}}%
\pgfpathlineto{\pgfqpoint{3.005588in}{1.048270in}}%
\pgfpathlineto{\pgfqpoint{3.076425in}{1.030041in}}%
\pgfpathlineto{\pgfqpoint{3.140331in}{1.011512in}}%
\pgfpathlineto{\pgfqpoint{3.197067in}{0.992819in}}%
\pgfpathlineto{\pgfqpoint{3.246631in}{0.974090in}}%
\pgfpathlineto{\pgfqpoint{3.289263in}{0.955444in}}%
\pgfpathlineto{\pgfqpoint{3.325446in}{0.936991in}}%
\pgfpathlineto{\pgfqpoint{3.355871in}{0.918833in}}%
\pgfpathlineto{\pgfqpoint{3.380736in}{0.901079in}}%
\pgfpathlineto{\pgfqpoint{3.400803in}{0.883791in}}%
\pgfpathlineto{\pgfqpoint{3.416989in}{0.867010in}}%
\pgfpathlineto{\pgfqpoint{3.429974in}{0.850774in}}%
\pgfpathlineto{\pgfqpoint{3.440206in}{0.835114in}}%
\pgfpathlineto{\pgfqpoint{3.447898in}{0.820060in}}%
\pgfpathlineto{\pgfqpoint{3.453027in}{0.805634in}}%
\pgfpathlineto{\pgfqpoint{3.455336in}{0.791856in}}%
\pgfpathlineto{\pgfqpoint{3.454333in}{0.778741in}}%
\pgfpathlineto{\pgfqpoint{3.450169in}{0.766307in}}%
\pgfpathlineto{\pgfqpoint{3.443852in}{0.754573in}}%
\pgfpathlineto{\pgfqpoint{3.435525in}{0.743542in}}%
\pgfpathlineto{\pgfqpoint{3.425285in}{0.733217in}}%
\pgfpathlineto{\pgfqpoint{3.413189in}{0.723598in}}%
\pgfpathlineto{\pgfqpoint{3.399256in}{0.714686in}}%
\pgfpathlineto{\pgfqpoint{3.383468in}{0.706481in}}%
\pgfpathlineto{\pgfqpoint{3.356169in}{0.695492in}}%
\pgfpathlineto{\pgfqpoint{3.324246in}{0.686077in}}%
\pgfpathlineto{\pgfqpoint{3.287686in}{0.678245in}}%
\pgfpathlineto{\pgfqpoint{3.246303in}{0.672013in}}%
\pgfpathlineto{\pgfqpoint{3.199756in}{0.667404in}}%
\pgfpathlineto{\pgfqpoint{3.147627in}{0.664448in}}%
\pgfpathlineto{\pgfqpoint{3.089417in}{0.663184in}}%
\pgfpathlineto{\pgfqpoint{3.024547in}{0.663656in}}%
\pgfpathlineto{\pgfqpoint{2.926547in}{0.667080in}}%
\pgfpathlineto{\pgfqpoint{2.813890in}{0.673848in}}%
\pgfpathlineto{\pgfqpoint{2.684752in}{0.684215in}}%
\pgfpathlineto{\pgfqpoint{2.538050in}{0.698439in}}%
\pgfpathlineto{\pgfqpoint{2.374488in}{0.716743in}}%
\pgfpathlineto{\pgfqpoint{2.196584in}{0.739309in}}%
\pgfpathlineto{\pgfqpoint{2.057724in}{0.759036in}}%
\pgfpathlineto{\pgfqpoint{1.919255in}{0.781030in}}%
\pgfpathlineto{\pgfqpoint{1.785661in}{0.805058in}}%
\pgfpathlineto{\pgfqpoint{1.701455in}{0.822050in}}%
\pgfpathlineto{\pgfqpoint{1.622593in}{0.839680in}}%
\pgfpathlineto{\pgfqpoint{1.550293in}{0.857810in}}%
\pgfpathlineto{\pgfqpoint{1.485296in}{0.876286in}}%
\pgfpathlineto{\pgfqpoint{1.427477in}{0.894963in}}%
\pgfpathlineto{\pgfqpoint{1.376579in}{0.913703in}}%
\pgfpathlineto{\pgfqpoint{1.332303in}{0.932382in}}%
\pgfpathlineto{\pgfqpoint{1.294311in}{0.950888in}}%
\pgfpathlineto{\pgfqpoint{1.262224in}{0.969119in}}%
\pgfpathlineto{\pgfqpoint{1.235625in}{0.986986in}}%
\pgfpathlineto{\pgfqpoint{1.214053in}{1.004413in}}%
\pgfpathlineto{\pgfqpoint{1.197001in}{1.021330in}}%
\pgfpathlineto{\pgfqpoint{1.183833in}{1.037692in}}%
\pgfpathlineto{\pgfqpoint{1.173992in}{1.053466in}}%
\pgfpathlineto{\pgfqpoint{1.167053in}{1.068623in}}%
\pgfpathlineto{\pgfqpoint{1.162729in}{1.083141in}}%
\pgfpathlineto{\pgfqpoint{1.160868in}{1.097003in}}%
\pgfpathlineto{\pgfqpoint{1.161454in}{1.110193in}}%
\pgfpathlineto{\pgfqpoint{1.164608in}{1.122707in}}%
\pgfpathlineto{\pgfqpoint{1.170261in}{1.134534in}}%
\pgfpathlineto{\pgfqpoint{1.178021in}{1.145666in}}%
\pgfpathlineto{\pgfqpoint{1.187764in}{1.156100in}}%
\pgfpathlineto{\pgfqpoint{1.199403in}{1.165831in}}%
\pgfpathlineto{\pgfqpoint{1.212888in}{1.174860in}}%
\pgfpathlineto{\pgfqpoint{1.228204in}{1.183185in}}%
\pgfpathlineto{\pgfqpoint{1.254675in}{1.194353in}}%
\pgfpathlineto{\pgfqpoint{1.285562in}{1.203944in}}%
\pgfpathlineto{\pgfqpoint{1.321253in}{1.211965in}}%
\pgfpathlineto{\pgfqpoint{1.361795in}{1.218403in}}%
\pgfpathlineto{\pgfqpoint{1.407442in}{1.223231in}}%
\pgfpathlineto{\pgfqpoint{1.458632in}{1.226421in}}%
\pgfpathlineto{\pgfqpoint{1.515863in}{1.227932in}}%
\pgfpathlineto{\pgfqpoint{1.579692in}{1.227716in}}%
\pgfpathlineto{\pgfqpoint{1.676145in}{1.224639in}}%
\pgfpathlineto{\pgfqpoint{1.787075in}{1.218225in}}%
\pgfpathlineto{\pgfqpoint{1.914167in}{1.208262in}}%
\pgfpathlineto{\pgfqpoint{2.058682in}{1.194482in}}%
\pgfpathlineto{\pgfqpoint{2.220336in}{1.176643in}}%
\pgfpathlineto{\pgfqpoint{2.396613in}{1.154566in}}%
\pgfpathlineto{\pgfqpoint{2.534706in}{1.135209in}}%
\pgfpathlineto{\pgfqpoint{2.673478in}{1.113557in}}%
\pgfpathlineto{\pgfqpoint{2.808282in}{1.089823in}}%
\pgfpathlineto{\pgfqpoint{2.893342in}{1.072986in}}%
\pgfpathlineto{\pgfqpoint{2.972950in}{1.055464in}}%
\pgfpathlineto{\pgfqpoint{3.046387in}{1.037422in}}%
\pgfpathlineto{\pgfqpoint{3.113147in}{1.019019in}}%
\pgfpathlineto{\pgfqpoint{3.172924in}{1.000401in}}%
\pgfpathlineto{\pgfqpoint{3.225611in}{0.981702in}}%
\pgfpathlineto{\pgfqpoint{3.271302in}{0.963045in}}%
\pgfpathlineto{\pgfqpoint{3.310287in}{0.944541in}}%
\pgfpathlineto{\pgfqpoint{3.343056in}{0.926289in}}%
\pgfpathlineto{\pgfqpoint{3.370299in}{0.908377in}}%
\pgfpathlineto{\pgfqpoint{3.392750in}{0.890885in}}%
\pgfpathlineto{\pgfqpoint{3.410483in}{0.873895in}}%
\pgfpathlineto{\pgfqpoint{3.424261in}{0.857448in}}%
\pgfpathlineto{\pgfqpoint{3.434816in}{0.841576in}}%
\pgfpathlineto{\pgfqpoint{3.442680in}{0.826305in}}%
\pgfpathlineto{\pgfqpoint{3.448188in}{0.811657in}}%
\pgfpathlineto{\pgfqpoint{3.451476in}{0.797651in}}%
\pgfpathlineto{\pgfqpoint{3.452481in}{0.784301in}}%
\pgfpathlineto{\pgfqpoint{3.450943in}{0.771617in}}%
\pgfpathlineto{\pgfqpoint{3.446402in}{0.759605in}}%
\pgfpathlineto{\pgfqpoint{3.439007in}{0.748277in}}%
\pgfpathlineto{\pgfqpoint{3.429581in}{0.737651in}}%
\pgfpathlineto{\pgfqpoint{3.418222in}{0.727729in}}%
\pgfpathlineto{\pgfqpoint{3.404988in}{0.718512in}}%
\pgfpathlineto{\pgfqpoint{3.389909in}{0.710003in}}%
\pgfpathlineto{\pgfqpoint{3.363812in}{0.698563in}}%
\pgfpathlineto{\pgfqpoint{3.333402in}{0.688712in}}%
\pgfpathlineto{\pgfqpoint{3.298371in}{0.680444in}}%
\pgfpathlineto{\pgfqpoint{3.258444in}{0.673758in}}%
\pgfpathlineto{\pgfqpoint{3.213504in}{0.668680in}}%
\pgfpathlineto{\pgfqpoint{3.163097in}{0.665238in}}%
\pgfpathlineto{\pgfqpoint{3.106712in}{0.663468in}}%
\pgfpathlineto{\pgfqpoint{3.043791in}{0.663419in}}%
\pgfpathlineto{\pgfqpoint{2.973724in}{0.665150in}}%
\pgfpathlineto{\pgfqpoint{2.868043in}{0.670344in}}%
\pgfpathlineto{\pgfqpoint{2.746794in}{0.679012in}}%
\pgfpathlineto{\pgfqpoint{2.608492in}{0.691398in}}%
\pgfpathlineto{\pgfqpoint{2.452635in}{0.707770in}}%
\pgfpathlineto{\pgfqpoint{2.280952in}{0.728331in}}%
\pgfpathlineto{\pgfqpoint{2.144452in}{0.746564in}}%
\pgfpathlineto{\pgfqpoint{2.005281in}{0.767157in}}%
\pgfpathlineto{\pgfqpoint{1.867851in}{0.789951in}}%
\pgfpathlineto{\pgfqpoint{1.737023in}{0.814676in}}%
\pgfpathlineto{\pgfqpoint{1.656064in}{0.832069in}}%
\pgfpathlineto{\pgfqpoint{1.581266in}{0.850026in}}%
\pgfpathlineto{\pgfqpoint{1.513098in}{0.868384in}}%
\pgfpathlineto{\pgfqpoint{1.451850in}{0.886989in}}%
\pgfpathlineto{\pgfqpoint{1.397638in}{0.905703in}}%
\pgfpathlineto{\pgfqpoint{1.350405in}{0.924398in}}%
\pgfpathlineto{\pgfqpoint{1.309913in}{0.942960in}}%
\pgfpathlineto{\pgfqpoint{1.275753in}{0.961286in}}%
\pgfpathlineto{\pgfqpoint{1.247339in}{0.979289in}}%
\pgfpathlineto{\pgfqpoint{1.223908in}{0.996890in}}%
\pgfpathlineto{\pgfqpoint{1.205075in}{1.014010in}}%
\pgfpathlineto{\pgfqpoint{1.190545in}{1.030590in}}%
\pgfpathlineto{\pgfqpoint{1.179497in}{1.046597in}}%
\pgfpathlineto{\pgfqpoint{1.171301in}{1.062006in}}%
\pgfpathlineto{\pgfqpoint{1.165520in}{1.076794in}}%
\pgfpathlineto{\pgfqpoint{1.161911in}{1.090942in}}%
\pgfpathlineto{\pgfqpoint{1.160425in}{1.104436in}}%
\pgfpathlineto{\pgfqpoint{1.161206in}{1.117266in}}%
\pgfpathlineto{\pgfqpoint{1.164590in}{1.129425in}}%
\pgfpathlineto{\pgfqpoint{1.171075in}{1.140911in}}%
\pgfpathlineto{\pgfqpoint{1.180059in}{1.151704in}}%
\pgfpathlineto{\pgfqpoint{1.191014in}{1.161793in}}%
\pgfpathlineto{\pgfqpoint{1.203869in}{1.171177in}}%
\pgfpathlineto{\pgfqpoint{1.218584in}{1.179854in}}%
\pgfpathlineto{\pgfqpoint{1.244130in}{1.191542in}}%
\pgfpathlineto{\pgfqpoint{1.273934in}{1.201640in}}%
\pgfpathlineto{\pgfqpoint{1.308238in}{1.210148in}}%
\pgfpathlineto{\pgfqpoint{1.347413in}{1.217071in}}%
\pgfpathlineto{\pgfqpoint{1.391584in}{1.222394in}}%
\pgfpathlineto{\pgfqpoint{1.441120in}{1.226088in}}%
\pgfpathlineto{\pgfqpoint{1.496560in}{1.228116in}}%
\pgfpathlineto{\pgfqpoint{1.558472in}{1.228432in}}%
\pgfpathlineto{\pgfqpoint{1.627463in}{1.226979in}}%
\pgfpathlineto{\pgfqpoint{1.731569in}{1.222168in}}%
\pgfpathlineto{\pgfqpoint{1.851007in}{1.213899in}}%
\pgfpathlineto{\pgfqpoint{1.987348in}{1.201945in}}%
\pgfpathlineto{\pgfqpoint{2.141291in}{1.186033in}}%
\pgfpathlineto{\pgfqpoint{2.311439in}{1.165944in}}%
\pgfpathlineto{\pgfqpoint{2.447134in}{1.148072in}}%
\pgfpathlineto{\pgfqpoint{2.586357in}{1.127818in}}%
\pgfpathlineto{\pgfqpoint{2.724367in}{1.105334in}}%
\pgfpathlineto{\pgfqpoint{2.856390in}{1.080876in}}%
\pgfpathlineto{\pgfqpoint{2.938869in}{1.063647in}}%
\pgfpathlineto{\pgfqpoint{3.015448in}{1.045828in}}%
\pgfpathlineto{\pgfqpoint{3.084653in}{1.027572in}}%
\pgfpathlineto{\pgfqpoint{3.146502in}{1.009032in}}%
\pgfpathlineto{\pgfqpoint{3.201494in}{0.990343in}}%
\pgfpathlineto{\pgfqpoint{3.250075in}{0.971632in}}%
\pgfpathlineto{\pgfqpoint{3.292639in}{0.953014in}}%
\pgfpathlineto{\pgfqpoint{3.329528in}{0.934594in}}%
\pgfpathlineto{\pgfqpoint{3.361032in}{0.916465in}}%
\pgfpathlineto{\pgfqpoint{3.387390in}{0.898709in}}%
\pgfpathlineto{\pgfqpoint{3.408788in}{0.881399in}}%
\pgfpathlineto{\pgfqpoint{3.425360in}{0.864593in}}%
\pgfpathlineto{\pgfqpoint{3.437247in}{0.848345in}}%
\pgfpathlineto{\pgfqpoint{3.445312in}{0.832712in}}%
\pgfpathlineto{\pgfqpoint{3.450312in}{0.817721in}}%
\pgfpathlineto{\pgfqpoint{3.452751in}{0.803389in}}%
\pgfpathlineto{\pgfqpoint{3.453008in}{0.789728in}}%
\pgfpathlineto{\pgfqpoint{3.451329in}{0.776748in}}%
\pgfpathlineto{\pgfqpoint{3.447834in}{0.764456in}}%
\pgfpathlineto{\pgfqpoint{3.442512in}{0.752854in}}%
\pgfpathlineto{\pgfqpoint{3.435222in}{0.741944in}}%
\pgfpathlineto{\pgfqpoint{3.425696in}{0.731720in}}%
\pgfpathlineto{\pgfqpoint{3.413547in}{0.722178in}}%
\pgfpathlineto{\pgfqpoint{3.399107in}{0.713331in}}%
\pgfpathlineto{\pgfqpoint{3.382756in}{0.705189in}}%
\pgfpathlineto{\pgfqpoint{3.354658in}{0.694302in}}%
\pgfpathlineto{\pgfqpoint{3.322210in}{0.685013in}}%
\pgfpathlineto{\pgfqpoint{3.285234in}{0.677329in}}%
\pgfpathlineto{\pgfqpoint{3.243437in}{0.671257in}}%
\pgfpathlineto{\pgfqpoint{3.196412in}{0.666808in}}%
\pgfpathlineto{\pgfqpoint{3.143748in}{0.663994in}}%
\pgfpathlineto{\pgfqpoint{3.085233in}{0.662852in}}%
\pgfpathlineto{\pgfqpoint{3.019912in}{0.663447in}}%
\pgfpathlineto{\pgfqpoint{2.920715in}{0.667069in}}%
\pgfpathlineto{\pgfqpoint{2.806226in}{0.674099in}}%
\pgfpathlineto{\pgfqpoint{2.675371in}{0.684748in}}%
\pgfpathlineto{\pgfqpoint{2.527633in}{0.699246in}}%
\pgfpathlineto{\pgfqpoint{2.363023in}{0.717847in}}%
\pgfpathlineto{\pgfqpoint{2.182690in}{0.740819in}}%
\pgfpathlineto{\pgfqpoint{2.043092in}{0.760847in}}%
\pgfpathlineto{\pgfqpoint{1.905244in}{0.783079in}}%
\pgfpathlineto{\pgfqpoint{1.773401in}{0.807265in}}%
\pgfpathlineto{\pgfqpoint{1.690660in}{0.824320in}}%
\pgfpathlineto{\pgfqpoint{1.613135in}{0.841992in}}%
\pgfpathlineto{\pgfqpoint{1.541624in}{0.860157in}}%
\pgfpathlineto{\pgfqpoint{1.476802in}{0.878677in}}%
\pgfpathlineto{\pgfqpoint{1.419220in}{0.897401in}}%
\pgfpathlineto{\pgfqpoint{1.369241in}{0.916167in}}%
\pgfpathlineto{\pgfqpoint{1.326309in}{0.934841in}}%
\pgfpathlineto{\pgfqpoint{1.289644in}{0.953319in}}%
\pgfpathlineto{\pgfqpoint{1.258620in}{0.971506in}}%
\pgfpathlineto{\pgfqpoint{1.232710in}{0.989319in}}%
\pgfpathlineto{\pgfqpoint{1.211485in}{1.006684in}}%
\pgfpathlineto{\pgfqpoint{1.194613in}{1.023540in}}%
\pgfpathlineto{\pgfqpoint{1.181765in}{1.039838in}}%
\pgfpathlineto{\pgfqpoint{1.172362in}{1.055538in}}%
\pgfpathlineto{\pgfqpoint{1.166010in}{1.070613in}}%
\pgfpathlineto{\pgfqpoint{1.162407in}{1.085043in}}%
\pgfpathlineto{\pgfqpoint{1.161330in}{1.098810in}}%
\pgfpathlineto{\pgfqpoint{1.162635in}{1.111901in}}%
\pgfpathlineto{\pgfqpoint{1.166257in}{1.124308in}}%
\pgfpathlineto{\pgfqpoint{1.172149in}{1.136026in}}%
\pgfpathlineto{\pgfqpoint{1.180158in}{1.147051in}}%
\pgfpathlineto{\pgfqpoint{1.190148in}{1.157379in}}%
\pgfpathlineto{\pgfqpoint{1.202024in}{1.167006in}}%
\pgfpathlineto{\pgfqpoint{1.215731in}{1.175931in}}%
\pgfpathlineto{\pgfqpoint{1.231248in}{1.184152in}}%
\pgfpathlineto{\pgfqpoint{1.257979in}{1.195165in}}%
\pgfpathlineto{\pgfqpoint{1.289077in}{1.204601in}}%
\pgfpathlineto{\pgfqpoint{1.324973in}{1.212468in}}%
\pgfpathlineto{\pgfqpoint{1.365919in}{1.218762in}}%
\pgfpathlineto{\pgfqpoint{1.411975in}{1.223447in}}%
\pgfpathlineto{\pgfqpoint{1.463611in}{1.226494in}}%
\pgfpathlineto{\pgfqpoint{1.521349in}{1.227861in}}%
\pgfpathlineto{\pgfqpoint{1.585757in}{1.227500in}}%
\pgfpathlineto{\pgfqpoint{1.683088in}{1.224220in}}%
\pgfpathlineto{\pgfqpoint{1.794982in}{1.217585in}}%
\pgfpathlineto{\pgfqpoint{1.923117in}{1.207387in}}%
\pgfpathlineto{\pgfqpoint{2.068685in}{1.193358in}}%
\pgfpathlineto{\pgfqpoint{2.231349in}{1.175256in}}%
\pgfpathlineto{\pgfqpoint{2.408302in}{1.152911in}}%
\pgfpathlineto{\pgfqpoint{2.546614in}{1.133360in}}%
\pgfpathlineto{\pgfqpoint{2.685185in}{1.111527in}}%
\pgfpathlineto{\pgfqpoint{2.819357in}{1.087636in}}%
\pgfpathlineto{\pgfqpoint{2.903861in}{1.070721in}}%
\pgfpathlineto{\pgfqpoint{2.982888in}{1.053152in}}%
\pgfpathlineto{\pgfqpoint{3.055614in}{1.035082in}}%
\pgfpathlineto{\pgfqpoint{3.121484in}{1.016659in}}%
\pgfpathlineto{\pgfqpoint{3.180201in}{0.998023in}}%
\pgfpathlineto{\pgfqpoint{3.231730in}{0.979307in}}%
\pgfpathlineto{\pgfqpoint{3.276295in}{0.960639in}}%
\pgfpathlineto{\pgfqpoint{3.314382in}{0.942136in}}%
\pgfpathlineto{\pgfqpoint{3.314382in}{0.942136in}}%
\pgfusepath{stroke}%
\end{pgfscope}%
\begin{pgfscope}%
\pgfsetrectcap%
\pgfsetmiterjoin%
\pgfsetlinewidth{0.803000pt}%
\definecolor{currentstroke}{rgb}{0.000000,0.000000,0.000000}%
\pgfsetstrokecolor{currentstroke}%
\pgfsetdash{}{0pt}%
\pgfpathmoveto{\pgfqpoint{0.562500in}{0.275000in}}%
\pgfpathlineto{\pgfqpoint{0.562500in}{2.200000in}}%
\pgfusepath{stroke}%
\end{pgfscope}%
\begin{pgfscope}%
\pgfsetrectcap%
\pgfsetmiterjoin%
\pgfsetlinewidth{0.803000pt}%
\definecolor{currentstroke}{rgb}{0.000000,0.000000,0.000000}%
\pgfsetstrokecolor{currentstroke}%
\pgfsetdash{}{0pt}%
\pgfpathmoveto{\pgfqpoint{4.050000in}{0.275000in}}%
\pgfpathlineto{\pgfqpoint{4.050000in}{2.200000in}}%
\pgfusepath{stroke}%
\end{pgfscope}%
\begin{pgfscope}%
\pgfsetrectcap%
\pgfsetmiterjoin%
\pgfsetlinewidth{0.803000pt}%
\definecolor{currentstroke}{rgb}{0.000000,0.000000,0.000000}%
\pgfsetstrokecolor{currentstroke}%
\pgfsetdash{}{0pt}%
\pgfpathmoveto{\pgfqpoint{0.562500in}{0.275000in}}%
\pgfpathlineto{\pgfqpoint{4.050000in}{0.275000in}}%
\pgfusepath{stroke}%
\end{pgfscope}%
\begin{pgfscope}%
\pgfsetrectcap%
\pgfsetmiterjoin%
\pgfsetlinewidth{0.803000pt}%
\definecolor{currentstroke}{rgb}{0.000000,0.000000,0.000000}%
\pgfsetstrokecolor{currentstroke}%
\pgfsetdash{}{0pt}%
\pgfpathmoveto{\pgfqpoint{0.562500in}{2.200000in}}%
\pgfpathlineto{\pgfqpoint{4.050000in}{2.200000in}}%
\pgfusepath{stroke}%
\end{pgfscope}%
\end{pgfpicture}%
\makeatother%
\endgroup%

    \caption{Lösungen des Differentialgleichungssystems %\ref{TODO}
    Diverse Anfangspunkte haben den selben periodischer Orbit als Omega-Limesmenge.}
\label{poinbendix:fig:fall_2}
\end{figure}

\subsection{Fall 3: $\omega(p)$ ist ein Geschlossener Orbit welcher Singularitäten verbindet} \label{poinbendix:subsection:fall3}

