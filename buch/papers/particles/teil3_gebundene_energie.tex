%
% teil2.tex -- Beispiel-File für teil2 
%
% (c) 2020 Prof Dr Andreas Müller, Hochschule Rapperswil
%
% !TEX root = ../../buch.tex
% !TEX encoding = UTF-8
%
\section{Nichtlineares Medium
\label{particles:section:nichtlinear}}
\kopfrechts{Nichtlineares Medium}
Im Gegensatz zu einem linearen Medium, in dem der Proportionalitätsfaktor nicht von der wirkenden Kraft abhängig ist, 
wird dies bei einem nichtlinearen Medium nun zum Problem.
Dies bedeutet also, dass sich das hookesche Gesetz von 
\index{hookesches Gesetz}%
\index{Gesetz, hooksch}%
\[
    \Delta l
    = 
    \frac{F}{D}
\]
zu
\[
    \Delta l
    = 
    \frac{F}{D(F)}
\]
abändert. 
Wie dieser Proportionalitätsfaktor genau von der Kraft abhängt, ist je nach Medium unterschiedlich.
Ein Beispiel davon wird im Video~\cite{particles:huygens-optics} von Huygens Optics gezeigt, wobei für $D$ die Formel 
\[
    D(F)
    =
    D_0
    \cdot
    (1 + \alpha |F|^n)
\]
eingesetzt wird. Ein Beispiel der daraus resultierenden Graphen ist in \autoref{particles:fig:nichtlin-medium:deform} und \autoref{particles:fig:nichtlin-medium:elast-modul} abgebildet.
\begin{figure}
    \centering
    \subfigure[]{\includegraphics{./papers/particles/figures/out/nichtlineares_medium_deformation.pdf}\label{particles:fig:nichtlin-medium:deform}}\hfill
    \subfigure[]{\includegraphics{./papers/particles/figures/out/nichtlineares_medium_elast_modul.pdf}\label{particles:fig:nichtlin-medium:elast-modul}}
    \caption{Deformation~(a) und Elastizitätsmodul~(b) eines nichtlinearen Mediums mit $D_0 = 1$, $\alpha = 0.005$ und $n = 1.5$.}
\end{figure}

Wiederholen wir nun wieder die Simulation aus \autoref{particles:section:lin-medium:superposition}, so fällt zunächst kein Unterschied auf.
\autoref{particles:fig:nichtlin-medium:beams} zeigt einen Vergleich der beiden Resultate.
Damit die Auswirkung der nichtlinearen Effekte erkannt werden können, müssen die Feldstärken anscheinend bedeutend grösser sein.
\begin{figure}
    \centering
    \subfigure[0:20]{\includegraphics[width=.32\linewidth]{./papers/particles/figures/simulations/beams_nonlin_1_10.png}}\label{particles:fig:nichtlin-medium:beams-1}\hfill
    \subfigure[0:40]{\includegraphics[width=.32\linewidth]{./papers/particles/figures/simulations/beams_nonlin_3_20.png}}\label{particles:fig:nichtlin-medium:beams-2}\hfill
    \subfigure[1:04]{\includegraphics[width=.32\linewidth]{./papers/particles/figures/simulations/beams_nonlin_4_32.png}}\label{particles:fig:nichtlin-medium:beams-3}
    \caption{Strahlen im nichtlinearen Medium zu Beginn~(a), während sie sich kreuzen~(b) und danach~(c) mit den jeweiligen Zeitstempeln aus dem Video \cite{particles:video-beams-nonlin}.}\label{particles:fig:nichtlin-medium:beams}
\end{figure}


\subsection{Selbstfokussierung --- Confined Energy}
% TODO: Was für Bedingungen müssen erfüllt sein, damit dies geschieht? 
%       Kann man das überhaupt so einfach klassifizieren?
%       Allenfalls Analogie zu Schwingkreisen herstellen?
\begin{figure}
    \includegraphics{papers/particles/figures/wavesim/particle_initial_state.png}
    \caption{Ausgangszustand, der zu sehr hohen Feldstärken und schliesslich einer Energieansammlung im Zentrum des Simulationsfelds führt.}\label{particles:fig:partikel:ausgangszustand}
\end{figure}
\begin{figure}
    \begin{center}
        % \includegraphics[width=0.5\textwidth]{papers/particles/figures/wavesim/particle_initial_state.png}
        \includegraphics[width=0.5\textwidth]{papers/particles/figures/simulations/particle_frames/frame_02.png}
        % Alternative 1: \includegraphics[width=0.5\textwidth]{papers/particles/figures/simulations/particle_frames/frame_04.png}
        \caption{Feldkonfigurationen mit zunehmend wachsender Energieansammlung im Zentrum der Simulation.\ \mytodo{Bild anpassen}}\label{particles:fig:partikel:wachsen}
    \end{center}
\end{figure}
Der in Abbildung~\ref{particles:fig:partikel:ausgangszustand} gezeigte Anfangszustand kann durch das Aufeinandertreffen der Wellen im Zentrum besonders grosse Feldstärken hervorrufen.
Wird die Simulation nun abgespielt, so geschieht etwas Unerwartetes: Die Energie im Feld scheint sich im Zentrum zu sammeln.
Die besondere Region inmitten des Simulationsfelds wächst, 
bis die im Ausgangszustand definierten Wellen entweder gegen den Rand der Simulation verschwinden oder ihre Energie in die Region im Zentrum gesteckt haben.
Dieser Prozess wird in den Abbildungen~\ref{particles:fig:partikel:wachsen-1} und~\ref{particles:fig:partikel:wachsen-2} veranschaulicht.

\begin{figure}
    \begin{center}
        % \includegraphics[width=0.5\textwidth]{papers/particles/figures/wavesim/particle_initial_state.png}
        \subfigure[]{\includegraphics[width=0.45\textwidth]{papers/particles/figures/simulations/particle_frames/frame_06.png}}\hfill
        \subfigure[]{\includegraphics[width=0.45\textwidth]{papers/particles/figures/simulations/particle_frames/frame_10.png}}
        \caption{Symmetrische Feldkonfiguration mit langsam abnehmender Energiekonzentration.}\label{particles:fig:partikel:abnehmen:symmetrisch}
    \end{center}
\end{figure}
Die nächste Phase zeichnet sich durch Oszillieren der symmetrischen Feldanordnung aus. 
Weiter sind einige schwache Wellen zu erkennen, welche die Region der konzentrierten Energie verlassen.
So schrumpft die gefangene Energie wie auch die Grösse der Feldanordnung fortlaufend.
Einige Bilder sind in Abbildung~\ref{particles:fig:partikel:abnehmen:symmetrisch} gezeigt.

\begin{figure}
    \includegraphics{papers/particles/figures/wavesim/particle_initial_state.png}
    \caption{Asymmetrischer Feldkonfiguration, die entsteht, wenn die gefangenen Energiemenge ausreichend abgenommen hat.\ \mytodo{Bild anpassen}}\label{particles:fig:partikel:abnehmen:asymmetrisch}
\end{figure}
In einer nächsten Phase wird die Symmetrie der Feldanordnung gebrochen, 
vermutlich da die gespeicherte Energie und somit die Grösse der Anordnung zu klein wird.
Aufgrund des Längenelements der ortsdiskreten Simulation sind wohl ab einer gewissen Grösse keine symmetrischen Anordnungen mehr möglich, welche stabil sind.
Die konzentrierte Energie bleibt jedoch einen weiteren Moment bestehen, 
wie die Abbildung~\ref{particles:fig:partikel:abnehmen:asymmetrisch} zeigt.

\begin{figure}
    \begin{center}
        % \includegraphics[width=0.5\textwidth]{papers/particles/figures/wavesim/particle_initial_state.png}
        \subfigure[]{\includegraphics[width=0.45\textwidth]{./papers/particles/figures/simulations/particle_frames/frame_17.png}}\hfill%
% Alternativ könnte man zusätzlich noch diese Grafik einfügen:
        \subfigure[]{\includegraphics[width=0.45\textwidth]{./papers/particles/figures/simulations/particle_frames/frame_18.png}}
        \caption{Schlussendlicher Zerfall der Energieansammlung.}\label{particles:fig:partikel:abnehmen:zerfall}
    \end{center}
\end{figure}
In der letzten Phase wird die Energie zu klein, um die Anordnung aufrechtzuerhalten. 
Sie verfällt schliesslich in eine konzentrische, sehr hochfrequente Welle. 
Dessen Wellenlänge ist die in der Simulation kleinstmögliche, nämlich zweimal das Längenelement.
Der Zerfall der Feldanordnung ist in Abbildung~\ref{particles:fig:partikel:abnehmen:zerfall} gezeigt.

% Doch was geschieht hier genau? 
% \mynote{Analogie zu Schwingkreis herstellen.}
% \mynote{Hat das Medium eine Eigenfrequenz? Falls ja, ist diese ausschlaggebend, ob sich die Energie fängt?}
% \mynote{Kann man dies als \emph{Soliton} bezeichnen?}

