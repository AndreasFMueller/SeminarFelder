%
% Schnittkrümmung
%
\section{Schnittkrümmung
\label{buch:kruemmung:section:schnittkruemmung}}
\kopfrechts{Schnittkrümmung}

\subsection{Geodätische Untermannigfaltigkeiten}

\subsection{Schnittkrümmungen bestimmen den riemannschen Krümmungstensor}

\begin{satz}
Der riemannsche Krümmungstensor einer Mannigfaltigkeit $M$
ist vollständig bestimmt durch die Schnittkrümmungen.
\end{satz}

Der Beweis wird zeigen, dass sich sogar eine konkrete Formel
für die Werte des Krümmungstensors angeben lässt.

\begin{proof}
Die Komponenten $R_{lhik}$ des riemannsche Krümmungstensors ist
antisymmetrisch im ersten und letzten Indexpaar.
Es gilt also
\[
R_{lhik}
-R_{hlik}
=
-R_{lhki}.
\]
Dies bedeutet, dass die 4-Linearform
\[
R(X,Y,Z,W)
=
\xi^l\eta^h\zeta^i\omega^k R_{lhik}
\]
nur von den 2-Vektoren $X\wedge Y$ bzw.~$Z\wedge W$ abhängt.
Der Krümmungstensor kann daher auch als blineare Funktion
\begin{equation}
\wedge^2 TM\otimes\wedge^2 TM
\to
\mathbb{R}
:
(X\wedge Y,Z\wedge W)
\mapsto
R(X\wedge Y, Z\wedge W)
=
R(X,Y,Z,W)
\label{buch:kruemmung:schnittkruemmung:eqn:Rbivektor}
\end{equation}
auf dem Vektorraum $\bigwedge^2 TM$ der 2-Vektoren von Tangentialvektoren
geschrieben werden.

Der Krümmungstensor ist ausserdem symmetrisch unter der Vertauschung
der Indexpaare $lh$ und $ik$.
Dies bedeutet, dass die
in
\eqref{buch:kruemmung:schnittkruemmung:eqn:Rbivektor}
definierte Funktion $R$ auf den 2-Vektoren symmetrisch ist:
\begin{align*}
R(X\wedge Y,Z\wedge W)
=
R(Z\wedge W,X\wedge Y)
\end{align*}
für beliebige 2-Vektoren $X\wedge Y,Z\wedge W\in\bigwedge^2TM$.
Wir schreiben die symmetrische, bilineare Funktion auch als $R(u,v)$ mit
$u,v\in\bigwedge^2 TM$.

Wir möchten zeigen, dass sich $R(u,v)$ aus den Werten $R(w,w)$ für beliebige
2-Vektoren $w$ berechnen lässt.
Um dies etwas prägnanter auszudrücken, schreiben wir
\[
Q(w) = R(w,w).
\]
Die Funktion $Q$ ist eine quadratische Form.

Die Polarisationsformel \cite[p. 347]{buch:linalg} besagt, dass eine
symmetrische, bilineare Funktion $R(u,v)$ vollständig durch die
Werte $R(u,v)$ bestimmt ist.
Wir führen die Rechnung 
\begin{align}
R(u+v,u+v)
&=
R(u,u) + R(u,v) + R(v,u) + R(v,v)
\notag
\\
\Rightarrow\qquad
Q(u+v)
&=
R(u,u) + 2R(u,v) + R(v,v)
\notag
\\
R(u-v,u-v)
&=
R(u,u) - R(u,v) - R(v,u) + R(v,v)
\notag
\\
\Rightarrow\qquad
Q(u-v)
&=
R(u,u) - 2R(u,v) + R(v,v)
\notag
\intertext{mit der Differenz}
Q(u+v)-Q(u-v)
&=
4 R(u,v),
\notag
\intertext{oder aufgelöst nach dem gemischten Term:}
R(u,v)
&=
{\textstyle\frac14}\bigl( Q(u+v) - Q(u-v) \bigr).
\label{buch:kruemmung:schnittkruemmung:eqn:RQQ}
\end{align}
Die Formel \eqref{buch:kruemmung:schnittkruemmung:eqn:RQQ}
drückt $R(u,v)$ durch Werte der Form $R(w,w)$ mit $w\in\bigwedge^2 TM$ aus.
\end{proof}

\begin{satz}
Der Krümmungstensor ist durch die Schnittkrümmungen vollständig
bestimmt.
\end{satz}

\begin{proof}
Die Schnittkrümmung zu den Vektoren $\vec{s}$ und $\vec{v}$ ist
definiert durch
\begin{align}
K(\vec{s},\vec{v})
&=
\frac{
R(\vec{v},\vec{s},\vec{v},\vec{s})
}{
|\vec{v}\wedge\vec{s}|^2
}.
\intertext{Für einen Basis-2-Vektor $\vec{e}_i\wedge\vec{e}_k$ wird dies zu}
K(\vec{e}_i,\vec{e}_k)
&=
\frac{R(\vec{e}_i,\vec{e}_k,\vec{e}_i,\vec{e}_k)}{|\vec{e}_i\wedge\vec{e}_k|^2}
\notag
\\
&=
\frac{R_{ikik}}{g^{ii}g^{kk}-g^{ik}g^{ik}}.
\label{buch:kruemmung:schnittkruemmung:eqn:Rgggg}
\end{align}
Da $g$ als Metrik definit ist, ist der Nenner von
\eqref{buch:kruemmung:schnittkruemmung:eqn:Rgggg}
niemals 0.
Der Riemannsche Krümmungstensor kann also aus den Werten
\[
R(\vec{e}_i,\vec{e}_k,\vec{e}_i,\vec{e}_k)
=
R(\vec{e}_i\wedge\vec{e}_k,\vec{e}_i\wedge\vec{e}_k)
=
Q(\vec{e}_i\wedge\vec{e}_k)
=
(g^{ii}g^{kk}-g^{ik}g^{ik}) K(\vec{e}_i,\vec{e}_k)
\]
bestimmt werden.
Sie können also aus den Schnittkrümmungen ermittelt werden.
\end{proof}

Der Satz besagt also, dass die vom riemannschen Krümmungstensor
beschriebene Krümmung des Raumes vollständig ergründet werden kann,
indem man Messungen in divergierenden Gedäten durchführt.
Eine andere Art, dies auszudrücken, ist, dass der riemannsche
Krümmungstensor keine zusätzliche Information erhält, die man nicht
in der Funktion der Schnittkrümmungen gefunden werden kann.
