%
% gebiet.tex -- template for standalon tikz images
%
% (c) 2021 Prof Dr Andreas Müller, OST Ostschweizer Fachhochschule
%
\documentclass[tikz]{standalone}
\usepackage{amsmath}
\usepackage{times}
\usepackage{txfonts}
\usepackage{pgfplots}
\usepackage{csvsimple}
\usetikzlibrary{arrows,intersections,math,calc}
\definecolor{darkred}{rgb}{0.8,0,0}
\begin{document}
\def\skala{1}
\begin{tikzpicture}[>=latex,thick,scale=\skala]

\coordinate (A) at (2,1.5);
\coordinate (B) at (6.5,1.5);
\coordinate (B1) at (5,3);
\coordinate (C) at (2.5,5.5);
\coordinate (D) at (0.8,2.8);
\coordinate (E) at (2,3.5);

\def\p{
	(A) to[out=-20,in=-135]
	(B) to[out=45,in=10]
	(B1) to[out=-170,in=-10]
	(C) to[out=170,in=100]
	(D) to[out=-80,in=160] (A) -- cycle;
}

\fill[color=darkred!20] \p;
\draw[color=darkred,line width=1.2pt] \p;
\fill[color=darkred] (B1) circle[radius=0.08];

\draw[->,color=darkred,line width=1.4pt] (B1) -- ++(100:1.5);
\node at ($(B1)+(100:1.5)$) [right] {$n$};
\node[color=darkred] at (A) [below] {$\partial G$};

\fill[color=blue!40,opacity=0.5] (C) circle[radius=0.3];
\fill[color=blue] (C) circle[radius=0.05];
\node[color=blue] at ($(C)+(0,0)$) [above] {$Q$};

\fill[color=blue!40,opacity=0.5] (E) circle[radius=0.3];
\fill[color=blue] (E) circle[radius=0.05];
\node[color=blue] at ($(E)+(0,0)$) [above] {$P$};

\draw[->] (0,-0.1) -- (0,6.3) coordinate[label={right:$x^n$}];
\draw[->] (-0.1,0) -- (7.5,0) coordinate[label={$x^1$}];
\node[color=darkred] at (3.5,2.2) {$G$};

\end{tikzpicture}
\end{document}

