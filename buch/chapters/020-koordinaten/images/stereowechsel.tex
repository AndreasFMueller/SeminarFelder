%
% stereowechsel.tex -- Koordinatenwechsel in stereographischer Projektion
%
% (c) 2021 Prof Dr Andreas Müller, OST Ostschweizer Fachhochschule
%
\documentclass[tikz]{standalone}
\usepackage{amsmath}
\usepackage{times}
\usepackage{txfonts}
\usepackage{pgfplots}
\usepackage{csvsimple}
\usetikzlibrary{arrows,intersections,math,calc}
\begin{document}
\definecolor{darkred}{rgb}{0.8,0,0}
\def\skala{2}
\def\r{3.5}
\pgfmathparse{atan(\r)}
\xdef\w{\pgfmathresult}
\pgfmathparse{2*\w-90}
\xdef\v{\pgfmathresult}
\def\s{0.03}
\begin{tikzpicture}[>=latex,thick,scale=\skala]

\coordinate (P) at (\r,0);
\coordinate (P1) at ({1/\r},0);
\coordinate (A) at (\v:1);
\coordinate (N) at (0,1);
\coordinate (S) at (0,-1);
\coordinate (B) at (-0.1,-0.7);

\draw (0,0) circle[radius=1];

\fill[color=violet!30]
	(A) -- ++({90+\w}:0.2) arc ({90+\w}:{180+\w}:0.2) -- cycle;

\fill[color=blue!20]
	(N) -- ++(0,-0.3) arc (-90:{-90+\w}:0.3) -- cycle;
\node[color=blue] at ($(N)+({-90+\w/2}:0.2)$) {$\alpha_+$};

\fill[color=darkred!20]
	(S) -- ++(0,0.4) arc (90:\w:0.4) -- cycle;
\node[color=darkred] at (B) [left] {$\alpha_-$};
\draw[color=darkred,line width=0.3pt]
	($(B)+(-0.04,0)$) -- ($(0,-1)+({0.5*(90+\w)}:0.3)$);

\begin{scope}
\color{blue}
\draw[line width=1pt] (0,1) -- (\r,0);
\node at (P) [below] {$r$};
\node at (P) [above] {$P_+$};
\end{scope}

\begin{scope}
\color{darkred}
\draw[line width=1pt] (0,-1) -- (A);
\node at (P1) [below right] {$\displaystyle\frac{1}{r}$};
\node at ($(P1)+(0.05,0)$) [above left] {$P_-$};
\end{scope}

\fill (N) circle[radius=\s];
\node at (N) [above left] {$N$};
\fill (S) circle[radius=\s];
\node at (S) [below left] {$S$};
\draw[->] (-1.1,0) -- (4.3,0) coordinate[label={$r$}];
\draw[->] (0,-1.2) -- (0,1.3) coordinate[label={right:$z$}];

\fill[color=blue] (P) circle[radius=\s];
\fill[color=darkred] (P1) circle[radius=\s];

\fill[color=violet] (A) circle[radius=\s];
\node[color=violet] at (A) [above right] {$P$};

\node at (140:1) [above left] {$S^2$};

\end{tikzpicture}
\end{document}

