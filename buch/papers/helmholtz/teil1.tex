
+%
% teil1.tex -- Beispiel-File für das Paper
%
% (c) 2020 Prof Dr Andreas Müller, Hochschule Rapperswil
%
% !TEX root = ../../buch.tex
% !TEX encoding = UTF-8
%
\section{Titel noch unklar
\label{helmholtz:section:teil1}}
\kopfrechts{Problemstellung}


\subsection{Ebene Welle $\mathbf{Q} = 0$
\label{helmholtz:subsection:ebeneWelle}}

Ebene Welle bedeutet, dass ...
\begin{enumerate}
\item bei einer planaren Welle ist $P (\mathbf{r}) = P_0 = konst.\quad bzw. \quad \nabla P = 0$.
\item Aus der Gleichung \eqref{helmholtz:equationReaktiveIntensitaet} folgt $\mathbf{Q} = \mathbf{0}$.
\item sowie aus der Gleichung \eqref{helmholtz:equationAktiveIntensitaet} folgt $\mathbf{I} \neq \mathbf{0}$ da $ \nabla \phi$ parallel zur Ausbreitungsrichtung ist.
\item Schalldruck $P$ und Schall-Geschwindigkeit $v$ sind in Phase ($\phi = 0^{\circ}$).
\end{enumerate}


\begin{equation}
	\mathbf{I}_c ~(\mathbf{r}) = \underbrace{\mathbf{I}~(\mathbf{r})}_{\textit{aktive Intensität}} + \underbrace{ \cancel{j\,\mathbf{Q}~(\mathbf{r})}}_{\textit{reaktive~Intensität}~=~0}.
	\label{helmholtz:equationIntensitaetComplexplanar}
\end{equation}	


\subsection{stehende Welle $\mathbf{I} = 0,\mathbf{Q} \neq 0$
\label{helmholtz:subsection:stehendeWelle}}

Bei einer stehenden Welle überlagern (Superposition) sich zwei gleiche planare Wellen mit gleicher Amplitude in entgegengesetzter Richtung und hat folgende Charakteristik:

\begin{enumerate}
\item an jedem Punkt ist entweder  $P = maximal$ und $v = 0$ bei einem Bauch (Darstellung) oder $P = 0$  und $ v = maximal$ bei einem Knoten (Darstellung).
\item fehlt noch was ??
\item Schalldruck $P$ und Schall-Geschwindigkeit $v$ sind 90\textdegree phasenverschoben ($\phi = 90^{\circ}$).
\end{enumerate}



\begin{equation}
	\mathbf{I}_c ~(\mathbf{r}) = \underbrace{\cancel{\mathbf{I}~(\mathbf{r})}}_{\textit{aktive Intensität}~=~0} + \underbrace{j\,\mathbf{Q}~(\mathbf{r})}_{\textit{reaktive~Intensität}}
	\label{helmholtz:equationIntensitaetComplexStehend}
\end{equation}	