%
% teil1.tex -- Beispiel-File für das Paper
%
% (c) 2020 Prof Dr Andreas Müller, Hochschule Rapperswil
%
% !TEX root = ../../buch.tex
% !TEX encoding = UTF-8
%
\section{Grundbegriffe der Baustatik und Mechanik}
\label{elastomechanik:section:teil1}
In diesem Abschnitt werden die zentralen Begriffe und Gesetzmässigkeiten der Elastizitätstheorie in kompakter Form zusammengetragen. 
Dabei wird bewusst auf eine systematische Herleitung verzichtet, da angenommen wird, dass der Leser mit vielen der verwendeten Begriffe bereits vertraut ist.
Eine vertiefte Behandlung und Illustration erfolgt in späteren Abschnitten, beispielsweise im Rahmen der Herleitung der Feldgleichungen.

\subsection{Begriffe}
\begin{description}
\item[\textbf{Druckkräfte:}] Druckkräfte sind Kräfte, die senkrecht auf die Oberfläche eines Körpers wirken und dabei eine Kompression bzw. Zusammenstauchung des Körpers verursachen. 
Eine negative Normalkraft entspricht einer Druckkraft.
	
\item[\textbf{Druckspannungen:}] Druckspannungen sind Spannungen, die im Inneren eines Materials entstehen, wenn Druckkräfte senkrecht auf die Oberfläche eines Körpers ausgeübt werden. 
Sie bewirken eine Stauchung oder Kompression des Materials.
Druckspannungen haben ein negatives Vorzeichen, da die Kräfte nach innen gerichtet sind.
	
\item[\textbf{Elastizitätsmodul ($E$):}] Der Elastizitätsmodul (E-Modul) ist eine Materialkonstante, die die Steifigkeit eines Werkstoffs beschreibt. 
Er gibt an, wie stark ein Material auf eine mechanische Spannung mit Dehnung reagiert. 
Je grösser der Elastizitätsmodul, desto weniger verformt sich das Material unter einer gegebenen Belastung.
Der Elastizitätsmodul wird definiert als:
	\begin{equation}
		E=
		\frac{\sigma}{\varepsilon},
	\end{equation}
	$\sigma$ = Spannung,
	
	$\varepsilon$ = Dehnung, siehe auch (15.1)
	
\item[\textbf{Isotropie:}] Isotropie beschreibt eine Materialeigenschaft, bei der die mechanischen Eigenschaften in allen Raumrichtungen identisch sind. 
Ein isotropes Material reagiert auf Belastungen unabhängig von der Belastungsrichtung stets gleich.
In der linearen Elastizitätstheorie bedeutet Isotropie, dass das Materialverhalten vollständig durch zwei unabhängige Materialkonstanten beschrieben werden kann, etwa den Elastizitätsmodul ($E$) und die Poisson-Zahl ($\nu$).  
Isotrope Annahmen vereinfachen die mathematische Modellierung deutlich und finden in der technischen Mechanik insbesondere bei homogenen Metallen, Glas und einigen Kunststoffen Anwendung \cite{elastomechanik:Isotropie}.
	
\item[\textbf{Kinematische Gleichungen (Verzerrungsdefinition):}] Die kinematischen Gleichungen beschreiben den Zusammenhang zwischen den Verschiebungen $u_i$ eines Punktes im Raum und den daraus resultierenden Verzerrungen $\varepsilon_{ij}$. 
Sie sind die Grundlage für die geometrische Beschreibung der Deformation in der Elastizitätstheorie. 
Für kleine Verformungen gilt \cite{elastomechanik:Technische_Mechanik_2:Elastostatik}:
	\begin{equation}
		\varepsilon_{ij} = 
		\frac{1}{2} \left( \frac{\partial u_i}{\partial x_j} + \frac{\partial u_j}{\partial x_i} \right).
	\end{equation}
Dabei ist:
	
	$u_i$ = Verschiebung in Richtung $i$
	
	$x_j$ = Raumkoordinate in Richtung $j$
	
Diese Gleichungen stellen sicher, dass die Verzerrungen aus einem stetigen Verschiebungsfeld berechnet werden können.

\item[\textbf{Lamé-Konstanten ($\lambda{,\mu}$):}] Die Lamé-Konstanten werden die Konstanten $\lambda$ und $\mu$ bezeichnet. 
Die sind zwei materialabhängige Parameter, die das mechanische Verhalten eines isotropen elastischen Materials vollständig beschreiben.
Die Lamé-Konstante ($\lambda$) lässt sich aus dem Elastizitätsmodul und der Poisson-Zahl wie folgt berechnen \cite{elastomechanik:Grundlagen_der_Elastizitaetstheorie}:
\begin{equation}
	\lambda = 
	\frac{E \cdot \nu}{(1 + \nu)(1 - 2\nu)}
\end{equation}
$\mu$ = Schubmodul

$\lambda$ = steht in Verbindung mit der Volumenverformung
	
\item[\textbf{Normalkraft ($N$):}] Die Normalkraft ist eine Kraft, die senkrecht (normal) zur Querschnittsfläche eines Körpers oder Bauteils wirkt. 
Sie entsteht durch Zug oder Druck entlang der Längsachse des Körpers und ist eine der wichtigsten Grundkräfte in der technischen Mechanik und Statik.
	
\item[\textbf{Normalspannungen ($\sigma_N$):}] Die Normalspannung ist eine Form der Spannung, die auftritt, wenn eine Kraft senkrecht (normal) zur Fläche eines Körpers wirkt. 
Sie beschreibt den inneren Spannungszustand eines Materials bei Zug- oder Druckbelastung.
	
\item[\textbf{Poisson-Zahl ($\nu$):}] Die Poisson-Zahl, auch bekannt als Querkontraktionszahl, ist eine materialabhängige Konstante, die das Verhältnis zwischen der Querdehnung und der Längsdehnung eines Körpers unter Zug- oder Druckbelastung beschreibt.
Sie ist definiert als
	\begin{equation}
		\nu=
		-\frac{\varepsilon_\text{quer}}{\varepsilon_\text{laengs}},
	\end{equation}
	$\varepsilon_\text{quer}$ = Dehnung senkrecht zur Belastungsrichtung
	
	$\varepsilon_\text{laengs}$ = Dehnung in Belastungsrichtung
	
\item[\textbf{Schubmodul ($\mu$) oder ($G$):}] Der Schubmodul ist eine mechanische Materialkonstante, die beschreibt, wie steif ein Material gegenüber Scherkräften ist.
Scherkräfte (auch Schubkräfte) sind Kräfte, die parallel zur Fläche eines Körpers wirken. 
Bei isotropen Materialien wird er folgend berechnet:
	\begin{equation}
		\mu = 
		\frac{E}{2(1 + \nu)} =
		G.
	\end{equation}
\item[\textbf{Spannung ($\sigma$):}] Die Spannung beschreibt die innere Kraftverteilung in einem Material, die infolge externer Belastungen wie Zug, Druck oder Scherung entsteht. 
Sie gibt an, welche Kraft pro Flächeneinheit innerhalb eines Körpers wirkt.
	
\item[\textbf{Spannungtensor ($\sigma_{ij}$):}] Der Spannungstensor (auch Cauchy-Spannungstensor genannt) beschreibt die Verteilung der inneren Spannungen im Material. 
Der erste Index	$i$ gibt die Richtung der Flächennormalen an, der zweite Index $j$ die Richtung der Spannungswirkung. 
Er ist in der Regel symmetrisch und wird als 3×3-Matrix notiert \cite{elastomechanik:Grundlagen_der_Elastizitaetstheorie}:
	\begin{equation}
		\boldsymbol{\sigma} =
		\begin{bmatrix}
			\sigma_{xx} & \sigma_{xy} & \sigma_{xz} \\
			\sigma_{yx} & \sigma_{yy} & \sigma_{yz} \\
			\sigma_{zx} & \sigma_{zy} & \sigma_{zz}
		\end{bmatrix}.
	\end{equation}
Da der Spannungstensor für reine Elastizität symmetrisch ist, gilt:
	\begin{equation}
		\sigma_{ij} = 
		\sigma_{ji}.
	\end{equation}
	
\item[\textbf{Verschiebung ($u_i$):}] Die Verschiebung beschreibt die Ortsänderung eines Punktes im Körper infolge einer Belastung.

Sei $\vec{x} = (x_1, x_2, x_3)$ der ursprüngliche Ort eines Materialpunktes im unbelasteten Zustand und $\vec{x}' = (x_1', x_2', x_3')$ der Ort desselben Punktes im belasteten Zustand.  
Die \emph{Verschiebungsvektorkomponenten} $u_i$ ergeben sich aus:
\begin{equation}
	u_i(\vec{x}) = x_i' - x_i,
	\quad i = 1,2,3.
\end{equation}

In kartesischen Koordinaten gilt damit:
\[
\vec{u}(\vec{x}) =
\begin{bmatrix}
	u_1(\vec{x}) \\
	u_2(\vec{x}) \\
	u_3(\vec{x})
\end{bmatrix}
=
\begin{bmatrix}
	u(\vec{x}) \\
	v(\vec{x}) \\
	w(\vec{x})
\end{bmatrix}
\]
wobei $u, v, w$ die Verschiebungskomponenten in den Richtungen $x, y, z$ darstellen.

Die Verschiebung bildet die Grundlage zur Definition der Verzerrung. 
Aus den partiellen Ableitungen der Verschiebungskomponenten werden im nächsten Abschnitt die Verzerrungskomponenten $\varepsilon_{ij}$ bestimmt:

\begin{equation}
	\varepsilon_{ij} = \frac{1}{2}
	\left(
	\frac{\partial u_i}{\partial x_j}
	+
	\frac{\partial u_j}{\partial x_i}
	\right).
\end{equation}

Hierbei sind $i,j \in \{1,2,3\}$ und es gilt die \emph{Einstein’sche Summationskonvention}.

\item[\textbf{Verzerrung ($\varepsilon$):}] Die Verzerrung (auch Dehnung genannt) beschreibt die relative Änderung der Form oder Länge eines Körpers infolge einer äusseren Belastung. 
Sie ist ein massloser Wert, da sie ein Verhältnis zweier Längen ist.
Die Normalverrung wird berechnet mit
	\begin{equation}
		\varepsilon 
		= \frac{\Delta L}{L_0},
	\end{equation}
	$\varepsilon$ = Verzerrung
	
	$\Delta L$ = Längenänderung
	
	$\L_0$ = ursprüngliche Länge des Körpers
	
\item[\textbf{Verzerrungstensor ($\varepsilon_{ij}$):}] Der Verzerrungstensor beschreibt die Verformung (Dehnung und Scherung) eines Körpers infolge äusserer Belastungen in drei Raumrichtungen. 
Er erfasst nicht nur Längenänderungen (Normalverzerrungen), sondern auch Winkeländerungen (Schubverzerrungen) und stellt somit die vollständige lokale Verformung eines Materials im Raum dar.
Die Verzerrungstensor wird über die Ableitungen des Verschiebungsfeldes definiert als
	\begin{equation}
		\varepsilon_{ij} = 
		\frac{1}{2} \left( \frac{\partial u_i}{\partial x_j} + \frac{\partial u_j}{\partial x_i} \right),
	\end{equation}
	wobei er in drei Raumdimensionen die Form einer symmetrischen 3×3-Matrix annimmt:
	\begin{equation}
		\boldsymbol{\varepsilon} =
		\begin{bmatrix}
			\varepsilon_{xx} & \varepsilon_{xy} & \varepsilon_{xz} \\
			\varepsilon_{yx} & \varepsilon_{yy} & \varepsilon_{yz} \\
			\varepsilon_{zx} & \varepsilon_{zy} & \varepsilon_{zz},
		\end{bmatrix}.
	\end{equation}
\item[\textbf{Zugkraft:}] Die Zugkraft ist eine Kraft, die längs der Achse eines Körpers wirkt und diesen in Richtung Verlängerung belastet. 
Sie verursacht Zugverformungen im Material.
	
\item[\textbf{Zugspannungen:}] Die Zugspannung ist die innere Spannung, die durch eine Zugkraft im Material erzeugt wird. 
Sie wirkt senkrecht zur Querschnittsfläche des Körpers und beschreibt die Kraft pro Flächeneinheit, mit der das Material dem Zug widersteht.
\end{description}

\subsection{Materialgesetzlichkeiten}
\begin{description}	
\item[\textbf{Elastizitätsgesetz (hookesches Gesetz):}] Das hookesche Gesetz stellt im eindimensionalen Fall den linearen Zusammenhang zwischen Spannung $\sigma$ und Dehnung $\varepsilon$ durch die Beziehung 
	\begin{equation}
		\sigma = 
		E \cdot \varepsilon
	\end{equation}
her, wobei $E$ der Elastizitätsmodul des Materials ist \cite{elastomechanik:Kontinuumsmechanik}.
\item[\textbf{Isotropes Materialverhalten:}] Für isotrope Materialien (die in alle Richtungen die gleichen mechanischen Eigenschaften haben) reicht die Beschreibung mit zwei Materialkonstanten $\lambda$ und $\mu$ aus.
	\begin{equation}
		\sigma_{ij} = 
		\lambda \cdot \delta_{ij} \cdot \varepsilon_{kk} + 2\mu \cdot \varepsilon_{ij},
	\end{equation}
	$\sigma_{ij}$ = Spannungstensor (Komponenten der Spannung in Richtung $i$ auf der Fläche senkrecht zu Richtung $j$)
	
	$\varepsilon_{ij}$ = Dehnungstensor (Komponenten der Verzerrung)
	
	$\varepsilon_{kk}$ = Spur des Verzerrungstensors, also
	\begin{equation}
		\varepsilon_{kk} =
		\varepsilon_{11} + \varepsilon_{22} + \varepsilon_{33},
	\end{equation}
	dies ist die Volumendehnung \cite{elastomechanik:Grundlagen_der_Elastizitaetstheorie}.
	
	$\delta_{ij}$ = Kronecker-Delta 
	
	$\lambda{,\mu}$ = Lamé-Konstanten
	
\item[\textbf{Isotropes Verhalten:}] Wird durch zwei Konstanten ($E$, $u$) beschrieben. 
Gilt für viele Metalle wie Stahl näherungsweise.
	
\item[\textbf{Anisotropes Verhalten}] Mechanische Eigenschaften sind richtungsabhängig. 
Wird durch einen allgemeinen Elastizitätstensor vierter Stufe $C_{ijkl}$ beschrieben.
	
Für unsere Arbeit wird die Herleitung des anisotropen Materialverhaltens nicht thematisiert, da sie den Rahmen dieser Seminararbeit sprengen würde.
	
\item[\textbf{Materialverhaltenseigenschaften:}] Für ideal elastisches Verhalten gilt:
	\begin{enumerate}
		\item Dehnung verschwindet vollständig bei Entlastung.
		
		\item Der Energieinhalt ist vollständig reversibel.
		
		\item Das Verhalten ist unabhängig von der Geschwindigkeit der Belastung.
		
		\item Es existiert eine eindeutige Beziehung zwischen Spannung und Dehnung
	\end{enumerate}
\item[\textbf{Reziproke Form (für Verzerrungen aus Spannungen):}] Die reziproke Form des hookeschen Gesetzes beschreibt den linearen Zusammenhang zwischen Spannung und Verzerrung bei isotropen Materialien durch die Gleichung
	\begin{equation}
		\varepsilon_{ij} = 
		\frac{1+\nu}{E} \cdot \sigma_{ij} - \frac{\nu}{E} \cdot \sigma_{kk} \cdot \delta_{ij}
	\end{equation}
wobei die Verzerrung aus der direkten Spannung und der Spur des Spannungstensors unter Berücksichtigung des Elastizitätsmoduls und der Poisson-Zahl berechnet wird \cite{elastomechanik:Grundlagen_der_Elastizitaetstheorie}.

\item[\textbf{Anisotropes Materialverhalten:}] Bei anisotropen Materialien hängen die mechanischen Eigenschaften von der Richtung ab. 
Solche Materialien können nicht mehr durch zwei Konstanten beschrieben werden, sondern benötigen einen allgemeinen Elastizitätstensor vierter Stufe $C_{ijkl}$:
\begin{equation}
	\sigma_{ij} = C_{ijkl} \varepsilon_{kl}
\end{equation}
Die Anzahl der unabhängigen Komponenten hängt von der Symmetrie ab (bis zu 81 bei vollanisotropem Verhalten). 
Ein Beispiel ist Holz, dessen Steifigkeit stark von der Faserrichtung abhängt.

Für unsere Arbeit wird die Herleitung des anisotropen Verhaltens nicht weiter behandelt, da dies den Rahmen der Seminararbeit sprengen würde.
\end{description}