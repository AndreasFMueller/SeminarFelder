%
% helmholtz.tex -- Geht auf die Helmholzschen Wirbelsätze ein
%
% !TEX root = ../../buch.tex
% !TEX encoding = UTF-8
%
\section{Helmholzsche Wirbelsätze}

\subsection{Historisches}

Um das Verhalten von Wirbelringen besser zu verstehen, sind die sogenannten helmholzschen Wirbelsätze sehr nützlich. 
Mitte 19. Jahrhundert formulierte der deutsche Physiker Hermann von Helmholtz drei Wirbelsätze und veröffentlichte diese im Journal für die reine und angewandte Mathematik \cite{Wirbelringe:JournalHelmholz}.
In seinem veröffentlichten Paper definiert er auch gleich Linien, um die Wirbelbewegungen besser beschreiben zu können:

\subsubsection*{Wirbellinien \label{paper:Wirbelringe:Wirbellinien}}

Wirbellinien sind die „Mittelachse“ eines Wirbels. 
Um diese Achse rotieren die Teile, welche Teil eines Wirbels sind. 
Diese hat an sich kein Volumen, allerdings kann es sein, dass Teilchen auf dieser Achse zu liegen kommen. 
Diese Linie kann an einem Bild eines Wirbels (zum Beispiel in Abbildung \ref{buch:papers:Wirbelringe:fig:flacher_wirbel}) beobachtet werden. 
In der Praxis ist eine Wirbellinie nicht gerade, sondern gekrümmt oder sogar spiralenähnlich. 
Ein idealer Wirbelring besitzt eine Wirbellinie in der Form eines Kreises.
Des Weiteren ist eine sehr wichtige Eigenschaft, dass Wirbellinien nur auf einer Grenzfläche enden können, 
wie wir in Abschnitt \ref{paper:Wirbelringe:Grenzflaechen} sehen werden.

\subsubsection*{Wirbelfäden \label{paper:Wirbelringe:Wirbelfaden}}

Ein Wirbelfaden ist ein Zylinder, welcher eine Wirbellinie als Zentrum des Zylinders hat. 
Schneidet man nun diesen Zylinder senkrecht zu der Wirbellinie, ergibt sich ein einzelner Wirbel. 
Wirbelfäden werden auch Wirbelröhren genannt.

\subsection{Erster Helmholzscher Wirbelsatz}

\begin{satz}
    In Abwesenheit von wirbel anfachenden äusseren Kräften bleiben wirbelfreie Strömungsgebiete wirbelfrei.
\end{satz}

Kurz gesagt, Teilchen die ruhen, bleiben in Ruhe. 
Siehe Abbildung \ref{buch:papers:Wirbelringe:fig:Helmholtz_1}: 
Alle Teilchen in Blau bewegen sich nicht, da der Betrachtungsraum abgeschlossen ist und diese {\em nicht} teil eines Wirbels sind.

\subsection{Zweiter Helmholzscher Wirbelsatz}

\begin{satz}
    Fluidelemente, die auf einer Wirbellinie liegen, verbleiben auf dieser Wirbellinie.
\end{satz}

Dies gilt auch, wenn sich diese Wirbellinie fortbewegt. 
Allerdings heisst das nicht, dass Teilchen, die sich nicht von Anfang an auf einer Wirbellinie befinden, dort nicht mehr hingelangen können
Siehe Abbildung \ref{buch:papers:Wirbelringe:fig:Helmholtz_2}:  
Alle Teilchen in Blau bewegen sich nicht, da sich die Wirbellinie (relativ zum Betrachtungsraum) nicht bewegt.

\subsection{Dritter Helmholzscher Wirbelsatz}

\begin{satz}
    Die Zirkulation entlang einer Wirbelröhre ist konstant. 
\end{satz}

Dieser Wirbelsatz kann mit dem Integral 
\[
\Gamma
= 
\oint_{c} \vec{u} \cdot d \vec{l}
=
\text{const}
\]
ausgedrückt werden. 
\(\Gamma\) ist die Zirkulation des jeweiligen Flächenstücks der Gesamtwirbelröhre, \(c\) der Umfang des jeweiligen Flächenstücks und \(\vec{u}\) die Geschwindigkeit der rotierenden Partikel. 
Siehe Abbildung \ref{buch:papers:Wirbelringe:fig:Helmholtz_3}: 
Die Zirkulation, dargestellt als Grösse der Pfeile, ist bei allen 3 dargestellten Konturen der Wirbelröhre gleich gross 
\input{papers/wirbelringe/fig/Helmholz_wirbelsätze.tex}

\subsection{Stokes auf Wirbelringe angewendet \label{paper:Wirbelringe:Stokes}}

Die Zirkulation \(\Gamma\) kann auch anders beschrieben werden. 
Ein Wirbelring hat eine Wirbelstärke \(\omega\) welche als
\[
\omega
=
\operatorname{rot}\left( \vec{u} \right)
\]
definiert ist.
Integriert man $\omega$ über eine Fläche \(A\) mit Rand \(c\), erhält man nach dem Satz von Stokes (Satz \ref{buch:green:green:satz:stokes})
\begin{align*}
\iint_{A} \vec{\omega} \cdot d \vec{A}
&=
\iint_{A} \operatorname{rot}\left(\vec{u}\right)\cdot  d \vec{A}\\
&=
\int_{\partial A} \vec{u} \cdot d\vec{l},
\end{align*}
also die Zirkulation.

\subsection{Verhalten an Grenzflächen \label{paper:Wirbelringe:Grenzflaechen}}

Mit den vorhergegangenen Regel kann das Verhalten von Wirbelringen nun einfacher nachvollzogen werden. 
Allerdings können noch nicht alle Verhalten begründet werden. 
Bisher wurde nur das Verhalten im freien Raum betrachtet und Grenzflächen ignoriert. 
Allerdings sind diese wichtig, da in der Realität immer solche Grenzflächen vorhanden sind. 
Grenzflächen entstehen immer dort, wo es Übergänge von Materialien gibt oder am Rand dem Betrachtungsraum.

Aus der Identität der Quellenfreiheit von Wirbel (Gleichung \ref{paper:Wirbelringe:eq:wIdent}) lässt sich schliessen das Teilchen in einem Wirbelring Grenzflächen nicht durchqueren, da sonst die Divergenz nicht \(0\) ist. 
Allerdings können Wirbellinien darauf enden, da die Teilchen zwar auf der Grenzfläche sind, diese aber nicht überschreiten. 

Was man allerdings nicht vergessen sollte ist, bei Grenzflächen welche durch Übergänge von Materialien entstehen, die Teilchen diese Grenzfläche nicht überschreiten können. 
Zum Beispiel ein Wirbel welcher in Wasser entstanden ist, kann sich nicht auf die daran grenzende Wand ausweiten.  
