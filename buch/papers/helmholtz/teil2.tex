%
% teil2.tex -- Beispiel-File für teil2 
%
% (c) 2020 Prof Dr Andreas Müller, Hochschule Rapperswil
%
% !TEX root = ../../buch.tex
% !TEX encoding = UTF-8
%
\section{Helmholtz-Zerlegung für $\mathbb{R}^3$
\label{helmholtz:section:teil2}}
\kopfrechts{Teil 2}

Die Grundidee der Helmholtz-Zerlegung besteht darin, dass ein glattes (differenziebares) Vektorfeld $\mathbf{A}$, welches unendlich schnell genug abfällt,eindeutig in ein wirbelfreien (irrotationalen) Teil und einen quellenfreien (solenoidalen) Teil zerlegt werden kann.

\begin{equation}
\mathbf{A} =  \underbrace{-\nabla \Phi}_{Skalarpotential} + \underbrace{\nabla \times \mathbf{\Psi}}_{Vektorpotential}
\label{helmholtz:equationAllgemein}
\end{equation}


\begin{tabular}[h]{l|l|l|l}
\hline
zu zerlegende Vektorfeld & $\mathbf{A}$ & & \\
\hline 
skalares Potential & $\Phi $ & $\nabla \times (-\nabla \Phi) = 0$ & wirbelfrei\\
\hline
Vektorpotential & $\mathbf{\Psi}$ & $\nabla \cdot (\nabla \times \mathbf{\Psi}) = 0$ & quellenfrei\\
\hline
\end{tabular}

\subsection{Bedingungen
\label{helmholtz:subsection:Bedingung}}

Damit die Zerlegung vorgenommen werden kann bzw. darf, muss das Vektorfeld $\mathbf{A}$ 

\begin{itemize}
\item \textbf{Glatt sein:} $\mathbf{A}$ stetig/kontinuirlich differenziebar sein (in $\mathbb{R}^3$ 2 mal differenziebar) in dem Gebiet
\item \textbf{Abfallverhalten:} $\mathbf{A}$ schnell abfallen  $\mathbf{A}$ schneller als $1/r$. 
\end{itemize}
(Unsicher bei dieser Aussage. Formel?)

%https://www.youtube.com/watch?v=RVlUe5gsAX4
%https://icmp.lviv.ua/journal/zbirnyk.89/13002/art13002.pdf

\subsection{Berechnung der Potentiale unbeschränkte Gebiete
\label{helmholtz:subsection:Berechnung}}

%https://people.iith.ac.in/ashok/Maths_Lectures/TutorialB/Topic_03_(Helmholtz%27%2520Theorem).pdf

Berechnung der Potentiale $\Phi $ und $\mathbf{\Psi}$ lassen sich mit Hilfe der Greenschen Funktion herleiten. Wer Wie Wo Was?


\begin{itemize}
\item \textbf{skalares Potential}
\begin{equation}
\phi = \frac{1}{4 \pi} \int_{V} \frac{\nabla \cdot \mathbf{A}}{\mu}, dV
\end{equation}
\item \textbf{Vektorpotential}
\begin{equation}
\mathbf{\Psi} = \frac{1}{4 \pi} \int_{V} \frac{\nabla \times \mathbf{A}}{\mu}, dV
\end{equation}
\item $\vec{\mu} = \mid \vec{r} - \vec{r}^{\prime} \mid$ was ist das genau?
\end{itemize}

\subsection{Berechnung der Potentiale beschränktem Gebiet
\label{helmholtz:subsection:BerechnungBeschr}}

endliches (beschränktes) Volumen $V$ Oberflächenintegrale ?

\begin{itemize}
\item \textbf{skalares Potential}
\begin{equation}
\Phi (\mathbf{r}) = \frac{1}{4\pi} \int_V \frac{\nabla' \cdot \mathbf{F}(\mathbf{r}')}{|\mathbf{r} - \mathbf{r}'|} dV' + \frac{1}{4\pi} \oint_S \frac{\mathbf{F}(\mathbf{r}')}{|\mathbf{r} - \mathbf{r}'|} dxXx'
\end{equation}
\item \textbf{Vektorpotential}
\begin{equation}
\mathbf{\Psi}(\mathbf{r}) = \frac{1}{4\pi} \int_V \frac{\nabla' \times \mathbf{F}(\mathbf{r}')}{|\mathbf{r} - \mathbf{r}'|} dV' + \frac{1}{4\pi} \oint_S \frac{\mathbf{F}(\mathbf{r}')}{|\mathbf{r} - \mathbf{r}'|} dxXx'
\end{equation}
\end{itemize}


\subsection{Eindeutigkeit der Zerlegung 
\label{helmholtz:subsection:EindeutigkeitS}}

\begin{itemize}
\item \textbf{Für unbeschränkte Gebiete}
\item \textbf{Für beschränkte Gebiete}
\end{itemize}


\subsection{Orthogonalität
\label{helmholtz:subsection:Orthogonalitaet}}

