%
% Finite Volumina
%
\section{Finite Volumina
\label{buch:pdenumerik:section:fvm}}
\index{finite Volumina}%
\index{Volumina, finite}%
Für gewisse Erhaltungssätze wie zum Beispiel die Kontinuitätsgleichung
\index{Erhaltungssatz}%
ist eine alternative Diskretisation besser geeignet.
\index{Kontinuitätsgleichung}%
Wir betrachten als Beispiel die eindimensionale Kontinuitätsgleichung
in der eindimensionalen Form
\[
\frac{\partial \varrho}{\partial t}
+
\frac{\partial j}{\partial x}
=
0.
\]
Sie drückt aus, dass die zeitliche Dichteänderung davon abhängt, wie
schnell der zugehörige Strom $j$ von $x$ abhängt.
\index{Strom}
Die zugehörige dreidimensionale Form ist
\[
\frac{\partial \varrho}{\partial t}
+
\operatorname{div}\vec{\jmath} 
=
0.
\]

Statt die Werte in einem Punkt zu betrachten, wie das die Methode
der finiten Differenzen macht, betrachten wir jetzt die Ereignisse
ein einem kleinen Volumen.
Integrieren wir die Kontinuitätsgleichung über einen Quader $V$, dann
kann das Volumenintegral der Divergenz mit dem Satz von Gauss 
\index{Satz von Gauss}%
\index{Gauss, Satz von}%
in ein Oberflächenintegral umgewandelt werden:
\begin{align}
0
&=
\int_V \frac{\partial\varrho}{\partial t}\,dV
+
\int_V \operatorname{div}\vec\jmath\,dV
\notag
\\
&=
\int_V \frac{\partial\varrho}{\partial t}\,dV
+
\int_{\partial V} \vec\jmath\,d\vec{n}
\notag
\intertext{und die Zeitableitung kann aus dem Intergral genommen
werden.
Es bleibt die Gleichung}
0
&=
\frac{\partial}{\partial t}
\int_V\varrho\,dV
+
\int_{\partial V} \vec{\jmath}\,d\vec{n}.
\label{buch:pdenumerik:fvm:eqn:integralform}
\end{align}
Dies ist die Integralform der Kontinuitätsgleichung.

Die Gleichung~\eqref{buch:pdenumerik:fvm:eqn:integralform}
suggeriert eine alternative Art der Diskretisation.
Statt einer Variablen, die für den Wert von $\varrho$ in den Gitterpunkten
stehen, verwenden wir Variablen, die für den Wert von $\varrho$ im ganzen
Volumen $V$ stehen.
Statt eines Gitters zerlegen wir den Definitionsbereich daher
in kleine Polyeder $V$. 
Für den Strom brauchen wir dann Variablen, die für den Strom durch 
jede Seitenfläche des Polyeders stehen.

Im eindimensionalen Fall ist das Volumen einfach nur ein kleines 
Intervall $V_k = [x_k,x_{k+1}]$.
Die Variable $j_k$ ist die Stromdichte im Punkt $x_k$.
Die Variable $\varrho_k$ steht für die Dichte im ganzen Intervall $V_k$.
Die Gleichung \eqref{buch:pdenumerik:fvm:eqn:integralform}
wird jetzt zu
\[
0
=
\frac{\partial}{\partial t}
\varrho_k\cdot|V_k|
+
(j_{k+1}-j_{k}).
\]
Für die Zeitdiskretisation können wir wieder finite Differenzen
verwenden und damit
\[
\frac{
\varrho_k(t+\Delta t)
\varrho_k(t)
}{
\Delta t
}
=
\frac{j_k-j_{k+1}}{|V_{k}|}
=
\frac{j_k-j_{k+1}}{x_{k+1}-x_k}.
\]

Im mehrdimensionalen Fall müssen die Variablen $\vec{\jmath}_k$ mit
der Flächennormale der Seitenflächen des Volumens $V_k$ multiplizieren
und addieren.
Die Variablen $\varrho_k$ müssen mit dem Volumen $|V_k|$ multipliziert
werden.
So entstehen wieder lineare Gleichungen für die Variablen $\varrho_k$
und $\vec{\jmath}_k$.




