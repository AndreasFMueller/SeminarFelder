%
% einleitung.tex -- Einleitung und Motivation
%
% (c) 2020 Prof Dr Andreas Müller, Hochschule Rapperswil
%
% !TEX root = ../../buch.tex
% !TEX encoding = UTF-8
%

\section{Einleitung\label{fourier:section:einleitung}}
%\kopfrechts{Grundlagen der Fourier-Analyse}

%1.Version
%Felder kann man sich als eine unendliche Ansammlung von Feldlinien vorstellen, die sich kreuz und quer durch Raum-Zeit schlängeln. 
%Die Fourier-Analyse hilft dabei, dieses scheinbare Chaos in einfache Sinus- und Cosinus-Schwingungen zu zerlegen. 
%In der Quantenfeldtheorie werden diese Schwingungen anders interpretiert, was letztlich dazu führt, dass das Feld als Summe einer beschränkten Anzahl von Teilchen verstanden wird.


%2.Version
Man kann sich Felder wie ein unsichtbares Gewebe vorstellen, das aus unzähligen Linien besteht, die sich durch Raum und Zeit winden. 
Auf den ersten Blick wirkt dieses Geflecht unstrukturiert und chaotisch. 
Mit der Fourier-Analyse lässt sich jedoch eine verborgene Ordnung erkennen, denn sie zerlegt das komplexe Muster in reine Sinus- und Kosinusschwingungen. 
In der Quantenfeldtheorie verändert sich die Bedeutung dieser Schwingungen grundlegend. 
Dort werden sie nicht mehr nur als mathematische Bausteine verstanden, sondern als Ausdruck dafür, dass sich ein Feld aus einer endlichen Anzahl von Teilchen zusammensetzt. 
Aus einer scheinbar unendlichen und kontinuierlichen Struktur entsteht so ein zählbares System, in dem Wellen und Teilchen untrennbar miteinander verbunden sind.







