%
% einleitung.tex -- Beispiel-File für die Einleitung
%
% (c) 2020 Prof Dr Andreas Müller, Hochschule Rapperswil
%
% !TEX root = ../../buch.tex
% !TEX encoding = UTF-8
%
\section{Simulation\label{particles:section:simulation}}
\kopfrechts{Simulation}

Für die konventionelle Wellengleichung~\eqref{particles:eq:wellengleichung} 
können Lösungen besonders für einfache Fälle auch analytisch gefunden werden.
Sobald jedoch das Medium nichtlinear und 
der Proportionalitätsfaktor $c$ nicht mehr unabhängig von der Deformation $u$ ist, 
wird eine analytische Lösung beinahe unmöglich.
Entsprechend wurde in der Hinsicht darauf, 
dass schliesslich auch nichtlineare Medien simuliert werden können, 
ein numerischer Ansatz gewählt.

So wurde für die Simulationen eine von \textbf{insert name} angepasste Version des Open-Source Wellensimulators WaveSimulator2D~\cite{repo:wavesim2d} verwendet. % TODO: Insert name and cite
Dieser simuliert eine zweidimensionale Szene, indem er die Wellengleichung für einen diskreten Zeitschritt simuliert und das resultierende Feld anzeigt.
Die Szene wird dabei ebenfalls örtlich diskretisiert, indem sie in Pixel aufgeteilt wird.

\subsection{Wellengleichung\label{particles:section:simulation:wellengleichung}}
Die Diskretisierung der Raumdimensionen erlaubt die Berechnung von $\nabla^2 u$ als Faltung des Felds $u$ mit einem Laplace kernel.
WaveSim2D verwendet dazu den Kernel
\[
    \mathbf{D}^2_{xy} = 
    \begin{pmatrix}
        0.066 &  0.184 & 0.066\\
        0.184 & -1.000 & 0.184\\
        0.066 &  0.184 & 0.066\\
    \end{pmatrix},
\]
was zu der angepassten Wellengleichung 
\[
    \frac{\d^2 u}{\d t^2} = c^2 \cdot (\mathbf{D}^2_{xy} * u)
\]
führt.

Für die Ableitung im Zeitbereich wird der zentrale Differenzquotient
\[
    \frac{\d^2 u(t)}{\d t^2} \approx \frac{u(t-T) - 2u(t) + u(t+T)}{T^2}
\]
eingesetzt.
$T$ ist dabei der diskrete Zeitschritt der Simulation.

Die angepasste Wellengleichung 
\[
    \frac{u(t-T) - 2 u(t) + u(t+T)}{T^2} = c^2 \cdot (\mathbf{D}^2_{xy} * u)
\]
kann nun nach $u(t+T)$, also nach dem Wert von $u$ einen Zeitschritt in der Zukunft, umgestellt werden, was in
\[
    \underbrace{%
        u(t+T) = 2 u(t) - u(t+T) + T^2      % Equation part 1
        \vphantom{\mathbf{D}^2_{xy}}        % Adjust height
    }_{%
        \textstyle
        \frac{\d^2 u(t)}{\d t^2}            % Description
    } 
    \cdot 
    \underbrace{%
        c^2 \cdot (\mathbf{D}^2_{xy} * u)   % Equation part 2
    }_{%
        \textstyle
        \nabla^2 u                          % Description
        \vphantom{\frac{\d^2 u(t)}{\d t^2}} % Adjust height
    }
\]
resultiert.
Diese Gleichung kann verwendet werden, um ungedämpfte Wellen in linearen Medien zu simulieren.

\subsection{Dämpfung}
Damit der später erwähnte Farbkanal zur Dämpfung des Mediums realisiert werden kann, 
wird die Wellengleichung durch einen Dämpfungsterm erweitert.
Die angepasste Wellengleichung 
\[
    \frac{\d^2 u}{\d t^2} + d \frac{\d u}{\d t}= c^2 \cdot \nabla^2 u
\]
mit der Dämpfung $d$ als Skalarfeld kann entlang dem selben Weg wie in Abschnitt~\ref{particles:section:simulation:wellengleichung} dargestellt nach $u(t+T)$ umgeformt werden.
Dabei wird für die Ableitung $\frac{\d u}{\d t}$ der einseitige Differenzquotient
\[
    \frac{\d u}{\d t} \approx \frac{ - u(t-T)}{T}
\]
verwendet.

Die resultierende Wellengleichung lautet
\[
    \underbrace{%
        u(t+T) = 2 u(t) - u(t+T) + T^2      % Equation part 1
        \vphantom{\mathbf{D}^2_{xy}}        % Adjust height
    }_{%
        \textstyle
        \frac{\d^2 u(t)}{\d t^2}            % Description
    } 
    \cdot 
    \underbrace{%
        c^2 \cdot (\mathbf{D}^2_{xy} * u)   % Equation part 2
    }_{%
        \textstyle
        \nabla^2 u                          % Description
        \vphantom{\frac{\d^2 u(t)}{\d t^2}} % Adjust height
    }
     - 
    \underbrace{%
        d \cdot T \cdot (u(t) - u(T-1))
        \vphantom{\mathbf{D}^2_{xy}}        % Adjust height
    }_{%
        \textstyle
        d \cdot \frac{\d u(t)}{\d t}        % Description
    }.
\]

\subsection{Nichtlinearität}
Die Nichtlinearität des Mediums wird durch die Formel
\[
    \text{stress} = a \cdot \text{strain} \cdot e^{b \cdot \text{strain}} % TODO: Tatsächliche Formel
\]
gegeben, wobei das nichtlineare Verhalten durch die Parameter $a$ und $b$ angepasst werden kann.
Einige Beispiele der daraus resultierenden Spannungs-Deformations-Kurven sind in Abb.~\ref{particles:abb:nonlin} gezeigt. % TODO: Abbildung
\mytodo{Beschreiben, wie der Simulator das einrechnet}

\subsection{Szenendefinition}
Zur Definition des Ausgangszustands des Feldes sowie der Eigenschaften des Mediums nimmt der Simulator ein RGB-Bild entgegen.
Den drei Farbkanälen wird jeweils eine Funktion zugeteilt.
Der rote Kanal bestimmt den Brechungsindex des jeweiligen Pixels, während über den blauen Kanal die Dämpfung bestimmt werden kann.
Über den grünen Kanal können im unveränderten Simulator Quellen dargestellt werden, wobei der Farbwert die Frequenz der Quelle darstellt.
Damit die in der Simulation enthaltene Energie nicht fortlaufend ansteigt, wurde die Funktion des grünen Kanals durch \textbf{insert name} angepasst. % TODO: Echter Name von Huygens einfügen
Der grüne Kanal diktiert nun das Spannungsfeld zu Beginn der Simulation. % CHECK: ist es die Spannung oder die Deformation?
% TODO: Beispielszene

\subsection{Performance} % TODO: Evtl. weg lassen...
Die Simulation ist trotz relativ tiefer Auflösung nicht trivial. 
Der WaveSimulator2D zieht Nutzen aus der \texttt{CuPy} Bibliothek um die Faltung des Laplace Kernels mit den aktuellen Feldwerten auf der Grafikkarte laufen zu lassen.
