%
% Feldgleichunge, die sich durch zusätzliche Annahmen ergeben
%
\section{Feldgleichungen, die sich durch zusätzliche Annahmen ergeben}
\kopfrechts{Zusätzliche Annahmen}
Oft lassen sich Feldgleichungen mit Hilfe zusätzlicher Annahmen
vereinfachen.
Eine stationäre Lösung einer partiellen Differentialgleichung ist
eine Lösung, die nicht von der Zeit abhängt, in der also alle
Zeitableitungen verschwinden.

\subsection{Die Poisson-Gleichung}
In der Elektrostatik sind die Ladungen, die das elektrostatische
Feld erzeugen, unbeweglich, und das Feld ändert sich mit der Zeit
nicht.
Aus den Maxwell-Gleichungen, die in Kapitel~\ref{chapter:4d}
ausführlicher diskutiert werden, folgt dann, dass das elektrische
Feld ein Potentialfeld ist und dass das Potential $\varphi$ die
Differentialgleichung
\begin{equation}
\Delta \varphi = 4\pi \varrho
\label{buch:feldgleichungen:zusätzlich:eqn:poisson}
\end{equation}
erfüllt, wobei $\varrho$ die Ladungsdichte ist.
Die Gleichung~\eqref{buch:feldgleichungen:zusätzlich:eqn:poisson}
heisst auch die {\em Poisson-Gleichung}.
\index{Poisson-Gleichung}%
Sie tritt zum Beispiel auch bei Minimalflächen in linearer Näherung
auf, wenn zwischen den beiden Seiten der Fläche ein Druckunterschied
vorliegt.

\subsection{Die Laplace-Gleichung}
Aus der Poisson-Gleichung \eqref{buch:feldgleichungen:zusätzlich:eqn:poisson}
ergibt sich im Vakuum, wo keine Ladungen vorhanden sind, die noch
einfachere Gleichung
\[
\Delta \varphi = 0,
\]
die auch die {\em Laplace-Gleichung} genannt wird.
\index{Laplace-Gleichung}%
Sie tritt auch im Grenzfall einer stationären Temperaturverteilung
auf, denn in diesem Fall verschwindet die Zeitableitung in der
Wärmeleitungsgleichung.
Die Lösungen der Laplace-Gleichung heissen auch {\em harmonische} 
Funktionen.

\subsection{Die Helmholtz-Gleichung}
Die Wellengleichung in zwei Dimension hat die Form
\[
\frac{1}{a^2}
\frac{\partial^2u}{\partial t^2}
=
\frac{\partial^2 u}{\partial x^2}
+
\frac{\partial^2 u}{\partial y^2}.
\]
Auch hier kann man einen Separationsansatz versuchen und
\[
u(x,y,t) = T(t) v(x,y)
\]
ansetzen.
Einsetzen in die Differentialgleichung ergibt
\[
\frac{1}{a^2}
T''(t) v(x,y)
=
T(t)
\biggl(
\frac{\partial^2 v}{\partial x^2}
+
\frac{\partial^2 v}{\partial y^2}
\biggr).
\]
Division durch $u$ separiert die Variablen, den in
\[
\frac{1}{a^2}
\frac{T''(t)}{T(t)}
=
\frac{\delta v(x,y)}{v(x,y)}
\]
hängt die linke Seite nur von $t$, die rechte nur von $x$ und $y$ ab.
Beide Seiten sind daher konstant.
Wir bezeichnen die Konstante mit mit $-\lambda^2$ und bekommen die
beiden separierten Differentialgleichungen
\[
T''(t) = -a^2\lambda^2 T(t)
\qquad\text{und}\qquad
\Delta v = -\lambda^2 v.
\]
Die Differentialgleichung für $T$ ist eine gewöhnliche
Schwingungsdifferentialgleichung mit der allgemeinen Lösung
\[
T(t) = A\cos a\lambda t + B \sin a\lambda t.
\]
Es muss daher nur noch die zweite Differentialgleichung gelöst
werden.
In allgemeinster Form ist dies die Gleichung
\begin{equation}
\Delta v = \lambda v,
\end{equation}
die auch die {\em Helmholtz-Gleichung} heisst.
\index{Helmholtz-Gleichung}%


Bei der schwingenden Saite reduziert sich die Helmholtz-Gleichung
auf die Differentialgleichung
\[
X''(x) = \lambda X(x),
\]
die sich ebenfalls mit trigonometrischen Funktionen lösen lässt.
Auf einem endlichen Intervall der Länge $l$ mit Dirichlet-Randbedingungen
sind Lösungen nur für $\lambda=n\pi/l$, $n\in\mathbb{N}$ möglich.

Die allgemeine Theorie der Helmholtz-Gleichung zeigt, dass für
kompakte Gebiete mit ausreichend glattem Rand und Dirichlet-Randbedingungen
eine diskrete Folge $(\lambda_n)_{n\in\mathbb{N}}$ von möglichen Werten
für $\lambda$ mit jeweils endlichdimensionalen Lösungsraum der Dimension
$k(n)$ existiert, aus denen sich alle Lösungen
\[
u(x,t)
=
\sum_{n=0}^\infty
\sum_{k=1}^{k(n)}
\bigl(
a_n
\cos a\lambda_n t
+
b_n
\sin a\lambda_n t
\bigr)
u_{n,k}(x)
\]
zusammensetzen lassen, wobei die Funktionen $u_{n,k}$, $k=1,\dots,k(n)$
eine Basis des Lösungsraumes der Helmholtz-Gleichung für $\lambda=\lambda_n$
ist.

Auch aus der Wärmeleitungsgleichung auf dem Gebiet $G$ entsteht
mit Hilfe des obigen Separationsansatzes eine Helmholtz-Gleichung.

