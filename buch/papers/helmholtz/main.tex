%
% main.tex -- Paper zum Thema <helmholtz>
%
% (c) 2020 Autor, OST Ostschweizer Fachhochschule
%
% !TEX root = ../../buch.tex
% !TEX encoding = UTF-8
%
\chapter{Helmholtz-Zerlegung\label{chapter:helmholtz}}
\kopflinks{Helmholtz-Zerlegung}
\begin{refsection}
\chapterauthor{Joël Rechsteiner}

Im Mittelpunkt in diesem Kapitel steht die Helmholtz-Zerlegung.
Anhand eines Beispiels aus der Physik in der Akustik aus dem Paper
\cite{helmholtz:paper}
wird eine Anwendung der Helmholtz-Zerlegung
\index{Helmholtz-Zerlegung}%
dargestellt, wo die komplexe akustischen Intensität in einem
kartesischen Koordinatensystem in wirbelfreie
\index{wirbelfrei}%
und quellenfreie Anteile zerlegt wird.
\index{quellenfrei}%
Allerdings kann die Helmholtz-Zerlegung auch auf ein beliebiges
Koordinatensystem angewendet werden, worauf hier nicht weiter
eingegangen wird.

Schritt für Schritt wird erklärt, wie die relevanten Gleichungen
mit mathematischen Tools vorangegangener Kapitel der Vektoranalysis
mit der physikalischen Deutung zusammenhängen.
Anschliessend wird die Aufteilung der akustischen Intensität in
aktive und reaktive Komponenten aufgezeigt.
\index{aktive Intensitat@aktive Intensität}%
\index{reaktive Intensitat@reaktive Intensität}%
Die Ableitung der Differentialgleichungen für aktive und reaktive
Intensitätsfelder und die Wirbelbildung in der Intensitätsverteilung
zeigen, wie diese Konzepte in numerischen Simulationen wie zum
Beispiel in COMSOL und praktischen Messungen etwa zur Interpretation
\index{COMSOL}%
von Nah- und Fernfeld oder der Quellenkopplung angewendet werden.


%
% teilNeu2.tex -- Beispiel-File für Teil 2
%
% (c) 2020 Prof Dr Andreas Müller, Hochschule Rapperswil
%
% !TEX root = ../../buch.tex
% !TEX encoding = UTF-8
%
\section{Mahtematische Grundlagen
\label{helmholtz:section:Mahtematische_Grundlagen}}
\kopfrechts{Mahtematische Grundlagen}

Hier werden kurz die Tools vorgestellt, welche für die Helmholtz-Zerlegung benötigt werden. Um die Helmholtz-Zerlegung in der Akustik zu verstehen, ist ein solides Verständnis der grundlegenden mathematischen Konzepte notwendig. In diesem Abschnitt werden die wichtigsten Vektoroperatoren und ihre Eigenschaften vorgestellt, die für die Zerlegung von Vektorfeldern benötigt werden.

\subsection{Vektorfelder und Operatoren
\label{helmholtz:subsection:Vektorfelder_Operatoren}}

Vektorfelder stellen in jedem Punkt eines Raumes einen Vektor dar, der sowohl eine Richtung als auch einen Betrag besitzt. In der Akustik sind Vektorfelder von zentraler Bedeutung, da sie physikalische Größen wie die Schallschnelle und die Intensität beschreiben. Um diese Felder zu analysieren, werden spezielle Operatoren verwendet, die lokale Eigenschaften des Feldes charakterisieren.

\subsubsection{Der Gradient eines Skalarfeldes}

Der Gradient eines Skalarfeldes $a$ ist ein Vektorfeld, das in jedem Punkt in Richtung des steilsten Anstiegs von $a$ zeigt. Mathematisch wird der Gradient durch folgende Gleichung definiert
\begin{equation}
\boldsymbol{\nabla} a (\boldsymbol{r}) = \frac{\partial a}{\partial x}\boldsymbol{e}_x + \frac{\partial a}{\partial y}\boldsymbol{e}_y + \frac{\partial a}{\partial z}\boldsymbol{e}_z.
\end{equation}



\subsubsection{Die Divergenz eines Vektorfeldes}

Die Divergenz eines Vektorfeldes $\boldsymbol{F}$ ist ein Skalarfeld, das beschreibt, wie stark Feldlinien auseinanderstreben oder zusammenlaufen. Sie wird durch folgende Formel ausgedrückt
\begin{equation}
\nabla \cdot \boldsymbol{F}(\boldsymbol{r}) = \frac{\partial F_x}{\partial x} + \frac{\partial F_y}{\partial y} + \frac{\partial F_z}{\partial z}.
\end{equation}
Die Divergenz kann als Mass für die Quellen- oder Senken-stärke an einem bestimmten Punkt im Vektorfeld interpretiert werden. Ein Bereich mit positiver Divergenz stellt eine Quelle dar, aus der Feldlinien hervorgehen, während ein Bereich mit negativer Divergenz eine Senke darstellt, in der Feldlinien zusammenlaufen, wie in Abbildung \ref{fig:DivergenzAlg} dargestellt.

\begin{figure}
    \centering
    \includegraphics[scale=0.4]{papers/helmholtz/images/divergentes_Feld.png}
    \caption{Beispiel eines divergenten Vektorfeldes, das Quellen und Senken aufweist}
    \label{fig:DivergenzAlg}
\end{figure}

\subsubsection{Die Rotation eines Vektorfeldes}

Die Rotation eines Vektorfeldes $\boldsymbol{A}$ ist ein Vektorfeld, das die Wirbeldichte des ursprünglichen Feldes beschreibt. Sie wird definiert als
\begin{equation}
\nabla \times \boldsymbol{A}(\boldsymbol{r}) = \begin{vmatrix}
    \boldsymbol{e}_x & \boldsymbol{e}_y & \boldsymbol{e}_z \\
    \frac{\partial}{\partial x} & \frac{\partial}{\partial y} & \frac{\partial}{\partial z}\\
    A_x & A_y & A_z
\end{vmatrix}.
\end{equation}
Die Rotation misst, wie stark sich die Feldlinien um eine Achse drehen und gibt sowohl die Achsenorientierung als auch die Stärke eines solchen Wirbels an, wie in Abbildung \ref{fig:RotationAlg} zu sehen ist.

\begin{figure}
    \centering
    \includegraphics[scale=0.4]{papers/helmholtz/images/rotierendes_Feld.png}
    \caption{Beispiel eines rotierenden Vektorfeldes mit deutlich erkennbarer Wirbelstruktur}
    \label{fig:RotationAlg}
\end{figure}

\subsubsection{Der Laplace-Operator}

Der Laplace-Operator, oft als $\nabla^2$ oder $\Delta$ geschrieben, ist ein skalarer Differentialoperator zweiter Ordnung. Angewendet auf ein Skalarfeld $a$ ergibt er
\begin{equation}
\nabla^2 a(\boldsymbol{r}) = \frac{\partial^2 a}{\partial x^2} + \frac{\partial^2 a}{\partial y^2} + \frac{\partial^2 a}{\partial z^2}.
\end{equation}
Der Laplace-Operator spielt eine zentrale Rolle in vielen physikalischen Gleichungen, insbesondere in der Potentialtheorie und der Wellenausbreitung. Für ein Vektorfeld $\boldsymbol{F}$ wird der Laplace-Operator komponentenweise angewendet und das Resultat kann wie in \ref{fig:LaplaceAlg} dargestellt aussehen. Die Formel für den Laplace-Operation ist
\begin{equation}
\nabla^2 \boldsymbol{F} = (\nabla^2 F_x)\boldsymbol{e}_x + (\nabla^2 F_y)\boldsymbol{e}_y + (\nabla^2 F_z)\boldsymbol{e}_z.
\end{equation}

\begin{figure}
    \centering
    \includegraphics[scale=0.4]{papers/helmholtz/images/Laplace_Feld.png}
    \caption{Visualisierung des Laplace-Operators auf ein Skalarfeld}
    \label{fig:LaplaceAlg}
\end{figure}


\subsection{Vektoridentitäten
\label{helmholtz:subsection:Vektoridentitaeten}}
Für die mathematische Herleitung der Helmholtz-Zerlegung werden verschiedene Vektoridentitäten benötigt.

\subsubsection{Wichtige Vektoridentitäten}

Die folgenden Identitäten sind für die Ableitung der Helmholtz-Zerlegung besonders relevant:

\begin{itemize}
\item Für jedes Skalarfeld $\Phi$ gilt: $\nabla \times (\nabla \Phi) = 0$.
\item Für jedes Vektorfeld $\boldsymbol{F}$ gilt: $\nabla \cdot (\nabla \times \boldsymbol{F}) = 0$.
\item Der Laplace-Operator eines Vektorfeldes lässt sich umformen zu:
\begin{equation}
\nabla^2 \boldsymbol{F} = \nabla(\nabla \cdot \boldsymbol{F}) - \nabla \times (\nabla \times \boldsymbol{F}).
\end{equation}
Diese Identität ist fundamental für die Ableitung der Helmholtz-Zerlegung.
\end{itemize}





%
% teil3.tex -- Beispiel-File für Teil 3
%
% (c) 2020 Prof Dr Andreas Müller, Hochschule Rapperswil
%
% !TEX root = ../../buch.tex
% !TEX encoding = UTF-8
%
\section{Die Helmholtz-Zerlegung
\label{helmholtz:section:Helmholtz_Zerlegung}}
\kopfrechts{Helmholtz-Zerlegung}

Die Grundidee der Helmholtz-Zerlegung besteht darin, dass ein
differenzierbares  Vektorfeld $\boldsymbol{F}$, welches im Unendlichen
schnell genug abfällt, sich in zwei in noch zu spezifizierendem
Sinne orthogonale Komponenten zerlegen lässt: einen eindeutig
wirbelfreien Teil und einen quellenfreien Teil.
Diese Zerlegung spiegelt die physikalischen Eigenschaften des
Vektorfeldes wider und ist insbesondere in der Akustik, Strömungsmechanik
\index{Stromungsmechanik@Strömungsmechanik}%
und Elektrodynamik von zentraler Bedeutung.
\index{Elektrodynamik}%

\subsection{Formale Definition der Helmholtz-Zerlegung
\label{helmholtz:subsection:def_Helmholtz_Zerlegung}}

Mit der Idee der Helmholtz-Zerlegung wird das Vektorfeld geschrieben als
\begin{equation}
\underbrace{
\boldsymbol{F}
}_{
\textstyle\text{zu zerlegendes Vektorfeld}
}
=
\underbrace{
-\nabla \Phi
}_{
\textstyle\text{irrotationaler Teil}
}
+
\underbrace{
\nabla \times \boldsymbol{\Psi}
}_{
\textstyle\text{solenoidaler Teil}
}
\label{helmholtz:equationAllgemein}
\end{equation}
und die einzelnen Komponenten lässt sich wie folgt beschreiben:

\begin{itemize}
\item
Der erste Term $ -\nabla \Phi $ ist wirbelfrei, da seine Rotation verschwindet.
\begin{itemize}
\item
Dieser Anteil hat nach 
\eqref{helmholtz:eqn:rotgradphi}
keine Rotation: $\nabla \times (-\nabla \Phi) = 0$.
(Nachweis später in Abschitt~\ref{helmholtz:subsection:math-Nachweis}..)
\item
Die Feldlinien verlaufen strahlenförmig von Quellen zu Senken.
\end{itemize}

\item
Der zweite Term $\nabla \times \boldsymbol{\Psi}$ hat gemäss der
Definition immer Divergenz gleich null und wird auch solenoidal
genannt.
\begin{itemize}
\item
Dieser Anteil hat nach
\eqref{helmholtz:eqn:divrotF}
keine Divergenz: $\nabla \cdot (\nabla \times
\boldsymbol{\Psi}) = 0$.
(Nachweis später in Abschitt~\ref{helmholtz:subsection:math-Nachweis}.)
\item
Die Feldlinien bilden geschlossene Schleifen ähnlich einem Magnetfeld.
\end{itemize}
\end{itemize}

\begin{table}
\centering
\begin{tabular}{l|l|l|l}
\hline
Komponente & Potential & Eigenschaft & Bezeichnung \\
\hline 
Irrotationaler Teil & $\Phi$ (Skalarpotential) & $\nabla \times (-\nabla \Phi) = 0$ & wirbelfrei\\
Solenoidaler Teil & $\boldsymbol{\Psi}$ (Vektorpotential) & $\nabla \cdot (\nabla \times \boldsymbol{\Psi}) = 0$ & quellenfrei\\
\hline
\end{tabular}
\caption{Übersicht der Helmholtz-Zerlegung}
\label{tab:helmholtz_overview}
\end{table}

In der Tabelle \ref{tab:helmholtz_overview} ist die komplementäre
Eigeschaft, die Bezeichnung und die Potentialbeschreibung der
jeweiligen Komponenten zusammengefasst.

Um die Zerlegung anwenden zu können, müssen die Potentiale $\Phi$
und $\boldsymbol{\Psi}$ berechnet werden.
Dies kann auf verschiedene Weisen erfolgen, abhängig von den Randbedingungen.

\subsection{Mathematischer Nachweis der Eigenschaften
\label{helmholtz:subsection:math-Nachweis}}

\subsubsection{Nachweis der Wirbelfreiheit des irrotationalen Anteils
(Identität \eqref{helmholtz:eqn:rotgradphi})}
Sei $\Phi$ ein zweimal stetig differenzierbares Skalarfeld.

\begin{enumerate}
    \item Zuerst wird der Gradient $\nabla\Phi$ berechnet:
    Der Gradient ergibt
    \[
    \nabla \Phi =
	\renewcommand{\arraystretch}{2.0}
    \begin{pmatrix}
        \displaystyle \frac{\partial \Phi}{\partial x} \\
        \displaystyle \frac{\partial \Phi}{\partial y} \\
        \displaystyle \frac{\partial \Phi}{\partial z}
    \end{pmatrix}.
    \]

    \item Nun wird die Rotation dieses Gradientenfeldes berechnet:
    Die Definitionsformel
    der Rotation $\nabla \times \boldsymbol{F}$ auf den Gradientenvektor
    anwenden ergibt
    \[
    \nabla \times (\nabla \Phi) =
	\renewcommand{\arraystretch}{2.0}
    \begin{pmatrix}
        \displaystyle\frac{\partial}{\partial y}\left(\frac{\partial \Phi}{\partial z}\right) - \frac{\partial}{\partial z}\left(\frac{\partial \Phi}{\partial y}\right) \\
        \displaystyle\frac{\partial}{\partial z}\left(\frac{\partial \Phi}{\partial x}\right) - \frac{\partial}{\partial x}\left(\frac{\partial \Phi}{\partial z}\right) \\
        \displaystyle\frac{\partial}{\partial x}\left(\frac{\partial \Phi}{\partial y}\right) - \frac{\partial}{\partial y}\left(\frac{\partial \Phi}{\partial x}\right)
    \end{pmatrix}.
    \]

    \item %Den \emph{Satz von Schwarz} anwenden:
    
    Zur besseren Übersicht werden die gemischten Ableitungen ausgeschrieben:
    \[
    \nabla \times (\nabla \Phi) =
	\renewcommand{\arraystretch}{2.0}
    \begin{pmatrix}
        \displaystyle\frac{\partial^2 \Phi}{\partial y\, \partial z} - \frac{\partial^2 \Phi}{\partial z\, \partial y} \\
        \displaystyle\frac{\partial^2 \Phi}{\partial z\, \partial x} - \frac{\partial^2 \Phi}{\partial x\, \partial z} \\
        \displaystyle\frac{\partial^2 \Phi}{\partial x\, \partial y} - \frac{\partial^2 \Phi}{\partial y\, \partial x}
    \end{pmatrix}.
    \]
    Der \emph{Satz von Schwarz} besagt, dass bei stetigen differenzierbaren
    Funktionen die Reihenfolge der partiellen Ableitungen vertauscht werden
    kann.
    Damit heben sich die Terme in jeder Komponente gegenseitig auf
    und es bleibt
    \[
    \nabla \times (\nabla \Phi)  =
    \begin{pmatrix}
        0 \\
        0 \\
        0
    \end{pmatrix} = \boldsymbol{0}.
    \]
\end{enumerate}
Damit ist die Identität \eqref{helmholtz:eqn:rotgradphi} beweisen.

\subsubsection{Nachweis der Quellenfreiheit des solenoidalen Anteils
(Identität \eqref{helmholtz:eqn:divrotF})}

\begin{enumerate}
    \item Berechnung der Rotation $\nabla \times \boldsymbol{\Psi}$:
	Zuerst wird der Rotations-Operator auf $\boldsymbol{\Psi}$ angewendet:
    \[
    \nabla \times \boldsymbol{\Psi} =
	\renewcommand{\arraystretch}{2.0}
    \begin{pmatrix}
        \displaystyle\frac{\partial \Psi_z}{\partial y} - \frac{\partial \Psi_y}{\partial z} \\
        \displaystyle\frac{\partial \Psi_x}{\partial z} - \frac{\partial \Psi_z}{\partial x} \\
        \displaystyle\frac{\partial \Psi_y}{\partial x} - \frac{\partial \Psi_x}{\partial y}
    \end{pmatrix}.
    \]

    \item Berechnung der Divergenz des Ergebnisses: Nun wird der
    Divergenz-Operator auf das resultierende Vektorfeld angewendet.
    Dies geschieht durch die Summe der partiellen Ableitungen der
    Komponenten nach der jeweiligen Koordinate:
    \begin{align*}
    % Schritt 1: Definition der Divergenz anwenden
    \nabla \cdot (\nabla \times \boldsymbol{\Psi}) &= \frac{\partial}{\partial x}\left( \frac{\partial \Psi_z}{\partial y} - \frac{\partial \Psi_y}{\partial z} \right) + \frac{\partial}{\partial y}\left( \frac{\partial \Psi_x}{\partial z} - \frac{\partial \Psi_z}{\partial x} \right) + \frac{\partial}{\partial z}\left( \frac{\partial \Psi_y}{\partial x} - \frac{\partial \Psi_x}{\partial y} \right) \\
    % Schritt 2: Die Ableitungen ausführen
    &= \frac{\partial^2 \Psi_z}{\partial x\, \partial y} - \frac{\partial^2 \Psi_y}{\partial x\, \partial z} + \frac{\partial^2 \Psi_x}{\partial y\, \partial z} - \frac{\partial^2 \Psi_z}{\partial y\, \partial x} + \frac{\partial^2 \Psi_y}{\partial z\, \partial x} - \frac{\partial^2 \Psi_x}{\partial z\, \partial y} \\
    % Schritt 3: Terme umsortieren, um Paare zu bilden
    &= \left( \frac{\partial^2 \Psi_x}{\partial y\, \partial z} - \frac{\partial^2 \Psi_x}{\partial z\, \partial y} \right) + \left( \frac{\partial^2 \Psi_y}{\partial z\, \partial x} - \frac{\partial^2 \Psi_y}{\partial x\, \partial z} \right) + \left( \frac{\partial^2 \Psi_z}{\partial x\, \partial y} - \frac{\partial^2 \Psi_z}{\partial y\, \partial x} \right) = 0.
    \end{align*}
    
    \item Nach dem \emph{Satz von Schwarz} ist die Reihenfolge der
    partiellen Ableitungen bei zweimal stetig differenzierbaren
    Funktionen irrelevant.
    Daher gilt:
    \[
    \frac{\partial^2 f}{\partial x\, \partial y} = \frac{\partial^2 f}{\partial y\, \partial x}.
    \]
    Angewendet auf die obige Gleichung bedeutet dies, dass jeder
    Term in den Klammern Null ergibt:
    \begin{align*}
    \nabla \cdot (\nabla \times \boldsymbol{\Psi})
&=
\underbrace{\left( \frac{\partial^2 \Psi_x}{\partial y\, \partial z} - \frac{\partial^2 \Psi_x}{\partial y\, \partial z} \right)}_{\displaystyle=0}
+
\underbrace{\left( \frac{\partial^2 \Psi_y}{\partial z\, \partial x} - \frac{\partial^2 \Psi_y}{\partial z\, \partial x} \right)}_{\displaystyle=0}
+
\underbrace{\left( \frac{\partial^2 \Psi_z}{\partial x\, \partial y} - \frac{\partial^2 \Psi_z}{\partial x\, \partial y} \right)}_{\displaystyle=0} \\
%    &= 0 + 0 + 0 \\
    &= 0.
    \end{align*}
\end{enumerate}
Damit ist die Identität $\nabla \cdot (\nabla \times \boldsymbol{\Psi})
= 0$ bewiesen.

\subsection{Berechnung der Potentiale
\label{helmholtz:subsection:Berechnung der Potentiale}}

\subsubsection{Greenscher Ansatz}
Ein möglicher Ansatz zur Lösung der partiellen Differentialgleichung
ist die Verwendung der Dirac-$\delta$-Funktion und der
greenschen Funktionen,
\index{greensche Funktion}%
wie dies in \cite{baird_helmholtz} beschrieben ist.

Die Vorgehensweise zur Lösung der Differentialgleichung
\begin{equation}
\Delta F = b
\label{helmholtz:green:eqn:gl}
\end{equation}
ist analog zur Lösung eines linearen Gleichungsysstems
$Ax=b$.
Dazu muss eine Matrix $A^{-1}$ gefunden werden, welche $A^{-1}A=I$
erfüllt.
Die Rolle der Multiplikation $A^{-1}b$ wird übernommen von einem
Integral
\[
F(\boldsymbol{x})
=
\int_V
G(\boldsymbol{x},\boldsymbol{x}')
b(\boldsymbol{x}')
\,dV'
\]
über das Argument der Funktion auf der rechten Seite von
\eqref{helmholtz:green:eqn:gl}.
Die Rolle der Matrix wird übernommen von einer Funktion
$G(\boldsymbol{x},\boldsymbol{x}')$, der sogenannten
\emph{greenschen Funktion}.
Die Zusammensetzung mit dem Laplace-Operator muss den ``Einheitsoperator''
geben, der die Funktion reproduziert.
Die Dirac-$\delta$-Funktion hat diese Eigenschaft, denn es gilt
\[
f(\boldsymbol{x})
=
\int_V
\delta(\boldsymbol{x}-\boldsymbol{x}')
f(\boldsymbol{x}')
\,dV'.
\]
Man kann zeigen, dass 
\[
\delta(\boldsymbol{x}-\boldsymbol{x}')
=
\frac{1}{4\pi} \nabla^2 \frac{1}{|\boldsymbol{x}-\boldsymbol{x}'|}
\]
die gewünschte Eigenschaft hat,
dass also das Integral mit der Funktion
\[
G(\boldsymbol{x},\boldsymbol{x}')
=
\frac{1}{4\pi}\,
\frac{1}{|\boldsymbol{x}-\boldsymbol{x}'|}
\]
über $\boldsymbol{x}'$ die gewünschte Inverse des Laplace-Operators ist.
%Dies entspricht der Integralform einer Matrixmultiplikation, wie
%in Tabelle \ref{tab:helmholtz_matrix_analogie} veranschaulicht.
%Dabei wird die Funktion $\boldsymbol{F}$, die in diesem Fall einen
%Vektor darstellt, mit der Delta-Funktion multipliziert wie folgt:
%
%\begin{equation}
%\Delta F = -4 \pi \rho. 
%\label{helmholtz:DGL_idee}
%\end{equation}
%
%\begin{equation}
%\underbrace{\frac{1}{4 \pi} \Delta F}_{\text{invertiert}} = \rho 
%\label{helmholtz:DGL_idee_umformung}
%\end{equation}

%\begin{table}
%\centering
%\begin{tabular}{l|l}
%  & Idee in Matrixnotation \\
%\hline
%$\Delta F = -4 \pi \rho$  & $Au = f$ \\
%$\underbrace{\frac{1}{4 \pi} \Delta F}_{invertiert} = \rho$ & $A^{-1}Au = A^{-1}f$  \\
% & $u = A^{-1}f$  \\
% \hline
%\end{tabular}
%\caption{Analogie zur Matrixinversion}
%\label{tab:helmholtz_matrix_analogie}
%\end{table}
Es gilt also
\[
F(\boldsymbol{r})
=
\int_V
F(\boldsymbol{r}')
\delta(\boldsymbol{r}-\boldsymbol{r}')
\,dV'
=
-\frac{1}{4\pi}
\int_V
\nabla^2 \frac{1}{|\boldsymbol{r}-\boldsymbol{r}'|}
F(\boldsymbol{r}')
\,dV'
\]
für jede Funktion $F$.
Setzt man anstelle von $F$ ein Vektorfeld $\boldsymbol{F}$ ein, gilt
sie auch für die Komponenten des Vektorfeldes.

%Grundlegend für diesen Ansatz ist die Beziehung
%\begin{equation}
%\delta (\boldsymbol{x} - \boldsymbol{x}')
%=
%\frac{1}{4 \pi} \nabla^2 \frac{1}{|\boldsymbol{x} - \boldsymbol{x}'|}.
%\label{helmholtz:dirac}
%\end{equation}
%Diese Beziehung lässt sich analog zur Multiplikation mit der
%Einheitsmatrix verstehen, was es ermöglicht, das Vektorfeld
%$\boldsymbol{F}$ durch das Volumenintegral
%\begin{equation}
%\boldsymbol{F}(\boldsymbol{r})
%=
%-\frac{1}{4\pi} \int_V \boldsymbol{F}(\boldsymbol{r}') \nabla^2
%\left( \frac{1}{|\boldsymbol{r} - \boldsymbol{r}'|} \right)
%\,dV'
%\end{equation}
%auszudrücken.

Unter Verwendung der Vektoridentität
\begin{equation}
\nabla^2 \boldsymbol{A}
=
\underbrace{
\nabla ( \nabla \cdot \boldsymbol{A} )
}_{
\textstyle\text{Gradient\strut}
}
-
\underbrace{
\nabla \times (\nabla \times \boldsymbol{A} )
}_{
\textstyle\text{Rotation\strut}
},
\end{equation}
kann das Vektorfeld $\boldsymbol{F}$ als
\begin{equation}
\boldsymbol{F}(\boldsymbol{r})
=
- \frac{1}{4 \pi} \nabla \biggl(
\underbrace{
\nabla \cdot \int_V
\frac{\boldsymbol{F}(\boldsymbol{r}')
}{
|\boldsymbol{r}
-
\boldsymbol{r}'|} dV'
}_{\displaystyle = \Phi(\boldsymbol{r})}
\biggr)
+
\frac{1}{4 \pi} \nabla \times \biggl(
\underbrace{
\nabla \times
\int_V
\frac{\boldsymbol{F}(\boldsymbol{r}')
}{
|\boldsymbol{r} - \boldsymbol{r}'|
}
\,dV'
}_{\displaystyle = \boldsymbol{\Psi}(\boldsymbol{r})}
\biggr)
\label{helmholtz:eqn:potentiale}
\end{equation}
geschrieben werden.
Diese Gleichung repräsentiert die Helmholtz-Zerlegung des Vektorfeldes
$\boldsymbol{F}$.
Die Klammerausdrücke von \eqref{helmholtz:eqn:potentiale}
sind die gesuchten Poteniale.

In der Form \eqref{helmholtz:eqn:potentiale} ist die Helmholtz-Zerlegung
noch nicht wirklich nützlich.
Dazu muss es in ein Integral über eine geeignete Ableitung des
Vektorfeldes $\boldsymbol{F}(\boldsymbol{r})$ umgewandelt werden.
Dies ist mithilfe geeigneter Randbedingungen und partieller Integration
möglich.

\subsubsection{Berechnung der Potentiale in unbeschränkten Gebieten}

Im Falle eines unbeschränkten Gebietes gibt es keine Randterme und
die Potentiale $\Phi$ und $\boldsymbol{\Psi}$ lassen sich mit Hilfe
der greenschen Funktion wie folgt berechnen:
\begin{itemize}
\item Skalares Potential:
\begin{equation}
\Phi(\boldsymbol{r})
=
\frac{1}{4 \pi}
\int_{V}
\frac{\nabla' \cdot \boldsymbol{F}(\boldsymbol{r}')}{|\boldsymbol{r} - \boldsymbol{r}'|}
\,dV'.
\end{equation}

\item Vektorpotential:
\begin{equation}
\boldsymbol{\Psi}(\boldsymbol{r})
=
\frac{1}{4 \pi}
\int_{V}
\frac{\nabla' \times \boldsymbol{F}(\boldsymbol{r}')}{|\boldsymbol{r} - \boldsymbol{r}'|}
\,dV'.
\end{equation}
\end{itemize}
Hier bezeichnet $\nabla'$ den Nabla-Operator bezüglich der
Integrationsvariablen $\boldsymbol{r}'$.

\subsubsection{Berechnung der Potentiale in beschränkten Gebieten}
Für endliche Volumina $V$ mit Oberfläche $S$ bleiben bei der partiellen
Integraltion zusätzliche Oberflächenintegrale stehen:
\begin{itemize}
\item Skalares Potential:
\begin{equation}
\Phi (\boldsymbol{r})
=
\frac{1}{4\pi}
\int_V
\frac{\nabla' \cdot \boldsymbol{F}(\boldsymbol{r}')}{|\boldsymbol{r} - \boldsymbol{r}'|}
\,dV'
+
\frac{1}{4\pi}
\oint_S \frac{\boldsymbol{F}(\boldsymbol{r}') \cdot \boldsymbol{n}}{|\boldsymbol{r} - \boldsymbol{r}'|}
\,dS'
\end{equation}

\item Vektorpotential:
\begin{equation}
\boldsymbol{\Psi}(\boldsymbol{r})
=
\frac{1}{4\pi}
\int_V
\frac{\nabla' \times \boldsymbol{F}(\boldsymbol{r}')}{|\boldsymbol{r} - \boldsymbol{r}'|}
\,dV'
+
\frac{1}{4\pi}
\oint_S
\frac{\boldsymbol{n} \times \boldsymbol{F}(\boldsymbol{r}')}{|\boldsymbol{r} - \boldsymbol{r}'|}
\,dS'
\end{equation}
\end{itemize}
Hier bezeichnet $\boldsymbol{n}$ den nach aussen gerichteten
Normalenvektor auf der Oberfläche $S$.

\subsection{Bedingungen und Eindeutigkeit
\label{helmholtz:subsection:Bedingungen_Eindeutigkeit}}

\subsubsection{Voraussetzungen für die Anwendbarkeit}
Damit die Zerlegung korrekt durchgeführt werden kann, muss das
Vektorfeld $\boldsymbol{F}$ folgende Bedingungen erfüllen:

\begin{itemize}
\item Glattheit: $\boldsymbol{F}$ muss in dem betrachteten Gebiet
stetig differenzierbar sein.
\item Abfallverhalten: $\boldsymbol{F}$ muss im Unendlichen schneller
als $\frac{1}{r}$ abfallen, d.h.
\begin{equation}
\lim_{r \to \infty} r|\boldsymbol{F}(\boldsymbol{r})| = 0,
\end{equation}
wobei $r = |\boldsymbol{r}|$ der Abstand vom Ursprung ist.
\end{itemize}
Diese Bedingungen stellen sicher, dass die Integrale in der greenschen
Formulierung konvergieren und die Zerlegung eindeutig ist.

\subsubsection{Eindeutigkeit der Zerlegung
\label{helmholtz:subsection:Bedingungen_onlyEindeutigkeit}}

Die Eindeutigkeit der Helmholtz-Zerlegung hängt von den Randbedingungen ab:
\index{Randbedingung}%

\begin{itemize}
\item Für unbeschränkte Gebiete: Die Zerlegung ist eindeutig, wenn
das Vektorfeld im Unendlichen gegen null geht.
Diese Bedingung stellt sicher, dass bestimmte Oberflächenintegrale
bei der Herleitung der Zerlegung verschwinden \cite{wiki:helmholtz}.

\item
Für beschränkte Gebiete: Die Eindeutigkeit erfordert zusätzliche
Randbedingungen.
Obwohl das Vektorfeld auch als Summe eines irrotationalen und eines
solenoidalen Anteils geschrieben werden kann, ist für eine eindeutige
Lösung die Festlegung von Randbedingungen notwendig \cite{wiki:helmholtz}.
Typischerweise sind dies:
  \begin{itemize}
    \item Festlegung der Normalkomponente des irrotationalen Teils
    auf dem Rand.
    \item Festlegung der Tangentialkomponente des solenoidalen Teils
    auf dem Rand.
  \end{itemize}
\end{itemize}

\subsubsection{Orthogonalität der Komponenten}

Eine wichtige Eigenschaft der Helmholtz-Zerlegung ist die Orthogonalität
der beiden Komponenten.
\index{Orthogonalitat@Orthogonalität}%
Für ein Vektorfeld
$\boldsymbol{F} = \boldsymbol{F}_{\text{irr}} + \boldsymbol{F}_{\text{sol}}$
gilt unter geeigneten Randbedingungen:
\begin{equation}
\int_V \boldsymbol{F}_{\text{irr}} \cdot \boldsymbol{F}_{\text{sol}} \, dV = 0.
\end{equation}
%
% XXX Inkorrekter Satz
%
Das Integral ist null, weil es mithilfe des gaussschen Integralsatzes
in ein Oberflächenintegral umgewandelt werden kann, das unter den
für die Helmholtz-Zerlegung Randbedingungen, wie oben
%\ref{helmholtz:subsection:Bedingungen_onlyEindeutigkeit}
beschrieben, verschwindet.
 
In physikalischen Anwendungen, wie beispielsweise in der Akustik,
bedeutet diese mathematische Orthogonalität, dass die beiden
Feldkomponenten unterschiedliche physikalische Phänomene beschreiben,
die unabhängig voneinander analysiert werden können.
Mehr dazu wird im Abschnitt~\ref{helmholtz:Energie_Interpretation}
erläutert.



%
% teil3.tex -- Beispiel-File für Teil 3
%
% (c) 2020 Prof Dr Andreas Müller, Hochschule Rapperswil
%
% !TEX root = ../../buch.tex
% !TEX encoding = UTF-8
%
\section{Akustische Grundlagen und Feldtheorie
\label{helmholtz:section:akustische_Grundlagen}}
\kopfrechts{Akustische Grundlagen und Feldtheorie}

Um die Anwendung der Helmholtz-Zerlegung in der Akustik zu verstehen, bedarf es grundlegender Kenntnisse der physikalischen Akustik. Hier werden die zentralen physikalischen Grössen Schalldruck, Schallschnelle und deren Zusammenhänge erläutert. In vielen akustischen Aufgaben wird mit der komplexen Schreibweise gearbeitet, weshalb die Formeln vorzugsweise im Frequenzbereich dargestellt werden.

\subsection{Grundbegriffe der physiklalischen Akustik
\label{helmholtz:subsection:Grundbegriffe_Akustik}}

\subsubsection{Schalldruck $p$}
 
Der Schalldruck ist eine der fundamentalen Größen in der Akustik und beschreibt lokale Druckschwankungen im Medium:
 
\begin{itemize}
\item $P \: (\boldsymbol{r})$ bezeichnet die Amplitude des Schalldrucks am Ort $\boldsymbol{r}$ und ist eine reelle Grösse.
\item $\omega$ ist die Kreisfrequenz.
\item $\phi \: (\boldsymbol{r})$ beschreibt die Phase am Ort $\boldsymbol{r}$ und ist ebenfalls eine reelle Grösse.
\end{itemize}
 
Die zeitabhängige Darstellung des Schalldrucks lautet:
\begin{equation}
p(r,t) = P(\boldsymbol{r}) \: e^{j( \omega t - \phi(\boldsymbol{r}))} (\si{\pascal}).
\end{equation}
 
Im Frequenzbereich, auch Phasor-Darstellung genannt, bei einer festen Frequenz $\omega$ vereinfacht sich dies zu:
\begin{equation}
p(r) = P(r) \: e^{j \phi (r)}.
\label{helmholtz:PhasorSchalldruck}
\end{equation}
 
\subsubsection{Schallschnelle $\boldsymbol{u}$}
 
Die Schallschnelle $\boldsymbol{u}$ auch Teilchengeschwindigkeit genannt, wird aus dem Druckgradienten $\nabla p$ abgeleitet. Für die linearisierte Form im Frequenzbereich $\boldsymbol{u} = \frac{j}{\rho_0 \omega} \nabla p$ bei harmonischen Feldern $\boldsymbol{u}(r,t)$ ergibt sich:
\begin{equation}
\boldsymbol{u}(r,t) = \frac{1}{\omega \rho} \nabla \: p(r,t) = \frac{1}{\omega \rho} [ P(r) \nabla \Phi(r) + j\nabla P(r) ] \: e^{j[\omega t -\Phi(r)]} (\si{\metre / \second}).
\end{equation}
Dabei sind:
\begin{itemize}
\item $\nabla \phi \: (r)$ der Gradient der Phase zeigt in Richtung der steilsten Phasenänderung, senkrecht zur Wellenfront.
\item $\nabla P \:(r)$ der Gradient der Druckamplitude zeigt in Richtung der schnellsten Amplitudenänderung.
\item $\rho$ die Dichte des Mediums.
\end{itemize}
Als Phasor bei einer fixen Frequenz $\omega$ ergibt sich die Schallschnelle zu:
\begin{equation}
\boldsymbol{u}(r) = \frac{1}{\omega \rho} \: [ P(r) \: \nabla \phi(r) + j - \nabla P(r) ] \: e^{j\phi (r)}.
\label{helmholtz:PhasorSchallschnelle}
\end{equation}
 
%\subsubsection{Wellengleichung und Helmholtz-Gleichung}
 
%In der linearen Akustik sind Schalldruck und Schallschnelle durch folgende Grundgleichungen verbunden:
 
%\begin{itemize}
%\item Die Euler-Gleichung %(Impulserhaltung):
%\begin{equation}
%\rho_0 \frac{\partial \boldsymbol{u}}{\partial t} = -\nabla p
%\end{equation}
 
%\item Die Kontinuitätsgleichung %(Massenerhaltung):
%\begin{equation}
%\frac{\partial \rho}{\partial t} + \rho_0 \nabla \cdot \boldsymbol{u} = 0
%\end{equation}
%\end{itemize}
 
%\noindent Durch Kombination dieser Gleichungen erhält man die Wellengleichung für den Schalldruck:
%\begin{equation}
%\nabla^2 p - \frac{1}{c^2}\frac{\partial^2 p}{\partial t^2} = 0
%\end{equation}
 
%\noindent Für zeitharmonische Anregung mit $e^{j\omega t}$ führt dies zur Helmholtz-Gleichung:
%\begin{equation}
%\nabla^2 p + k^2 p = 0
%\end{equation}
 
%\noindent wobei $k = \omega/c$ die Wellenzahl ist und $c$ die Schallgeschwindigkeit im Medium.

\subsection{Energiebetrachtungen in akustischen Feldern
\label{helmholtz:subsection:Energiebetrachtung}}

Die akustische Energie in einem Schallfeld setzt sich aus kinetischer und potentieller Energie zusammen. Die zeitlich gemittelte Energiedichte kann durch folgende Gleichung ausgedrückt werden

\begin{equation}
\langle w \rangle = \frac{1}{4}\left(\frac{|p|^2}{\rho_0 c^2} + \rho_0 |\boldsymbol{u}|^2 \right),
\end{equation}
wobei der erste Term die potentielle Energiedichte (durch die Kompression des Mediums) und der zweite Term die kinetische Energiedichte (durch die Bewegung der Teilchen) darstellt.

Der Transport dieser Energie wird durch die Schallintensität beschrieben. Für harmonische Schallfelder betrachten wir die zeitlich gemittelte aktive Intensität:

\begin{equation}
\boldsymbol{I} = \frac{1}{T}\int_0^T \boldsymbol{I}_i(\boldsymbol{r},t)\,\mathrm{d}t = \frac{1}{2}\Re\{p(\boldsymbol{r})~\boldsymbol{u}^*(\boldsymbol{r})\},
\end{equation}
wobei der Asterisk die komplexe Konjugation bezeichnet. Diese aktive Intensität beschreibt den Netto-Energiefluss und ist für die Schallausbreitung von zentraler Bedeutung.

\subsubsection{Zeitabhängige und zeitgemittelte Größen}

In der Akustik unterscheiden wir zwischen momentanen, zeitabhängigen Größen und zeitgemittelten Größen:

\begin{itemize}
\item \textbf{Momentane Größen} wie die instantane Intensität $\boldsymbol{I}(r,t)$ beschreiben den augenblicklichen Energietransport an einem bestimmten Ort und zu einem bestimmten Zeitpunkt.

\item \textbf{Zeitgemittelte Größen} wie die aktive Intensität $\boldsymbol{I}(r)$ beschreiben den durchschnittlichen Energietransport über eine Periode der harmonischen Schwingung und sind besonders wichtig für die Analyse der effektiven Energieübertragung in akustischen Feldern.
\end{itemize}

Diese Unterscheidung ist von großer Bedeutung, da in vielen praktischen Anwendungen nicht die instantanen, sondern die zeitlich gemittelten Größen von Interesse sind.





%
% teil3.tex -- Beispiel-File für Teil 3
%
% (c) 2020 Prof Dr Andreas Müller, Hochschule Rapperswil
%
% !TEX root = ../../buch.tex
% !TEX encoding = UTF-8
%
\section{Helmholtz-Zerlegung in der Akustik
\label{helmholtz:section:Helmholtz_Zerlegung_Akustik}}
\kopfrechts{Helmholtz-Zerlegung in der Akustik}


\subsection{Zerlegung des Schallschnellefeldes
\label{helmholtz:subsection:Zerlegung_Schallschnelle}}
Angewendet auf das Schallschnellefeld $\boldsymbol{u}$ können wir dieses wie folgt zerlegen:
 
\begin{equation}
\underbrace{\boldsymbol{u}}_{\text{Schallschnellefeld}} =  \underbrace{-\nabla \Phi}_{\text{irrotationaler~Anteil}} + \underbrace{\nabla \times \boldsymbol{\Psi}}_{\text{solenoidaler~Anteil}}.
\end{equation}
Die physikalische Interpretation der Komponenten des Schallschnellefeldes lässt sich wie folgt beschreiben:
 
\begin{itemize}
\item Der irrotationale Anteil $-\nabla \Phi$ ist wirbelfrei, da die Rotation verschwindet.
\begin{itemize}
\item Dieser Anteil hat keine Rotation: $\nabla \times (-\nabla \Phi) = 0$.
\item Die Feldlinien verlaufen strahlenförmig von Quellen zu Senken.
\end{itemize}
 
\item Der solenoidale Anteil $\nabla \times \boldsymbol{\Psi}$ hat gemäss der Definition immer Divergenz null und wird auch quellenfrei genannt.
\begin{itemize}
\item Dieser Anteil hat keine Divergenz: $\nabla \cdot (\nabla \times \boldsymbol{\Psi}) = 0$.
\item Die Feldlinien bilden geschlossene Schleifen ähnlich einem Magnetfeld.
\end{itemize}
\end{itemize}

\subsection{Direkte Verbindung zur komplexen Schallintensität
\label{helmholtz:subsection:Zerlegung_Schallschnelle}}

Wie bereits beschrieben, stellt die komplexe Schallintensität $\boldsymbol{I}_c$ den Energietransport in einem Schallfeld dar und lässt sich in zwei fundamentale Komponenten zerlegen:
 
\begin{itemize}
\item Aktive Intensität $\boldsymbol{I}$
\item Reaktive Intensität $\boldsymbol{Q}$
\end{itemize}
Die komplexe Schallintensität ist definiert als:
\begin{equation}
\boldsymbol{I}_c (\boldsymbol{r})
=
\frac{1}{2} p(\boldsymbol{r})  \boldsymbol{u}^{*}(\boldsymbol{r}),
\end{equation}
wobei $p(\boldsymbol{r})$ den komplexen Schalldruck und $\boldsymbol{u}^{*}(\boldsymbol{r})$ die komplexe Konjugation der Schallschnelle $\boldsymbol{u}(\boldsymbol{r})$ beschreibt. Diese Formulierung berücksichtigt die Phasenverschiebung zwischen Schalldruck und Schallgeschwindigkeit und ermöglicht die Zerlegung des Energieflusses in quellenfreie und wirbelfreie Anteile gemäss der Helmholtz-Zerlegung.
 
Bei näherer Betrachtung ist zu erkennen, dass die komplexe Schallintensität eine ähnliche Struktur wie die Helmholtz-Zerlegung aufweist und sich wie folgt darstellen lässt:
\begin{equation}
\boldsymbol{I}_c (\boldsymbol{r})
=
\underbrace{\boldsymbol{I}(\boldsymbol{r})}_{\displaystyle\frac{1}{2}
\operatorname{Re} ( p(\boldsymbol{r}) \boldsymbol{u}^*(\boldsymbol{r}) )
}
+
\underbrace{j\boldsymbol{Q}(\boldsymbol{r})}_{\displaystyle\frac{1}{2}
\operatorname{Im} ( p(\boldsymbol{r}) \boldsymbol{u}^*(\boldsymbol{r}) )}.
\end{equation}
Es zeigt sich eine direkte Korrespondenz zwischen den Komponenten der Helmholtz-Zerlegung und der komplexen Schallintensität:
\begin{equation}
\boldsymbol{I}_c (\boldsymbol{r})
=
\underbrace{\boldsymbol{I}(\boldsymbol{r})}_{\displaystyle\nabla \cdot \boldsymbol{I}
=
0 \text{ (quellenfrei)}}
+
\underbrace{j\boldsymbol{Q}(\boldsymbol{r})}_{\displaystyle\nabla \times \boldsymbol{Q}
=
0\qquad \text{ (wirbelfrei)}}.
\end{equation}
Durch diese Äquivalenz können wir folgende Korrespondenz feststellen:
 
\begin{itemize}
\item Der irrotationale Anteil der Schallschnelle korrespondiert mit der aktiven Intensität.
\item Der solenoidale Anteil der Schallschnelle korrespondiert mit der reaktiven Intensität.
\end{itemize}

\subsection{Energie-Interpretation der Zerlegung
\label{helmholtz:Energie_Interpretation}}
 
Die Helmholtz-Zerlegung ermöglicht eine tiefe physikalische Interpretation der Energieverteilung und -übertragung in akustischen Feldern:
 
\begin{itemize}
\item Die aktive Intensität $\boldsymbol{I}(r)$ beschreibt den zeitgemittelten Netto-Energiefluss pro Fläche an dem Ort $r$ und lässt sich ausdrücken als:
\begin{equation}
\boldsymbol{I}(\boldsymbol{r})
=
\frac{1}{T}\int_0^T \boldsymbol{I}_i(\boldsymbol{r},t)\,dt
=
\frac{1}{2}\Re\left( p(\boldsymbol{r})~\boldsymbol{u}^*(\boldsymbol{r})\right).
\end{equation}
 
In quellenfreien, stationären Feldern ohne Energieabsorption gilt:
\begin{equation}
\nabla \cdot \boldsymbol{I}
=
0.
\end{equation}
 
Diese Komponente ist mit dem irrotationalen Anteil des Schallschnellefeldes verknüpft und repräsentiert den tatsächlichen Energiefluss durch das Medium.
 
\item Die reaktive Intensität $\boldsymbol{Q}(r)$ beschreibt die zeitlich gemittelte Dichte der nicht-propagierenden, oszillierenden Energie:
\begin{equation}
\boldsymbol{Q}(\boldsymbol{r})
=
\frac{1}{2}\Im\left(p(\boldsymbol{r})~\boldsymbol{u}^*(\boldsymbol{r})\right).
\end{equation}
 
Sie ist wirbelfrei:
\begin{equation}
\nabla \times \boldsymbol{Q}
=
0.
\end{equation}
 
Und steht in direkter Beziehung zur Differenz zwischen kinetischer Energie T und potentieller Energie V:
\begin{equation}
\nabla \cdot \boldsymbol{Q}
=
-2 \omega (T-V).
\end{equation}
 
Diese Komponente ist mit dem solenoidalen Anteil des Schallschnellefeldes verknüpft und stellt die oszillierende, lokal gespeicherte Energie dar, die nicht zum Netto-Energietransport beiträgt.
\end{itemize}
Anhand von zwei Extremfällen lässt sich diese Interpretation besonders gut veranschaulichen:

\begin{itemize}


 
\item Ebene Welle ($\boldsymbol{Q} = 0$):
Bei einer ebene Welle ist die Amplitude konstant $P(\boldsymbol{r})
= A = \text{konst.}$ bzw. $\nabla P = 0$, was dazu führt, dass die
reaktive Intensität verschwindet. Schalldruck und Schallschnelle
sind in Phase $\phi = 0^{\circ}$. Die komplexe Intensität reduziert
sich zu:
 
\begin{equation}
\boldsymbol{I}_c (\boldsymbol{r})
=
\boldsymbol{I}(\boldsymbol{r}) + \cancel{j\boldsymbol{Q}(\boldsymbol{r})}
\end{equation}
 
Die aktive Intensität ergibt sich zu: $\boldsymbol{I} = \frac{|\boldsymbol{A}|^2}{2 \rho_0 c_0}$ und zeigt in Richtung der Wellenausbreitung.
 
\item Stehende Welle ($\boldsymbol{I} = 0, \boldsymbol{Q} \neq 0$): 
Bei einer stehenden Welle überlagern sich zwei gleiche ebene Wellen mit gleicher Amplitude in entgegengesetzter Richtung. An jedem Punkt ist entweder $P = maximal$ und $\boldsymbol{u} = 0$. Oder $P = 0$ und $\boldsymbol{u} = maximal$. Schalldruck und Schallschnelle sind um 90° phasenverschoben. Die komplexe Intensität reduziert sich zu:
 
\begin{equation}
\boldsymbol{I}_c (\boldsymbol{r})
=
\cancel{\boldsymbol{I}(\boldsymbol{r})} + j\boldsymbol{Q}(\boldsymbol{r})
\end{equation}

\end{itemize}
In diesem Fall gibt es keinen Netto-Energietransport, sondern nur lokale Energieoszillation.


\subsection{Visualisierung und Feldmuster
\label{helmholtz:subsection:Visualisierung}}
Die Helmholtz-Zerlegung führt zu charakteristischen Feldmustern, die sich wie folgt grafisch darstellen lassen.
 
\begin{itemize}
\item Der irrotationale Anteil bildet Quellenfelder mit radialen Feldlinien, die von Quellen ausgehen oder in Senken münden:
 
\begin{figure}
\centering
\includegraphics[width=0.8\textwidth]{papers/helmholtz/images/Quelle.png}
\caption{Quellenmuster im irrotationalen Feldanteil}
\label{fig:quelle}
\end{figure}
 
\begin{figure}
\centering
\includegraphics[width=0.8\textwidth]{papers/helmholtz/images/Senke.png}
\caption{Senkenmuster im irrotationalen Feldanteil}
\label{fig:senke}
\end{figure}
 
\item Der solenoidale Anteil bildet Wirbelfelder mit geschlossenen Feldlinien, ähnlich einem Magnetfeld.
\end{itemize}
  
\begin{figure}
\centering
\includegraphics[scale=0.6]{papers/helmholtz/images/aktiveSchallintensitaet.png}
\caption{Visualisierung des aktiven Schallintensitätsfeldes für eine Anordnung von drei Punktquellen. Die Pfeile stellen die Vektoren der aktiven Intensität dar und zeigen die Richtung des zeitgemittelten Netto-Energieflusses. In der Nähe der Quellen ist die radial wegführende Energie zu erkennen, während in den Interferenzzonen komplexe Muster, einschliesslich der Bildung von Schallintensitätswirbeln, sichtbar werden (Abbildung aus \cite{helmholtz:paper}).}
\label{fig:aktive_intensitaet_3quellen}
\end{figure}
 
Die Visualisierung der Feldmuster erlaubt eine intuitive Interpretation von Schallfeldern:
 
\begin{itemize}
\item in Bereiche mit starker aktiver Intensität zeigen Energieausbreitung und -übertragung.
\item in Bereiche mit starker reaktiver Intensität zeigen Energieoszillation ohne Nettotransport, wie sie typischerweise im Nahfeld von Schallquellen oder bei stehenden Wellen auftritt.
\item Die räumliche Verteilung von Quellen, Senken und Wirbeln erlaubt Rückschlüsse auf die zugrundeliegenden Schallquellen und Reflexionsmuster. 

Abbildung \ref{fig:aktive_intensitaet_3quellen} verdeutlicht eindrücklich, wie die Interferenz von nur drei Quellen ausreicht, um solche komplexen Wirbelmuster zu erzeugen.
\end{itemize}
Diese Feldmuster sind besonders wichtig für die akustische Messtechnik, da sie eine visuelle Methode zur Identifikation von Schallquellen und zur Analyse von Energieflüssen in komplexen akustischen Feldern bieten.






\printbibliography[heading=subbibliography]
\end{refsection}
