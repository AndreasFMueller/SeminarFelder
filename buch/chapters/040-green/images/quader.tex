%
% quader.tex -- Approximation eines Gebietes durch Quader
%
% (c) 2021 Prof Dr Andreas Müller, OST Ostschweizer Fachhochschule
%
\documentclass[tikz]{standalone}
\usepackage{amsmath}
\usepackage{times}
\usepackage{txfonts}
\usepackage{pgfplots}
\usepackage{csvsimple}
\definecolor{darkgreen}{rgb}{0,0.6,0}
\definecolor{darkred}{rgb}{0.8,0,0}
\usetikzlibrary{arrows,intersections,math,calc}
\begin{document}
\def\skala{1}
\def\quader#1#2#3{
	\fill[color=blue!20,opacity=0.5] #1 rectangle #2;
	\draw[color=blue] #1 rectangle #2;
	\node[color=blue] at ($0.5*#1+0.5*#2$) {$#3$};
}
\begin{tikzpicture}[>=latex,thick,scale=\skala]

\clip (-1.5,-1.2) rectangle (5.2,4.60);

\coordinate (B1) at (0.2,0.2);
\coordinate (B2) at (1.8,-0.4);
\coordinate (B3) at (4.3,2.8);
\coordinate (B4) at (0.7,4.1);
\coordinate (B5) at (-0.3,3.5);

\coordinate (A1) at (-0.2,-0.2);
\coordinate (A2) at (2,-1);
\coordinate (A3) at (5,3);
\coordinate (A4) at (1,4.5);
\coordinate (A5) at (-0.5,4);

\fill[color=gray!20] (A1)
	to[out=10,in=-140] (A2)
	to[out=40,in=-70] (A3)
	to[out=110,in=-20] (A4)
	to[out=160,in=40] (A5)
	to[out=-140,in=-170] (A1);

%\fill (A1) circle[radius=0.08];
%\fill (A2) circle[radius=0.08];
%\fill (A3) circle[radius=0.08];
%\fill (A4) circle[radius=0.08];
%\fill (A5) circle[radius=0.08];

\fill[color=darkred!20] (B1)
	to[out=10,in=-140] (B2)
	to[out=40,in=-70] (B3)
	to[out=110,in=-20] (B4)
	to[out=160,in=40] (B5)
	to[out=-140,in=-170] (B1);

\quader{(-0.4,0)}{(3,4)}{Q_1}
\quader{(1,-0.7)}{(2.3,0)}{Q_2}
\quader{(3,0.7)}{(4,3.9)}{Q_3}
\quader{(3,0.3)}{(3.5,0.7)}{}
\quader{(4,1.3)}{(4.7,3.4)}{Q_4}
\quader{(0,4)}{(1.5,4.3)}{}
\quader{(-1,0.2)}{(-0.4,3.2)}{Q_5}
\quader{(-0.8,3.2)}{(-0.4,3.6)}{}
\quader{(0.5,-0.3)}{(1,0)}{}

%\foreach \x in {-2,...,5}{
%	\draw[color=gray,line width=0.1pt] (\x,-2) -- (\x,5);
%}
%\foreach \y in {-2,...,5}{
%	\draw[color=gray,line width=0.1pt] (-2,\y) -- (5,\y);
%}

\coordinate (C1) at (3.0,-0.7);
\node at (C1) [below right] {$U$};
\draw[line width=0.2pt] ($(C1)+(0.1,-0.1)$) -- ++(-0.5,0.5);

\coordinate (C2) at (0,-0.5);
\node[color=darkred] at (C2) [below left] {$F$};
\draw[color=darkred,line width=0.2pt] ($(C2)+(-0.1,-0.1)$) -- ++(0.9,1.2);

\end{tikzpicture}
\end{document}

