%
% teil4.tex -- Beispiel-File für Teil 4
%
% (c) 2020 Prof Dr Andreas Müller, Hochschule Rapperswil
%
% !TEX root = ../../buch.tex
% !TEX encoding = UTF-8
%
%\usepackage{siunitx}
%\usepackage{amsmath}
%\usepackage[ngerman]{babel}
\section{Beispiele aus der Bauingenieurspraxis}
\label{elastomechanik:section:teil4}
\kopfrechts{Beispiele aus der Bauingenieurpraxis}
In den vorangegangenen Abschnitten
(\ref{elastomechanik:section:Einleitung}--\ref{elastomechanik:section:herleitung2})
haben wir die allgemeinen Grundlagen der Elastizitätstheorie entwickelt: 
von der Definition der Verzerrungen über das hookesche Gesetz bis
hin zu den Gleichgewichtsbedingungen und der Formulierung der
Navier-Cauchy-Gleichungen.
Diese theoretischen Resultate bilden die Grundlage, um reale
Ingenieurprobleme zu beschreiben. 
Im Folgenden wollen wir an zwei typischen Beispielen, einem Balken
unter Mittellast und einem Oedometerversuch, aufzeigen, wie sich
\index{Oedometerversuch}%
aus den allgemeinen Gleichungen praktische Berechnungen ableiten
lassen.
Zur Erinnerung fassen wir die wesentlichen Zusammenhänge zusammen,
die wir in Abschnitt
%15.1–15.4
\ref{elastomechanik:section:Einleitung}--\ref{elastomechanik:section:herleitung2}
hergeleitet haben:
die Kinematik
\[
  \varepsilon_{ij} = \tfrac12(\partial_i u_j + \partial_j u_i),
\]
das konstitutive Gesetz
\[
  \sigma_{ij} = \lambda\, \delta_{ij}\,\varepsilon_{kk} + 2\mu\, \varepsilon_{ij},
\]
und das Gleichgewicht
\[
  \partial_j \sigma_{ij} + f_i = 0,
\]
die zusammen zu den Navier-Cauchy-Gleichungen führen.
Für schlanke Balken werden zusätzlich die Euler-Bernoulli-Annahmen getroffen. 
\index{Euler-Bernoulli-Annahmen}%
Diese lauten:
\begin{itemize}
  \item Querschnitte bleiben bei Biegung eben und unverzerrt,
  \item Querschnitte bleiben senkrecht zur Balkenachse,
  \item die Einflüsse der Querkraftverzerrung werden vernachlässigt.
\end{itemize}
Unter diesen Voraussetzungen ergibt sich die Balkendifferentialgleichung. 
Aus den allgemeinen Gleichgewichtsbedingungen der Elastizitätstheorie 
(vgl.~Abschnitt~\ref{elastomechanik:math:subsection:gleichgewicht})
folgt zunächst, dass die Querkraft \(Q(x)\) mit der 
Streckenlast \(q(x)\) durch
\[
  Q'(x) + q(x) = 0
\]
verknüpft ist.
Das Moment \(M(x)\) hängt wiederum mit der Querkraft nach
\[
  M'(x) + Q(x) = 0
\]
zusammen.
Kombiniert man diese Beziehungen, erhält man
\begin{equation*}
  M''(x) = q(x).
\label{elastomechanik:eqn:Mpp}
\end{equation*}

Die Euler-Bernoulli-Hypothese verknüpft das Moment mit der 
Krümmung des Balkens:
\begin{equation*}
  M(x) = EIw''(x),
\label{elastomechanik:eqn:kruemmung}
\end{equation*}
wobei \(E\) der Elastizitätsmodul und \(I\) das Flächenträgheitsmoment ist. 
Einsetzen
von
\eqref{elastomechanik:eqn:kruemmung}
in
\eqref{elastomechanik:eqn:Mpp}
liefert die Balkendifferentialgleichung
\begin{equation*}
  EIw^{(4)}(x) = q(x).
  \label{eq:beam_dgl}
\end{equation*}
Die fundamentale Gleichung \eqref{eq:beam_dgl}
bildet die Grundlage für die folgenden 
Berechnungen von Momenten, Spannungen und Durchbiegungen.

%Für den Oedometerversuch gilt ein eindimensionaler Verzerrungszustand
%\(\varepsilon_x=\varepsilon_y=0\); daraus folgt
%\[
%  \sigma_z = M \,\varepsilon_z, \qquad
%  M=\frac{E(1-\nu)}{(1+\nu)(1-2\nu)},
%\]
%wobei \(M\) als eingeschränkter (oedometrischer) Modul bezeichnet wird.

\subsection{Balken unter Mittellast}
\index{Balken}%
\index{Mittellast}%
Wir betrachten einen Einfeldträger der Länge \(l = 5\,\mathrm{m}\),
der in der Mitte mit einer Einzellast \(F = 100\,\mathrm{kN}\)
belastet wird.
Das System ist statisch bestimmt, sodass sich die Auflagerreaktionen
unmittelbar aus den Gleichgewichtsbedingungen ergeben:
\[
  R_A = R_B = \frac{F}{2}.
\]
Das Moment an einer Stelle \(x\) zwischen dem linken Auflager und
der Last ergibt sich zu
\[
  M(x) = R_A   x = \frac{F}{2} x.
\]
Für \(x=l/2\), also an der Stelle der Belastung, nimmt das Moment seinen 
maximalen Wert
\begin{equation*}
  M_{\max} = \frac{F l}{4} = 125\,\mathrm{kN\,m}
  \label{eq:Mmax}
\end{equation*}
an. Für den Rechteckquerschnitt mit Breite \(b=200\,\mathrm{mm}\) 
und Höhe \(h=300\,\mathrm{mm}\) ist das Flächenträgheitsmoment
\[
  I = \frac{b h^3}{12} = 4.5 \times 10^8\,\mathrm{mm^4}
  \qquad 
\]
Der Abstand \(y\) gibt den maximalen Abstand der äussersten Faser
von der neutralen Achse an.
Für einen rechteckigen Querschnitt liegt die neutrale Achse 
in der Mitte, daher gilt:
\[
  y = \frac{h}{2}= 150\,\mathrm{mm}.
\]
Damit ergibt sich die maximale Biegespannung zu
\begin{equation*}
  \sigma_{\max} = \frac{M_{\max} y}{I} = 41.7\,\mathrm{MPa}.
  \label{eq:sigma_max}
\end{equation*}
Die Durchbiegung in der Feldmitte erhält man schließlich aus der
Balkendifferentialgleichung \(EI\,w''(x)=M(x)\).
Einmaliges Integrieren liefert die Querkraftlinie, zweimaliges
Integrieren die Biegelinie.
Mit den Randbedingungen \(w(0)=w(l)=0\) ergibt sich für die 
maximale Durchbiegung:
\begin{equation*}
  w_{\max} = \frac{F l^3}{48 E I}.
  \label{eq:wmax}
\end{equation*}
Eingesetzt ergeben sich 
\(w_{\max} \approx 48.2\,\mathrm{mm}\) für 
\(\,E_{\text{Holz}}=12\,\mathrm{GPa}\)
und
\(w_{\max} \approx 2.76\,\mathrm{mm}\) für 
\(\,E_{\text{Stahl}}=210\,\mathrm{GPa}\).

\subsection{Oedometerversuch}
Im Oedometerversuch wird eine Bodenprobe so eingespannt, dass
\index{Oedometerversuch}%
seitliche Dehnungen nicht auftreten können.
Belastet man die Probe mit einer vertikalen Spannung \(\sigma_z\),
so kann sie sich ausschliesslich in vertikaler Richtung verkürzen.
Der Zusammenhang zwischen Spannung und Dehnung lässt sich aus dem
isotropen hookeschen Gesetz ableiten
(vgl. Abschnitt \ref{elastomechanik:section:teil1})
Da die seitlichen Dehnungen unter den Oedometerversuchsbedingungen
verschwinden, setzen wir
\[
  \varepsilon_x = \varepsilon_y = 0.
\]
Es gilt immer noch ein linearer Zusammenhang
\[
  \sigma_z = E_o \varepsilon_z,% \qquad
  %E_{o}=\frac{E(1-\nu)}{(1+\nu)(1-2\nu)},
\]
wobei \(E_{o}\) als eingeschränkter (oedometrischer) Modul
bezeichnet wird.
\index{oedometrischer Modul}%
\index{eingeschrankter Modul@eingeschränkter Modul}%
Im Folgenden soll \(E_{o}\) bestimmt werden.

Wenn wir die genannten Bedingung in das hookesche Gesetz einsetzen,
zeigt sich, dass in horizontaler Richtung eine Spannung \(\sigma_h\)
auftritt, die proportional zur vertikalen Spannung \(\sigma_z\)
ist.
Diese Spannung lässt sich schreiben als:
\[
  \sigma_h = \frac{\nu}{1-\nu}\,\sigma_z,
\]
wobei \(\nu\) die Poisson-Zahl ist.  
Sobald die seitliche Spannung berücksichtigt ist, können wir die
vertikale Dehnung \(\varepsilon_z\) bestimmen.
Durch Einsetzen von \(\sigma_h\) in das hookesche Gesetz ergibt sich:
\[
  \varepsilon_z = \frac{\sigma_z}{E}\,\frac{1-\nu-2\nu^2}{1-\nu}.
\]
Aus der Definition des Oedometermoduls als Verhältnis von Spannung
zu vertikaler Dehnung folgt dann direkt:
\[
  E_o = \frac{\sigma_z}{\varepsilon_z} = \frac{E(1-\nu)}{(1+\nu)(1-2\nu)}.
\]

\begin{beispiel}   
Betrachten wir eine Probe mit einer Höhe von \(h_0 = 20\,\mathrm{mm}\),
die unter einer Belastung von \(\sigma_z = 100\,\mathrm{kPa}\) um
\(\Delta h = 0.5\,\mathrm{mm}\) verkürzt wird.
Zuerst berechnen wir die vertikale Dehnung:
\[
  \varepsilon_z
= \frac{\Delta h}{h_0}
= \frac{0.5\,\mathrm{mm}}{20\,\mathrm{mm}} = 0.025.
\]
Anschliessend ergibt sich der Oedometermodul direkt aus der Spannung
geteilt durch die Dehnung:
\[
  E_o = \frac{\sigma_z}{\varepsilon_z}
=
\frac{100\,\mathrm{kPa}}{0.025} = 4000\,\mathrm{kPa}.
\qedhere
\]
\end{beispiel}  
Dieser Versuch verdeutlicht, dass der Oedometermodul \(E_o\) nicht
mit dem dreidimensionalen Elastizitätsmodul \(E\) identisch ist.
Er hängt sowohl von \(E\) als auch von der Poisson-Zahl \(\nu\) ab
und ergibt sich unmittelbar aus dem hookeschen Gesetz in Verbindung
mit den speziellen Deformationsbedingungen des Oedometerversuchs.
