%
% Wärmeleitung und Diffusion
%
\section{Wärmeleitung und Diffusion}
\kopfrechts{Wärmeleitung und Diffunsion}
Wärmeleitung und Diffusion führen auf sehr ähnliche parabolische
partielle Differentialgleichungen.

%
% Wärmeleitung
%
\subsection{Wärmeleitung}
Die Wärmeenergiedichte $\varrho$ eines homogenenen Mediums mit der
\index{Wärmeenergiedichte}%
konstanten Wärmekapazität $c$ ist proportional zur Temperatur,
es gilt daher $\varrho=cT$.
\index{Temperatur}%
Der Wärmeleitungskoeffizient $k$ definiert, wieviel
\index{Wärmeleitungskoeffizient}%
Wärmeenergie pro Zeiteinheit den Temperaturunterschied
$\Delta T$ über eine Distanz $\Delta x$ überwindet.
In differentieller Schreibweise ist der Wärmeenergiefluss
\index{Wärmeenergiefluss}%
\[
\vec{\jmath}
=
-k \operatorname{grad} T.
\]
Da die Wärmeenergie erhalten bleibt, gilt die Kontinuitätsgleichung
\index{Kontinuitätsgleichung}%
für $\varrho$ und $\vec{\jmath}$, sie lautet
\begin{align*}
-\frac{\partial\varrho}{\partial t}
&=
\operatorname{div}\bigl(
-\kappa\operatorname{grad}T
\bigr)
\\
-c\frac{\partial T}{\partial t}
&=
-k\varrho\operatorname{div}\operatorname{grad}T
\intertext{oder nach Division durch $-c$}
\frac{\partial T}{\partial t}
&=
\kappa\,
\Delta T
\end{align*}
mit $\kappa = \frac{c}{k}$.

%
% Diffusion
%
\subsection{Diffusion}
In einem homogenen Medium ist ein Stoff mit der Konzentration 
$c$ gelöst.
\index{Konzentration}%
Konzentrationsunterschiede werden durch Diffusion des gelösten
Stoffs ausgeglichen.
\index{Diffusion}%
Je grösser der Konzentrationsunterschied, desto grösser ist
der Stofffluss, es gilt also
\[
\vec{\jmath}\, = -D\operatorname{grad}c
\]
für den Stofffluss, mit einer Materialkonstanten $D$.
\index{Stofffluss}%
Die Konzentration $c$ ist die Dichte des gelösten Stoffs,
$\vec{\jmath}$\, ist der Stofffluss im Medium.
Sofern der Stoff keine Reaktionen mit dem Medium eingeht,
bleibt die gelöste Stoffmenge erhalten, es muss daher die
Kontinuitätsgleichung
\begin{align*}
-\frac{\partial c}{\partial t}
&=
\operatorname{div}\bigl(
-D\operatorname{grad}c
\bigr)
\\
\frac{\partial c}{\partial t}
&=
D\operatorname{div}\operatorname{grad}c
\intertext{gelten.
Mit dem Laplace-Operator geschrieben ist dies}
\frac{\partial c}{\partial t}
&=
D\,\Delta c,
\end{align*}
\index{Laplace-Operator}%
eine zur Wärmeleitungsgleichung äquivalente partielle
Differentialgleichung.

%
% Fundamentallösung
%
\subsection{Fundamentallösung}
Ersetzt man die Zeitkoordinate $t$ durch $\tau=\kappa t$, wird die
Zeitableitung zu
\[
\frac{1}{\kappa}
\frac{\partial}{\partial t}
=
\frac{\partial}{\partial (\kappa t)}
=
\frac{\partial}{\partial\tau}.
\]
Durch Wahl einer geeigneten Zeiteinheit lässt sich die 
Wärmeleitungsgleichung auf $\mathbb{R}^n$ also in die Form
\begin{equation}
\frac{\partial u}{\partial t}
=
\Delta u
\label{buch:feldgleichungen:waermeleitung:eqn:normalisiert}
\end{equation}
bringen.

Die Gleichung \eqref{buch:feldgleichungen:waermeleitung:eqn:normalisiert}
hat die Fundamentallösung
\begin{equation}
u(x,t)
=
\frac{1}{(\!\sqrt{4\pi t})^{\frac{n}2}}
e^{-\frac{|x|^2}{4t}}
,
\label{buch:feldgleichungen:waermeleitung:eqn:fundamentalloesung}
\end{equation}
wie man durch Einsetzen prüfen kann.
In einer Dimension, also für $n=1$, sind die Ableitungen
\[
\left.
\begin{aligned}
\frac{\partial u}{\partial t}
&=
\frac{1}{4\!\sqrt{\pi t}}
\biggl(
\frac{x^2}{2t^2}
-
\frac{1}{t}
\biggr)
e^{-\frac{x^2}{4t}}
\\
\frac{\partial u}{\partial x}
&=
-
\frac{1}{4\!\sqrt{\pi t}} \frac{x}{t} e^{-\frac{x^2}{4t}}
&&\Rightarrow&
\frac{\partial^2 u}{\partial x^2}
&=
\frac{1}{4\!\sqrt{\pi t}}
\biggl(
\frac{x^2}{2t^2}
-\frac{1}{t}
\biggr)
e^{-\frac{x^2}{4t}}
\end{aligned}
\right\}
\quad
\Rightarrow
\quad
\frac{\partial u}{\partial t}=\Delta u.
\]
Die Wärmeleitungsgleichung ist also in diesem Fall erfüllt.

Für $n>1$ muss $\operatorname{grad}u$ und davon die Divergenz
berechnet werden.
Dies wird einfacher, wenn man beachtet, dass
\[
u(x,t)
=
u_0(x^1,t)
\dots
u_0(x^2,t)
\]
ist, wobei $u_0(x,t)$, die eindimensionale Fundamentallösung
von \eqref{buch:feldgleichungen:waermeleitung:eqn:fundamentalloesung}
ist.
Die Zeitableitung von $u(x,t)$ ist nach der Produktregel
\[
\frac{\partial u}{\partial t}(x,t)
=
\sum_{i=1}^n
u_0(x^1,t)\cdots u_0(x^{i-1})
\cdot
\frac{\partial u_0}{\partial t}(x^i,t)
\cdot
u_0(x^{i+1},t)\cdots u_0(x^n).
\]
Der Gradient $\operatorname{grad}u$ hat die Komponenten
\[
\frac{\partial u}{\partial x^i}
=
\frac{\partial}{\partial x^i}
\prod_{k=1}^n
u_0(x^k,t)
=
u_0(x^1,t)\cdots u_0(x^{i-1},t)
\cdot
\frac{\partial u_0}{\partial x}(x^i,t)
\cdot
u_0(x^{i+1},t) \cdots u_0(x^n,t).
\]
Daraus kann man die Divergenz berechnen als
\begin{align*}
\Delta u(x,t)
=
\operatorname{div}
\operatorname{grad}
u(x,t)
&=
\sum_{i=1}^n
u_0(x^1,t)\cdots u_0(x^{i-1},t)
\cdot
\frac{\partial^2 u_0}{\partial x^2}(x^i,t)
\cdot
u_0(x^{i+1},t) \cdots u_0(x^n,t).
\end{align*}
Die Differenz ist
\begin{align*}
\frac{\partial u}{\partial t}
-
\Delta u
&=
\sum_{i=1}^n
u_0(x^1,t)\cdots u_0(x^{i-1},t)
\cdot
\biggl(
\underbrace{
\frac{\partial u_0}{\partial t}(x^i,t)
-
\frac{\partial^2 u_0}{\partial x^2}(x^i,t)
}_{\displaystyle = 0}
\biggr)
\cdot
u_0(x^{i+1},t) \cdots u_0(x^n,t)
\\
&=
0.
\end{align*}
Damit ist gezeigt, dass $u(x,t)$ eine Lösung der Wärmeleitungsgleichung
ist.

Die Funktion $u_0(x,t)$ ist die Normalerteilung mit der Varianz $2t$ und
$u(x,t)$ ist eine $n$-dimensionale Normalverteilung.
Dies deckt sich mit den Erwartungen, die sich aus dem zentralen
Grenzwertsatz ergeben.
Die Diffusion von Molekülen ist setzt sich aus sehr vielen sehr
kleinen, zufälligen Stössen zusammen.
Der zentrale Grenzwertsatz sagt unter diesen Voraussetzungen eine
Normalverteilung voraus.
