%
% Der riemannsche Krümmungstensor
%
\section{Berechnung der Krümmung
\label{buch:kruemmung:section:riemann}}
\kopfrechts{Berechnung der Krümmung}
Das eingangs am Beispiel des Paralleltransports entlang eines
Dreiecksweges auf einer Kugeloberfläche illustrierte Phänomen
der Drehung eines paralleltransportierten Vektors soll jetzt
als Ausgangspunkt der Definition eines Krümmungsmasses sein.
Im Geiste der früheren Kapitel muss die Krümmung als eine 2-Form
beschrieben werden können, da die Krümmung offenbar durch Bewegung
entlang des Randes eines Flächenstücks erkennbar wird.
Allerdings können die Werte nicht nur skalare Werte sein, da
sich die Drehung eines Vektors nur mit Hilfe einer Matrix
beschreiben lässt.
Wir erwarten daher einen Tensor vierter Stufe.
Dieser Tensor ist der riemannsche Krümmungstensor.

%
% Der riemannsche Krümmungstensor
%
\subsection{Der riemannsche Krümmungstensor}
Das Einführungsbeispiel der Winkeldrehung eines paralleltransportieren
Vektors motiviert die Definition der Krümmung als die Abbildung 
der Tangentialvektoren beim Paralleltransport.
Ein Koordinatenparallelgramm mit den Ecken
\begin{center}
\begin{tikzpicture}[>=latex,thick]
\begin{scope}[xshift=-6cm,yshift=-0cm]
\node at (0,0) {$\displaystyle
\begin{aligned}
A &= x,\\
B &= x+\vec{h},\\
C &= x+\vec{h}+\vec{k}\\
\text{und}\quad
D &= x+\vec{k}
\end{aligned}$};
\end{scope}
\begin{scope}
\coordinate (A) at (-1.5,-1);
\coordinate (B) at (1.0,-0.5);
\coordinate (C) at (1.5,1);
\coordinate (D) at (-1.0,0.5);
\fill[color=gray!20] (A) -- (B) -- (C) -- (D) -- cycle;
\node at ($0.25*(A)+0.25*(B)+0.25*(C)+0.25*(D)$) {$\vec{h}\wedge\vec{k}$};
\draw[->] (A) -- (B);
\draw[->] (B) -- (C);
\draw[->] (C) -- (D);
\draw[->] (D) -- (A);
\fill (A) circle[radius=0.05];
\fill (B) circle[radius=0.05];
\fill (C) circle[radius=0.05];
\fill (D) circle[radius=0.05];
\node at (A) [below left] {$A$};
\node at (B) [below right] {$B$};
\node at (C) [above right] {$C$};
\node at (D) [above left] {$D$};
\node at ($0.5*(A)+0.5*(B)$) [below] {$\vec{h}$};
\node at ($0.5*(A)+0.5*(D)$) [left] {$-\vec{k}$};
\node at ($0.5*(B)+0.5*(C)$) [right] {$\vec{k}$};
\node at ($0.5*(D)+0.5*(C)$) [above] {$-\vec{h}$};
\end{scope}
\end{tikzpicture}
\end{center}
hat als Rand einen geschlossen Pfad, entlang dem ein Vektor parallel
transportiert werden soll.
Ein Vektor $A$ mit den Komponenten $a^i$ wird beim Transport entlang
des Weges linear abgebildet auf einen Vektor
$R(\vec{h}\wedge\vec{k})\cdot A$ abgebildet, der in linearer Näherung
nur vom 2-Vektor $\vec{h}\wedge\vec{k}$ und vom Vektor $A$ abhängt.
Der {\em riemannsche Krümmungstensor} ist daher die lineare Abbildung
\index{riemannscher Krummungstensor@riemannscher Krümmungstensor}%
$R(\vec{h}\wedge\vec{k})$.

\subsubsection{Tangentialvektoren der Drehgruppen}
Der Krümmungstensor hängt antisymmetrisch von den Komponenten von
$\vec{h}$ und $\vec{k}$ ab und liefert eine Matrix, die von einer
Drehung herrührt.
Sei also $D(t)\in\operatorname{SO}(n)$ ein Weg in der Gruppe
der $n$-dimensionalen Drehmatrizen, die beim Parameterwert $t=0$
durch die Einheitsmatrix $I\in \operatorname{SO}(n)$ geht.
Er erfüllt
\begin{align*}
D(t)^tD(t) &= I
&&\text{und}&
\det D(t) &= 1.
\intertext{Die Ableitung nach $t$ an der Stelle $t=0$ ist}
\bigl(
\dot{D}(t)^tD(t)
+
D(t)^t\dot{D}^t
\bigr)\Bigl|_{t=0}
&=
0
&&\text{und}&
\operatorname{Spur}\dot{D}(0)
&=
0.
\end{align*}
Aus der linken Gleichung wird mit $D(0)=I$ die Bedingung
\[
\dot{D}(0) + \dot{D}(0)^t
\qquad\Rightarrow\qquad
\dot{D}(0)^t
=
-\dot{D}(0).
\]
Die Tangentialvektoren an die Drehgruppe sind als antisymmetrische
Matrizen mit verschwindender Spur.

\begin{beispiel}
Die Matrizen in der zweidimensionalen Drehmatrix sind
\[
D(t)
=
\begin{pmatrix}
\cos kt &          - \sin kt\\
\sin kt & \phantom{-}\cos kt
\end{pmatrix}
\]
mit der Ableitung
\[
\dot{D}(0)
=
\begin{pmatrix}
0 & -k\\
k &\phantom{-}0
\end{pmatrix}.
\qedhere
\]
\end{beispiel}

Die oben durchgeführten Rechnungen gelten nur in $\operatorname{SO}(n)$,
die mit dem Standardskalarprodukt definiert wird.
Auf einer beliebigen riemannschen Mannigfaltigkeit sind Drehmatrizen $D$
durch die Eigenschaft gegeben, dass
\[
D^t(t)GD(t)
=
G
\]
gilt.
Durch Ableitung nach $t$ an der Stelle $t=0$ folgt
\[
\dot{D}(0)^tG + G\dot{D}(0)
=
0
\qquad\Rightarrow\qquad
G\dot{D}(0)
=
-\dot{D}(0)^tG.
\]
Schreibt man die Komponenten von $\dot{D}(0)$ als $D^i\mathstrut_l$,
dann bedeutet dies
\[
g_{uv}D^{v}\mathstrut_l
=
-
\sum_{v}
D^{u}\mathstrut_v
g_{vl}
\]
Zieht man den oberen Index von $D^u\mathstrut_v$ mit herunter,
entsteht ein symmetrischer Tensor
\[
D_{uv}
=
g_{ui} D^i\mathstrut_v
\qquad\Rightarrow\qquad
D_{uv}
=
-D_{vu}.
\]

\subsubsection{Symmetrien des Krümmungstensors}
Der Krümmungstensor $R(\vec{h}\wedge\vec{k})$ hat als Wert eine Matrix,
deren Komponenten wir mit $R^i\mathstrut_l$ bezeichnen können.
Sie hängt antisymmetrisch von den Komponenten der beiden Vektoren
$\vec{h}\wedge\vec{k}$ ab.
In Komponenten ausgedrückt ist also
\[
R^i\mathstrut_l(\vec{h}\wedge\vec{k})
=
R^i\mathstrut_{luv} h^uk^v,
\]
wobei wie üblich nach der einsteinschen Summationskonvention über $u$
$v$ summiert werden muss.
Falls der metrische Tensor die Einheitsmatrix ist, muss $R^i\mathstrut_l$
antisymmetrisch sein, doch dies ist keine Eigenschaft, die allgemein
kovariant definiert ist.
Aus den Resultaten des letzten Abschnitts folgt aber, dass nach
Herunterziehen des oberen Index mithilfe von $g_{uv}$ eine
antismmetrische Matrix entsteht.
Für beliebige Vektoren $\vec{h}$ und $\vec{v}$ muss daher
$g_{li}R^i\mathstrut_l(\vec{h}\wedge\vec{k})=R_{il}(\vec{h}\wedge\vec{k})$
ein antisymmetrischer Tensor zweiter Stufe sein.
In Komponenten muss für beliebige Indizes $u$ und $v$ muss
$R_{jluv}=g_{ji}R^i\mathstrut_{luv}$ antisymmetrisch in den
Indizes $j$ und $l$ sein.

\subsubsection{Kovariante Ableitung und Krümmungstensor}
Wir berechnen jetzt den Paralleltransport des Vektors mit den
Komponenten $a^i$ entlang der Kantenpfade $ABC$ und $ADC$ 
des von $\vec{h}$ und $\vec{k}$ aufgespannten Parallelogramms.
Der Transport von $A$ nach $B$ ändert den Vektor um die Komponenten
\[
-\Gamma^l_{uv} a^u h^v 
\]
Der neue Vektor muss jetzt um den Vektor $\vec{k}$ transportiert
werden, der transportierte Vektor ist
\[
b^l
=
a^l-\Gamma^l_{uv} a^u h^v.
\]
Dies ergibt
\begin{align*}
\biggl(
\frac{\partial b^s}{\partial x^m}
+
\Gamma^s_{nm} b^n
\biggr)
k^m
&=
\biggl(
\frac{\partial \Gamma^s_{uv}}{\partial x^m}
a^u
h^v
-
\Gamma^s_{nm}
a^n
+
\Gamma^s_{nm} \Gamma^n_{uv}
a^u
h^v
\biggr)
k^m
\\
&=
\biggl(
\frac{\partial \Gamma^s_{uv}}{\partial x^m}
+
\Gamma^s_{nm}\Gamma^n_{uv}
\biggr)
a^u
h^v k^m
-
\Gamma^s_{nm}a^nk^m.
\end{align*}
Vertauschung von $v$ und $m$ führt auf den entsprechenden
Ausdruck für den Weg $ADC$.
Die Differenz ist der gesucht Krümmungstensor
\begin{align*}
R^s\mathstrut_{uvm}
&=
\frac{\partial\Gamma^s_{uv}}{\partial x^m}
-
\frac{\partial\Gamma^s_{um}}{\partial x^v}
-
\Gamma^s_{nm}\Gamma^n_{uv}
+
\Gamma^s_{nv}\Gamma^n_{um}
\end{align*}
Dies sind die Komponenten des riemannschen Krümmungstensors.
Durch herunterziehen des ersten Index erhält man die Komponenten
\[
R_{ruvm}
=
g_{rs}
\biggl(
\frac{\partial\Gamma^s_{uv}}{\partial x^m}
-
\frac{\partial\Gamma^s_{um}}{\partial x^v}
-
\Gamma^s_{nm}\Gamma^n_{uv}
+
\Gamma^s_{nv}\Gamma^n_{um}
\biggr)
\]
des kovarianten riemannschen Tensors.
Er ist antisymmetrisch in den ersten zwei und den letzten zwei
Indizes.

\begin{beispiel}
Der riemannsche Krümmungstensor für die Ebene in Polarkoordinaten.

In Polarkoordinaten $(r,\vartheta)$ wird die euklidische Metrik
durch die Koeffizienten
\[
g_{11} = 1,\quad
g_{22} = r^2,\qquad
g
=
\begin{pmatrix}
1&0\\
0&r^2
\end{pmatrix},
\quad
g^{-1}
=
\begin{pmatrix}
1&0\\
0&r^{-2}
\end{pmatrix},
\]
gegeben.
Daraus kann man die Christoffel-Symbole erster Art
\begin{align*}
\Gamma_{1,11}                 & =  0 &
\Gamma_{1,12} = \Gamma^1_{21} & =  0 &
\Gamma_{1,22}                 & = -r
\\
\Gamma_{2,11}                 & = 0 &
\Gamma_{2,12} = \Gamma_{2,21} & = r &
\Gamma_{2,22}                 & = 0
\intertext{und zweiter Art}
\Gamma^1_{11}                 & = 0  &
\Gamma^1_{12} = \Gamma^1_{21} & =  0 &
\Gamma^1_{22}                 & = -r
\\
\Gamma^2_{11}                 & = 0  &
\Gamma^2_{12} = \Gamma_{2,21} & = \frac{1}{r} &
\Gamma^2_{22}                 & = 0
\end{align*}
berechnen.
Verwendet man diese Koeffizienten, um den riemannschen Krümmungstensor
zu berechnen, erhält man $0$ für alle Komponenten.
\end{beispiel}

\begin{beispiel}
Der riemannsche Krümmungstensor für die Kugeloberfläche.
\end{beispiel}

\subsubsection{Herleitung mit Hilfe des Satzes von Green}
Für die Berechnung des Krümmungstensors haben wir ein konkretes
Parallelogramm verwendet und den Transport eines Vektors entlang
seiner Kanten approximiert.
Der Krümmungstensor ergab sich dann in linearer Näherung.
Dieser Zugang hat aber den Nachteil, dass er einen speziellen, nicht
differenzierbaren Weg und eine etwas holperig wirkende Approximation
verwendet, der Zulässigkeit nicht nachgeprüft wurde.

Der Satz von Green ermöglicht, ein Wegintegral über einen geschlossenen
Weg direkt zu formulieren und in ein Flächenintegral umzuwandeln.
Seien also $a^k$ die Komponenten eines Vektors $A$ und $\gamma(t)$ in
Weg mit den Koordinaten $x^i(t)$.
Der Vektor ändert beim Paralleltransport entlang der Kurve um den
Betrag $\Delta a^k$.
Die kovariante Ableitung $\nabla_{\dot{\gamma}} A$ gibt die infinitesimale
Änderung.
Die Änderung $\Delta a^k$ kann daher als das Integral
\begin{align*}
\Delta A
&=
\oint_{\gamma}
\nabla_{\dot{\gamma}}
A
\,d\gamma(t)
\intertext{über den Weg $\gamma$, oder in Komponenten}
\Delta a^k
&=
\int 
\biggl(
\frac{\partial a^k}{\partial x^v}
+
\Gamma^k_{uv} a^u
\biggr)
\dot{x}^v
\,dt.
\end{align*}
Um den Satz von Green anwenden zu können, brauchen wir ein ebenes
Kurvenintegral.
Wir betrachten daher einen Weg in der $x^v$-$x^s$-Koordinaten-Ebene.
Die Änderung entlang eines solchen Weges ist dann das Integral
\begin{align}
\Delta a^k
&=
\int
\biggl(
\frac{\partial a^k}{\partial x^v} + \Gamma^k_{uv}a^u
\biggr)
\,
dx^v
+
\biggl(
\frac{\partial a^k}{\partial x^s} + \Gamma^k_{us}a^u
\biggr)
dx^s,
\intertext{welches mit dem Satz von Green in das Integral}
&=
\int
\frac{\partial}{\partial x^s}
\biggl(
\frac{\partial a^k}{\partial x^v} + \Gamma^k_{uv}a^u
\biggr)
-
\frac{\partial}{\partial x^v}
\biggl(
\frac{\partial a^k}{\partial x^s} + \Gamma^k_{us}a^u
\biggr)
\,dx^u\,dx^s
\notag
\\
&=
\int
\frac{\partial^2 a^k}{\partial x^s\,\partial x^v}
-
\frac{\partial^2 a^k}{\partial x^v\,\partial x^s}
+
\frac{\partial \Gamma^k_{uv}}{\partial x^s} a^u
+
\Gamma^k_{uv}\frac{\partial a^u}{\partial x^s}
-
\frac{\partial \Gamma^k_{us}}{\partial x^v} a^u
-
\Gamma^k_{us}\frac{\partial a^u}{\partial x^v}
\,dx^v\,dx^s.
\label{buch:kruemmung:eqn:greendelta1}
\end{align}
Die zweiten Ableitungen heben sich weg und müssen daher nicht weiter
berücksichtigt werden.
Für den paralleltransportierten Vektor verschwindet die kovariante
Ableitung, so dass sie sich nach der partiellen Ableitung von $a^k$
nach den Koordinaten auflössen lässt.
Dies ergibt
\[
\frac{\partial a^u}{\partial x^s}
=
\Gamma^u_{sl}a^l.
\]
Eingesetzt in den Ausdruck
\eqref{buch:kruemmung:eqn:greendelta1}
ergibt sich
\begin{align*}
\Delta a^k
&=
\int
\frac{\partial\Gamma^k_{uv}}{\partial x^s} a^u
-
\frac{\partial\Gamma^k_{us}}{\partial x^v} a^u
+
\Gamma^k_{uv}\Gamma^u_{sl} a^l
-
\Gamma^k_{us}\Gamma^u_{vl} a^l
\,dx^s\,dx^v
\\
&=
\int
\biggl(
\frac{\partial\Gamma^k_{uv}}{\partial x^s}
-
\frac{\partial\Gamma^k_{us}}{\partial x^v}
+
\Gamma^k_{lv}\Gamma^l_{su}
-
\Gamma^k_{ls}\Gamma^l_{vu}
\biggr)
a^u
\,dx^s\,dx^v.
\end{align*}
Der Klammerausdruck im Integral ist wieder der riemannsche Krümmungstensor.

%
% Die Bianchi-Identitäten
%
\subsection{Die Bianchi-Identitäten}
Im vorangegangenen Abschnitt wurde erkannt, dass der riemannsche 
Krümmungstensor antisymmetrisch im ersten und letzten Indexpaar ist,
und symmetrisch bei Vertauschung der beiden Indexpaare.
Die Bianchi-Identitäten sind weitere Beziehungen der Komponenten
des Krümmungstensors untereinander, so dass sich die Information 

%
% Die ersrte Bianchi-Identität
%
\subsubsection{Die algebraische Bianchi-Identität}
Die algebraische Bianchi-Identität ist die Eigenschaft, dass die
Summe der Werte des Krümmungstensors bei zyklischer Vertauschung
dreier Vektoren verschwindet.
Dies erinnert an die Jacobi-Identität von
Satz~\ref{buch:koordinaten:diffop:satz}
und tatsächlich wird der
Beweis die Jacobi-Identität für die Lie-Ableitung wesentlich
verwenden.

\begin{satz}[algebraische Bianchi-Identität]
Für den riemannschen Krümmungstensor $\textit{Rm}$ und für vier beliebige
Vektorfelder $X$, $Y$, $Z$, $W$ gilt
\[
\textit{Rm}(X,Y,Z,W)
+
\textit{Rm}(X,Z,W,Y)
+
\textit{Rm}(X,W,Y,Z)
=
0.
\]
In Komponenten ausgedrückt lautet sie
\[
R_{abcd}
+
R_{acdb}
+
R_{adbc}
=
0.
\]
\end{satz}

Die algebraische Bianchi-Identität wird traditionellerweise auch
mit dem wenig deskriptiven Namen erste Bianchi-Identität bezeichnet.
\index{Bianchi-Identitat@Bianchi-Identität!erste}%
\index{erste Bianchi-Identitat@erste Bianchi-Identität}%

\begin{proof}
Wir zeigen durch direkte Rechnung dass der Ausdruck
\[
B
=
\textit{Rm}(X,Y,Z,W)
+
\textit{Rm}(X,Z,W,Y)
+
\textit{Rm}(X,W,Y,Z)
\]
verschwindet.
Da
\[
\textit{Rm}(X,Y,Z,W)
=
R(Z,W)Y
\]
ist, ist das verschwinden von $B$ gleichbedeutend damit, dass
\[
C
=
R(Z,W)Y
+
R(W,Y)Z
+
R(Y,Z)W
\]
verschwindet.

Um dies zu zeigen, expandieren wir die Terme in $C$ mit der Definition
des Krümmungstensors 
\begin{align*}
C
&=
(\nabla_W\nabla_Z Y - \nabla_Z\nabla_W Y - \nabla_{[W,Z]}Y)
\\
&\qquad \mathstrut +
(\nabla_Y\nabla_W Z - \nabla_W\nabla_Y Z - \nabla_{[Y,W]}Z)
\\
&\qquad \mathstrut +
(\nabla_Z\nabla_Y W - \nabla_Y\nabla_Z W - \nabla_{[Z,Y]}W)
\intertext{Wir gruppieren die zweiten Ableitungen nach dem gleichen Vektor}
&=
\nabla_W(\nabla_Z Y - \nabla_Y Z)
+
\nabla_Z(\nabla_Y W - \nabla_W Y)
+
\nabla_Y(\nabla_W Z - \nabla_Z W)
\\
&\qquad\mathstrut
-
\nabla_{[W,Z]}Y
-
\nabla_{[Y,W]}Z
-
\nabla_{[Z,Y]}W.
\intertext{Mit der Symmetrie der kovarianten Ableitung können die ersten
drei Terme vereinfacht werden zu}
&=
\nabla_W[Z,Y] + \nabla_Z[Y,W] + \nabla_Y[W,Z]
-
\nabla_{[W,Z]}Y
-
\nabla_{[Y,W]}Z
-
\nabla_{[Z,Y]}W.
\intertext{Mit erneuter Verwendung der Symmetrie der kovarianten Ableitung
folgt schliesslich}
&=
[W,[Z,Y]]
+
[Z,[Y,W]]
+
[Y,[W,Z]]
=
0
\end{align*}
aufgrund der Jacobi-Identität für die Lie-Klammer.
\end{proof}

%
% Die Anzahl unabhängiger Komponenten des Krümmungstensors
%
\subsubsection{Die Anzahl unabhängiger Komponenten des Krümmungstensors}
Wir werden im Abschnitt~\ref{buch:kruemmung:section:flaeche} für den Fall
zweidimensionaler Mannigfaltigkeiten zeigen, dass der Krümmungstensor
nur eine unabhängige Komponente hat.
Dazu werden nur die früher diskutierten Symmetrieeigenschaften des
Krümmungstensors benötigt, die Bianchi-Identität ist automatisch immer
erfüllt.

Für dreidimensionale Mannigfaltigkeiten ist dies nicht mehr der Fall.
In diesem Abschnitt zählen wir die unabhängigen Komponenten des
kovarianten Krümmungstensors im Falle $n=3$.
Aus den grundlegenden Symmetrieeigenschaften folgt, dass nur die
Komponenten $R_{abcd}$ mit $a<b$ und $c<d$ betrachtet werden müssen,
die Komponenten mit $a=b$ oder $c=d$ verschwinden und die Fälle
$a>b$ und $c>d$ können durch Indexvertauschung erhalten werden.
Es gibt nur drei Indexpaare $(a,b)$, die die Bedingung $a<b$
erfüllen, nämlich $(1,2)$, $(1,3)$ und $(2,3)$.
Es verbleiben dann neun Komponenten, von denen aber wegen der
Symmetrie bei Vertauschung der Indexpaare zwei übereinstimmen:
\begin{equation}
R_{1212},
R_{1213} = R_{1312},
R_{1223} = R_{2312},
R_{1313},
R_{1323} = R_{2313},
R_{2323}.
\label{buch:kruemmung:kruemmung:eqn:R3}
\end{equation}
Der Krümmungstensor hat nach dieser Rechnung sechs unabhängige Komponenten.
Insbesondere kann man aus \eqref{buch:kruemmung:kruemmung:eqn:R3} ablesen,
dass Komponenten mit mehr als zwei gleichen Indizes verschwinden.

Die Bianchi-Identität auferlegt den Komponenten des Krümmungstensors
weitere Einschränkungen.
Da es nur drei verschiedene Indizes gibt, kommt ein Index notwendigerweise
zweimal vor.
Wir betrachten die folgenden zwei Fälle:
\begin{itemize}
\item
Der erste Index kommt auch in der zyklisch permutierten
Dreiergruppe vor.
Wir schreiben die Bianchi-Identität mit den Indizes $aabc$,
wobei $b\ne a$ und $c\ne a$ ist.
Sie erhält so die Form
\begin{align*}
R_{aabc}
+
R_{abca}
+
R_{acab}
&=
0
\quad
\Rightarrow
\quad
R_{abca}
+
R_{abac}
=
0
\quad\Rightarrow\quad
R_{abca} = -R_{abac}.
\end{align*}
Diese Beziehung ist aber wegen der grundlegenden Symmetrieeigenschaften
immer erfüllt.
\item
Der erste Index kommt in der zyklisch permutierten Gruppe nicht mehr vor.
In diesem Fall hat das Indexquadrupel die Form $abbc$, wobei $b\ne a$ und
$c\ne a$.
Damit bekommt die Bianchi-Identität die Form
\begin{align*}
R_{abbc}
+
R_{abcb}
+
R_{acbb}
&=
0
\quad \Rightarrow\quad
R_{abbc}
+
R_{abcb}
=
0
\quad\Rightarrow\quad
R_{abbc} = -R_{abcb}.
\end{align*}
Auch dies ist eine der grundlegenden Symmetriebeziehungen.
\end{itemize}
Wir schliessen daher, dass die Bianchi-Identitäten keine weiteren
Symmetrien zwischen den Komponenten des Krümmungstensors liefern
können.

Erst in Dimension $n\ge 4$ ergeben sich aus den Bianchi-Identitäten
zusätzliche Einschränkungen an die Tensorkomponenten.
Zum Beispiel ist die Beziehung
\[
R_{1234}
+
R_{1342}
+
R_{1423}
=
0
\quad\Leftrightarrow\quad
R_{1234}
-
R_{1324}
+
R_{1423}
=
0
\]
nicht aus den grundlegenden Symmetrieeigenschaften abzuleiten.

%
% Die differentielle Bianchi-Identität
%
\subsubsection{Die differentielle Bianchi-Identität}
Die zweite Bianchi-Identität stellt eine Symmetrie der kovarianten Ableitung
des Krümmungstensors her.
\index{Bianchi-Identitat@Bianchi-Identität!zweite}%
\index{zweite Bianchi-Identitat@zweite Bianchi-Identität}%
Sie heisst deshalb auch die \emph{differentielle Bianchi-Identität}.
\index{Bianchi-Identitat@Bianchi-Identität!differentiell}%
\index{differentielle Bianchi-Identitat@differentielle Bianchi-Identität}%

