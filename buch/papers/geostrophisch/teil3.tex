%
% teil3.tex -- Beispiel-File für Teil 3
%
% (c) 2020 Prof Dr Andreas Müller, Hochschule Rapperswil
%
% !TEX root = ../../buch.tex
% !TEX encoding = UTF-8
%
\section{Geostrophische Näherung / heutiger Ansatz
\label{geostrophisch:section:teil3}}
\kopfrechts{Teil 3}
Sed ut perspiciatis unde omnis iste natus error sit voluptatem
accusantium doloremque laudantium, totam rem aperiam, eaque ipsa
quae ab illo inventore veritatis et quasi architecto beatae vitae
dicta sunt explicabo. Nemo enim ipsam voluptatem quia voluptas sit
aspernatur aut odit aut fugit, sed quia consequuntur magni dolores
eos qui ratione voluptatem sequi nesciunt. 
\begin{equation}
\boldsymbol{
f\, *\vec{k} \times \vec{v}_g 
+
\frac{1}{\rho} \nabla p
=
0
}
\label{geostrophisch:equation3}
\end{equation}
Die Gleichung besagt, der geostrophische Wind entsteht genau dann, wenn sich die Druckgradientenkraft\eqref{geostrophisch:equation2} und die Corioliskraft\eqref{geostrophisch:equation1} ausgleichen.
Heisst \eqref{geostrophisch:equation1} Corioliskraft ist gleich \eqref{geostrophisch:equation2} Gradientenkraft.




\subsection{De finibus bonorum et malorum
\label{geostrophisch:subsection:malorum}}
At vero eos et accusamus et iusto odio dignissimos ducimus qui
blanditiis praesentium voluptatum deleniti atque corrupti quos
dolores et quas molestias excepturi sint occaecati cupiditate non
provident, similique sunt in culpa qui officia deserunt mollitia
animi, id est laborum et dolorum fuga. Et harum quidem rerum facilis
est et expedita distinctio. Nam libero tempore, cum soluta nobis
est eligendi optio cumque nihil impedit quo minus id quod maxime
placeat facere possimus, omnis voluptas assumenda est, omnis dolor
repellendus. Temporibus autem quibusdam et aut officiis debitis aut
rerum necessitatibus saepe eveniet ut et voluptates repudiandae
sint et molestiae non recusandae. Itaque earum rerum hic tenetur a
sapiente delectus, ut aut reiciendis voluptatibus maiores alias
consequatur aut perferendis doloribus asperiores repellat.


