%
% anwendungenFelder.tex -- 3.	Anwendungen auf Felder, PDEs in ODE umwandeln (Grundlage, um Feldgleichungen als eine Familie von Oszillator-Gleichungen zu interpretieren)
%
% (c) 2020 Prof Dr Andreas Müller, Hochschule Rapperswil
%
% !TEX root = ../../buch.tex
% !TEX encoding = UTF-8
%

%    Anwendungen auf Felder, PDEs in ODE umwandeln (Grundlage, um Feldgleichungen als eine Familie von Oszillator-Gleichungen zu interpretieren) 


\section{Anwendung auf Feld\label{fourier:section:teil0}}

\kopfrechts{Anwendung auf Feld}
%Motivation. E & B Felder enfüllen wellengleichung. noch schreiben!
Gewöhnliche Differentialgleichungen sind berühmt berüchtigt schwierig zu lösen.
Schlimmer geht es immer.
Partielle Differenzialgleichungen enthaltet Ableitungen nach mehreren Variabeln.
In diesem Abschnitt vereinfachen wir eine Partielle, mithilfe der Fourierreihe, in eine Gewöhnliche.
Alle Ableitungen sollten schlussendlich den identischen Nenner besitzen. 
Im unteren Beispiel wäre das die Zeit $\partial t$.
Als partielle Differentialgleichung verwenden wir die eindimensionale Wellengleichung mit der Lichtgeschwindigkeit als Ausbreitungsgeschwindigkeit. 
Dieses Modell eignet sich besonders gut, da es auch auf elektromagnetische Felder anwendbar ist!

\begin{equation}
	\frac{\partial^2 u(x, t)}{\partial t^2} = c^2 \cdot \frac{\partial^2 u(x, t)}{\partial x^2}
\end{equation}

Unter der Annahme, dass $u(x, t)$ eine sich periodisch wiederholende Funktion ist, führen wir eine Fourierreihen-Entwicklung durch. 
Neu besteht $u(x, t)$ aus einer unendlichen Anzahl von Schwingungen:

\begin{equation}
	u(x,t) = \frac{a_0(t)}{2} + \sum_{n=1}^{\infty} \left( a_n(t) \cos(n \omega x) + b_n(t) \sin(n \omega x) \right)
\end{equation}

Alle Fourier Koeffizienten $a_0(t)$, $a_n(t)$ und $b_n(t)$ sind neu von der Zeit abhängig. 
Nun wird $u(x,t)$ in seiner Summenform in die Wellengleichung eingebaut. 
Dazu muss $u(x,t)$ je zwei Mal nach $t$ und $x$ abgeleitet werden:

\begin{equation}
	\frac{\partial^2 u(x,t)}{\partial t^2} = \frac{1}{2} \frac{\partial^2 a_0(t)}{\partial t^2} +  \sum_{n=1}^{\infty} \left( \frac{\partial^2}{\partial t^2} a_n(t) \cos(n \omega x) + \frac{\partial^2}{\partial t^2} b_n(t) \sin(n \omega x) \right)
\end{equation}

\begin{equation}
	\frac{\partial^2 u(x,t)}{\partial x^2} = \sum_{n=1}^{\infty} \left( -a_n(t) n^2 \omega^2 \cos(n \omega t) - b_n(t) n^2 \omega^2 \sin(n \omega t) \right)
\end{equation}


Von nun an, wird als ableitungsoperator $\frac{d}{dt}$ verwendet, da die Gleichung nur noch von $t$ abhängig ist.
Die Resultate können jetzt in die Wellengleichung einsetzt werden. 

%\begin{equation}
%	 \frac{1}{2} \frac{d^2 a_0(t)}{d t^2} + \sum_{n=1}^{\infty} \left( \frac{d^2}{dt^2} a_n(t) \cos(n \omega x) + \frac{d^2}{dt^2} b_n(t) \sin(n \omega x) \right) = c^2  \sum_{n=1}^{\infty} \left( -a_n(t) n^2 \omega^2 \cos(n \omega t) - b_n(t) n^2 \omega^2 \sin(n \omega t) \right) 
%\end{equation}


\begin{multline}
	\frac{1}{2}\frac{d^2 a_0(t)}{dt^2}
	+ \sum_{n=1}^{\infty}\Bigl(
	\frac{d^2 a_n(t)}{dt^2}\cos(n\omega x)
	+ \frac{d^2 b_n(t)}{dt^2}\sin(n\omega x)
	\Bigr)
	= \\[-0.8ex]
	c^2 \sum_{n=1}^{\infty}\Bigl(
	-a_n(t)\,n^2\omega^2\cos(n\omega x)
	-b_n(t)\,n^2\omega^2\sin(n\omega x)
	\Bigr)
\end{multline}


Diese etwas grosse Gleichung kann nun mithilfe eines Koeffizientenvergleichs gelöst werden. so entstehen folgende drei Gleichungen:

\begin{equation}
	\frac{1}{2} \frac{d^2 a_0(t)}{d t^2} = 0
\end{equation}

\begin{equation}
	\sum_{n=1}^{\infty}
	\frac{d^2 a_n(t)}{dt^2}\cos(n\omega x)
	 = c^2 \sum_{n=1}^{\infty}
	-a_n(t)\,n^2\omega^2\cos(n\omega x)
	\end{equation}

\begin{equation}
	\sum_{n=1}^{\infty}
	\frac{d^2 b_n(t)}{dt^2}\sin(n\omega x) = c^2 \sum_{n=1}^{\infty}
	-b_n(t)\,n^2\omega^2\sin(n\omega x)
\end{equation}

Der variable Mittelwert $a_0(t)$ kann somit gleich Null gesetzt werden.
Die Gleichungen mit $a_n(t)$ und $b_n(t)$ besitzen beidseitig dasselbe Summenzeichen, daher gilt die Gleichung für jedes $n$. 
Zudem sind die $cos(n\omega x)$ und $sin(n\omega x)$ zu kürzen. 
Somit entstehen folgende zwei Gleichungen:

\begin{equation}
	\frac{d^2}{dt^2} a_n(t) + a_n(t) n^2 \omega^2 c^2 = 0
	  \quad   \text{und} \quad  \frac{d^2}{dt^2} b_n(t) + b_n(t) n^2 \omega^2 c^2 = 0
\end{equation}

Wenn man im Physik unterricht aufgepasst hat, sollte einem diese Form bekannt vorkommen. Die Differentialgleichung einem ungedämpften Federpendels besitzt dieselbe Form. 

\begin{equation}
	\frac{d^2}{dt^2} x(t) + x(t) \omega^2  = 0
\end{equation}

Die lösung von dieser DGL ist einfach eine ungedämpfte schwingung der Form 

\begin{equation}
x(t) = A \cos(\omega t) + B \sin(\omega t) \quad \text{oder} \quad x(t) = C \cos(\omega t - \varphi)
\end{equation}

Dieses Resultat beschreib mit welcher Frequenz und Amplitude eine Masse an einer Feder schwingt. 

Könnte man nun dasselbe mit der resultierenden Lösung der Wellengleichung machen?
Das Resultat wäre eine unendliche Anzahl von Felderpendeln, da $n$ für jede natürliche Zahl steht. 
Die Kreisfrequenz wäre  $n \omega c$. 
Dies macht mathematisch Sinn, physikalisch müsste man sich jedoch fragen, was genau hin und her pendelt und ob eine unendliche Frequenz Sinn macht.


In diesem Kapitel wurde die Thematik stark vereinfacht dargestellt. 
Anstelle eines dreidimensionales Feldes $u(x,y,z,t)$ wurde lediglich eine eindimensionale Funktion $u(x,t)$ betrachtet. Die Fourier-Reihe lässt sich zwar auch auf mehrdimensionale Felder anwenden, der damit verbundene mathematische Aufwand ist jedoch deutlich höher. 








