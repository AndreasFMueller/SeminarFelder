%
% 3-divergenz.tex
%
% (c) 2025 Prof Dr Andreas Müller
%

%
% Kovariante Ableitung von Tensoren
%
\section{Divergenz und Erhaltungssätze
\label{buch:zusammenhang:section:divergenz}}
\kopfrechts{Divergenz und Erhaltungssätze}
In Abschnitt~\ref{buch:gauss:section:erhaltungssatz} wurde gezeigt,
wie Erhaltungssätze mit Hilfe der Divergenz formuliert werden können.
Beschreibt man den Fluss eines Feldes durch die $n-1$-dimensionale
Volumenelemente durch eine $n-1$-Form $\omega\in \Omega^{n-1}(M)$,
dann ist die äussere Ableitung $d\omega\in \Omega^n(M)$ eine $n$-Form.
Der Satz von Gauss sagt dann, dass für ein $n$-dimensionales $V$
mit dem geschlossenen, $n-1$-dimensionalen Rand $\partial V$
\[
\int_{V}d\omega
=
\int_{\partial V}\omega
\]
gilt.
Die $n$-Form $d\omega$ beschreibt die Quelldichte für das Feld im
Gebiet $V$.
Da man $d\omega$ in der Form
\[
d\omega
=
\varrho(x^1,\dots,x^n)
\,
dx^1\wedge\dots\wedge dx^n
\]
schreiben kann, betrachtet man $\varrho(x^1,\dots,x^n)$ als die
Dichte der erhaltenen Grösse.
Die Dichtefunktion $\varrho(x^1,\dots,x^n)$  ist aber abhängig
von der Wahl des Koordinatensystems, sie kann also nicht direkt 
mit einer physikalischen Grösse identifiziert werden.

Die äussere Ableitung $d\omega$ wurde mit der Divergenz des Vektorfeldes
identifiziert, dessen Komponenten die Koeffizienten der $n$ Basis-$n-1$-Formen
$dx^{i_1}\wedge\dots\wedge dx^{i_{n-1}}$, $1\le i_1<\dots<i_{n-1}\le n$, in
einem gewählten Koordinatensystem sind.
Die Kontinuitätsgleichung wurde in Abschnitt
\ref{buch:zusammenhang:erhaltungssatz:subsection:kontinuitaetsgleichung}
als der Ausdruck eines Erhaltungssatzes.
Die Komponenten des Vektorfeldes bilden den zur Dichte $\varrho$ gehörigen
erhaltenen Strom.
Auch dieses Vektorfeld hängt, genau wie die Dichtefunktion
$\varrho(x^1,\dots,x^n)$, von der Wahl des Koordinatensystems ab.

%
% Metrik und Dichte
%
\subsection{Metrik und Dichte
\label{buch:zusammenhang:divergenz:subsection:metrik}}
Die Dichtefunktion $\varrho(x^1,\dots,x^n)$ liess sich nicht 
unabhängig vom Koordinatensystem definieren, weil es keine vom
Koorinatensystem unabhängig Wahl einer Volumenform
\[
v(x^1,\dots,x^n)\,dx^1\wedge\dots\wedge dx^n
\]
gibt.
Eine Orientierung der Mannigfaltigkeit verlang zwar, dass auf
jedem Kartengebiet eine solche Form ausgezeichnet wird.
Von den Kartenwechseln wird aber nur verlangt, dass sie das Vorzeichen
der $n$-Form nicht ändern.
Es wird nichts über die Werte der $n$-Formen verlangt.

Mit einer Metrik wird es möglich, eine koordinatenunabhängige Volumenform
\[
\sqrt{\det g}
\,dx^1\wedge\dots\wedge dx^n
\]
zu definieren.
Eine beliebige $n$-Form $\omega$ kann dann in der Form
\[
\varrho(x^1,\dots,x^n)
\sqrt{\det g}
\,dx^1\wedge \dots \wedge dx^n
\]
geschrieben werden.
Die Dichtefunktion $\varrho(x^1,\dots,x^n)$ ist jetzt eine von der
Wahl des Koordinatensystems unabhängige Grösse.

Eine entsprechende ``Referenzform'' lässt sich auch für $p$-Formen mit
$p<n$ finden, die auf einer eingebetteten $p$-Mannifaltigkeit
definiert sind.
Ein $p$-dimensionale Untermannigfaltigkeit $N\subset M$ einer
riemannschen Mannigfaltigkeit $M$ mit der Metrik $g$ erbt die Metrik
von $M$.
Wegen $N\subset M$ folgt $T_xN\subset T_xM$ für jeden Punkt $x\in N$
und daher sind Tangentialvektoren von $N$ im Punkt $x$ auch
Tangentialvektoren von $M$ im selben Punkt.
Die Metrik von $M$ definiert daher auch ein Skalarprodukt von
Tangentialvektoren von $N$ und damit eine Metrik auf $N$.
Die Länge einer Kurve in $N$ wird natürlich nicht verändert.

Eine $p$-dimensionale Untermannigfaltigkeit hat daher eine $p$-dimensionale
Volumenform, die mit der gleichen Metrik definiert ist.
Auf dem $p-1$-dimensionalen Rand eines $p$-dimensionalen Gebietes definiert
die Metrik eine $p-1$-Form, die mit der $p$-Form auf dem Gebiet kompatibel
ist.

%
% Symmetrien und Erhaltungssätze
%
\subsection{Symmetrien und Erhaltungssätze
\label{buch:zusammenhang:divergenz:subsection:noether}}
Aus der Variationsrechnung \cite[Abschnitt 10.2]{buch:seminarvariation}
ist bekannt, dass zu jeder Symmetrie der Lagrange-Funktion eines
Variationsproblems ein Erhaltungssatz gehört.

XXX Divergenzform des Erhaltungssatzes

%
% Kovariante Ableitung und erhaltene Ströme
%
\subsection{Kovariante Ableitung und erhaltene Ströme
\label{buch:zusammenhang:divergenz:subsetion:kovariant}}
Die Divergenzform ermöglicht, Erhaltungssätze in Form einer 
Differentialgleichung zu formulieren.
Die Koeffizienten der Basis-$p$-Formen sind aber koordinatenabhängig,
wir können also nicht erwarten, damit einen koordinatenunabhängig
formulierten Erhaltungssatz zu konstruieren.

Die Divergenzform des Satzes von Nöther zeigt uns aber die Richtung,
in die wir gehen müssen.
Da die $n$-Form $v=\sqrt{g} dx^1\wedge \dots\wedge dx^n$
koordinatenunabhängig ist, sind Grössen, die als Dichtefunktionen
relativ zu $v$ definiert sind, Kandidaten für Grössen, für die
ein Erhaltungssatz in Divergenzform formuliert werden kann.

Etwas formaler betrachten wir jetzt eine Grösse mit Komponenten $A^i$
derart, dass $\sqrt{g}A^i$ ein Vektor ist.
Wir bezeichnen eine solche Grösse als eine Vektordichte.
In Divergenzform lautet ein Erhaltungssatz für diese Grösse
\[
\sum_{i=1}^n \frac{\partial}{\partial x^i} (\!\sqrt{g}A^i)
=
0.
\]
