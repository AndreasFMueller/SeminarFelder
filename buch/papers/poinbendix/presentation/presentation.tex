\documentclass[ngerman, aspectratio=169]{beamer}

%style
\mode<presentation>{
	\usetheme{Frankfurt}
}
%packages
\usepackage[utf8]{inputenc}
\usepackage[english]{babel}
\usepackage{graphicx}
\usepackage{array}

\newcolumntype{L}[1]{>{\raggedright\let\newline\\\arraybackslash\hspace{0pt}}m{#1}}
\usepackage{ragged2e}

\usepackage{bm} % bold math
\usepackage{amsfonts}
\usepackage{amssymb}
\usepackage{mathtools}
\usepackage{amsmath}
\usepackage{multirow} % multi row in tables
\usepackage{scrextend}

\usepackage{tikz}

\usepackage{algorithmic}

%\usepackage{algorithm} % http://ctan.org/pkg/algorithm
%\usepackage{algpseudocode} % http://ctan.org/pkg/algorithmicx

%\usepackage{algorithmicx}


%citations
\usepackage[style=verbose,backend=biber]{biblatex}
\addbibresource{references.bib}



\usefonttheme[onlymath]{serif}

%Beamer Template modifications
%\definecolor{mainColor}{HTML}{0065A3} % HSR blue
\definecolor{mainColor}{HTML}{D72864} % OST pink
\definecolor{invColor}{HTML}{28d79b} % OST pink
\definecolor{dgreen}{HTML}{38ad36} % Dark green

%\definecolor{mainColor}{HTML}{000000} % HSR blue
\setbeamercolor{palette primary}{bg=white,fg=mainColor}
\setbeamercolor{palette secondary}{bg=orange,fg=mainColor}
\setbeamercolor{palette tertiary}{bg=yellow,fg=red}
\setbeamercolor{palette quaternary}{bg=mainColor,fg=white} %bg = Top bar, fg = active top bar topic
\setbeamercolor{structure}{fg=black} % itemize, enumerate, etc (bullet points)
\setbeamercolor{section in toc}{fg=black} % TOC sections
\setbeamertemplate{section in toc}[sections numbered]
\setbeamertemplate{subsection in toc}{%
	\hspace{1.2em}{$\bullet$}~\inserttocsubsection\par}

\setbeamertemplate{itemize items}[circle]
\setbeamertemplate{description item}[circle]
\setbeamertemplate{title page}[default][colsep=-4bp,rounded=true]
\beamertemplatenavigationsymbolsempty

\setbeamercolor{footline}{fg=gray}
\setbeamertemplate{footline}{%
	\hfill\usebeamertemplate***{navigation symbols}
	\hspace{0.5cm}
	\insertframenumber{}\hspace{0.2cm}\vspace{0.2cm}
}

\usepackage{caption}
\captionsetup{labelformat=empty}

%Title Page
\title{Der Satz von Poincaré-Bendixson}
\author{Raphael Unterer}
\institute{Mathematisches Seminar 2025: Felder}

\newcommand*{\HL}{\textcolor{mainColor}}
\newcommand*{\RD}{\textcolor{red}}
\newcommand*{\BL}{\textcolor{blue}}
\newcommand*{\GN}{\textcolor{dgreen}}
\newcommand*{\YE}{\textcolor{violet}}




\makeatletter
\newcount\my@repeat@count
\newcommand{\myrepeat}[2]{%
	\begingroup
	\my@repeat@count=\z@
	\@whilenum\my@repeat@count<#1\do{#2\advance\my@repeat@count\@ne}%
	\endgroup
}
\makeatother




\usetikzlibrary{automata,arrows,positioning,calc}


\begin{document}

	%Titelseite
	\begin{frame}
		\titlepage
	\end{frame}

	\section{Motivation}

    \begin{frame}
        \frametitle{Recharge Oscillator}
        \begin{align*}
            \dot{x} &= -x + \gamma \left(bx + y\right) - \epsilon \left(bx + y\right)^3, \\
            \dot{y} &= -ry - \alpha b x.
        \end{align*}
    \end{frame}
    \begin{frame}
        \frametitle{Summe aller Natürlichen Zahlen}
        % \begin{center}
        %     \includegraphics[width=0.7\textwidth]{../images/youtube_screenshot.png}
        % \end{center}
    \end{frame}
    \begin{frame}
        \frametitle{Riemannsche Zeta Funktion}
        \begin{equation*}
            \zeta(s)
            =
            \sum_{n=1}^{\infty}
            \frac{1}{n^s}
        \end{equation*}
        \pause
        \begin{equation*}
            \zeta(-1)
            =
            \sum_{n=1}^{\infty}
            \frac{1}{n^{-1}}
            =
            \sum_{n=1}^{\infty} n
        \end{equation*}
    \end{frame}
    \begin{frame}
        \frametitle{Originaler Definitionsbereich}
        Wir kennen die divergierende harmonische Reihe
        \begin{equation*}
            \zeta(1)
            =
            \sum_{n=1}^{\infty}
            \frac{1}{n}
            \rightarrow
            \infty,
        \end{equation*}
        und somit ist $\Re(s) > 1$.
    \end{frame}

    \section{Analytische Fortsetzung}
    \begin{frame}
        \frametitle{Plan für die Analytische Fortsetzung von $\zeta(s)$}
        % \begin{center}
        %     \input{../images/continuation_overview.tikz.tex}
        % \end{center}
    \end{frame}
    \begin{frame}
        \frametitle{Fortsetzung auf $\Re(s) > 0$}
        Dirichletsche Etafunktion ist
        \begin{equation*}\label{zeta:equation:eta}
            \eta(s)
            =
            \sum_{n=1}^{\infty}
            \frac{(-1)^{n-1}}{n^s},
        \end{equation*}
        und konvergiert im Bereich $\Re(s) > 0$.
    \end{frame}
    \begin{frame}
        \frametitle{Fortsetzung auf $\Re(s) > 0$}
        \begin{align}
            \zeta(s)
            &=
            \RD{
            \sum_{n=1}^{\infty}
            \frac{1}{n^s} \label{zeta:align1}
            }
            \\
            \frac{1}{2^{s-1}}
            \zeta(s)
            &=
            \BL{
            \sum_{n=1}^{\infty}
            \frac{2}{(2n)^s} \label{zeta:align2}
            }
        \end{align}
        \pause
        \eqref{zeta:align1} - \eqref{zeta:align2}:
        \begin{align*}
            \left(1 - \frac{1}{2^{s-1}} \right)
            \zeta(s)
            &=
            \RD{\frac{1}{1^s}}
            \underbrace{-\BL{\frac{2}{2^s}} + \RD{\frac{1}{2^s}}}_{-\frac{1}{2^s}}
            + \RD{\frac{1}{3^s}}
            \underbrace{-\BL{\frac{2}{4^s}} + \RD{\frac{1}{4^s}}}_{-\frac{1}{4^s}}
            \ldots
            \\
            &= \eta(s)
        \end{align*}
    \end{frame}
    \begin{frame}
        \frametitle{Fortsetzung auf $\Re(s) > 0$}
        Somit haben wir die Fortsetzung gefunden als
        \begin{equation} \label{zeta:equation:fortsetzung1}
            \zeta(s)
            :=
            \left(1 - \frac{1}{2^{s-1}} \right)^{-1} \eta(s).
        \end{equation}
    \end{frame}
    \begin{frame}
        \frametitle{Spiegelungseigenschaft für $\Re(s) < 0$}
        \begin{equation*}\label{zeta:equation:functional}
            \frac{\Gamma \left( \frac{s}{2} \right)}{\pi^{\frac{s}{2}}}
            \zeta(s)
            =
            \frac{\Gamma \left( \frac{1-s}{2} \right)}{\pi^{\frac{1-s}{2}}}
            \zeta(1-s).
        \end{equation*}
    \end{frame}
    %TODO maybe explain gamma-fct

    \section{Euler Produkt und Primzahlen}
    \begin{frame}
        \frametitle{Wieso ist die Zeta Funktion so bekannt?}
        \begin{itemize}
            \item Interessante Funktionswerte z.B. $\zeta(2) = \frac{\pi^2}{6}$
            \item Primzahlenverteilung (Riemannhypothese)
            \item Forschungsgebiet der analytischen Zahlentheorie seit dem 18. Jahrhundert
            \item ...
        \end{itemize}
    \end{frame}
    \begin{frame}
        \frametitle{Euler Produkt: Verbindung von Zeta und Primzahlen}
        \begin{equation*}
            \zeta(s)
            =
            \sum_{n=1}^\infty
            \frac{1}{n^s}
            =
            \prod_{p \in P}
            \frac{1}{1-p^{-s}}
        \end{equation*}
        \pause
        Geometrische Reihe
        \begin{equation*}
            \prod_{p \in P}
            \frac{1}{1-p^{-s}}
            =
            \prod_{p \in P}
            \left(
            1
            +
            \frac{1}{p^s}
            +
            \frac{1}{p^{2s}}
            +
            \frac{1}{p^{3s}}
            +
            \ldots
            \right)
        \end{equation*}
        \pause
        Erste Terme ausmultiplizieren
        \begin{align*}
            \left(
            1
            +
            \RD{\frac{1}{2^s}}
            +
            \GN{\frac{1}{2^{2s}}}
            +
            \frac{1}{2^{3s}}
            +
            \ldots
            \right)
            \left(
            1
            +
            \BL{\frac{1}{3^s}}
            +
            \frac{1}{3^{2s}}
            +
            \frac{1}{3^{3s}}
            +
            \ldots
            \right)
            \left(
            1
            +
            \YE{\frac{1}{5^s}}
            +
            \frac{1}{5^{2s}}
            +
            \frac{1}{5^{3s}}
            +
            \ldots
            \right)
            \ldots
            \\
            =
            1
            +
            \RD{\frac{1}{2^s}}
            +
            \BL{\frac{1}{3^s}}
            +
            \GN{\frac{1}{4^s}}
            +
            \YE{\frac{1}{5^s}}
            +
            \ldots
        \end{align*}
    \end{frame}
    \begin{frame}
        \frametitle{Primzahlfunktion}
        % \begin{center}
        %    \scalebox{0.5}{\input{../images/primzahlfunktion.pgf}}
        % \end{center}
    \end{frame}


    \section{Darstellungen}

    \begin{frame}
        \frametitle{Farbcodierung}
        % \begin{center}
        %     \scalebox{0.6}{\input{zeta_color_plot.pgf}}
        % \end{center}
    \end{frame}

    \begin{frame}
        \frametitle{Konstanter Realteil $\Re(s)=-1$ und $\Im(s)=0\ldots40$}
        % \begin{center}
        %     \scalebox{0.6}{\input{../images/zeta_re_-1_plot.pgf}}
        % \end{center}
    \end{frame}
    \begin{frame}
        \frametitle{Konstanter Realteil $\Re(s)=0$ und $\Im(s)=0\ldots40$}
        % \begin{center}
        %     \scalebox{0.6}{\input{../images/zeta_re_0_plot.pgf}}
        % \end{center}
    \end{frame}
    \begin{frame}
        \frametitle{Konstanter Realteil $\Re(s)=0.5$ und $\Im(s)=0\ldots40$}
        % \begin{center}
        %     \scalebox{0.6}{\input{../images/zeta_re_0.5_plot.pgf}}
        % \end{center}
    \end{frame}

\end{document}

