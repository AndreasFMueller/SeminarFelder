\newcommand{\eot}{\mathbf{e}_{12}}
% \newcommand{\eott}{\mathbf{e}_{123}}
\section{Berechnungen mit der geometrischen Algebra}
Anstelle immer $\eone \wedge \etwo$ zu schreiben, wird in der geometrischen Algebra oft auch
die Schreibweise $\eot$ verwendet.
Wie wir bereits wissen, nennen wir $\eot$ einen \emph{Bivektor}. In der geometrischen Algebra
spricht man auch von \emph{Graden}. Ein Bivektor ist von Grad $2$, ein Vektor ist von Grad $1$, ein Skalar von
Grad $0$, ein Trivektor von Grad $3$ und so weiter.

Desweiteren wird von \emph{Multivektoren} gesprochen, wenn eine Summe von verschiedenen Graden vorliegt, zum Beispiel
\begin{align}
A &= 2 + 3 \eone + \eot \\
B &= 1 + \eone + 2 \eot.
\end{align}

Das Tolle ist, dass mit der geometrischen Algebra ähnlich zu den komplexen Zahlen eine Algebra definiert
ist, mit welcher normal gerechnet werden kann. So können wir zum Beispiel Multivektoren addieren
\begin{equation}
\begin{aligned}
A + B &= (2 + 3 \eone + \eot) + (1 + \eone + 2 \eot) \\
&= 3 + 4 \eone + 3 \eot,
\end{aligned}
\end{equation}

oder sogar multiplizieren

\begin{equation}
\begin{aligned}
A  B &= (2 + 3 \eone + \eot) (1 + \eone + 2 \eot) \\
&= 2 + 2 \eone + 4 \eot + 3 \eone + 3 \eone^2 + 6 \eone \eot + \eot + \eot \eone + 2 \eot^2 \\
&= 2 + 2 \eone + 4 \eot + 3 \eone + 3 + 6 \etwo + \eot - \etwo - 2 \\
&= 3 + 5 \eone + 5 \etwo + 5 \eot \\
\end{aligned}
\end{equation}
