%
% integrand.tex -- Integrand der 1-Form für den Weg \gamma_s
%
% (c) 2021 Prof Dr Andreas Müller, OST Ostschweizer Fachhochschule
%
\documentclass[tikz]{standalone}
\usepackage{amsmath}
\usepackage{times}
\usepackage{txfonts}
\usepackage{pgfplots}
\usepackage{csvsimple}
\definecolor{darkred}{rgb}{0.8,0,0}
\usetikzlibrary{arrows,intersections,math,calc}
\begin{document}
\def\skala{1}
\def\s{0.2}
\begin{tikzpicture}[>=latex,thick,scale=\skala,
declare function={
	z(\t,\s) = \s*(1-\t)*(1+\t);
	n(\t,\s) = sqrt(z(\t,\s)*z(\t,\s)+1);
	f(\t,\s) = \s(\t*\t+1)/n(\t,\s);
	s(\t) = \t / sqrt(4-\t*\t);
}]
\begin{scope}
%\clip (-6,-0.1) rectangle (6,4.1);
\foreach \y in {1,0.8,0.6,0.4,0.2}{
	\pgfmathparse{s(\y)}
	\xdef\s{\pgfmathresult}
	\draw[color=darkred,line width=1.4pt]
		plot[domain=-1:1,samples=100] ({6*\x},{8*f(\x,\s)});
}
\end{scope}

\draw[->] (-6.1,0) -- (6.3,0) coordinate[label={$t$}];
\draw[->] (0,-0.05) -- (0,4.3) coordinate[label={right:$f(t,s)$}];

\draw (-6,-0.05) -- ++(0,0.1);
\node at (-6,0) [below] {$-1$};
\draw (6,-0.05) -- ++(0,0.1);
\node at (6,0) [below] {$1$};

\foreach \t in {0.1,0.2,...,0.9}{
	\draw ({6*\t},-0.05) -- ++(0,0.1);
	\draw ({-6*\t},-0.05) -- ++(0,0.1);
}
\foreach \t in {0.1,0.2,...,0.7}{
	\draw (-0.05,{4*sqrt(2)*\t}) -- ++(0.1,0);
}
\node at (0,{4*sqrt(2)*0.1}) [left] {$0.1$};
\node at (0,{4*sqrt(2)*0.2}) [left] {$0.2$};
\node at (0,{4*sqrt(2)*0.5}) [left] {$0.5$};

\node at ({6*0.5},0) [below] {$0.5$};
\node at ({-6*0.5},0) [below] {$-0.5$};
\node at (0,0) [below] {$0$};

\node[color=darkred] at ({6*0.5},{4*sqrt(2)*f(0.5,1)}) [above right] {$s=1$};
\node[color=darkred] at ({6*0.5},{8*f(0.5,\s)}) [below left] {$s=\s$};

\end{tikzpicture}
\end{document}

