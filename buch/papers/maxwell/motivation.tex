\section{Motivation für eine vierdimensionale Darstellung der Maxwell-Gleichungen}
\label{maxwell:motivation}

Das elektromagnetische Feld besteht klassisch aus zwei scheinbar getrennten Vektorfeldern: der elektrischen Feldstärke \( \vec{E} \) und der magnetischen Flussdichte \( \vec{B} \). Aus Sicht der speziellen Relativitätstheorie zeigt sich jedoch, dass diese Trennung nicht fundamental ist. Die folgenden Gedankenexperimente veranschaulichen, warum eine gemeinsame, vierdimensionale Beschreibung sinnvoll ist.

\subsection{Gedankenexperiment: Elektromagnetisches Feld}

Betrachten wir eine elektrische Punktladung, die sich gleichförmig bewegt.
Ein ruhender Beobachter erkennt in ihrer Umgebung sowohl ein elektrisches Feld \( \vec{E} \) als auch ein magnetisches Feld \( \vec{B} \).
Bewegt sich jedoch ein Beobachter gemeinsam mit der Ladung, so erscheint diese ruhend und es tritt nur noch ein elektrisches Feld auf. 

Die Tatsache, dass dieselbe physikalische Situation vom Beobachter abhängt, legt nahe, dass \( \vec{E} \) und \( \vec{B} \) keine voneinander unabhängigen Größen sind.
Vielmehr handelt es sich um zwei Erscheinungsformen des gleichen Feldes.
Ihre Trennung ist abhängig vom Bezugssystem.
Dies spricht dafür, beide Felder gemeinsam durch ein einziges mathematisches Objekt zu beschreiben.

\subsection{Gedankenexperiment: Ladungs- und Stromdichte}

Ein ähnlicher Effekt zeigt sich bei der Ladungsdichte \( \rho \) und der Stromdichte \( \vec{J} \).
Bewegen wir uns mit einer Ladung mit, erscheint sie aus unserer Sicht ruhend.
Wir beobachten eine reine Ladungsverteilung, aber keinen Strom.
Für einen ruhenden Beobachter hingegen, der die Ladung vorbeiziehen sieht, ergibt sich zusätzlich eine Stromdichte.
Auch hier sind die beobachteten Größen also vom Bewegungszustand abhängig.

Diese Abhängigkeit legt nahe, dass \( \rho \) und \( \vec{J} \) keine unabhängig voneinander bestehenden Objekte sind, sondern zusammengehören und gemeinsam beschrieben werden.

\subsection{Elektrisches und magnetisches Potential}
\label{maxwell:gedankenexperiment:potentiale}

Auch das elektrisches Potential \( \varphi \) und das magnetische Vektorpotential \( \vec{A} \) sind beobachterabhängig.
In bestimmten Bezugssystemen kann ein elektrisches Feld allein durch \( \varphi \) beschrieben werden, während in anderen das magnetische Potential \( \vec{A} \) zusätzlich erforderlich ist.

Dass die Beschreibung des Feldes also entweder über \( \varphi \), über \( \vec{A} \) oder über beide gleichzeitig erfolgt, abhängig vom Bezugssystem, deutet darauf hin, dass auch diese beiden Potentiale nicht fundamental getrennt sind.
Es liegt daher nahe, sie zu einem einzigen Objekt zusammenzufassen, das unabhängig vom Beobachter die gleiche physikalische Information enthält.

In den folgenden Abschnitten entwickeln wir eine Theorie, welche diese Problematiken beheben wird.




%\section{Motivation für eine vierdimensionale Darstellung der Maxwell-Gleichungen}
%Das elektromagnetische Feld besteht aus zwei scheinbar getrennten Vektorfeldern, der elektrischen Feldstärke $E$ und der magnetischen Flussdichte $B$.
%
%\subsection{Gedankenexperiment Elektromagnetisches Feld}
%Stellen wir uns vor, wir sind ein ruhender Beobachter und betrachten eine bewegte elektrische Punktladung.
%Aufgrund der Bewegung ist für uns ein magnetisches Feld $B$ sichtbar und auch eine elektrische Feldstärke $E$.
%
%Wenn wir aber ein bewegter Beobachter sind, der sich mit der Punktladung mitbewegt, erscheint diese ruhend und wir können nur die elektrische Feldstärke beobachten. Dies bedeutet, dass das Auftreten der beiden Felder vom Beobachter abhängig ist.
%
%Diese Beobachterabhängigkeit legt nahe, dass $E$ und $B$ nicht fundamental getrennt sein können, sondern verschiedene Aspekte eines physikalischen Feldes beschreiben, welches wir durch ein einziges mathematisches Objekt beschreiben wollen.
%
%\subsection{Gedankenexperiment Ladungs- und Stromdichte}
%Ein ähnliches Phänomen lässt sich auch bei der Ladungs- und Stromdichte beobachten.
%Stellen wir uns wiederum vor, wir sind ein bewegter Beobachter und bewegen uns mit der Ladung mit, die Ladung wirkt wiederum ruhend.
%Dabei beobachten wir keine weiteren Phänomene, wenn wir das elektromagnetische Feld vernachlässigen.
%
%Wenn wir uns aber nicht mitbewegen, stellen wir dabei fest, dass wir zusätzlich zur Ladungsdichte auch noch eine Stromdichte feststellen können.
%
%Durch diese Beobachtung liegt auch hier nahe, dass Ladungs- und Stromdichte auch als ein einziges Objekt betrachtet werden müssen.
%
%\subsection{Elektrisches Potential und magnetisches Vektorpotential}
%Aufgrund der Gedankenexperimente liegt nahe, dass es auch eine Vereinheitlichung für das elektrische Potential und das magnetische Vektorpotential geben muss. 
