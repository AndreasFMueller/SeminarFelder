\section{Spiegelungen}
\renewcommand{\v}{\hat{\mathbf{v}}}
\subsection{Spiegelung in der traditionellen Vektorgeometrie}
Wir betrachten den Vektor
\begin{equation}
\a = a_1 \eone + a_2 \etwo.
\end{equation}
Wenn man \autoref{geoalgebra:fig:reflection} betrachtet, fällt schnell auf, dass bei dem an $\etwo$ gespiegelte Vektor
$\a'$ genau eine Komponente negativ wird:
\begin{equation}
\a' = -a_1 \eone + a_2 \etwo.
\end{equation}

\subsection{Projektion eines Vektors}
Sei ein beliebiger Vektor $\a$ und ein beliebiger Einheitsvektor $\v$ (\autoref{geoalgebra:fig:reflection}).
Der auf $\v$ projezierte Vektor $\a_{||}$ erhält man durch
\begin{equation}
  \a_{||} = (\v \cdot \a) \v.
\end{equation}
Der Senkrechte $\a_{\perp}$ erhält man durch
\begin{equation}
  \a_{\perp} = (\v \wedge \a) \v
  \label{geoalgebra:eq:perpendicular}
\end{equation}
\begin{proof}[Beweis von \eqref{geoalgebra:eq:perpendicular}]
  $\a$ lässt sich ausdrücken als
  \begin{equation}
    \a = a_1 \eone + a_2 \etwo.
  \end{equation}
  Ausserdem ist aus \eqref{geoalgebra:eq:geo-product-wedge} bekannt, dass
  \begin{equation}
    \a \wedge \v = \frac{1}{2} (\a \v - \v \a)
  \end{equation}
  Zur Einfachheit sei $\v = \etwo$.
  Somit ist
  \begin{equation}
    \begin{aligned}
      \a_{\perp} &= \etwo (\a \wedge \etwo) = \etwo \frac{1}{2} (\a \etwo - \etwo \a) \\
                 &= \frac{1}{2} \etwo ((a_1 \eone + a_2 \etwo) \etwo - \etwo (a_1 \eone + a_2 \etwo)) \\
                 &= \frac{1}{2} \etwo (a_1 \eone \etwo + a_2 \etwo \etwo - a_1 \etwo \eone - a_2 \etwo \etwo) \\
                 &= \frac{1}{2} \etwo (a_1 \eot + a_2 + a_1 \eot - a_2) \\
                 &= \frac{1}{2} \etwo 2 a_1 \eot = a_1 \mathbf{e}_{212} = -a_1 \eone.
  \end{aligned}
  \end{equation}
\end{proof}
\begin{figure}
  \begin{center}
  \begin{tikzpicture}[>=latex, scale=3]
  \draw [->](0, 0) -- (0, 1.2) node [right]{$\v$};
\draw [darkred, ->](0, 0) -- (0, 1) node [right]{$\a_{||}$};
\draw [blue, ->](0, 0) -- (1, 1) node [right]{$\a$};
\draw [blue!30, ->](0, 0) -- (-1, 1) node [left]{$\a'$};
  \draw [darkred, ->](0, 0) -- (-1, 0) node [above]{$\a_{\perp}$};
\end{tikzpicture}

  \end{center}
\caption{Spiegelung}
\label{geoalgebra:fig:reflection}
\end{figure}
