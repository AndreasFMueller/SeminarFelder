%
% Schnittkrümmung
%
\section{Schnittkrümmung
\label{buch:kruemmung:section:schnittkruemmung}}
\kopfrechts{Schnittkrümmung}
In Beispiel~\ref{buch:kruemmung:kruemmung:bsp:kugel} wurde der 
riemannsche Krümmungstensor einer Kugeloberfläche berechnet.
Es wurde gefunden, dass $R_{1212}=\sin^2\vartheta$ immer noch von der 
Position auf der Kugeloberfläche abhängt.
Wir erwarten aber auch eine Abhängigkeit der Krümmung von der umlaufenen
Fläche.
Der Flächeninhalt eines infinitesimalen Rechtecks mit Seitenlänge
$d\vartheta$ und $d\varphi$ ist
\[
F
=
\sin\vartheta\, d\vartheta\,d\varphi.
\]
Die kovariante riemannsche Tensorkomponenten $R_{1212}$ hängt zweimal
(einmal für jedes Indexpaar $12$) von diesem Flächeninhalt ab.
Wir erwarten daher, dass
\[
K
=
\frac{R_{1212}}{\sin^2\vartheta}
=
1
\]
eine geometrische Invariante ist, die nicht von der Wahl des
Koordinatensystems abhängt.
Dies ist ein Beispiel der Schnittkrümmung, die in diesem Abschnitt
eingeführt werden soll.
Im Gegensatz zu den Komponenten des riemannschen Krümmungstensors ist
die Schnittkrümmung eine geometrische Invariante.
Einerseits lässt sie sich direkt aus dem riemannschen Krümmungstensor
ablesen, der riemannsche Krümmungstensor lässt sich aber auch aus
den Schnittkrümmungen wiedergewinnen.

%
% Geodätische Untermannigfaltigkeiten
%
\subsection{Geodätische Untermannigfaltigkeiten
\label{buch:kruemmung:schnittkrüemmung:subsection:geodum}}

%
% Definitin der Schnittkrümmung
%
\subsection{Definition der Schnittkrümmung}
Der riemannsche Krümmungstensor liefert zu einem Bivektor
$\vec{h}\wedge\vec{k}$ eine lineare Abbildung, die die Drehung
eines paralleltransportierten Vektors beschreibt.
Die Abhängigkeit $\vec{h}\wedge\vec{k}\mapsto R(\vec{h}\wedge\vec{k})$
ist bilinear in $\vec{h}$ und $\vec{k}$.
Die Bivektoren aufgespannt von Tangentialvektoren in der
$\vec{h}$-$\vec{k}$-Ebene bilden den eindimensionalen Vektorraum
$\mathbb{R}\vec{h}\wedge\vec{k}$ mit der Basis $\vec{h}\wedge\vec{k}$.

Aus der Theorie der geodätischen Untermannigfaltigkeiten von
Abschnitt~\ref{buch:kruemmung:schnittkrüemmung:subsection:geodum}
ergibt sich, dass Tangentialvektoren in der $\vec{h}$-$\vec{k}$-Ebene
in dieser Ebene bleiben.
Der Wert $R(\vec{h}\wedge\vec{k})$ beschreibt eine infinitesimale Drehung
in der $\vec{h}\wedge\vec{k}$-Ebene.
In einer orthonormierten Basis wird sie beschrieben durch eine
antisymmetrische Matrix.
Die Antisymmetrie bedeutet, dass der Wert $R(\vec{h}\wedge\vec{k})$ ist
also wieder ein Vielfaches von $\vec{h}\wedge\vec{k}$, es gibt
eine Zahl $s(\vec{h}\wedge\vec{h})$ mit der Eigenschaft
\[
R(\vec{h}\wedge\vec{k})
=
s(\vec{h}\wedge\vec{k})
\vec{h}\wedge\vec{k}.
\]
Da $R(\vec{h}\wedge\vec{k})$ linear von $\vec{h}\wedge\vec{k}$ abhängt,
muss auch $s(\vec{h}\wedge\vec{k})$ linear von $\vec{h}\wedge\vec{k}$
abhängen.

Der Bivektor $\vec{h}\wedge\vec{k}$ repräsentiert den Flächeninhalt
des von $\vec{h}$ und $\vec{k}$ aufgespannten Parallelogramms. 
In einer orthonormierten Basis wird es durch die Determinante wiedergegeben,
die mit $|\vec{h}\wedge\vec{k}|$ geschrieben werden kann.
es muss also ein Zahl $S$ geben derart, dass
\[
R(\vec{h}\wedge\vec{k})
=
s(\vec{h}\wedge\vec{k})
\begin{pmatrix*}[r] 0&-1\\ 1&0\end{pmatrix*}
|\vec{h}\wedge\vec{k}|
=
S |\vec{h}\wedge\vec{k}|^2
\begin{pmatrix*}[r] 0&-1\\1&0\end{pmatrix*}.
\]
Der Faktor $S$ ist unabängig von der Basis in der $\vec{h}$-$\vec{k}$-Ebene
und damit eine innere Eigenschaft der Mannigfaltigkeit.

\begin{definition}[Schnittkrümmung]
\label{buch:kruemmung:schnittkruemmung:def:schnittkruemmung}
Sie $M$ eine riemannsche Mannigfaltigkeit mit dem kovarianten riemannschen
Krümmungstensor $Rm(X,Y,Z,W)$.
Die Grösse
\[
S(X\wedge Y) = \frac{Rm(X,Y,X,Y)}{|X\wedge Y|^2}.
\]
heisst die \emph{Schnittkrümmung} in der von $X$ und $Y$ aufgespannten
geodätischen Untermannigfaltigkeit von $M$.
\index{Schnittkrümmung}%
\end{definition}


%
% Schnittkrümmungen bestimmen den riemannschen Krümmungstensor
%
\subsection{Schnittkrümmungen bestimmen den riemannschen Krümmungstensor}

\begin{satz}
Der riemannsche Krümmungstensor einer Mannigfaltigkeit $M$
ist vollständig bestimmt durch die Schnittkrümmungen.
\end{satz}

Der Beweis wird zeigen, dass sich sogar eine konkrete Formel
für die Werte des Krümmungstensors angeben lässt.

\begin{proof}
Die Komponenten $R_{lhik}$ des riemannsche Krümmungstensors ist
antisymmetrisch im ersten und letzten Indexpaar.
Es gilt also
\[
R_{lhik}
-R_{hlik}
=
-R_{lhki}.
\]
Dies bedeutet, dass die 4-Linearform
\[
Rm(X,Y,Z,W)
=
\xi^l\eta^h\zeta^i\omega^k R_{lhik}
\]
nur von den 2-Vektoren $X\wedge Y$ bzw.~$Z\wedge W$ abhängt.
Der Krümmungstensor kann daher auch als blineare Funktion
\begin{equation}
\bigwedge^2 TM\otimes\bigwedge^2 TM
\to
\mathbb{R}
:
(X\wedge Y,Z\wedge W)
\mapsto
R(X\wedge Y, Z\wedge W)
=
Rm(X,Y,Z,W)
\label{buch:kruemmung:schnittkruemmung:eqn:Rbivektor}
\end{equation}
auf dem Vektorraum $\bigwedge^2 TM$ der 2-Vektoren von Tangentialvektoren
geschrieben werden.

Der Krümmungstensor ist ausserdem symmetrisch unter der Vertauschung
der Indexpaare $lh$ und $ik$.
Dies bedeutet, dass die
in
\eqref{buch:kruemmung:schnittkruemmung:eqn:Rbivektor}
definierte Funktion $R$ auf den 2-Vektoren symmetrisch ist:
\begin{align*}
R(X\wedge Y,Z\wedge W)
=
R(Z\wedge W,X\wedge Y)
\end{align*}
für beliebige 2-Vektoren $X\wedge Y,Z\wedge W\in\bigwedge^2TM$.
Wir schreiben die symmetrische, bilineare Funktion auch als $R(u,v)$ mit
$u,v\in\bigwedge^2 TM$.

Wir möchten zeigen, dass sich $R(u,v)$ aus den Werten $R(w,w)$ für beliebige
2-Vektoren $w$ berechnen lässt.
Um dies etwas prägnanter auszudrücken, schreiben wir
\[
Q(w) = R(w,w).
\]
Die Funktion $Q$ ist eine quadratische Form.

Die Polarisationsformel \cite[p. 347]{buch:linalg} besagt, dass eine
symmetrische, bilineare Funktion $R(u,v)$ vollständig durch die
Werte $R(u,v)$ bestimmt ist.
Wir führen die Rechnung 
\begin{align}
R(u+v,u+v)
&=
R(u,u) + R(u,v) + R(v,u) + R(v,v)
\notag
\\
\Rightarrow\qquad
Q(u+v)
&=
R(u,u) + 2R(u,v) + R(v,v)
\notag
\\
R(u-v,u-v)
&=
R(u,u) - R(u,v) - R(v,u) + R(v,v)
\notag
\\
\Rightarrow\qquad
Q(u-v)
&=
R(u,u) - 2R(u,v) + R(v,v)
\notag
\intertext{mit der Differenz}
Q(u+v)-Q(u-v)
&=
4 R(u,v),
\notag
\intertext{oder aufgelöst nach dem gemischten Term:}
R(u,v)
&=
{\textstyle\frac14}\bigl( Q(u+v) - Q(u-v) \bigr).
\label{buch:kruemmung:schnittkruemmung:eqn:RQQ}
\end{align}
Die Formel \eqref{buch:kruemmung:schnittkruemmung:eqn:RQQ}
drückt $R(u,v)$ durch Werte der Form $R(w,w)$ mit $w\in\bigwedge^2 TM$ aus.
\end{proof}

\begin{satz}
Der Krümmungstensor ist durch die Schnittkrümmungen vollständig
bestimmt.
\end{satz}

\begin{proof}
Die Schnittkrümmung zu den Vektoren $\vec{s}$ und $\vec{v}$ ist
definiert durch
\begin{align}
K(\vec{s},\vec{v})
&=
\frac{
R(\vec{v},\vec{s},\vec{v},\vec{s})
}{
|\vec{v}\wedge\vec{s}|^2
}.
\intertext{Für einen Basis-2-Vektor $\vec{e}_i\wedge\vec{e}_k$ wird dies zu}
K(\vec{e}_i,\vec{e}_k)
&=
\frac{Rm(\vec{e}_i,\vec{e}_k,\vec{e}_i,\vec{e}_k)}{|\vec{e}_i\wedge\vec{e}_k|^2}
\notag
\\
&=
\frac{R_{ikik}}{g^{ii}g^{kk}-g^{ik}g^{ik}}.
\label{buch:kruemmung:schnittkruemmung:eqn:Rgggg}
\end{align}
Da $g$ als Metrik definit ist, ist der Nenner von
\eqref{buch:kruemmung:schnittkruemmung:eqn:Rgggg}
niemals 0.
Der Riemannsche Krümmungstensor kann also aus den Werten
\[
Rm(\vec{e}_i,\vec{e}_k,\vec{e}_i,\vec{e}_k)
=
R(\vec{e}_i\wedge\vec{e}_k,\vec{e}_i\wedge\vec{e}_k)
=
Q(\vec{e}_i\wedge\vec{e}_k)
=
(g^{ii}g^{kk}-g^{ik}g^{ik}) K(\vec{e}_i,\vec{e}_k)
\]
bestimmt werden.
Sie können also aus den Schnittkrümmungen ermittelt werden.
\end{proof}

Der Satz besagt also, dass die vom riemannschen Krümmungstensor
beschriebene Krümmung des Raumes vollständig ergründet werden kann,
indem man Messungen in divergierenden Gedäten durchführt.
Eine andere Art, dies auszudrücken, ist, dass der riemannsche
Krümmungstensor keine zusätzliche Information erhält, die man nicht
in der Funktion der Schnittkrümmungen gefunden werden kann.
