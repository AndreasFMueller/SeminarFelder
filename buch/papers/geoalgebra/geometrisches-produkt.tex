\section{Geometrisches Produkt}
Im Rahmen der geometrischen Algebra ist das \emph{geometrische Produkt} eine sehr
nützliche Operation. Es kombiniert das \emph{innere Produkt} (Skalarprodukt) und das
\emph{äussere Produkt} (Wedgeprodukt) in einer Operation
\begin{equation}
\a \b = \a \cdot \b + \a \wedge \b
\end{equation}
und kombiniert somit einen Skalar (0-Vektor) mit einem Bivektor (2-Vektor).
Es gilt
\begin{lemma}
\begin{equation}
  \label{geoalgebra:lemma:a_squared}
  \begin{aligned}
    \a^2 &= \a \cdot \a + \a \wedge \a \\
    &= \a \cdot \a \\
    &= |\a|^2
  \end{aligned}
\end{equation}
\end{lemma}
Mithilfe des Axioms \eqref{geoalgebra:eq:antikommutativ} erkennt man, dass
\begin{align}
  \a \b &= \a \cdot \b + \a \wedge \b \\
  \b \a &= \a \cdot \b - \a \wedge \b.
\end{align}

Wir können die beiden Gleichungen addieren und erhalten
\begin{align}
  \a \b + \b \a &= 2 \a \cdot \b \\
  \a \cdot \b = \frac{1}{2} (\a \b + \b \a),
\end{align}
wenn wir sie voneinander subtrahieren, erhalten wir
\begin{align}
  \a \b - \b \a &= 2 \a \wedge \b \\
  \a \wedge \b = \frac{1}{2} (\a \b - \b \a).
\end{align}
