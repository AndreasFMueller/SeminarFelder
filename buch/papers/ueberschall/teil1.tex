%
% teil2.tex -- Beispiel-File für teil2 
%
% (c) 2020 Prof Dr Andreas Müller, Hochschule Rapperswil
%
% !TEX root = ../../buch.tex
% !TEX encoding = UTF-8
%
\section{Potentialströmung\label{ueberschall:section:potentialstroemung}}
\kopfrechts{Potentialströmung}
Bei der Potentialströmung handelt es sich um eine 
quellenfreie und wirbelfreie Strömung.
Mathematisch wird dies durch die folgenden Bedingungen 
an das Geschwindigkeitsfeld~$\vec{v}$ ausgedrückt:
\begin{align*}
    \mathrm{div}\,\vec{v} &= 0, \\
    \mathrm{rot}\,\vec{v} &= 0,
\end{align*}
wobei~$\vec{v}$ die Geschwindigkeit der Strömung bezeichnet.
Diese Idealisierung stellt einen Spezialfall der
laminaren Strömung dar. 
Unter Einhaltung der genannten Bedingungen 
besitzt das Geschwindigkeitsfeld ein skalares
Potential~$\Phi$, sodass gilt:
\begin{align}
    \vec{v} 
    = 
    \mathrm{grad}\,\Phi 
    = 
    \nabla \Phi.\label{eq:potential}
\end{align}
Das bedeutet, dass~$\vec{v}$ dem Gradienten des Potentials 
entspricht, also der Richtungsableitung in jeder 
Koordinatenrichtung:
\begin{align*}
    \nabla \Phi =
    \begin{pmatrix}
        \frac{\partial \Phi}{\partial x} \\
        \frac{\partial \Phi}{\partial y} \\
        \frac{\partial \Phi}{\partial z}
    \end{pmatrix}.
\end{align*}
Da es sich bei~$\Phi$ um eine skalare Funktion handelt, 
kann der Gradient formal als gewöhnliche Multiplikation 
mit dem Nabla-Operator~$\nabla$ interpretiert werden.

Zudem betrachten wir eine stationäre Strömung.  
Natürlich muss dabei auch die Kontinuitätsgleichung
erfüllt sein:
\begin{align*}
    \frac{\partial \rho}{\partial t}
    + 
    \mathrm{div}\,\vec{v} 
    = 
    0.
\end{align*}
Für inkompressible Medien vereinfacht sich dies, 
da die Dichte zeitlich konstant bleibt:
\begin{align*}
    \frac{\partial \rho}{\partial t} 
    = 
    0,
\end{align*}
woraus folgt, dass das Geschwindigkeitsfeld 
quellen- und senkenfrei ist, also
\begin{align*}
    \mathrm{div}\,\vec{v} = 0.
\end{align*}
Die Divergenz kann auch als Skalarprodukt 
des Nabla-Operators mit dem Geschwindigkeitsfeld 
geschrieben werden:
\begin{align*}
    \mathrm{div}\,\vec{v}
    = 
    \nabla \cdot \vec{v}
    = 
    \frac{\partial v_x}{\partial x} 
    +
    \frac{\partial v_y}{\partial y} 
    +
    \frac{\partial v_z}{\partial z}.
\end{align*}
Sie ist ein Mass dafür, wie stark sich ein Feld lokal 
wie eine Quelle oder eine Senke verhält.  
Ist $\mathrm{div}\,\vec{v} = 0$, 
so ist das Feld weder Quelle noch Senke.

Das Kreuzprodukt des Nabla-Operators mit 
dem Geschwindigkeitsfeld entspricht der Rotation 
(auch Wirbel) des Feldes:
\begin{align*}
    \mathrm{rot}\,\vec{v} 
    = 
    \nabla \times \vec{v}
    =
    \begin{pmatrix}
        \frac{\partial v_z}{\partial y} - \frac{\partial v_y}{\partial z} \\
        \frac{\partial v_x}{\partial z} - \frac{\partial v_z}{\partial x} \\
        \frac{\partial v_y}{\partial x} - \frac{\partial v_x}{\partial y}
    \end{pmatrix}.
\end{align*}
Diese Grösse beschreibt, wie ein Körper, 
der sich im Strömungsfeld befindet, zu rotieren beginnt.  
Ein wirbelfreies Feld erfüllt demnach
 $\mathrm{rot}\,\vec{v} = 0$,  
woraus folgt, dass das Vektorfeld ein Gradientenfeld 
ist – wie in Gleichung~\eqref{eq:potential} gezeigt.

Wenn nun in die Kontinuitätsgleichung des inkompressiblen 
Falls, $\nabla \cdot \vec{v} = 0$, 
die Definition des Geschwindigkeitsfeldes 
aus Gleichung~\eqref{eq:potential} eingesetzt wird, 
ergibt sich:
\begin{align}
    \nabla \cdot \vec{v}
    = 
    \nabla \cdot \nabla \Phi
    = 
    \nabla^2 \Phi
    = 
    \Delta \Phi
    = 
    0, \label{eq:laplace}
\end{align}
auch bekannt als Laplace-Gleichung mit dem 
Laplace-Operator~$\Delta$.



