%
% teil3.tex -- Beispiel-File für Teil 3
%
% (c) 2020 Prof Dr Andreas Müller, Hochschule Rapperswil
%
% !TEX root = ../../buch.tex
% !TEX encoding = UTF-8
%
\section{Das Muster im Kapitel}
\kopfrechts{Das Muster im Kapitel}
In diesem Kapitel wurde den Lesern gezeigt, wie sie mithilfe mathematischer Mittel wunderschöne Muster erstellen können.
Aufmerksamen Lesern ist möglicherweise aufgefallen, dass der Ablauf in den verschiedenen Abschnitten dieses Kapitels sehr ähnlich ist.
Ausgangspunkt ist stets die Reaktionsdiffusionsgleichung.
Zunächst wird das System mit einer kleinen Störung angeregt und anschliessend mithilfe der Fourier-Theorie die Entwicklung dieser Störung untersucht.
Wenn die Störungsentwicklung gewissen Regeln folgt, können Muster entstehen.
Bei manchen Beispielen wurde die Theorie ebenfalls überprüft, indem sie mit der FFT des Musters verglichen wurde.

Auch wenn in diesem Kapitel viele Beispiele besprochen wurden, gibt es noch viele weitere spannende Reaktionen, wie beispielsweise die Belousov-Zhabotinsky-Reaktion, auf welche die beschriebene Vorgehensweise ebenfalls angewendet werden kann.
