
\section{Wedgeprodukt
\label{geoalgebra:section:wedgeprodukt}}
\kopfrechts{Wedgeprodukt}

Im Rahmen der Geometrischen Algebra möchten wir eine neue Operation
definieren, das \emph{Wedgeprodukt} $\wedge$.
Es ist die grundlegende Operation der geometrischen
Algebra.
Angenommen, wir haben die
zwei Vektoren $\mathbf{v_1}, \mathbf{v_2}$ und bilden das Wedgeprodukt
\begin{equation}
  B = \mathbf{v_1} \wedge \mathbf{v_2}
\end{equation}
erhalten wir ein neues mathematisches Konzept, den \emph{Bivektor} $B$.
Ein Bivektor unterscheidet sich elementar von einem Vektor (siehe \ref{geoalgebra:section:bivektoren}).

\subsection{Definition in zwei Dimensionen}
Um das Wedgeprodukt definieren zu können, benötigen wir einige wenige Axiome, d.h. grundlegende Annahmen, die wir treffen,
um uns eine Algebra darauf aufzubauen. Somit definieren wir

\begin{satz}
  Das Wedgeprodukt zwischen einem Vektor und sich selbst ist 0.

  $
  \mathbf{v_i} \wedge \mathbf{v_i} = 0
  $
\end{satz}
und
\begin{satz}
Das Wedgeprodukt ist antikommutativ.
  $
  \mathbf{v_i} \wedge \mathbf{v_j} = -\mathbf{v_j} \wedge \mathbf{v_i}
  $
  \label{geoalgebra:satz:antikommutativ}
\end{satz}
und
\begin{satz}
Das Distributivgesetz gilt.

  $
  \mathbf{v_i} \wedge (\mathbf{v_j} + \mathbf{v_k}) = \mathbf{v_i} \wedge \mathbf{v_j} + \mathbf{v_i} \wedge \mathbf{v_k}
  $
\end{satz}

Mit diesen Axiomen können wir bereits das Wedgeprodukt zwischen zwei
Vektoren in der zweiten Dimension herleiten.
Dazu machen wir uns zu nutze, dass wir jeden Vektor als
Linearkombination von den Einheitsvektoren beschreiben können.
Es gilt
\begin{align}
  \mathbf{v} &= \begin{pmatrix} v_1 \\ v_2 \end{pmatrix} \\
    &= v_1 \mathbf{e_1} + v_2 \mathbf{e_2}
\end{align}

\begin{definition}
  Wir betrachten die beiden (zweidimensionalen) Vektoren $\mathbf{u},
  \mathbf{v}$.
  Das Wedgeprodukt dazwischen ist

  \begin{equation}
    \begin{aligned}
    \mathbf{u} \wedge \mathbf{v} &= (u_1 e_1 + u_2 e_2) \wedge
    (v_1 e_1 + v_2 e_2) \\
    &= u_1 e_1 \wedge (v_1 e_1 + v_2 e_2) + u_2 e_2 \wedge (v_1 e_1 + v_2 e_2) \\
    &= u_1 e_1 \wedge v_1 e_1 + u_1 e_1 \wedge v_2 e_2 + u_2 e_2 \wedge v_1 e_1 + u_2 e_2 \wedge v_2 e_2 \\
    &= u_1 v_1 (\underbrace{e_1 \wedge e_1}_{0}) + u_1 v_2 (e_1 \wedge e_2) + u_2 v_1 (e_2 \wedge e_1) + u_2 v_2 (\underbrace{e_2 \wedge e_2}_{0}) \\
    &= (u_1 v_2 - u_2 v_1) e_1 \wedge e_2
    \end{aligned}
  \end{equation}
\end{definition}

Das Wedgeprodukt in zwei Dimensionen kann auch als Determinante
\begin{equation}
  \begin{vmatrix}
    u_1 & v_1 \\
    u_2 & v_2
  \end{vmatrix}
  \mathbf{e_1} \wedge \mathbf{e_2}
\end{equation}
beschrieben werden.

\subsection{Das Wedgeprodukt in höheren Dimensionen}
Zunächst wollen wir uns nochmals unser bereits bekanntes
Vektorprodukt ansehen.
\begin{definition}
  Wir betrachten die beiden dreidimensionalen Vektoren $\mathbf{a}, \mathbf{b}$
  und bilden das Kreuzprodukt
  \begin{equation}
    \mathbf{a} \times \mathbf{b} = \begin{pmatrix} a_2 b_3 - a_3 b_2 \\ a_3 b_1 - a_1 b_3 \\ a_1 b_2 - a_2 b_1 \end{pmatrix}
    \label{geoalgebra:eq:vektorprodukt}
  \end{equation}
  
\end{definition}

Das Wedgeprodukt funktioniert nicht nur in der zweiten, sondern
in jeder beliebiger Dimension. Im folgenden möchten wir kurz
das Wedgeprodukt zwischen zwei dreidimensionalen Vektoren
herleiten.

\begin{definition}
Wir betrachten die \emph{dreidimensionalen} Vektoren $\mathbf{a}, \mathbf{b}$.
Das Wedgeprodukt dazwischen ist
  \newcommand{\eone}{\mathbf{e_1}}
  \newcommand{\etwo}{\mathbf{e_2}}
  \newcommand{\ethree}{\mathbf{e_3}}
\begin{equation}
  \begin{aligned}
      \mathbf{a} \wedge \mathbf{b} &= (a_1 \eone + a_2 \etwo + a_3 \ethree) \wedge (b_1 \eone + b_2 \etwo + b_3 \ethree) \\
      &= a_1 \eone \wedge (b_1 \eone + b_2 \etwo + b_3 \ethree) \\
      &\quad+ a_2 \etwo \wedge (b_1 \eone + b_2 \etwo + b_3 \ethree) \\
      &\quad+ a_3 \ethree \wedge (b_1 \eone + b_2 \etwo + b_3 \ethree) \\
      &= (a_1 b_2 - a_2 b_1) \eone \wedge \etwo \\
      &\quad+ (a_2 b_3 - a_3 b_2) \etwo \wedge \ethree \\
      &\quad+ (a_3 b_1 - a_1 b_3) \ethree \wedge \eone
  \end{aligned}
\end{equation}
.
\end{definition}
Wir stellen fest, dass die Koeffizienten hinter den drei entstandenen Wedgeprodukten
gleich sind, wie die Komponenten des Vektorprodukts (siehe \autoref{geoalgebra:eq:vektorprodukt}).

\subsection{Rückblick}
Hier wollen wir einen kurzen Rückblick vornehmen. Wir haben zunächst
die komplexen Zahlen nochmals kurz betrachtet und haben gesehen,
wie sie eine natürliche Erweiterung der reellen Zahlen bieten.
Analog dazu bietet die geometrische Algebra eine natürliche
Erweiterung zur bereits bekannten Vektorgeometrie. Wir wissen,
dass das Vektorprodukt $\mathbf{v_1} \times \mathbf{v_2}$ nur
in drei Dimensionen funktioniert und haben erkannt, dass
die neue Operation des Wedgeprodukts $\mathbf{v_1} \wedge \mathbf{v_2}$
im Gegensatz dazu auch in der zweiten Dimension funktioniert.
Später wollen wir das Ganze noch auf beliebige Dimensionen erweitern.
