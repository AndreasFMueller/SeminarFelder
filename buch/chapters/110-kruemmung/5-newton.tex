%
% Newtonsche Gravitation als Raumkrümmung
%
\section{Newtonsche Gravitation als Raumkrümmung
\label{buch:kruemmung:section:newton}}
\kopfrechts{Newtonsche Gravitation als Raumkrümmung}
Es gehört schon fast zum Allgemeinwissen, dass die einsteinsche
Relativitätstheorie die Gravitation als Raumkrümmung beschreibt.
Tatsächlich erlaubt aber auch die newtonsche Theorie eine solche
Interpretation, wie in diesem Abschnitt gezeigt werden soll.

%
% Koordinatensystem und newtonsche Gravitation
%
\subsection{Koordinatensystem und newtonsche Gravitation}
Wir verwenden ein vierdimensionales Koordinatensystem mit
den Koordinaten $x^0=ct$, $y^1=x$, $x^2=y$ und $x^3=z$.
Die Bahn eines Teilchens im Gravitationsfeld eines Himmelskörpers
der Masse $M$, der sich im Nullpunkt des Koordinatensystems befindet,
ist dann durch die Funktion
\[
t\mapsto
\begin{pmatrix}
ct\\
x(t)\\
y(t)\\
z(t)
\end{pmatrix}
\]
gegeben, die die Differentialgleichung
\begin{equation}
\frac{d^2}{dt^2}
\begin{pmatrix}
ct\\
x(t)\\
y(t)\\
z(t)
\end{pmatrix}
=
\begin{pmatrix}
0\\
\ddot{x}(t)\\
\ddot{y}(t)\\
\ddot{z}(t)
\end{pmatrix}
=
-
\frac{GMm}{r(t)^3}
\begin{pmatrix}
0\\
x(t)\\
y(t)\\
z(t)
\end{pmatrix}
\label{buch:kruemmung:newton:eqn:dgl}
\end{equation}
erfüllt, wobei
\[
r(t)
=
\sqrt{x(t)^2 + y(t)^2 + z(t)^2}
\]
ist.
Es ist möglich, die Bahn im Gravitationsfeld eines einzelnen,
ruhenden Himmelskörpers in geschlossener Form zu lösen.
Sobald mehrere massereiche Himmelskörper die Bahn beeinflussen,
wie dies im Sonnensystem mit dem Einfluss des Planeten Jupiter
\index{Sonnensystem}%
\index{Jupiter}%
der Fall ist, sind nur noch numerische Lösungen möglich.
Daher wird im folgenden nicht versucht, Lösungen direkt miteinander
zu vergleichen.
Stattdessen soll gezeigt werden, dass die Differentialgleichung
\eqref{buch:kruemmung:newton:eqn:dgl}
auch als Geodätengleichung einer geeigneten Metrik geschrieben
werden kann.
Zu diesem Zweck schreiben wir die Differentialgleichung noch in
der Komponentenform
\begin{equation}
\begin{aligned}
\ddot{x}^0(t) &= 0 \\
\ddot{x}^i(t) &= - \frac{GM}{r(t)^2}\,x^i(t)\qquad \text{für $i>0$}.
\end{aligned}
\label{buch:kruemmung:newton:eqn:dglcomponents}
\end{equation}
Die erste Gleichung bedeutet nichts anderes, als dass es in der
newtonschen Gravitationstheorie eine absolute Zeit gibt.
\index{Gravitation!newtonsch}%

%
% Gravitationspotential
%
\subsection{Gravitationspotential}
Das newtonsche Gravitationsfeld ist ein Potentialfeld.
Das Potential der Masse $M$ im Nullpunkt des Koordinatensystems
hat das Potential
\[
U(x) = \frac{GM}{r}.
\]
\index{Gravitationspotential}%
Tatsächlich ist der Gradient 
\[
\operatorname{grad}U
=
-\frac{GM}{r^2} \operatorname{grad}{r}
=
-\frac{GM}{r^2}
\cdot
\frac{1}{2r}
\begin{pmatrix}
2x(t)\\
2y(t)\\
2z(t)
\end{pmatrix}
=
-\frac{GM}{r^3}
\begin{pmatrix}
x(t)\\
y(t)\\
z(t)
\end{pmatrix}
.
\]
Die rechte Seite stimmt mit der rechten Seite der Differentialgleichung
\eqref{buch:kruemmung:newton:eqn:dgl}
überein.

Sind mehrere Massepunkte gegeben, kann ihr gemeinsames Potential 
als Summe
\[
U(x)
=
\sum_{i=1}^n U_i(x)
\]
geschrieben werden, wobei
\[
U_i(x)
=
\frac{GM_i}{|x-x_i|}
\]
das Potential der Masse $M_i$ an der Stelle $x_i$ ist.

Für die folgende Rechnung gehen wir von der Annahme aus, dass sich die
Massen nicht bewegen, dass das Gravitationsfeld also statisch ist.
Diese Annahme ist näherungsweise dadurch gerechtfertigt, dass die
Geschwindigkeit der Massen in schwachen Gravitationsfeldern wie in
unserem Sonnensystem viel kleiner sind als die Lichtgeschwindigkeit.

%
% Metrik
%
\subsection{Metrik}
Wir verwenden die Metrik
\begin{equation}
\begin{aligned}
g_{00} &= \phantom{-}1 + \frac{2U(x)}{c^2} \\
g_{ii} &= -1\qquad &i&>0
\end{aligned}
\label{buch:kruemmung:newton:eqn:metrik}
\end{equation}
für die vierdimensionale Mannigfaltigkeit mit den Koordinaten
$(x^0,x^1,x^2,x^3)$.
Nur die Zeitkomponente der Metrik weicht von der Minkowski-Metrik ab.
\index{Minkowski-Metrik}%

Da das Potential im Vergleich zu $c^2$ klein ist, werden wir nur in
erster Näherung arbeiten.
Terme zweiter Ordnung in $U/c^2$ enthalten Quadrate von $U/c^2$ und sind
noch viel kleiner, sie dürfen daher vernachlässigt werden.

In den folgenden Abschnitten werden schrittweise die Christoffel-Symbole
berechnet, mit denen dann in
Abschnitt~\ref{buch:kruemmung:newtion:subsection:geodaeten}
die Geodätengleichung aufgestellt werden soll.

%
% Die Inverse der Metrik
%
\subsubsection{Die Inverse der Metrik}
Da die Metrik im gewählten Koordinatensystem eine Diagonalmatrix ist,
ist auch die inverse Matrix diagonal mit den reziproken Diagonalelementen.
Wir brauchen nur eine Approximation in erster Ordnung in $2U/c^2$
für diese Elemente.

Das Element $g_{00}$ ist von der Form $1+x$.
Die geometrische Reihe
\[
1+q+q^2+q^3+\dots = \frac{1}{1-q}
\]
liefert für $q=-x$ die Approximation
\[
\frac{1}{1+x} = 1-x+x^2-x^3+o(x^3)
\]
für das reziproke Element.
Angewendet auf den Tensor $g_{ik}$ ergeben sich in erster Ordnung
die Diagonalelemente
\begin{equation}
\begin{aligned}
g^{00}
&=
\frac{1}{\displaystyle 1+\frac{2U}{c^2}}
=
1-\frac{2U}{c^2}+o\biggl(\frac{2U}{c^2}\biggr)
\\
g^{ii}&= -1
&&\text{für $i>0$.}
\end{aligned}
\end{equation}

%
% Ableitung der metrischen Koeffizienten
%
\subsubsection{Ableitungen der metrischen Koeffizienten}
Da nur das Element $g_{00}$ nicht konstant ist, gibt es nur die eine
nicht verschwindende Ableitung
\[
\frac{\partial g_{00}}{\partial x^k}
=
\begin{cases}
0
&\qquad\text{für $k=0$}\\[4pt]
\displaystyle
\frac{2}{c^2}
\frac{\partial U}{\partial x^k}
&\qquad\text{für $k>0$.}
\end{cases}
\]

%
% Christoffel-Symbole 1. Art
%
\subsubsection{Christoffel-Symbole erster Art}
Die Christoffel-Symbole erster Art sind
\[
\Gamma_{l,ik}
=
\frac{1}{2}\biggl(
\frac{\partial g_{lk}}{\partial x^i}
+
\frac{\partial g_{li}}{\partial x^k}
-
\frac{\partial g_{ik}}{\partial x^l}
\biggr).
\]
Es folgt, dass nur diejenigen Christoffel-Symboel von $0$ verschieden
sind, die genau zwei $0$-Indizes haben.
Diese sind
\begin{align}
\Gamma_{l,00}
&=
-\frac12 \frac{\partial g_{00}}{\partial x^l}
=
-
\frac{1}{c^2}\frac{\partial U}{\partial x^l}
\notag
\\
\Gamma_{0,l0}
=
\Gamma_{0,0l}
&=
\phantom{-}
\frac{1}{2}\frac{g_{00}}{\partial x^l}
=
\frac{1}{c^2}\frac{\partial U}{\partial x^l},
\label{buch:kruemmung:newton:eqn:bewegungsgleichung}
\end{align}
wobei $l>0$ ist.
Alle anderen Christoffel-Symbole verschwinden.

%
% Christoffel-Symbole 2. Art
%
\subsubsection{Christoffel-Symbole zweiter Art}
Die Christoffel-Symbole zweiter Art entstehen durch Multiplikation
mit von $\Gamma_{l,ik}$ mit $g^{jl}$.
Da nur die Elemente mit $j=l$ von Null verschieden sind, sind wieder
nur die Christoffel-Symbole zweiter Art von Null verschieden, die zwei
$0$-Indizes haben:
\begin{align*}
\Gamma^l_{00}
&=
(-1)\cdot\biggl(-\frac{1}{c^2}\frac{\partial U}{\partial x^l}\biggr)
=
\frac{1}{c^2}\frac{\partial U}{\partial x^l}
\\
\Gamma^0_{l0}
=
\Gamma^0_{0l}
&=
\biggl(1-\frac{2U}{c^2}\biggr)
\frac{1}{c^2}\frac{\partial U}{\partial x^l}
=
\frac{1}{c^2}\frac{\partial U}{\partial x^l}
+
o\biggl(
\frac{2U}{c^2}
\biggr)
\end{align*}
für $l>0$.
Die nicht verschwindenden Christoffel-Symbole zweiter Art hängen nur
vom Index $l$ ab.
Abkürzend können wir
\[
\Gamma^l_{00}
=
\Gamma^0_{l0}
=
\Gamma^0_{0l}
=
\Gamma_l
= 
\frac{1}{c^2} \frac{\partial U}{\partial x^l}
\]
schreiben.

%
% Geodäten
%
\subsection{Geodäten
\label{buch:kruemmung:newtion:subsection:geodaeten}}
Die Geodätengleichung ist
\[
\frac{d^2 x^l}{ds^2}
=
-
\Gamma^l_{ik} \frac{dx^i}{ds}\frac{dx^k}{ds}.
\]
Mit den oben berechneten Werten für die Christoffel-Symbole zweiter
Art wird daraus für $l>0$ die Gleichung
\begin{align}
\frac{d^2x^l}{ds^2}
&=
-
\Gamma^l_{00}\frac{dx^0}{ds}\frac{dx^0}{ds}
=
-
\frac{1}{c^2}
\frac{\partial U}{\partial x^l}
\frac{dx^0}{ds}
\frac{dx^0}{ds}.
\notag
\intertext{
Man möchte aber $t$ als Kurvenparameter verwenden, nicht $s$, was man
durch Multiplikation mit $(ds/dt)^2$ und Anwendung der Kettenregel
erreichen kann.
Es entsteht}
\frac{d^2x^l}{ds^2}
\biggl(\frac{ds}{dt}\biggr)^2
&=
-
\frac{1}{c^2}
\frac{\partial U}{\partial x^l}
\frac{dx^0}{ds}
\frac{dx^0}{ds}
\biggl(\frac{ds}{dt}\biggr)^2
\notag
\\
\frac{d^2x^l}{dt^2}
&=
-
\frac{1}{c^2}
\frac{\partial U}{\partial x^l}
\underbrace{
\frac{dx^0}{dt}
\frac{dx^0}{dt}
}_{\displaystyle = c^2}
=
-\frac{\partial U}{\partial x^l}.
\label{buch:kruemmung:newton:geodaeten:eqn:final}
\end{align}
Dabei wurde verwendet, dass wegen $x^0=ct$ auch $dx^0/dt=c$ gilt.
\eqref{buch:kruemmung:newton:geodaeten:eqn:final}
ist die Bewegungsgleichung für die Bewegung eines Teilchens
im Gravitationspotential $U(x)$.

Die Geodätengleichung für die $0$-Komponente ist
\begin{align*}
\frac{d^2x^0}{ds^2}
&=
-
\Gamma^0_{ik}
\frac{dx^i}{ds}
\frac{dx^k}{ds}.
\end{align*}
Wegen $x^0=ct$ sagt diese Gleichung nur etwas über die Abhängigkeit 
zwischen $s$ und $t$ aus, die nicht weiter interessiert,
da in der
Bewegungsgleichung~\eqref{buch:kruemmung:newton:eqn:bewegungsgleichung}
der Parameter $s$ gar nicht mehr vorkommt.

%
% Raumkrümmung
%
\subsection{Raumkrümmung für die newtonsche Gravitation}
Die Tatsache, dass aus der Metrik
\eqref{buch:kruemmung:newton:eqn:metrik}
die
Bewegungsgleichungen~\eqref{buch:kruemmung:newton:eqn:bewegungsgleichung}
der newtonschen Gravitation entstehen, rechtfertigt die Idee, die
Gravitation als Konsequenz der Raumkrümmung zu betrachten.
\index{Raumkrümmung für newtonsche Gravitation}%

%
% Der riemannsche Krümmungstensor
%
\subsubsection{Der riemannsche Krümmungstensor}
Mit den Christoffel-Symbolen kann jetzt auch der riemannsche Krümmungstensor
berechnet werden.
Die Definition ist
\begin{equation}
R^{i}\mathstrut_{klm}
=
\frac{\partial \Gamma^i_{km}}{\partial x^l}
-
\frac{\partial \Gamma^i_{kl}}{\partial x^m}
+
\Gamma^i_{nl}
\Gamma^n_{km}
-
\Gamma^i_{nm}
\Gamma^n_{kl}.
\end{equation}
Aus den Symmetrieeigenschaften des Krümmungstensors folgt, dass die
Komponenten mit $i=k$ und $l=m$ verschwinden.
Insbesondere müssen mindestens zwei der Indizes von Null verschieden sein.
In den Produkten der Christoffel-Symbole kommen nur Terme höherer Ordnung
vor, es reicht also, die Ableitungsterme zu berücksichtigen.

Damit die partiellen Ableitungen in den ersten beiden Termen nicht
verschwinden, muss nach einer Raumvariablen abgeleitet werden und von
den anderen drei Indizes muss genau einer von Null verschieden sein.
Für $i\ne 0\ne l$ ist
\begin{align}
R^i\mathstrut_{0l0}
&=
\frac{\partial \Gamma^i_{00}}{\partial x^l}
-
\frac{\partial \Gamma^i_{k0}}{\partial x^0}
=
\frac{\partial \Gamma^i_{00}}{\partial x^l}
=
\frac{1}{c^2}
\frac{\partial^2 U}{\partial x^i\,\partial x^l}.
\label{buch:kruemmung:newton:eqn:riemann}
\end{align}
Ist ein weiterer Index von 0 verschieden, sind in den Ableitungstermen
mindestens zwei Indizes der Christoffel-Symboel von 0 verschieden.
Somit verschwinden alle diese Komponenten.

%
% Ricci-Krümmung
%
\subsubsection{Ricci-Krümmung}
Die Ricci-Krümmung entsteht durch Kontraktion des riemannschen
Krümmungstensors über den ersten und dritten Index.
Die 0-Komponente ist
\[
R_{00}
=
R^l\mathstrut_{0l0}
=
\frac{1}{c^2}
\sum_{l=1}^3
\frac{\partial^2 U}{\partial (x^l)^2}
=
\frac{1}{c^2}\Delta U.
\]
Die newtonsche Gravitationstheorie besagt, dass es einen Zusammenhang
zwischen der Materiedichte $\varrho$ und dem Potential gibt, es gilt
nämlich die Differentialgleichung
\begin{equation}
\Delta U = 4\pi \varrho.
\label{buch:kruemmung:newton:potential}
\end{equation}
ersetzt man $\varrho$ durch die zugehörige Energiedichte
$\varepsilon = \varrho c^2$, ergibt sich die Gleichung
\[
R_{00}
=
\frac{4\pi G}{c^4}\varepsilon.
\]
Damit ist die später abzuleitende einsteinsche Feldgleichung bereits
angedeutet.

Die anderen Komponenten des Ricci-Tensors sind
\begin{align}
R_{0m}
&=
R^l\mathstrut_{0lm}
=
R^0\mathstrut_{00m}
=
0
&k>0
\notag
\\
R_{km}
&=
R^l\mathstrut_{klm}
=
R^0\mathstrut_{k0m}
=
R^k\mathstrut_{0m0}
=
\frac{1}{c^2}
\frac{\partial^2 U}{\partial x^k\,\partial x^m}
&k,m&>0.
\label{buch:kruemmung:newton:Rik}
\end{align}
Der Ricci-Tensor besteht also im Wesentlichen aus den zweiten
partiellen Ableitungen des Potentials.
Eine Feldgleichung für den Ricci-Tensor ist also ein lineare
partielle Differentialgleichung zweiter Ordnung für das Potential.
Die Einstein-Gleichungen werden sich als nichtlinear herausstellen,
für die vorliegende newtonsche Näherung haben wir aber in linearer
Näherung gearbeitet, so dass das verschwinden der nichtlinearen
Terme nicht überrascht.

%
% Krümmungsskalar
%
\subsubsection{Der Krümmungsskalar}
Da die Metrik $g_{ik}$ ebenso wie $g^{ik}$ diagonal sind, werden
für den Krümmungsskalar nur die diagonalen Elemente des Ricci-Tensors
benötigt.
Da Elemente von $R_{ik}$ alle von erster Ordnung sind, kann der
zusätzliche Term in $g_{00}$ bzw.~$g^{00}$ vernachlässigt werden.
Damit ist 
\begin{align*}
R
&=
g^{ik}R_{ik}
=
g^{00}R_{00}
+
\sum_{i=1}^3
g^{ii}R_{ii}
=
\frac{1}{c^2}\Delta U
-
\frac{1}{c^2}
\sum_{i=1}^3
\frac{\partial^2 U}{\partial (x^i)^2}
=
0.
\end{align*}
Der Krümmungsskalar verschwindet also.
