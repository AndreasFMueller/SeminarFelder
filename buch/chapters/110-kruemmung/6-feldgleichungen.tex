%
% Energie-Impuls-Tensor und die Feldgleichungen der Gravitation
%
\section{Energie-Impuls-Tensor und die Feldgleichungen der Gravitation
\label{buch:kruemmung:section:gravitation}}
\kopfrechts{Energie-Impuls-Tensor und die Feldgleichungen der Gravitation}

Der Abschnitt~\ref{buch:kruemmung:section:newton} suggeriert, dass sich
die Bewegung eines Teilchens in einem newtonschen Gravitationsfeld
vollständig durch die die Metrik und den daraus abgeleiteten riemannschen
Krümmungstensor beschreiben lässt.
Die Komponenten des Ricci-Tensors scheinen nach
\eqref{buch:kruemmung:newton:Rik} die zweiten Ableitungen des
newtonschen Gravitationspotentials zu verallgemeinern.
Newtons Gravitationsgesetz besagt, dass das Gravitationspotential von
der Masseverteilung durch die Poisson-Gleichung
\eqref{buch:kruemmung:newton:potential}
mit der Massedichte $\varrho$ bestimmt ist.
Es ist daher zu erwarten, dass eine relativistische Gravitationstheorie
konstruiert werden kann.
Die Rolle des Gravitationsfeldes wird von der Metrik übernommen und der
Laplace-Operator wird ersetzt durch den Ricci-Tensor.
Zur Vervollständigung einer Feldgleichung braucht es aber noch einen
Term, der die Massedichte $\varrho$ verallgemeinert.

%
% Konsequenzen der allgmeinen Kovarianz
%
\subsubsection{Konsequenzen der allgemeinen Kovarianz}
Die allgemeine Kovarianz fordert, dass die konstruierte Feldgleichung
von der Wahl des Koordinatensystems unabhängig sein muss.
Sie verlangt, dass die kinetische Energie bewegter Massen ebenfalls
gravitative Wirkung entfaltet.
Die Massedichte $\varrho$ fängt jedoch nur die Ruhemasse ein.
Ein Skalar oder eine Dichte ist grundsätzlich nicht fähig, die
Relativgeschwindigkeiten zu berücksichtigen.
Die Bewegung eines Teilchens äussert sich im Impuls, den es trägt.
Der Viererimpuls eines Teilchens, der als Zeitkomponente die Energie
und als Raumkomponenten die Impulskomponenten enthält, ist ein
besserer Kandidat.
Energie kann aber auch in Form von Spannungen in einem Material 
gespeichert werden.
Solche können jedoch nur mit einem Tensor zweiter Stufe wiedergegeben
werden, im Kapitel~\ref{chapter:elastomechanik} zur Elastomechanik
ausgeführt wird.

Gesucht ist also ein Feldgesetz in der Form
\begin{equation}
\left(
\begin{minipage}{1.8cm}
\raggedright
Ausdruck in $R_{ik}$ und $S$
\end{minipage}
\right)
=
\left(
\text{Kopplungskonstante}
\right)
\cdot
\left(
\begin{minipage}{3.5cm}
\raggedright
Ausdrück für Energie, Impuls und Spannungen
\end{minipage}
\right).
\label{buch:kruemmung:feldgleichungen:eqn:schema}
\end{equation}
Die allgemeine Kovarianz verlangt, dass die Terme $R_{ik}$ und $S$
nur linear auf der linken Seite auftreten können.
Somit muss auch für die rechte Seite ein Tensor zweiter Stufe sein.

Die Kopplungskonstante gibt an, wie stark die Gravitationswirkung
der Masse ist. 
In der newtonschen Theorie ist dies die Gravitationskonstante $G$
und ist durch Messungen sehr genau bestimmt worden.

%
% Der Energie-Impuls-Tensor
%
\subsubsection{Der Energie-Impuls-Tensor}
Gesucht ist daher ein Tensor $T$ zweiter Stufe mit Komponenten
$T_{ik}$, der als $T_{00}$-Kompo\-nen\-te die Ruheenergie des enthält.
Die Komponenten $T_{0k}$ mit $k>0$ beschreiben die Impulsdichte.
Die Komponenten $T_{ik}$ mit $i,k>0$ bieten genügend Platz, die
Spannungen unterzubringen.
Auf der Diagonalen stehen die Druckspannungen, ausserhalb der Diagonalen
die Scherspannungen.
So entsteht der sogenannte \emph{Energie-Impuls-Tensor} $T$.
Er konsolidiert auf allgemein kovariante Art alle Formen von Energie
in einem einzigen Tensor.

In einem ruhenden idealen Gas gibt es keine Scherspannungen,
der Energie-Impuls-Tensor ist daher Diagonal.
Man kann ihn in Matrixform als
\begin{equation}
T_{ik}
=
\begin{pmatrix}
\varrho     c^2 & 0             & 0             & 0             \\
0               & p_x           & 0             & 0             \\
0               & 0             & p_y           & 0             \\
0               & 0             & 0             & p_z
\end{pmatrix}
\label{buch:kruemmung:feldgleichung:eqn:Tik}
\end{equation}
schreiben kann.
Das Element $T_{00} = \varepsilon = \varrho c^2$ ist die
Ruheenergiedichte des Gases.
Falls das Gas nicht ruht, kann man den Energie-Impuls-Tensor durch
eine Lorentztransformation in ein mit dem Gas mitbewegtes
Koordinatensystem transformieren und so einen allgemeinen Ausdruck
$T_{ik}$ finden.

Da der Ricci-Tensor symmetrisch ist, muss auch der Tensor $T_{ik}$
symmetrisch sein, wenn eine lineare Feldgleichung aus dem Ricci-Tensor
und dem Energie-Impuls-Tensor gelten soll.
Die Definition~\eqref{buch:kruemmung:feldgleichung:eqn:Tik} zeigt,
dass $T_{ik}$ symmetrisch ist.
Für die Impulskomponenten ist dies offensichtlich.
Für die Spannungskomponenten beruht dies auf der in der
Elastiztitätstheorie (Kapitel~\ref{chapter:elastomechanik})
diskutierten Tatsache, dass ein asymmetrischer Spannungstensor dazu
führen würde, dass das Medium durch verbleibende Drehmomente zu
beliebig schneller Rotation angeregt würde, was nicht sein kann.

Auch das elektromagnetische Feld kann gravitative Wirkung haben.
Es gibt daher auch einen Energie-Impuls-Tensor für das elektromagnetische
Feld, der sich aus dem Faraday-Tensor konstruieren lässt, der in
Kapitel~\ref{chapter:maxwell} erklärt wird.
Er ist
\[
T^{ik}
=
F^{il}F_l\mathstrut^k - \frac12 g^{ik}F^{ab}F_{ab}.
\]
Auch dieser Tensor hat verschwindende kovariante Divergenz.

%
% Erhaltungssätze
%
\subsubsection{Erhaltungssätze}
Sowohl der Ricci-Tensor wie auch der Krümmungsskalar können in einem
Gravitationsgesetz der Form \eqref{buch:kruemmung:feldgleichungen:eqn:schema}
auftreten.
Auf der linken Seite einer Feldgleichung für die Gravitation könnten daher
eine beliebige Linearkombination $\alpha R_{ik}+\beta Sg_{ik}$ stehen.
Der Faktor $g_{ik}$ ist notwendig, um aus dem Krümmungsskalar $S$ einen
Tensor zweiter Stufe zu machen.

Der Energie-Impuls-Tensor auf der rechten Seite der Gleichung schränkt
die Linearkombination weiter ein.
Die Energieerhaltung bedeutet, dass für $T_{ik}$ eine kovariante 
Kontinuitätsgleichung gilt, die in Komponenten durch die kovariante
Divergenz
\[
T^i\mathstrut_{k;i}
=
0
\]
gegeben ist.
Der Linearkombination auf der linken Seite der Feldgleichung muss daher
ebenfalls verschwindende Divergenz haben.

Die Komponenten $R^i\mathstrut_{k;i}$ verschwinden nicht unbedingt,
es gelten die kontrahierten Bianchi-Identitäten.
Ebensowenig verschwinden die kovarianten Ableitungen $S_{;k}$.
Gesucht ist daher eine Linearkombination von $R_{ik}$ und $Sg_{ik}$,
deren kovariante Divergenz verschwindet.
Die kontrahierten Bianchi-Identitäten für den Ricci-Tensor zeigen, dass  der
\emph{Einstein-Tensor}
\index{Einstein-Tensor}%
\[
G_{ik}
=
R_{ik}
-\frac12 g_{ik}S
\]
diese Eigenschaft hat, denn in
\[
G^i\mathstrut_{k;i}
=
R^i\mathstrut_{k;i}
-
\frac12 S_{;k}
\]
ist der erste Term auf der rechten Seite
nach \eqref{buch:kruemmung:eqn:kontrahiertS}
die halbe Ableitung des Krümmungsskalars, hebt sich also mit dem
zweiten Term weg und ergibt $G^i\mathstrut_{k;i}=0$.

%
% Die Feldgleichungen
%
\subsubsection{Die Feldgleichungen}
Die Form der gesuchten Feldgleichungen der Gravitations zeichnet sich
jetzt klar ab.
Es muss nur noch die Kopplungskonstante ermittelt werden.
Durch Vergleichung mit der newtonschen Theorie schwacher Felder 
kann man zeigen, dass 
\begin{equation}
G_{ik}
=
R_{ik}
-
\frac12 Rg_{ik}
=
\frac{8\pi G}{c^4}T_{ik}
\label{buch:kruemmung:feldgleichung:eqn:feldgleichung}
\end{equation}
gelten muss.
Dies sind die \emph{einsteinschen Feldgleichungen} der Gravitation.
\index{Feldgleichung!einsteinsche}%
\index{Gravitation, Feldgleichung}%
Die folgenden Abschnitte enthalten Anwendungen auf schwarze Löcher
und die Kosmologie.

