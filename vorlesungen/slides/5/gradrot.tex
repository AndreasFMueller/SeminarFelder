%
% gradrot.tex -- slide template
%
% (c) 2021 Prof Dr Andreas Müller, OST Ostschweizer Fachhochschule
%
\bgroup
\begin{frame}[t]
\setlength{\abovedisplayskip}{5pt}
\setlength{\belowdisplayskip}{5pt}
\frametitle{Gradient und Rotation}
\vspace*{-10pt}
\begin{columns}[t,onlytextwidth]
\begin{column}{0.48\textwidth}
\begin{block}{Differential einer Funktion}
Eine Funktion
\[
f\colon M\to\mathbb{R}
\]
\uncover<2->{hat die äussere Ableitung}
\begin{align*}
\uncover<3->{df
&=
\sum_{i=1}^n \frac{\partial f}{\partial x^i} \,dx^i
}
\\
\uncover<4->{
&=
\begin{pmatrix}
\displaystyle\frac{\partial f}{\partial x^1}
&
\dots
&
\displaystyle\frac{\partial f}{\partial x^n}
\end{pmatrix}
}
\\
\uncover<5->{
&=
\text{``
$
\operatorname{grad} f(x)
$
''}
}
\end{align*}
\end{block}
\end{column}
\begin{column}{0.48\textwidth}
\begin{block}{Ableitung einer 1-Form}
\uncover<6->{%
Eine 1-Form
\[
\alpha = \displaystyle\sum_{i=1}^n a_i(x)\,dx^i \in \Omega^1(M)
\]
hat die äussere Ableitung}
\begin{align*}
\uncover<7->{
d\alpha
&=
\sum_{i,k=1}^n \frac{\partial a_i}{\partial x^k}\, dx^k\wedge dx^i
}
\\
\uncover<8->{
&=
\sum_{k<i}
\biggl(
\frac{\partial a_i}{\partial x^k}
-
\frac{\partial a_k}{\partial x^i}
\biggr)
\,dx^k\wedge dx^i
}
\\
\uncover<9->{
&=
\text{``
\(
\operatorname{rot}\vec{a}
\)
''}
}
\end{align*}
\end{block}
\end{column}
\end{columns}
\end{frame}
\egroup
