%
% template.tex -- slide template
%
% (c) 2021 Prof Dr Andreas Müller, OST Ostschweizer Fachhochschule
%
\bgroup
\definecolor{blau}{rgb}{0.4,0.6,1.0}
\begin{frame}[t]
\setlength{\abovedisplayskip}{5pt}
\setlength{\belowdisplayskip}{5pt}
\frametitle{Begriffe}
\vspace{-20pt}
\begin{columns}[t,onlytextwidth]
\begin{column}{0.48\textwidth}
\begin{block}{Tangentialvektoren\strut
\uncover<13->{{\color{orange}~``Forscher''}}}
\begin{itemize}
\item<1-> mit gleicher Geschwindigkeit durch einen Punkt gehende Kurven
\item<2->
\(
X = \xi^{\color{blau}1}\vec{e}_{\color{blau}1} + \dots +
\xi^{\color{blau}n} \vec{e}_{\color{blau}n}
\)
\item<3->
\(
\dot{\gamma}(t)
=
\dot{x}^{\color{blau}1}(t) \vec{e}_{\color{blau}1} + \dots
+ \dot{x}^{\color{blau}n}\vec{e}_{\color{blau}n}
\)
\item<4-> Koordinatentransformation:
\(
\dot{y}^k
=
{\color{blau}\sum_{i=1}^n}J^k\mathstrut_{\color{blau}i}\dot{x}^{\color{blau}i}
\)
\item<5-> $\xi^{\color{blau}i}$ {\color{blau}kontravariant}
\item<6-> {\color{blau}Spalten}vektoren
\item<7-> Differentialoperator:
\(
X
=
{\color{blau}\sum_{i=1}^n}
\xi^{\color{blau}i} \frac{\partial}{\partial x^{\color{blau}i}}
\)
\end{itemize}
\end{block}
\end{column}
\begin{column}{0.48\textwidth}
\begin{block}{1-Formen\strut
\uncover<14->{{\color{orange}~``Felder''}}}
\begin{itemize}
\item<1-> Etwas, was man entlang einer Kurve messen kann
\item<2->
\(
\alpha
=
a_{\color{red}1}\,dx^{\color{red}1}
+ \dots +
a_{\color{red}n}\,dx^{\color{red}n}
\)
\item<3->
\(
df
=
\frac{\partial f}{\partial x^{\color{red}1}} \,dx^{\color{red}1}
+ \dots +
\frac{\partial f}{\partial x^{\color{red}n}}\,dx^{\color{red}n}
\)
\item<4-> Koordinatentransformation:
\(
\frac{\partial f}{\partial x^i}
=
{\color{red}
\sum_{k=1}^n}
\frac{\partial f}{\partial y^{\color{red}k}}J^{\color{red}k}\mathstrut_i
\)
\item<5-> $a_{\color{red}i}$ {\color{red}kovariant}
\item<6-> {\color{red}Zeilen}vektoren
\item<7-> Richtungsableitung und Differential $df$:
\(
X\cdot f
=
\langle
df,X
\rangle
\)
\end{itemize}
\end{block}
\end{column}
\end{columns}
\only<8>{allgemein kovariant:}
\(
\displaystyle
\uncover<8->{
\langle \alpha,X\rangle
}
\only<8->{=
\biggl\langle
{\color{red}\sum_{i=1}^n} a_{\color{red}i}(x)\,dx^{\color{red}i},
{\color{blau}\sum_{k=1}^n} \xi^{\color{blau}k}\frac{\partial}{\partial x^{\color{blau}k}}
\biggr\rangle
}
\only<9>{=
\sum_{{\color{red}i},{\color{blau}k}=1}^n
a_{\color{red}i}(x)\xi^{\color{blau}k}
\biggl\langle
dx^{\color{red}i},\frac{\partial}{\partial x^{\color{blau}k}}
\biggr\rangle
}
\only<10>{=
\sum_{{\color{red}i},{\color{blau}k}=1}^n
a_{\color{red}i}(x)\xi^{\color{blau}k}
\delta^{\color{red}i}_{\color{blau}k}
}
\only<11->{=
\sum_{i=1}^n
a_i(x)\xi^i
}
\only<12->{
=
{\color{red}
\begin{pmatrix}a_1(x)\dots a_n(x)\end{pmatrix}
}
\smash{
\color{blau}
\begin{pmatrix}
\xi^1\\
\vdots\\[-2pt]
\xi^n
\end{pmatrix}
}
}
\)
\end{frame}
\egroup
