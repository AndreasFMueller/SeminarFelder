%
% parabolisch.tex -- parabolisch
%
% (c) 2021 Prof Dr Andreas Müller, OST Ostschweizer Fachhochschule
%
\documentclass[tikz]{standalone}
\usepackage{amsmath}
\usepackage{times}
\usepackage{txfonts}
\usepackage{pgfplots}
\usepackage{csvsimple}
\usetikzlibrary{arrows,intersections,math}
\definecolor{darkred}{rgb}{0.8,0,0}
\begin{document}
\def\skala{3}
\begin{tikzpicture}[>=latex,thick,scale=\skala]

\draw[->] (-1.6,0) -- (1.7,0) coordinate[label={$x^1$}];
\draw[->] (0,-1.3) -- (0,1.37) coordinate[label={right:$x^2$}];
\begin{scope}
	\clip (-1.6,-1.3) rectangle (1.6,1.3);
	\foreach \m in {0.2,0.4,0.6,0.8,1,1.2,1.4,1.6}{
		\draw[color=darkred] plot[domain=-1.6:1.6,samples=100]
			({\x},{-\m*\m/2+\x*\x/(2*\m*\m)});
		\draw[color=blue] plot[domain=-1.6:1.6,samples=100]
			({\x},{\m*\m/2-\x*\x/(2*\m*\m)});
	}
\end{scope}

\end{tikzpicture}
\end{document}

