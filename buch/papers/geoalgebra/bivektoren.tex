
\section{Bivektoren
\label{geoalgebra:section:bivektoren}}
\kopfrechts{Bivektoren}
Durch das Wedgeprodukt zwischen $\mathbf{v_1}$ und $\mathbf{v_2}$ erhalten wir
den
Bivektor $\mathbf{v_1} \wedge \mathbf{v_2}$. Einen Bivektor können wir uns als \emph{gerichtete Fläche}, die zwischen den beiden Vektoren aufgespannt
wird, vorstellen (siehe \autoref{geoalgebra:fig:bivektor-gegenuhrzeigersinn}).
\begin{figure}
\begin{center}
\begin{tikzpicture}
  \draw[step=.5cm,gray,very thin] (-0.25,-0.25) grid (5.25,4.25);
  \coordinate (0) at (0, 0);
  \coordinate (A) at (3, 1);
  \coordinate (B) at (2, 3);
  \coordinate (C) at ($(A) + (B)$);
  \fill[blue!30, opacity=0.7] (0) -- (A) -- (C) -- (B) -- cycle;

  \draw[thick, ->, red] (0) -- (A) node[midway,below] {$\mathbf{v_1}$};
  \draw[thick, ->, blue] (0) -- (B) node[midway,left] {$\mathbf{v_2}$};
  \draw[dashed] (A) -- (C);
  \draw[dashed] (B) -- (C);
  \node (K) at ($0.25*(0)+0.25*(A)+0.25*(B)+0.25*($(C)$)$)
  {$\color{red}\mathbf{v_1} \color{black}\wedge \color{blue}\mathbf{v_2}$};
\draw[thick,->] (K) ++(0.6,0) arc (0:270:0.6);
\end{tikzpicture}
\end{center}
  \caption{Bivektor mit Richtung im
  Gegenuhrzeigersinn}\label{geoalgebra:fig:bivektor-gegenuhrzeigersinn}
\end{figure}

