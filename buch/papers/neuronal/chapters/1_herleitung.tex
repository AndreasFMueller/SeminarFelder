%
% 1_herleitung.tex -- Herleitung der Methode
%
% (c) 2025 Roman Cvijanovic & Nicola Dall'Acqua, Hochschule Rapperswil
%
% !TEX root = ../../buch.tex
% !TEX encoding = UTF-8
%

\section{Herleitung der Methode\label{neuronal:section:herleitung}}
\kopfrechts{Herleitung der Methode}

Im Folgenden wird die Methode zum Lösen von Feldgleichungen mittels eines neuronalen Netzes theoretisch hergeleitet.
Dies wird anhand des Beispiels der Wellengleichung in zwei räumlichen Dimensionen im Bereich \( x, y \in [-10,10], t \in [0,10] \) gemacht
\begin{equation}
    \frac{\partial^2 u}{\partial t^2} = c^2 \left( \frac{\partial^2 u}{\partial x^2} + \frac{\partial^2 u}{\partial y^2} \right).
    \label{neuronal:wellengleichung}
\end{equation}

Wobei \( u(x, y, t) \) die z-Koordinate am Punkt \( (x, y) \) zum Zeitpunkt \( t \) darstellt. 
Zudem ist \( c \in \mathbb{R} \) eine Konstante und stellt die Verbreitungsgeschwindigkeit der Welle dar.

Zusätzlich werden die folgenden Initialbedingungen
\begin{equation}
    \begin{aligned}
        u(x, y, 0) &= \sin(\pi x) \sin(\pi y)\\
        \frac{\partial u(x, y, 0)}{\partial t} &= 0
    \end{aligned}
    \label{neuronal:initial}
\end{equation}
und die Randbedingungen
\begin{equation}
    \begin{aligned}
        u(-10, y, t) = 0\\
        u(10, y, t) = 0\\
        u(x, -10, t) = 0\\
        u(x, 10, t) = 0
    \end{aligned}
    \label{neuronal:rand}
\end{equation}
verwendet.
Weiter ist das neuronale Netzwerk gegeben als
\begin{equation}
    \hat{u}(x, y, t; \vartheta).
    \label{neuronal:nn}
\end{equation}
Das Netz hängt von den gleichen Variablen ab wie \( u \).
Zusätzlich besitzt es einen Vektor \( \vartheta \in \mathbb{R}^n \) der n \emph{trainierbaren Parameter}.
Das Ziel des Trainings eines neuronalen Netzes ist es, diese Parameter so zu wählen, dass das Netz die gesuchte Funktion (hier \( u \)) möglichst gut approximiert.


\subsection{Formulierung als Optimierungsproblem}\label{neuronal:subsection:optimierungsproblem}
Durch Subtrahieren der rechten Seite von der Wellengleichung \eqref{neuronal:wellengleichung}, anschliessendem substituieren des neuronalen Netzes \eqref{neuronal:nn} für \( u \) und anschliessendem Quadrieren, erhält man
\begin{equation}
    \left(\frac{\partial^2 \hat{u}}{\partial t^2} - c^2 \left( \frac{\partial^2 \hat{u}}{\partial x^2} + \frac{\partial^2 \hat{u}}{\partial y^2} \right)\right)^2 = 0.
\end{equation}

Macht man das gleiche mit den Initial- und Randbedingungen, erhält man
\begin{equation}
    \begin{aligned}
        \left(\sin(\pi x) \sin(\pi y) - \hat{u}(x, y, 0)\right)^2 = 0\\
        \left(\frac{\partial \hat{u}(x, y, 0)}{\partial t}\right)^2 = 0
    \end{aligned}
\end{equation}
und
\begin{equation}
    \begin{aligned}
        \left(\hat{u}(-10, y, t)\right)^2 = 0\\
        \left(\hat{u}(10, y, t)\right)^2 = 0\\
        \left(\hat{u}(x, -10, t)\right)^2 = 0\\
        \left(\hat{u}(x, 10, t)\right)^2 = 0.
    \end{aligned}
\end{equation}

Durch das Quadrieren wird erreicht, dass die Terme für alle \( x, y, t \) immer positiv sind.
Somit sind Minima der Terme tiefstens 0 und nicht beliebig klein.
Das Training des neuronalen Netzwerks \( \hat{u} \) lässt sich nun als Optimierungsproblem ausdrücken:

\begin{tcolorbox}[colback=gray!10, colframe=black, boxrule=0.5pt]
Wähle die Parameter \( \vartheta \) des Netzwerks so, dass die obigen Terme für die Wellengleichung und deren Bedingungen minimal (d.h. möglichst nahe bei 0) werden.
\end{tcolorbox}

Um all diese Terme gleichzeitig zu minimieren, werden diese addiert. So bleibt nur ein einziger Term \( L(x, y, t, \vartheta) \) der schlussendlich minimiert werden soll
\begin{equation}
    \begin{aligned}
        L(x, y, t, \vartheta) &= \left(\frac{\partial^2 \hat{u}}{\partial t^2} - c^2 \left( \frac{\partial^2 \hat{u}}{\partial x^2} + \frac{\partial^2 \hat{u}}{\partial y^2} \right)\right)^2\\
        &+ \left(\sin(\pi x) \sin(\pi y) - \hat{u}(x, y, 0)\right)^2
        + \left(\frac{\partial \hat{u}(x, y, 0)}{\partial t}\right)^2\\
        &+ \left(\hat{u}(-10, y, t)\right)^2
        + \left(\hat{u}(10, y, t)\right)^2
        + \left(\hat{u}(x, -10, t)\right)^2
        + \left(\hat{u}(x, 10, t)\right)^2.
    \end{aligned}
    \label{neuronal:optimierung}
\end{equation}

Zu beachten ist, dass \( L \) für alle \( x, y, t \) minimal bzw. möglichst nahe bei 0 sein soll.
D.h. \( L \) soll nur in \( \vartheta \) minimiert werden.
Wie man dies erreicht, wird im Kapitel \ref{neuronal:subsection:training_nn} beschrieben.

\subsection{Struktur des neuronalen Netzes}\label{neuronal:subsection:struktur_nn}

Im vorangegangenen Kapitel wurde nicht weiter auf die Struktur des neuronalen Netzes \eqref{neuronal:nn} eingegangen.
Diese wird innerhalb dieses Kapitels entwickelt.

Generell können neuronale Netze als Kompositionen von mehreren Teilfunktionen betrachtet werden.
Dabei besteht jede Teilfunktion \( f_i \) aus einer Affintransformation, gefolgt von einer nicht-linearen Aktivierungsfunktion \( g_i \)

\begin{align*}
    f_i\colon \mathbb{R}^q & \longrightarrow\mathbb{R}^p \\[-1ex]
    v & \longmapsto g_i(A_iv + b_i)
\end{align*}

mit \( v \in \mathbb{R}^q, A_i \in \mathbb{R}^{p \times q}, b_i \in \mathbb{R}^p \). 
Die Elemente aller \( A_i \) und \( b_i \) bilden den Vektor \( \vartheta \) der \emph{trainierbaren Parameter}.

Weiter ist die Aktivierungsfunktion \( g_i\colon \mathbb{R} \longrightarrow\mathbb{R} \), da das Resultat der Affintransformation ein Vektor ist, wird \( g_i \) einzeln auf jeden Vektor-Komponenten angewendet.
Für \( g_i \) gibt es viele verschiedene Möglichkeiten, eine häufig verwendete Aktivierungsfunktion ist die hyperbolische Tangens \( \tanh() \).

Das gesamte neuronale Netzwerk ist somit gegeben als
\begin{equation}
    \hat{u}(x, y, t; \vartheta) = f_k(\ldots(f_i(\ldots(f_1(x, y, t))))) = f_k \circ \ldots \circ f_i \ldots \circ f_1(x, y, t).
    \label{neuronal:nn_ausformuliert}
\end{equation}

Die Domänen und Bereiche der Teilfunktionen \( f_i \) sind mehr oder weniger frei wählbar.
Damit es mit den Dimensionen aufgeht, muss jeweils der Bereich von \( f_i \) gleich der Domäne von \( f_{i+1} \) sein.
Ebenfalls ist man relativ frei in der Wahl der Anzahl Teilfunktion.

Die einzigen strikten Vorgaben sind die Domäne der ersten Teilfunktion \( f_1 \) und der Bereich der letzten Teilfunktion \( f_k \).
Da das Netzwerk die Wellenfunktion
\begin{align*}
    u\colon \mathbb{R}^3 & \longrightarrow\mathbb{R}
\end{align*}

in zwei räumlichen und der zeitlichen Dimension approximieren soll, muss die Domäne von \( f_1, \mathbb{R}^3\) sein und der Bereich von \( f_k, \mathbb{R}\).

Je mehr Teilfunktionen und Anzahl Dimensionen in den Teilfunktionen, desto mehr trainierbare Parameter besitzt das Modell.
Auf die Frage nach der Anzahl Parameter gibt es keine ``richtige'' Antwort.
Als Faustregel gilt: Je komplizierter die Funktion die approximiert wird, desto mehr Parameter werden benötigt.

Die konkrete Struktur vom neuronalen Netzwerk für die Lösung der Wellengleichung wird im Kapitel \ref{neuronal:section:rechenbeispiel} beschrieben.


\subsection{Training des neuronalen Netzes}\label{neuronal:subsection:training_nn}

Nachdem nun das neuronale Netzwerk als Funktion definiert ist, ist der nächste Schritt die Wahl vom optimalen \( \vartheta \) bzw. die Wahl der optimalen Parametern.
Dies wird als Training des Netzes bezeichnet.

Wie im Kapitel \ref{neuronal:subsection:optimierungsproblem} beschrieben, soll das Netz, bzw. die gewählten Parameter, \( L(x, y, t, \vartheta) \) \eqref{neuronal:optimierung} minimieren.
Hierzu werden drei Dinge benötigt:
\begin{itemize}
    \item Ein Trainings-Datensatz an x-, y- und t-Werten
    \item Eine Loss-Funktion
    \item Einen Optimierungsalgorithmus
\end{itemize}

\paragraph{Trainings-Datensatz}

Aktuell hängt \( L(x, y, t, \vartheta) \) sowohl von den Parametern des Netzes \( \vartheta \), als auch den drei Variablen \( x, y, t \) ab.
Wie bereits erwähnt, soll aber nicht in den drei Variablen minimiert werden, sondern in \( \vartheta \).
Daher müssen die Variablen durch konkrete Werte ersetzt werden.

Dies wird mit einem Trainings-Datensatz gemacht, welcher aus x-, y- und t-Werten besteht, für die die Wellengleichung und deren Bedingungen erfüllt sind.
Gem. Kapitel \ref{neuronal:section:herleitung} sollen x und y in [-10,10] sein und t in [0,10].
Innerhalb dieser Bereiche werden gleichverteilte, reelle Zahlen generiert.
Somit ist der Datensatz
\begin{center}
    \( x_1, x_2, \ldots, x_k \in [-10,10] \)\\
    \( y_1, y_2, \ldots, y_k \in [-10,10] \)\\
    \( t_1, t_2, \ldots, t_k \in [0,10] \)\\
\end{center}

Die Anzahl Datenpunkte hängt davon ab, wieviele Parameter das neuronale Netzwerk besitzt.
Je mehr Parameter, desto mehr Datenpunkte werden benötigt.

Wie viele Datenpunkte konkret für die Lösung der Wellengleichung verwendet werden, wird im Kapitel \ref{neuronal:section:rechenbeispiel} beschrieben.

\paragraph{Loss-Funktion}

Die Loss-Funktion ist die Summe von \( L(x, y, t, \vartheta) \), über alle Datensatz-Werte und hängt somit nur noch von \( \vartheta \) ab.
Um das Netzwerk zu trainieren, kann nun diese Funktion minimiert werden

\begin{equation}
    J(\vartheta) = \sum_{i=1}^{k} L(x_i, y_i, t_i, \vartheta).
    \label{neuronal:loss}
\end{equation}

Ausgewertet an einem bestimmten \( \vartheta \) liefert sie eine reelle Zahl als Resultat.
Diese reelle Zahl ist ein Mass dafür, wie gut das neuronale Netzwerk, mit den verwendeten Parametern, die Wellengleichung an den x-, y- und t-Werten aus dem Trainings-Datensatzes approximiert.
Anders ausgedrückt ist die Loss-Funktion \( J \) ein Mass für den Fehler, den das Netzwerk macht.

Durch das Minimieren von \( J \) wird das Netzwerk nur darauf trainiert, die Wellengleichung an den Punkten aus dem Trainings-Datensatz zu lösen.
Warum das Netzwerk aber trotzdem auch an Datenpunkten funktioniert, welche nicht im Trainings-Datensatz waren, wird im Kapitel \ref{neuronal:subsection:qualität_nn} beschrieben.

\paragraph{Optimierungsalgorithmus}

Der Optimierungsalgorithmus hat das Ziel, die Loss-Funktion \( J \) zu minimieren.
Da \( J \) ein hochdimensionales Hyper-Surface ist, ist es sehr unwahrscheinlich dass man analytisch ein Minimum findet.
Das bedeutet, dass stattdessen ein numerischer Algorithmus verwendet werden muss.

\begin{tcolorbox}[colback=gray!10, colframe=black, boxrule=0.5pt]
    Die Schritte die der Algorithmus durchläuft sind:
    \begin{enumerate}
        \item Initialisiere \( \vartheta_i \) mit Anfangswerten.
        \item \textbf{Loop} von \( i = 1 \) bis \( i = m - 1 \):
        \begin{enumerate}
            \item Berechne neue Parameterwerte: \( \vartheta_{i+1} = \vartheta_i - \epsilon \nabla_\vartheta J\left(\vartheta_i\right) \). (Erklärung unten)
        \end{enumerate}
        \item Gebe die Parameter \( \vartheta_m \) zurück.
    \end{enumerate}
\end{tcolorbox}

Die Anzahl Schleifendurchläufe bzw. \( m \) wird so gewählt, dass \( J(\vartheta_m) \) genügend nahe bei 0 ist.
Nach dem letzten Schritt dieses Algorithmus ist das neuronale Netzwerk mit den Parametern \( \vartheta_m \) fertig trainiert.

\textbf{Als abschliessende Notiz:} Im Schritt 2 des Algorithmus werden die Parameterwerte neu berechnet.
Durch die verwendete Formel wird erreicht, dass \( J(\vartheta_{i+1}) \leq J(\vartheta_i) \) ist, d.h. der Fehler des Netzwerks nimmt mit jedem Schritt ab.
Dies ist das Optimierungsverfahren \emph{Gradient Descent}.

Das funktioniert, da der Gradient von \( J \) ausgewertet an \( \vartheta_i \) ein Vektor ist, der in die Richtung des stärksten Anstiegs auf \( J \), von \(\vartheta_i \) aus, weist.
Durch das Minus in der Formel geht man in die entgegen gesetzte Richtung, wo es auf \( J \) abwärts geht. 
Mit \( \epsilon \) wird die ``Schrittgrösse'' gesteuert, um zu verhindern dass man über ein Minimum auf \( J \) ``springt''.


\subsection{Qualität der Approximation}\label{neuronal:subsection:qualität_nn}

\begin{itemize}
    \item Wie gut ist das Modell?
    \item Wieso funktioniert das Modell auch an Datenpunkten welche nicht im Trainings-Datensatz waren?
    \item Wie bewertet man die Qualität?
\end{itemize}
