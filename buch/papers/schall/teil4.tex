\section{Überschallknall: Entstehung und Low-Boom-Strategien}
\label{schall:section:boom}

Ein Flugzeug mit Machzahl $Ma=V/c_0>1$ erzeugt entlang seiner Länge diskrete
Schockbeiträge an Bug, Tragflächen, Rumpfänderungen, Leitwerke und Auspuff,
die sich bei der Ausbreitung in der Atmosphäre überlagern und am Boden als impulsförmige
Signatur (\emph{N-Welle}) wahrgenommen werden.
Die \emph{Nahfeld}-Geometrie, definiert durch die Flugzeugform sowie die  Last- oder
Auftriebsverteilung, bestimmt die anfängliche Drucksignatur;
die \emph{Fernfeld}-Ausbreitung formt diese durch Nichtlinearität und atmosphärische Dämpfung.

\subsection{Nahfeld: F-Funktion, Äquivalentfläche und Initialsignatur}
Für schlanke Überschallkörper liefert Whithams Linear-Theorie eine Kopplung
zwischen der Rumpf- oder Flügel-Geometrie (Äquivalentfläche $A(x)$) und der
Nahfeldsignatur über die sogenannte \emph{F-Funktion}.
Whithams modifizierte Lineartheorie definiert die zugehörige \emph{F-Funktion}
als Abel-Transformierte von $A_e''$:
\begin{equation}
    F(x_e)=\frac{1}{2\pi}\int_{0}^{x_e}\frac{A_e''(t)}{\sqrt{x_e-t}}\,dt,
\end{equation}
wobei $x_e$ eine entlang der Flugrichtung skalierte Koordinate ist.
Die dimensionslose Nahfeld-Überdrucksignatur ist, bis auf bekannte
Mach-/Gasfaktoren, proportional zu $F(x_e)$.
Glatte, lobe-arme $F$-Verläufe führen zu \emph{Low-Boom}-Signaturen.
Intuitiv gilt: stark gekrümmte Änderungen in $A'(x)$ erzeugen starke lokale
Schockanteile; eine „sanft“ verteilte Geometrie verteilt die Druckanstiege
längs der Rumpflänge.
Die klassische Low-Boom-Minimierung nach Seebass–George–Darden entwirft
Zielverteilungen für $A(x)$, die eine geringe Schockstärke am Eintritt in
die Atmosphäre bewirken \cite{schall:whitham,schall:seebassgeorge,schall:darden75}.

\subsection{Ausbreitung: (Augmentierte) Burgers-Gleichung}
Bei der Fernfeld-Übertragung durch eine ruhende, schichtweise homogene
Atmosphäre (ohne Wind) wird die skalierte Schaltdrucksignatur $p(\tau,x)$
(hier $\tau$ = „retardierte Zeit“) häufig mit einer \emph{augmentierten Burgers-Gleichung}
modelliert:
\begin{equation}
  \frac{\partial p}{\partial x}
  \;=\;
  -\,\frac{m}{x}\,p
  \;-\;\frac{\beta}{\rho_0 c_0^{3}}\,p\,\frac{\partial p}{\partial \tau}
  \;+\;\delta\,\frac{\partial^2 p}{\partial \tau^2},
  \label{eq:aug-burgers}
\end{equation}
wobei $\frac{m}{x}$ geometrische Abschwächung beschreibt, der nichtlineare
Term, welcher mit $\beta$ multipliziert wird, zu Schocksteilen führt und
$\delta$ frequenzabhängige atmosphärische Verluste aggregiert.
Grobe Faustregel: Nichtlinearität „zieht“ die Front zu einem N-Profil;
Dämpfung rundet Spitzen ab und reduziert den Spitzenüberdruck
$\Delta p_\mathrm{max}$\cite{schall:rallabhandi2023,schall:rallabhandiAIAA2023,schall:burgersJASA}.

\subsection{Boden-Signatur und Metriken}
Die für Wahrnehmung maßgebliche Metrik ist häufig der \emph{Perceived Level},
der die zeitliche Impulsform und psychoakustische Bewertungen kombiniert.
Ziel moderner Low-Boom-Demonstratoren ist ein \emph{„sonic thump“} statt
eines klassischen Booms, mit einem Zielwerte um
${\sim}75$\ PLdB in Standardatmosphäre \cite{schall:x59pldb}.

\subsection{Hebel zur Boom-Reduktion}
\paragraph{Aerodynamische Formgebung (Low-Boom Shaping).}
Gestaltung von $A(x)$ und der Auftriebsverteilung entlang der Länge,
um große Einzel-Schocks zu vermeiden und verteilte, kleinere Sprünge zu generieren.
Dies kann durch eine lange, schlanke Nase, durch Flügel- oder Canard-Anordnung,
verdeckte Triebwerksintegration oder durch Vermeidung starker
Querschnittssprünge realisiert werden.
Formal: Minimierung geeigneter Funktionale der F-Funktion bzw.\ der Nahfeldsignatur
gemäß Seebass–George–Darden-Ideen \cite{schall:seebassgeorge,schall:darden75}.

\paragraph{Flughöhe und Machzahl}
Höhere Flughöhe erhöht die geometrische Abschwächung, nachvollziehbar durch
größeres $x$ in \eqref{eq:aug-burgers}, und die atmosphärische Dämpfung durch
größere effektive $\delta$; moderate Machzahl reduziert die erzeugten
Schockstärken bereits im Nahfeld.

\paragraph{Flugweg- und Operationsführung}
Überland Flug mit Unterschall, Kurswahl, Steigprofile, und Wetter-Routing
durch Vermeidung fokussierender Schichtungen, können die
Boom-Teppich-Geometrie signifikant beeinflussen.

\paragraph{Atmosphärische Effekte berücksichtigen.}
Temperatur- und Wind-Schichtung $c(z)$, Windschubvektoren und Turbulenz
modulieren Divergenz und effektive Dämpfung in \eqref{eq:aug-burgers};
in inversen Lagen kann es zu \emph{Fokussierung} kommen (lokal erhöhte
$\Delta p$), in Normal-Lapsen eher zu \emph{Defokussierung}.