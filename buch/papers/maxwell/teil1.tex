%
% teil1.tex -- Beispiel-File für das Paper
%
% (c) 2020 Prof Dr Andreas Müller, Hochschule Rapperswil
%
% !TEX root = ../../buch.tex
% !TEX encoding = UTF-8
%
\section{Hodge Theorie provisorisch
\label{maxwell:section:teil1}}
\kopfrechts{Hodge Theorie provisorisch}
Für den Teil der Inhomogenen Maxwellgleichungen benötigen wir die Hodge-Duale auf 1-Formen, 2-Formen und 3-Formen im vierdimensionalen Minkowski-Raum.
Um dabei die korrekten Vorzeichen zu erhalten, muss die Hodge-Dualität mit Hilfe des metrischen Tensors $g^{ik}$  der Minkowski-Metrik verwendet werden.



\subsection{Minkowski Metrik}
In der Speziellen Relativitätstheorie (SRT) wird die Minkowski-Metrik verwendet.
Da es in der SRT keine Krümmung und Gravitation gibt, sind alle Elemente ausserhalb der Diagonale des Metrischen Tensors null und somit ist die Raum-Zeit flach.
Zwei Signaturen sind üblich.
Einerseits gibt es die (-+++)-Signatur, bei welcher die Zeitkomponente negativ und die Raumkomponenten positiv gezählt werden.
Andererseits gibt es die (+ - - -)-Signatur, bei welcher die Zeitkomponente positiv und die Raumkomponenten negativ gezählt werden.
Beide Signaturen sind gleichwertig, solange man sich auf eine Metrik festlegt und diese konsequent beibehält.
Im folgenden werden wir uns an die (-+++)-Signatur halten.
Daher definieren wir den Metrischen Tensor als
\begin{equation}
	g^{ik} = \begin{pmatrix}
		-1 & 0 & 0 & 0 \\ 0 & 1 & 0 & 0 \\ 0 & 0 & 1 & 0 \\ 0 & 0 & 0 & 1 
	\end{pmatrix}.
	\label{maxwell:section:teil1:metrik}
\end{equation}
Der Ausdruck für ein Linienelement in dieser Metrik ist definiert als
\begin{equation}
	dl^2 = -c^2dt^2 +dx^2+dy^2+dz^2.
\end{equation}
Eine Konsequenz dieser Signatur ist, dass Zeitartige Abstände $ds^2 < 0$ und raumartige Abstände $ds^2 > 0$ sind.


\subsection{Herleitung Hodgeduale(verweis auf Abschnitt im Buch)}
Folgend alle Berechungen der Hodgeduale auf 1-/2- und 3-Formen mit $(-+++)$ -Metrik.
Es gilt für $g^{ik}$ \eqref{maxwell:section:teil1:metrik} und für $ vol(M) = dx^0 \wedge dx^1 \wedge dx^2 \wedge dx^3$.

\subsubsection{Auf 1-Formen}
\begin{align*}
	\star dx^0 
	&=
	s \, dx^1 \wedge dx^2 \wedge dx^3
	\\
	dx^0 \wedge \star dx^0 
	&=
	dx^0 \wedge s \, dx^1 \wedge dx^2 \wedge dx^3 
	\\
	&=
	s \, dx^0 \wedge dx^1 \wedge dx^2 \wedge dx^3
	\\
	&=
	\underbrace{\langle dx^0, dx^0 \rangle}_{g^{00}} \, vol(M) 
	\\
	&= - dx^0 \wedge dx^1 \wedge dx^2 \wedge dx^3 \rightarrow s = -1
	\\
	\Rightarrow \star dx^0 
	&=
	- dx^1 \wedge dx^2 \wedge dx^3
\\
\\
	\star dx^1 
	&=
	s \, dx^0 \wedge dx^2 \wedge dx^3
	\\
	dx^1 \wedge \star dx^1 
	&=
	dx^1 \wedge s \, dx^0 \wedge dx^2 \wedge dx^3 
	\\
	&=
	-s \, dx^0 \wedge dx^1 \wedge dx^2 \wedge dx^3
	\\
	&=
	\underbrace{\langle dx^1, dx^1 \rangle}_{g^{11}} \, vol(M) 
	\\
	&=
	- dx^0 \wedge dx^1 \wedge dx^2 \wedge dx^3 \rightarrow s = -1
	\\
	\Rightarrow \star dx^1 
	&=
	- dx^0 \wedge dx^2 \wedge dx^3
\\
\\
	\star dx^2 
	&=
	s \, dx^0 \wedge dx^1 \wedge dx^3
	\\
	dx^2 \wedge \star dx^2 
	&=
	dx^2 \wedge s \, dx^0 \wedge dx^1 \wedge dx^3 
	\\
	&=
	s \, dx^0 \wedge dx^1 \wedge dx^2 \wedge dx^3
	\\
	&=
	\underbrace{\langle dx^2, dx^2 \rangle}_{g^{22}} \, vol(M) 
	\\
	&=
	dx^0 \wedge dx^1 \wedge dx^2 \wedge dx^3 \rightarrow s = 1
	\\
	\Rightarrow \star dx^2 
	&=
	dx^0 \wedge dx^1 \wedge dx^3
\\
\\
	\star dx^3 
	&=
	s \, dx^0 \wedge dx^1 \wedge dx^2
	\\
	dx^3 \wedge \star dx^3 
	&=
	dx^3 \wedge s \, dx^0 \wedge dx^1 \wedge dx^2 
	\\
	&= -s \, dx^0 \wedge dx^1 \wedge dx^2 \wedge dx^3
	\\
	&=
	\underbrace{\langle dx^3, dx^3 \rangle}_{g^{33}} \, vol(M) 
	\\
	&=
	dx^0 \wedge dx^1 \wedge dx^2 \wedge dx^3 \rightarrow s = -1
	\\
	\Rightarrow \star dx^3 
	&=
	-dx^0 \wedge dx^1 \wedge dx^2
\end{align*}

\subsubsection{Auf 2-Formen}
\begin{align*}
	\star \underbrace{(dx^0 \wedge dx^1)}_{\omega}
	&=
	s \, dx^2 \wedge dx^3
	\\
	\omega \wedge \star \omega
	&=
	dx^0 \wedge dx^1 \wedge s \, dx^2 \wedge dx^3
	\\
	&=
	s \, dx^0 \wedge dx^1 \wedge dx^2 \wedge dx^3
	\\
	\omega \wedge \star \omega
	&=
	\langle \omega, \omega \rangle \, vol(M)
	\\
	&=
	\langle dx^0 \wedge dx^1, dx^0 \wedge dx^1 \rangle \, vol(M)
	\\
	&=
	g^{00} \cdot g^{11} \cdot vol(M)
	\\
	&=
	-1 \cdot 1 \cdot vol(M)
	\\
	&=
	-dx^0 \wedge dx^1 \wedge dx^2 \wedge dx^3 \rightarrow s = -1
	\\
	\Rightarrow \star(dx^0 \wedge dx^1)
	&=
	-dx^2 \wedge dx^3
\\
\\
	\star \underbrace{(dx^1 \wedge dx^2)}_{\omega}
	&=
	s \, dx^0 \wedge dx^3
	\\
	\omega \wedge \star \omega
	&=
	dx^1 \wedge dx^2 \wedge s \, dx^0 \wedge dx^3
	\\
	&=
	s \, dx^0 \wedge dx^1 \wedge dx^2 \wedge dx^3
	\\
	\omega \wedge \star \omega
	&=
	\langle \omega, \omega \rangle \, vol(M)
	\\
	&=
	\langle dx^1 \wedge dx^2, dx^1 \wedge dx^2 \rangle \, vol(M)
	\\
	&=
	g^{11} \cdot g^{22} \cdot vol(M)
	\\
	&=
	1 \cdot 1 \cdot vol(M)
	\\
	&=
	dx^0 \wedge dx^1 \wedge dx^2 \wedge dx^3 \rightarrow s = 1
	\\
	\Rightarrow \star(dx^1 \wedge dx^2)
	&=
	dx^0 \wedge dx^3
\\
\\
	\star \underbrace{(dx^2 \wedge dx^3)}_{\omega}
	&=
	s \, dx^0 \wedge dx^1
	\\
	\omega \wedge \star \omega
	&=
	dx^2 \wedge dx^3 \wedge s \, dx^0 \wedge dx^1
	\\
	&=
	s \, dx^0 \wedge dx^1 \wedge dx^2 \wedge dx^3
	\\
	\omega \wedge \star \omega
	&=
	\langle \omega, \omega \rangle \, vol(M)
	\\
	&=
	\langle dx^2 \wedge dx^3, dx^2 \wedge dx^3 \rangle \, vol(M)
	\\
	&=
	g^{22} \cdot g^{33} \cdot vol(M)
	\\
	&=
	1 \cdot 1 \cdot vol(M)
	\\
	&=
	dx^0 \wedge dx^1 \wedge dx^2 \wedge dx^3 \rightarrow s = 1
	\\
	\Rightarrow \star(dx^2 \wedge dx^3)
	&=
	dx^0 \wedge dx^1
\\
\\
	\star \underbrace{(dx^0 \wedge dx^2)}_{\omega}
	&=
	s \, dx^1 \wedge dx^3
	\\
	\omega \wedge \star \omega
	&=
	dx^0 \wedge dx^2 \wedge s \, dx^1 \wedge dx^3
	\\
	&=
	-s \, dx^0 \wedge dx^1 \wedge dx^2 \wedge dx^3
	\\
	\omega \wedge \star \omega
	&=
	\langle \omega, \omega \rangle \, vol(M)
	\\
	&=
	\langle dx^0 \wedge dx^2, dx^0 \wedge dx^2 \rangle \, vol(M)
	\\
	&=
	g^{00} \cdot g^{22} \cdot vol(M)
	\\
	&=
	-1 \cdot 1 \cdot vol(M)
	\\
	&=
	-dx^0 \wedge dx^1 \wedge dx^2 \wedge dx^3 \rightarrow s = 1
	\\
	\Rightarrow \star(dx^0 \wedge dx^2)
	&=
	dx^1 \wedge dx^3
\\
\\
	\star \underbrace{(dx^0 \wedge dx^3)}_{\omega}
	&=
	s \, dx^1 \wedge dx^2
	\\
	\omega \wedge \star \omega
	&=
	dx^0 \wedge dx^3 \wedge s \, dx^1 \wedge dx^2
	\\
	&=
	s \, dx^0 \wedge dx^1 \wedge dx^2 \wedge dx^3
	\\
	\omega \wedge \star \omega
	&=
	\langle \omega, \omega \rangle \, vol(M)
	\\
	&=
	\langle dx^0 \wedge dx^3, dx^0 \wedge dx^3 \rangle \, vol(M)
	\\
	&=
	g^{00} \cdot g^{33} \cdot vol(M)
	\\
	&=
	-1 \cdot 1 \cdot vol(M)
	\\
	&=
	-dx^0 \wedge dx^1 \wedge dx^2 \wedge dx^3 \rightarrow s = -1
	\\
	\Rightarrow \star(dx^0 \wedge dx^3)
	&=
	-dx^1 \wedge dx^2
\\
\\
	\star \underbrace{(dx^1 \wedge dx^3)}_{\omega}
	&=
	s \, dx^0 \wedge dx^2
	\\
	\omega \wedge \star \omega
	&=
	dx^1 \wedge dx^3 \wedge s \, dx^0 \wedge dx^2
	\\
	&=
	-s \, dx^0 \wedge dx^1 \wedge dx^2 \wedge dx^3
	\\
	\omega \wedge \star \omega
	&=
	\langle \omega, \omega \rangle \, vol(M)
	\\
	&=
	\langle dx^1 \wedge dx^3, dx^1 \wedge dx^3 \rangle \, vol(M)
	\\
	&=
	g^{11} \cdot g^{33} \cdot vol(M)
	\\
	&=
	1 \cdot 1 \cdot vol(M)
	\\
	&=
	dx^0 \wedge dx^1 \wedge dx^2 \wedge dx^3 \rightarrow s = -1
	\\
	\Rightarrow \star(dx^1 \wedge dx^3)
	&=
	-dx^0 \wedge dx^2
\end{align*}
\subsubsection{Auf 3-Formen}
\begin{align*}
	\star \underbrace{dx^0 \wedge dx^1 \wedge dx^2}_{\omega}
	&=
	s \, dx^3
	\\
	\omega \wedge \star \omega 
	&=
	dx^0 \wedge dx^1 \wedge dx^2 \wedge s \, dx^3
	\\
	&=
	s \, dx^0 \wedge dx^1 \wedge dx^2 \wedge dx^3
	\\
	\omega \wedge \star \omega
	&=
	\langle dx^0 \wedge dx^1 \wedge dx^2 , dx^0 \wedge dx^1 \wedge dx^2 \rangle \, vol(M)
	\\
	&=
	g^{00} \cdot g^{11} \cdot g^{22} \cdot vol(M)
	\\
	&=
	-1 \cdot 1 \cdot 1 \cdot vol(M) \rightarrow s = -1
	\\
	\Rightarrow \star (dx^0 \wedge dx^1 \wedge dx^2) 
	&= - dx^3
\\
\\
	\star \underbrace{dx^0 \wedge dx^1 \wedge dx^3}_{\omega}
	&=
	s \, dx^2
	\\
	\omega \wedge \star \omega 
	&=
	dx^0 \wedge dx^1 \wedge dx^3 \wedge s \, dx^2
	\\
	&=
	-s \, dx^0 \wedge dx^1 \wedge dx^2 \wedge dx^3
	\\
	\omega \wedge \star \omega
	&=
	\langle dx^0 \wedge dx^1 \wedge dx^3 , dx^0 \wedge dx^1 \wedge dx^3 \rangle \, vol(M)
	\\
	&=
	g^{00} \cdot g^{11} \cdot g^{33} \cdot vol(M)
	\\
	&=
	-1 \cdot 1 \cdot 1 \cdot vol(M) \rightarrow s = 1
	\\
	\Rightarrow \star (dx^0 \wedge dx^1 \wedge dx^3) 
	&= dx^2
\\
\\
	\star \underbrace{dx^0 \wedge dx^2 \wedge dx^23}_{\omega}
	&=
	s \, dx^1
	\\
	\omega \wedge \star \omega 
	&=
	dx^0 \wedge dx^2 \wedge dx^3 \wedge s \, dx^1
	\\
	&=
	s \, dx^0 \wedge dx^1 \wedge dx^2 \wedge dx^3
	\\
	\omega \wedge \star \omega
	&=
	\langle dx^0 \wedge dx^2 \wedge dx^3 , dx^0 \wedge dx^2 \wedge dx^3 \rangle \, vol(M)
	\\
	&=
	g^{00} \cdot g^{22} \cdot g^{33} \cdot vol(M)
	\\
	&=
	-1 \cdot 1 \cdot 1 \cdot vol(M) \rightarrow s = -1
	\\
	\Rightarrow \star (dx^0 \wedge dx^2 \wedge dx^3) 
	&= - dx^1
\\
\\
	\star \underbrace{dx^1 \wedge dx^2 \wedge dx^3}_{\omega}
	&=
	s \, dx^0
	\\
	\omega \wedge \star \omega 
	&=
	dx^1 \wedge dx^2 \wedge dx^3 \wedge s \, dx^0
	\\
	&=
	-s \, dx^0 \wedge dx^1 \wedge dx^2 \wedge dx^3
	\\
	\omega \wedge \star \omega
	&=
	\langle dx^1 \wedge dx^2 \wedge dx^3 , dx^1 \wedge dx^2 \wedge dx^3 \rangle \, vol(M)
	\\
	&=
	g^{11} \cdot g^{22} \cdot g^{33} \cdot vol(M)
	\\
	&=
	1 \cdot 1 \cdot 1 \cdot vol(M) \rightarrow s = -1
	\\
	\Rightarrow \star (dx^1 \wedge dx^2 \wedge dx^3) 
	&= - dx^0
\end{align*}





\subsection{De finibus bonorum et malorum
\label{maxwell:subsection:finibus}}
At vero eos et accusamus et iusto odio dignissimos ducimus qui
blanditiis praesentium voluptatum deleniti atque corrupti quos
dolores et quas molestias excepturi sint occaecati cupiditate non
provident, similique sunt in culpa qui officia deserunt mollitia
animi, id est laborum et dolorum fuga \eqref{maxwell:equation1}.

Et harum quidem rerum facilis est et expedita distinctio
\ref{maxwell:section:teil2}.
Nam libero tempore, cum soluta nobis est eligendi optio cumque nihil
impedit quo minus id quod maxime placeat facere possimus, omnis
voluptas assumenda est, omnis dolor repellendus
\ref{maxwell:section:teil3}.
Temporibus autem quibusdam et aut officiis debitis aut rerum
necessitatibus saepe eveniet ut et voluptates repudiandae sint et
molestiae non recusandae.
Itaque earum rerum hic tenetur a sapiente delectus, ut aut reiciendis
voluptatibus maiores alias consequatur aut perferendis doloribus
asperiores repellat.


