%
% einleitung.tex -- Beispiel-File für die Einleitung
%
% (c) 2020 Prof Dr Andreas Müller, Hochschule Rapperswil
%
% !TEX root = ../../buch.tex
% !TEX encoding = UTF-8
%
\section{Einleitung\label{geoalgebra:section:einfuehrung}}
\kopfrechts{Einleitung}
Die geometrische Algebra, auch als Clifford-Algebra bekannt, bietet ein einheitliches mathematisches
Framework für das Rechnen mit Vektoren.
Neben den herkömmlichen Vektoren definiert die geometrische
Algebra neue mathematische Objekte, wie den \emph{Bivektor}, den \emph{Trivektor}, allgemein gesprochen
\emph{$n$-Vektoren}.
Der uns bereits bekannte normale Vektor bildet dabei den 1-Vektor. Diese neuen $n$-Vektoren werden auch
\emph{Multivektoren} genannt.

\subsection{Entstehung}
Das geometrische Produkt wurde erstmals durch Hermann Grassmann in seinem Meisterwerk ``A1'' \cite{geoalgebra:grassmann1844lineale} in 1844
erwähnt. Im Jahr 1878 erweiterte der britische Mathematiker William Kingdon Clifford Grassmanns Idee
in die allgemeine geometrische Algebra. 

