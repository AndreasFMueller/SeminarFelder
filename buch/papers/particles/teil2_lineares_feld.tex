%
% teil1.tex -- Beispiel-File für das Paper
%
% (c) 2020 Prof Dr Andreas Müller, Hochschule Rapperswil
%
% !TEX root = ../../buch.tex
% !TEX encoding = UTF-8
%
\section{Lineares Medium\label{particles:section:linear}}
\kopfrechts{Lineares Medium}
% TODO: Quellenangaben
% [ ]: https://en.wikipedia.org/wiki/Linearity
% [ ]: Lineare Algebra: Eine anwendungsorientierte Einführung, Seite: 27, ISBN: 978-3-662-67865-7, Published: 01 September 2023, DOI: https://doi.org/10.1007/978-3-662-67866-4
% \subsection{Linearität} % TODO: Allenfalls folgendes als subsection "Linearität" schreiben.
Für den einfachsten und üblichsten Fall, nimmt man oft ein \emph{lineares Medium} an.
Solch ein Medium nennt man \emph{linear}, wenn dessen Definition sowohl \emph{additiv}
\[
    f(x_{1} + y_{1}, \ldots, x_{n} + y_{n}) 
    = f(x_{1}, \ldots, x_{n}) + f(y_{1}, \ldots, y_{n}),
\]
als auch \emph{homogen}
\[
    f(\lambda x_{1}, \ldots, \lambda x_{n}) 
    = \lambda f(x_{1}, \ldots, x_{n})
\]
ist.
Hierbei ist angemerkt, dass $x_{k}$, $y_{k}$ und $\lambda$ nicht rein reell sein müssen, 
sondern einem beliebigen Vektorraum angehören können. % Alternativ: "[...] angehörig sein können."
Diese Definition ist auch bekannt unter dem Term \emph{Superpositionsprinzip}, 
worauf im nächsten Unterabschnitt noch weiter darauf eingegangen wird. % TODO: Etwas umschreiben, dass es schöner in den Fluss des Textes passt.
Im Kontext dieses Papers wird besonders die Elastizität eines Mediums betrachtet, 
welche durch
\[
    D
    = \frac{\varepsilon}{E}
    \quad
    (E = \text{const.})\label{particles:eq:lin-elastizitaet}
\]
gegeben ist. 
$D$ beschreibt hier die \emph{Deformierung}, % NOTE: oder Deformation?
$\varepsilon$ den mechanischen Stress oder Druck und
$E$ eine Elastizitätskonstante.


\subsection{Superpositionsprinzip}
Das Superpositionsprinzip fasst die Bedingungen zur Linearität in eine Formel zusammen, nämlich
\[
    T(\lambda x + \mu y)
    = \lambda T(x) + \mu T(y).
\]
Drückt man die Deformierung $D$ aus Formel~\ref{particles:eq:lin-elastizitaet} als eine Funktion von $\varepsilon$ aus
und stellt man die Formel ein wenig um zu 
\[
    D(\varepsilon)
    = \frac{1}{E} \cdot \varepsilon,
\]
so kann man daraus folgern, dass es sich hierbei ebenfalls um eine lineare Funktion handelt, 
sofern $E = \text{const.}$ eingehalten wird.
Folglich ist das angenommene Medium also linear und erfüllt somit auch die Superpositionsbedingungen.
Dies vereinfacht die Berechnung des Feldes enorm, 
da nicht für jedes einzelne $\varepsilon$ ein separates $E$ berechnet werden muss. 

% TODO: Folgendes irgendwo besser einbringen:
\subsection{Feldgleichung}
Da die Elastizitätsgleichung allgemeingültig geschrieben ist, % NOTE: Stimmt das überhaupt?
wird hier noch die Feldgleichung für ein zweidimensionales, diskretes Feld hergeleitet,
welches so in der Simulation verwendet wird.
% TODO: Herleitung Feldgleichung aus Elastizitätsgleichung etc.
% In der Elastizitätsgleichung ist die Feldgleichung zwar gewissermassen enthalten,
% jedoch ist sie nicht direkt ersichtlich. 


% Solch ein Medium nennt man \emph{linear}, wenn dessen Definition die folgenden Anforderungen zur Linearität erfüllen.
% \[
%     \left.\begin{array}{rl}
%         \text{Additivität: } 
%         &   f(x_{1} + y_{1}, \ldots, x_{n} + y_{n}) 
%         = f(x_{1}, \ldots, x_{n}) + f(y_{1}, \ldots, y_{n})\\ 
%         \text{Homogenität: } 
%         &   f(\lambda x_{1}, \ldots, \lambda x_{n}) 
%         = \lambda f(x_{1}, \ldots, x_{n})
%     \end{array}\right\rbrace
%     \quad 
%     \forall x_{k}, y_{k}, \lambda \in \mathbb{k} % TODO: k hier als Platzhalter für eine beliebige Zahl in einem beliebigen Vektorraum. Bitte anpassen, falls es ein "Standard" für dieses Buch gibt.
% \]



