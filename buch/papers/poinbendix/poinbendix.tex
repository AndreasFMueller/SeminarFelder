\section{Der Satz von Poincaré-Bendixson} \label{poinbendix:section:poinbendix}

\begin{satz}[Poincaré-Bendixson]
\label{poinbendix:satz:poinbendix}
$\Phi_t(p) \in \Xi^r(\mathbb{S}^2)$ sei ein $r$-Fach differenzierbares, zweidimensionales dynamisches System mit Startpunkt $p \in \mathbb{S}^2$.
Dann gilt für das Omega-Limit Set $\omega(p)$ eine der folgenden Optionen:
\begin{enumerate}
\item $\omega(p)$ ist eine Nullstelle
\item $\omega(p)$ ist ein geschlossener Orbit
\item $\omega(p)$ ist ein geschlossener Orbit welcher Singularitäten verbindet
\end{enumerate}
\end{satz}

Dies gilt auf der Kugeloberfläche $\mathbb{S}^2$.\cite{poinbendix:melo}
Bis anhin haben wir aber mehrheitlich über planare Fälle gesprochen, weshalb der nächste Abschnitt darauf eingeht wieso wir hier die Kugeloberfläche nehmen.
In den darauffolgenden Abschnitten wird für dann jeden Fall jeweils ein Beispiel gezeigt.

\subsection{Kugeloberfläche und Ebene} \label{poinbendix:subsection:kugeloberflaeche}

Im Beispiel \ref{buch:koordinaten:diffmannig:beispiel:stereographisch} wird die stereographische Projektion eingeführt, welche ein planares Polarkoordinatensystem auf die Kugeloberfläche projiziert.
In Abbildung \ref{buch:koordinaten:diffmannig:fig:stereographisch} sieht man, dass dabei der Punkt auf dem Nordpol der Kugeloberfläche ins unendliche auf der Ebene abgebildet wird.
Somit ergibt sich für den Fall einer Nullstelle auf dem Nordpol eine divergierende Bahnkurve auf der Ebene.

Auch wenn die Formulierung auf der Kugeloberfläche etwas eleganter ist, gilt der Satz von Poincaré-Bendixson sowohl auf der Ebene $\mathbb{R}^2$ wie auch auf der Kugeloberfläche $\mathbb{S}^2$.
Allerdings gilt er nicht auf allen zweidimensionalen Abbildungen, so können zum Beispiel auf dem Torus Bahnkurven auftreten welche nicht wirklich periodisch sind, aber auch nicht auf eine Singularität fallen.\cite{poinbendix:wiki}


\subsection{Fall 1: $\omega(p)$ ist eine Nullstelle} \label{poinbendix:subsection:fall1}

Der erste Fall ist intuitiv am einfachsten zu verstehen.
Sobald auf einer Bahnkurve eine Nullstelle auftritt, bleibt die Kurve für alle Zeiten stehen.
Die eindimensionale Version von Fall 1  wurde bereits im Beispiel \ref{poinbendix:beispiel:1dlimesmengen} über die Limesmengen gezeigt.

\begin{beispiel} \label{poinbendix:beispiel:fall1}
Wir betrachten das folgende Differentialgleichungssystem:
\begin{align*}
    \dot{x} &= -x + y(1-x^2-y^2) \\
    \dot{y} &= -y - x(1-x^2-y^2).
\end{align*}
Nullstellen sind Punkte wo beide Ableitungen Null sind.
Zunächst substituieren wir den Klammerausdruck durch $a = (1-x^2-y^2)$ und setzen die Ableitung nach $x$ gleich Null
\begin{align*}
    -x + ya &= 0 \\
    ya &= x.
\end{align*}
Nun setzen wir dieses Resultat in die Ableitung nach $y$ und setzen es gleich Null
\begin{align*}
    -y - ya^2 &= 0 \\
    y(a^2+1) &= 0.
\end{align*}
Nach dieser Untersuchung sehen wir, dass es nur eine reelle Nullstelle $x=y=0$ geben kann, da $a^2 + 1$ im reellen nie Null werden kann.

Dieses Beispiel wurde gewählt da sehr schöne Bahnkurven entstehen.
Wie diese Bahnkurven aussehen können sieht man in Abbildung \ref{poinbendix:fig:fixed_point_omega_set}.
\end{beispiel}

Zu beachten gilt es, dass Satz von Poincaré-Bendixson nur eine Aussage zu möglichen Lösungen ab einem Startpunkt $p$ macht.
Dies würde unterschiedliche Fälle erlauben für ein einzelnes Differentialgleichungssystem.
Das obige Beispiel zeigt aber einen häufigen Fall, dass nämlich diverse Startpunkte $p$ denselben Nullpunkt als Omega Limesmenge haben.

\begin{figure}
    \centering
    %% Creator: Matplotlib, PGF backend
%%
%% To include the figure in your LaTeX document, write
%%   \input{<filename>.pgf}
%%
%% Make sure the required packages are loaded in your preamble
%%   \usepackage{pgf}
%%
%% Also ensure that all the required font packages are loaded; for instance,
%% the lmodern package is sometimes necessary when using math font.
%%   \usepackage{lmodern}
%%
%% Figures using additional raster images can only be included by \input if
%% they are in the same directory as the main LaTeX file. For loading figures
%% from other directories you can use the `import` package
%%   \usepackage{import}
%%
%% and then include the figures with
%%   \import{<path to file>}{<filename>.pgf}
%%
%% Matplotlib used the following preamble
%%   \usepackage{bm}
%%   \usepackage{amsmath}
%%   \usepackage{xcolor}
%%   \usepackage{tgtermes}
%%   \makeatletter\@ifpackageloaded{underscore}{}{\usepackage[strings]{underscore}}\makeatother
%%
\begingroup%
\makeatletter%
\begin{pgfpicture}%
\pgfpathrectangle{\pgfpointorigin}{\pgfqpoint{4.500000in}{2.500000in}}%
\pgfusepath{use as bounding box, clip}%
\begin{pgfscope}%
\pgfsetbuttcap%
\pgfsetmiterjoin%
\definecolor{currentfill}{rgb}{1.000000,1.000000,1.000000}%
\pgfsetfillcolor{currentfill}%
\pgfsetlinewidth{0.000000pt}%
\definecolor{currentstroke}{rgb}{1.000000,1.000000,1.000000}%
\pgfsetstrokecolor{currentstroke}%
\pgfsetdash{}{0pt}%
\pgfpathmoveto{\pgfqpoint{0.000000in}{0.000000in}}%
\pgfpathlineto{\pgfqpoint{4.500000in}{0.000000in}}%
\pgfpathlineto{\pgfqpoint{4.500000in}{2.500000in}}%
\pgfpathlineto{\pgfqpoint{0.000000in}{2.500000in}}%
\pgfpathlineto{\pgfqpoint{0.000000in}{0.000000in}}%
\pgfpathclose%
\pgfusepath{fill}%
\end{pgfscope}%
\begin{pgfscope}%
\pgfsetbuttcap%
\pgfsetmiterjoin%
\definecolor{currentfill}{rgb}{1.000000,1.000000,1.000000}%
\pgfsetfillcolor{currentfill}%
\pgfsetlinewidth{0.000000pt}%
\definecolor{currentstroke}{rgb}{0.000000,0.000000,0.000000}%
\pgfsetstrokecolor{currentstroke}%
\pgfsetstrokeopacity{0.000000}%
\pgfsetdash{}{0pt}%
\pgfpathmoveto{\pgfqpoint{0.562500in}{0.275000in}}%
\pgfpathlineto{\pgfqpoint{4.050000in}{0.275000in}}%
\pgfpathlineto{\pgfqpoint{4.050000in}{2.200000in}}%
\pgfpathlineto{\pgfqpoint{0.562500in}{2.200000in}}%
\pgfpathlineto{\pgfqpoint{0.562500in}{0.275000in}}%
\pgfpathclose%
\pgfusepath{fill}%
\end{pgfscope}%
\begin{pgfscope}%
\pgfpathrectangle{\pgfqpoint{0.562500in}{0.275000in}}{\pgfqpoint{3.487500in}{1.925000in}}%
\pgfusepath{clip}%
\pgfsetrectcap%
\pgfsetroundjoin%
\pgfsetlinewidth{0.803000pt}%
\definecolor{currentstroke}{rgb}{0.690196,0.690196,0.690196}%
\pgfsetstrokecolor{currentstroke}%
\pgfsetdash{}{0pt}%
\pgfpathmoveto{\pgfqpoint{0.732791in}{0.275000in}}%
\pgfpathlineto{\pgfqpoint{0.732791in}{2.200000in}}%
\pgfusepath{stroke}%
\end{pgfscope}%
\begin{pgfscope}%
\pgfsetbuttcap%
\pgfsetroundjoin%
\definecolor{currentfill}{rgb}{0.000000,0.000000,0.000000}%
\pgfsetfillcolor{currentfill}%
\pgfsetlinewidth{0.803000pt}%
\definecolor{currentstroke}{rgb}{0.000000,0.000000,0.000000}%
\pgfsetstrokecolor{currentstroke}%
\pgfsetdash{}{0pt}%
\pgfsys@defobject{currentmarker}{\pgfqpoint{0.000000in}{-0.048611in}}{\pgfqpoint{0.000000in}{0.000000in}}{%
\pgfpathmoveto{\pgfqpoint{0.000000in}{0.000000in}}%
\pgfpathlineto{\pgfqpoint{0.000000in}{-0.048611in}}%
\pgfusepath{stroke,fill}%
}%
\begin{pgfscope}%
\pgfsys@transformshift{0.732791in}{0.275000in}%
\pgfsys@useobject{currentmarker}{}%
\end{pgfscope}%
\end{pgfscope}%
\begin{pgfscope}%
\definecolor{textcolor}{rgb}{0.000000,0.000000,0.000000}%
\pgfsetstrokecolor{textcolor}%
\pgfsetfillcolor{textcolor}%
\pgftext[x=0.732791in,y=0.177778in,,top]{\color{textcolor}\rmfamily\fontsize{10.000000}{12.000000}\selectfont \(\displaystyle {-1.0}\)}%
\end{pgfscope}%
\begin{pgfscope}%
\pgfpathrectangle{\pgfqpoint{0.562500in}{0.275000in}}{\pgfqpoint{3.487500in}{1.925000in}}%
\pgfusepath{clip}%
\pgfsetrectcap%
\pgfsetroundjoin%
\pgfsetlinewidth{0.803000pt}%
\definecolor{currentstroke}{rgb}{0.690196,0.690196,0.690196}%
\pgfsetstrokecolor{currentstroke}%
\pgfsetdash{}{0pt}%
\pgfpathmoveto{\pgfqpoint{1.522463in}{0.275000in}}%
\pgfpathlineto{\pgfqpoint{1.522463in}{2.200000in}}%
\pgfusepath{stroke}%
\end{pgfscope}%
\begin{pgfscope}%
\pgfsetbuttcap%
\pgfsetroundjoin%
\definecolor{currentfill}{rgb}{0.000000,0.000000,0.000000}%
\pgfsetfillcolor{currentfill}%
\pgfsetlinewidth{0.803000pt}%
\definecolor{currentstroke}{rgb}{0.000000,0.000000,0.000000}%
\pgfsetstrokecolor{currentstroke}%
\pgfsetdash{}{0pt}%
\pgfsys@defobject{currentmarker}{\pgfqpoint{0.000000in}{-0.048611in}}{\pgfqpoint{0.000000in}{0.000000in}}{%
\pgfpathmoveto{\pgfqpoint{0.000000in}{0.000000in}}%
\pgfpathlineto{\pgfqpoint{0.000000in}{-0.048611in}}%
\pgfusepath{stroke,fill}%
}%
\begin{pgfscope}%
\pgfsys@transformshift{1.522463in}{0.275000in}%
\pgfsys@useobject{currentmarker}{}%
\end{pgfscope}%
\end{pgfscope}%
\begin{pgfscope}%
\definecolor{textcolor}{rgb}{0.000000,0.000000,0.000000}%
\pgfsetstrokecolor{textcolor}%
\pgfsetfillcolor{textcolor}%
\pgftext[x=1.522463in,y=0.177778in,,top]{\color{textcolor}\rmfamily\fontsize{10.000000}{12.000000}\selectfont \(\displaystyle {-0.5}\)}%
\end{pgfscope}%
\begin{pgfscope}%
\pgfpathrectangle{\pgfqpoint{0.562500in}{0.275000in}}{\pgfqpoint{3.487500in}{1.925000in}}%
\pgfusepath{clip}%
\pgfsetrectcap%
\pgfsetroundjoin%
\pgfsetlinewidth{0.803000pt}%
\definecolor{currentstroke}{rgb}{0.690196,0.690196,0.690196}%
\pgfsetstrokecolor{currentstroke}%
\pgfsetdash{}{0pt}%
\pgfpathmoveto{\pgfqpoint{2.312134in}{0.275000in}}%
\pgfpathlineto{\pgfqpoint{2.312134in}{2.200000in}}%
\pgfusepath{stroke}%
\end{pgfscope}%
\begin{pgfscope}%
\pgfsetbuttcap%
\pgfsetroundjoin%
\definecolor{currentfill}{rgb}{0.000000,0.000000,0.000000}%
\pgfsetfillcolor{currentfill}%
\pgfsetlinewidth{0.803000pt}%
\definecolor{currentstroke}{rgb}{0.000000,0.000000,0.000000}%
\pgfsetstrokecolor{currentstroke}%
\pgfsetdash{}{0pt}%
\pgfsys@defobject{currentmarker}{\pgfqpoint{0.000000in}{-0.048611in}}{\pgfqpoint{0.000000in}{0.000000in}}{%
\pgfpathmoveto{\pgfqpoint{0.000000in}{0.000000in}}%
\pgfpathlineto{\pgfqpoint{0.000000in}{-0.048611in}}%
\pgfusepath{stroke,fill}%
}%
\begin{pgfscope}%
\pgfsys@transformshift{2.312134in}{0.275000in}%
\pgfsys@useobject{currentmarker}{}%
\end{pgfscope}%
\end{pgfscope}%
\begin{pgfscope}%
\definecolor{textcolor}{rgb}{0.000000,0.000000,0.000000}%
\pgfsetstrokecolor{textcolor}%
\pgfsetfillcolor{textcolor}%
\pgftext[x=2.312134in,y=0.177778in,,top]{\color{textcolor}\rmfamily\fontsize{10.000000}{12.000000}\selectfont \(\displaystyle {0.0}\)}%
\end{pgfscope}%
\begin{pgfscope}%
\pgfpathrectangle{\pgfqpoint{0.562500in}{0.275000in}}{\pgfqpoint{3.487500in}{1.925000in}}%
\pgfusepath{clip}%
\pgfsetrectcap%
\pgfsetroundjoin%
\pgfsetlinewidth{0.803000pt}%
\definecolor{currentstroke}{rgb}{0.690196,0.690196,0.690196}%
\pgfsetstrokecolor{currentstroke}%
\pgfsetdash{}{0pt}%
\pgfpathmoveto{\pgfqpoint{3.101806in}{0.275000in}}%
\pgfpathlineto{\pgfqpoint{3.101806in}{2.200000in}}%
\pgfusepath{stroke}%
\end{pgfscope}%
\begin{pgfscope}%
\pgfsetbuttcap%
\pgfsetroundjoin%
\definecolor{currentfill}{rgb}{0.000000,0.000000,0.000000}%
\pgfsetfillcolor{currentfill}%
\pgfsetlinewidth{0.803000pt}%
\definecolor{currentstroke}{rgb}{0.000000,0.000000,0.000000}%
\pgfsetstrokecolor{currentstroke}%
\pgfsetdash{}{0pt}%
\pgfsys@defobject{currentmarker}{\pgfqpoint{0.000000in}{-0.048611in}}{\pgfqpoint{0.000000in}{0.000000in}}{%
\pgfpathmoveto{\pgfqpoint{0.000000in}{0.000000in}}%
\pgfpathlineto{\pgfqpoint{0.000000in}{-0.048611in}}%
\pgfusepath{stroke,fill}%
}%
\begin{pgfscope}%
\pgfsys@transformshift{3.101806in}{0.275000in}%
\pgfsys@useobject{currentmarker}{}%
\end{pgfscope}%
\end{pgfscope}%
\begin{pgfscope}%
\definecolor{textcolor}{rgb}{0.000000,0.000000,0.000000}%
\pgfsetstrokecolor{textcolor}%
\pgfsetfillcolor{textcolor}%
\pgftext[x=3.101806in,y=0.177778in,,top]{\color{textcolor}\rmfamily\fontsize{10.000000}{12.000000}\selectfont \(\displaystyle {0.5}\)}%
\end{pgfscope}%
\begin{pgfscope}%
\pgfpathrectangle{\pgfqpoint{0.562500in}{0.275000in}}{\pgfqpoint{3.487500in}{1.925000in}}%
\pgfusepath{clip}%
\pgfsetrectcap%
\pgfsetroundjoin%
\pgfsetlinewidth{0.803000pt}%
\definecolor{currentstroke}{rgb}{0.690196,0.690196,0.690196}%
\pgfsetstrokecolor{currentstroke}%
\pgfsetdash{}{0pt}%
\pgfpathmoveto{\pgfqpoint{3.891477in}{0.275000in}}%
\pgfpathlineto{\pgfqpoint{3.891477in}{2.200000in}}%
\pgfusepath{stroke}%
\end{pgfscope}%
\begin{pgfscope}%
\pgfsetbuttcap%
\pgfsetroundjoin%
\definecolor{currentfill}{rgb}{0.000000,0.000000,0.000000}%
\pgfsetfillcolor{currentfill}%
\pgfsetlinewidth{0.803000pt}%
\definecolor{currentstroke}{rgb}{0.000000,0.000000,0.000000}%
\pgfsetstrokecolor{currentstroke}%
\pgfsetdash{}{0pt}%
\pgfsys@defobject{currentmarker}{\pgfqpoint{0.000000in}{-0.048611in}}{\pgfqpoint{0.000000in}{0.000000in}}{%
\pgfpathmoveto{\pgfqpoint{0.000000in}{0.000000in}}%
\pgfpathlineto{\pgfqpoint{0.000000in}{-0.048611in}}%
\pgfusepath{stroke,fill}%
}%
\begin{pgfscope}%
\pgfsys@transformshift{3.891477in}{0.275000in}%
\pgfsys@useobject{currentmarker}{}%
\end{pgfscope}%
\end{pgfscope}%
\begin{pgfscope}%
\definecolor{textcolor}{rgb}{0.000000,0.000000,0.000000}%
\pgfsetstrokecolor{textcolor}%
\pgfsetfillcolor{textcolor}%
\pgftext[x=3.891477in,y=0.177778in,,top]{\color{textcolor}\rmfamily\fontsize{10.000000}{12.000000}\selectfont \(\displaystyle {1.0}\)}%
\end{pgfscope}%
\begin{pgfscope}%
\pgfpathrectangle{\pgfqpoint{0.562500in}{0.275000in}}{\pgfqpoint{3.487500in}{1.925000in}}%
\pgfusepath{clip}%
\pgfsetrectcap%
\pgfsetroundjoin%
\pgfsetlinewidth{0.803000pt}%
\definecolor{currentstroke}{rgb}{0.690196,0.690196,0.690196}%
\pgfsetstrokecolor{currentstroke}%
\pgfsetdash{}{0pt}%
\pgfpathmoveto{\pgfqpoint{0.562500in}{0.362500in}}%
\pgfpathlineto{\pgfqpoint{4.050000in}{0.362500in}}%
\pgfusepath{stroke}%
\end{pgfscope}%
\begin{pgfscope}%
\pgfsetbuttcap%
\pgfsetroundjoin%
\definecolor{currentfill}{rgb}{0.000000,0.000000,0.000000}%
\pgfsetfillcolor{currentfill}%
\pgfsetlinewidth{0.803000pt}%
\definecolor{currentstroke}{rgb}{0.000000,0.000000,0.000000}%
\pgfsetstrokecolor{currentstroke}%
\pgfsetdash{}{0pt}%
\pgfsys@defobject{currentmarker}{\pgfqpoint{-0.048611in}{0.000000in}}{\pgfqpoint{-0.000000in}{0.000000in}}{%
\pgfpathmoveto{\pgfqpoint{-0.000000in}{0.000000in}}%
\pgfpathlineto{\pgfqpoint{-0.048611in}{0.000000in}}%
\pgfusepath{stroke,fill}%
}%
\begin{pgfscope}%
\pgfsys@transformshift{0.562500in}{0.362500in}%
\pgfsys@useobject{currentmarker}{}%
\end{pgfscope}%
\end{pgfscope}%
\begin{pgfscope}%
\definecolor{textcolor}{rgb}{0.000000,0.000000,0.000000}%
\pgfsetstrokecolor{textcolor}%
\pgfsetfillcolor{textcolor}%
\pgftext[x=0.287808in, y=0.315799in, left, base]{\color{textcolor}\rmfamily\fontsize{10.000000}{12.000000}\selectfont \(\displaystyle {-1}\)}%
\end{pgfscope}%
\begin{pgfscope}%
\pgfpathrectangle{\pgfqpoint{0.562500in}{0.275000in}}{\pgfqpoint{3.487500in}{1.925000in}}%
\pgfusepath{clip}%
\pgfsetrectcap%
\pgfsetroundjoin%
\pgfsetlinewidth{0.803000pt}%
\definecolor{currentstroke}{rgb}{0.690196,0.690196,0.690196}%
\pgfsetstrokecolor{currentstroke}%
\pgfsetdash{}{0pt}%
\pgfpathmoveto{\pgfqpoint{0.562500in}{0.943734in}}%
\pgfpathlineto{\pgfqpoint{4.050000in}{0.943734in}}%
\pgfusepath{stroke}%
\end{pgfscope}%
\begin{pgfscope}%
\pgfsetbuttcap%
\pgfsetroundjoin%
\definecolor{currentfill}{rgb}{0.000000,0.000000,0.000000}%
\pgfsetfillcolor{currentfill}%
\pgfsetlinewidth{0.803000pt}%
\definecolor{currentstroke}{rgb}{0.000000,0.000000,0.000000}%
\pgfsetstrokecolor{currentstroke}%
\pgfsetdash{}{0pt}%
\pgfsys@defobject{currentmarker}{\pgfqpoint{-0.048611in}{0.000000in}}{\pgfqpoint{-0.000000in}{0.000000in}}{%
\pgfpathmoveto{\pgfqpoint{-0.000000in}{0.000000in}}%
\pgfpathlineto{\pgfqpoint{-0.048611in}{0.000000in}}%
\pgfusepath{stroke,fill}%
}%
\begin{pgfscope}%
\pgfsys@transformshift{0.562500in}{0.943734in}%
\pgfsys@useobject{currentmarker}{}%
\end{pgfscope}%
\end{pgfscope}%
\begin{pgfscope}%
\definecolor{textcolor}{rgb}{0.000000,0.000000,0.000000}%
\pgfsetstrokecolor{textcolor}%
\pgfsetfillcolor{textcolor}%
\pgftext[x=0.395833in, y=0.897032in, left, base]{\color{textcolor}\rmfamily\fontsize{10.000000}{12.000000}\selectfont \(\displaystyle {0}\)}%
\end{pgfscope}%
\begin{pgfscope}%
\pgfpathrectangle{\pgfqpoint{0.562500in}{0.275000in}}{\pgfqpoint{3.487500in}{1.925000in}}%
\pgfusepath{clip}%
\pgfsetrectcap%
\pgfsetroundjoin%
\pgfsetlinewidth{0.803000pt}%
\definecolor{currentstroke}{rgb}{0.690196,0.690196,0.690196}%
\pgfsetstrokecolor{currentstroke}%
\pgfsetdash{}{0pt}%
\pgfpathmoveto{\pgfqpoint{0.562500in}{1.524967in}}%
\pgfpathlineto{\pgfqpoint{4.050000in}{1.524967in}}%
\pgfusepath{stroke}%
\end{pgfscope}%
\begin{pgfscope}%
\pgfsetbuttcap%
\pgfsetroundjoin%
\definecolor{currentfill}{rgb}{0.000000,0.000000,0.000000}%
\pgfsetfillcolor{currentfill}%
\pgfsetlinewidth{0.803000pt}%
\definecolor{currentstroke}{rgb}{0.000000,0.000000,0.000000}%
\pgfsetstrokecolor{currentstroke}%
\pgfsetdash{}{0pt}%
\pgfsys@defobject{currentmarker}{\pgfqpoint{-0.048611in}{0.000000in}}{\pgfqpoint{-0.000000in}{0.000000in}}{%
\pgfpathmoveto{\pgfqpoint{-0.000000in}{0.000000in}}%
\pgfpathlineto{\pgfqpoint{-0.048611in}{0.000000in}}%
\pgfusepath{stroke,fill}%
}%
\begin{pgfscope}%
\pgfsys@transformshift{0.562500in}{1.524967in}%
\pgfsys@useobject{currentmarker}{}%
\end{pgfscope}%
\end{pgfscope}%
\begin{pgfscope}%
\definecolor{textcolor}{rgb}{0.000000,0.000000,0.000000}%
\pgfsetstrokecolor{textcolor}%
\pgfsetfillcolor{textcolor}%
\pgftext[x=0.395833in, y=1.478266in, left, base]{\color{textcolor}\rmfamily\fontsize{10.000000}{12.000000}\selectfont \(\displaystyle {1}\)}%
\end{pgfscope}%
\begin{pgfscope}%
\pgfpathrectangle{\pgfqpoint{0.562500in}{0.275000in}}{\pgfqpoint{3.487500in}{1.925000in}}%
\pgfusepath{clip}%
\pgfsetrectcap%
\pgfsetroundjoin%
\pgfsetlinewidth{0.803000pt}%
\definecolor{currentstroke}{rgb}{0.690196,0.690196,0.690196}%
\pgfsetstrokecolor{currentstroke}%
\pgfsetdash{}{0pt}%
\pgfpathmoveto{\pgfqpoint{0.562500in}{2.106201in}}%
\pgfpathlineto{\pgfqpoint{4.050000in}{2.106201in}}%
\pgfusepath{stroke}%
\end{pgfscope}%
\begin{pgfscope}%
\pgfsetbuttcap%
\pgfsetroundjoin%
\definecolor{currentfill}{rgb}{0.000000,0.000000,0.000000}%
\pgfsetfillcolor{currentfill}%
\pgfsetlinewidth{0.803000pt}%
\definecolor{currentstroke}{rgb}{0.000000,0.000000,0.000000}%
\pgfsetstrokecolor{currentstroke}%
\pgfsetdash{}{0pt}%
\pgfsys@defobject{currentmarker}{\pgfqpoint{-0.048611in}{0.000000in}}{\pgfqpoint{-0.000000in}{0.000000in}}{%
\pgfpathmoveto{\pgfqpoint{-0.000000in}{0.000000in}}%
\pgfpathlineto{\pgfqpoint{-0.048611in}{0.000000in}}%
\pgfusepath{stroke,fill}%
}%
\begin{pgfscope}%
\pgfsys@transformshift{0.562500in}{2.106201in}%
\pgfsys@useobject{currentmarker}{}%
\end{pgfscope}%
\end{pgfscope}%
\begin{pgfscope}%
\definecolor{textcolor}{rgb}{0.000000,0.000000,0.000000}%
\pgfsetstrokecolor{textcolor}%
\pgfsetfillcolor{textcolor}%
\pgftext[x=0.395833in, y=2.059500in, left, base]{\color{textcolor}\rmfamily\fontsize{10.000000}{12.000000}\selectfont \(\displaystyle {2}\)}%
\end{pgfscope}%
\begin{pgfscope}%
\pgfpathrectangle{\pgfqpoint{0.562500in}{0.275000in}}{\pgfqpoint{3.487500in}{1.925000in}}%
\pgfusepath{clip}%
\pgfsetbuttcap%
\pgfsetroundjoin%
\definecolor{currentfill}{rgb}{0.121569,0.466667,0.705882}%
\pgfsetfillcolor{currentfill}%
\pgfsetlinewidth{1.003750pt}%
\definecolor{currentstroke}{rgb}{0.121569,0.466667,0.705882}%
\pgfsetstrokecolor{currentstroke}%
\pgfsetdash{}{0pt}%
\pgfsys@defobject{currentmarker}{\pgfqpoint{-0.020833in}{-0.020833in}}{\pgfqpoint{0.020833in}{0.020833in}}{%
\pgfpathmoveto{\pgfqpoint{0.000000in}{-0.020833in}}%
\pgfpathcurveto{\pgfqpoint{0.005525in}{-0.020833in}}{\pgfqpoint{0.010825in}{-0.018638in}}{\pgfqpoint{0.014731in}{-0.014731in}}%
\pgfpathcurveto{\pgfqpoint{0.018638in}{-0.010825in}}{\pgfqpoint{0.020833in}{-0.005525in}}{\pgfqpoint{0.020833in}{0.000000in}}%
\pgfpathcurveto{\pgfqpoint{0.020833in}{0.005525in}}{\pgfqpoint{0.018638in}{0.010825in}}{\pgfqpoint{0.014731in}{0.014731in}}%
\pgfpathcurveto{\pgfqpoint{0.010825in}{0.018638in}}{\pgfqpoint{0.005525in}{0.020833in}}{\pgfqpoint{0.000000in}{0.020833in}}%
\pgfpathcurveto{\pgfqpoint{-0.005525in}{0.020833in}}{\pgfqpoint{-0.010825in}{0.018638in}}{\pgfqpoint{-0.014731in}{0.014731in}}%
\pgfpathcurveto{\pgfqpoint{-0.018638in}{0.010825in}}{\pgfqpoint{-0.020833in}{0.005525in}}{\pgfqpoint{-0.020833in}{0.000000in}}%
\pgfpathcurveto{\pgfqpoint{-0.020833in}{-0.005525in}}{\pgfqpoint{-0.018638in}{-0.010825in}}{\pgfqpoint{-0.014731in}{-0.014731in}}%
\pgfpathcurveto{\pgfqpoint{-0.010825in}{-0.018638in}}{\pgfqpoint{-0.005525in}{-0.020833in}}{\pgfqpoint{0.000000in}{-0.020833in}}%
\pgfpathlineto{\pgfqpoint{0.000000in}{-0.020833in}}%
\pgfpathclose%
\pgfusepath{stroke,fill}%
}%
\begin{pgfscope}%
\pgfsys@transformshift{2.312134in}{0.943734in}%
\pgfsys@useobject{currentmarker}{}%
\end{pgfscope}%
\begin{pgfscope}%
\pgfsys@transformshift{2.312134in}{0.943734in}%
\pgfsys@useobject{currentmarker}{}%
\end{pgfscope}%
\begin{pgfscope}%
\pgfsys@transformshift{2.312134in}{0.943734in}%
\pgfsys@useobject{currentmarker}{}%
\end{pgfscope}%
\begin{pgfscope}%
\pgfsys@transformshift{2.312134in}{0.943734in}%
\pgfsys@useobject{currentmarker}{}%
\end{pgfscope}%
\begin{pgfscope}%
\pgfsys@transformshift{2.312134in}{0.943734in}%
\pgfsys@useobject{currentmarker}{}%
\end{pgfscope}%
\begin{pgfscope}%
\pgfsys@transformshift{2.312134in}{0.943734in}%
\pgfsys@useobject{currentmarker}{}%
\end{pgfscope}%
\begin{pgfscope}%
\pgfsys@transformshift{2.312134in}{0.943734in}%
\pgfsys@useobject{currentmarker}{}%
\end{pgfscope}%
\begin{pgfscope}%
\pgfsys@transformshift{2.312134in}{0.943734in}%
\pgfsys@useobject{currentmarker}{}%
\end{pgfscope}%
\begin{pgfscope}%
\pgfsys@transformshift{2.312134in}{0.943734in}%
\pgfsys@useobject{currentmarker}{}%
\end{pgfscope}%
\begin{pgfscope}%
\pgfsys@transformshift{2.312134in}{0.943734in}%
\pgfsys@useobject{currentmarker}{}%
\end{pgfscope}%
\begin{pgfscope}%
\pgfsys@transformshift{2.312134in}{0.943734in}%
\pgfsys@useobject{currentmarker}{}%
\end{pgfscope}%
\begin{pgfscope}%
\pgfsys@transformshift{2.312134in}{0.943734in}%
\pgfsys@useobject{currentmarker}{}%
\end{pgfscope}%
\begin{pgfscope}%
\pgfsys@transformshift{2.312134in}{0.943734in}%
\pgfsys@useobject{currentmarker}{}%
\end{pgfscope}%
\begin{pgfscope}%
\pgfsys@transformshift{2.312134in}{0.943734in}%
\pgfsys@useobject{currentmarker}{}%
\end{pgfscope}%
\begin{pgfscope}%
\pgfsys@transformshift{2.312134in}{0.943734in}%
\pgfsys@useobject{currentmarker}{}%
\end{pgfscope}%
\begin{pgfscope}%
\pgfsys@transformshift{2.312134in}{0.943734in}%
\pgfsys@useobject{currentmarker}{}%
\end{pgfscope}%
\begin{pgfscope}%
\pgfsys@transformshift{2.312134in}{0.943734in}%
\pgfsys@useobject{currentmarker}{}%
\end{pgfscope}%
\begin{pgfscope}%
\pgfsys@transformshift{2.312134in}{0.943734in}%
\pgfsys@useobject{currentmarker}{}%
\end{pgfscope}%
\begin{pgfscope}%
\pgfsys@transformshift{2.312134in}{0.943734in}%
\pgfsys@useobject{currentmarker}{}%
\end{pgfscope}%
\begin{pgfscope}%
\pgfsys@transformshift{2.312134in}{0.943734in}%
\pgfsys@useobject{currentmarker}{}%
\end{pgfscope}%
\begin{pgfscope}%
\pgfsys@transformshift{2.312134in}{0.943734in}%
\pgfsys@useobject{currentmarker}{}%
\end{pgfscope}%
\begin{pgfscope}%
\pgfsys@transformshift{2.312134in}{0.943734in}%
\pgfsys@useobject{currentmarker}{}%
\end{pgfscope}%
\begin{pgfscope}%
\pgfsys@transformshift{2.312134in}{0.943734in}%
\pgfsys@useobject{currentmarker}{}%
\end{pgfscope}%
\begin{pgfscope}%
\pgfsys@transformshift{2.312134in}{0.943734in}%
\pgfsys@useobject{currentmarker}{}%
\end{pgfscope}%
\begin{pgfscope}%
\pgfsys@transformshift{2.312134in}{0.943734in}%
\pgfsys@useobject{currentmarker}{}%
\end{pgfscope}%
\begin{pgfscope}%
\pgfsys@transformshift{2.312134in}{0.943734in}%
\pgfsys@useobject{currentmarker}{}%
\end{pgfscope}%
\begin{pgfscope}%
\pgfsys@transformshift{2.312134in}{0.943734in}%
\pgfsys@useobject{currentmarker}{}%
\end{pgfscope}%
\begin{pgfscope}%
\pgfsys@transformshift{2.312134in}{0.943734in}%
\pgfsys@useobject{currentmarker}{}%
\end{pgfscope}%
\begin{pgfscope}%
\pgfsys@transformshift{2.312134in}{0.943734in}%
\pgfsys@useobject{currentmarker}{}%
\end{pgfscope}%
\begin{pgfscope}%
\pgfsys@transformshift{2.312134in}{0.943734in}%
\pgfsys@useobject{currentmarker}{}%
\end{pgfscope}%
\begin{pgfscope}%
\pgfsys@transformshift{2.312134in}{0.943734in}%
\pgfsys@useobject{currentmarker}{}%
\end{pgfscope}%
\begin{pgfscope}%
\pgfsys@transformshift{2.312134in}{0.943734in}%
\pgfsys@useobject{currentmarker}{}%
\end{pgfscope}%
\begin{pgfscope}%
\pgfsys@transformshift{2.312134in}{0.943734in}%
\pgfsys@useobject{currentmarker}{}%
\end{pgfscope}%
\begin{pgfscope}%
\pgfsys@transformshift{2.312134in}{0.943734in}%
\pgfsys@useobject{currentmarker}{}%
\end{pgfscope}%
\begin{pgfscope}%
\pgfsys@transformshift{2.312134in}{0.943734in}%
\pgfsys@useobject{currentmarker}{}%
\end{pgfscope}%
\begin{pgfscope}%
\pgfsys@transformshift{2.312134in}{0.943734in}%
\pgfsys@useobject{currentmarker}{}%
\end{pgfscope}%
\begin{pgfscope}%
\pgfsys@transformshift{2.312134in}{0.943734in}%
\pgfsys@useobject{currentmarker}{}%
\end{pgfscope}%
\begin{pgfscope}%
\pgfsys@transformshift{2.312134in}{0.943734in}%
\pgfsys@useobject{currentmarker}{}%
\end{pgfscope}%
\begin{pgfscope}%
\pgfsys@transformshift{2.312134in}{0.943734in}%
\pgfsys@useobject{currentmarker}{}%
\end{pgfscope}%
\begin{pgfscope}%
\pgfsys@transformshift{2.312134in}{0.943734in}%
\pgfsys@useobject{currentmarker}{}%
\end{pgfscope}%
\begin{pgfscope}%
\pgfsys@transformshift{2.312134in}{0.943734in}%
\pgfsys@useobject{currentmarker}{}%
\end{pgfscope}%
\begin{pgfscope}%
\pgfsys@transformshift{2.312134in}{0.943734in}%
\pgfsys@useobject{currentmarker}{}%
\end{pgfscope}%
\begin{pgfscope}%
\pgfsys@transformshift{2.312134in}{0.943734in}%
\pgfsys@useobject{currentmarker}{}%
\end{pgfscope}%
\begin{pgfscope}%
\pgfsys@transformshift{2.312134in}{0.943734in}%
\pgfsys@useobject{currentmarker}{}%
\end{pgfscope}%
\begin{pgfscope}%
\pgfsys@transformshift{2.312134in}{0.943734in}%
\pgfsys@useobject{currentmarker}{}%
\end{pgfscope}%
\begin{pgfscope}%
\pgfsys@transformshift{2.312134in}{0.943734in}%
\pgfsys@useobject{currentmarker}{}%
\end{pgfscope}%
\begin{pgfscope}%
\pgfsys@transformshift{2.312134in}{0.943734in}%
\pgfsys@useobject{currentmarker}{}%
\end{pgfscope}%
\begin{pgfscope}%
\pgfsys@transformshift{2.312134in}{0.943734in}%
\pgfsys@useobject{currentmarker}{}%
\end{pgfscope}%
\begin{pgfscope}%
\pgfsys@transformshift{2.312134in}{0.943734in}%
\pgfsys@useobject{currentmarker}{}%
\end{pgfscope}%
\begin{pgfscope}%
\pgfsys@transformshift{2.312134in}{0.943734in}%
\pgfsys@useobject{currentmarker}{}%
\end{pgfscope}%
\begin{pgfscope}%
\pgfsys@transformshift{2.312134in}{0.943734in}%
\pgfsys@useobject{currentmarker}{}%
\end{pgfscope}%
\begin{pgfscope}%
\pgfsys@transformshift{2.312134in}{0.943734in}%
\pgfsys@useobject{currentmarker}{}%
\end{pgfscope}%
\begin{pgfscope}%
\pgfsys@transformshift{2.312134in}{0.943734in}%
\pgfsys@useobject{currentmarker}{}%
\end{pgfscope}%
\begin{pgfscope}%
\pgfsys@transformshift{2.312134in}{0.943734in}%
\pgfsys@useobject{currentmarker}{}%
\end{pgfscope}%
\begin{pgfscope}%
\pgfsys@transformshift{2.312134in}{0.943734in}%
\pgfsys@useobject{currentmarker}{}%
\end{pgfscope}%
\begin{pgfscope}%
\pgfsys@transformshift{2.312134in}{0.943734in}%
\pgfsys@useobject{currentmarker}{}%
\end{pgfscope}%
\begin{pgfscope}%
\pgfsys@transformshift{2.312134in}{0.943734in}%
\pgfsys@useobject{currentmarker}{}%
\end{pgfscope}%
\begin{pgfscope}%
\pgfsys@transformshift{2.312134in}{0.943734in}%
\pgfsys@useobject{currentmarker}{}%
\end{pgfscope}%
\begin{pgfscope}%
\pgfsys@transformshift{2.312134in}{0.943734in}%
\pgfsys@useobject{currentmarker}{}%
\end{pgfscope}%
\begin{pgfscope}%
\pgfsys@transformshift{2.312134in}{0.943734in}%
\pgfsys@useobject{currentmarker}{}%
\end{pgfscope}%
\begin{pgfscope}%
\pgfsys@transformshift{2.312134in}{0.943734in}%
\pgfsys@useobject{currentmarker}{}%
\end{pgfscope}%
\begin{pgfscope}%
\pgfsys@transformshift{2.312134in}{0.943734in}%
\pgfsys@useobject{currentmarker}{}%
\end{pgfscope}%
\begin{pgfscope}%
\pgfsys@transformshift{2.312134in}{0.943734in}%
\pgfsys@useobject{currentmarker}{}%
\end{pgfscope}%
\begin{pgfscope}%
\pgfsys@transformshift{2.312134in}{0.943734in}%
\pgfsys@useobject{currentmarker}{}%
\end{pgfscope}%
\begin{pgfscope}%
\pgfsys@transformshift{2.312134in}{0.943734in}%
\pgfsys@useobject{currentmarker}{}%
\end{pgfscope}%
\begin{pgfscope}%
\pgfsys@transformshift{2.312134in}{0.943734in}%
\pgfsys@useobject{currentmarker}{}%
\end{pgfscope}%
\begin{pgfscope}%
\pgfsys@transformshift{2.312134in}{0.943734in}%
\pgfsys@useobject{currentmarker}{}%
\end{pgfscope}%
\begin{pgfscope}%
\pgfsys@transformshift{2.312134in}{0.943734in}%
\pgfsys@useobject{currentmarker}{}%
\end{pgfscope}%
\begin{pgfscope}%
\pgfsys@transformshift{2.312134in}{0.943734in}%
\pgfsys@useobject{currentmarker}{}%
\end{pgfscope}%
\begin{pgfscope}%
\pgfsys@transformshift{2.312134in}{0.943734in}%
\pgfsys@useobject{currentmarker}{}%
\end{pgfscope}%
\begin{pgfscope}%
\pgfsys@transformshift{2.312134in}{0.943734in}%
\pgfsys@useobject{currentmarker}{}%
\end{pgfscope}%
\begin{pgfscope}%
\pgfsys@transformshift{2.312134in}{0.943734in}%
\pgfsys@useobject{currentmarker}{}%
\end{pgfscope}%
\begin{pgfscope}%
\pgfsys@transformshift{2.312134in}{0.943734in}%
\pgfsys@useobject{currentmarker}{}%
\end{pgfscope}%
\begin{pgfscope}%
\pgfsys@transformshift{2.312134in}{0.943734in}%
\pgfsys@useobject{currentmarker}{}%
\end{pgfscope}%
\begin{pgfscope}%
\pgfsys@transformshift{2.312134in}{0.943734in}%
\pgfsys@useobject{currentmarker}{}%
\end{pgfscope}%
\begin{pgfscope}%
\pgfsys@transformshift{2.312134in}{0.943734in}%
\pgfsys@useobject{currentmarker}{}%
\end{pgfscope}%
\begin{pgfscope}%
\pgfsys@transformshift{2.312134in}{0.943734in}%
\pgfsys@useobject{currentmarker}{}%
\end{pgfscope}%
\begin{pgfscope}%
\pgfsys@transformshift{2.312134in}{0.943734in}%
\pgfsys@useobject{currentmarker}{}%
\end{pgfscope}%
\begin{pgfscope}%
\pgfsys@transformshift{2.312134in}{0.943734in}%
\pgfsys@useobject{currentmarker}{}%
\end{pgfscope}%
\begin{pgfscope}%
\pgfsys@transformshift{2.312134in}{0.943734in}%
\pgfsys@useobject{currentmarker}{}%
\end{pgfscope}%
\begin{pgfscope}%
\pgfsys@transformshift{2.312134in}{0.943734in}%
\pgfsys@useobject{currentmarker}{}%
\end{pgfscope}%
\begin{pgfscope}%
\pgfsys@transformshift{2.312134in}{0.943734in}%
\pgfsys@useobject{currentmarker}{}%
\end{pgfscope}%
\begin{pgfscope}%
\pgfsys@transformshift{2.312134in}{0.943734in}%
\pgfsys@useobject{currentmarker}{}%
\end{pgfscope}%
\begin{pgfscope}%
\pgfsys@transformshift{2.312134in}{0.943734in}%
\pgfsys@useobject{currentmarker}{}%
\end{pgfscope}%
\begin{pgfscope}%
\pgfsys@transformshift{2.312134in}{0.943734in}%
\pgfsys@useobject{currentmarker}{}%
\end{pgfscope}%
\begin{pgfscope}%
\pgfsys@transformshift{2.312134in}{0.943734in}%
\pgfsys@useobject{currentmarker}{}%
\end{pgfscope}%
\begin{pgfscope}%
\pgfsys@transformshift{2.312134in}{0.943734in}%
\pgfsys@useobject{currentmarker}{}%
\end{pgfscope}%
\begin{pgfscope}%
\pgfsys@transformshift{2.312134in}{0.943734in}%
\pgfsys@useobject{currentmarker}{}%
\end{pgfscope}%
\begin{pgfscope}%
\pgfsys@transformshift{2.312134in}{0.943734in}%
\pgfsys@useobject{currentmarker}{}%
\end{pgfscope}%
\begin{pgfscope}%
\pgfsys@transformshift{2.312134in}{0.943734in}%
\pgfsys@useobject{currentmarker}{}%
\end{pgfscope}%
\begin{pgfscope}%
\pgfsys@transformshift{2.312134in}{0.943734in}%
\pgfsys@useobject{currentmarker}{}%
\end{pgfscope}%
\begin{pgfscope}%
\pgfsys@transformshift{2.312134in}{0.943734in}%
\pgfsys@useobject{currentmarker}{}%
\end{pgfscope}%
\begin{pgfscope}%
\pgfsys@transformshift{2.312134in}{0.943734in}%
\pgfsys@useobject{currentmarker}{}%
\end{pgfscope}%
\begin{pgfscope}%
\pgfsys@transformshift{2.312134in}{0.943734in}%
\pgfsys@useobject{currentmarker}{}%
\end{pgfscope}%
\begin{pgfscope}%
\pgfsys@transformshift{2.312134in}{0.943734in}%
\pgfsys@useobject{currentmarker}{}%
\end{pgfscope}%
\begin{pgfscope}%
\pgfsys@transformshift{2.312134in}{0.943734in}%
\pgfsys@useobject{currentmarker}{}%
\end{pgfscope}%
\begin{pgfscope}%
\pgfsys@transformshift{2.312134in}{0.943734in}%
\pgfsys@useobject{currentmarker}{}%
\end{pgfscope}%
\begin{pgfscope}%
\pgfsys@transformshift{2.312134in}{0.943734in}%
\pgfsys@useobject{currentmarker}{}%
\end{pgfscope}%
\begin{pgfscope}%
\pgfsys@transformshift{2.312134in}{0.943734in}%
\pgfsys@useobject{currentmarker}{}%
\end{pgfscope}%
\begin{pgfscope}%
\pgfsys@transformshift{2.312134in}{0.943734in}%
\pgfsys@useobject{currentmarker}{}%
\end{pgfscope}%
\begin{pgfscope}%
\pgfsys@transformshift{2.312134in}{0.943734in}%
\pgfsys@useobject{currentmarker}{}%
\end{pgfscope}%
\begin{pgfscope}%
\pgfsys@transformshift{2.312134in}{0.943734in}%
\pgfsys@useobject{currentmarker}{}%
\end{pgfscope}%
\begin{pgfscope}%
\pgfsys@transformshift{2.312134in}{0.943734in}%
\pgfsys@useobject{currentmarker}{}%
\end{pgfscope}%
\begin{pgfscope}%
\pgfsys@transformshift{2.312134in}{0.943734in}%
\pgfsys@useobject{currentmarker}{}%
\end{pgfscope}%
\begin{pgfscope}%
\pgfsys@transformshift{2.312134in}{0.943734in}%
\pgfsys@useobject{currentmarker}{}%
\end{pgfscope}%
\begin{pgfscope}%
\pgfsys@transformshift{2.312134in}{0.943734in}%
\pgfsys@useobject{currentmarker}{}%
\end{pgfscope}%
\begin{pgfscope}%
\pgfsys@transformshift{2.312134in}{0.943734in}%
\pgfsys@useobject{currentmarker}{}%
\end{pgfscope}%
\begin{pgfscope}%
\pgfsys@transformshift{2.312134in}{0.943734in}%
\pgfsys@useobject{currentmarker}{}%
\end{pgfscope}%
\begin{pgfscope}%
\pgfsys@transformshift{2.312134in}{0.943734in}%
\pgfsys@useobject{currentmarker}{}%
\end{pgfscope}%
\begin{pgfscope}%
\pgfsys@transformshift{2.312134in}{0.943734in}%
\pgfsys@useobject{currentmarker}{}%
\end{pgfscope}%
\begin{pgfscope}%
\pgfsys@transformshift{2.312134in}{0.943734in}%
\pgfsys@useobject{currentmarker}{}%
\end{pgfscope}%
\begin{pgfscope}%
\pgfsys@transformshift{2.312134in}{0.943734in}%
\pgfsys@useobject{currentmarker}{}%
\end{pgfscope}%
\begin{pgfscope}%
\pgfsys@transformshift{2.312134in}{0.943734in}%
\pgfsys@useobject{currentmarker}{}%
\end{pgfscope}%
\begin{pgfscope}%
\pgfsys@transformshift{2.312134in}{0.943734in}%
\pgfsys@useobject{currentmarker}{}%
\end{pgfscope}%
\begin{pgfscope}%
\pgfsys@transformshift{2.312134in}{0.943734in}%
\pgfsys@useobject{currentmarker}{}%
\end{pgfscope}%
\begin{pgfscope}%
\pgfsys@transformshift{2.312134in}{0.943734in}%
\pgfsys@useobject{currentmarker}{}%
\end{pgfscope}%
\begin{pgfscope}%
\pgfsys@transformshift{2.312134in}{0.943734in}%
\pgfsys@useobject{currentmarker}{}%
\end{pgfscope}%
\begin{pgfscope}%
\pgfsys@transformshift{2.312134in}{0.943734in}%
\pgfsys@useobject{currentmarker}{}%
\end{pgfscope}%
\begin{pgfscope}%
\pgfsys@transformshift{2.312134in}{0.943734in}%
\pgfsys@useobject{currentmarker}{}%
\end{pgfscope}%
\begin{pgfscope}%
\pgfsys@transformshift{2.312134in}{0.943734in}%
\pgfsys@useobject{currentmarker}{}%
\end{pgfscope}%
\begin{pgfscope}%
\pgfsys@transformshift{2.312134in}{0.943734in}%
\pgfsys@useobject{currentmarker}{}%
\end{pgfscope}%
\begin{pgfscope}%
\pgfsys@transformshift{2.312134in}{0.943734in}%
\pgfsys@useobject{currentmarker}{}%
\end{pgfscope}%
\begin{pgfscope}%
\pgfsys@transformshift{2.312134in}{0.943734in}%
\pgfsys@useobject{currentmarker}{}%
\end{pgfscope}%
\begin{pgfscope}%
\pgfsys@transformshift{2.312134in}{0.943734in}%
\pgfsys@useobject{currentmarker}{}%
\end{pgfscope}%
\begin{pgfscope}%
\pgfsys@transformshift{2.312134in}{0.943734in}%
\pgfsys@useobject{currentmarker}{}%
\end{pgfscope}%
\begin{pgfscope}%
\pgfsys@transformshift{2.312134in}{0.943734in}%
\pgfsys@useobject{currentmarker}{}%
\end{pgfscope}%
\begin{pgfscope}%
\pgfsys@transformshift{2.312134in}{0.943734in}%
\pgfsys@useobject{currentmarker}{}%
\end{pgfscope}%
\begin{pgfscope}%
\pgfsys@transformshift{2.312134in}{0.943734in}%
\pgfsys@useobject{currentmarker}{}%
\end{pgfscope}%
\begin{pgfscope}%
\pgfsys@transformshift{2.312134in}{0.943734in}%
\pgfsys@useobject{currentmarker}{}%
\end{pgfscope}%
\begin{pgfscope}%
\pgfsys@transformshift{2.312134in}{0.943734in}%
\pgfsys@useobject{currentmarker}{}%
\end{pgfscope}%
\begin{pgfscope}%
\pgfsys@transformshift{2.312134in}{0.943734in}%
\pgfsys@useobject{currentmarker}{}%
\end{pgfscope}%
\begin{pgfscope}%
\pgfsys@transformshift{2.312134in}{0.943734in}%
\pgfsys@useobject{currentmarker}{}%
\end{pgfscope}%
\begin{pgfscope}%
\pgfsys@transformshift{2.312134in}{0.943734in}%
\pgfsys@useobject{currentmarker}{}%
\end{pgfscope}%
\begin{pgfscope}%
\pgfsys@transformshift{2.312134in}{0.943734in}%
\pgfsys@useobject{currentmarker}{}%
\end{pgfscope}%
\begin{pgfscope}%
\pgfsys@transformshift{2.312134in}{0.943734in}%
\pgfsys@useobject{currentmarker}{}%
\end{pgfscope}%
\begin{pgfscope}%
\pgfsys@transformshift{2.312134in}{0.943734in}%
\pgfsys@useobject{currentmarker}{}%
\end{pgfscope}%
\begin{pgfscope}%
\pgfsys@transformshift{2.312134in}{0.943734in}%
\pgfsys@useobject{currentmarker}{}%
\end{pgfscope}%
\begin{pgfscope}%
\pgfsys@transformshift{2.312134in}{0.943734in}%
\pgfsys@useobject{currentmarker}{}%
\end{pgfscope}%
\begin{pgfscope}%
\pgfsys@transformshift{2.312134in}{0.943734in}%
\pgfsys@useobject{currentmarker}{}%
\end{pgfscope}%
\begin{pgfscope}%
\pgfsys@transformshift{2.312134in}{0.943734in}%
\pgfsys@useobject{currentmarker}{}%
\end{pgfscope}%
\begin{pgfscope}%
\pgfsys@transformshift{2.312134in}{0.943734in}%
\pgfsys@useobject{currentmarker}{}%
\end{pgfscope}%
\begin{pgfscope}%
\pgfsys@transformshift{2.312134in}{0.943734in}%
\pgfsys@useobject{currentmarker}{}%
\end{pgfscope}%
\begin{pgfscope}%
\pgfsys@transformshift{2.312134in}{0.943734in}%
\pgfsys@useobject{currentmarker}{}%
\end{pgfscope}%
\begin{pgfscope}%
\pgfsys@transformshift{2.312134in}{0.943734in}%
\pgfsys@useobject{currentmarker}{}%
\end{pgfscope}%
\begin{pgfscope}%
\pgfsys@transformshift{2.312134in}{0.943734in}%
\pgfsys@useobject{currentmarker}{}%
\end{pgfscope}%
\begin{pgfscope}%
\pgfsys@transformshift{2.312134in}{0.943734in}%
\pgfsys@useobject{currentmarker}{}%
\end{pgfscope}%
\begin{pgfscope}%
\pgfsys@transformshift{2.312134in}{0.943734in}%
\pgfsys@useobject{currentmarker}{}%
\end{pgfscope}%
\begin{pgfscope}%
\pgfsys@transformshift{2.312134in}{0.943734in}%
\pgfsys@useobject{currentmarker}{}%
\end{pgfscope}%
\begin{pgfscope}%
\pgfsys@transformshift{2.312134in}{0.943734in}%
\pgfsys@useobject{currentmarker}{}%
\end{pgfscope}%
\begin{pgfscope}%
\pgfsys@transformshift{2.312134in}{0.943734in}%
\pgfsys@useobject{currentmarker}{}%
\end{pgfscope}%
\begin{pgfscope}%
\pgfsys@transformshift{2.312134in}{0.943734in}%
\pgfsys@useobject{currentmarker}{}%
\end{pgfscope}%
\begin{pgfscope}%
\pgfsys@transformshift{2.312134in}{0.943734in}%
\pgfsys@useobject{currentmarker}{}%
\end{pgfscope}%
\begin{pgfscope}%
\pgfsys@transformshift{2.312134in}{0.943734in}%
\pgfsys@useobject{currentmarker}{}%
\end{pgfscope}%
\begin{pgfscope}%
\pgfsys@transformshift{2.312134in}{0.943734in}%
\pgfsys@useobject{currentmarker}{}%
\end{pgfscope}%
\begin{pgfscope}%
\pgfsys@transformshift{2.312134in}{0.943734in}%
\pgfsys@useobject{currentmarker}{}%
\end{pgfscope}%
\begin{pgfscope}%
\pgfsys@transformshift{2.312134in}{0.943734in}%
\pgfsys@useobject{currentmarker}{}%
\end{pgfscope}%
\begin{pgfscope}%
\pgfsys@transformshift{2.312134in}{0.943734in}%
\pgfsys@useobject{currentmarker}{}%
\end{pgfscope}%
\begin{pgfscope}%
\pgfsys@transformshift{2.312134in}{0.943734in}%
\pgfsys@useobject{currentmarker}{}%
\end{pgfscope}%
\begin{pgfscope}%
\pgfsys@transformshift{2.312134in}{0.943734in}%
\pgfsys@useobject{currentmarker}{}%
\end{pgfscope}%
\begin{pgfscope}%
\pgfsys@transformshift{2.312134in}{0.943734in}%
\pgfsys@useobject{currentmarker}{}%
\end{pgfscope}%
\begin{pgfscope}%
\pgfsys@transformshift{2.312134in}{0.943734in}%
\pgfsys@useobject{currentmarker}{}%
\end{pgfscope}%
\begin{pgfscope}%
\pgfsys@transformshift{2.312134in}{0.943734in}%
\pgfsys@useobject{currentmarker}{}%
\end{pgfscope}%
\begin{pgfscope}%
\pgfsys@transformshift{2.312134in}{0.943734in}%
\pgfsys@useobject{currentmarker}{}%
\end{pgfscope}%
\begin{pgfscope}%
\pgfsys@transformshift{2.312134in}{0.943734in}%
\pgfsys@useobject{currentmarker}{}%
\end{pgfscope}%
\begin{pgfscope}%
\pgfsys@transformshift{2.312134in}{0.943734in}%
\pgfsys@useobject{currentmarker}{}%
\end{pgfscope}%
\begin{pgfscope}%
\pgfsys@transformshift{2.312134in}{0.943734in}%
\pgfsys@useobject{currentmarker}{}%
\end{pgfscope}%
\begin{pgfscope}%
\pgfsys@transformshift{2.312134in}{0.943734in}%
\pgfsys@useobject{currentmarker}{}%
\end{pgfscope}%
\begin{pgfscope}%
\pgfsys@transformshift{2.312134in}{0.943734in}%
\pgfsys@useobject{currentmarker}{}%
\end{pgfscope}%
\begin{pgfscope}%
\pgfsys@transformshift{2.312134in}{0.943734in}%
\pgfsys@useobject{currentmarker}{}%
\end{pgfscope}%
\begin{pgfscope}%
\pgfsys@transformshift{2.312134in}{0.943734in}%
\pgfsys@useobject{currentmarker}{}%
\end{pgfscope}%
\begin{pgfscope}%
\pgfsys@transformshift{2.312134in}{0.943734in}%
\pgfsys@useobject{currentmarker}{}%
\end{pgfscope}%
\begin{pgfscope}%
\pgfsys@transformshift{2.312134in}{0.943734in}%
\pgfsys@useobject{currentmarker}{}%
\end{pgfscope}%
\begin{pgfscope}%
\pgfsys@transformshift{2.312134in}{0.943734in}%
\pgfsys@useobject{currentmarker}{}%
\end{pgfscope}%
\begin{pgfscope}%
\pgfsys@transformshift{2.312134in}{0.943734in}%
\pgfsys@useobject{currentmarker}{}%
\end{pgfscope}%
\begin{pgfscope}%
\pgfsys@transformshift{2.312134in}{0.943734in}%
\pgfsys@useobject{currentmarker}{}%
\end{pgfscope}%
\begin{pgfscope}%
\pgfsys@transformshift{2.312134in}{0.943734in}%
\pgfsys@useobject{currentmarker}{}%
\end{pgfscope}%
\begin{pgfscope}%
\pgfsys@transformshift{2.312134in}{0.943734in}%
\pgfsys@useobject{currentmarker}{}%
\end{pgfscope}%
\begin{pgfscope}%
\pgfsys@transformshift{2.312134in}{0.943734in}%
\pgfsys@useobject{currentmarker}{}%
\end{pgfscope}%
\begin{pgfscope}%
\pgfsys@transformshift{2.312134in}{0.943734in}%
\pgfsys@useobject{currentmarker}{}%
\end{pgfscope}%
\begin{pgfscope}%
\pgfsys@transformshift{2.312134in}{0.943734in}%
\pgfsys@useobject{currentmarker}{}%
\end{pgfscope}%
\begin{pgfscope}%
\pgfsys@transformshift{2.312134in}{0.943734in}%
\pgfsys@useobject{currentmarker}{}%
\end{pgfscope}%
\begin{pgfscope}%
\pgfsys@transformshift{2.312134in}{0.943734in}%
\pgfsys@useobject{currentmarker}{}%
\end{pgfscope}%
\begin{pgfscope}%
\pgfsys@transformshift{2.312134in}{0.943734in}%
\pgfsys@useobject{currentmarker}{}%
\end{pgfscope}%
\begin{pgfscope}%
\pgfsys@transformshift{2.312134in}{0.943734in}%
\pgfsys@useobject{currentmarker}{}%
\end{pgfscope}%
\begin{pgfscope}%
\pgfsys@transformshift{2.312134in}{0.943734in}%
\pgfsys@useobject{currentmarker}{}%
\end{pgfscope}%
\begin{pgfscope}%
\pgfsys@transformshift{2.312134in}{0.943734in}%
\pgfsys@useobject{currentmarker}{}%
\end{pgfscope}%
\begin{pgfscope}%
\pgfsys@transformshift{2.312134in}{0.943734in}%
\pgfsys@useobject{currentmarker}{}%
\end{pgfscope}%
\begin{pgfscope}%
\pgfsys@transformshift{2.312134in}{0.943734in}%
\pgfsys@useobject{currentmarker}{}%
\end{pgfscope}%
\begin{pgfscope}%
\pgfsys@transformshift{2.312134in}{0.943734in}%
\pgfsys@useobject{currentmarker}{}%
\end{pgfscope}%
\begin{pgfscope}%
\pgfsys@transformshift{2.312134in}{0.943734in}%
\pgfsys@useobject{currentmarker}{}%
\end{pgfscope}%
\begin{pgfscope}%
\pgfsys@transformshift{2.312134in}{0.943734in}%
\pgfsys@useobject{currentmarker}{}%
\end{pgfscope}%
\begin{pgfscope}%
\pgfsys@transformshift{2.312134in}{0.943734in}%
\pgfsys@useobject{currentmarker}{}%
\end{pgfscope}%
\begin{pgfscope}%
\pgfsys@transformshift{2.312134in}{0.943734in}%
\pgfsys@useobject{currentmarker}{}%
\end{pgfscope}%
\begin{pgfscope}%
\pgfsys@transformshift{2.312134in}{0.943734in}%
\pgfsys@useobject{currentmarker}{}%
\end{pgfscope}%
\begin{pgfscope}%
\pgfsys@transformshift{2.312134in}{0.943734in}%
\pgfsys@useobject{currentmarker}{}%
\end{pgfscope}%
\begin{pgfscope}%
\pgfsys@transformshift{2.312134in}{0.943734in}%
\pgfsys@useobject{currentmarker}{}%
\end{pgfscope}%
\begin{pgfscope}%
\pgfsys@transformshift{2.312134in}{0.943734in}%
\pgfsys@useobject{currentmarker}{}%
\end{pgfscope}%
\begin{pgfscope}%
\pgfsys@transformshift{2.312134in}{0.943734in}%
\pgfsys@useobject{currentmarker}{}%
\end{pgfscope}%
\begin{pgfscope}%
\pgfsys@transformshift{2.312134in}{0.943734in}%
\pgfsys@useobject{currentmarker}{}%
\end{pgfscope}%
\begin{pgfscope}%
\pgfsys@transformshift{2.312134in}{0.943734in}%
\pgfsys@useobject{currentmarker}{}%
\end{pgfscope}%
\begin{pgfscope}%
\pgfsys@transformshift{2.312134in}{0.943734in}%
\pgfsys@useobject{currentmarker}{}%
\end{pgfscope}%
\begin{pgfscope}%
\pgfsys@transformshift{2.312134in}{0.943734in}%
\pgfsys@useobject{currentmarker}{}%
\end{pgfscope}%
\begin{pgfscope}%
\pgfsys@transformshift{2.312134in}{0.943734in}%
\pgfsys@useobject{currentmarker}{}%
\end{pgfscope}%
\begin{pgfscope}%
\pgfsys@transformshift{2.312134in}{0.943734in}%
\pgfsys@useobject{currentmarker}{}%
\end{pgfscope}%
\begin{pgfscope}%
\pgfsys@transformshift{2.312134in}{0.943734in}%
\pgfsys@useobject{currentmarker}{}%
\end{pgfscope}%
\begin{pgfscope}%
\pgfsys@transformshift{2.312134in}{0.943734in}%
\pgfsys@useobject{currentmarker}{}%
\end{pgfscope}%
\begin{pgfscope}%
\pgfsys@transformshift{2.312134in}{0.943734in}%
\pgfsys@useobject{currentmarker}{}%
\end{pgfscope}%
\begin{pgfscope}%
\pgfsys@transformshift{2.312134in}{0.943734in}%
\pgfsys@useobject{currentmarker}{}%
\end{pgfscope}%
\begin{pgfscope}%
\pgfsys@transformshift{2.312134in}{0.943734in}%
\pgfsys@useobject{currentmarker}{}%
\end{pgfscope}%
\begin{pgfscope}%
\pgfsys@transformshift{2.312134in}{0.943734in}%
\pgfsys@useobject{currentmarker}{}%
\end{pgfscope}%
\begin{pgfscope}%
\pgfsys@transformshift{2.312134in}{0.943734in}%
\pgfsys@useobject{currentmarker}{}%
\end{pgfscope}%
\begin{pgfscope}%
\pgfsys@transformshift{2.312134in}{0.943734in}%
\pgfsys@useobject{currentmarker}{}%
\end{pgfscope}%
\begin{pgfscope}%
\pgfsys@transformshift{2.312134in}{0.943734in}%
\pgfsys@useobject{currentmarker}{}%
\end{pgfscope}%
\begin{pgfscope}%
\pgfsys@transformshift{2.312134in}{0.943734in}%
\pgfsys@useobject{currentmarker}{}%
\end{pgfscope}%
\begin{pgfscope}%
\pgfsys@transformshift{2.312134in}{0.943734in}%
\pgfsys@useobject{currentmarker}{}%
\end{pgfscope}%
\begin{pgfscope}%
\pgfsys@transformshift{2.312134in}{0.943734in}%
\pgfsys@useobject{currentmarker}{}%
\end{pgfscope}%
\begin{pgfscope}%
\pgfsys@transformshift{2.312134in}{0.943734in}%
\pgfsys@useobject{currentmarker}{}%
\end{pgfscope}%
\begin{pgfscope}%
\pgfsys@transformshift{2.312134in}{0.943734in}%
\pgfsys@useobject{currentmarker}{}%
\end{pgfscope}%
\begin{pgfscope}%
\pgfsys@transformshift{2.312134in}{0.943734in}%
\pgfsys@useobject{currentmarker}{}%
\end{pgfscope}%
\begin{pgfscope}%
\pgfsys@transformshift{2.312134in}{0.943734in}%
\pgfsys@useobject{currentmarker}{}%
\end{pgfscope}%
\begin{pgfscope}%
\pgfsys@transformshift{2.312134in}{0.943734in}%
\pgfsys@useobject{currentmarker}{}%
\end{pgfscope}%
\begin{pgfscope}%
\pgfsys@transformshift{2.312134in}{0.943734in}%
\pgfsys@useobject{currentmarker}{}%
\end{pgfscope}%
\begin{pgfscope}%
\pgfsys@transformshift{2.312134in}{0.943734in}%
\pgfsys@useobject{currentmarker}{}%
\end{pgfscope}%
\begin{pgfscope}%
\pgfsys@transformshift{2.312134in}{0.943734in}%
\pgfsys@useobject{currentmarker}{}%
\end{pgfscope}%
\begin{pgfscope}%
\pgfsys@transformshift{2.312134in}{0.943734in}%
\pgfsys@useobject{currentmarker}{}%
\end{pgfscope}%
\begin{pgfscope}%
\pgfsys@transformshift{2.312134in}{0.943734in}%
\pgfsys@useobject{currentmarker}{}%
\end{pgfscope}%
\begin{pgfscope}%
\pgfsys@transformshift{2.312134in}{0.943734in}%
\pgfsys@useobject{currentmarker}{}%
\end{pgfscope}%
\begin{pgfscope}%
\pgfsys@transformshift{2.312134in}{0.943734in}%
\pgfsys@useobject{currentmarker}{}%
\end{pgfscope}%
\begin{pgfscope}%
\pgfsys@transformshift{2.312134in}{0.943734in}%
\pgfsys@useobject{currentmarker}{}%
\end{pgfscope}%
\begin{pgfscope}%
\pgfsys@transformshift{2.312134in}{0.943734in}%
\pgfsys@useobject{currentmarker}{}%
\end{pgfscope}%
\begin{pgfscope}%
\pgfsys@transformshift{2.312134in}{0.943734in}%
\pgfsys@useobject{currentmarker}{}%
\end{pgfscope}%
\begin{pgfscope}%
\pgfsys@transformshift{2.312134in}{0.943734in}%
\pgfsys@useobject{currentmarker}{}%
\end{pgfscope}%
\begin{pgfscope}%
\pgfsys@transformshift{2.312134in}{0.943734in}%
\pgfsys@useobject{currentmarker}{}%
\end{pgfscope}%
\begin{pgfscope}%
\pgfsys@transformshift{2.312134in}{0.943734in}%
\pgfsys@useobject{currentmarker}{}%
\end{pgfscope}%
\begin{pgfscope}%
\pgfsys@transformshift{2.312134in}{0.943734in}%
\pgfsys@useobject{currentmarker}{}%
\end{pgfscope}%
\begin{pgfscope}%
\pgfsys@transformshift{2.312134in}{0.943734in}%
\pgfsys@useobject{currentmarker}{}%
\end{pgfscope}%
\begin{pgfscope}%
\pgfsys@transformshift{2.312134in}{0.943734in}%
\pgfsys@useobject{currentmarker}{}%
\end{pgfscope}%
\begin{pgfscope}%
\pgfsys@transformshift{2.312134in}{0.943734in}%
\pgfsys@useobject{currentmarker}{}%
\end{pgfscope}%
\begin{pgfscope}%
\pgfsys@transformshift{2.312134in}{0.943734in}%
\pgfsys@useobject{currentmarker}{}%
\end{pgfscope}%
\begin{pgfscope}%
\pgfsys@transformshift{2.312134in}{0.943734in}%
\pgfsys@useobject{currentmarker}{}%
\end{pgfscope}%
\begin{pgfscope}%
\pgfsys@transformshift{2.312134in}{0.943734in}%
\pgfsys@useobject{currentmarker}{}%
\end{pgfscope}%
\begin{pgfscope}%
\pgfsys@transformshift{2.312134in}{0.943734in}%
\pgfsys@useobject{currentmarker}{}%
\end{pgfscope}%
\begin{pgfscope}%
\pgfsys@transformshift{2.312134in}{0.943734in}%
\pgfsys@useobject{currentmarker}{}%
\end{pgfscope}%
\begin{pgfscope}%
\pgfsys@transformshift{2.312134in}{0.943734in}%
\pgfsys@useobject{currentmarker}{}%
\end{pgfscope}%
\begin{pgfscope}%
\pgfsys@transformshift{2.312134in}{0.943734in}%
\pgfsys@useobject{currentmarker}{}%
\end{pgfscope}%
\begin{pgfscope}%
\pgfsys@transformshift{2.312134in}{0.943734in}%
\pgfsys@useobject{currentmarker}{}%
\end{pgfscope}%
\begin{pgfscope}%
\pgfsys@transformshift{2.312134in}{0.943734in}%
\pgfsys@useobject{currentmarker}{}%
\end{pgfscope}%
\begin{pgfscope}%
\pgfsys@transformshift{2.312134in}{0.943734in}%
\pgfsys@useobject{currentmarker}{}%
\end{pgfscope}%
\begin{pgfscope}%
\pgfsys@transformshift{2.312134in}{0.943734in}%
\pgfsys@useobject{currentmarker}{}%
\end{pgfscope}%
\begin{pgfscope}%
\pgfsys@transformshift{2.312134in}{0.943734in}%
\pgfsys@useobject{currentmarker}{}%
\end{pgfscope}%
\begin{pgfscope}%
\pgfsys@transformshift{2.312134in}{0.943734in}%
\pgfsys@useobject{currentmarker}{}%
\end{pgfscope}%
\begin{pgfscope}%
\pgfsys@transformshift{2.312134in}{0.943734in}%
\pgfsys@useobject{currentmarker}{}%
\end{pgfscope}%
\begin{pgfscope}%
\pgfsys@transformshift{2.312134in}{0.943734in}%
\pgfsys@useobject{currentmarker}{}%
\end{pgfscope}%
\begin{pgfscope}%
\pgfsys@transformshift{2.312134in}{0.943734in}%
\pgfsys@useobject{currentmarker}{}%
\end{pgfscope}%
\begin{pgfscope}%
\pgfsys@transformshift{2.312134in}{0.943734in}%
\pgfsys@useobject{currentmarker}{}%
\end{pgfscope}%
\begin{pgfscope}%
\pgfsys@transformshift{2.312134in}{0.943734in}%
\pgfsys@useobject{currentmarker}{}%
\end{pgfscope}%
\begin{pgfscope}%
\pgfsys@transformshift{2.312134in}{0.943734in}%
\pgfsys@useobject{currentmarker}{}%
\end{pgfscope}%
\begin{pgfscope}%
\pgfsys@transformshift{2.312134in}{0.943734in}%
\pgfsys@useobject{currentmarker}{}%
\end{pgfscope}%
\begin{pgfscope}%
\pgfsys@transformshift{2.312134in}{0.943734in}%
\pgfsys@useobject{currentmarker}{}%
\end{pgfscope}%
\begin{pgfscope}%
\pgfsys@transformshift{2.312134in}{0.943734in}%
\pgfsys@useobject{currentmarker}{}%
\end{pgfscope}%
\begin{pgfscope}%
\pgfsys@transformshift{2.312134in}{0.943734in}%
\pgfsys@useobject{currentmarker}{}%
\end{pgfscope}%
\begin{pgfscope}%
\pgfsys@transformshift{2.312134in}{0.943734in}%
\pgfsys@useobject{currentmarker}{}%
\end{pgfscope}%
\begin{pgfscope}%
\pgfsys@transformshift{2.312134in}{0.943734in}%
\pgfsys@useobject{currentmarker}{}%
\end{pgfscope}%
\begin{pgfscope}%
\pgfsys@transformshift{2.312134in}{0.943734in}%
\pgfsys@useobject{currentmarker}{}%
\end{pgfscope}%
\begin{pgfscope}%
\pgfsys@transformshift{2.312134in}{0.943734in}%
\pgfsys@useobject{currentmarker}{}%
\end{pgfscope}%
\begin{pgfscope}%
\pgfsys@transformshift{2.312134in}{0.943734in}%
\pgfsys@useobject{currentmarker}{}%
\end{pgfscope}%
\begin{pgfscope}%
\pgfsys@transformshift{2.312134in}{0.943734in}%
\pgfsys@useobject{currentmarker}{}%
\end{pgfscope}%
\begin{pgfscope}%
\pgfsys@transformshift{2.312134in}{0.943734in}%
\pgfsys@useobject{currentmarker}{}%
\end{pgfscope}%
\begin{pgfscope}%
\pgfsys@transformshift{2.312134in}{0.943734in}%
\pgfsys@useobject{currentmarker}{}%
\end{pgfscope}%
\begin{pgfscope}%
\pgfsys@transformshift{2.312134in}{0.943734in}%
\pgfsys@useobject{currentmarker}{}%
\end{pgfscope}%
\begin{pgfscope}%
\pgfsys@transformshift{2.312134in}{0.943734in}%
\pgfsys@useobject{currentmarker}{}%
\end{pgfscope}%
\begin{pgfscope}%
\pgfsys@transformshift{2.312134in}{0.943734in}%
\pgfsys@useobject{currentmarker}{}%
\end{pgfscope}%
\begin{pgfscope}%
\pgfsys@transformshift{2.312134in}{0.943734in}%
\pgfsys@useobject{currentmarker}{}%
\end{pgfscope}%
\begin{pgfscope}%
\pgfsys@transformshift{2.312134in}{0.943734in}%
\pgfsys@useobject{currentmarker}{}%
\end{pgfscope}%
\begin{pgfscope}%
\pgfsys@transformshift{2.312134in}{0.943734in}%
\pgfsys@useobject{currentmarker}{}%
\end{pgfscope}%
\begin{pgfscope}%
\pgfsys@transformshift{2.312134in}{0.943734in}%
\pgfsys@useobject{currentmarker}{}%
\end{pgfscope}%
\begin{pgfscope}%
\pgfsys@transformshift{2.312134in}{0.943734in}%
\pgfsys@useobject{currentmarker}{}%
\end{pgfscope}%
\begin{pgfscope}%
\pgfsys@transformshift{2.312134in}{0.943734in}%
\pgfsys@useobject{currentmarker}{}%
\end{pgfscope}%
\begin{pgfscope}%
\pgfsys@transformshift{2.312134in}{0.943734in}%
\pgfsys@useobject{currentmarker}{}%
\end{pgfscope}%
\begin{pgfscope}%
\pgfsys@transformshift{2.312134in}{0.943734in}%
\pgfsys@useobject{currentmarker}{}%
\end{pgfscope}%
\begin{pgfscope}%
\pgfsys@transformshift{2.312134in}{0.943734in}%
\pgfsys@useobject{currentmarker}{}%
\end{pgfscope}%
\begin{pgfscope}%
\pgfsys@transformshift{2.312134in}{0.943734in}%
\pgfsys@useobject{currentmarker}{}%
\end{pgfscope}%
\begin{pgfscope}%
\pgfsys@transformshift{2.312134in}{0.943734in}%
\pgfsys@useobject{currentmarker}{}%
\end{pgfscope}%
\begin{pgfscope}%
\pgfsys@transformshift{2.312134in}{0.943734in}%
\pgfsys@useobject{currentmarker}{}%
\end{pgfscope}%
\begin{pgfscope}%
\pgfsys@transformshift{2.312134in}{0.943734in}%
\pgfsys@useobject{currentmarker}{}%
\end{pgfscope}%
\begin{pgfscope}%
\pgfsys@transformshift{2.312134in}{0.943734in}%
\pgfsys@useobject{currentmarker}{}%
\end{pgfscope}%
\begin{pgfscope}%
\pgfsys@transformshift{2.312134in}{0.943734in}%
\pgfsys@useobject{currentmarker}{}%
\end{pgfscope}%
\begin{pgfscope}%
\pgfsys@transformshift{2.312134in}{0.943734in}%
\pgfsys@useobject{currentmarker}{}%
\end{pgfscope}%
\begin{pgfscope}%
\pgfsys@transformshift{2.312134in}{0.943734in}%
\pgfsys@useobject{currentmarker}{}%
\end{pgfscope}%
\begin{pgfscope}%
\pgfsys@transformshift{2.312134in}{0.943734in}%
\pgfsys@useobject{currentmarker}{}%
\end{pgfscope}%
\begin{pgfscope}%
\pgfsys@transformshift{2.312134in}{0.943734in}%
\pgfsys@useobject{currentmarker}{}%
\end{pgfscope}%
\begin{pgfscope}%
\pgfsys@transformshift{2.312134in}{0.943734in}%
\pgfsys@useobject{currentmarker}{}%
\end{pgfscope}%
\begin{pgfscope}%
\pgfsys@transformshift{2.312134in}{0.943734in}%
\pgfsys@useobject{currentmarker}{}%
\end{pgfscope}%
\begin{pgfscope}%
\pgfsys@transformshift{2.312134in}{0.943734in}%
\pgfsys@useobject{currentmarker}{}%
\end{pgfscope}%
\begin{pgfscope}%
\pgfsys@transformshift{2.312134in}{0.943734in}%
\pgfsys@useobject{currentmarker}{}%
\end{pgfscope}%
\begin{pgfscope}%
\pgfsys@transformshift{2.312134in}{0.943734in}%
\pgfsys@useobject{currentmarker}{}%
\end{pgfscope}%
\begin{pgfscope}%
\pgfsys@transformshift{2.312134in}{0.943734in}%
\pgfsys@useobject{currentmarker}{}%
\end{pgfscope}%
\begin{pgfscope}%
\pgfsys@transformshift{2.312134in}{0.943734in}%
\pgfsys@useobject{currentmarker}{}%
\end{pgfscope}%
\begin{pgfscope}%
\pgfsys@transformshift{2.312134in}{0.943734in}%
\pgfsys@useobject{currentmarker}{}%
\end{pgfscope}%
\begin{pgfscope}%
\pgfsys@transformshift{2.312134in}{0.943734in}%
\pgfsys@useobject{currentmarker}{}%
\end{pgfscope}%
\begin{pgfscope}%
\pgfsys@transformshift{2.312134in}{0.943734in}%
\pgfsys@useobject{currentmarker}{}%
\end{pgfscope}%
\begin{pgfscope}%
\pgfsys@transformshift{2.312134in}{0.943734in}%
\pgfsys@useobject{currentmarker}{}%
\end{pgfscope}%
\begin{pgfscope}%
\pgfsys@transformshift{2.312134in}{0.943734in}%
\pgfsys@useobject{currentmarker}{}%
\end{pgfscope}%
\begin{pgfscope}%
\pgfsys@transformshift{2.312134in}{0.943734in}%
\pgfsys@useobject{currentmarker}{}%
\end{pgfscope}%
\begin{pgfscope}%
\pgfsys@transformshift{2.312134in}{0.943734in}%
\pgfsys@useobject{currentmarker}{}%
\end{pgfscope}%
\begin{pgfscope}%
\pgfsys@transformshift{2.312134in}{0.943734in}%
\pgfsys@useobject{currentmarker}{}%
\end{pgfscope}%
\begin{pgfscope}%
\pgfsys@transformshift{2.312134in}{0.943734in}%
\pgfsys@useobject{currentmarker}{}%
\end{pgfscope}%
\begin{pgfscope}%
\pgfsys@transformshift{2.312134in}{0.943734in}%
\pgfsys@useobject{currentmarker}{}%
\end{pgfscope}%
\begin{pgfscope}%
\pgfsys@transformshift{2.312134in}{0.943734in}%
\pgfsys@useobject{currentmarker}{}%
\end{pgfscope}%
\begin{pgfscope}%
\pgfsys@transformshift{2.312134in}{0.943734in}%
\pgfsys@useobject{currentmarker}{}%
\end{pgfscope}%
\begin{pgfscope}%
\pgfsys@transformshift{2.312134in}{0.943734in}%
\pgfsys@useobject{currentmarker}{}%
\end{pgfscope}%
\begin{pgfscope}%
\pgfsys@transformshift{2.312134in}{0.943734in}%
\pgfsys@useobject{currentmarker}{}%
\end{pgfscope}%
\begin{pgfscope}%
\pgfsys@transformshift{2.312134in}{0.943734in}%
\pgfsys@useobject{currentmarker}{}%
\end{pgfscope}%
\begin{pgfscope}%
\pgfsys@transformshift{2.312134in}{0.943734in}%
\pgfsys@useobject{currentmarker}{}%
\end{pgfscope}%
\begin{pgfscope}%
\pgfsys@transformshift{2.312134in}{0.943734in}%
\pgfsys@useobject{currentmarker}{}%
\end{pgfscope}%
\begin{pgfscope}%
\pgfsys@transformshift{2.312134in}{0.943734in}%
\pgfsys@useobject{currentmarker}{}%
\end{pgfscope}%
\begin{pgfscope}%
\pgfsys@transformshift{2.312134in}{0.943734in}%
\pgfsys@useobject{currentmarker}{}%
\end{pgfscope}%
\begin{pgfscope}%
\pgfsys@transformshift{2.312134in}{0.943734in}%
\pgfsys@useobject{currentmarker}{}%
\end{pgfscope}%
\begin{pgfscope}%
\pgfsys@transformshift{2.312134in}{0.943734in}%
\pgfsys@useobject{currentmarker}{}%
\end{pgfscope}%
\begin{pgfscope}%
\pgfsys@transformshift{2.312134in}{0.943734in}%
\pgfsys@useobject{currentmarker}{}%
\end{pgfscope}%
\begin{pgfscope}%
\pgfsys@transformshift{2.312134in}{0.943734in}%
\pgfsys@useobject{currentmarker}{}%
\end{pgfscope}%
\begin{pgfscope}%
\pgfsys@transformshift{2.312134in}{0.943734in}%
\pgfsys@useobject{currentmarker}{}%
\end{pgfscope}%
\begin{pgfscope}%
\pgfsys@transformshift{2.312134in}{0.943734in}%
\pgfsys@useobject{currentmarker}{}%
\end{pgfscope}%
\begin{pgfscope}%
\pgfsys@transformshift{2.312134in}{0.943734in}%
\pgfsys@useobject{currentmarker}{}%
\end{pgfscope}%
\begin{pgfscope}%
\pgfsys@transformshift{2.312134in}{0.943734in}%
\pgfsys@useobject{currentmarker}{}%
\end{pgfscope}%
\begin{pgfscope}%
\pgfsys@transformshift{2.312134in}{0.943734in}%
\pgfsys@useobject{currentmarker}{}%
\end{pgfscope}%
\begin{pgfscope}%
\pgfsys@transformshift{2.312134in}{0.943734in}%
\pgfsys@useobject{currentmarker}{}%
\end{pgfscope}%
\begin{pgfscope}%
\pgfsys@transformshift{2.312134in}{0.943734in}%
\pgfsys@useobject{currentmarker}{}%
\end{pgfscope}%
\begin{pgfscope}%
\pgfsys@transformshift{2.312134in}{0.943734in}%
\pgfsys@useobject{currentmarker}{}%
\end{pgfscope}%
\begin{pgfscope}%
\pgfsys@transformshift{2.312134in}{0.943734in}%
\pgfsys@useobject{currentmarker}{}%
\end{pgfscope}%
\begin{pgfscope}%
\pgfsys@transformshift{2.312134in}{0.943734in}%
\pgfsys@useobject{currentmarker}{}%
\end{pgfscope}%
\begin{pgfscope}%
\pgfsys@transformshift{2.312134in}{0.943734in}%
\pgfsys@useobject{currentmarker}{}%
\end{pgfscope}%
\begin{pgfscope}%
\pgfsys@transformshift{2.312134in}{0.943734in}%
\pgfsys@useobject{currentmarker}{}%
\end{pgfscope}%
\begin{pgfscope}%
\pgfsys@transformshift{2.312134in}{0.943734in}%
\pgfsys@useobject{currentmarker}{}%
\end{pgfscope}%
\begin{pgfscope}%
\pgfsys@transformshift{2.312134in}{0.943734in}%
\pgfsys@useobject{currentmarker}{}%
\end{pgfscope}%
\begin{pgfscope}%
\pgfsys@transformshift{2.312134in}{0.943734in}%
\pgfsys@useobject{currentmarker}{}%
\end{pgfscope}%
\begin{pgfscope}%
\pgfsys@transformshift{2.312134in}{0.943734in}%
\pgfsys@useobject{currentmarker}{}%
\end{pgfscope}%
\begin{pgfscope}%
\pgfsys@transformshift{2.312134in}{0.943734in}%
\pgfsys@useobject{currentmarker}{}%
\end{pgfscope}%
\begin{pgfscope}%
\pgfsys@transformshift{2.312134in}{0.943734in}%
\pgfsys@useobject{currentmarker}{}%
\end{pgfscope}%
\begin{pgfscope}%
\pgfsys@transformshift{2.312134in}{0.943734in}%
\pgfsys@useobject{currentmarker}{}%
\end{pgfscope}%
\begin{pgfscope}%
\pgfsys@transformshift{2.312134in}{0.943734in}%
\pgfsys@useobject{currentmarker}{}%
\end{pgfscope}%
\begin{pgfscope}%
\pgfsys@transformshift{2.312134in}{0.943734in}%
\pgfsys@useobject{currentmarker}{}%
\end{pgfscope}%
\begin{pgfscope}%
\pgfsys@transformshift{2.312134in}{0.943734in}%
\pgfsys@useobject{currentmarker}{}%
\end{pgfscope}%
\begin{pgfscope}%
\pgfsys@transformshift{2.312134in}{0.943734in}%
\pgfsys@useobject{currentmarker}{}%
\end{pgfscope}%
\begin{pgfscope}%
\pgfsys@transformshift{2.312134in}{0.943734in}%
\pgfsys@useobject{currentmarker}{}%
\end{pgfscope}%
\begin{pgfscope}%
\pgfsys@transformshift{2.312134in}{0.943734in}%
\pgfsys@useobject{currentmarker}{}%
\end{pgfscope}%
\begin{pgfscope}%
\pgfsys@transformshift{2.312134in}{0.943734in}%
\pgfsys@useobject{currentmarker}{}%
\end{pgfscope}%
\begin{pgfscope}%
\pgfsys@transformshift{2.312134in}{0.943734in}%
\pgfsys@useobject{currentmarker}{}%
\end{pgfscope}%
\begin{pgfscope}%
\pgfsys@transformshift{2.312134in}{0.943734in}%
\pgfsys@useobject{currentmarker}{}%
\end{pgfscope}%
\begin{pgfscope}%
\pgfsys@transformshift{2.312134in}{0.943734in}%
\pgfsys@useobject{currentmarker}{}%
\end{pgfscope}%
\begin{pgfscope}%
\pgfsys@transformshift{2.312134in}{0.943734in}%
\pgfsys@useobject{currentmarker}{}%
\end{pgfscope}%
\begin{pgfscope}%
\pgfsys@transformshift{2.312134in}{0.943734in}%
\pgfsys@useobject{currentmarker}{}%
\end{pgfscope}%
\begin{pgfscope}%
\pgfsys@transformshift{2.312134in}{0.943734in}%
\pgfsys@useobject{currentmarker}{}%
\end{pgfscope}%
\begin{pgfscope}%
\pgfsys@transformshift{2.312134in}{0.943734in}%
\pgfsys@useobject{currentmarker}{}%
\end{pgfscope}%
\begin{pgfscope}%
\pgfsys@transformshift{2.312134in}{0.943734in}%
\pgfsys@useobject{currentmarker}{}%
\end{pgfscope}%
\begin{pgfscope}%
\pgfsys@transformshift{2.312134in}{0.943734in}%
\pgfsys@useobject{currentmarker}{}%
\end{pgfscope}%
\begin{pgfscope}%
\pgfsys@transformshift{2.312134in}{0.943734in}%
\pgfsys@useobject{currentmarker}{}%
\end{pgfscope}%
\begin{pgfscope}%
\pgfsys@transformshift{2.312134in}{0.943734in}%
\pgfsys@useobject{currentmarker}{}%
\end{pgfscope}%
\begin{pgfscope}%
\pgfsys@transformshift{2.312134in}{0.943734in}%
\pgfsys@useobject{currentmarker}{}%
\end{pgfscope}%
\begin{pgfscope}%
\pgfsys@transformshift{2.312134in}{0.943734in}%
\pgfsys@useobject{currentmarker}{}%
\end{pgfscope}%
\begin{pgfscope}%
\pgfsys@transformshift{2.312134in}{0.943734in}%
\pgfsys@useobject{currentmarker}{}%
\end{pgfscope}%
\begin{pgfscope}%
\pgfsys@transformshift{2.312134in}{0.943734in}%
\pgfsys@useobject{currentmarker}{}%
\end{pgfscope}%
\begin{pgfscope}%
\pgfsys@transformshift{2.312134in}{0.943734in}%
\pgfsys@useobject{currentmarker}{}%
\end{pgfscope}%
\begin{pgfscope}%
\pgfsys@transformshift{2.312134in}{0.943734in}%
\pgfsys@useobject{currentmarker}{}%
\end{pgfscope}%
\begin{pgfscope}%
\pgfsys@transformshift{2.312134in}{0.943734in}%
\pgfsys@useobject{currentmarker}{}%
\end{pgfscope}%
\begin{pgfscope}%
\pgfsys@transformshift{2.312134in}{0.943734in}%
\pgfsys@useobject{currentmarker}{}%
\end{pgfscope}%
\begin{pgfscope}%
\pgfsys@transformshift{2.312134in}{0.943734in}%
\pgfsys@useobject{currentmarker}{}%
\end{pgfscope}%
\begin{pgfscope}%
\pgfsys@transformshift{2.312134in}{0.943734in}%
\pgfsys@useobject{currentmarker}{}%
\end{pgfscope}%
\begin{pgfscope}%
\pgfsys@transformshift{2.312134in}{0.943734in}%
\pgfsys@useobject{currentmarker}{}%
\end{pgfscope}%
\begin{pgfscope}%
\pgfsys@transformshift{2.312134in}{0.943734in}%
\pgfsys@useobject{currentmarker}{}%
\end{pgfscope}%
\begin{pgfscope}%
\pgfsys@transformshift{2.312134in}{0.943734in}%
\pgfsys@useobject{currentmarker}{}%
\end{pgfscope}%
\begin{pgfscope}%
\pgfsys@transformshift{2.312134in}{0.943734in}%
\pgfsys@useobject{currentmarker}{}%
\end{pgfscope}%
\begin{pgfscope}%
\pgfsys@transformshift{2.312134in}{0.943734in}%
\pgfsys@useobject{currentmarker}{}%
\end{pgfscope}%
\begin{pgfscope}%
\pgfsys@transformshift{2.312134in}{0.943734in}%
\pgfsys@useobject{currentmarker}{}%
\end{pgfscope}%
\begin{pgfscope}%
\pgfsys@transformshift{2.312134in}{0.943734in}%
\pgfsys@useobject{currentmarker}{}%
\end{pgfscope}%
\begin{pgfscope}%
\pgfsys@transformshift{2.312134in}{0.943734in}%
\pgfsys@useobject{currentmarker}{}%
\end{pgfscope}%
\begin{pgfscope}%
\pgfsys@transformshift{2.312134in}{0.943734in}%
\pgfsys@useobject{currentmarker}{}%
\end{pgfscope}%
\begin{pgfscope}%
\pgfsys@transformshift{2.312134in}{0.943734in}%
\pgfsys@useobject{currentmarker}{}%
\end{pgfscope}%
\begin{pgfscope}%
\pgfsys@transformshift{2.312134in}{0.943734in}%
\pgfsys@useobject{currentmarker}{}%
\end{pgfscope}%
\begin{pgfscope}%
\pgfsys@transformshift{2.312134in}{0.943734in}%
\pgfsys@useobject{currentmarker}{}%
\end{pgfscope}%
\begin{pgfscope}%
\pgfsys@transformshift{2.312134in}{0.943734in}%
\pgfsys@useobject{currentmarker}{}%
\end{pgfscope}%
\begin{pgfscope}%
\pgfsys@transformshift{2.312134in}{0.943734in}%
\pgfsys@useobject{currentmarker}{}%
\end{pgfscope}%
\begin{pgfscope}%
\pgfsys@transformshift{2.312134in}{0.943734in}%
\pgfsys@useobject{currentmarker}{}%
\end{pgfscope}%
\begin{pgfscope}%
\pgfsys@transformshift{2.312134in}{0.943734in}%
\pgfsys@useobject{currentmarker}{}%
\end{pgfscope}%
\begin{pgfscope}%
\pgfsys@transformshift{2.312134in}{0.943734in}%
\pgfsys@useobject{currentmarker}{}%
\end{pgfscope}%
\begin{pgfscope}%
\pgfsys@transformshift{2.312134in}{0.943734in}%
\pgfsys@useobject{currentmarker}{}%
\end{pgfscope}%
\begin{pgfscope}%
\pgfsys@transformshift{2.312134in}{0.943734in}%
\pgfsys@useobject{currentmarker}{}%
\end{pgfscope}%
\begin{pgfscope}%
\pgfsys@transformshift{2.312134in}{0.943734in}%
\pgfsys@useobject{currentmarker}{}%
\end{pgfscope}%
\begin{pgfscope}%
\pgfsys@transformshift{2.312134in}{0.943734in}%
\pgfsys@useobject{currentmarker}{}%
\end{pgfscope}%
\begin{pgfscope}%
\pgfsys@transformshift{2.312134in}{0.943734in}%
\pgfsys@useobject{currentmarker}{}%
\end{pgfscope}%
\begin{pgfscope}%
\pgfsys@transformshift{2.312134in}{0.943734in}%
\pgfsys@useobject{currentmarker}{}%
\end{pgfscope}%
\begin{pgfscope}%
\pgfsys@transformshift{2.312134in}{0.943734in}%
\pgfsys@useobject{currentmarker}{}%
\end{pgfscope}%
\begin{pgfscope}%
\pgfsys@transformshift{2.312134in}{0.943734in}%
\pgfsys@useobject{currentmarker}{}%
\end{pgfscope}%
\begin{pgfscope}%
\pgfsys@transformshift{2.312134in}{0.943734in}%
\pgfsys@useobject{currentmarker}{}%
\end{pgfscope}%
\begin{pgfscope}%
\pgfsys@transformshift{2.312134in}{0.943734in}%
\pgfsys@useobject{currentmarker}{}%
\end{pgfscope}%
\begin{pgfscope}%
\pgfsys@transformshift{2.312134in}{0.943734in}%
\pgfsys@useobject{currentmarker}{}%
\end{pgfscope}%
\begin{pgfscope}%
\pgfsys@transformshift{2.312134in}{0.943734in}%
\pgfsys@useobject{currentmarker}{}%
\end{pgfscope}%
\begin{pgfscope}%
\pgfsys@transformshift{2.312134in}{0.943734in}%
\pgfsys@useobject{currentmarker}{}%
\end{pgfscope}%
\begin{pgfscope}%
\pgfsys@transformshift{2.312134in}{0.943734in}%
\pgfsys@useobject{currentmarker}{}%
\end{pgfscope}%
\begin{pgfscope}%
\pgfsys@transformshift{2.312134in}{0.943734in}%
\pgfsys@useobject{currentmarker}{}%
\end{pgfscope}%
\begin{pgfscope}%
\pgfsys@transformshift{2.312134in}{0.943734in}%
\pgfsys@useobject{currentmarker}{}%
\end{pgfscope}%
\begin{pgfscope}%
\pgfsys@transformshift{2.312134in}{0.943734in}%
\pgfsys@useobject{currentmarker}{}%
\end{pgfscope}%
\begin{pgfscope}%
\pgfsys@transformshift{2.312134in}{0.943734in}%
\pgfsys@useobject{currentmarker}{}%
\end{pgfscope}%
\begin{pgfscope}%
\pgfsys@transformshift{2.312134in}{0.943734in}%
\pgfsys@useobject{currentmarker}{}%
\end{pgfscope}%
\begin{pgfscope}%
\pgfsys@transformshift{2.312134in}{0.943734in}%
\pgfsys@useobject{currentmarker}{}%
\end{pgfscope}%
\begin{pgfscope}%
\pgfsys@transformshift{2.312134in}{0.943734in}%
\pgfsys@useobject{currentmarker}{}%
\end{pgfscope}%
\begin{pgfscope}%
\pgfsys@transformshift{2.312134in}{0.943734in}%
\pgfsys@useobject{currentmarker}{}%
\end{pgfscope}%
\begin{pgfscope}%
\pgfsys@transformshift{2.312134in}{0.943734in}%
\pgfsys@useobject{currentmarker}{}%
\end{pgfscope}%
\begin{pgfscope}%
\pgfsys@transformshift{2.312134in}{0.943734in}%
\pgfsys@useobject{currentmarker}{}%
\end{pgfscope}%
\begin{pgfscope}%
\pgfsys@transformshift{2.312134in}{0.943734in}%
\pgfsys@useobject{currentmarker}{}%
\end{pgfscope}%
\begin{pgfscope}%
\pgfsys@transformshift{2.312134in}{0.943734in}%
\pgfsys@useobject{currentmarker}{}%
\end{pgfscope}%
\begin{pgfscope}%
\pgfsys@transformshift{2.312134in}{0.943734in}%
\pgfsys@useobject{currentmarker}{}%
\end{pgfscope}%
\begin{pgfscope}%
\pgfsys@transformshift{2.312134in}{0.943734in}%
\pgfsys@useobject{currentmarker}{}%
\end{pgfscope}%
\begin{pgfscope}%
\pgfsys@transformshift{2.312134in}{0.943734in}%
\pgfsys@useobject{currentmarker}{}%
\end{pgfscope}%
\begin{pgfscope}%
\pgfsys@transformshift{2.312134in}{0.943734in}%
\pgfsys@useobject{currentmarker}{}%
\end{pgfscope}%
\begin{pgfscope}%
\pgfsys@transformshift{2.312134in}{0.943734in}%
\pgfsys@useobject{currentmarker}{}%
\end{pgfscope}%
\begin{pgfscope}%
\pgfsys@transformshift{2.312134in}{0.943734in}%
\pgfsys@useobject{currentmarker}{}%
\end{pgfscope}%
\begin{pgfscope}%
\pgfsys@transformshift{2.312134in}{0.943734in}%
\pgfsys@useobject{currentmarker}{}%
\end{pgfscope}%
\begin{pgfscope}%
\pgfsys@transformshift{2.312134in}{0.943734in}%
\pgfsys@useobject{currentmarker}{}%
\end{pgfscope}%
\begin{pgfscope}%
\pgfsys@transformshift{2.312134in}{0.943734in}%
\pgfsys@useobject{currentmarker}{}%
\end{pgfscope}%
\begin{pgfscope}%
\pgfsys@transformshift{2.312134in}{0.943734in}%
\pgfsys@useobject{currentmarker}{}%
\end{pgfscope}%
\begin{pgfscope}%
\pgfsys@transformshift{2.312134in}{0.943734in}%
\pgfsys@useobject{currentmarker}{}%
\end{pgfscope}%
\begin{pgfscope}%
\pgfsys@transformshift{2.312134in}{0.943734in}%
\pgfsys@useobject{currentmarker}{}%
\end{pgfscope}%
\begin{pgfscope}%
\pgfsys@transformshift{2.312134in}{0.943734in}%
\pgfsys@useobject{currentmarker}{}%
\end{pgfscope}%
\begin{pgfscope}%
\pgfsys@transformshift{2.312134in}{0.943734in}%
\pgfsys@useobject{currentmarker}{}%
\end{pgfscope}%
\begin{pgfscope}%
\pgfsys@transformshift{2.312134in}{0.943734in}%
\pgfsys@useobject{currentmarker}{}%
\end{pgfscope}%
\begin{pgfscope}%
\pgfsys@transformshift{2.312134in}{0.943734in}%
\pgfsys@useobject{currentmarker}{}%
\end{pgfscope}%
\begin{pgfscope}%
\pgfsys@transformshift{2.312134in}{0.943734in}%
\pgfsys@useobject{currentmarker}{}%
\end{pgfscope}%
\begin{pgfscope}%
\pgfsys@transformshift{2.312134in}{0.943734in}%
\pgfsys@useobject{currentmarker}{}%
\end{pgfscope}%
\begin{pgfscope}%
\pgfsys@transformshift{2.312134in}{0.943734in}%
\pgfsys@useobject{currentmarker}{}%
\end{pgfscope}%
\begin{pgfscope}%
\pgfsys@transformshift{2.312134in}{0.943734in}%
\pgfsys@useobject{currentmarker}{}%
\end{pgfscope}%
\begin{pgfscope}%
\pgfsys@transformshift{2.312134in}{0.943734in}%
\pgfsys@useobject{currentmarker}{}%
\end{pgfscope}%
\begin{pgfscope}%
\pgfsys@transformshift{2.312134in}{0.943734in}%
\pgfsys@useobject{currentmarker}{}%
\end{pgfscope}%
\begin{pgfscope}%
\pgfsys@transformshift{2.312134in}{0.943734in}%
\pgfsys@useobject{currentmarker}{}%
\end{pgfscope}%
\begin{pgfscope}%
\pgfsys@transformshift{2.312134in}{0.943734in}%
\pgfsys@useobject{currentmarker}{}%
\end{pgfscope}%
\begin{pgfscope}%
\pgfsys@transformshift{2.312134in}{0.943734in}%
\pgfsys@useobject{currentmarker}{}%
\end{pgfscope}%
\begin{pgfscope}%
\pgfsys@transformshift{2.312134in}{0.943734in}%
\pgfsys@useobject{currentmarker}{}%
\end{pgfscope}%
\begin{pgfscope}%
\pgfsys@transformshift{2.312134in}{0.943734in}%
\pgfsys@useobject{currentmarker}{}%
\end{pgfscope}%
\begin{pgfscope}%
\pgfsys@transformshift{2.312134in}{0.943734in}%
\pgfsys@useobject{currentmarker}{}%
\end{pgfscope}%
\begin{pgfscope}%
\pgfsys@transformshift{2.312134in}{0.943734in}%
\pgfsys@useobject{currentmarker}{}%
\end{pgfscope}%
\begin{pgfscope}%
\pgfsys@transformshift{2.312134in}{0.943734in}%
\pgfsys@useobject{currentmarker}{}%
\end{pgfscope}%
\begin{pgfscope}%
\pgfsys@transformshift{2.312134in}{0.943734in}%
\pgfsys@useobject{currentmarker}{}%
\end{pgfscope}%
\begin{pgfscope}%
\pgfsys@transformshift{2.312134in}{0.943734in}%
\pgfsys@useobject{currentmarker}{}%
\end{pgfscope}%
\begin{pgfscope}%
\pgfsys@transformshift{2.312134in}{0.943734in}%
\pgfsys@useobject{currentmarker}{}%
\end{pgfscope}%
\begin{pgfscope}%
\pgfsys@transformshift{2.312134in}{0.943734in}%
\pgfsys@useobject{currentmarker}{}%
\end{pgfscope}%
\begin{pgfscope}%
\pgfsys@transformshift{2.312134in}{0.943734in}%
\pgfsys@useobject{currentmarker}{}%
\end{pgfscope}%
\begin{pgfscope}%
\pgfsys@transformshift{2.312134in}{0.943734in}%
\pgfsys@useobject{currentmarker}{}%
\end{pgfscope}%
\begin{pgfscope}%
\pgfsys@transformshift{2.312134in}{0.943734in}%
\pgfsys@useobject{currentmarker}{}%
\end{pgfscope}%
\begin{pgfscope}%
\pgfsys@transformshift{2.312134in}{0.943734in}%
\pgfsys@useobject{currentmarker}{}%
\end{pgfscope}%
\begin{pgfscope}%
\pgfsys@transformshift{2.312134in}{0.943734in}%
\pgfsys@useobject{currentmarker}{}%
\end{pgfscope}%
\begin{pgfscope}%
\pgfsys@transformshift{2.312134in}{0.943734in}%
\pgfsys@useobject{currentmarker}{}%
\end{pgfscope}%
\begin{pgfscope}%
\pgfsys@transformshift{2.312134in}{0.943734in}%
\pgfsys@useobject{currentmarker}{}%
\end{pgfscope}%
\begin{pgfscope}%
\pgfsys@transformshift{2.312134in}{0.943734in}%
\pgfsys@useobject{currentmarker}{}%
\end{pgfscope}%
\begin{pgfscope}%
\pgfsys@transformshift{2.312134in}{0.943734in}%
\pgfsys@useobject{currentmarker}{}%
\end{pgfscope}%
\begin{pgfscope}%
\pgfsys@transformshift{2.312134in}{0.943734in}%
\pgfsys@useobject{currentmarker}{}%
\end{pgfscope}%
\begin{pgfscope}%
\pgfsys@transformshift{2.312134in}{0.943734in}%
\pgfsys@useobject{currentmarker}{}%
\end{pgfscope}%
\begin{pgfscope}%
\pgfsys@transformshift{2.312134in}{0.943734in}%
\pgfsys@useobject{currentmarker}{}%
\end{pgfscope}%
\begin{pgfscope}%
\pgfsys@transformshift{2.312134in}{0.943734in}%
\pgfsys@useobject{currentmarker}{}%
\end{pgfscope}%
\begin{pgfscope}%
\pgfsys@transformshift{2.312134in}{0.943734in}%
\pgfsys@useobject{currentmarker}{}%
\end{pgfscope}%
\begin{pgfscope}%
\pgfsys@transformshift{2.312134in}{0.943734in}%
\pgfsys@useobject{currentmarker}{}%
\end{pgfscope}%
\begin{pgfscope}%
\pgfsys@transformshift{2.312134in}{0.943734in}%
\pgfsys@useobject{currentmarker}{}%
\end{pgfscope}%
\begin{pgfscope}%
\pgfsys@transformshift{2.312134in}{0.943734in}%
\pgfsys@useobject{currentmarker}{}%
\end{pgfscope}%
\begin{pgfscope}%
\pgfsys@transformshift{2.312134in}{0.943734in}%
\pgfsys@useobject{currentmarker}{}%
\end{pgfscope}%
\begin{pgfscope}%
\pgfsys@transformshift{2.312134in}{0.943734in}%
\pgfsys@useobject{currentmarker}{}%
\end{pgfscope}%
\begin{pgfscope}%
\pgfsys@transformshift{2.312134in}{0.943734in}%
\pgfsys@useobject{currentmarker}{}%
\end{pgfscope}%
\begin{pgfscope}%
\pgfsys@transformshift{2.312134in}{0.943734in}%
\pgfsys@useobject{currentmarker}{}%
\end{pgfscope}%
\begin{pgfscope}%
\pgfsys@transformshift{2.312134in}{0.943734in}%
\pgfsys@useobject{currentmarker}{}%
\end{pgfscope}%
\begin{pgfscope}%
\pgfsys@transformshift{2.312134in}{0.943734in}%
\pgfsys@useobject{currentmarker}{}%
\end{pgfscope}%
\begin{pgfscope}%
\pgfsys@transformshift{2.312134in}{0.943734in}%
\pgfsys@useobject{currentmarker}{}%
\end{pgfscope}%
\begin{pgfscope}%
\pgfsys@transformshift{2.312134in}{0.943734in}%
\pgfsys@useobject{currentmarker}{}%
\end{pgfscope}%
\begin{pgfscope}%
\pgfsys@transformshift{2.312134in}{0.943734in}%
\pgfsys@useobject{currentmarker}{}%
\end{pgfscope}%
\begin{pgfscope}%
\pgfsys@transformshift{2.312134in}{0.943734in}%
\pgfsys@useobject{currentmarker}{}%
\end{pgfscope}%
\begin{pgfscope}%
\pgfsys@transformshift{2.312134in}{0.943734in}%
\pgfsys@useobject{currentmarker}{}%
\end{pgfscope}%
\begin{pgfscope}%
\pgfsys@transformshift{2.312134in}{0.943734in}%
\pgfsys@useobject{currentmarker}{}%
\end{pgfscope}%
\begin{pgfscope}%
\pgfsys@transformshift{2.312134in}{0.943734in}%
\pgfsys@useobject{currentmarker}{}%
\end{pgfscope}%
\begin{pgfscope}%
\pgfsys@transformshift{2.312134in}{0.943734in}%
\pgfsys@useobject{currentmarker}{}%
\end{pgfscope}%
\begin{pgfscope}%
\pgfsys@transformshift{2.312134in}{0.943734in}%
\pgfsys@useobject{currentmarker}{}%
\end{pgfscope}%
\begin{pgfscope}%
\pgfsys@transformshift{2.312134in}{0.943734in}%
\pgfsys@useobject{currentmarker}{}%
\end{pgfscope}%
\begin{pgfscope}%
\pgfsys@transformshift{2.312134in}{0.943734in}%
\pgfsys@useobject{currentmarker}{}%
\end{pgfscope}%
\begin{pgfscope}%
\pgfsys@transformshift{2.312134in}{0.943734in}%
\pgfsys@useobject{currentmarker}{}%
\end{pgfscope}%
\begin{pgfscope}%
\pgfsys@transformshift{2.312134in}{0.943734in}%
\pgfsys@useobject{currentmarker}{}%
\end{pgfscope}%
\begin{pgfscope}%
\pgfsys@transformshift{2.312134in}{0.943734in}%
\pgfsys@useobject{currentmarker}{}%
\end{pgfscope}%
\begin{pgfscope}%
\pgfsys@transformshift{2.312134in}{0.943734in}%
\pgfsys@useobject{currentmarker}{}%
\end{pgfscope}%
\begin{pgfscope}%
\pgfsys@transformshift{2.312134in}{0.943734in}%
\pgfsys@useobject{currentmarker}{}%
\end{pgfscope}%
\begin{pgfscope}%
\pgfsys@transformshift{2.312134in}{0.943734in}%
\pgfsys@useobject{currentmarker}{}%
\end{pgfscope}%
\begin{pgfscope}%
\pgfsys@transformshift{2.312134in}{0.943734in}%
\pgfsys@useobject{currentmarker}{}%
\end{pgfscope}%
\begin{pgfscope}%
\pgfsys@transformshift{2.312134in}{0.943734in}%
\pgfsys@useobject{currentmarker}{}%
\end{pgfscope}%
\begin{pgfscope}%
\pgfsys@transformshift{2.312134in}{0.943734in}%
\pgfsys@useobject{currentmarker}{}%
\end{pgfscope}%
\begin{pgfscope}%
\pgfsys@transformshift{2.312134in}{0.943734in}%
\pgfsys@useobject{currentmarker}{}%
\end{pgfscope}%
\begin{pgfscope}%
\pgfsys@transformshift{2.312134in}{0.943734in}%
\pgfsys@useobject{currentmarker}{}%
\end{pgfscope}%
\begin{pgfscope}%
\pgfsys@transformshift{2.312134in}{0.943734in}%
\pgfsys@useobject{currentmarker}{}%
\end{pgfscope}%
\begin{pgfscope}%
\pgfsys@transformshift{2.312134in}{0.943734in}%
\pgfsys@useobject{currentmarker}{}%
\end{pgfscope}%
\begin{pgfscope}%
\pgfsys@transformshift{2.312134in}{0.943734in}%
\pgfsys@useobject{currentmarker}{}%
\end{pgfscope}%
\begin{pgfscope}%
\pgfsys@transformshift{2.312134in}{0.943734in}%
\pgfsys@useobject{currentmarker}{}%
\end{pgfscope}%
\begin{pgfscope}%
\pgfsys@transformshift{2.312134in}{0.943734in}%
\pgfsys@useobject{currentmarker}{}%
\end{pgfscope}%
\begin{pgfscope}%
\pgfsys@transformshift{2.312134in}{0.943734in}%
\pgfsys@useobject{currentmarker}{}%
\end{pgfscope}%
\begin{pgfscope}%
\pgfsys@transformshift{2.312134in}{0.943734in}%
\pgfsys@useobject{currentmarker}{}%
\end{pgfscope}%
\begin{pgfscope}%
\pgfsys@transformshift{2.312134in}{0.943734in}%
\pgfsys@useobject{currentmarker}{}%
\end{pgfscope}%
\begin{pgfscope}%
\pgfsys@transformshift{2.312134in}{0.943734in}%
\pgfsys@useobject{currentmarker}{}%
\end{pgfscope}%
\begin{pgfscope}%
\pgfsys@transformshift{2.312134in}{0.943734in}%
\pgfsys@useobject{currentmarker}{}%
\end{pgfscope}%
\begin{pgfscope}%
\pgfsys@transformshift{2.312134in}{0.943734in}%
\pgfsys@useobject{currentmarker}{}%
\end{pgfscope}%
\begin{pgfscope}%
\pgfsys@transformshift{2.312134in}{0.943734in}%
\pgfsys@useobject{currentmarker}{}%
\end{pgfscope}%
\begin{pgfscope}%
\pgfsys@transformshift{2.312134in}{0.943734in}%
\pgfsys@useobject{currentmarker}{}%
\end{pgfscope}%
\begin{pgfscope}%
\pgfsys@transformshift{2.312134in}{0.943734in}%
\pgfsys@useobject{currentmarker}{}%
\end{pgfscope}%
\begin{pgfscope}%
\pgfsys@transformshift{2.312134in}{0.943734in}%
\pgfsys@useobject{currentmarker}{}%
\end{pgfscope}%
\begin{pgfscope}%
\pgfsys@transformshift{2.312134in}{0.943734in}%
\pgfsys@useobject{currentmarker}{}%
\end{pgfscope}%
\begin{pgfscope}%
\pgfsys@transformshift{2.312134in}{0.943734in}%
\pgfsys@useobject{currentmarker}{}%
\end{pgfscope}%
\begin{pgfscope}%
\pgfsys@transformshift{2.312134in}{0.943734in}%
\pgfsys@useobject{currentmarker}{}%
\end{pgfscope}%
\begin{pgfscope}%
\pgfsys@transformshift{2.312134in}{0.943734in}%
\pgfsys@useobject{currentmarker}{}%
\end{pgfscope}%
\begin{pgfscope}%
\pgfsys@transformshift{2.312134in}{0.943734in}%
\pgfsys@useobject{currentmarker}{}%
\end{pgfscope}%
\begin{pgfscope}%
\pgfsys@transformshift{2.312134in}{0.943734in}%
\pgfsys@useobject{currentmarker}{}%
\end{pgfscope}%
\begin{pgfscope}%
\pgfsys@transformshift{2.312134in}{0.943734in}%
\pgfsys@useobject{currentmarker}{}%
\end{pgfscope}%
\begin{pgfscope}%
\pgfsys@transformshift{2.312134in}{0.943734in}%
\pgfsys@useobject{currentmarker}{}%
\end{pgfscope}%
\begin{pgfscope}%
\pgfsys@transformshift{2.312134in}{0.943734in}%
\pgfsys@useobject{currentmarker}{}%
\end{pgfscope}%
\begin{pgfscope}%
\pgfsys@transformshift{2.312134in}{0.943734in}%
\pgfsys@useobject{currentmarker}{}%
\end{pgfscope}%
\begin{pgfscope}%
\pgfsys@transformshift{2.312134in}{0.943734in}%
\pgfsys@useobject{currentmarker}{}%
\end{pgfscope}%
\begin{pgfscope}%
\pgfsys@transformshift{2.312134in}{0.943734in}%
\pgfsys@useobject{currentmarker}{}%
\end{pgfscope}%
\begin{pgfscope}%
\pgfsys@transformshift{2.312134in}{0.943734in}%
\pgfsys@useobject{currentmarker}{}%
\end{pgfscope}%
\begin{pgfscope}%
\pgfsys@transformshift{2.312134in}{0.943734in}%
\pgfsys@useobject{currentmarker}{}%
\end{pgfscope}%
\begin{pgfscope}%
\pgfsys@transformshift{2.312134in}{0.943734in}%
\pgfsys@useobject{currentmarker}{}%
\end{pgfscope}%
\begin{pgfscope}%
\pgfsys@transformshift{2.312134in}{0.943734in}%
\pgfsys@useobject{currentmarker}{}%
\end{pgfscope}%
\begin{pgfscope}%
\pgfsys@transformshift{2.312134in}{0.943734in}%
\pgfsys@useobject{currentmarker}{}%
\end{pgfscope}%
\begin{pgfscope}%
\pgfsys@transformshift{2.312134in}{0.943734in}%
\pgfsys@useobject{currentmarker}{}%
\end{pgfscope}%
\begin{pgfscope}%
\pgfsys@transformshift{2.312134in}{0.943734in}%
\pgfsys@useobject{currentmarker}{}%
\end{pgfscope}%
\begin{pgfscope}%
\pgfsys@transformshift{2.312134in}{0.943734in}%
\pgfsys@useobject{currentmarker}{}%
\end{pgfscope}%
\begin{pgfscope}%
\pgfsys@transformshift{2.312134in}{0.943734in}%
\pgfsys@useobject{currentmarker}{}%
\end{pgfscope}%
\begin{pgfscope}%
\pgfsys@transformshift{2.312134in}{0.943734in}%
\pgfsys@useobject{currentmarker}{}%
\end{pgfscope}%
\begin{pgfscope}%
\pgfsys@transformshift{2.312134in}{0.943734in}%
\pgfsys@useobject{currentmarker}{}%
\end{pgfscope}%
\begin{pgfscope}%
\pgfsys@transformshift{2.312134in}{0.943734in}%
\pgfsys@useobject{currentmarker}{}%
\end{pgfscope}%
\begin{pgfscope}%
\pgfsys@transformshift{2.312134in}{0.943734in}%
\pgfsys@useobject{currentmarker}{}%
\end{pgfscope}%
\begin{pgfscope}%
\pgfsys@transformshift{2.312134in}{0.943734in}%
\pgfsys@useobject{currentmarker}{}%
\end{pgfscope}%
\begin{pgfscope}%
\pgfsys@transformshift{2.312134in}{0.943734in}%
\pgfsys@useobject{currentmarker}{}%
\end{pgfscope}%
\begin{pgfscope}%
\pgfsys@transformshift{2.312134in}{0.943734in}%
\pgfsys@useobject{currentmarker}{}%
\end{pgfscope}%
\begin{pgfscope}%
\pgfsys@transformshift{2.312134in}{0.943734in}%
\pgfsys@useobject{currentmarker}{}%
\end{pgfscope}%
\begin{pgfscope}%
\pgfsys@transformshift{2.312134in}{0.943734in}%
\pgfsys@useobject{currentmarker}{}%
\end{pgfscope}%
\begin{pgfscope}%
\pgfsys@transformshift{2.312134in}{0.943734in}%
\pgfsys@useobject{currentmarker}{}%
\end{pgfscope}%
\begin{pgfscope}%
\pgfsys@transformshift{2.312134in}{0.943734in}%
\pgfsys@useobject{currentmarker}{}%
\end{pgfscope}%
\begin{pgfscope}%
\pgfsys@transformshift{2.312134in}{0.943734in}%
\pgfsys@useobject{currentmarker}{}%
\end{pgfscope}%
\begin{pgfscope}%
\pgfsys@transformshift{2.312134in}{0.943734in}%
\pgfsys@useobject{currentmarker}{}%
\end{pgfscope}%
\begin{pgfscope}%
\pgfsys@transformshift{2.312134in}{0.943734in}%
\pgfsys@useobject{currentmarker}{}%
\end{pgfscope}%
\begin{pgfscope}%
\pgfsys@transformshift{2.312134in}{0.943734in}%
\pgfsys@useobject{currentmarker}{}%
\end{pgfscope}%
\begin{pgfscope}%
\pgfsys@transformshift{2.312134in}{0.943734in}%
\pgfsys@useobject{currentmarker}{}%
\end{pgfscope}%
\begin{pgfscope}%
\pgfsys@transformshift{2.312134in}{0.943734in}%
\pgfsys@useobject{currentmarker}{}%
\end{pgfscope}%
\begin{pgfscope}%
\pgfsys@transformshift{2.312134in}{0.943734in}%
\pgfsys@useobject{currentmarker}{}%
\end{pgfscope}%
\begin{pgfscope}%
\pgfsys@transformshift{2.312134in}{0.943734in}%
\pgfsys@useobject{currentmarker}{}%
\end{pgfscope}%
\begin{pgfscope}%
\pgfsys@transformshift{2.312134in}{0.943734in}%
\pgfsys@useobject{currentmarker}{}%
\end{pgfscope}%
\begin{pgfscope}%
\pgfsys@transformshift{2.312134in}{0.943734in}%
\pgfsys@useobject{currentmarker}{}%
\end{pgfscope}%
\begin{pgfscope}%
\pgfsys@transformshift{2.312134in}{0.943734in}%
\pgfsys@useobject{currentmarker}{}%
\end{pgfscope}%
\begin{pgfscope}%
\pgfsys@transformshift{2.312134in}{0.943734in}%
\pgfsys@useobject{currentmarker}{}%
\end{pgfscope}%
\begin{pgfscope}%
\pgfsys@transformshift{2.312134in}{0.943734in}%
\pgfsys@useobject{currentmarker}{}%
\end{pgfscope}%
\begin{pgfscope}%
\pgfsys@transformshift{2.312134in}{0.943734in}%
\pgfsys@useobject{currentmarker}{}%
\end{pgfscope}%
\begin{pgfscope}%
\pgfsys@transformshift{2.312134in}{0.943734in}%
\pgfsys@useobject{currentmarker}{}%
\end{pgfscope}%
\begin{pgfscope}%
\pgfsys@transformshift{2.312134in}{0.943734in}%
\pgfsys@useobject{currentmarker}{}%
\end{pgfscope}%
\begin{pgfscope}%
\pgfsys@transformshift{2.312134in}{0.943734in}%
\pgfsys@useobject{currentmarker}{}%
\end{pgfscope}%
\begin{pgfscope}%
\pgfsys@transformshift{2.312134in}{0.943734in}%
\pgfsys@useobject{currentmarker}{}%
\end{pgfscope}%
\begin{pgfscope}%
\pgfsys@transformshift{2.312134in}{0.943734in}%
\pgfsys@useobject{currentmarker}{}%
\end{pgfscope}%
\begin{pgfscope}%
\pgfsys@transformshift{2.312134in}{0.943734in}%
\pgfsys@useobject{currentmarker}{}%
\end{pgfscope}%
\begin{pgfscope}%
\pgfsys@transformshift{2.312134in}{0.943734in}%
\pgfsys@useobject{currentmarker}{}%
\end{pgfscope}%
\begin{pgfscope}%
\pgfsys@transformshift{2.312134in}{0.943734in}%
\pgfsys@useobject{currentmarker}{}%
\end{pgfscope}%
\begin{pgfscope}%
\pgfsys@transformshift{2.312134in}{0.943734in}%
\pgfsys@useobject{currentmarker}{}%
\end{pgfscope}%
\begin{pgfscope}%
\pgfsys@transformshift{2.312134in}{0.943734in}%
\pgfsys@useobject{currentmarker}{}%
\end{pgfscope}%
\begin{pgfscope}%
\pgfsys@transformshift{2.312134in}{0.943734in}%
\pgfsys@useobject{currentmarker}{}%
\end{pgfscope}%
\begin{pgfscope}%
\pgfsys@transformshift{2.312134in}{0.943734in}%
\pgfsys@useobject{currentmarker}{}%
\end{pgfscope}%
\begin{pgfscope}%
\pgfsys@transformshift{2.312134in}{0.943734in}%
\pgfsys@useobject{currentmarker}{}%
\end{pgfscope}%
\begin{pgfscope}%
\pgfsys@transformshift{2.312134in}{0.943734in}%
\pgfsys@useobject{currentmarker}{}%
\end{pgfscope}%
\begin{pgfscope}%
\pgfsys@transformshift{2.312134in}{0.943734in}%
\pgfsys@useobject{currentmarker}{}%
\end{pgfscope}%
\begin{pgfscope}%
\pgfsys@transformshift{2.312134in}{0.943734in}%
\pgfsys@useobject{currentmarker}{}%
\end{pgfscope}%
\begin{pgfscope}%
\pgfsys@transformshift{2.312134in}{0.943734in}%
\pgfsys@useobject{currentmarker}{}%
\end{pgfscope}%
\begin{pgfscope}%
\pgfsys@transformshift{2.312134in}{0.943734in}%
\pgfsys@useobject{currentmarker}{}%
\end{pgfscope}%
\begin{pgfscope}%
\pgfsys@transformshift{2.312134in}{0.943734in}%
\pgfsys@useobject{currentmarker}{}%
\end{pgfscope}%
\begin{pgfscope}%
\pgfsys@transformshift{2.312134in}{0.943734in}%
\pgfsys@useobject{currentmarker}{}%
\end{pgfscope}%
\begin{pgfscope}%
\pgfsys@transformshift{2.312134in}{0.943734in}%
\pgfsys@useobject{currentmarker}{}%
\end{pgfscope}%
\begin{pgfscope}%
\pgfsys@transformshift{2.312134in}{0.943734in}%
\pgfsys@useobject{currentmarker}{}%
\end{pgfscope}%
\begin{pgfscope}%
\pgfsys@transformshift{2.312134in}{0.943734in}%
\pgfsys@useobject{currentmarker}{}%
\end{pgfscope}%
\begin{pgfscope}%
\pgfsys@transformshift{2.312134in}{0.943734in}%
\pgfsys@useobject{currentmarker}{}%
\end{pgfscope}%
\begin{pgfscope}%
\pgfsys@transformshift{2.312134in}{0.943734in}%
\pgfsys@useobject{currentmarker}{}%
\end{pgfscope}%
\begin{pgfscope}%
\pgfsys@transformshift{2.312134in}{0.943734in}%
\pgfsys@useobject{currentmarker}{}%
\end{pgfscope}%
\begin{pgfscope}%
\pgfsys@transformshift{2.312134in}{0.943734in}%
\pgfsys@useobject{currentmarker}{}%
\end{pgfscope}%
\begin{pgfscope}%
\pgfsys@transformshift{2.312134in}{0.943734in}%
\pgfsys@useobject{currentmarker}{}%
\end{pgfscope}%
\begin{pgfscope}%
\pgfsys@transformshift{2.312134in}{0.943734in}%
\pgfsys@useobject{currentmarker}{}%
\end{pgfscope}%
\begin{pgfscope}%
\pgfsys@transformshift{2.312134in}{0.943734in}%
\pgfsys@useobject{currentmarker}{}%
\end{pgfscope}%
\begin{pgfscope}%
\pgfsys@transformshift{2.312134in}{0.943734in}%
\pgfsys@useobject{currentmarker}{}%
\end{pgfscope}%
\begin{pgfscope}%
\pgfsys@transformshift{2.312134in}{0.943734in}%
\pgfsys@useobject{currentmarker}{}%
\end{pgfscope}%
\begin{pgfscope}%
\pgfsys@transformshift{2.312134in}{0.943734in}%
\pgfsys@useobject{currentmarker}{}%
\end{pgfscope}%
\begin{pgfscope}%
\pgfsys@transformshift{2.312134in}{0.943734in}%
\pgfsys@useobject{currentmarker}{}%
\end{pgfscope}%
\begin{pgfscope}%
\pgfsys@transformshift{2.312134in}{0.943734in}%
\pgfsys@useobject{currentmarker}{}%
\end{pgfscope}%
\begin{pgfscope}%
\pgfsys@transformshift{2.312134in}{0.943734in}%
\pgfsys@useobject{currentmarker}{}%
\end{pgfscope}%
\begin{pgfscope}%
\pgfsys@transformshift{2.312134in}{0.943734in}%
\pgfsys@useobject{currentmarker}{}%
\end{pgfscope}%
\begin{pgfscope}%
\pgfsys@transformshift{2.312134in}{0.943734in}%
\pgfsys@useobject{currentmarker}{}%
\end{pgfscope}%
\begin{pgfscope}%
\pgfsys@transformshift{2.312134in}{0.943734in}%
\pgfsys@useobject{currentmarker}{}%
\end{pgfscope}%
\begin{pgfscope}%
\pgfsys@transformshift{2.312134in}{0.943734in}%
\pgfsys@useobject{currentmarker}{}%
\end{pgfscope}%
\begin{pgfscope}%
\pgfsys@transformshift{2.312134in}{0.943734in}%
\pgfsys@useobject{currentmarker}{}%
\end{pgfscope}%
\begin{pgfscope}%
\pgfsys@transformshift{2.312134in}{0.943734in}%
\pgfsys@useobject{currentmarker}{}%
\end{pgfscope}%
\begin{pgfscope}%
\pgfsys@transformshift{2.312134in}{0.943734in}%
\pgfsys@useobject{currentmarker}{}%
\end{pgfscope}%
\begin{pgfscope}%
\pgfsys@transformshift{2.312134in}{0.943734in}%
\pgfsys@useobject{currentmarker}{}%
\end{pgfscope}%
\begin{pgfscope}%
\pgfsys@transformshift{2.312134in}{0.943734in}%
\pgfsys@useobject{currentmarker}{}%
\end{pgfscope}%
\begin{pgfscope}%
\pgfsys@transformshift{2.312134in}{0.943734in}%
\pgfsys@useobject{currentmarker}{}%
\end{pgfscope}%
\begin{pgfscope}%
\pgfsys@transformshift{2.312134in}{0.943734in}%
\pgfsys@useobject{currentmarker}{}%
\end{pgfscope}%
\begin{pgfscope}%
\pgfsys@transformshift{2.312134in}{0.943734in}%
\pgfsys@useobject{currentmarker}{}%
\end{pgfscope}%
\begin{pgfscope}%
\pgfsys@transformshift{2.312134in}{0.943734in}%
\pgfsys@useobject{currentmarker}{}%
\end{pgfscope}%
\begin{pgfscope}%
\pgfsys@transformshift{2.312134in}{0.943734in}%
\pgfsys@useobject{currentmarker}{}%
\end{pgfscope}%
\begin{pgfscope}%
\pgfsys@transformshift{2.312134in}{0.943734in}%
\pgfsys@useobject{currentmarker}{}%
\end{pgfscope}%
\begin{pgfscope}%
\pgfsys@transformshift{2.312134in}{0.943734in}%
\pgfsys@useobject{currentmarker}{}%
\end{pgfscope}%
\begin{pgfscope}%
\pgfsys@transformshift{2.312134in}{0.943734in}%
\pgfsys@useobject{currentmarker}{}%
\end{pgfscope}%
\begin{pgfscope}%
\pgfsys@transformshift{2.312134in}{0.943734in}%
\pgfsys@useobject{currentmarker}{}%
\end{pgfscope}%
\begin{pgfscope}%
\pgfsys@transformshift{2.312134in}{0.943734in}%
\pgfsys@useobject{currentmarker}{}%
\end{pgfscope}%
\begin{pgfscope}%
\pgfsys@transformshift{2.312134in}{0.943734in}%
\pgfsys@useobject{currentmarker}{}%
\end{pgfscope}%
\begin{pgfscope}%
\pgfsys@transformshift{2.312134in}{0.943734in}%
\pgfsys@useobject{currentmarker}{}%
\end{pgfscope}%
\begin{pgfscope}%
\pgfsys@transformshift{2.312134in}{0.943734in}%
\pgfsys@useobject{currentmarker}{}%
\end{pgfscope}%
\begin{pgfscope}%
\pgfsys@transformshift{2.312134in}{0.943734in}%
\pgfsys@useobject{currentmarker}{}%
\end{pgfscope}%
\begin{pgfscope}%
\pgfsys@transformshift{2.312134in}{0.943734in}%
\pgfsys@useobject{currentmarker}{}%
\end{pgfscope}%
\begin{pgfscope}%
\pgfsys@transformshift{2.312134in}{0.943734in}%
\pgfsys@useobject{currentmarker}{}%
\end{pgfscope}%
\begin{pgfscope}%
\pgfsys@transformshift{2.312134in}{0.943734in}%
\pgfsys@useobject{currentmarker}{}%
\end{pgfscope}%
\begin{pgfscope}%
\pgfsys@transformshift{2.312134in}{0.943734in}%
\pgfsys@useobject{currentmarker}{}%
\end{pgfscope}%
\begin{pgfscope}%
\pgfsys@transformshift{2.312134in}{0.943734in}%
\pgfsys@useobject{currentmarker}{}%
\end{pgfscope}%
\begin{pgfscope}%
\pgfsys@transformshift{2.312134in}{0.943734in}%
\pgfsys@useobject{currentmarker}{}%
\end{pgfscope}%
\begin{pgfscope}%
\pgfsys@transformshift{2.312134in}{0.943734in}%
\pgfsys@useobject{currentmarker}{}%
\end{pgfscope}%
\begin{pgfscope}%
\pgfsys@transformshift{2.312134in}{0.943734in}%
\pgfsys@useobject{currentmarker}{}%
\end{pgfscope}%
\begin{pgfscope}%
\pgfsys@transformshift{2.312134in}{0.943734in}%
\pgfsys@useobject{currentmarker}{}%
\end{pgfscope}%
\begin{pgfscope}%
\pgfsys@transformshift{2.312134in}{0.943734in}%
\pgfsys@useobject{currentmarker}{}%
\end{pgfscope}%
\begin{pgfscope}%
\pgfsys@transformshift{2.312134in}{0.943734in}%
\pgfsys@useobject{currentmarker}{}%
\end{pgfscope}%
\begin{pgfscope}%
\pgfsys@transformshift{2.312134in}{0.943734in}%
\pgfsys@useobject{currentmarker}{}%
\end{pgfscope}%
\begin{pgfscope}%
\pgfsys@transformshift{2.312134in}{0.943734in}%
\pgfsys@useobject{currentmarker}{}%
\end{pgfscope}%
\begin{pgfscope}%
\pgfsys@transformshift{2.312134in}{0.943734in}%
\pgfsys@useobject{currentmarker}{}%
\end{pgfscope}%
\begin{pgfscope}%
\pgfsys@transformshift{2.312134in}{0.943734in}%
\pgfsys@useobject{currentmarker}{}%
\end{pgfscope}%
\begin{pgfscope}%
\pgfsys@transformshift{2.312134in}{0.943734in}%
\pgfsys@useobject{currentmarker}{}%
\end{pgfscope}%
\begin{pgfscope}%
\pgfsys@transformshift{2.312134in}{0.943734in}%
\pgfsys@useobject{currentmarker}{}%
\end{pgfscope}%
\begin{pgfscope}%
\pgfsys@transformshift{2.312134in}{0.943734in}%
\pgfsys@useobject{currentmarker}{}%
\end{pgfscope}%
\begin{pgfscope}%
\pgfsys@transformshift{2.312134in}{0.943734in}%
\pgfsys@useobject{currentmarker}{}%
\end{pgfscope}%
\begin{pgfscope}%
\pgfsys@transformshift{2.312134in}{0.943734in}%
\pgfsys@useobject{currentmarker}{}%
\end{pgfscope}%
\begin{pgfscope}%
\pgfsys@transformshift{2.312134in}{0.943734in}%
\pgfsys@useobject{currentmarker}{}%
\end{pgfscope}%
\begin{pgfscope}%
\pgfsys@transformshift{2.312134in}{0.943734in}%
\pgfsys@useobject{currentmarker}{}%
\end{pgfscope}%
\begin{pgfscope}%
\pgfsys@transformshift{2.312134in}{0.943734in}%
\pgfsys@useobject{currentmarker}{}%
\end{pgfscope}%
\begin{pgfscope}%
\pgfsys@transformshift{2.312134in}{0.943734in}%
\pgfsys@useobject{currentmarker}{}%
\end{pgfscope}%
\begin{pgfscope}%
\pgfsys@transformshift{2.312134in}{0.943734in}%
\pgfsys@useobject{currentmarker}{}%
\end{pgfscope}%
\begin{pgfscope}%
\pgfsys@transformshift{2.312134in}{0.943734in}%
\pgfsys@useobject{currentmarker}{}%
\end{pgfscope}%
\begin{pgfscope}%
\pgfsys@transformshift{2.312134in}{0.943734in}%
\pgfsys@useobject{currentmarker}{}%
\end{pgfscope}%
\begin{pgfscope}%
\pgfsys@transformshift{2.312134in}{0.943734in}%
\pgfsys@useobject{currentmarker}{}%
\end{pgfscope}%
\begin{pgfscope}%
\pgfsys@transformshift{2.312134in}{0.943734in}%
\pgfsys@useobject{currentmarker}{}%
\end{pgfscope}%
\begin{pgfscope}%
\pgfsys@transformshift{2.312134in}{0.943734in}%
\pgfsys@useobject{currentmarker}{}%
\end{pgfscope}%
\begin{pgfscope}%
\pgfsys@transformshift{2.312134in}{0.943734in}%
\pgfsys@useobject{currentmarker}{}%
\end{pgfscope}%
\begin{pgfscope}%
\pgfsys@transformshift{2.312134in}{0.943734in}%
\pgfsys@useobject{currentmarker}{}%
\end{pgfscope}%
\begin{pgfscope}%
\pgfsys@transformshift{2.312134in}{0.943734in}%
\pgfsys@useobject{currentmarker}{}%
\end{pgfscope}%
\begin{pgfscope}%
\pgfsys@transformshift{2.312134in}{0.943734in}%
\pgfsys@useobject{currentmarker}{}%
\end{pgfscope}%
\begin{pgfscope}%
\pgfsys@transformshift{2.312134in}{0.943734in}%
\pgfsys@useobject{currentmarker}{}%
\end{pgfscope}%
\begin{pgfscope}%
\pgfsys@transformshift{2.312134in}{0.943734in}%
\pgfsys@useobject{currentmarker}{}%
\end{pgfscope}%
\begin{pgfscope}%
\pgfsys@transformshift{2.312134in}{0.943734in}%
\pgfsys@useobject{currentmarker}{}%
\end{pgfscope}%
\begin{pgfscope}%
\pgfsys@transformshift{2.312134in}{0.943734in}%
\pgfsys@useobject{currentmarker}{}%
\end{pgfscope}%
\begin{pgfscope}%
\pgfsys@transformshift{2.312134in}{0.943734in}%
\pgfsys@useobject{currentmarker}{}%
\end{pgfscope}%
\begin{pgfscope}%
\pgfsys@transformshift{2.312134in}{0.943734in}%
\pgfsys@useobject{currentmarker}{}%
\end{pgfscope}%
\begin{pgfscope}%
\pgfsys@transformshift{2.312134in}{0.943734in}%
\pgfsys@useobject{currentmarker}{}%
\end{pgfscope}%
\begin{pgfscope}%
\pgfsys@transformshift{2.312134in}{0.943734in}%
\pgfsys@useobject{currentmarker}{}%
\end{pgfscope}%
\begin{pgfscope}%
\pgfsys@transformshift{2.312134in}{0.943734in}%
\pgfsys@useobject{currentmarker}{}%
\end{pgfscope}%
\begin{pgfscope}%
\pgfsys@transformshift{2.312134in}{0.943734in}%
\pgfsys@useobject{currentmarker}{}%
\end{pgfscope}%
\begin{pgfscope}%
\pgfsys@transformshift{2.312134in}{0.943734in}%
\pgfsys@useobject{currentmarker}{}%
\end{pgfscope}%
\begin{pgfscope}%
\pgfsys@transformshift{2.312134in}{0.943734in}%
\pgfsys@useobject{currentmarker}{}%
\end{pgfscope}%
\begin{pgfscope}%
\pgfsys@transformshift{2.312134in}{0.943734in}%
\pgfsys@useobject{currentmarker}{}%
\end{pgfscope}%
\begin{pgfscope}%
\pgfsys@transformshift{2.312134in}{0.943734in}%
\pgfsys@useobject{currentmarker}{}%
\end{pgfscope}%
\begin{pgfscope}%
\pgfsys@transformshift{2.312134in}{0.943734in}%
\pgfsys@useobject{currentmarker}{}%
\end{pgfscope}%
\begin{pgfscope}%
\pgfsys@transformshift{2.312134in}{0.943734in}%
\pgfsys@useobject{currentmarker}{}%
\end{pgfscope}%
\begin{pgfscope}%
\pgfsys@transformshift{2.312134in}{0.943734in}%
\pgfsys@useobject{currentmarker}{}%
\end{pgfscope}%
\begin{pgfscope}%
\pgfsys@transformshift{2.312134in}{0.943734in}%
\pgfsys@useobject{currentmarker}{}%
\end{pgfscope}%
\begin{pgfscope}%
\pgfsys@transformshift{2.312134in}{0.943734in}%
\pgfsys@useobject{currentmarker}{}%
\end{pgfscope}%
\begin{pgfscope}%
\pgfsys@transformshift{2.312134in}{0.943734in}%
\pgfsys@useobject{currentmarker}{}%
\end{pgfscope}%
\begin{pgfscope}%
\pgfsys@transformshift{2.312134in}{0.943734in}%
\pgfsys@useobject{currentmarker}{}%
\end{pgfscope}%
\begin{pgfscope}%
\pgfsys@transformshift{2.312134in}{0.943734in}%
\pgfsys@useobject{currentmarker}{}%
\end{pgfscope}%
\begin{pgfscope}%
\pgfsys@transformshift{2.312134in}{0.943734in}%
\pgfsys@useobject{currentmarker}{}%
\end{pgfscope}%
\begin{pgfscope}%
\pgfsys@transformshift{2.312134in}{0.943734in}%
\pgfsys@useobject{currentmarker}{}%
\end{pgfscope}%
\begin{pgfscope}%
\pgfsys@transformshift{2.312134in}{0.943734in}%
\pgfsys@useobject{currentmarker}{}%
\end{pgfscope}%
\begin{pgfscope}%
\pgfsys@transformshift{2.312134in}{0.943734in}%
\pgfsys@useobject{currentmarker}{}%
\end{pgfscope}%
\begin{pgfscope}%
\pgfsys@transformshift{2.312134in}{0.943734in}%
\pgfsys@useobject{currentmarker}{}%
\end{pgfscope}%
\begin{pgfscope}%
\pgfsys@transformshift{2.312134in}{0.943734in}%
\pgfsys@useobject{currentmarker}{}%
\end{pgfscope}%
\begin{pgfscope}%
\pgfsys@transformshift{2.312134in}{0.943734in}%
\pgfsys@useobject{currentmarker}{}%
\end{pgfscope}%
\begin{pgfscope}%
\pgfsys@transformshift{2.312134in}{0.943734in}%
\pgfsys@useobject{currentmarker}{}%
\end{pgfscope}%
\begin{pgfscope}%
\pgfsys@transformshift{2.312134in}{0.943734in}%
\pgfsys@useobject{currentmarker}{}%
\end{pgfscope}%
\begin{pgfscope}%
\pgfsys@transformshift{2.312134in}{0.943734in}%
\pgfsys@useobject{currentmarker}{}%
\end{pgfscope}%
\begin{pgfscope}%
\pgfsys@transformshift{2.312134in}{0.943734in}%
\pgfsys@useobject{currentmarker}{}%
\end{pgfscope}%
\begin{pgfscope}%
\pgfsys@transformshift{2.312134in}{0.943734in}%
\pgfsys@useobject{currentmarker}{}%
\end{pgfscope}%
\begin{pgfscope}%
\pgfsys@transformshift{2.312134in}{0.943734in}%
\pgfsys@useobject{currentmarker}{}%
\end{pgfscope}%
\begin{pgfscope}%
\pgfsys@transformshift{2.312134in}{0.943734in}%
\pgfsys@useobject{currentmarker}{}%
\end{pgfscope}%
\begin{pgfscope}%
\pgfsys@transformshift{2.312134in}{0.943734in}%
\pgfsys@useobject{currentmarker}{}%
\end{pgfscope}%
\begin{pgfscope}%
\pgfsys@transformshift{2.312134in}{0.943734in}%
\pgfsys@useobject{currentmarker}{}%
\end{pgfscope}%
\begin{pgfscope}%
\pgfsys@transformshift{2.312134in}{0.943734in}%
\pgfsys@useobject{currentmarker}{}%
\end{pgfscope}%
\begin{pgfscope}%
\pgfsys@transformshift{2.312134in}{0.943734in}%
\pgfsys@useobject{currentmarker}{}%
\end{pgfscope}%
\begin{pgfscope}%
\pgfsys@transformshift{2.312134in}{0.943734in}%
\pgfsys@useobject{currentmarker}{}%
\end{pgfscope}%
\begin{pgfscope}%
\pgfsys@transformshift{2.312134in}{0.943734in}%
\pgfsys@useobject{currentmarker}{}%
\end{pgfscope}%
\begin{pgfscope}%
\pgfsys@transformshift{2.312134in}{0.943734in}%
\pgfsys@useobject{currentmarker}{}%
\end{pgfscope}%
\begin{pgfscope}%
\pgfsys@transformshift{2.312134in}{0.943734in}%
\pgfsys@useobject{currentmarker}{}%
\end{pgfscope}%
\begin{pgfscope}%
\pgfsys@transformshift{2.312134in}{0.943734in}%
\pgfsys@useobject{currentmarker}{}%
\end{pgfscope}%
\begin{pgfscope}%
\pgfsys@transformshift{2.312134in}{0.943734in}%
\pgfsys@useobject{currentmarker}{}%
\end{pgfscope}%
\begin{pgfscope}%
\pgfsys@transformshift{2.312134in}{0.943734in}%
\pgfsys@useobject{currentmarker}{}%
\end{pgfscope}%
\begin{pgfscope}%
\pgfsys@transformshift{2.312134in}{0.943734in}%
\pgfsys@useobject{currentmarker}{}%
\end{pgfscope}%
\begin{pgfscope}%
\pgfsys@transformshift{2.312134in}{0.943734in}%
\pgfsys@useobject{currentmarker}{}%
\end{pgfscope}%
\begin{pgfscope}%
\pgfsys@transformshift{2.312134in}{0.943734in}%
\pgfsys@useobject{currentmarker}{}%
\end{pgfscope}%
\begin{pgfscope}%
\pgfsys@transformshift{2.312134in}{0.943734in}%
\pgfsys@useobject{currentmarker}{}%
\end{pgfscope}%
\begin{pgfscope}%
\pgfsys@transformshift{2.312134in}{0.943734in}%
\pgfsys@useobject{currentmarker}{}%
\end{pgfscope}%
\begin{pgfscope}%
\pgfsys@transformshift{2.312134in}{0.943734in}%
\pgfsys@useobject{currentmarker}{}%
\end{pgfscope}%
\begin{pgfscope}%
\pgfsys@transformshift{2.312134in}{0.943734in}%
\pgfsys@useobject{currentmarker}{}%
\end{pgfscope}%
\begin{pgfscope}%
\pgfsys@transformshift{2.312134in}{0.943734in}%
\pgfsys@useobject{currentmarker}{}%
\end{pgfscope}%
\begin{pgfscope}%
\pgfsys@transformshift{2.312134in}{0.943734in}%
\pgfsys@useobject{currentmarker}{}%
\end{pgfscope}%
\begin{pgfscope}%
\pgfsys@transformshift{2.312134in}{0.943734in}%
\pgfsys@useobject{currentmarker}{}%
\end{pgfscope}%
\begin{pgfscope}%
\pgfsys@transformshift{2.312134in}{0.943734in}%
\pgfsys@useobject{currentmarker}{}%
\end{pgfscope}%
\begin{pgfscope}%
\pgfsys@transformshift{2.312134in}{0.943734in}%
\pgfsys@useobject{currentmarker}{}%
\end{pgfscope}%
\begin{pgfscope}%
\pgfsys@transformshift{2.312134in}{0.943734in}%
\pgfsys@useobject{currentmarker}{}%
\end{pgfscope}%
\begin{pgfscope}%
\pgfsys@transformshift{2.312134in}{0.943734in}%
\pgfsys@useobject{currentmarker}{}%
\end{pgfscope}%
\begin{pgfscope}%
\pgfsys@transformshift{2.312134in}{0.943734in}%
\pgfsys@useobject{currentmarker}{}%
\end{pgfscope}%
\begin{pgfscope}%
\pgfsys@transformshift{2.312134in}{0.943734in}%
\pgfsys@useobject{currentmarker}{}%
\end{pgfscope}%
\begin{pgfscope}%
\pgfsys@transformshift{2.312134in}{0.943734in}%
\pgfsys@useobject{currentmarker}{}%
\end{pgfscope}%
\begin{pgfscope}%
\pgfsys@transformshift{2.312134in}{0.943734in}%
\pgfsys@useobject{currentmarker}{}%
\end{pgfscope}%
\begin{pgfscope}%
\pgfsys@transformshift{2.312134in}{0.943734in}%
\pgfsys@useobject{currentmarker}{}%
\end{pgfscope}%
\begin{pgfscope}%
\pgfsys@transformshift{2.312134in}{0.943734in}%
\pgfsys@useobject{currentmarker}{}%
\end{pgfscope}%
\begin{pgfscope}%
\pgfsys@transformshift{2.312134in}{0.943734in}%
\pgfsys@useobject{currentmarker}{}%
\end{pgfscope}%
\begin{pgfscope}%
\pgfsys@transformshift{2.312134in}{0.943734in}%
\pgfsys@useobject{currentmarker}{}%
\end{pgfscope}%
\begin{pgfscope}%
\pgfsys@transformshift{2.312134in}{0.943734in}%
\pgfsys@useobject{currentmarker}{}%
\end{pgfscope}%
\begin{pgfscope}%
\pgfsys@transformshift{2.312134in}{0.943734in}%
\pgfsys@useobject{currentmarker}{}%
\end{pgfscope}%
\begin{pgfscope}%
\pgfsys@transformshift{2.312134in}{0.943734in}%
\pgfsys@useobject{currentmarker}{}%
\end{pgfscope}%
\begin{pgfscope}%
\pgfsys@transformshift{2.312134in}{0.943734in}%
\pgfsys@useobject{currentmarker}{}%
\end{pgfscope}%
\begin{pgfscope}%
\pgfsys@transformshift{2.312134in}{0.943734in}%
\pgfsys@useobject{currentmarker}{}%
\end{pgfscope}%
\begin{pgfscope}%
\pgfsys@transformshift{2.312134in}{0.943734in}%
\pgfsys@useobject{currentmarker}{}%
\end{pgfscope}%
\begin{pgfscope}%
\pgfsys@transformshift{2.312134in}{0.943734in}%
\pgfsys@useobject{currentmarker}{}%
\end{pgfscope}%
\begin{pgfscope}%
\pgfsys@transformshift{2.312134in}{0.943734in}%
\pgfsys@useobject{currentmarker}{}%
\end{pgfscope}%
\begin{pgfscope}%
\pgfsys@transformshift{2.312134in}{0.943734in}%
\pgfsys@useobject{currentmarker}{}%
\end{pgfscope}%
\begin{pgfscope}%
\pgfsys@transformshift{2.312134in}{0.943734in}%
\pgfsys@useobject{currentmarker}{}%
\end{pgfscope}%
\begin{pgfscope}%
\pgfsys@transformshift{2.312134in}{0.943734in}%
\pgfsys@useobject{currentmarker}{}%
\end{pgfscope}%
\begin{pgfscope}%
\pgfsys@transformshift{2.312134in}{0.943734in}%
\pgfsys@useobject{currentmarker}{}%
\end{pgfscope}%
\begin{pgfscope}%
\pgfsys@transformshift{2.312134in}{0.943734in}%
\pgfsys@useobject{currentmarker}{}%
\end{pgfscope}%
\begin{pgfscope}%
\pgfsys@transformshift{2.312134in}{0.943734in}%
\pgfsys@useobject{currentmarker}{}%
\end{pgfscope}%
\begin{pgfscope}%
\pgfsys@transformshift{2.312134in}{0.943734in}%
\pgfsys@useobject{currentmarker}{}%
\end{pgfscope}%
\begin{pgfscope}%
\pgfsys@transformshift{2.312134in}{0.943734in}%
\pgfsys@useobject{currentmarker}{}%
\end{pgfscope}%
\begin{pgfscope}%
\pgfsys@transformshift{2.312134in}{0.943734in}%
\pgfsys@useobject{currentmarker}{}%
\end{pgfscope}%
\begin{pgfscope}%
\pgfsys@transformshift{2.312134in}{0.943734in}%
\pgfsys@useobject{currentmarker}{}%
\end{pgfscope}%
\begin{pgfscope}%
\pgfsys@transformshift{2.312134in}{0.943734in}%
\pgfsys@useobject{currentmarker}{}%
\end{pgfscope}%
\begin{pgfscope}%
\pgfsys@transformshift{2.312134in}{0.943734in}%
\pgfsys@useobject{currentmarker}{}%
\end{pgfscope}%
\begin{pgfscope}%
\pgfsys@transformshift{2.312134in}{0.943734in}%
\pgfsys@useobject{currentmarker}{}%
\end{pgfscope}%
\begin{pgfscope}%
\pgfsys@transformshift{2.312134in}{0.943734in}%
\pgfsys@useobject{currentmarker}{}%
\end{pgfscope}%
\begin{pgfscope}%
\pgfsys@transformshift{2.312134in}{0.943734in}%
\pgfsys@useobject{currentmarker}{}%
\end{pgfscope}%
\begin{pgfscope}%
\pgfsys@transformshift{2.312134in}{0.943734in}%
\pgfsys@useobject{currentmarker}{}%
\end{pgfscope}%
\begin{pgfscope}%
\pgfsys@transformshift{2.312134in}{0.943734in}%
\pgfsys@useobject{currentmarker}{}%
\end{pgfscope}%
\begin{pgfscope}%
\pgfsys@transformshift{2.312134in}{0.943734in}%
\pgfsys@useobject{currentmarker}{}%
\end{pgfscope}%
\begin{pgfscope}%
\pgfsys@transformshift{2.312134in}{0.943734in}%
\pgfsys@useobject{currentmarker}{}%
\end{pgfscope}%
\begin{pgfscope}%
\pgfsys@transformshift{2.312134in}{0.943734in}%
\pgfsys@useobject{currentmarker}{}%
\end{pgfscope}%
\begin{pgfscope}%
\pgfsys@transformshift{2.312134in}{0.943734in}%
\pgfsys@useobject{currentmarker}{}%
\end{pgfscope}%
\begin{pgfscope}%
\pgfsys@transformshift{2.312134in}{0.943734in}%
\pgfsys@useobject{currentmarker}{}%
\end{pgfscope}%
\begin{pgfscope}%
\pgfsys@transformshift{2.312134in}{0.943734in}%
\pgfsys@useobject{currentmarker}{}%
\end{pgfscope}%
\begin{pgfscope}%
\pgfsys@transformshift{2.312134in}{0.943734in}%
\pgfsys@useobject{currentmarker}{}%
\end{pgfscope}%
\begin{pgfscope}%
\pgfsys@transformshift{2.312134in}{0.943734in}%
\pgfsys@useobject{currentmarker}{}%
\end{pgfscope}%
\begin{pgfscope}%
\pgfsys@transformshift{2.312134in}{0.943734in}%
\pgfsys@useobject{currentmarker}{}%
\end{pgfscope}%
\begin{pgfscope}%
\pgfsys@transformshift{2.312134in}{0.943734in}%
\pgfsys@useobject{currentmarker}{}%
\end{pgfscope}%
\begin{pgfscope}%
\pgfsys@transformshift{2.312134in}{0.943734in}%
\pgfsys@useobject{currentmarker}{}%
\end{pgfscope}%
\begin{pgfscope}%
\pgfsys@transformshift{2.312134in}{0.943734in}%
\pgfsys@useobject{currentmarker}{}%
\end{pgfscope}%
\begin{pgfscope}%
\pgfsys@transformshift{2.312134in}{0.943734in}%
\pgfsys@useobject{currentmarker}{}%
\end{pgfscope}%
\begin{pgfscope}%
\pgfsys@transformshift{2.312134in}{0.943734in}%
\pgfsys@useobject{currentmarker}{}%
\end{pgfscope}%
\begin{pgfscope}%
\pgfsys@transformshift{2.312134in}{0.943734in}%
\pgfsys@useobject{currentmarker}{}%
\end{pgfscope}%
\begin{pgfscope}%
\pgfsys@transformshift{2.312134in}{0.943734in}%
\pgfsys@useobject{currentmarker}{}%
\end{pgfscope}%
\begin{pgfscope}%
\pgfsys@transformshift{2.312134in}{0.943734in}%
\pgfsys@useobject{currentmarker}{}%
\end{pgfscope}%
\begin{pgfscope}%
\pgfsys@transformshift{2.312134in}{0.943734in}%
\pgfsys@useobject{currentmarker}{}%
\end{pgfscope}%
\begin{pgfscope}%
\pgfsys@transformshift{2.312134in}{0.943734in}%
\pgfsys@useobject{currentmarker}{}%
\end{pgfscope}%
\begin{pgfscope}%
\pgfsys@transformshift{2.312134in}{0.943734in}%
\pgfsys@useobject{currentmarker}{}%
\end{pgfscope}%
\begin{pgfscope}%
\pgfsys@transformshift{2.312134in}{0.943734in}%
\pgfsys@useobject{currentmarker}{}%
\end{pgfscope}%
\begin{pgfscope}%
\pgfsys@transformshift{2.312134in}{0.943734in}%
\pgfsys@useobject{currentmarker}{}%
\end{pgfscope}%
\begin{pgfscope}%
\pgfsys@transformshift{2.312134in}{0.943734in}%
\pgfsys@useobject{currentmarker}{}%
\end{pgfscope}%
\begin{pgfscope}%
\pgfsys@transformshift{2.312134in}{0.943734in}%
\pgfsys@useobject{currentmarker}{}%
\end{pgfscope}%
\begin{pgfscope}%
\pgfsys@transformshift{2.312134in}{0.943734in}%
\pgfsys@useobject{currentmarker}{}%
\end{pgfscope}%
\begin{pgfscope}%
\pgfsys@transformshift{2.312134in}{0.943734in}%
\pgfsys@useobject{currentmarker}{}%
\end{pgfscope}%
\begin{pgfscope}%
\pgfsys@transformshift{2.312134in}{0.943734in}%
\pgfsys@useobject{currentmarker}{}%
\end{pgfscope}%
\begin{pgfscope}%
\pgfsys@transformshift{2.312134in}{0.943734in}%
\pgfsys@useobject{currentmarker}{}%
\end{pgfscope}%
\begin{pgfscope}%
\pgfsys@transformshift{2.312134in}{0.943734in}%
\pgfsys@useobject{currentmarker}{}%
\end{pgfscope}%
\begin{pgfscope}%
\pgfsys@transformshift{2.312134in}{0.943734in}%
\pgfsys@useobject{currentmarker}{}%
\end{pgfscope}%
\begin{pgfscope}%
\pgfsys@transformshift{2.312134in}{0.943734in}%
\pgfsys@useobject{currentmarker}{}%
\end{pgfscope}%
\begin{pgfscope}%
\pgfsys@transformshift{2.312134in}{0.943734in}%
\pgfsys@useobject{currentmarker}{}%
\end{pgfscope}%
\begin{pgfscope}%
\pgfsys@transformshift{2.312134in}{0.943734in}%
\pgfsys@useobject{currentmarker}{}%
\end{pgfscope}%
\begin{pgfscope}%
\pgfsys@transformshift{2.312134in}{0.943734in}%
\pgfsys@useobject{currentmarker}{}%
\end{pgfscope}%
\begin{pgfscope}%
\pgfsys@transformshift{2.312134in}{0.943734in}%
\pgfsys@useobject{currentmarker}{}%
\end{pgfscope}%
\begin{pgfscope}%
\pgfsys@transformshift{2.312134in}{0.943734in}%
\pgfsys@useobject{currentmarker}{}%
\end{pgfscope}%
\begin{pgfscope}%
\pgfsys@transformshift{2.312134in}{0.943734in}%
\pgfsys@useobject{currentmarker}{}%
\end{pgfscope}%
\begin{pgfscope}%
\pgfsys@transformshift{2.312134in}{0.943734in}%
\pgfsys@useobject{currentmarker}{}%
\end{pgfscope}%
\begin{pgfscope}%
\pgfsys@transformshift{2.312134in}{0.943734in}%
\pgfsys@useobject{currentmarker}{}%
\end{pgfscope}%
\begin{pgfscope}%
\pgfsys@transformshift{2.312134in}{0.943734in}%
\pgfsys@useobject{currentmarker}{}%
\end{pgfscope}%
\begin{pgfscope}%
\pgfsys@transformshift{2.312134in}{0.943734in}%
\pgfsys@useobject{currentmarker}{}%
\end{pgfscope}%
\begin{pgfscope}%
\pgfsys@transformshift{2.312134in}{0.943734in}%
\pgfsys@useobject{currentmarker}{}%
\end{pgfscope}%
\begin{pgfscope}%
\pgfsys@transformshift{2.312134in}{0.943734in}%
\pgfsys@useobject{currentmarker}{}%
\end{pgfscope}%
\begin{pgfscope}%
\pgfsys@transformshift{2.312134in}{0.943734in}%
\pgfsys@useobject{currentmarker}{}%
\end{pgfscope}%
\begin{pgfscope}%
\pgfsys@transformshift{2.312134in}{0.943734in}%
\pgfsys@useobject{currentmarker}{}%
\end{pgfscope}%
\begin{pgfscope}%
\pgfsys@transformshift{2.312134in}{0.943734in}%
\pgfsys@useobject{currentmarker}{}%
\end{pgfscope}%
\begin{pgfscope}%
\pgfsys@transformshift{2.312134in}{0.943734in}%
\pgfsys@useobject{currentmarker}{}%
\end{pgfscope}%
\begin{pgfscope}%
\pgfsys@transformshift{2.312134in}{0.943734in}%
\pgfsys@useobject{currentmarker}{}%
\end{pgfscope}%
\begin{pgfscope}%
\pgfsys@transformshift{2.312134in}{0.943734in}%
\pgfsys@useobject{currentmarker}{}%
\end{pgfscope}%
\begin{pgfscope}%
\pgfsys@transformshift{2.312134in}{0.943734in}%
\pgfsys@useobject{currentmarker}{}%
\end{pgfscope}%
\begin{pgfscope}%
\pgfsys@transformshift{2.312134in}{0.943734in}%
\pgfsys@useobject{currentmarker}{}%
\end{pgfscope}%
\begin{pgfscope}%
\pgfsys@transformshift{2.312134in}{0.943734in}%
\pgfsys@useobject{currentmarker}{}%
\end{pgfscope}%
\begin{pgfscope}%
\pgfsys@transformshift{2.312134in}{0.943734in}%
\pgfsys@useobject{currentmarker}{}%
\end{pgfscope}%
\begin{pgfscope}%
\pgfsys@transformshift{2.312134in}{0.943734in}%
\pgfsys@useobject{currentmarker}{}%
\end{pgfscope}%
\begin{pgfscope}%
\pgfsys@transformshift{2.312134in}{0.943734in}%
\pgfsys@useobject{currentmarker}{}%
\end{pgfscope}%
\begin{pgfscope}%
\pgfsys@transformshift{2.312134in}{0.943734in}%
\pgfsys@useobject{currentmarker}{}%
\end{pgfscope}%
\begin{pgfscope}%
\pgfsys@transformshift{2.312134in}{0.943734in}%
\pgfsys@useobject{currentmarker}{}%
\end{pgfscope}%
\begin{pgfscope}%
\pgfsys@transformshift{2.312134in}{0.943734in}%
\pgfsys@useobject{currentmarker}{}%
\end{pgfscope}%
\begin{pgfscope}%
\pgfsys@transformshift{2.312134in}{0.943734in}%
\pgfsys@useobject{currentmarker}{}%
\end{pgfscope}%
\begin{pgfscope}%
\pgfsys@transformshift{2.312134in}{0.943734in}%
\pgfsys@useobject{currentmarker}{}%
\end{pgfscope}%
\begin{pgfscope}%
\pgfsys@transformshift{2.312134in}{0.943734in}%
\pgfsys@useobject{currentmarker}{}%
\end{pgfscope}%
\begin{pgfscope}%
\pgfsys@transformshift{2.312134in}{0.943734in}%
\pgfsys@useobject{currentmarker}{}%
\end{pgfscope}%
\begin{pgfscope}%
\pgfsys@transformshift{2.312134in}{0.943734in}%
\pgfsys@useobject{currentmarker}{}%
\end{pgfscope}%
\begin{pgfscope}%
\pgfsys@transformshift{2.312134in}{0.943734in}%
\pgfsys@useobject{currentmarker}{}%
\end{pgfscope}%
\begin{pgfscope}%
\pgfsys@transformshift{2.312134in}{0.943734in}%
\pgfsys@useobject{currentmarker}{}%
\end{pgfscope}%
\begin{pgfscope}%
\pgfsys@transformshift{2.312134in}{0.943734in}%
\pgfsys@useobject{currentmarker}{}%
\end{pgfscope}%
\begin{pgfscope}%
\pgfsys@transformshift{2.312134in}{0.943734in}%
\pgfsys@useobject{currentmarker}{}%
\end{pgfscope}%
\begin{pgfscope}%
\pgfsys@transformshift{2.312134in}{0.943734in}%
\pgfsys@useobject{currentmarker}{}%
\end{pgfscope}%
\begin{pgfscope}%
\pgfsys@transformshift{2.312134in}{0.943734in}%
\pgfsys@useobject{currentmarker}{}%
\end{pgfscope}%
\begin{pgfscope}%
\pgfsys@transformshift{2.312134in}{0.943734in}%
\pgfsys@useobject{currentmarker}{}%
\end{pgfscope}%
\begin{pgfscope}%
\pgfsys@transformshift{2.312134in}{0.943734in}%
\pgfsys@useobject{currentmarker}{}%
\end{pgfscope}%
\begin{pgfscope}%
\pgfsys@transformshift{2.312134in}{0.943734in}%
\pgfsys@useobject{currentmarker}{}%
\end{pgfscope}%
\begin{pgfscope}%
\pgfsys@transformshift{2.312134in}{0.943734in}%
\pgfsys@useobject{currentmarker}{}%
\end{pgfscope}%
\begin{pgfscope}%
\pgfsys@transformshift{2.312134in}{0.943734in}%
\pgfsys@useobject{currentmarker}{}%
\end{pgfscope}%
\begin{pgfscope}%
\pgfsys@transformshift{2.312134in}{0.943734in}%
\pgfsys@useobject{currentmarker}{}%
\end{pgfscope}%
\begin{pgfscope}%
\pgfsys@transformshift{2.312134in}{0.943734in}%
\pgfsys@useobject{currentmarker}{}%
\end{pgfscope}%
\begin{pgfscope}%
\pgfsys@transformshift{2.312134in}{0.943734in}%
\pgfsys@useobject{currentmarker}{}%
\end{pgfscope}%
\begin{pgfscope}%
\pgfsys@transformshift{2.312134in}{0.943734in}%
\pgfsys@useobject{currentmarker}{}%
\end{pgfscope}%
\begin{pgfscope}%
\pgfsys@transformshift{2.312134in}{0.943734in}%
\pgfsys@useobject{currentmarker}{}%
\end{pgfscope}%
\begin{pgfscope}%
\pgfsys@transformshift{2.312134in}{0.943734in}%
\pgfsys@useobject{currentmarker}{}%
\end{pgfscope}%
\begin{pgfscope}%
\pgfsys@transformshift{2.312134in}{0.943734in}%
\pgfsys@useobject{currentmarker}{}%
\end{pgfscope}%
\begin{pgfscope}%
\pgfsys@transformshift{2.312134in}{0.943734in}%
\pgfsys@useobject{currentmarker}{}%
\end{pgfscope}%
\begin{pgfscope}%
\pgfsys@transformshift{2.312134in}{0.943734in}%
\pgfsys@useobject{currentmarker}{}%
\end{pgfscope}%
\begin{pgfscope}%
\pgfsys@transformshift{2.312134in}{0.943734in}%
\pgfsys@useobject{currentmarker}{}%
\end{pgfscope}%
\begin{pgfscope}%
\pgfsys@transformshift{2.312134in}{0.943734in}%
\pgfsys@useobject{currentmarker}{}%
\end{pgfscope}%
\begin{pgfscope}%
\pgfsys@transformshift{2.312134in}{0.943734in}%
\pgfsys@useobject{currentmarker}{}%
\end{pgfscope}%
\begin{pgfscope}%
\pgfsys@transformshift{2.312134in}{0.943734in}%
\pgfsys@useobject{currentmarker}{}%
\end{pgfscope}%
\begin{pgfscope}%
\pgfsys@transformshift{2.312134in}{0.943734in}%
\pgfsys@useobject{currentmarker}{}%
\end{pgfscope}%
\begin{pgfscope}%
\pgfsys@transformshift{2.312134in}{0.943734in}%
\pgfsys@useobject{currentmarker}{}%
\end{pgfscope}%
\begin{pgfscope}%
\pgfsys@transformshift{2.312134in}{0.943734in}%
\pgfsys@useobject{currentmarker}{}%
\end{pgfscope}%
\begin{pgfscope}%
\pgfsys@transformshift{2.312134in}{0.943734in}%
\pgfsys@useobject{currentmarker}{}%
\end{pgfscope}%
\begin{pgfscope}%
\pgfsys@transformshift{2.312134in}{0.943734in}%
\pgfsys@useobject{currentmarker}{}%
\end{pgfscope}%
\begin{pgfscope}%
\pgfsys@transformshift{2.312134in}{0.943734in}%
\pgfsys@useobject{currentmarker}{}%
\end{pgfscope}%
\begin{pgfscope}%
\pgfsys@transformshift{2.312134in}{0.943734in}%
\pgfsys@useobject{currentmarker}{}%
\end{pgfscope}%
\begin{pgfscope}%
\pgfsys@transformshift{2.312134in}{0.943734in}%
\pgfsys@useobject{currentmarker}{}%
\end{pgfscope}%
\begin{pgfscope}%
\pgfsys@transformshift{2.312134in}{0.943734in}%
\pgfsys@useobject{currentmarker}{}%
\end{pgfscope}%
\begin{pgfscope}%
\pgfsys@transformshift{2.312134in}{0.943734in}%
\pgfsys@useobject{currentmarker}{}%
\end{pgfscope}%
\begin{pgfscope}%
\pgfsys@transformshift{2.312134in}{0.943734in}%
\pgfsys@useobject{currentmarker}{}%
\end{pgfscope}%
\begin{pgfscope}%
\pgfsys@transformshift{2.312134in}{0.943734in}%
\pgfsys@useobject{currentmarker}{}%
\end{pgfscope}%
\begin{pgfscope}%
\pgfsys@transformshift{2.312134in}{0.943734in}%
\pgfsys@useobject{currentmarker}{}%
\end{pgfscope}%
\begin{pgfscope}%
\pgfsys@transformshift{2.312134in}{0.943734in}%
\pgfsys@useobject{currentmarker}{}%
\end{pgfscope}%
\begin{pgfscope}%
\pgfsys@transformshift{2.312134in}{0.943734in}%
\pgfsys@useobject{currentmarker}{}%
\end{pgfscope}%
\begin{pgfscope}%
\pgfsys@transformshift{2.312134in}{0.943734in}%
\pgfsys@useobject{currentmarker}{}%
\end{pgfscope}%
\begin{pgfscope}%
\pgfsys@transformshift{2.312134in}{0.943734in}%
\pgfsys@useobject{currentmarker}{}%
\end{pgfscope}%
\begin{pgfscope}%
\pgfsys@transformshift{2.312134in}{0.943734in}%
\pgfsys@useobject{currentmarker}{}%
\end{pgfscope}%
\begin{pgfscope}%
\pgfsys@transformshift{2.312134in}{0.943734in}%
\pgfsys@useobject{currentmarker}{}%
\end{pgfscope}%
\begin{pgfscope}%
\pgfsys@transformshift{2.312134in}{0.943734in}%
\pgfsys@useobject{currentmarker}{}%
\end{pgfscope}%
\begin{pgfscope}%
\pgfsys@transformshift{2.312134in}{0.943734in}%
\pgfsys@useobject{currentmarker}{}%
\end{pgfscope}%
\begin{pgfscope}%
\pgfsys@transformshift{2.312134in}{0.943734in}%
\pgfsys@useobject{currentmarker}{}%
\end{pgfscope}%
\begin{pgfscope}%
\pgfsys@transformshift{2.312134in}{0.943734in}%
\pgfsys@useobject{currentmarker}{}%
\end{pgfscope}%
\begin{pgfscope}%
\pgfsys@transformshift{2.312134in}{0.943734in}%
\pgfsys@useobject{currentmarker}{}%
\end{pgfscope}%
\begin{pgfscope}%
\pgfsys@transformshift{2.312134in}{0.943734in}%
\pgfsys@useobject{currentmarker}{}%
\end{pgfscope}%
\begin{pgfscope}%
\pgfsys@transformshift{2.312134in}{0.943734in}%
\pgfsys@useobject{currentmarker}{}%
\end{pgfscope}%
\begin{pgfscope}%
\pgfsys@transformshift{2.312134in}{0.943734in}%
\pgfsys@useobject{currentmarker}{}%
\end{pgfscope}%
\begin{pgfscope}%
\pgfsys@transformshift{2.312134in}{0.943734in}%
\pgfsys@useobject{currentmarker}{}%
\end{pgfscope}%
\begin{pgfscope}%
\pgfsys@transformshift{2.312134in}{0.943734in}%
\pgfsys@useobject{currentmarker}{}%
\end{pgfscope}%
\begin{pgfscope}%
\pgfsys@transformshift{2.312134in}{0.943734in}%
\pgfsys@useobject{currentmarker}{}%
\end{pgfscope}%
\begin{pgfscope}%
\pgfsys@transformshift{2.312134in}{0.943734in}%
\pgfsys@useobject{currentmarker}{}%
\end{pgfscope}%
\begin{pgfscope}%
\pgfsys@transformshift{2.312134in}{0.943734in}%
\pgfsys@useobject{currentmarker}{}%
\end{pgfscope}%
\begin{pgfscope}%
\pgfsys@transformshift{2.312134in}{0.943734in}%
\pgfsys@useobject{currentmarker}{}%
\end{pgfscope}%
\begin{pgfscope}%
\pgfsys@transformshift{2.312134in}{0.943734in}%
\pgfsys@useobject{currentmarker}{}%
\end{pgfscope}%
\begin{pgfscope}%
\pgfsys@transformshift{2.312134in}{0.943734in}%
\pgfsys@useobject{currentmarker}{}%
\end{pgfscope}%
\begin{pgfscope}%
\pgfsys@transformshift{2.312134in}{0.943734in}%
\pgfsys@useobject{currentmarker}{}%
\end{pgfscope}%
\begin{pgfscope}%
\pgfsys@transformshift{2.312134in}{0.943734in}%
\pgfsys@useobject{currentmarker}{}%
\end{pgfscope}%
\begin{pgfscope}%
\pgfsys@transformshift{2.312134in}{0.943734in}%
\pgfsys@useobject{currentmarker}{}%
\end{pgfscope}%
\begin{pgfscope}%
\pgfsys@transformshift{2.312134in}{0.943734in}%
\pgfsys@useobject{currentmarker}{}%
\end{pgfscope}%
\begin{pgfscope}%
\pgfsys@transformshift{2.312134in}{0.943734in}%
\pgfsys@useobject{currentmarker}{}%
\end{pgfscope}%
\begin{pgfscope}%
\pgfsys@transformshift{2.312134in}{0.943734in}%
\pgfsys@useobject{currentmarker}{}%
\end{pgfscope}%
\begin{pgfscope}%
\pgfsys@transformshift{2.312134in}{0.943734in}%
\pgfsys@useobject{currentmarker}{}%
\end{pgfscope}%
\begin{pgfscope}%
\pgfsys@transformshift{2.312134in}{0.943734in}%
\pgfsys@useobject{currentmarker}{}%
\end{pgfscope}%
\begin{pgfscope}%
\pgfsys@transformshift{2.312134in}{0.943734in}%
\pgfsys@useobject{currentmarker}{}%
\end{pgfscope}%
\begin{pgfscope}%
\pgfsys@transformshift{2.312134in}{0.943734in}%
\pgfsys@useobject{currentmarker}{}%
\end{pgfscope}%
\begin{pgfscope}%
\pgfsys@transformshift{2.312134in}{0.943734in}%
\pgfsys@useobject{currentmarker}{}%
\end{pgfscope}%
\begin{pgfscope}%
\pgfsys@transformshift{2.312134in}{0.943734in}%
\pgfsys@useobject{currentmarker}{}%
\end{pgfscope}%
\begin{pgfscope}%
\pgfsys@transformshift{2.312134in}{0.943734in}%
\pgfsys@useobject{currentmarker}{}%
\end{pgfscope}%
\begin{pgfscope}%
\pgfsys@transformshift{2.312134in}{0.943734in}%
\pgfsys@useobject{currentmarker}{}%
\end{pgfscope}%
\begin{pgfscope}%
\pgfsys@transformshift{2.312134in}{0.943734in}%
\pgfsys@useobject{currentmarker}{}%
\end{pgfscope}%
\begin{pgfscope}%
\pgfsys@transformshift{2.312134in}{0.943734in}%
\pgfsys@useobject{currentmarker}{}%
\end{pgfscope}%
\begin{pgfscope}%
\pgfsys@transformshift{2.312134in}{0.943734in}%
\pgfsys@useobject{currentmarker}{}%
\end{pgfscope}%
\begin{pgfscope}%
\pgfsys@transformshift{2.312134in}{0.943734in}%
\pgfsys@useobject{currentmarker}{}%
\end{pgfscope}%
\begin{pgfscope}%
\pgfsys@transformshift{2.312134in}{0.943734in}%
\pgfsys@useobject{currentmarker}{}%
\end{pgfscope}%
\begin{pgfscope}%
\pgfsys@transformshift{2.312134in}{0.943734in}%
\pgfsys@useobject{currentmarker}{}%
\end{pgfscope}%
\begin{pgfscope}%
\pgfsys@transformshift{2.312134in}{0.943734in}%
\pgfsys@useobject{currentmarker}{}%
\end{pgfscope}%
\begin{pgfscope}%
\pgfsys@transformshift{2.312134in}{0.943734in}%
\pgfsys@useobject{currentmarker}{}%
\end{pgfscope}%
\begin{pgfscope}%
\pgfsys@transformshift{2.312134in}{0.943734in}%
\pgfsys@useobject{currentmarker}{}%
\end{pgfscope}%
\begin{pgfscope}%
\pgfsys@transformshift{2.312134in}{0.943734in}%
\pgfsys@useobject{currentmarker}{}%
\end{pgfscope}%
\begin{pgfscope}%
\pgfsys@transformshift{2.312134in}{0.943734in}%
\pgfsys@useobject{currentmarker}{}%
\end{pgfscope}%
\begin{pgfscope}%
\pgfsys@transformshift{2.312134in}{0.943734in}%
\pgfsys@useobject{currentmarker}{}%
\end{pgfscope}%
\begin{pgfscope}%
\pgfsys@transformshift{2.312134in}{0.943734in}%
\pgfsys@useobject{currentmarker}{}%
\end{pgfscope}%
\begin{pgfscope}%
\pgfsys@transformshift{2.312134in}{0.943734in}%
\pgfsys@useobject{currentmarker}{}%
\end{pgfscope}%
\begin{pgfscope}%
\pgfsys@transformshift{2.312134in}{0.943734in}%
\pgfsys@useobject{currentmarker}{}%
\end{pgfscope}%
\begin{pgfscope}%
\pgfsys@transformshift{2.312134in}{0.943734in}%
\pgfsys@useobject{currentmarker}{}%
\end{pgfscope}%
\begin{pgfscope}%
\pgfsys@transformshift{2.312134in}{0.943734in}%
\pgfsys@useobject{currentmarker}{}%
\end{pgfscope}%
\begin{pgfscope}%
\pgfsys@transformshift{2.312134in}{0.943734in}%
\pgfsys@useobject{currentmarker}{}%
\end{pgfscope}%
\begin{pgfscope}%
\pgfsys@transformshift{2.312134in}{0.943734in}%
\pgfsys@useobject{currentmarker}{}%
\end{pgfscope}%
\begin{pgfscope}%
\pgfsys@transformshift{2.312134in}{0.943734in}%
\pgfsys@useobject{currentmarker}{}%
\end{pgfscope}%
\begin{pgfscope}%
\pgfsys@transformshift{2.312134in}{0.943734in}%
\pgfsys@useobject{currentmarker}{}%
\end{pgfscope}%
\begin{pgfscope}%
\pgfsys@transformshift{2.312134in}{0.943734in}%
\pgfsys@useobject{currentmarker}{}%
\end{pgfscope}%
\begin{pgfscope}%
\pgfsys@transformshift{2.312134in}{0.943734in}%
\pgfsys@useobject{currentmarker}{}%
\end{pgfscope}%
\begin{pgfscope}%
\pgfsys@transformshift{2.312134in}{0.943734in}%
\pgfsys@useobject{currentmarker}{}%
\end{pgfscope}%
\begin{pgfscope}%
\pgfsys@transformshift{2.312134in}{0.943734in}%
\pgfsys@useobject{currentmarker}{}%
\end{pgfscope}%
\begin{pgfscope}%
\pgfsys@transformshift{2.312134in}{0.943734in}%
\pgfsys@useobject{currentmarker}{}%
\end{pgfscope}%
\begin{pgfscope}%
\pgfsys@transformshift{2.312134in}{0.943734in}%
\pgfsys@useobject{currentmarker}{}%
\end{pgfscope}%
\begin{pgfscope}%
\pgfsys@transformshift{2.312134in}{0.943734in}%
\pgfsys@useobject{currentmarker}{}%
\end{pgfscope}%
\begin{pgfscope}%
\pgfsys@transformshift{2.312134in}{0.943734in}%
\pgfsys@useobject{currentmarker}{}%
\end{pgfscope}%
\begin{pgfscope}%
\pgfsys@transformshift{2.312134in}{0.943734in}%
\pgfsys@useobject{currentmarker}{}%
\end{pgfscope}%
\begin{pgfscope}%
\pgfsys@transformshift{2.312134in}{0.943734in}%
\pgfsys@useobject{currentmarker}{}%
\end{pgfscope}%
\begin{pgfscope}%
\pgfsys@transformshift{2.312134in}{0.943734in}%
\pgfsys@useobject{currentmarker}{}%
\end{pgfscope}%
\begin{pgfscope}%
\pgfsys@transformshift{2.312134in}{0.943734in}%
\pgfsys@useobject{currentmarker}{}%
\end{pgfscope}%
\begin{pgfscope}%
\pgfsys@transformshift{2.312134in}{0.943734in}%
\pgfsys@useobject{currentmarker}{}%
\end{pgfscope}%
\begin{pgfscope}%
\pgfsys@transformshift{2.312134in}{0.943734in}%
\pgfsys@useobject{currentmarker}{}%
\end{pgfscope}%
\begin{pgfscope}%
\pgfsys@transformshift{2.312134in}{0.943734in}%
\pgfsys@useobject{currentmarker}{}%
\end{pgfscope}%
\begin{pgfscope}%
\pgfsys@transformshift{2.312134in}{0.943734in}%
\pgfsys@useobject{currentmarker}{}%
\end{pgfscope}%
\begin{pgfscope}%
\pgfsys@transformshift{2.312134in}{0.943734in}%
\pgfsys@useobject{currentmarker}{}%
\end{pgfscope}%
\begin{pgfscope}%
\pgfsys@transformshift{2.312134in}{0.943734in}%
\pgfsys@useobject{currentmarker}{}%
\end{pgfscope}%
\begin{pgfscope}%
\pgfsys@transformshift{2.312134in}{0.943734in}%
\pgfsys@useobject{currentmarker}{}%
\end{pgfscope}%
\begin{pgfscope}%
\pgfsys@transformshift{2.312134in}{0.943734in}%
\pgfsys@useobject{currentmarker}{}%
\end{pgfscope}%
\begin{pgfscope}%
\pgfsys@transformshift{2.312134in}{0.943734in}%
\pgfsys@useobject{currentmarker}{}%
\end{pgfscope}%
\begin{pgfscope}%
\pgfsys@transformshift{2.312134in}{0.943734in}%
\pgfsys@useobject{currentmarker}{}%
\end{pgfscope}%
\begin{pgfscope}%
\pgfsys@transformshift{2.312134in}{0.943734in}%
\pgfsys@useobject{currentmarker}{}%
\end{pgfscope}%
\begin{pgfscope}%
\pgfsys@transformshift{2.312134in}{0.943734in}%
\pgfsys@useobject{currentmarker}{}%
\end{pgfscope}%
\begin{pgfscope}%
\pgfsys@transformshift{2.312134in}{0.943734in}%
\pgfsys@useobject{currentmarker}{}%
\end{pgfscope}%
\begin{pgfscope}%
\pgfsys@transformshift{2.312134in}{0.943734in}%
\pgfsys@useobject{currentmarker}{}%
\end{pgfscope}%
\begin{pgfscope}%
\pgfsys@transformshift{2.312134in}{0.943734in}%
\pgfsys@useobject{currentmarker}{}%
\end{pgfscope}%
\begin{pgfscope}%
\pgfsys@transformshift{2.312134in}{0.943734in}%
\pgfsys@useobject{currentmarker}{}%
\end{pgfscope}%
\begin{pgfscope}%
\pgfsys@transformshift{2.312134in}{0.943734in}%
\pgfsys@useobject{currentmarker}{}%
\end{pgfscope}%
\begin{pgfscope}%
\pgfsys@transformshift{2.312134in}{0.943734in}%
\pgfsys@useobject{currentmarker}{}%
\end{pgfscope}%
\begin{pgfscope}%
\pgfsys@transformshift{2.312134in}{0.943734in}%
\pgfsys@useobject{currentmarker}{}%
\end{pgfscope}%
\begin{pgfscope}%
\pgfsys@transformshift{2.312134in}{0.943734in}%
\pgfsys@useobject{currentmarker}{}%
\end{pgfscope}%
\begin{pgfscope}%
\pgfsys@transformshift{2.312134in}{0.943734in}%
\pgfsys@useobject{currentmarker}{}%
\end{pgfscope}%
\begin{pgfscope}%
\pgfsys@transformshift{2.312134in}{0.943734in}%
\pgfsys@useobject{currentmarker}{}%
\end{pgfscope}%
\begin{pgfscope}%
\pgfsys@transformshift{2.312134in}{0.943734in}%
\pgfsys@useobject{currentmarker}{}%
\end{pgfscope}%
\begin{pgfscope}%
\pgfsys@transformshift{2.312134in}{0.943734in}%
\pgfsys@useobject{currentmarker}{}%
\end{pgfscope}%
\begin{pgfscope}%
\pgfsys@transformshift{2.312134in}{0.943734in}%
\pgfsys@useobject{currentmarker}{}%
\end{pgfscope}%
\begin{pgfscope}%
\pgfsys@transformshift{2.312134in}{0.943734in}%
\pgfsys@useobject{currentmarker}{}%
\end{pgfscope}%
\begin{pgfscope}%
\pgfsys@transformshift{2.312134in}{0.943734in}%
\pgfsys@useobject{currentmarker}{}%
\end{pgfscope}%
\begin{pgfscope}%
\pgfsys@transformshift{2.312134in}{0.943734in}%
\pgfsys@useobject{currentmarker}{}%
\end{pgfscope}%
\begin{pgfscope}%
\pgfsys@transformshift{2.312134in}{0.943734in}%
\pgfsys@useobject{currentmarker}{}%
\end{pgfscope}%
\begin{pgfscope}%
\pgfsys@transformshift{2.312134in}{0.943734in}%
\pgfsys@useobject{currentmarker}{}%
\end{pgfscope}%
\begin{pgfscope}%
\pgfsys@transformshift{2.312134in}{0.943734in}%
\pgfsys@useobject{currentmarker}{}%
\end{pgfscope}%
\begin{pgfscope}%
\pgfsys@transformshift{2.312134in}{0.943734in}%
\pgfsys@useobject{currentmarker}{}%
\end{pgfscope}%
\begin{pgfscope}%
\pgfsys@transformshift{2.312134in}{0.943734in}%
\pgfsys@useobject{currentmarker}{}%
\end{pgfscope}%
\begin{pgfscope}%
\pgfsys@transformshift{2.312134in}{0.943734in}%
\pgfsys@useobject{currentmarker}{}%
\end{pgfscope}%
\begin{pgfscope}%
\pgfsys@transformshift{2.312134in}{0.943734in}%
\pgfsys@useobject{currentmarker}{}%
\end{pgfscope}%
\begin{pgfscope}%
\pgfsys@transformshift{2.312134in}{0.943734in}%
\pgfsys@useobject{currentmarker}{}%
\end{pgfscope}%
\begin{pgfscope}%
\pgfsys@transformshift{2.312134in}{0.943734in}%
\pgfsys@useobject{currentmarker}{}%
\end{pgfscope}%
\begin{pgfscope}%
\pgfsys@transformshift{2.312134in}{0.943734in}%
\pgfsys@useobject{currentmarker}{}%
\end{pgfscope}%
\begin{pgfscope}%
\pgfsys@transformshift{2.312134in}{0.943734in}%
\pgfsys@useobject{currentmarker}{}%
\end{pgfscope}%
\begin{pgfscope}%
\pgfsys@transformshift{2.312134in}{0.943734in}%
\pgfsys@useobject{currentmarker}{}%
\end{pgfscope}%
\begin{pgfscope}%
\pgfsys@transformshift{2.312134in}{0.943734in}%
\pgfsys@useobject{currentmarker}{}%
\end{pgfscope}%
\begin{pgfscope}%
\pgfsys@transformshift{2.312134in}{0.943734in}%
\pgfsys@useobject{currentmarker}{}%
\end{pgfscope}%
\begin{pgfscope}%
\pgfsys@transformshift{2.312134in}{0.943734in}%
\pgfsys@useobject{currentmarker}{}%
\end{pgfscope}%
\begin{pgfscope}%
\pgfsys@transformshift{2.312134in}{0.943734in}%
\pgfsys@useobject{currentmarker}{}%
\end{pgfscope}%
\begin{pgfscope}%
\pgfsys@transformshift{2.312134in}{0.943734in}%
\pgfsys@useobject{currentmarker}{}%
\end{pgfscope}%
\begin{pgfscope}%
\pgfsys@transformshift{2.312134in}{0.943734in}%
\pgfsys@useobject{currentmarker}{}%
\end{pgfscope}%
\begin{pgfscope}%
\pgfsys@transformshift{2.312134in}{0.943734in}%
\pgfsys@useobject{currentmarker}{}%
\end{pgfscope}%
\begin{pgfscope}%
\pgfsys@transformshift{2.312134in}{0.943734in}%
\pgfsys@useobject{currentmarker}{}%
\end{pgfscope}%
\begin{pgfscope}%
\pgfsys@transformshift{2.312134in}{0.943734in}%
\pgfsys@useobject{currentmarker}{}%
\end{pgfscope}%
\begin{pgfscope}%
\pgfsys@transformshift{2.312134in}{0.943734in}%
\pgfsys@useobject{currentmarker}{}%
\end{pgfscope}%
\begin{pgfscope}%
\pgfsys@transformshift{2.312134in}{0.943734in}%
\pgfsys@useobject{currentmarker}{}%
\end{pgfscope}%
\begin{pgfscope}%
\pgfsys@transformshift{2.312134in}{0.943734in}%
\pgfsys@useobject{currentmarker}{}%
\end{pgfscope}%
\begin{pgfscope}%
\pgfsys@transformshift{2.312134in}{0.943734in}%
\pgfsys@useobject{currentmarker}{}%
\end{pgfscope}%
\begin{pgfscope}%
\pgfsys@transformshift{2.312134in}{0.943734in}%
\pgfsys@useobject{currentmarker}{}%
\end{pgfscope}%
\begin{pgfscope}%
\pgfsys@transformshift{2.312134in}{0.943734in}%
\pgfsys@useobject{currentmarker}{}%
\end{pgfscope}%
\begin{pgfscope}%
\pgfsys@transformshift{2.312134in}{0.943734in}%
\pgfsys@useobject{currentmarker}{}%
\end{pgfscope}%
\begin{pgfscope}%
\pgfsys@transformshift{2.312134in}{0.943734in}%
\pgfsys@useobject{currentmarker}{}%
\end{pgfscope}%
\begin{pgfscope}%
\pgfsys@transformshift{2.312134in}{0.943734in}%
\pgfsys@useobject{currentmarker}{}%
\end{pgfscope}%
\begin{pgfscope}%
\pgfsys@transformshift{2.312134in}{0.943734in}%
\pgfsys@useobject{currentmarker}{}%
\end{pgfscope}%
\begin{pgfscope}%
\pgfsys@transformshift{2.312134in}{0.943734in}%
\pgfsys@useobject{currentmarker}{}%
\end{pgfscope}%
\begin{pgfscope}%
\pgfsys@transformshift{2.312134in}{0.943734in}%
\pgfsys@useobject{currentmarker}{}%
\end{pgfscope}%
\begin{pgfscope}%
\pgfsys@transformshift{2.312134in}{0.943734in}%
\pgfsys@useobject{currentmarker}{}%
\end{pgfscope}%
\begin{pgfscope}%
\pgfsys@transformshift{2.312134in}{0.943734in}%
\pgfsys@useobject{currentmarker}{}%
\end{pgfscope}%
\begin{pgfscope}%
\pgfsys@transformshift{2.312134in}{0.943734in}%
\pgfsys@useobject{currentmarker}{}%
\end{pgfscope}%
\begin{pgfscope}%
\pgfsys@transformshift{2.312134in}{0.943734in}%
\pgfsys@useobject{currentmarker}{}%
\end{pgfscope}%
\begin{pgfscope}%
\pgfsys@transformshift{2.312134in}{0.943734in}%
\pgfsys@useobject{currentmarker}{}%
\end{pgfscope}%
\begin{pgfscope}%
\pgfsys@transformshift{2.312134in}{0.943734in}%
\pgfsys@useobject{currentmarker}{}%
\end{pgfscope}%
\begin{pgfscope}%
\pgfsys@transformshift{2.312134in}{0.943734in}%
\pgfsys@useobject{currentmarker}{}%
\end{pgfscope}%
\begin{pgfscope}%
\pgfsys@transformshift{2.312134in}{0.943734in}%
\pgfsys@useobject{currentmarker}{}%
\end{pgfscope}%
\begin{pgfscope}%
\pgfsys@transformshift{2.312134in}{0.943734in}%
\pgfsys@useobject{currentmarker}{}%
\end{pgfscope}%
\begin{pgfscope}%
\pgfsys@transformshift{2.312134in}{0.943734in}%
\pgfsys@useobject{currentmarker}{}%
\end{pgfscope}%
\begin{pgfscope}%
\pgfsys@transformshift{2.312134in}{0.943734in}%
\pgfsys@useobject{currentmarker}{}%
\end{pgfscope}%
\begin{pgfscope}%
\pgfsys@transformshift{2.312134in}{0.943734in}%
\pgfsys@useobject{currentmarker}{}%
\end{pgfscope}%
\begin{pgfscope}%
\pgfsys@transformshift{2.312134in}{0.943734in}%
\pgfsys@useobject{currentmarker}{}%
\end{pgfscope}%
\begin{pgfscope}%
\pgfsys@transformshift{2.312134in}{0.943734in}%
\pgfsys@useobject{currentmarker}{}%
\end{pgfscope}%
\begin{pgfscope}%
\pgfsys@transformshift{2.312134in}{0.943734in}%
\pgfsys@useobject{currentmarker}{}%
\end{pgfscope}%
\begin{pgfscope}%
\pgfsys@transformshift{2.312134in}{0.943734in}%
\pgfsys@useobject{currentmarker}{}%
\end{pgfscope}%
\begin{pgfscope}%
\pgfsys@transformshift{2.312134in}{0.943734in}%
\pgfsys@useobject{currentmarker}{}%
\end{pgfscope}%
\begin{pgfscope}%
\pgfsys@transformshift{2.312134in}{0.943734in}%
\pgfsys@useobject{currentmarker}{}%
\end{pgfscope}%
\begin{pgfscope}%
\pgfsys@transformshift{2.312134in}{0.943734in}%
\pgfsys@useobject{currentmarker}{}%
\end{pgfscope}%
\begin{pgfscope}%
\pgfsys@transformshift{2.312134in}{0.943734in}%
\pgfsys@useobject{currentmarker}{}%
\end{pgfscope}%
\begin{pgfscope}%
\pgfsys@transformshift{2.312134in}{0.943734in}%
\pgfsys@useobject{currentmarker}{}%
\end{pgfscope}%
\begin{pgfscope}%
\pgfsys@transformshift{2.312134in}{0.943734in}%
\pgfsys@useobject{currentmarker}{}%
\end{pgfscope}%
\begin{pgfscope}%
\pgfsys@transformshift{2.312134in}{0.943734in}%
\pgfsys@useobject{currentmarker}{}%
\end{pgfscope}%
\begin{pgfscope}%
\pgfsys@transformshift{2.312134in}{0.943734in}%
\pgfsys@useobject{currentmarker}{}%
\end{pgfscope}%
\begin{pgfscope}%
\pgfsys@transformshift{2.312134in}{0.943734in}%
\pgfsys@useobject{currentmarker}{}%
\end{pgfscope}%
\begin{pgfscope}%
\pgfsys@transformshift{2.312134in}{0.943734in}%
\pgfsys@useobject{currentmarker}{}%
\end{pgfscope}%
\begin{pgfscope}%
\pgfsys@transformshift{2.312134in}{0.943734in}%
\pgfsys@useobject{currentmarker}{}%
\end{pgfscope}%
\begin{pgfscope}%
\pgfsys@transformshift{2.312134in}{0.943734in}%
\pgfsys@useobject{currentmarker}{}%
\end{pgfscope}%
\begin{pgfscope}%
\pgfsys@transformshift{2.312134in}{0.943734in}%
\pgfsys@useobject{currentmarker}{}%
\end{pgfscope}%
\begin{pgfscope}%
\pgfsys@transformshift{2.312134in}{0.943734in}%
\pgfsys@useobject{currentmarker}{}%
\end{pgfscope}%
\begin{pgfscope}%
\pgfsys@transformshift{2.312134in}{0.943734in}%
\pgfsys@useobject{currentmarker}{}%
\end{pgfscope}%
\begin{pgfscope}%
\pgfsys@transformshift{2.312134in}{0.943734in}%
\pgfsys@useobject{currentmarker}{}%
\end{pgfscope}%
\end{pgfscope}%
\begin{pgfscope}%
\pgfpathrectangle{\pgfqpoint{0.562500in}{0.275000in}}{\pgfqpoint{3.487500in}{1.925000in}}%
\pgfusepath{clip}%
\pgfsetrectcap%
\pgfsetroundjoin%
\pgfsetlinewidth{1.505625pt}%
\definecolor{currentstroke}{rgb}{1.000000,0.498039,0.054902}%
\pgfsetstrokecolor{currentstroke}%
\pgfsetdash{}{0pt}%
\pgfpathmoveto{\pgfqpoint{2.312134in}{0.943734in}}%
\pgfpathlineto{\pgfqpoint{2.312134in}{0.943734in}}%
\pgfusepath{stroke}%
\end{pgfscope}%
\begin{pgfscope}%
\pgfpathrectangle{\pgfqpoint{0.562500in}{0.275000in}}{\pgfqpoint{3.487500in}{1.925000in}}%
\pgfusepath{clip}%
\pgfsetrectcap%
\pgfsetroundjoin%
\pgfsetlinewidth{1.505625pt}%
\definecolor{currentstroke}{rgb}{0.172549,0.627451,0.172549}%
\pgfsetstrokecolor{currentstroke}%
\pgfsetdash{}{0pt}%
\pgfpathmoveto{\pgfqpoint{3.891477in}{0.362500in}}%
\pgfpathlineto{\pgfqpoint{3.852525in}{0.462371in}}%
\pgfpathlineto{\pgfqpoint{3.761952in}{0.533896in}}%
\pgfpathlineto{\pgfqpoint{3.644763in}{0.583099in}}%
\pgfpathlineto{\pgfqpoint{3.516977in}{0.616087in}}%
\pgfpathlineto{\pgfqpoint{3.387283in}{0.639333in}}%
\pgfpathlineto{\pgfqpoint{3.259617in}{0.656646in}}%
\pgfpathlineto{\pgfqpoint{3.136438in}{0.669953in}}%
\pgfpathlineto{\pgfqpoint{2.808048in}{0.702394in}}%
\pgfpathlineto{\pgfqpoint{2.714816in}{0.713564in}}%
\pgfpathlineto{\pgfqpoint{2.630104in}{0.725426in}}%
\pgfpathlineto{\pgfqpoint{2.553950in}{0.738038in}}%
\pgfpathlineto{\pgfqpoint{2.486290in}{0.751356in}}%
\pgfpathlineto{\pgfqpoint{2.426956in}{0.765236in}}%
\pgfpathlineto{\pgfqpoint{2.375675in}{0.779443in}}%
\pgfpathlineto{\pgfqpoint{2.332026in}{0.793769in}}%
\pgfpathlineto{\pgfqpoint{2.295450in}{0.808028in}}%
\pgfpathlineto{\pgfqpoint{2.265386in}{0.822039in}}%
\pgfpathlineto{\pgfqpoint{2.241288in}{0.835639in}}%
\pgfpathlineto{\pgfqpoint{2.222620in}{0.848679in}}%
\pgfpathlineto{\pgfqpoint{2.208821in}{0.861035in}}%
\pgfpathlineto{\pgfqpoint{2.199259in}{0.872627in}}%
\pgfpathlineto{\pgfqpoint{2.193352in}{0.883399in}}%
\pgfpathlineto{\pgfqpoint{2.190568in}{0.893311in}}%
\pgfpathlineto{\pgfqpoint{2.190425in}{0.902339in}}%
\pgfpathlineto{\pgfqpoint{2.192489in}{0.910473in}}%
\pgfpathlineto{\pgfqpoint{2.196326in}{0.917728in}}%
\pgfpathlineto{\pgfqpoint{2.201541in}{0.924141in}}%
\pgfpathlineto{\pgfqpoint{2.214764in}{0.934617in}}%
\pgfpathlineto{\pgfqpoint{2.229969in}{0.942249in}}%
\pgfpathlineto{\pgfqpoint{2.245555in}{0.947450in}}%
\pgfpathlineto{\pgfqpoint{2.267239in}{0.951729in}}%
\pgfpathlineto{\pgfqpoint{2.285122in}{0.953006in}}%
\pgfpathlineto{\pgfqpoint{2.302044in}{0.952052in}}%
\pgfpathlineto{\pgfqpoint{2.313350in}{0.949240in}}%
\pgfpathlineto{\pgfqpoint{2.317133in}{0.946533in}}%
\pgfpathlineto{\pgfqpoint{2.317006in}{0.944687in}}%
\pgfpathlineto{\pgfqpoint{2.314493in}{0.943453in}}%
\pgfpathlineto{\pgfqpoint{2.311907in}{0.943643in}}%
\pgfpathlineto{\pgfqpoint{2.312135in}{0.943734in}}%
\pgfusepath{stroke}%
\end{pgfscope}%
\begin{pgfscope}%
\pgfpathrectangle{\pgfqpoint{0.562500in}{0.275000in}}{\pgfqpoint{3.487500in}{1.925000in}}%
\pgfusepath{clip}%
\pgfsetrectcap%
\pgfsetroundjoin%
\pgfsetlinewidth{1.505625pt}%
\definecolor{currentstroke}{rgb}{0.839216,0.152941,0.156863}%
\pgfsetstrokecolor{currentstroke}%
\pgfsetdash{}{0pt}%
\pgfpathmoveto{\pgfqpoint{3.891477in}{2.106201in}}%
\pgfpathlineto{\pgfqpoint{2.663098in}{2.112500in}}%
\pgfpathlineto{\pgfqpoint{1.850286in}{1.993740in}}%
\pgfpathlineto{\pgfqpoint{1.380991in}{1.842951in}}%
\pgfpathlineto{\pgfqpoint{1.151678in}{1.702442in}}%
\pgfpathlineto{\pgfqpoint{1.075929in}{1.586669in}}%
\pgfpathlineto{\pgfqpoint{1.090800in}{1.496720in}}%
\pgfpathlineto{\pgfqpoint{1.158731in}{1.428976in}}%
\pgfpathlineto{\pgfqpoint{1.255509in}{1.378783in}}%
\pgfpathlineto{\pgfqpoint{1.365520in}{1.340665in}}%
\pgfpathlineto{\pgfqpoint{1.480280in}{1.310630in}}%
\pgfpathlineto{\pgfqpoint{1.595185in}{1.286174in}}%
\pgfpathlineto{\pgfqpoint{2.000430in}{1.209557in}}%
\pgfpathlineto{\pgfqpoint{2.081617in}{1.191568in}}%
\pgfpathlineto{\pgfqpoint{2.154012in}{1.173503in}}%
\pgfpathlineto{\pgfqpoint{2.217599in}{1.155346in}}%
\pgfpathlineto{\pgfqpoint{2.272546in}{1.137189in}}%
\pgfpathlineto{\pgfqpoint{2.319199in}{1.119226in}}%
\pgfpathlineto{\pgfqpoint{2.358011in}{1.101632in}}%
\pgfpathlineto{\pgfqpoint{2.389605in}{1.084553in}}%
\pgfpathlineto{\pgfqpoint{2.414624in}{1.068132in}}%
\pgfpathlineto{\pgfqpoint{2.433688in}{1.052499in}}%
\pgfpathlineto{\pgfqpoint{2.447396in}{1.037771in}}%
\pgfpathlineto{\pgfqpoint{2.456342in}{1.024049in}}%
\pgfpathlineto{\pgfqpoint{2.461205in}{1.011383in}}%
\pgfpathlineto{\pgfqpoint{2.462628in}{0.999790in}}%
\pgfpathlineto{\pgfqpoint{2.461186in}{0.989275in}}%
\pgfpathlineto{\pgfqpoint{2.457395in}{0.979832in}}%
\pgfpathlineto{\pgfqpoint{2.451707in}{0.971445in}}%
\pgfpathlineto{\pgfqpoint{2.444537in}{0.964084in}}%
\pgfpathlineto{\pgfqpoint{2.436293in}{0.957689in}}%
\pgfpathlineto{\pgfqpoint{2.417896in}{0.947532in}}%
\pgfpathlineto{\pgfqpoint{2.398673in}{0.940456in}}%
\pgfpathlineto{\pgfqpoint{2.380126in}{0.935938in}}%
\pgfpathlineto{\pgfqpoint{2.355602in}{0.932739in}}%
\pgfpathlineto{\pgfqpoint{2.336355in}{0.932453in}}%
\pgfpathlineto{\pgfqpoint{2.319069in}{0.934443in}}%
\pgfpathlineto{\pgfqpoint{2.308478in}{0.938049in}}%
\pgfpathlineto{\pgfqpoint{2.305696in}{0.941102in}}%
\pgfpathlineto{\pgfqpoint{2.306629in}{0.943023in}}%
\pgfpathlineto{\pgfqpoint{2.310130in}{0.944200in}}%
\pgfpathlineto{\pgfqpoint{2.312414in}{0.943835in}}%
\pgfpathlineto{\pgfqpoint{2.312134in}{0.943734in}}%
\pgfusepath{stroke}%
\end{pgfscope}%
\begin{pgfscope}%
\pgfpathrectangle{\pgfqpoint{0.562500in}{0.275000in}}{\pgfqpoint{3.487500in}{1.925000in}}%
\pgfusepath{clip}%
\pgfsetrectcap%
\pgfsetroundjoin%
\pgfsetlinewidth{1.505625pt}%
\definecolor{currentstroke}{rgb}{0.580392,0.403922,0.741176}%
\pgfsetstrokecolor{currentstroke}%
\pgfsetdash{}{0pt}%
\pgfpathmoveto{\pgfqpoint{1.917299in}{2.106201in}}%
\pgfpathlineto{\pgfqpoint{1.209532in}{1.922923in}}%
\pgfpathlineto{\pgfqpoint{0.854164in}{1.738293in}}%
\pgfpathlineto{\pgfqpoint{0.721672in}{1.583888in}}%
\pgfpathlineto{\pgfqpoint{0.721023in}{1.466126in}}%
\pgfpathlineto{\pgfqpoint{0.791563in}{1.380657in}}%
\pgfpathlineto{\pgfqpoint{0.899796in}{1.321178in}}%
\pgfpathlineto{\pgfqpoint{1.024616in}{1.280808in}}%
\pgfpathlineto{\pgfqpoint{1.154807in}{1.252698in}}%
\pgfpathlineto{\pgfqpoint{1.284797in}{1.233036in}}%
\pgfpathlineto{\pgfqpoint{1.411661in}{1.218766in}}%
\pgfpathlineto{\pgfqpoint{1.851782in}{1.177168in}}%
\pgfpathlineto{\pgfqpoint{1.941478in}{1.166226in}}%
\pgfpathlineto{\pgfqpoint{2.022620in}{1.154465in}}%
\pgfpathlineto{\pgfqpoint{2.095203in}{1.141886in}}%
\pgfpathlineto{\pgfqpoint{2.159358in}{1.128644in}}%
\pgfpathlineto{\pgfqpoint{2.215310in}{1.114929in}}%
\pgfpathlineto{\pgfqpoint{2.263488in}{1.100930in}}%
\pgfpathlineto{\pgfqpoint{2.304373in}{1.086849in}}%
\pgfpathlineto{\pgfqpoint{2.338451in}{1.072875in}}%
\pgfpathlineto{\pgfqpoint{2.366209in}{1.059195in}}%
\pgfpathlineto{\pgfqpoint{2.388155in}{1.045979in}}%
\pgfpathlineto{\pgfqpoint{2.404925in}{1.033357in}}%
\pgfpathlineto{\pgfqpoint{2.417151in}{1.021425in}}%
\pgfpathlineto{\pgfqpoint{2.425411in}{1.010258in}}%
\pgfpathlineto{\pgfqpoint{2.430236in}{0.999913in}}%
\pgfpathlineto{\pgfqpoint{2.432106in}{0.990432in}}%
\pgfpathlineto{\pgfqpoint{2.431485in}{0.981835in}}%
\pgfpathlineto{\pgfqpoint{2.428846in}{0.974111in}}%
\pgfpathlineto{\pgfqpoint{2.424608in}{0.967229in}}%
\pgfpathlineto{\pgfqpoint{2.419135in}{0.961161in}}%
\pgfpathlineto{\pgfqpoint{2.405692in}{0.951317in}}%
\pgfpathlineto{\pgfqpoint{2.390544in}{0.944233in}}%
\pgfpathlineto{\pgfqpoint{2.375260in}{0.939444in}}%
\pgfpathlineto{\pgfqpoint{2.354250in}{0.935615in}}%
\pgfpathlineto{\pgfqpoint{2.337098in}{0.934595in}}%
\pgfpathlineto{\pgfqpoint{2.321064in}{0.935697in}}%
\pgfpathlineto{\pgfqpoint{2.310525in}{0.938502in}}%
\pgfpathlineto{\pgfqpoint{2.306998in}{0.941550in}}%
\pgfpathlineto{\pgfqpoint{2.307971in}{0.943331in}}%
\pgfpathlineto{\pgfqpoint{2.311545in}{0.944107in}}%
\pgfpathlineto{\pgfqpoint{2.312358in}{0.943818in}}%
\pgfpathlineto{\pgfqpoint{2.312134in}{0.943734in}}%
\pgfusepath{stroke}%
\end{pgfscope}%
\begin{pgfscope}%
\pgfpathrectangle{\pgfqpoint{0.562500in}{0.275000in}}{\pgfqpoint{3.487500in}{1.925000in}}%
\pgfusepath{clip}%
\pgfsetrectcap%
\pgfsetroundjoin%
\pgfsetlinewidth{1.505625pt}%
\definecolor{currentstroke}{rgb}{0.549020,0.337255,0.294118}%
\pgfsetstrokecolor{currentstroke}%
\pgfsetdash{}{0pt}%
\pgfpathmoveto{\pgfqpoint{1.127627in}{1.524967in}}%
\pgfpathlineto{\pgfqpoint{1.181776in}{1.452700in}}%
\pgfpathlineto{\pgfqpoint{1.268559in}{1.397830in}}%
\pgfpathlineto{\pgfqpoint{1.373825in}{1.356265in}}%
\pgfpathlineto{\pgfqpoint{1.487694in}{1.324410in}}%
\pgfpathlineto{\pgfqpoint{1.602599in}{1.298848in}}%
\pgfpathlineto{\pgfqpoint{2.005993in}{1.216938in}}%
\pgfpathlineto{\pgfqpoint{2.087403in}{1.197993in}}%
\pgfpathlineto{\pgfqpoint{2.159973in}{1.179057in}}%
\pgfpathlineto{\pgfqpoint{2.223655in}{1.160155in}}%
\pgfpathlineto{\pgfqpoint{2.278701in}{1.141371in}}%
\pgfpathlineto{\pgfqpoint{2.325501in}{1.122830in}}%
\pgfpathlineto{\pgfqpoint{2.364490in}{1.104677in}}%
\pgfpathlineto{\pgfqpoint{2.396154in}{1.087074in}}%
\pgfpathlineto{\pgfqpoint{2.421027in}{1.070200in}}%
\pgfpathlineto{\pgfqpoint{2.439775in}{1.054192in}}%
\pgfpathlineto{\pgfqpoint{2.453143in}{1.039134in}}%
\pgfpathlineto{\pgfqpoint{2.461819in}{1.025094in}}%
\pgfpathlineto{\pgfqpoint{2.466436in}{1.012127in}}%
\pgfpathlineto{\pgfqpoint{2.467571in}{1.000272in}}%
\pgfpathlineto{\pgfqpoint{2.465749in}{0.989553in}}%
\pgfpathlineto{\pgfqpoint{2.461524in}{0.979961in}}%
\pgfpathlineto{\pgfqpoint{2.455414in}{0.971448in}}%
\pgfpathlineto{\pgfqpoint{2.447868in}{0.963962in}}%
\pgfpathlineto{\pgfqpoint{2.439276in}{0.957452in}}%
\pgfpathlineto{\pgfqpoint{2.420219in}{0.947144in}}%
\pgfpathlineto{\pgfqpoint{2.400362in}{0.940025in}}%
\pgfpathlineto{\pgfqpoint{2.381280in}{0.935483in}}%
\pgfpathlineto{\pgfqpoint{2.356159in}{0.932328in}}%
\pgfpathlineto{\pgfqpoint{2.336493in}{0.932116in}}%
\pgfpathlineto{\pgfqpoint{2.318937in}{0.934225in}}%
\pgfpathlineto{\pgfqpoint{2.309571in}{0.937211in}}%
\pgfpathlineto{\pgfqpoint{2.305611in}{0.940539in}}%
\pgfpathlineto{\pgfqpoint{2.306139in}{0.942733in}}%
\pgfpathlineto{\pgfqpoint{2.309383in}{0.944133in}}%
\pgfpathlineto{\pgfqpoint{2.312422in}{0.943857in}}%
\pgfpathlineto{\pgfqpoint{2.312133in}{0.943734in}}%
\pgfusepath{stroke}%
\end{pgfscope}%
\begin{pgfscope}%
\pgfpathrectangle{\pgfqpoint{0.562500in}{0.275000in}}{\pgfqpoint{3.487500in}{1.925000in}}%
\pgfusepath{clip}%
\pgfsetrectcap%
\pgfsetroundjoin%
\pgfsetlinewidth{1.505625pt}%
\definecolor{currentstroke}{rgb}{0.890196,0.466667,0.760784}%
\pgfsetstrokecolor{currentstroke}%
\pgfsetdash{}{0pt}%
\pgfpathmoveto{\pgfqpoint{0.732791in}{0.362500in}}%
\pgfpathlineto{\pgfqpoint{1.004162in}{0.376835in}}%
\pgfpathlineto{\pgfqpoint{1.198512in}{0.410168in}}%
\pgfpathlineto{\pgfqpoint{1.332209in}{0.453296in}}%
\pgfpathlineto{\pgfqpoint{1.421844in}{0.500325in}}%
\pgfpathlineto{\pgfqpoint{1.485009in}{0.548055in}}%
\pgfpathlineto{\pgfqpoint{1.532052in}{0.595039in}}%
\pgfpathlineto{\pgfqpoint{1.568210in}{0.640372in}}%
\pgfpathlineto{\pgfqpoint{1.627436in}{0.723830in}}%
\pgfpathlineto{\pgfqpoint{1.656360in}{0.761226in}}%
\pgfpathlineto{\pgfqpoint{1.686710in}{0.795538in}}%
\pgfpathlineto{\pgfqpoint{1.718944in}{0.826714in}}%
\pgfpathlineto{\pgfqpoint{1.753214in}{0.854740in}}%
\pgfpathlineto{\pgfqpoint{1.789401in}{0.879640in}}%
\pgfpathlineto{\pgfqpoint{1.827117in}{0.901477in}}%
\pgfpathlineto{\pgfqpoint{1.865719in}{0.920349in}}%
\pgfpathlineto{\pgfqpoint{1.904645in}{0.936413in}}%
\pgfpathlineto{\pgfqpoint{1.943391in}{0.949874in}}%
\pgfpathlineto{\pgfqpoint{1.981463in}{0.960938in}}%
\pgfpathlineto{\pgfqpoint{2.018416in}{0.969807in}}%
\pgfpathlineto{\pgfqpoint{2.053849in}{0.976677in}}%
\pgfpathlineto{\pgfqpoint{2.087423in}{0.981755in}}%
\pgfpathlineto{\pgfqpoint{2.148190in}{0.987448in}}%
\pgfpathlineto{\pgfqpoint{2.199655in}{0.988525in}}%
\pgfpathlineto{\pgfqpoint{2.241470in}{0.986354in}}%
\pgfpathlineto{\pgfqpoint{2.274155in}{0.982136in}}%
\pgfpathlineto{\pgfqpoint{2.298637in}{0.976822in}}%
\pgfpathlineto{\pgfqpoint{2.315907in}{0.971091in}}%
\pgfpathlineto{\pgfqpoint{2.327235in}{0.965456in}}%
\pgfpathlineto{\pgfqpoint{2.333860in}{0.960256in}}%
\pgfpathlineto{\pgfqpoint{2.336862in}{0.955690in}}%
\pgfpathlineto{\pgfqpoint{2.337258in}{0.951844in}}%
\pgfpathlineto{\pgfqpoint{2.334738in}{0.947447in}}%
\pgfpathlineto{\pgfqpoint{2.328719in}{0.943852in}}%
\pgfpathlineto{\pgfqpoint{2.319739in}{0.941894in}}%
\pgfpathlineto{\pgfqpoint{2.312096in}{0.942502in}}%
\pgfpathlineto{\pgfqpoint{2.311058in}{0.943428in}}%
\pgfpathlineto{\pgfqpoint{2.311784in}{0.943812in}}%
\pgfpathlineto{\pgfqpoint{2.312182in}{0.943752in}}%
\pgfpathlineto{\pgfqpoint{2.312134in}{0.943734in}}%
\pgfusepath{stroke}%
\end{pgfscope}%
\begin{pgfscope}%
\pgfsetrectcap%
\pgfsetmiterjoin%
\pgfsetlinewidth{0.803000pt}%
\definecolor{currentstroke}{rgb}{0.000000,0.000000,0.000000}%
\pgfsetstrokecolor{currentstroke}%
\pgfsetdash{}{0pt}%
\pgfpathmoveto{\pgfqpoint{0.562500in}{0.275000in}}%
\pgfpathlineto{\pgfqpoint{0.562500in}{2.200000in}}%
\pgfusepath{stroke}%
\end{pgfscope}%
\begin{pgfscope}%
\pgfsetrectcap%
\pgfsetmiterjoin%
\pgfsetlinewidth{0.803000pt}%
\definecolor{currentstroke}{rgb}{0.000000,0.000000,0.000000}%
\pgfsetstrokecolor{currentstroke}%
\pgfsetdash{}{0pt}%
\pgfpathmoveto{\pgfqpoint{4.050000in}{0.275000in}}%
\pgfpathlineto{\pgfqpoint{4.050000in}{2.200000in}}%
\pgfusepath{stroke}%
\end{pgfscope}%
\begin{pgfscope}%
\pgfsetrectcap%
\pgfsetmiterjoin%
\pgfsetlinewidth{0.803000pt}%
\definecolor{currentstroke}{rgb}{0.000000,0.000000,0.000000}%
\pgfsetstrokecolor{currentstroke}%
\pgfsetdash{}{0pt}%
\pgfpathmoveto{\pgfqpoint{0.562500in}{0.275000in}}%
\pgfpathlineto{\pgfqpoint{4.050000in}{0.275000in}}%
\pgfusepath{stroke}%
\end{pgfscope}%
\begin{pgfscope}%
\pgfsetrectcap%
\pgfsetmiterjoin%
\pgfsetlinewidth{0.803000pt}%
\definecolor{currentstroke}{rgb}{0.000000,0.000000,0.000000}%
\pgfsetstrokecolor{currentstroke}%
\pgfsetdash{}{0pt}%
\pgfpathmoveto{\pgfqpoint{0.562500in}{2.200000in}}%
\pgfpathlineto{\pgfqpoint{4.050000in}{2.200000in}}%
\pgfusepath{stroke}%
\end{pgfscope}%
\end{pgfpicture}%
\makeatother%
\endgroup%

    \caption{Lösungen des Differentialgleichungssystems aus Beispiel \ref{poinbendix:beispiel:fall1}.
    Diverse Anfangspunkte haben die selbe Omega-Limesmenge auf der Nullstelle $(0,0)$.}
    \label{poinbendix:fig:fixed_point_omega_set}
\end{figure}

\subsection{Fall 2: $\omega(p)$ ist ein Geschlossener Orbit} \label{poinbendix:subsection:fall2}

Den zweiten Fall haben wir im Abschnitt zu den Nullklinen \ref{poinbendix:subsection:nullklinen} bereits angedeutet.
Bahnkurven werden dabei eingefangen und da sie sich nicht kreuzen können, bleiben sie auf alle Zeiten periodisch.

\begin{beispiel}
Das Gleichungssystem
\begin{align*}
    \dot{x} &= y \\
    \dot{y} &= x
\end{align*}
erzeugt dabei kreisrunde Orbitale um den Nullpunkt herum.
Der Nullpunkt stellt also eine weitere Omega-Limesmenge dar, und zwar vom Punkt $p = (0,0)$.
Die ganze Situation ist in Abbildung \ref{poinbendix:fig:fall_2} zu sehen.
\end{beispiel}

Es gibt aber auch die Möglichkeit, dass verschiedene Bahnkurven auf den selben Orbit konvergieren, wie im folgenden Beispiel gezeigt wird.

\begin{beispiel}
Das Gleichungssystem
% \begin{align*}
%     \dot{x} &= y \\ 
%     \dot{y} &= x
% \end{align*} %TODO Replace by correct formula
beschreibt ist eine vereinfachte Variante des Recharge Osziallator Models.
Dieser Oszillator konvergiert von verschiedensten Startpunkten auf den gleichen Orbit, wie in Abbildung \ref{poinbendix:fig:recharge_oszillator_fall_2} gezeigt.

\begin{figure}
\centering
    \input{papers/poinbendix/images/recharge_oszillator_fall_2.pgf}
    \caption{Lösungen des Differentialgleichungssystems %\ref{TODO}
    Anfangspunkte in unterschiedlicher Distanz zum Nullpunkt $(0,0)$ haben den selben periodischen Orbit als Omega-Limesmenge.}
\label{poinbendix:fig:fall_2}
\end{figure}

\begin{figure}
\centering
    %% Creator: Matplotlib, PGF backend
%%
%% To include the figure in your LaTeX document, write
%%   \input{<filename>.pgf}
%%
%% Make sure the required packages are loaded in your preamble
%%   \usepackage{pgf}
%%
%% Also ensure that all the required font packages are loaded; for instance,
%% the lmodern package is sometimes necessary when using math font.
%%   \usepackage{lmodern}
%%
%% Figures using additional raster images can only be included by \input if
%% they are in the same directory as the main LaTeX file. For loading figures
%% from other directories you can use the `import` package
%%   \usepackage{import}
%%
%% and then include the figures with
%%   \import{<path to file>}{<filename>.pgf}
%%
%% Matplotlib used the following preamble
%%   \usepackage{bm}
%%   \usepackage{amsmath}
%%   \usepackage{xcolor}
%%   \usepackage{tgtermes}
%%   \makeatletter\@ifpackageloaded{underscore}{}{\usepackage[strings]{underscore}}\makeatother
%%
\begingroup%
\makeatletter%
\begin{pgfpicture}%
\pgfpathrectangle{\pgfpointorigin}{\pgfqpoint{4.000000in}{4.000000in}}%
\pgfusepath{use as bounding box, clip}%
\begin{pgfscope}%
\pgfsetbuttcap%
\pgfsetmiterjoin%
\definecolor{currentfill}{rgb}{1.000000,1.000000,1.000000}%
\pgfsetfillcolor{currentfill}%
\pgfsetlinewidth{0.000000pt}%
\definecolor{currentstroke}{rgb}{1.000000,1.000000,1.000000}%
\pgfsetstrokecolor{currentstroke}%
\pgfsetdash{}{0pt}%
\pgfpathmoveto{\pgfqpoint{0.000000in}{0.000000in}}%
\pgfpathlineto{\pgfqpoint{4.000000in}{0.000000in}}%
\pgfpathlineto{\pgfqpoint{4.000000in}{4.000000in}}%
\pgfpathlineto{\pgfqpoint{0.000000in}{4.000000in}}%
\pgfpathlineto{\pgfqpoint{0.000000in}{0.000000in}}%
\pgfpathclose%
\pgfusepath{fill}%
\end{pgfscope}%
\begin{pgfscope}%
\pgfsetbuttcap%
\pgfsetmiterjoin%
\definecolor{currentfill}{rgb}{1.000000,1.000000,1.000000}%
\pgfsetfillcolor{currentfill}%
\pgfsetlinewidth{0.000000pt}%
\definecolor{currentstroke}{rgb}{0.000000,0.000000,0.000000}%
\pgfsetstrokecolor{currentstroke}%
\pgfsetstrokeopacity{0.000000}%
\pgfsetdash{}{0pt}%
\pgfpathmoveto{\pgfqpoint{0.500000in}{0.440000in}}%
\pgfpathlineto{\pgfqpoint{3.600000in}{0.440000in}}%
\pgfpathlineto{\pgfqpoint{3.600000in}{3.520000in}}%
\pgfpathlineto{\pgfqpoint{0.500000in}{3.520000in}}%
\pgfpathlineto{\pgfqpoint{0.500000in}{0.440000in}}%
\pgfpathclose%
\pgfusepath{fill}%
\end{pgfscope}%
\begin{pgfscope}%
\pgfpathrectangle{\pgfqpoint{0.500000in}{0.440000in}}{\pgfqpoint{3.100000in}{3.080000in}}%
\pgfusepath{clip}%
\pgfsetbuttcap%
\pgfsetroundjoin%
\pgfsetlinewidth{0.250937pt}%
\definecolor{currentstroke}{rgb}{0.501961,0.501961,0.501961}%
\pgfsetstrokecolor{currentstroke}%
\pgfsetdash{{0.250000pt}{0.412500pt}}{0.000000pt}%
\pgfpathmoveto{\pgfqpoint{0.814940in}{0.440000in}}%
\pgfpathlineto{\pgfqpoint{0.814940in}{3.520000in}}%
\pgfusepath{stroke}%
\end{pgfscope}%
\begin{pgfscope}%
\pgfsetbuttcap%
\pgfsetroundjoin%
\definecolor{currentfill}{rgb}{0.000000,0.000000,0.000000}%
\pgfsetfillcolor{currentfill}%
\pgfsetlinewidth{0.501875pt}%
\definecolor{currentstroke}{rgb}{0.000000,0.000000,0.000000}%
\pgfsetstrokecolor{currentstroke}%
\pgfsetdash{}{0pt}%
\pgfsys@defobject{currentmarker}{\pgfqpoint{0.000000in}{-0.041667in}}{\pgfqpoint{0.000000in}{0.000000in}}{%
\pgfpathmoveto{\pgfqpoint{0.000000in}{0.000000in}}%
\pgfpathlineto{\pgfqpoint{0.000000in}{-0.041667in}}%
\pgfusepath{stroke,fill}%
}%
\begin{pgfscope}%
\pgfsys@transformshift{0.814940in}{0.440000in}%
\pgfsys@useobject{currentmarker}{}%
\end{pgfscope}%
\end{pgfscope}%
\begin{pgfscope}%
\definecolor{textcolor}{rgb}{0.000000,0.000000,0.000000}%
\pgfsetstrokecolor{textcolor}%
\pgfsetfillcolor{textcolor}%
\pgftext[x=0.814940in,y=0.349722in,,top]{\color{textcolor}\rmfamily\fontsize{10.000000}{12.000000}\selectfont \(\displaystyle {-2}\)}%
\end{pgfscope}%
\begin{pgfscope}%
\pgfpathrectangle{\pgfqpoint{0.500000in}{0.440000in}}{\pgfqpoint{3.100000in}{3.080000in}}%
\pgfusepath{clip}%
\pgfsetbuttcap%
\pgfsetroundjoin%
\pgfsetlinewidth{0.250937pt}%
\definecolor{currentstroke}{rgb}{0.501961,0.501961,0.501961}%
\pgfsetstrokecolor{currentstroke}%
\pgfsetdash{{0.250000pt}{0.412500pt}}{0.000000pt}%
\pgfpathmoveto{\pgfqpoint{1.432479in}{0.440000in}}%
\pgfpathlineto{\pgfqpoint{1.432479in}{3.520000in}}%
\pgfusepath{stroke}%
\end{pgfscope}%
\begin{pgfscope}%
\pgfsetbuttcap%
\pgfsetroundjoin%
\definecolor{currentfill}{rgb}{0.000000,0.000000,0.000000}%
\pgfsetfillcolor{currentfill}%
\pgfsetlinewidth{0.501875pt}%
\definecolor{currentstroke}{rgb}{0.000000,0.000000,0.000000}%
\pgfsetstrokecolor{currentstroke}%
\pgfsetdash{}{0pt}%
\pgfsys@defobject{currentmarker}{\pgfqpoint{0.000000in}{-0.041667in}}{\pgfqpoint{0.000000in}{0.000000in}}{%
\pgfpathmoveto{\pgfqpoint{0.000000in}{0.000000in}}%
\pgfpathlineto{\pgfqpoint{0.000000in}{-0.041667in}}%
\pgfusepath{stroke,fill}%
}%
\begin{pgfscope}%
\pgfsys@transformshift{1.432479in}{0.440000in}%
\pgfsys@useobject{currentmarker}{}%
\end{pgfscope}%
\end{pgfscope}%
\begin{pgfscope}%
\definecolor{textcolor}{rgb}{0.000000,0.000000,0.000000}%
\pgfsetstrokecolor{textcolor}%
\pgfsetfillcolor{textcolor}%
\pgftext[x=1.432479in,y=0.349722in,,top]{\color{textcolor}\rmfamily\fontsize{10.000000}{12.000000}\selectfont \(\displaystyle {-1}\)}%
\end{pgfscope}%
\begin{pgfscope}%
\pgfpathrectangle{\pgfqpoint{0.500000in}{0.440000in}}{\pgfqpoint{3.100000in}{3.080000in}}%
\pgfusepath{clip}%
\pgfsetbuttcap%
\pgfsetroundjoin%
\pgfsetlinewidth{0.250937pt}%
\definecolor{currentstroke}{rgb}{0.501961,0.501961,0.501961}%
\pgfsetstrokecolor{currentstroke}%
\pgfsetdash{{0.250000pt}{0.412500pt}}{0.000000pt}%
\pgfpathmoveto{\pgfqpoint{2.050019in}{0.440000in}}%
\pgfpathlineto{\pgfqpoint{2.050019in}{3.520000in}}%
\pgfusepath{stroke}%
\end{pgfscope}%
\begin{pgfscope}%
\pgfsetbuttcap%
\pgfsetroundjoin%
\definecolor{currentfill}{rgb}{0.000000,0.000000,0.000000}%
\pgfsetfillcolor{currentfill}%
\pgfsetlinewidth{0.501875pt}%
\definecolor{currentstroke}{rgb}{0.000000,0.000000,0.000000}%
\pgfsetstrokecolor{currentstroke}%
\pgfsetdash{}{0pt}%
\pgfsys@defobject{currentmarker}{\pgfqpoint{0.000000in}{-0.041667in}}{\pgfqpoint{0.000000in}{0.000000in}}{%
\pgfpathmoveto{\pgfqpoint{0.000000in}{0.000000in}}%
\pgfpathlineto{\pgfqpoint{0.000000in}{-0.041667in}}%
\pgfusepath{stroke,fill}%
}%
\begin{pgfscope}%
\pgfsys@transformshift{2.050019in}{0.440000in}%
\pgfsys@useobject{currentmarker}{}%
\end{pgfscope}%
\end{pgfscope}%
\begin{pgfscope}%
\definecolor{textcolor}{rgb}{0.000000,0.000000,0.000000}%
\pgfsetstrokecolor{textcolor}%
\pgfsetfillcolor{textcolor}%
\pgftext[x=2.050019in,y=0.349722in,,top]{\color{textcolor}\rmfamily\fontsize{10.000000}{12.000000}\selectfont \(\displaystyle {0}\)}%
\end{pgfscope}%
\begin{pgfscope}%
\pgfpathrectangle{\pgfqpoint{0.500000in}{0.440000in}}{\pgfqpoint{3.100000in}{3.080000in}}%
\pgfusepath{clip}%
\pgfsetbuttcap%
\pgfsetroundjoin%
\pgfsetlinewidth{0.250937pt}%
\definecolor{currentstroke}{rgb}{0.501961,0.501961,0.501961}%
\pgfsetstrokecolor{currentstroke}%
\pgfsetdash{{0.250000pt}{0.412500pt}}{0.000000pt}%
\pgfpathmoveto{\pgfqpoint{2.667558in}{0.440000in}}%
\pgfpathlineto{\pgfqpoint{2.667558in}{3.520000in}}%
\pgfusepath{stroke}%
\end{pgfscope}%
\begin{pgfscope}%
\pgfsetbuttcap%
\pgfsetroundjoin%
\definecolor{currentfill}{rgb}{0.000000,0.000000,0.000000}%
\pgfsetfillcolor{currentfill}%
\pgfsetlinewidth{0.501875pt}%
\definecolor{currentstroke}{rgb}{0.000000,0.000000,0.000000}%
\pgfsetstrokecolor{currentstroke}%
\pgfsetdash{}{0pt}%
\pgfsys@defobject{currentmarker}{\pgfqpoint{0.000000in}{-0.041667in}}{\pgfqpoint{0.000000in}{0.000000in}}{%
\pgfpathmoveto{\pgfqpoint{0.000000in}{0.000000in}}%
\pgfpathlineto{\pgfqpoint{0.000000in}{-0.041667in}}%
\pgfusepath{stroke,fill}%
}%
\begin{pgfscope}%
\pgfsys@transformshift{2.667558in}{0.440000in}%
\pgfsys@useobject{currentmarker}{}%
\end{pgfscope}%
\end{pgfscope}%
\begin{pgfscope}%
\definecolor{textcolor}{rgb}{0.000000,0.000000,0.000000}%
\pgfsetstrokecolor{textcolor}%
\pgfsetfillcolor{textcolor}%
\pgftext[x=2.667558in,y=0.349722in,,top]{\color{textcolor}\rmfamily\fontsize{10.000000}{12.000000}\selectfont \(\displaystyle {1}\)}%
\end{pgfscope}%
\begin{pgfscope}%
\pgfpathrectangle{\pgfqpoint{0.500000in}{0.440000in}}{\pgfqpoint{3.100000in}{3.080000in}}%
\pgfusepath{clip}%
\pgfsetbuttcap%
\pgfsetroundjoin%
\pgfsetlinewidth{0.250937pt}%
\definecolor{currentstroke}{rgb}{0.501961,0.501961,0.501961}%
\pgfsetstrokecolor{currentstroke}%
\pgfsetdash{{0.250000pt}{0.412500pt}}{0.000000pt}%
\pgfpathmoveto{\pgfqpoint{3.285097in}{0.440000in}}%
\pgfpathlineto{\pgfqpoint{3.285097in}{3.520000in}}%
\pgfusepath{stroke}%
\end{pgfscope}%
\begin{pgfscope}%
\pgfsetbuttcap%
\pgfsetroundjoin%
\definecolor{currentfill}{rgb}{0.000000,0.000000,0.000000}%
\pgfsetfillcolor{currentfill}%
\pgfsetlinewidth{0.501875pt}%
\definecolor{currentstroke}{rgb}{0.000000,0.000000,0.000000}%
\pgfsetstrokecolor{currentstroke}%
\pgfsetdash{}{0pt}%
\pgfsys@defobject{currentmarker}{\pgfqpoint{0.000000in}{-0.041667in}}{\pgfqpoint{0.000000in}{0.000000in}}{%
\pgfpathmoveto{\pgfqpoint{0.000000in}{0.000000in}}%
\pgfpathlineto{\pgfqpoint{0.000000in}{-0.041667in}}%
\pgfusepath{stroke,fill}%
}%
\begin{pgfscope}%
\pgfsys@transformshift{3.285097in}{0.440000in}%
\pgfsys@useobject{currentmarker}{}%
\end{pgfscope}%
\end{pgfscope}%
\begin{pgfscope}%
\definecolor{textcolor}{rgb}{0.000000,0.000000,0.000000}%
\pgfsetstrokecolor{textcolor}%
\pgfsetfillcolor{textcolor}%
\pgftext[x=3.285097in,y=0.349722in,,top]{\color{textcolor}\rmfamily\fontsize{10.000000}{12.000000}\selectfont \(\displaystyle {2}\)}%
\end{pgfscope}%
\begin{pgfscope}%
\pgfpathrectangle{\pgfqpoint{0.500000in}{0.440000in}}{\pgfqpoint{3.100000in}{3.080000in}}%
\pgfusepath{clip}%
\pgfsetbuttcap%
\pgfsetroundjoin%
\pgfsetlinewidth{0.250937pt}%
\definecolor{currentstroke}{rgb}{0.501961,0.501961,0.501961}%
\pgfsetstrokecolor{currentstroke}%
\pgfsetdash{{0.250000pt}{0.412500pt}}{0.000000pt}%
\pgfpathmoveto{\pgfqpoint{0.500000in}{0.752655in}}%
\pgfpathlineto{\pgfqpoint{3.600000in}{0.752655in}}%
\pgfusepath{stroke}%
\end{pgfscope}%
\begin{pgfscope}%
\pgfsetbuttcap%
\pgfsetroundjoin%
\definecolor{currentfill}{rgb}{0.000000,0.000000,0.000000}%
\pgfsetfillcolor{currentfill}%
\pgfsetlinewidth{0.501875pt}%
\definecolor{currentstroke}{rgb}{0.000000,0.000000,0.000000}%
\pgfsetstrokecolor{currentstroke}%
\pgfsetdash{}{0pt}%
\pgfsys@defobject{currentmarker}{\pgfqpoint{-0.041667in}{0.000000in}}{\pgfqpoint{-0.000000in}{0.000000in}}{%
\pgfpathmoveto{\pgfqpoint{-0.000000in}{0.000000in}}%
\pgfpathlineto{\pgfqpoint{-0.041667in}{0.000000in}}%
\pgfusepath{stroke,fill}%
}%
\begin{pgfscope}%
\pgfsys@transformshift{0.500000in}{0.752655in}%
\pgfsys@useobject{currentmarker}{}%
\end{pgfscope}%
\end{pgfscope}%
\begin{pgfscope}%
\definecolor{textcolor}{rgb}{0.000000,0.000000,0.000000}%
\pgfsetstrokecolor{textcolor}%
\pgfsetfillcolor{textcolor}%
\pgftext[x=0.232253in, y=0.705954in, left, base]{\color{textcolor}\rmfamily\fontsize{10.000000}{12.000000}\selectfont \(\displaystyle {-2}\)}%
\end{pgfscope}%
\begin{pgfscope}%
\pgfpathrectangle{\pgfqpoint{0.500000in}{0.440000in}}{\pgfqpoint{3.100000in}{3.080000in}}%
\pgfusepath{clip}%
\pgfsetbuttcap%
\pgfsetroundjoin%
\pgfsetlinewidth{0.250937pt}%
\definecolor{currentstroke}{rgb}{0.501961,0.501961,0.501961}%
\pgfsetstrokecolor{currentstroke}%
\pgfsetdash{{0.250000pt}{0.412500pt}}{0.000000pt}%
\pgfpathmoveto{\pgfqpoint{0.500000in}{1.366237in}}%
\pgfpathlineto{\pgfqpoint{3.600000in}{1.366237in}}%
\pgfusepath{stroke}%
\end{pgfscope}%
\begin{pgfscope}%
\pgfsetbuttcap%
\pgfsetroundjoin%
\definecolor{currentfill}{rgb}{0.000000,0.000000,0.000000}%
\pgfsetfillcolor{currentfill}%
\pgfsetlinewidth{0.501875pt}%
\definecolor{currentstroke}{rgb}{0.000000,0.000000,0.000000}%
\pgfsetstrokecolor{currentstroke}%
\pgfsetdash{}{0pt}%
\pgfsys@defobject{currentmarker}{\pgfqpoint{-0.041667in}{0.000000in}}{\pgfqpoint{-0.000000in}{0.000000in}}{%
\pgfpathmoveto{\pgfqpoint{-0.000000in}{0.000000in}}%
\pgfpathlineto{\pgfqpoint{-0.041667in}{0.000000in}}%
\pgfusepath{stroke,fill}%
}%
\begin{pgfscope}%
\pgfsys@transformshift{0.500000in}{1.366237in}%
\pgfsys@useobject{currentmarker}{}%
\end{pgfscope}%
\end{pgfscope}%
\begin{pgfscope}%
\definecolor{textcolor}{rgb}{0.000000,0.000000,0.000000}%
\pgfsetstrokecolor{textcolor}%
\pgfsetfillcolor{textcolor}%
\pgftext[x=0.232253in, y=1.319535in, left, base]{\color{textcolor}\rmfamily\fontsize{10.000000}{12.000000}\selectfont \(\displaystyle {-1}\)}%
\end{pgfscope}%
\begin{pgfscope}%
\pgfpathrectangle{\pgfqpoint{0.500000in}{0.440000in}}{\pgfqpoint{3.100000in}{3.080000in}}%
\pgfusepath{clip}%
\pgfsetbuttcap%
\pgfsetroundjoin%
\pgfsetlinewidth{0.250937pt}%
\definecolor{currentstroke}{rgb}{0.501961,0.501961,0.501961}%
\pgfsetstrokecolor{currentstroke}%
\pgfsetdash{{0.250000pt}{0.412500pt}}{0.000000pt}%
\pgfpathmoveto{\pgfqpoint{0.500000in}{1.979818in}}%
\pgfpathlineto{\pgfqpoint{3.600000in}{1.979818in}}%
\pgfusepath{stroke}%
\end{pgfscope}%
\begin{pgfscope}%
\pgfsetbuttcap%
\pgfsetroundjoin%
\definecolor{currentfill}{rgb}{0.000000,0.000000,0.000000}%
\pgfsetfillcolor{currentfill}%
\pgfsetlinewidth{0.501875pt}%
\definecolor{currentstroke}{rgb}{0.000000,0.000000,0.000000}%
\pgfsetstrokecolor{currentstroke}%
\pgfsetdash{}{0pt}%
\pgfsys@defobject{currentmarker}{\pgfqpoint{-0.041667in}{0.000000in}}{\pgfqpoint{-0.000000in}{0.000000in}}{%
\pgfpathmoveto{\pgfqpoint{-0.000000in}{0.000000in}}%
\pgfpathlineto{\pgfqpoint{-0.041667in}{0.000000in}}%
\pgfusepath{stroke,fill}%
}%
\begin{pgfscope}%
\pgfsys@transformshift{0.500000in}{1.979818in}%
\pgfsys@useobject{currentmarker}{}%
\end{pgfscope}%
\end{pgfscope}%
\begin{pgfscope}%
\definecolor{textcolor}{rgb}{0.000000,0.000000,0.000000}%
\pgfsetstrokecolor{textcolor}%
\pgfsetfillcolor{textcolor}%
\pgftext[x=0.340277in, y=1.933117in, left, base]{\color{textcolor}\rmfamily\fontsize{10.000000}{12.000000}\selectfont \(\displaystyle {0}\)}%
\end{pgfscope}%
\begin{pgfscope}%
\pgfpathrectangle{\pgfqpoint{0.500000in}{0.440000in}}{\pgfqpoint{3.100000in}{3.080000in}}%
\pgfusepath{clip}%
\pgfsetbuttcap%
\pgfsetroundjoin%
\pgfsetlinewidth{0.250937pt}%
\definecolor{currentstroke}{rgb}{0.501961,0.501961,0.501961}%
\pgfsetstrokecolor{currentstroke}%
\pgfsetdash{{0.250000pt}{0.412500pt}}{0.000000pt}%
\pgfpathmoveto{\pgfqpoint{0.500000in}{2.593400in}}%
\pgfpathlineto{\pgfqpoint{3.600000in}{2.593400in}}%
\pgfusepath{stroke}%
\end{pgfscope}%
\begin{pgfscope}%
\pgfsetbuttcap%
\pgfsetroundjoin%
\definecolor{currentfill}{rgb}{0.000000,0.000000,0.000000}%
\pgfsetfillcolor{currentfill}%
\pgfsetlinewidth{0.501875pt}%
\definecolor{currentstroke}{rgb}{0.000000,0.000000,0.000000}%
\pgfsetstrokecolor{currentstroke}%
\pgfsetdash{}{0pt}%
\pgfsys@defobject{currentmarker}{\pgfqpoint{-0.041667in}{0.000000in}}{\pgfqpoint{-0.000000in}{0.000000in}}{%
\pgfpathmoveto{\pgfqpoint{-0.000000in}{0.000000in}}%
\pgfpathlineto{\pgfqpoint{-0.041667in}{0.000000in}}%
\pgfusepath{stroke,fill}%
}%
\begin{pgfscope}%
\pgfsys@transformshift{0.500000in}{2.593400in}%
\pgfsys@useobject{currentmarker}{}%
\end{pgfscope}%
\end{pgfscope}%
\begin{pgfscope}%
\definecolor{textcolor}{rgb}{0.000000,0.000000,0.000000}%
\pgfsetstrokecolor{textcolor}%
\pgfsetfillcolor{textcolor}%
\pgftext[x=0.340277in, y=2.546698in, left, base]{\color{textcolor}\rmfamily\fontsize{10.000000}{12.000000}\selectfont \(\displaystyle {1}\)}%
\end{pgfscope}%
\begin{pgfscope}%
\pgfpathrectangle{\pgfqpoint{0.500000in}{0.440000in}}{\pgfqpoint{3.100000in}{3.080000in}}%
\pgfusepath{clip}%
\pgfsetbuttcap%
\pgfsetroundjoin%
\pgfsetlinewidth{0.250937pt}%
\definecolor{currentstroke}{rgb}{0.501961,0.501961,0.501961}%
\pgfsetstrokecolor{currentstroke}%
\pgfsetdash{{0.250000pt}{0.412500pt}}{0.000000pt}%
\pgfpathmoveto{\pgfqpoint{0.500000in}{3.206981in}}%
\pgfpathlineto{\pgfqpoint{3.600000in}{3.206981in}}%
\pgfusepath{stroke}%
\end{pgfscope}%
\begin{pgfscope}%
\pgfsetbuttcap%
\pgfsetroundjoin%
\definecolor{currentfill}{rgb}{0.000000,0.000000,0.000000}%
\pgfsetfillcolor{currentfill}%
\pgfsetlinewidth{0.501875pt}%
\definecolor{currentstroke}{rgb}{0.000000,0.000000,0.000000}%
\pgfsetstrokecolor{currentstroke}%
\pgfsetdash{}{0pt}%
\pgfsys@defobject{currentmarker}{\pgfqpoint{-0.041667in}{0.000000in}}{\pgfqpoint{-0.000000in}{0.000000in}}{%
\pgfpathmoveto{\pgfqpoint{-0.000000in}{0.000000in}}%
\pgfpathlineto{\pgfqpoint{-0.041667in}{0.000000in}}%
\pgfusepath{stroke,fill}%
}%
\begin{pgfscope}%
\pgfsys@transformshift{0.500000in}{3.206981in}%
\pgfsys@useobject{currentmarker}{}%
\end{pgfscope}%
\end{pgfscope}%
\begin{pgfscope}%
\definecolor{textcolor}{rgb}{0.000000,0.000000,0.000000}%
\pgfsetstrokecolor{textcolor}%
\pgfsetfillcolor{textcolor}%
\pgftext[x=0.340277in, y=3.160280in, left, base]{\color{textcolor}\rmfamily\fontsize{10.000000}{12.000000}\selectfont \(\displaystyle {2}\)}%
\end{pgfscope}%
\begin{pgfscope}%
\pgfpathrectangle{\pgfqpoint{0.500000in}{0.440000in}}{\pgfqpoint{3.100000in}{3.080000in}}%
\pgfusepath{clip}%
\pgfsetbuttcap%
\pgfsetroundjoin%
\definecolor{currentfill}{rgb}{0.121569,0.466667,0.705882}%
\pgfsetfillcolor{currentfill}%
\pgfsetlinewidth{1.003750pt}%
\definecolor{currentstroke}{rgb}{0.121569,0.466667,0.705882}%
\pgfsetstrokecolor{currentstroke}%
\pgfsetdash{}{0pt}%
\pgfsys@defobject{currentmarker}{\pgfqpoint{-0.020833in}{-0.020833in}}{\pgfqpoint{0.020833in}{0.020833in}}{%
\pgfpathmoveto{\pgfqpoint{0.000000in}{-0.020833in}}%
\pgfpathcurveto{\pgfqpoint{0.005525in}{-0.020833in}}{\pgfqpoint{0.010825in}{-0.018638in}}{\pgfqpoint{0.014731in}{-0.014731in}}%
\pgfpathcurveto{\pgfqpoint{0.018638in}{-0.010825in}}{\pgfqpoint{0.020833in}{-0.005525in}}{\pgfqpoint{0.020833in}{0.000000in}}%
\pgfpathcurveto{\pgfqpoint{0.020833in}{0.005525in}}{\pgfqpoint{0.018638in}{0.010825in}}{\pgfqpoint{0.014731in}{0.014731in}}%
\pgfpathcurveto{\pgfqpoint{0.010825in}{0.018638in}}{\pgfqpoint{0.005525in}{0.020833in}}{\pgfqpoint{0.000000in}{0.020833in}}%
\pgfpathcurveto{\pgfqpoint{-0.005525in}{0.020833in}}{\pgfqpoint{-0.010825in}{0.018638in}}{\pgfqpoint{-0.014731in}{0.014731in}}%
\pgfpathcurveto{\pgfqpoint{-0.018638in}{0.010825in}}{\pgfqpoint{-0.020833in}{0.005525in}}{\pgfqpoint{-0.020833in}{0.000000in}}%
\pgfpathcurveto{\pgfqpoint{-0.020833in}{-0.005525in}}{\pgfqpoint{-0.018638in}{-0.010825in}}{\pgfqpoint{-0.014731in}{-0.014731in}}%
\pgfpathcurveto{\pgfqpoint{-0.010825in}{-0.018638in}}{\pgfqpoint{-0.005525in}{-0.020833in}}{\pgfqpoint{0.000000in}{-0.020833in}}%
\pgfpathlineto{\pgfqpoint{0.000000in}{-0.020833in}}%
\pgfpathclose%
\pgfusepath{stroke,fill}%
}%
\begin{pgfscope}%
\pgfsys@transformshift{2.050019in}{1.979818in}%
\pgfsys@useobject{currentmarker}{}%
\end{pgfscope}%
\begin{pgfscope}%
\pgfsys@transformshift{2.050019in}{1.979818in}%
\pgfsys@useobject{currentmarker}{}%
\end{pgfscope}%
\begin{pgfscope}%
\pgfsys@transformshift{2.050019in}{1.979818in}%
\pgfsys@useobject{currentmarker}{}%
\end{pgfscope}%
\begin{pgfscope}%
\pgfsys@transformshift{2.050019in}{1.979818in}%
\pgfsys@useobject{currentmarker}{}%
\end{pgfscope}%
\begin{pgfscope}%
\pgfsys@transformshift{2.050019in}{1.979818in}%
\pgfsys@useobject{currentmarker}{}%
\end{pgfscope}%
\begin{pgfscope}%
\pgfsys@transformshift{2.050019in}{1.979818in}%
\pgfsys@useobject{currentmarker}{}%
\end{pgfscope}%
\begin{pgfscope}%
\pgfsys@transformshift{2.050019in}{1.979818in}%
\pgfsys@useobject{currentmarker}{}%
\end{pgfscope}%
\begin{pgfscope}%
\pgfsys@transformshift{2.050019in}{1.979818in}%
\pgfsys@useobject{currentmarker}{}%
\end{pgfscope}%
\begin{pgfscope}%
\pgfsys@transformshift{2.050019in}{1.979818in}%
\pgfsys@useobject{currentmarker}{}%
\end{pgfscope}%
\begin{pgfscope}%
\pgfsys@transformshift{2.050019in}{1.979818in}%
\pgfsys@useobject{currentmarker}{}%
\end{pgfscope}%
\begin{pgfscope}%
\pgfsys@transformshift{2.050019in}{1.979818in}%
\pgfsys@useobject{currentmarker}{}%
\end{pgfscope}%
\begin{pgfscope}%
\pgfsys@transformshift{2.050019in}{1.979818in}%
\pgfsys@useobject{currentmarker}{}%
\end{pgfscope}%
\begin{pgfscope}%
\pgfsys@transformshift{2.050019in}{1.979818in}%
\pgfsys@useobject{currentmarker}{}%
\end{pgfscope}%
\begin{pgfscope}%
\pgfsys@transformshift{2.050019in}{1.979818in}%
\pgfsys@useobject{currentmarker}{}%
\end{pgfscope}%
\begin{pgfscope}%
\pgfsys@transformshift{2.050019in}{1.979818in}%
\pgfsys@useobject{currentmarker}{}%
\end{pgfscope}%
\begin{pgfscope}%
\pgfsys@transformshift{2.050019in}{1.979818in}%
\pgfsys@useobject{currentmarker}{}%
\end{pgfscope}%
\begin{pgfscope}%
\pgfsys@transformshift{2.050019in}{1.979818in}%
\pgfsys@useobject{currentmarker}{}%
\end{pgfscope}%
\begin{pgfscope}%
\pgfsys@transformshift{2.050019in}{1.979818in}%
\pgfsys@useobject{currentmarker}{}%
\end{pgfscope}%
\begin{pgfscope}%
\pgfsys@transformshift{2.050019in}{1.979818in}%
\pgfsys@useobject{currentmarker}{}%
\end{pgfscope}%
\begin{pgfscope}%
\pgfsys@transformshift{2.050019in}{1.979818in}%
\pgfsys@useobject{currentmarker}{}%
\end{pgfscope}%
\begin{pgfscope}%
\pgfsys@transformshift{2.050019in}{1.979818in}%
\pgfsys@useobject{currentmarker}{}%
\end{pgfscope}%
\begin{pgfscope}%
\pgfsys@transformshift{2.050019in}{1.979818in}%
\pgfsys@useobject{currentmarker}{}%
\end{pgfscope}%
\begin{pgfscope}%
\pgfsys@transformshift{2.050019in}{1.979818in}%
\pgfsys@useobject{currentmarker}{}%
\end{pgfscope}%
\begin{pgfscope}%
\pgfsys@transformshift{2.050019in}{1.979818in}%
\pgfsys@useobject{currentmarker}{}%
\end{pgfscope}%
\begin{pgfscope}%
\pgfsys@transformshift{2.050019in}{1.979818in}%
\pgfsys@useobject{currentmarker}{}%
\end{pgfscope}%
\begin{pgfscope}%
\pgfsys@transformshift{2.050019in}{1.979818in}%
\pgfsys@useobject{currentmarker}{}%
\end{pgfscope}%
\begin{pgfscope}%
\pgfsys@transformshift{2.050019in}{1.979818in}%
\pgfsys@useobject{currentmarker}{}%
\end{pgfscope}%
\begin{pgfscope}%
\pgfsys@transformshift{2.050019in}{1.979818in}%
\pgfsys@useobject{currentmarker}{}%
\end{pgfscope}%
\begin{pgfscope}%
\pgfsys@transformshift{2.050019in}{1.979818in}%
\pgfsys@useobject{currentmarker}{}%
\end{pgfscope}%
\begin{pgfscope}%
\pgfsys@transformshift{2.050019in}{1.979818in}%
\pgfsys@useobject{currentmarker}{}%
\end{pgfscope}%
\begin{pgfscope}%
\pgfsys@transformshift{2.050019in}{1.979818in}%
\pgfsys@useobject{currentmarker}{}%
\end{pgfscope}%
\begin{pgfscope}%
\pgfsys@transformshift{2.050019in}{1.979818in}%
\pgfsys@useobject{currentmarker}{}%
\end{pgfscope}%
\begin{pgfscope}%
\pgfsys@transformshift{2.050019in}{1.979818in}%
\pgfsys@useobject{currentmarker}{}%
\end{pgfscope}%
\begin{pgfscope}%
\pgfsys@transformshift{2.050019in}{1.979818in}%
\pgfsys@useobject{currentmarker}{}%
\end{pgfscope}%
\begin{pgfscope}%
\pgfsys@transformshift{2.050019in}{1.979818in}%
\pgfsys@useobject{currentmarker}{}%
\end{pgfscope}%
\begin{pgfscope}%
\pgfsys@transformshift{2.050019in}{1.979818in}%
\pgfsys@useobject{currentmarker}{}%
\end{pgfscope}%
\begin{pgfscope}%
\pgfsys@transformshift{2.050019in}{1.979818in}%
\pgfsys@useobject{currentmarker}{}%
\end{pgfscope}%
\begin{pgfscope}%
\pgfsys@transformshift{2.050019in}{1.979818in}%
\pgfsys@useobject{currentmarker}{}%
\end{pgfscope}%
\begin{pgfscope}%
\pgfsys@transformshift{2.050019in}{1.979818in}%
\pgfsys@useobject{currentmarker}{}%
\end{pgfscope}%
\begin{pgfscope}%
\pgfsys@transformshift{2.050019in}{1.979818in}%
\pgfsys@useobject{currentmarker}{}%
\end{pgfscope}%
\begin{pgfscope}%
\pgfsys@transformshift{2.050019in}{1.979818in}%
\pgfsys@useobject{currentmarker}{}%
\end{pgfscope}%
\begin{pgfscope}%
\pgfsys@transformshift{2.050019in}{1.979818in}%
\pgfsys@useobject{currentmarker}{}%
\end{pgfscope}%
\begin{pgfscope}%
\pgfsys@transformshift{2.050019in}{1.979818in}%
\pgfsys@useobject{currentmarker}{}%
\end{pgfscope}%
\begin{pgfscope}%
\pgfsys@transformshift{2.050019in}{1.979818in}%
\pgfsys@useobject{currentmarker}{}%
\end{pgfscope}%
\begin{pgfscope}%
\pgfsys@transformshift{2.050019in}{1.979818in}%
\pgfsys@useobject{currentmarker}{}%
\end{pgfscope}%
\begin{pgfscope}%
\pgfsys@transformshift{2.050019in}{1.979818in}%
\pgfsys@useobject{currentmarker}{}%
\end{pgfscope}%
\begin{pgfscope}%
\pgfsys@transformshift{2.050019in}{1.979818in}%
\pgfsys@useobject{currentmarker}{}%
\end{pgfscope}%
\begin{pgfscope}%
\pgfsys@transformshift{2.050019in}{1.979818in}%
\pgfsys@useobject{currentmarker}{}%
\end{pgfscope}%
\begin{pgfscope}%
\pgfsys@transformshift{2.050019in}{1.979818in}%
\pgfsys@useobject{currentmarker}{}%
\end{pgfscope}%
\begin{pgfscope}%
\pgfsys@transformshift{2.050019in}{1.979818in}%
\pgfsys@useobject{currentmarker}{}%
\end{pgfscope}%
\begin{pgfscope}%
\pgfsys@transformshift{2.050019in}{1.979818in}%
\pgfsys@useobject{currentmarker}{}%
\end{pgfscope}%
\begin{pgfscope}%
\pgfsys@transformshift{2.050019in}{1.979818in}%
\pgfsys@useobject{currentmarker}{}%
\end{pgfscope}%
\begin{pgfscope}%
\pgfsys@transformshift{2.050019in}{1.979818in}%
\pgfsys@useobject{currentmarker}{}%
\end{pgfscope}%
\begin{pgfscope}%
\pgfsys@transformshift{2.050019in}{1.979818in}%
\pgfsys@useobject{currentmarker}{}%
\end{pgfscope}%
\begin{pgfscope}%
\pgfsys@transformshift{2.050019in}{1.979818in}%
\pgfsys@useobject{currentmarker}{}%
\end{pgfscope}%
\begin{pgfscope}%
\pgfsys@transformshift{2.050019in}{1.979818in}%
\pgfsys@useobject{currentmarker}{}%
\end{pgfscope}%
\begin{pgfscope}%
\pgfsys@transformshift{2.050019in}{1.979818in}%
\pgfsys@useobject{currentmarker}{}%
\end{pgfscope}%
\begin{pgfscope}%
\pgfsys@transformshift{2.050019in}{1.979818in}%
\pgfsys@useobject{currentmarker}{}%
\end{pgfscope}%
\begin{pgfscope}%
\pgfsys@transformshift{2.050019in}{1.979818in}%
\pgfsys@useobject{currentmarker}{}%
\end{pgfscope}%
\begin{pgfscope}%
\pgfsys@transformshift{2.050019in}{1.979818in}%
\pgfsys@useobject{currentmarker}{}%
\end{pgfscope}%
\begin{pgfscope}%
\pgfsys@transformshift{2.050019in}{1.979818in}%
\pgfsys@useobject{currentmarker}{}%
\end{pgfscope}%
\begin{pgfscope}%
\pgfsys@transformshift{2.050019in}{1.979818in}%
\pgfsys@useobject{currentmarker}{}%
\end{pgfscope}%
\begin{pgfscope}%
\pgfsys@transformshift{2.050019in}{1.979818in}%
\pgfsys@useobject{currentmarker}{}%
\end{pgfscope}%
\begin{pgfscope}%
\pgfsys@transformshift{2.050019in}{1.979818in}%
\pgfsys@useobject{currentmarker}{}%
\end{pgfscope}%
\begin{pgfscope}%
\pgfsys@transformshift{2.050019in}{1.979818in}%
\pgfsys@useobject{currentmarker}{}%
\end{pgfscope}%
\begin{pgfscope}%
\pgfsys@transformshift{2.050019in}{1.979818in}%
\pgfsys@useobject{currentmarker}{}%
\end{pgfscope}%
\begin{pgfscope}%
\pgfsys@transformshift{2.050019in}{1.979818in}%
\pgfsys@useobject{currentmarker}{}%
\end{pgfscope}%
\begin{pgfscope}%
\pgfsys@transformshift{2.050019in}{1.979818in}%
\pgfsys@useobject{currentmarker}{}%
\end{pgfscope}%
\begin{pgfscope}%
\pgfsys@transformshift{2.050019in}{1.979818in}%
\pgfsys@useobject{currentmarker}{}%
\end{pgfscope}%
\begin{pgfscope}%
\pgfsys@transformshift{2.050019in}{1.979818in}%
\pgfsys@useobject{currentmarker}{}%
\end{pgfscope}%
\begin{pgfscope}%
\pgfsys@transformshift{2.050019in}{1.979818in}%
\pgfsys@useobject{currentmarker}{}%
\end{pgfscope}%
\begin{pgfscope}%
\pgfsys@transformshift{2.050019in}{1.979818in}%
\pgfsys@useobject{currentmarker}{}%
\end{pgfscope}%
\begin{pgfscope}%
\pgfsys@transformshift{2.050019in}{1.979818in}%
\pgfsys@useobject{currentmarker}{}%
\end{pgfscope}%
\begin{pgfscope}%
\pgfsys@transformshift{2.050019in}{1.979818in}%
\pgfsys@useobject{currentmarker}{}%
\end{pgfscope}%
\begin{pgfscope}%
\pgfsys@transformshift{2.050019in}{1.979818in}%
\pgfsys@useobject{currentmarker}{}%
\end{pgfscope}%
\begin{pgfscope}%
\pgfsys@transformshift{2.050019in}{1.979818in}%
\pgfsys@useobject{currentmarker}{}%
\end{pgfscope}%
\begin{pgfscope}%
\pgfsys@transformshift{2.050019in}{1.979818in}%
\pgfsys@useobject{currentmarker}{}%
\end{pgfscope}%
\begin{pgfscope}%
\pgfsys@transformshift{2.050019in}{1.979818in}%
\pgfsys@useobject{currentmarker}{}%
\end{pgfscope}%
\begin{pgfscope}%
\pgfsys@transformshift{2.050019in}{1.979818in}%
\pgfsys@useobject{currentmarker}{}%
\end{pgfscope}%
\begin{pgfscope}%
\pgfsys@transformshift{2.050019in}{1.979818in}%
\pgfsys@useobject{currentmarker}{}%
\end{pgfscope}%
\begin{pgfscope}%
\pgfsys@transformshift{2.050019in}{1.979818in}%
\pgfsys@useobject{currentmarker}{}%
\end{pgfscope}%
\begin{pgfscope}%
\pgfsys@transformshift{2.050019in}{1.979818in}%
\pgfsys@useobject{currentmarker}{}%
\end{pgfscope}%
\begin{pgfscope}%
\pgfsys@transformshift{2.050019in}{1.979818in}%
\pgfsys@useobject{currentmarker}{}%
\end{pgfscope}%
\begin{pgfscope}%
\pgfsys@transformshift{2.050019in}{1.979818in}%
\pgfsys@useobject{currentmarker}{}%
\end{pgfscope}%
\begin{pgfscope}%
\pgfsys@transformshift{2.050019in}{1.979818in}%
\pgfsys@useobject{currentmarker}{}%
\end{pgfscope}%
\begin{pgfscope}%
\pgfsys@transformshift{2.050019in}{1.979818in}%
\pgfsys@useobject{currentmarker}{}%
\end{pgfscope}%
\begin{pgfscope}%
\pgfsys@transformshift{2.050019in}{1.979818in}%
\pgfsys@useobject{currentmarker}{}%
\end{pgfscope}%
\begin{pgfscope}%
\pgfsys@transformshift{2.050019in}{1.979818in}%
\pgfsys@useobject{currentmarker}{}%
\end{pgfscope}%
\begin{pgfscope}%
\pgfsys@transformshift{2.050019in}{1.979818in}%
\pgfsys@useobject{currentmarker}{}%
\end{pgfscope}%
\begin{pgfscope}%
\pgfsys@transformshift{2.050019in}{1.979818in}%
\pgfsys@useobject{currentmarker}{}%
\end{pgfscope}%
\begin{pgfscope}%
\pgfsys@transformshift{2.050019in}{1.979818in}%
\pgfsys@useobject{currentmarker}{}%
\end{pgfscope}%
\begin{pgfscope}%
\pgfsys@transformshift{2.050019in}{1.979818in}%
\pgfsys@useobject{currentmarker}{}%
\end{pgfscope}%
\begin{pgfscope}%
\pgfsys@transformshift{2.050019in}{1.979818in}%
\pgfsys@useobject{currentmarker}{}%
\end{pgfscope}%
\begin{pgfscope}%
\pgfsys@transformshift{2.050019in}{1.979818in}%
\pgfsys@useobject{currentmarker}{}%
\end{pgfscope}%
\begin{pgfscope}%
\pgfsys@transformshift{2.050019in}{1.979818in}%
\pgfsys@useobject{currentmarker}{}%
\end{pgfscope}%
\begin{pgfscope}%
\pgfsys@transformshift{2.050019in}{1.979818in}%
\pgfsys@useobject{currentmarker}{}%
\end{pgfscope}%
\begin{pgfscope}%
\pgfsys@transformshift{2.050019in}{1.979818in}%
\pgfsys@useobject{currentmarker}{}%
\end{pgfscope}%
\begin{pgfscope}%
\pgfsys@transformshift{2.050019in}{1.979818in}%
\pgfsys@useobject{currentmarker}{}%
\end{pgfscope}%
\begin{pgfscope}%
\pgfsys@transformshift{2.050019in}{1.979818in}%
\pgfsys@useobject{currentmarker}{}%
\end{pgfscope}%
\begin{pgfscope}%
\pgfsys@transformshift{2.050019in}{1.979818in}%
\pgfsys@useobject{currentmarker}{}%
\end{pgfscope}%
\begin{pgfscope}%
\pgfsys@transformshift{2.050019in}{1.979818in}%
\pgfsys@useobject{currentmarker}{}%
\end{pgfscope}%
\begin{pgfscope}%
\pgfsys@transformshift{2.050019in}{1.979818in}%
\pgfsys@useobject{currentmarker}{}%
\end{pgfscope}%
\begin{pgfscope}%
\pgfsys@transformshift{2.050019in}{1.979818in}%
\pgfsys@useobject{currentmarker}{}%
\end{pgfscope}%
\begin{pgfscope}%
\pgfsys@transformshift{2.050019in}{1.979818in}%
\pgfsys@useobject{currentmarker}{}%
\end{pgfscope}%
\begin{pgfscope}%
\pgfsys@transformshift{2.050019in}{1.979818in}%
\pgfsys@useobject{currentmarker}{}%
\end{pgfscope}%
\begin{pgfscope}%
\pgfsys@transformshift{2.050019in}{1.979818in}%
\pgfsys@useobject{currentmarker}{}%
\end{pgfscope}%
\begin{pgfscope}%
\pgfsys@transformshift{2.050019in}{1.979818in}%
\pgfsys@useobject{currentmarker}{}%
\end{pgfscope}%
\begin{pgfscope}%
\pgfsys@transformshift{2.050019in}{1.979818in}%
\pgfsys@useobject{currentmarker}{}%
\end{pgfscope}%
\begin{pgfscope}%
\pgfsys@transformshift{2.050019in}{1.979818in}%
\pgfsys@useobject{currentmarker}{}%
\end{pgfscope}%
\begin{pgfscope}%
\pgfsys@transformshift{2.050019in}{1.979818in}%
\pgfsys@useobject{currentmarker}{}%
\end{pgfscope}%
\begin{pgfscope}%
\pgfsys@transformshift{2.050019in}{1.979818in}%
\pgfsys@useobject{currentmarker}{}%
\end{pgfscope}%
\begin{pgfscope}%
\pgfsys@transformshift{2.050019in}{1.979818in}%
\pgfsys@useobject{currentmarker}{}%
\end{pgfscope}%
\begin{pgfscope}%
\pgfsys@transformshift{2.050019in}{1.979818in}%
\pgfsys@useobject{currentmarker}{}%
\end{pgfscope}%
\begin{pgfscope}%
\pgfsys@transformshift{2.050019in}{1.979818in}%
\pgfsys@useobject{currentmarker}{}%
\end{pgfscope}%
\begin{pgfscope}%
\pgfsys@transformshift{2.050019in}{1.979818in}%
\pgfsys@useobject{currentmarker}{}%
\end{pgfscope}%
\begin{pgfscope}%
\pgfsys@transformshift{2.050019in}{1.979818in}%
\pgfsys@useobject{currentmarker}{}%
\end{pgfscope}%
\begin{pgfscope}%
\pgfsys@transformshift{2.050019in}{1.979818in}%
\pgfsys@useobject{currentmarker}{}%
\end{pgfscope}%
\begin{pgfscope}%
\pgfsys@transformshift{2.050019in}{1.979818in}%
\pgfsys@useobject{currentmarker}{}%
\end{pgfscope}%
\begin{pgfscope}%
\pgfsys@transformshift{2.050019in}{1.979818in}%
\pgfsys@useobject{currentmarker}{}%
\end{pgfscope}%
\begin{pgfscope}%
\pgfsys@transformshift{2.050019in}{1.979818in}%
\pgfsys@useobject{currentmarker}{}%
\end{pgfscope}%
\begin{pgfscope}%
\pgfsys@transformshift{2.050019in}{1.979818in}%
\pgfsys@useobject{currentmarker}{}%
\end{pgfscope}%
\begin{pgfscope}%
\pgfsys@transformshift{2.050019in}{1.979818in}%
\pgfsys@useobject{currentmarker}{}%
\end{pgfscope}%
\begin{pgfscope}%
\pgfsys@transformshift{2.050019in}{1.979818in}%
\pgfsys@useobject{currentmarker}{}%
\end{pgfscope}%
\begin{pgfscope}%
\pgfsys@transformshift{2.050019in}{1.979818in}%
\pgfsys@useobject{currentmarker}{}%
\end{pgfscope}%
\begin{pgfscope}%
\pgfsys@transformshift{2.050019in}{1.979818in}%
\pgfsys@useobject{currentmarker}{}%
\end{pgfscope}%
\begin{pgfscope}%
\pgfsys@transformshift{2.050019in}{1.979818in}%
\pgfsys@useobject{currentmarker}{}%
\end{pgfscope}%
\begin{pgfscope}%
\pgfsys@transformshift{2.050019in}{1.979818in}%
\pgfsys@useobject{currentmarker}{}%
\end{pgfscope}%
\begin{pgfscope}%
\pgfsys@transformshift{2.050019in}{1.979818in}%
\pgfsys@useobject{currentmarker}{}%
\end{pgfscope}%
\begin{pgfscope}%
\pgfsys@transformshift{2.050019in}{1.979818in}%
\pgfsys@useobject{currentmarker}{}%
\end{pgfscope}%
\begin{pgfscope}%
\pgfsys@transformshift{2.050019in}{1.979818in}%
\pgfsys@useobject{currentmarker}{}%
\end{pgfscope}%
\begin{pgfscope}%
\pgfsys@transformshift{2.050019in}{1.979818in}%
\pgfsys@useobject{currentmarker}{}%
\end{pgfscope}%
\begin{pgfscope}%
\pgfsys@transformshift{2.050019in}{1.979818in}%
\pgfsys@useobject{currentmarker}{}%
\end{pgfscope}%
\begin{pgfscope}%
\pgfsys@transformshift{2.050019in}{1.979818in}%
\pgfsys@useobject{currentmarker}{}%
\end{pgfscope}%
\begin{pgfscope}%
\pgfsys@transformshift{2.050019in}{1.979818in}%
\pgfsys@useobject{currentmarker}{}%
\end{pgfscope}%
\begin{pgfscope}%
\pgfsys@transformshift{2.050019in}{1.979818in}%
\pgfsys@useobject{currentmarker}{}%
\end{pgfscope}%
\begin{pgfscope}%
\pgfsys@transformshift{2.050019in}{1.979818in}%
\pgfsys@useobject{currentmarker}{}%
\end{pgfscope}%
\begin{pgfscope}%
\pgfsys@transformshift{2.050019in}{1.979818in}%
\pgfsys@useobject{currentmarker}{}%
\end{pgfscope}%
\begin{pgfscope}%
\pgfsys@transformshift{2.050019in}{1.979818in}%
\pgfsys@useobject{currentmarker}{}%
\end{pgfscope}%
\begin{pgfscope}%
\pgfsys@transformshift{2.050019in}{1.979818in}%
\pgfsys@useobject{currentmarker}{}%
\end{pgfscope}%
\begin{pgfscope}%
\pgfsys@transformshift{2.050019in}{1.979818in}%
\pgfsys@useobject{currentmarker}{}%
\end{pgfscope}%
\begin{pgfscope}%
\pgfsys@transformshift{2.050019in}{1.979818in}%
\pgfsys@useobject{currentmarker}{}%
\end{pgfscope}%
\begin{pgfscope}%
\pgfsys@transformshift{2.050019in}{1.979818in}%
\pgfsys@useobject{currentmarker}{}%
\end{pgfscope}%
\begin{pgfscope}%
\pgfsys@transformshift{2.050019in}{1.979818in}%
\pgfsys@useobject{currentmarker}{}%
\end{pgfscope}%
\begin{pgfscope}%
\pgfsys@transformshift{2.050019in}{1.979818in}%
\pgfsys@useobject{currentmarker}{}%
\end{pgfscope}%
\begin{pgfscope}%
\pgfsys@transformshift{2.050019in}{1.979818in}%
\pgfsys@useobject{currentmarker}{}%
\end{pgfscope}%
\begin{pgfscope}%
\pgfsys@transformshift{2.050019in}{1.979818in}%
\pgfsys@useobject{currentmarker}{}%
\end{pgfscope}%
\begin{pgfscope}%
\pgfsys@transformshift{2.050019in}{1.979818in}%
\pgfsys@useobject{currentmarker}{}%
\end{pgfscope}%
\begin{pgfscope}%
\pgfsys@transformshift{2.050019in}{1.979818in}%
\pgfsys@useobject{currentmarker}{}%
\end{pgfscope}%
\begin{pgfscope}%
\pgfsys@transformshift{2.050019in}{1.979818in}%
\pgfsys@useobject{currentmarker}{}%
\end{pgfscope}%
\begin{pgfscope}%
\pgfsys@transformshift{2.050019in}{1.979818in}%
\pgfsys@useobject{currentmarker}{}%
\end{pgfscope}%
\begin{pgfscope}%
\pgfsys@transformshift{2.050019in}{1.979818in}%
\pgfsys@useobject{currentmarker}{}%
\end{pgfscope}%
\begin{pgfscope}%
\pgfsys@transformshift{2.050019in}{1.979818in}%
\pgfsys@useobject{currentmarker}{}%
\end{pgfscope}%
\begin{pgfscope}%
\pgfsys@transformshift{2.050019in}{1.979818in}%
\pgfsys@useobject{currentmarker}{}%
\end{pgfscope}%
\begin{pgfscope}%
\pgfsys@transformshift{2.050019in}{1.979818in}%
\pgfsys@useobject{currentmarker}{}%
\end{pgfscope}%
\begin{pgfscope}%
\pgfsys@transformshift{2.050019in}{1.979818in}%
\pgfsys@useobject{currentmarker}{}%
\end{pgfscope}%
\begin{pgfscope}%
\pgfsys@transformshift{2.050019in}{1.979818in}%
\pgfsys@useobject{currentmarker}{}%
\end{pgfscope}%
\begin{pgfscope}%
\pgfsys@transformshift{2.050019in}{1.979818in}%
\pgfsys@useobject{currentmarker}{}%
\end{pgfscope}%
\begin{pgfscope}%
\pgfsys@transformshift{2.050019in}{1.979818in}%
\pgfsys@useobject{currentmarker}{}%
\end{pgfscope}%
\begin{pgfscope}%
\pgfsys@transformshift{2.050019in}{1.979818in}%
\pgfsys@useobject{currentmarker}{}%
\end{pgfscope}%
\begin{pgfscope}%
\pgfsys@transformshift{2.050019in}{1.979818in}%
\pgfsys@useobject{currentmarker}{}%
\end{pgfscope}%
\begin{pgfscope}%
\pgfsys@transformshift{2.050019in}{1.979818in}%
\pgfsys@useobject{currentmarker}{}%
\end{pgfscope}%
\begin{pgfscope}%
\pgfsys@transformshift{2.050019in}{1.979818in}%
\pgfsys@useobject{currentmarker}{}%
\end{pgfscope}%
\begin{pgfscope}%
\pgfsys@transformshift{2.050019in}{1.979818in}%
\pgfsys@useobject{currentmarker}{}%
\end{pgfscope}%
\begin{pgfscope}%
\pgfsys@transformshift{2.050019in}{1.979818in}%
\pgfsys@useobject{currentmarker}{}%
\end{pgfscope}%
\begin{pgfscope}%
\pgfsys@transformshift{2.050019in}{1.979818in}%
\pgfsys@useobject{currentmarker}{}%
\end{pgfscope}%
\begin{pgfscope}%
\pgfsys@transformshift{2.050019in}{1.979818in}%
\pgfsys@useobject{currentmarker}{}%
\end{pgfscope}%
\begin{pgfscope}%
\pgfsys@transformshift{2.050019in}{1.979818in}%
\pgfsys@useobject{currentmarker}{}%
\end{pgfscope}%
\begin{pgfscope}%
\pgfsys@transformshift{2.050019in}{1.979818in}%
\pgfsys@useobject{currentmarker}{}%
\end{pgfscope}%
\begin{pgfscope}%
\pgfsys@transformshift{2.050019in}{1.979818in}%
\pgfsys@useobject{currentmarker}{}%
\end{pgfscope}%
\begin{pgfscope}%
\pgfsys@transformshift{2.050019in}{1.979818in}%
\pgfsys@useobject{currentmarker}{}%
\end{pgfscope}%
\begin{pgfscope}%
\pgfsys@transformshift{2.050019in}{1.979818in}%
\pgfsys@useobject{currentmarker}{}%
\end{pgfscope}%
\begin{pgfscope}%
\pgfsys@transformshift{2.050019in}{1.979818in}%
\pgfsys@useobject{currentmarker}{}%
\end{pgfscope}%
\begin{pgfscope}%
\pgfsys@transformshift{2.050019in}{1.979818in}%
\pgfsys@useobject{currentmarker}{}%
\end{pgfscope}%
\begin{pgfscope}%
\pgfsys@transformshift{2.050019in}{1.979818in}%
\pgfsys@useobject{currentmarker}{}%
\end{pgfscope}%
\begin{pgfscope}%
\pgfsys@transformshift{2.050019in}{1.979818in}%
\pgfsys@useobject{currentmarker}{}%
\end{pgfscope}%
\begin{pgfscope}%
\pgfsys@transformshift{2.050019in}{1.979818in}%
\pgfsys@useobject{currentmarker}{}%
\end{pgfscope}%
\begin{pgfscope}%
\pgfsys@transformshift{2.050019in}{1.979818in}%
\pgfsys@useobject{currentmarker}{}%
\end{pgfscope}%
\begin{pgfscope}%
\pgfsys@transformshift{2.050019in}{1.979818in}%
\pgfsys@useobject{currentmarker}{}%
\end{pgfscope}%
\begin{pgfscope}%
\pgfsys@transformshift{2.050019in}{1.979818in}%
\pgfsys@useobject{currentmarker}{}%
\end{pgfscope}%
\begin{pgfscope}%
\pgfsys@transformshift{2.050019in}{1.979818in}%
\pgfsys@useobject{currentmarker}{}%
\end{pgfscope}%
\begin{pgfscope}%
\pgfsys@transformshift{2.050019in}{1.979818in}%
\pgfsys@useobject{currentmarker}{}%
\end{pgfscope}%
\begin{pgfscope}%
\pgfsys@transformshift{2.050019in}{1.979818in}%
\pgfsys@useobject{currentmarker}{}%
\end{pgfscope}%
\begin{pgfscope}%
\pgfsys@transformshift{2.050019in}{1.979818in}%
\pgfsys@useobject{currentmarker}{}%
\end{pgfscope}%
\begin{pgfscope}%
\pgfsys@transformshift{2.050019in}{1.979818in}%
\pgfsys@useobject{currentmarker}{}%
\end{pgfscope}%
\begin{pgfscope}%
\pgfsys@transformshift{2.050019in}{1.979818in}%
\pgfsys@useobject{currentmarker}{}%
\end{pgfscope}%
\begin{pgfscope}%
\pgfsys@transformshift{2.050019in}{1.979818in}%
\pgfsys@useobject{currentmarker}{}%
\end{pgfscope}%
\begin{pgfscope}%
\pgfsys@transformshift{2.050019in}{1.979818in}%
\pgfsys@useobject{currentmarker}{}%
\end{pgfscope}%
\begin{pgfscope}%
\pgfsys@transformshift{2.050019in}{1.979818in}%
\pgfsys@useobject{currentmarker}{}%
\end{pgfscope}%
\begin{pgfscope}%
\pgfsys@transformshift{2.050019in}{1.979818in}%
\pgfsys@useobject{currentmarker}{}%
\end{pgfscope}%
\begin{pgfscope}%
\pgfsys@transformshift{2.050019in}{1.979818in}%
\pgfsys@useobject{currentmarker}{}%
\end{pgfscope}%
\begin{pgfscope}%
\pgfsys@transformshift{2.050019in}{1.979818in}%
\pgfsys@useobject{currentmarker}{}%
\end{pgfscope}%
\begin{pgfscope}%
\pgfsys@transformshift{2.050019in}{1.979818in}%
\pgfsys@useobject{currentmarker}{}%
\end{pgfscope}%
\begin{pgfscope}%
\pgfsys@transformshift{2.050019in}{1.979818in}%
\pgfsys@useobject{currentmarker}{}%
\end{pgfscope}%
\begin{pgfscope}%
\pgfsys@transformshift{2.050019in}{1.979818in}%
\pgfsys@useobject{currentmarker}{}%
\end{pgfscope}%
\begin{pgfscope}%
\pgfsys@transformshift{2.050019in}{1.979818in}%
\pgfsys@useobject{currentmarker}{}%
\end{pgfscope}%
\begin{pgfscope}%
\pgfsys@transformshift{2.050019in}{1.979818in}%
\pgfsys@useobject{currentmarker}{}%
\end{pgfscope}%
\begin{pgfscope}%
\pgfsys@transformshift{2.050019in}{1.979818in}%
\pgfsys@useobject{currentmarker}{}%
\end{pgfscope}%
\begin{pgfscope}%
\pgfsys@transformshift{2.050019in}{1.979818in}%
\pgfsys@useobject{currentmarker}{}%
\end{pgfscope}%
\begin{pgfscope}%
\pgfsys@transformshift{2.050019in}{1.979818in}%
\pgfsys@useobject{currentmarker}{}%
\end{pgfscope}%
\begin{pgfscope}%
\pgfsys@transformshift{2.050019in}{1.979818in}%
\pgfsys@useobject{currentmarker}{}%
\end{pgfscope}%
\begin{pgfscope}%
\pgfsys@transformshift{2.050019in}{1.979818in}%
\pgfsys@useobject{currentmarker}{}%
\end{pgfscope}%
\begin{pgfscope}%
\pgfsys@transformshift{2.050019in}{1.979818in}%
\pgfsys@useobject{currentmarker}{}%
\end{pgfscope}%
\begin{pgfscope}%
\pgfsys@transformshift{2.050019in}{1.979818in}%
\pgfsys@useobject{currentmarker}{}%
\end{pgfscope}%
\begin{pgfscope}%
\pgfsys@transformshift{2.050019in}{1.979818in}%
\pgfsys@useobject{currentmarker}{}%
\end{pgfscope}%
\begin{pgfscope}%
\pgfsys@transformshift{2.050019in}{1.979818in}%
\pgfsys@useobject{currentmarker}{}%
\end{pgfscope}%
\begin{pgfscope}%
\pgfsys@transformshift{2.050019in}{1.979818in}%
\pgfsys@useobject{currentmarker}{}%
\end{pgfscope}%
\begin{pgfscope}%
\pgfsys@transformshift{2.050019in}{1.979818in}%
\pgfsys@useobject{currentmarker}{}%
\end{pgfscope}%
\begin{pgfscope}%
\pgfsys@transformshift{2.050019in}{1.979818in}%
\pgfsys@useobject{currentmarker}{}%
\end{pgfscope}%
\begin{pgfscope}%
\pgfsys@transformshift{2.050019in}{1.979818in}%
\pgfsys@useobject{currentmarker}{}%
\end{pgfscope}%
\begin{pgfscope}%
\pgfsys@transformshift{2.050019in}{1.979818in}%
\pgfsys@useobject{currentmarker}{}%
\end{pgfscope}%
\begin{pgfscope}%
\pgfsys@transformshift{2.050019in}{1.979818in}%
\pgfsys@useobject{currentmarker}{}%
\end{pgfscope}%
\begin{pgfscope}%
\pgfsys@transformshift{2.050019in}{1.979818in}%
\pgfsys@useobject{currentmarker}{}%
\end{pgfscope}%
\begin{pgfscope}%
\pgfsys@transformshift{2.050019in}{1.979818in}%
\pgfsys@useobject{currentmarker}{}%
\end{pgfscope}%
\begin{pgfscope}%
\pgfsys@transformshift{2.050019in}{1.979818in}%
\pgfsys@useobject{currentmarker}{}%
\end{pgfscope}%
\begin{pgfscope}%
\pgfsys@transformshift{2.050019in}{1.979818in}%
\pgfsys@useobject{currentmarker}{}%
\end{pgfscope}%
\begin{pgfscope}%
\pgfsys@transformshift{2.050019in}{1.979818in}%
\pgfsys@useobject{currentmarker}{}%
\end{pgfscope}%
\begin{pgfscope}%
\pgfsys@transformshift{2.050019in}{1.979818in}%
\pgfsys@useobject{currentmarker}{}%
\end{pgfscope}%
\begin{pgfscope}%
\pgfsys@transformshift{2.050019in}{1.979818in}%
\pgfsys@useobject{currentmarker}{}%
\end{pgfscope}%
\begin{pgfscope}%
\pgfsys@transformshift{2.050019in}{1.979818in}%
\pgfsys@useobject{currentmarker}{}%
\end{pgfscope}%
\begin{pgfscope}%
\pgfsys@transformshift{2.050019in}{1.979818in}%
\pgfsys@useobject{currentmarker}{}%
\end{pgfscope}%
\begin{pgfscope}%
\pgfsys@transformshift{2.050019in}{1.979818in}%
\pgfsys@useobject{currentmarker}{}%
\end{pgfscope}%
\begin{pgfscope}%
\pgfsys@transformshift{2.050019in}{1.979818in}%
\pgfsys@useobject{currentmarker}{}%
\end{pgfscope}%
\begin{pgfscope}%
\pgfsys@transformshift{2.050019in}{1.979818in}%
\pgfsys@useobject{currentmarker}{}%
\end{pgfscope}%
\begin{pgfscope}%
\pgfsys@transformshift{2.050019in}{1.979818in}%
\pgfsys@useobject{currentmarker}{}%
\end{pgfscope}%
\begin{pgfscope}%
\pgfsys@transformshift{2.050019in}{1.979818in}%
\pgfsys@useobject{currentmarker}{}%
\end{pgfscope}%
\begin{pgfscope}%
\pgfsys@transformshift{2.050019in}{1.979818in}%
\pgfsys@useobject{currentmarker}{}%
\end{pgfscope}%
\begin{pgfscope}%
\pgfsys@transformshift{2.050019in}{1.979818in}%
\pgfsys@useobject{currentmarker}{}%
\end{pgfscope}%
\begin{pgfscope}%
\pgfsys@transformshift{2.050019in}{1.979818in}%
\pgfsys@useobject{currentmarker}{}%
\end{pgfscope}%
\begin{pgfscope}%
\pgfsys@transformshift{2.050019in}{1.979818in}%
\pgfsys@useobject{currentmarker}{}%
\end{pgfscope}%
\begin{pgfscope}%
\pgfsys@transformshift{2.050019in}{1.979818in}%
\pgfsys@useobject{currentmarker}{}%
\end{pgfscope}%
\begin{pgfscope}%
\pgfsys@transformshift{2.050019in}{1.979818in}%
\pgfsys@useobject{currentmarker}{}%
\end{pgfscope}%
\begin{pgfscope}%
\pgfsys@transformshift{2.050019in}{1.979818in}%
\pgfsys@useobject{currentmarker}{}%
\end{pgfscope}%
\begin{pgfscope}%
\pgfsys@transformshift{2.050019in}{1.979818in}%
\pgfsys@useobject{currentmarker}{}%
\end{pgfscope}%
\begin{pgfscope}%
\pgfsys@transformshift{2.050019in}{1.979818in}%
\pgfsys@useobject{currentmarker}{}%
\end{pgfscope}%
\begin{pgfscope}%
\pgfsys@transformshift{2.050019in}{1.979818in}%
\pgfsys@useobject{currentmarker}{}%
\end{pgfscope}%
\begin{pgfscope}%
\pgfsys@transformshift{2.050019in}{1.979818in}%
\pgfsys@useobject{currentmarker}{}%
\end{pgfscope}%
\begin{pgfscope}%
\pgfsys@transformshift{2.050019in}{1.979818in}%
\pgfsys@useobject{currentmarker}{}%
\end{pgfscope}%
\begin{pgfscope}%
\pgfsys@transformshift{2.050019in}{1.979818in}%
\pgfsys@useobject{currentmarker}{}%
\end{pgfscope}%
\begin{pgfscope}%
\pgfsys@transformshift{2.050019in}{1.979818in}%
\pgfsys@useobject{currentmarker}{}%
\end{pgfscope}%
\begin{pgfscope}%
\pgfsys@transformshift{2.050019in}{1.979818in}%
\pgfsys@useobject{currentmarker}{}%
\end{pgfscope}%
\begin{pgfscope}%
\pgfsys@transformshift{2.050019in}{1.979818in}%
\pgfsys@useobject{currentmarker}{}%
\end{pgfscope}%
\begin{pgfscope}%
\pgfsys@transformshift{2.050019in}{1.979818in}%
\pgfsys@useobject{currentmarker}{}%
\end{pgfscope}%
\begin{pgfscope}%
\pgfsys@transformshift{2.050019in}{1.979818in}%
\pgfsys@useobject{currentmarker}{}%
\end{pgfscope}%
\begin{pgfscope}%
\pgfsys@transformshift{2.050019in}{1.979818in}%
\pgfsys@useobject{currentmarker}{}%
\end{pgfscope}%
\begin{pgfscope}%
\pgfsys@transformshift{2.050019in}{1.979818in}%
\pgfsys@useobject{currentmarker}{}%
\end{pgfscope}%
\begin{pgfscope}%
\pgfsys@transformshift{2.050019in}{1.979818in}%
\pgfsys@useobject{currentmarker}{}%
\end{pgfscope}%
\begin{pgfscope}%
\pgfsys@transformshift{2.050019in}{1.979818in}%
\pgfsys@useobject{currentmarker}{}%
\end{pgfscope}%
\begin{pgfscope}%
\pgfsys@transformshift{2.050019in}{1.979818in}%
\pgfsys@useobject{currentmarker}{}%
\end{pgfscope}%
\begin{pgfscope}%
\pgfsys@transformshift{2.050019in}{1.979818in}%
\pgfsys@useobject{currentmarker}{}%
\end{pgfscope}%
\begin{pgfscope}%
\pgfsys@transformshift{2.050019in}{1.979818in}%
\pgfsys@useobject{currentmarker}{}%
\end{pgfscope}%
\begin{pgfscope}%
\pgfsys@transformshift{2.050019in}{1.979818in}%
\pgfsys@useobject{currentmarker}{}%
\end{pgfscope}%
\begin{pgfscope}%
\pgfsys@transformshift{2.050019in}{1.979818in}%
\pgfsys@useobject{currentmarker}{}%
\end{pgfscope}%
\begin{pgfscope}%
\pgfsys@transformshift{2.050019in}{1.979818in}%
\pgfsys@useobject{currentmarker}{}%
\end{pgfscope}%
\begin{pgfscope}%
\pgfsys@transformshift{2.050019in}{1.979818in}%
\pgfsys@useobject{currentmarker}{}%
\end{pgfscope}%
\begin{pgfscope}%
\pgfsys@transformshift{2.050019in}{1.979818in}%
\pgfsys@useobject{currentmarker}{}%
\end{pgfscope}%
\begin{pgfscope}%
\pgfsys@transformshift{2.050019in}{1.979818in}%
\pgfsys@useobject{currentmarker}{}%
\end{pgfscope}%
\begin{pgfscope}%
\pgfsys@transformshift{2.050019in}{1.979818in}%
\pgfsys@useobject{currentmarker}{}%
\end{pgfscope}%
\begin{pgfscope}%
\pgfsys@transformshift{2.050019in}{1.979818in}%
\pgfsys@useobject{currentmarker}{}%
\end{pgfscope}%
\begin{pgfscope}%
\pgfsys@transformshift{2.050019in}{1.979818in}%
\pgfsys@useobject{currentmarker}{}%
\end{pgfscope}%
\begin{pgfscope}%
\pgfsys@transformshift{2.050019in}{1.979818in}%
\pgfsys@useobject{currentmarker}{}%
\end{pgfscope}%
\begin{pgfscope}%
\pgfsys@transformshift{2.050019in}{1.979818in}%
\pgfsys@useobject{currentmarker}{}%
\end{pgfscope}%
\begin{pgfscope}%
\pgfsys@transformshift{2.050019in}{1.979818in}%
\pgfsys@useobject{currentmarker}{}%
\end{pgfscope}%
\begin{pgfscope}%
\pgfsys@transformshift{2.050019in}{1.979818in}%
\pgfsys@useobject{currentmarker}{}%
\end{pgfscope}%
\begin{pgfscope}%
\pgfsys@transformshift{2.050019in}{1.979818in}%
\pgfsys@useobject{currentmarker}{}%
\end{pgfscope}%
\begin{pgfscope}%
\pgfsys@transformshift{2.050019in}{1.979818in}%
\pgfsys@useobject{currentmarker}{}%
\end{pgfscope}%
\begin{pgfscope}%
\pgfsys@transformshift{2.050019in}{1.979818in}%
\pgfsys@useobject{currentmarker}{}%
\end{pgfscope}%
\begin{pgfscope}%
\pgfsys@transformshift{2.050019in}{1.979818in}%
\pgfsys@useobject{currentmarker}{}%
\end{pgfscope}%
\begin{pgfscope}%
\pgfsys@transformshift{2.050019in}{1.979818in}%
\pgfsys@useobject{currentmarker}{}%
\end{pgfscope}%
\begin{pgfscope}%
\pgfsys@transformshift{2.050019in}{1.979818in}%
\pgfsys@useobject{currentmarker}{}%
\end{pgfscope}%
\begin{pgfscope}%
\pgfsys@transformshift{2.050019in}{1.979818in}%
\pgfsys@useobject{currentmarker}{}%
\end{pgfscope}%
\begin{pgfscope}%
\pgfsys@transformshift{2.050019in}{1.979818in}%
\pgfsys@useobject{currentmarker}{}%
\end{pgfscope}%
\begin{pgfscope}%
\pgfsys@transformshift{2.050019in}{1.979818in}%
\pgfsys@useobject{currentmarker}{}%
\end{pgfscope}%
\begin{pgfscope}%
\pgfsys@transformshift{2.050019in}{1.979818in}%
\pgfsys@useobject{currentmarker}{}%
\end{pgfscope}%
\begin{pgfscope}%
\pgfsys@transformshift{2.050019in}{1.979818in}%
\pgfsys@useobject{currentmarker}{}%
\end{pgfscope}%
\begin{pgfscope}%
\pgfsys@transformshift{2.050019in}{1.979818in}%
\pgfsys@useobject{currentmarker}{}%
\end{pgfscope}%
\begin{pgfscope}%
\pgfsys@transformshift{2.050019in}{1.979818in}%
\pgfsys@useobject{currentmarker}{}%
\end{pgfscope}%
\begin{pgfscope}%
\pgfsys@transformshift{2.050019in}{1.979818in}%
\pgfsys@useobject{currentmarker}{}%
\end{pgfscope}%
\begin{pgfscope}%
\pgfsys@transformshift{2.050019in}{1.979818in}%
\pgfsys@useobject{currentmarker}{}%
\end{pgfscope}%
\begin{pgfscope}%
\pgfsys@transformshift{2.050019in}{1.979818in}%
\pgfsys@useobject{currentmarker}{}%
\end{pgfscope}%
\begin{pgfscope}%
\pgfsys@transformshift{2.050019in}{1.979818in}%
\pgfsys@useobject{currentmarker}{}%
\end{pgfscope}%
\begin{pgfscope}%
\pgfsys@transformshift{2.050019in}{1.979818in}%
\pgfsys@useobject{currentmarker}{}%
\end{pgfscope}%
\begin{pgfscope}%
\pgfsys@transformshift{2.050019in}{1.979818in}%
\pgfsys@useobject{currentmarker}{}%
\end{pgfscope}%
\begin{pgfscope}%
\pgfsys@transformshift{2.050019in}{1.979818in}%
\pgfsys@useobject{currentmarker}{}%
\end{pgfscope}%
\begin{pgfscope}%
\pgfsys@transformshift{2.050019in}{1.979818in}%
\pgfsys@useobject{currentmarker}{}%
\end{pgfscope}%
\begin{pgfscope}%
\pgfsys@transformshift{2.050019in}{1.979818in}%
\pgfsys@useobject{currentmarker}{}%
\end{pgfscope}%
\begin{pgfscope}%
\pgfsys@transformshift{2.050019in}{1.979818in}%
\pgfsys@useobject{currentmarker}{}%
\end{pgfscope}%
\begin{pgfscope}%
\pgfsys@transformshift{2.050019in}{1.979818in}%
\pgfsys@useobject{currentmarker}{}%
\end{pgfscope}%
\begin{pgfscope}%
\pgfsys@transformshift{2.050019in}{1.979818in}%
\pgfsys@useobject{currentmarker}{}%
\end{pgfscope}%
\begin{pgfscope}%
\pgfsys@transformshift{2.050019in}{1.979818in}%
\pgfsys@useobject{currentmarker}{}%
\end{pgfscope}%
\begin{pgfscope}%
\pgfsys@transformshift{2.050019in}{1.979818in}%
\pgfsys@useobject{currentmarker}{}%
\end{pgfscope}%
\begin{pgfscope}%
\pgfsys@transformshift{2.050019in}{1.979818in}%
\pgfsys@useobject{currentmarker}{}%
\end{pgfscope}%
\begin{pgfscope}%
\pgfsys@transformshift{2.050019in}{1.979818in}%
\pgfsys@useobject{currentmarker}{}%
\end{pgfscope}%
\begin{pgfscope}%
\pgfsys@transformshift{2.050019in}{1.979818in}%
\pgfsys@useobject{currentmarker}{}%
\end{pgfscope}%
\begin{pgfscope}%
\pgfsys@transformshift{2.050019in}{1.979818in}%
\pgfsys@useobject{currentmarker}{}%
\end{pgfscope}%
\begin{pgfscope}%
\pgfsys@transformshift{2.050019in}{1.979818in}%
\pgfsys@useobject{currentmarker}{}%
\end{pgfscope}%
\begin{pgfscope}%
\pgfsys@transformshift{2.050019in}{1.979818in}%
\pgfsys@useobject{currentmarker}{}%
\end{pgfscope}%
\begin{pgfscope}%
\pgfsys@transformshift{2.050019in}{1.979818in}%
\pgfsys@useobject{currentmarker}{}%
\end{pgfscope}%
\begin{pgfscope}%
\pgfsys@transformshift{2.050019in}{1.979818in}%
\pgfsys@useobject{currentmarker}{}%
\end{pgfscope}%
\begin{pgfscope}%
\pgfsys@transformshift{2.050019in}{1.979818in}%
\pgfsys@useobject{currentmarker}{}%
\end{pgfscope}%
\begin{pgfscope}%
\pgfsys@transformshift{2.050019in}{1.979818in}%
\pgfsys@useobject{currentmarker}{}%
\end{pgfscope}%
\begin{pgfscope}%
\pgfsys@transformshift{2.050019in}{1.979818in}%
\pgfsys@useobject{currentmarker}{}%
\end{pgfscope}%
\begin{pgfscope}%
\pgfsys@transformshift{2.050019in}{1.979818in}%
\pgfsys@useobject{currentmarker}{}%
\end{pgfscope}%
\begin{pgfscope}%
\pgfsys@transformshift{2.050019in}{1.979818in}%
\pgfsys@useobject{currentmarker}{}%
\end{pgfscope}%
\begin{pgfscope}%
\pgfsys@transformshift{2.050019in}{1.979818in}%
\pgfsys@useobject{currentmarker}{}%
\end{pgfscope}%
\begin{pgfscope}%
\pgfsys@transformshift{2.050019in}{1.979818in}%
\pgfsys@useobject{currentmarker}{}%
\end{pgfscope}%
\begin{pgfscope}%
\pgfsys@transformshift{2.050019in}{1.979818in}%
\pgfsys@useobject{currentmarker}{}%
\end{pgfscope}%
\begin{pgfscope}%
\pgfsys@transformshift{2.050019in}{1.979818in}%
\pgfsys@useobject{currentmarker}{}%
\end{pgfscope}%
\begin{pgfscope}%
\pgfsys@transformshift{2.050019in}{1.979818in}%
\pgfsys@useobject{currentmarker}{}%
\end{pgfscope}%
\begin{pgfscope}%
\pgfsys@transformshift{2.050019in}{1.979818in}%
\pgfsys@useobject{currentmarker}{}%
\end{pgfscope}%
\begin{pgfscope}%
\pgfsys@transformshift{2.050019in}{1.979818in}%
\pgfsys@useobject{currentmarker}{}%
\end{pgfscope}%
\begin{pgfscope}%
\pgfsys@transformshift{2.050019in}{1.979818in}%
\pgfsys@useobject{currentmarker}{}%
\end{pgfscope}%
\begin{pgfscope}%
\pgfsys@transformshift{2.050019in}{1.979818in}%
\pgfsys@useobject{currentmarker}{}%
\end{pgfscope}%
\begin{pgfscope}%
\pgfsys@transformshift{2.050019in}{1.979818in}%
\pgfsys@useobject{currentmarker}{}%
\end{pgfscope}%
\begin{pgfscope}%
\pgfsys@transformshift{2.050019in}{1.979818in}%
\pgfsys@useobject{currentmarker}{}%
\end{pgfscope}%
\begin{pgfscope}%
\pgfsys@transformshift{2.050019in}{1.979818in}%
\pgfsys@useobject{currentmarker}{}%
\end{pgfscope}%
\begin{pgfscope}%
\pgfsys@transformshift{2.050019in}{1.979818in}%
\pgfsys@useobject{currentmarker}{}%
\end{pgfscope}%
\begin{pgfscope}%
\pgfsys@transformshift{2.050019in}{1.979818in}%
\pgfsys@useobject{currentmarker}{}%
\end{pgfscope}%
\begin{pgfscope}%
\pgfsys@transformshift{2.050019in}{1.979818in}%
\pgfsys@useobject{currentmarker}{}%
\end{pgfscope}%
\begin{pgfscope}%
\pgfsys@transformshift{2.050019in}{1.979818in}%
\pgfsys@useobject{currentmarker}{}%
\end{pgfscope}%
\begin{pgfscope}%
\pgfsys@transformshift{2.050019in}{1.979818in}%
\pgfsys@useobject{currentmarker}{}%
\end{pgfscope}%
\begin{pgfscope}%
\pgfsys@transformshift{2.050019in}{1.979818in}%
\pgfsys@useobject{currentmarker}{}%
\end{pgfscope}%
\begin{pgfscope}%
\pgfsys@transformshift{2.050019in}{1.979818in}%
\pgfsys@useobject{currentmarker}{}%
\end{pgfscope}%
\begin{pgfscope}%
\pgfsys@transformshift{2.050019in}{1.979818in}%
\pgfsys@useobject{currentmarker}{}%
\end{pgfscope}%
\begin{pgfscope}%
\pgfsys@transformshift{2.050019in}{1.979818in}%
\pgfsys@useobject{currentmarker}{}%
\end{pgfscope}%
\begin{pgfscope}%
\pgfsys@transformshift{2.050019in}{1.979818in}%
\pgfsys@useobject{currentmarker}{}%
\end{pgfscope}%
\begin{pgfscope}%
\pgfsys@transformshift{2.050019in}{1.979818in}%
\pgfsys@useobject{currentmarker}{}%
\end{pgfscope}%
\begin{pgfscope}%
\pgfsys@transformshift{2.050019in}{1.979818in}%
\pgfsys@useobject{currentmarker}{}%
\end{pgfscope}%
\begin{pgfscope}%
\pgfsys@transformshift{2.050019in}{1.979818in}%
\pgfsys@useobject{currentmarker}{}%
\end{pgfscope}%
\begin{pgfscope}%
\pgfsys@transformshift{2.050019in}{1.979818in}%
\pgfsys@useobject{currentmarker}{}%
\end{pgfscope}%
\begin{pgfscope}%
\pgfsys@transformshift{2.050019in}{1.979818in}%
\pgfsys@useobject{currentmarker}{}%
\end{pgfscope}%
\begin{pgfscope}%
\pgfsys@transformshift{2.050019in}{1.979818in}%
\pgfsys@useobject{currentmarker}{}%
\end{pgfscope}%
\begin{pgfscope}%
\pgfsys@transformshift{2.050019in}{1.979818in}%
\pgfsys@useobject{currentmarker}{}%
\end{pgfscope}%
\begin{pgfscope}%
\pgfsys@transformshift{2.050019in}{1.979818in}%
\pgfsys@useobject{currentmarker}{}%
\end{pgfscope}%
\begin{pgfscope}%
\pgfsys@transformshift{2.050019in}{1.979818in}%
\pgfsys@useobject{currentmarker}{}%
\end{pgfscope}%
\begin{pgfscope}%
\pgfsys@transformshift{2.050019in}{1.979818in}%
\pgfsys@useobject{currentmarker}{}%
\end{pgfscope}%
\begin{pgfscope}%
\pgfsys@transformshift{2.050019in}{1.979818in}%
\pgfsys@useobject{currentmarker}{}%
\end{pgfscope}%
\begin{pgfscope}%
\pgfsys@transformshift{2.050019in}{1.979818in}%
\pgfsys@useobject{currentmarker}{}%
\end{pgfscope}%
\begin{pgfscope}%
\pgfsys@transformshift{2.050019in}{1.979818in}%
\pgfsys@useobject{currentmarker}{}%
\end{pgfscope}%
\begin{pgfscope}%
\pgfsys@transformshift{2.050019in}{1.979818in}%
\pgfsys@useobject{currentmarker}{}%
\end{pgfscope}%
\begin{pgfscope}%
\pgfsys@transformshift{2.050019in}{1.979818in}%
\pgfsys@useobject{currentmarker}{}%
\end{pgfscope}%
\begin{pgfscope}%
\pgfsys@transformshift{2.050019in}{1.979818in}%
\pgfsys@useobject{currentmarker}{}%
\end{pgfscope}%
\begin{pgfscope}%
\pgfsys@transformshift{2.050019in}{1.979818in}%
\pgfsys@useobject{currentmarker}{}%
\end{pgfscope}%
\begin{pgfscope}%
\pgfsys@transformshift{2.050019in}{1.979818in}%
\pgfsys@useobject{currentmarker}{}%
\end{pgfscope}%
\begin{pgfscope}%
\pgfsys@transformshift{2.050019in}{1.979818in}%
\pgfsys@useobject{currentmarker}{}%
\end{pgfscope}%
\begin{pgfscope}%
\pgfsys@transformshift{2.050019in}{1.979818in}%
\pgfsys@useobject{currentmarker}{}%
\end{pgfscope}%
\begin{pgfscope}%
\pgfsys@transformshift{2.050019in}{1.979818in}%
\pgfsys@useobject{currentmarker}{}%
\end{pgfscope}%
\begin{pgfscope}%
\pgfsys@transformshift{2.050019in}{1.979818in}%
\pgfsys@useobject{currentmarker}{}%
\end{pgfscope}%
\begin{pgfscope}%
\pgfsys@transformshift{2.050019in}{1.979818in}%
\pgfsys@useobject{currentmarker}{}%
\end{pgfscope}%
\begin{pgfscope}%
\pgfsys@transformshift{2.050019in}{1.979818in}%
\pgfsys@useobject{currentmarker}{}%
\end{pgfscope}%
\begin{pgfscope}%
\pgfsys@transformshift{2.050019in}{1.979818in}%
\pgfsys@useobject{currentmarker}{}%
\end{pgfscope}%
\begin{pgfscope}%
\pgfsys@transformshift{2.050019in}{1.979818in}%
\pgfsys@useobject{currentmarker}{}%
\end{pgfscope}%
\begin{pgfscope}%
\pgfsys@transformshift{2.050019in}{1.979818in}%
\pgfsys@useobject{currentmarker}{}%
\end{pgfscope}%
\begin{pgfscope}%
\pgfsys@transformshift{2.050019in}{1.979818in}%
\pgfsys@useobject{currentmarker}{}%
\end{pgfscope}%
\begin{pgfscope}%
\pgfsys@transformshift{2.050019in}{1.979818in}%
\pgfsys@useobject{currentmarker}{}%
\end{pgfscope}%
\begin{pgfscope}%
\pgfsys@transformshift{2.050019in}{1.979818in}%
\pgfsys@useobject{currentmarker}{}%
\end{pgfscope}%
\begin{pgfscope}%
\pgfsys@transformshift{2.050019in}{1.979818in}%
\pgfsys@useobject{currentmarker}{}%
\end{pgfscope}%
\begin{pgfscope}%
\pgfsys@transformshift{2.050019in}{1.979818in}%
\pgfsys@useobject{currentmarker}{}%
\end{pgfscope}%
\begin{pgfscope}%
\pgfsys@transformshift{2.050019in}{1.979818in}%
\pgfsys@useobject{currentmarker}{}%
\end{pgfscope}%
\begin{pgfscope}%
\pgfsys@transformshift{2.050019in}{1.979818in}%
\pgfsys@useobject{currentmarker}{}%
\end{pgfscope}%
\begin{pgfscope}%
\pgfsys@transformshift{2.050019in}{1.979818in}%
\pgfsys@useobject{currentmarker}{}%
\end{pgfscope}%
\begin{pgfscope}%
\pgfsys@transformshift{2.050019in}{1.979818in}%
\pgfsys@useobject{currentmarker}{}%
\end{pgfscope}%
\begin{pgfscope}%
\pgfsys@transformshift{2.050019in}{1.979818in}%
\pgfsys@useobject{currentmarker}{}%
\end{pgfscope}%
\begin{pgfscope}%
\pgfsys@transformshift{2.050019in}{1.979818in}%
\pgfsys@useobject{currentmarker}{}%
\end{pgfscope}%
\begin{pgfscope}%
\pgfsys@transformshift{2.050019in}{1.979818in}%
\pgfsys@useobject{currentmarker}{}%
\end{pgfscope}%
\begin{pgfscope}%
\pgfsys@transformshift{2.050019in}{1.979818in}%
\pgfsys@useobject{currentmarker}{}%
\end{pgfscope}%
\begin{pgfscope}%
\pgfsys@transformshift{2.050019in}{1.979818in}%
\pgfsys@useobject{currentmarker}{}%
\end{pgfscope}%
\begin{pgfscope}%
\pgfsys@transformshift{2.050019in}{1.979818in}%
\pgfsys@useobject{currentmarker}{}%
\end{pgfscope}%
\begin{pgfscope}%
\pgfsys@transformshift{2.050019in}{1.979818in}%
\pgfsys@useobject{currentmarker}{}%
\end{pgfscope}%
\begin{pgfscope}%
\pgfsys@transformshift{2.050019in}{1.979818in}%
\pgfsys@useobject{currentmarker}{}%
\end{pgfscope}%
\begin{pgfscope}%
\pgfsys@transformshift{2.050019in}{1.979818in}%
\pgfsys@useobject{currentmarker}{}%
\end{pgfscope}%
\begin{pgfscope}%
\pgfsys@transformshift{2.050019in}{1.979818in}%
\pgfsys@useobject{currentmarker}{}%
\end{pgfscope}%
\begin{pgfscope}%
\pgfsys@transformshift{2.050019in}{1.979818in}%
\pgfsys@useobject{currentmarker}{}%
\end{pgfscope}%
\begin{pgfscope}%
\pgfsys@transformshift{2.050019in}{1.979818in}%
\pgfsys@useobject{currentmarker}{}%
\end{pgfscope}%
\begin{pgfscope}%
\pgfsys@transformshift{2.050019in}{1.979818in}%
\pgfsys@useobject{currentmarker}{}%
\end{pgfscope}%
\begin{pgfscope}%
\pgfsys@transformshift{2.050019in}{1.979818in}%
\pgfsys@useobject{currentmarker}{}%
\end{pgfscope}%
\begin{pgfscope}%
\pgfsys@transformshift{2.050019in}{1.979818in}%
\pgfsys@useobject{currentmarker}{}%
\end{pgfscope}%
\begin{pgfscope}%
\pgfsys@transformshift{2.050019in}{1.979818in}%
\pgfsys@useobject{currentmarker}{}%
\end{pgfscope}%
\begin{pgfscope}%
\pgfsys@transformshift{2.050019in}{1.979818in}%
\pgfsys@useobject{currentmarker}{}%
\end{pgfscope}%
\begin{pgfscope}%
\pgfsys@transformshift{2.050019in}{1.979818in}%
\pgfsys@useobject{currentmarker}{}%
\end{pgfscope}%
\begin{pgfscope}%
\pgfsys@transformshift{2.050019in}{1.979818in}%
\pgfsys@useobject{currentmarker}{}%
\end{pgfscope}%
\begin{pgfscope}%
\pgfsys@transformshift{2.050019in}{1.979818in}%
\pgfsys@useobject{currentmarker}{}%
\end{pgfscope}%
\begin{pgfscope}%
\pgfsys@transformshift{2.050019in}{1.979818in}%
\pgfsys@useobject{currentmarker}{}%
\end{pgfscope}%
\begin{pgfscope}%
\pgfsys@transformshift{2.050019in}{1.979818in}%
\pgfsys@useobject{currentmarker}{}%
\end{pgfscope}%
\begin{pgfscope}%
\pgfsys@transformshift{2.050019in}{1.979818in}%
\pgfsys@useobject{currentmarker}{}%
\end{pgfscope}%
\begin{pgfscope}%
\pgfsys@transformshift{2.050019in}{1.979818in}%
\pgfsys@useobject{currentmarker}{}%
\end{pgfscope}%
\begin{pgfscope}%
\pgfsys@transformshift{2.050019in}{1.979818in}%
\pgfsys@useobject{currentmarker}{}%
\end{pgfscope}%
\begin{pgfscope}%
\pgfsys@transformshift{2.050019in}{1.979818in}%
\pgfsys@useobject{currentmarker}{}%
\end{pgfscope}%
\begin{pgfscope}%
\pgfsys@transformshift{2.050019in}{1.979818in}%
\pgfsys@useobject{currentmarker}{}%
\end{pgfscope}%
\begin{pgfscope}%
\pgfsys@transformshift{2.050019in}{1.979818in}%
\pgfsys@useobject{currentmarker}{}%
\end{pgfscope}%
\begin{pgfscope}%
\pgfsys@transformshift{2.050019in}{1.979818in}%
\pgfsys@useobject{currentmarker}{}%
\end{pgfscope}%
\begin{pgfscope}%
\pgfsys@transformshift{2.050019in}{1.979818in}%
\pgfsys@useobject{currentmarker}{}%
\end{pgfscope}%
\begin{pgfscope}%
\pgfsys@transformshift{2.050019in}{1.979818in}%
\pgfsys@useobject{currentmarker}{}%
\end{pgfscope}%
\begin{pgfscope}%
\pgfsys@transformshift{2.050019in}{1.979818in}%
\pgfsys@useobject{currentmarker}{}%
\end{pgfscope}%
\begin{pgfscope}%
\pgfsys@transformshift{2.050019in}{1.979818in}%
\pgfsys@useobject{currentmarker}{}%
\end{pgfscope}%
\begin{pgfscope}%
\pgfsys@transformshift{2.050019in}{1.979818in}%
\pgfsys@useobject{currentmarker}{}%
\end{pgfscope}%
\begin{pgfscope}%
\pgfsys@transformshift{2.050019in}{1.979818in}%
\pgfsys@useobject{currentmarker}{}%
\end{pgfscope}%
\begin{pgfscope}%
\pgfsys@transformshift{2.050019in}{1.979818in}%
\pgfsys@useobject{currentmarker}{}%
\end{pgfscope}%
\begin{pgfscope}%
\pgfsys@transformshift{2.050019in}{1.979818in}%
\pgfsys@useobject{currentmarker}{}%
\end{pgfscope}%
\begin{pgfscope}%
\pgfsys@transformshift{2.050019in}{1.979818in}%
\pgfsys@useobject{currentmarker}{}%
\end{pgfscope}%
\begin{pgfscope}%
\pgfsys@transformshift{2.050019in}{1.979818in}%
\pgfsys@useobject{currentmarker}{}%
\end{pgfscope}%
\begin{pgfscope}%
\pgfsys@transformshift{2.050019in}{1.979818in}%
\pgfsys@useobject{currentmarker}{}%
\end{pgfscope}%
\begin{pgfscope}%
\pgfsys@transformshift{2.050019in}{1.979818in}%
\pgfsys@useobject{currentmarker}{}%
\end{pgfscope}%
\begin{pgfscope}%
\pgfsys@transformshift{2.050019in}{1.979818in}%
\pgfsys@useobject{currentmarker}{}%
\end{pgfscope}%
\begin{pgfscope}%
\pgfsys@transformshift{2.050019in}{1.979818in}%
\pgfsys@useobject{currentmarker}{}%
\end{pgfscope}%
\begin{pgfscope}%
\pgfsys@transformshift{2.050019in}{1.979818in}%
\pgfsys@useobject{currentmarker}{}%
\end{pgfscope}%
\begin{pgfscope}%
\pgfsys@transformshift{2.050019in}{1.979818in}%
\pgfsys@useobject{currentmarker}{}%
\end{pgfscope}%
\begin{pgfscope}%
\pgfsys@transformshift{2.050019in}{1.979818in}%
\pgfsys@useobject{currentmarker}{}%
\end{pgfscope}%
\begin{pgfscope}%
\pgfsys@transformshift{2.050019in}{1.979818in}%
\pgfsys@useobject{currentmarker}{}%
\end{pgfscope}%
\begin{pgfscope}%
\pgfsys@transformshift{2.050019in}{1.979818in}%
\pgfsys@useobject{currentmarker}{}%
\end{pgfscope}%
\begin{pgfscope}%
\pgfsys@transformshift{2.050019in}{1.979818in}%
\pgfsys@useobject{currentmarker}{}%
\end{pgfscope}%
\begin{pgfscope}%
\pgfsys@transformshift{2.050019in}{1.979818in}%
\pgfsys@useobject{currentmarker}{}%
\end{pgfscope}%
\begin{pgfscope}%
\pgfsys@transformshift{2.050019in}{1.979818in}%
\pgfsys@useobject{currentmarker}{}%
\end{pgfscope}%
\begin{pgfscope}%
\pgfsys@transformshift{2.050019in}{1.979818in}%
\pgfsys@useobject{currentmarker}{}%
\end{pgfscope}%
\begin{pgfscope}%
\pgfsys@transformshift{2.050019in}{1.979818in}%
\pgfsys@useobject{currentmarker}{}%
\end{pgfscope}%
\begin{pgfscope}%
\pgfsys@transformshift{2.050019in}{1.979818in}%
\pgfsys@useobject{currentmarker}{}%
\end{pgfscope}%
\begin{pgfscope}%
\pgfsys@transformshift{2.050019in}{1.979818in}%
\pgfsys@useobject{currentmarker}{}%
\end{pgfscope}%
\begin{pgfscope}%
\pgfsys@transformshift{2.050019in}{1.979818in}%
\pgfsys@useobject{currentmarker}{}%
\end{pgfscope}%
\begin{pgfscope}%
\pgfsys@transformshift{2.050019in}{1.979818in}%
\pgfsys@useobject{currentmarker}{}%
\end{pgfscope}%
\begin{pgfscope}%
\pgfsys@transformshift{2.050019in}{1.979818in}%
\pgfsys@useobject{currentmarker}{}%
\end{pgfscope}%
\begin{pgfscope}%
\pgfsys@transformshift{2.050019in}{1.979818in}%
\pgfsys@useobject{currentmarker}{}%
\end{pgfscope}%
\begin{pgfscope}%
\pgfsys@transformshift{2.050019in}{1.979818in}%
\pgfsys@useobject{currentmarker}{}%
\end{pgfscope}%
\begin{pgfscope}%
\pgfsys@transformshift{2.050019in}{1.979818in}%
\pgfsys@useobject{currentmarker}{}%
\end{pgfscope}%
\begin{pgfscope}%
\pgfsys@transformshift{2.050019in}{1.979818in}%
\pgfsys@useobject{currentmarker}{}%
\end{pgfscope}%
\begin{pgfscope}%
\pgfsys@transformshift{2.050019in}{1.979818in}%
\pgfsys@useobject{currentmarker}{}%
\end{pgfscope}%
\begin{pgfscope}%
\pgfsys@transformshift{2.050019in}{1.979818in}%
\pgfsys@useobject{currentmarker}{}%
\end{pgfscope}%
\begin{pgfscope}%
\pgfsys@transformshift{2.050019in}{1.979818in}%
\pgfsys@useobject{currentmarker}{}%
\end{pgfscope}%
\begin{pgfscope}%
\pgfsys@transformshift{2.050019in}{1.979818in}%
\pgfsys@useobject{currentmarker}{}%
\end{pgfscope}%
\begin{pgfscope}%
\pgfsys@transformshift{2.050019in}{1.979818in}%
\pgfsys@useobject{currentmarker}{}%
\end{pgfscope}%
\begin{pgfscope}%
\pgfsys@transformshift{2.050019in}{1.979818in}%
\pgfsys@useobject{currentmarker}{}%
\end{pgfscope}%
\begin{pgfscope}%
\pgfsys@transformshift{2.050019in}{1.979818in}%
\pgfsys@useobject{currentmarker}{}%
\end{pgfscope}%
\begin{pgfscope}%
\pgfsys@transformshift{2.050019in}{1.979818in}%
\pgfsys@useobject{currentmarker}{}%
\end{pgfscope}%
\begin{pgfscope}%
\pgfsys@transformshift{2.050019in}{1.979818in}%
\pgfsys@useobject{currentmarker}{}%
\end{pgfscope}%
\begin{pgfscope}%
\pgfsys@transformshift{2.050019in}{1.979818in}%
\pgfsys@useobject{currentmarker}{}%
\end{pgfscope}%
\begin{pgfscope}%
\pgfsys@transformshift{2.050019in}{1.979818in}%
\pgfsys@useobject{currentmarker}{}%
\end{pgfscope}%
\begin{pgfscope}%
\pgfsys@transformshift{2.050019in}{1.979818in}%
\pgfsys@useobject{currentmarker}{}%
\end{pgfscope}%
\begin{pgfscope}%
\pgfsys@transformshift{2.050019in}{1.979818in}%
\pgfsys@useobject{currentmarker}{}%
\end{pgfscope}%
\begin{pgfscope}%
\pgfsys@transformshift{2.050019in}{1.979818in}%
\pgfsys@useobject{currentmarker}{}%
\end{pgfscope}%
\begin{pgfscope}%
\pgfsys@transformshift{2.050019in}{1.979818in}%
\pgfsys@useobject{currentmarker}{}%
\end{pgfscope}%
\begin{pgfscope}%
\pgfsys@transformshift{2.050019in}{1.979818in}%
\pgfsys@useobject{currentmarker}{}%
\end{pgfscope}%
\begin{pgfscope}%
\pgfsys@transformshift{2.050019in}{1.979818in}%
\pgfsys@useobject{currentmarker}{}%
\end{pgfscope}%
\begin{pgfscope}%
\pgfsys@transformshift{2.050019in}{1.979818in}%
\pgfsys@useobject{currentmarker}{}%
\end{pgfscope}%
\begin{pgfscope}%
\pgfsys@transformshift{2.050019in}{1.979818in}%
\pgfsys@useobject{currentmarker}{}%
\end{pgfscope}%
\begin{pgfscope}%
\pgfsys@transformshift{2.050019in}{1.979818in}%
\pgfsys@useobject{currentmarker}{}%
\end{pgfscope}%
\begin{pgfscope}%
\pgfsys@transformshift{2.050019in}{1.979818in}%
\pgfsys@useobject{currentmarker}{}%
\end{pgfscope}%
\begin{pgfscope}%
\pgfsys@transformshift{2.050019in}{1.979818in}%
\pgfsys@useobject{currentmarker}{}%
\end{pgfscope}%
\begin{pgfscope}%
\pgfsys@transformshift{2.050019in}{1.979818in}%
\pgfsys@useobject{currentmarker}{}%
\end{pgfscope}%
\begin{pgfscope}%
\pgfsys@transformshift{2.050019in}{1.979818in}%
\pgfsys@useobject{currentmarker}{}%
\end{pgfscope}%
\begin{pgfscope}%
\pgfsys@transformshift{2.050019in}{1.979818in}%
\pgfsys@useobject{currentmarker}{}%
\end{pgfscope}%
\begin{pgfscope}%
\pgfsys@transformshift{2.050019in}{1.979818in}%
\pgfsys@useobject{currentmarker}{}%
\end{pgfscope}%
\begin{pgfscope}%
\pgfsys@transformshift{2.050019in}{1.979818in}%
\pgfsys@useobject{currentmarker}{}%
\end{pgfscope}%
\begin{pgfscope}%
\pgfsys@transformshift{2.050019in}{1.979818in}%
\pgfsys@useobject{currentmarker}{}%
\end{pgfscope}%
\begin{pgfscope}%
\pgfsys@transformshift{2.050019in}{1.979818in}%
\pgfsys@useobject{currentmarker}{}%
\end{pgfscope}%
\begin{pgfscope}%
\pgfsys@transformshift{2.050019in}{1.979818in}%
\pgfsys@useobject{currentmarker}{}%
\end{pgfscope}%
\begin{pgfscope}%
\pgfsys@transformshift{2.050019in}{1.979818in}%
\pgfsys@useobject{currentmarker}{}%
\end{pgfscope}%
\begin{pgfscope}%
\pgfsys@transformshift{2.050019in}{1.979818in}%
\pgfsys@useobject{currentmarker}{}%
\end{pgfscope}%
\begin{pgfscope}%
\pgfsys@transformshift{2.050019in}{1.979818in}%
\pgfsys@useobject{currentmarker}{}%
\end{pgfscope}%
\begin{pgfscope}%
\pgfsys@transformshift{2.050019in}{1.979818in}%
\pgfsys@useobject{currentmarker}{}%
\end{pgfscope}%
\begin{pgfscope}%
\pgfsys@transformshift{2.050019in}{1.979818in}%
\pgfsys@useobject{currentmarker}{}%
\end{pgfscope}%
\begin{pgfscope}%
\pgfsys@transformshift{2.050019in}{1.979818in}%
\pgfsys@useobject{currentmarker}{}%
\end{pgfscope}%
\begin{pgfscope}%
\pgfsys@transformshift{2.050019in}{1.979818in}%
\pgfsys@useobject{currentmarker}{}%
\end{pgfscope}%
\begin{pgfscope}%
\pgfsys@transformshift{2.050019in}{1.979818in}%
\pgfsys@useobject{currentmarker}{}%
\end{pgfscope}%
\begin{pgfscope}%
\pgfsys@transformshift{2.050019in}{1.979818in}%
\pgfsys@useobject{currentmarker}{}%
\end{pgfscope}%
\begin{pgfscope}%
\pgfsys@transformshift{2.050019in}{1.979818in}%
\pgfsys@useobject{currentmarker}{}%
\end{pgfscope}%
\begin{pgfscope}%
\pgfsys@transformshift{2.050019in}{1.979818in}%
\pgfsys@useobject{currentmarker}{}%
\end{pgfscope}%
\begin{pgfscope}%
\pgfsys@transformshift{2.050019in}{1.979818in}%
\pgfsys@useobject{currentmarker}{}%
\end{pgfscope}%
\begin{pgfscope}%
\pgfsys@transformshift{2.050019in}{1.979818in}%
\pgfsys@useobject{currentmarker}{}%
\end{pgfscope}%
\begin{pgfscope}%
\pgfsys@transformshift{2.050019in}{1.979818in}%
\pgfsys@useobject{currentmarker}{}%
\end{pgfscope}%
\begin{pgfscope}%
\pgfsys@transformshift{2.050019in}{1.979818in}%
\pgfsys@useobject{currentmarker}{}%
\end{pgfscope}%
\begin{pgfscope}%
\pgfsys@transformshift{2.050019in}{1.979818in}%
\pgfsys@useobject{currentmarker}{}%
\end{pgfscope}%
\begin{pgfscope}%
\pgfsys@transformshift{2.050019in}{1.979818in}%
\pgfsys@useobject{currentmarker}{}%
\end{pgfscope}%
\begin{pgfscope}%
\pgfsys@transformshift{2.050019in}{1.979818in}%
\pgfsys@useobject{currentmarker}{}%
\end{pgfscope}%
\begin{pgfscope}%
\pgfsys@transformshift{2.050019in}{1.979818in}%
\pgfsys@useobject{currentmarker}{}%
\end{pgfscope}%
\begin{pgfscope}%
\pgfsys@transformshift{2.050019in}{1.979818in}%
\pgfsys@useobject{currentmarker}{}%
\end{pgfscope}%
\begin{pgfscope}%
\pgfsys@transformshift{2.050019in}{1.979818in}%
\pgfsys@useobject{currentmarker}{}%
\end{pgfscope}%
\begin{pgfscope}%
\pgfsys@transformshift{2.050019in}{1.979818in}%
\pgfsys@useobject{currentmarker}{}%
\end{pgfscope}%
\begin{pgfscope}%
\pgfsys@transformshift{2.050019in}{1.979818in}%
\pgfsys@useobject{currentmarker}{}%
\end{pgfscope}%
\begin{pgfscope}%
\pgfsys@transformshift{2.050019in}{1.979818in}%
\pgfsys@useobject{currentmarker}{}%
\end{pgfscope}%
\begin{pgfscope}%
\pgfsys@transformshift{2.050019in}{1.979818in}%
\pgfsys@useobject{currentmarker}{}%
\end{pgfscope}%
\begin{pgfscope}%
\pgfsys@transformshift{2.050019in}{1.979818in}%
\pgfsys@useobject{currentmarker}{}%
\end{pgfscope}%
\begin{pgfscope}%
\pgfsys@transformshift{2.050019in}{1.979818in}%
\pgfsys@useobject{currentmarker}{}%
\end{pgfscope}%
\begin{pgfscope}%
\pgfsys@transformshift{2.050019in}{1.979818in}%
\pgfsys@useobject{currentmarker}{}%
\end{pgfscope}%
\begin{pgfscope}%
\pgfsys@transformshift{2.050019in}{1.979818in}%
\pgfsys@useobject{currentmarker}{}%
\end{pgfscope}%
\begin{pgfscope}%
\pgfsys@transformshift{2.050019in}{1.979818in}%
\pgfsys@useobject{currentmarker}{}%
\end{pgfscope}%
\begin{pgfscope}%
\pgfsys@transformshift{2.050019in}{1.979818in}%
\pgfsys@useobject{currentmarker}{}%
\end{pgfscope}%
\begin{pgfscope}%
\pgfsys@transformshift{2.050019in}{1.979818in}%
\pgfsys@useobject{currentmarker}{}%
\end{pgfscope}%
\begin{pgfscope}%
\pgfsys@transformshift{2.050019in}{1.979818in}%
\pgfsys@useobject{currentmarker}{}%
\end{pgfscope}%
\begin{pgfscope}%
\pgfsys@transformshift{2.050019in}{1.979818in}%
\pgfsys@useobject{currentmarker}{}%
\end{pgfscope}%
\begin{pgfscope}%
\pgfsys@transformshift{2.050019in}{1.979818in}%
\pgfsys@useobject{currentmarker}{}%
\end{pgfscope}%
\begin{pgfscope}%
\pgfsys@transformshift{2.050019in}{1.979818in}%
\pgfsys@useobject{currentmarker}{}%
\end{pgfscope}%
\begin{pgfscope}%
\pgfsys@transformshift{2.050019in}{1.979818in}%
\pgfsys@useobject{currentmarker}{}%
\end{pgfscope}%
\begin{pgfscope}%
\pgfsys@transformshift{2.050019in}{1.979818in}%
\pgfsys@useobject{currentmarker}{}%
\end{pgfscope}%
\begin{pgfscope}%
\pgfsys@transformshift{2.050019in}{1.979818in}%
\pgfsys@useobject{currentmarker}{}%
\end{pgfscope}%
\begin{pgfscope}%
\pgfsys@transformshift{2.050019in}{1.979818in}%
\pgfsys@useobject{currentmarker}{}%
\end{pgfscope}%
\begin{pgfscope}%
\pgfsys@transformshift{2.050019in}{1.979818in}%
\pgfsys@useobject{currentmarker}{}%
\end{pgfscope}%
\begin{pgfscope}%
\pgfsys@transformshift{2.050019in}{1.979818in}%
\pgfsys@useobject{currentmarker}{}%
\end{pgfscope}%
\begin{pgfscope}%
\pgfsys@transformshift{2.050019in}{1.979818in}%
\pgfsys@useobject{currentmarker}{}%
\end{pgfscope}%
\begin{pgfscope}%
\pgfsys@transformshift{2.050019in}{1.979818in}%
\pgfsys@useobject{currentmarker}{}%
\end{pgfscope}%
\begin{pgfscope}%
\pgfsys@transformshift{2.050019in}{1.979818in}%
\pgfsys@useobject{currentmarker}{}%
\end{pgfscope}%
\begin{pgfscope}%
\pgfsys@transformshift{2.050019in}{1.979818in}%
\pgfsys@useobject{currentmarker}{}%
\end{pgfscope}%
\begin{pgfscope}%
\pgfsys@transformshift{2.050019in}{1.979818in}%
\pgfsys@useobject{currentmarker}{}%
\end{pgfscope}%
\begin{pgfscope}%
\pgfsys@transformshift{2.050019in}{1.979818in}%
\pgfsys@useobject{currentmarker}{}%
\end{pgfscope}%
\begin{pgfscope}%
\pgfsys@transformshift{2.050019in}{1.979818in}%
\pgfsys@useobject{currentmarker}{}%
\end{pgfscope}%
\begin{pgfscope}%
\pgfsys@transformshift{2.050019in}{1.979818in}%
\pgfsys@useobject{currentmarker}{}%
\end{pgfscope}%
\begin{pgfscope}%
\pgfsys@transformshift{2.050019in}{1.979818in}%
\pgfsys@useobject{currentmarker}{}%
\end{pgfscope}%
\begin{pgfscope}%
\pgfsys@transformshift{2.050019in}{1.979818in}%
\pgfsys@useobject{currentmarker}{}%
\end{pgfscope}%
\begin{pgfscope}%
\pgfsys@transformshift{2.050019in}{1.979818in}%
\pgfsys@useobject{currentmarker}{}%
\end{pgfscope}%
\begin{pgfscope}%
\pgfsys@transformshift{2.050019in}{1.979818in}%
\pgfsys@useobject{currentmarker}{}%
\end{pgfscope}%
\begin{pgfscope}%
\pgfsys@transformshift{2.050019in}{1.979818in}%
\pgfsys@useobject{currentmarker}{}%
\end{pgfscope}%
\begin{pgfscope}%
\pgfsys@transformshift{2.050019in}{1.979818in}%
\pgfsys@useobject{currentmarker}{}%
\end{pgfscope}%
\begin{pgfscope}%
\pgfsys@transformshift{2.050019in}{1.979818in}%
\pgfsys@useobject{currentmarker}{}%
\end{pgfscope}%
\begin{pgfscope}%
\pgfsys@transformshift{2.050019in}{1.979818in}%
\pgfsys@useobject{currentmarker}{}%
\end{pgfscope}%
\begin{pgfscope}%
\pgfsys@transformshift{2.050019in}{1.979818in}%
\pgfsys@useobject{currentmarker}{}%
\end{pgfscope}%
\begin{pgfscope}%
\pgfsys@transformshift{2.050019in}{1.979818in}%
\pgfsys@useobject{currentmarker}{}%
\end{pgfscope}%
\begin{pgfscope}%
\pgfsys@transformshift{2.050019in}{1.979818in}%
\pgfsys@useobject{currentmarker}{}%
\end{pgfscope}%
\begin{pgfscope}%
\pgfsys@transformshift{2.050019in}{1.979818in}%
\pgfsys@useobject{currentmarker}{}%
\end{pgfscope}%
\begin{pgfscope}%
\pgfsys@transformshift{2.050019in}{1.979818in}%
\pgfsys@useobject{currentmarker}{}%
\end{pgfscope}%
\begin{pgfscope}%
\pgfsys@transformshift{2.050019in}{1.979818in}%
\pgfsys@useobject{currentmarker}{}%
\end{pgfscope}%
\begin{pgfscope}%
\pgfsys@transformshift{2.050019in}{1.979818in}%
\pgfsys@useobject{currentmarker}{}%
\end{pgfscope}%
\begin{pgfscope}%
\pgfsys@transformshift{2.050019in}{1.979818in}%
\pgfsys@useobject{currentmarker}{}%
\end{pgfscope}%
\begin{pgfscope}%
\pgfsys@transformshift{2.050019in}{1.979818in}%
\pgfsys@useobject{currentmarker}{}%
\end{pgfscope}%
\begin{pgfscope}%
\pgfsys@transformshift{2.050019in}{1.979818in}%
\pgfsys@useobject{currentmarker}{}%
\end{pgfscope}%
\begin{pgfscope}%
\pgfsys@transformshift{2.050019in}{1.979818in}%
\pgfsys@useobject{currentmarker}{}%
\end{pgfscope}%
\begin{pgfscope}%
\pgfsys@transformshift{2.050019in}{1.979818in}%
\pgfsys@useobject{currentmarker}{}%
\end{pgfscope}%
\begin{pgfscope}%
\pgfsys@transformshift{2.050019in}{1.979818in}%
\pgfsys@useobject{currentmarker}{}%
\end{pgfscope}%
\begin{pgfscope}%
\pgfsys@transformshift{2.050019in}{1.979818in}%
\pgfsys@useobject{currentmarker}{}%
\end{pgfscope}%
\begin{pgfscope}%
\pgfsys@transformshift{2.050019in}{1.979818in}%
\pgfsys@useobject{currentmarker}{}%
\end{pgfscope}%
\begin{pgfscope}%
\pgfsys@transformshift{2.050019in}{1.979818in}%
\pgfsys@useobject{currentmarker}{}%
\end{pgfscope}%
\begin{pgfscope}%
\pgfsys@transformshift{2.050019in}{1.979818in}%
\pgfsys@useobject{currentmarker}{}%
\end{pgfscope}%
\begin{pgfscope}%
\pgfsys@transformshift{2.050019in}{1.979818in}%
\pgfsys@useobject{currentmarker}{}%
\end{pgfscope}%
\begin{pgfscope}%
\pgfsys@transformshift{2.050019in}{1.979818in}%
\pgfsys@useobject{currentmarker}{}%
\end{pgfscope}%
\begin{pgfscope}%
\pgfsys@transformshift{2.050019in}{1.979818in}%
\pgfsys@useobject{currentmarker}{}%
\end{pgfscope}%
\begin{pgfscope}%
\pgfsys@transformshift{2.050019in}{1.979818in}%
\pgfsys@useobject{currentmarker}{}%
\end{pgfscope}%
\begin{pgfscope}%
\pgfsys@transformshift{2.050019in}{1.979818in}%
\pgfsys@useobject{currentmarker}{}%
\end{pgfscope}%
\begin{pgfscope}%
\pgfsys@transformshift{2.050019in}{1.979818in}%
\pgfsys@useobject{currentmarker}{}%
\end{pgfscope}%
\begin{pgfscope}%
\pgfsys@transformshift{2.050019in}{1.979818in}%
\pgfsys@useobject{currentmarker}{}%
\end{pgfscope}%
\begin{pgfscope}%
\pgfsys@transformshift{2.050019in}{1.979818in}%
\pgfsys@useobject{currentmarker}{}%
\end{pgfscope}%
\begin{pgfscope}%
\pgfsys@transformshift{2.050019in}{1.979818in}%
\pgfsys@useobject{currentmarker}{}%
\end{pgfscope}%
\begin{pgfscope}%
\pgfsys@transformshift{2.050019in}{1.979818in}%
\pgfsys@useobject{currentmarker}{}%
\end{pgfscope}%
\begin{pgfscope}%
\pgfsys@transformshift{2.050019in}{1.979818in}%
\pgfsys@useobject{currentmarker}{}%
\end{pgfscope}%
\begin{pgfscope}%
\pgfsys@transformshift{2.050019in}{1.979818in}%
\pgfsys@useobject{currentmarker}{}%
\end{pgfscope}%
\begin{pgfscope}%
\pgfsys@transformshift{2.050019in}{1.979818in}%
\pgfsys@useobject{currentmarker}{}%
\end{pgfscope}%
\begin{pgfscope}%
\pgfsys@transformshift{2.050019in}{1.979818in}%
\pgfsys@useobject{currentmarker}{}%
\end{pgfscope}%
\begin{pgfscope}%
\pgfsys@transformshift{2.050019in}{1.979818in}%
\pgfsys@useobject{currentmarker}{}%
\end{pgfscope}%
\begin{pgfscope}%
\pgfsys@transformshift{2.050019in}{1.979818in}%
\pgfsys@useobject{currentmarker}{}%
\end{pgfscope}%
\begin{pgfscope}%
\pgfsys@transformshift{2.050019in}{1.979818in}%
\pgfsys@useobject{currentmarker}{}%
\end{pgfscope}%
\begin{pgfscope}%
\pgfsys@transformshift{2.050019in}{1.979818in}%
\pgfsys@useobject{currentmarker}{}%
\end{pgfscope}%
\begin{pgfscope}%
\pgfsys@transformshift{2.050019in}{1.979818in}%
\pgfsys@useobject{currentmarker}{}%
\end{pgfscope}%
\begin{pgfscope}%
\pgfsys@transformshift{2.050019in}{1.979818in}%
\pgfsys@useobject{currentmarker}{}%
\end{pgfscope}%
\begin{pgfscope}%
\pgfsys@transformshift{2.050019in}{1.979818in}%
\pgfsys@useobject{currentmarker}{}%
\end{pgfscope}%
\begin{pgfscope}%
\pgfsys@transformshift{2.050019in}{1.979818in}%
\pgfsys@useobject{currentmarker}{}%
\end{pgfscope}%
\begin{pgfscope}%
\pgfsys@transformshift{2.050019in}{1.979818in}%
\pgfsys@useobject{currentmarker}{}%
\end{pgfscope}%
\begin{pgfscope}%
\pgfsys@transformshift{2.050019in}{1.979818in}%
\pgfsys@useobject{currentmarker}{}%
\end{pgfscope}%
\begin{pgfscope}%
\pgfsys@transformshift{2.050019in}{1.979818in}%
\pgfsys@useobject{currentmarker}{}%
\end{pgfscope}%
\begin{pgfscope}%
\pgfsys@transformshift{2.050019in}{1.979818in}%
\pgfsys@useobject{currentmarker}{}%
\end{pgfscope}%
\begin{pgfscope}%
\pgfsys@transformshift{2.050019in}{1.979818in}%
\pgfsys@useobject{currentmarker}{}%
\end{pgfscope}%
\begin{pgfscope}%
\pgfsys@transformshift{2.050019in}{1.979818in}%
\pgfsys@useobject{currentmarker}{}%
\end{pgfscope}%
\begin{pgfscope}%
\pgfsys@transformshift{2.050019in}{1.979818in}%
\pgfsys@useobject{currentmarker}{}%
\end{pgfscope}%
\begin{pgfscope}%
\pgfsys@transformshift{2.050019in}{1.979818in}%
\pgfsys@useobject{currentmarker}{}%
\end{pgfscope}%
\begin{pgfscope}%
\pgfsys@transformshift{2.050019in}{1.979818in}%
\pgfsys@useobject{currentmarker}{}%
\end{pgfscope}%
\begin{pgfscope}%
\pgfsys@transformshift{2.050019in}{1.979818in}%
\pgfsys@useobject{currentmarker}{}%
\end{pgfscope}%
\begin{pgfscope}%
\pgfsys@transformshift{2.050019in}{1.979818in}%
\pgfsys@useobject{currentmarker}{}%
\end{pgfscope}%
\begin{pgfscope}%
\pgfsys@transformshift{2.050019in}{1.979818in}%
\pgfsys@useobject{currentmarker}{}%
\end{pgfscope}%
\begin{pgfscope}%
\pgfsys@transformshift{2.050019in}{1.979818in}%
\pgfsys@useobject{currentmarker}{}%
\end{pgfscope}%
\begin{pgfscope}%
\pgfsys@transformshift{2.050019in}{1.979818in}%
\pgfsys@useobject{currentmarker}{}%
\end{pgfscope}%
\begin{pgfscope}%
\pgfsys@transformshift{2.050019in}{1.979818in}%
\pgfsys@useobject{currentmarker}{}%
\end{pgfscope}%
\begin{pgfscope}%
\pgfsys@transformshift{2.050019in}{1.979818in}%
\pgfsys@useobject{currentmarker}{}%
\end{pgfscope}%
\begin{pgfscope}%
\pgfsys@transformshift{2.050019in}{1.979818in}%
\pgfsys@useobject{currentmarker}{}%
\end{pgfscope}%
\begin{pgfscope}%
\pgfsys@transformshift{2.050019in}{1.979818in}%
\pgfsys@useobject{currentmarker}{}%
\end{pgfscope}%
\begin{pgfscope}%
\pgfsys@transformshift{2.050019in}{1.979818in}%
\pgfsys@useobject{currentmarker}{}%
\end{pgfscope}%
\begin{pgfscope}%
\pgfsys@transformshift{2.050019in}{1.979818in}%
\pgfsys@useobject{currentmarker}{}%
\end{pgfscope}%
\begin{pgfscope}%
\pgfsys@transformshift{2.050019in}{1.979818in}%
\pgfsys@useobject{currentmarker}{}%
\end{pgfscope}%
\begin{pgfscope}%
\pgfsys@transformshift{2.050019in}{1.979818in}%
\pgfsys@useobject{currentmarker}{}%
\end{pgfscope}%
\begin{pgfscope}%
\pgfsys@transformshift{2.050019in}{1.979818in}%
\pgfsys@useobject{currentmarker}{}%
\end{pgfscope}%
\begin{pgfscope}%
\pgfsys@transformshift{2.050019in}{1.979818in}%
\pgfsys@useobject{currentmarker}{}%
\end{pgfscope}%
\begin{pgfscope}%
\pgfsys@transformshift{2.050019in}{1.979818in}%
\pgfsys@useobject{currentmarker}{}%
\end{pgfscope}%
\begin{pgfscope}%
\pgfsys@transformshift{2.050019in}{1.979818in}%
\pgfsys@useobject{currentmarker}{}%
\end{pgfscope}%
\begin{pgfscope}%
\pgfsys@transformshift{2.050019in}{1.979818in}%
\pgfsys@useobject{currentmarker}{}%
\end{pgfscope}%
\begin{pgfscope}%
\pgfsys@transformshift{2.050019in}{1.979818in}%
\pgfsys@useobject{currentmarker}{}%
\end{pgfscope}%
\begin{pgfscope}%
\pgfsys@transformshift{2.050019in}{1.979818in}%
\pgfsys@useobject{currentmarker}{}%
\end{pgfscope}%
\begin{pgfscope}%
\pgfsys@transformshift{2.050019in}{1.979818in}%
\pgfsys@useobject{currentmarker}{}%
\end{pgfscope}%
\begin{pgfscope}%
\pgfsys@transformshift{2.050019in}{1.979818in}%
\pgfsys@useobject{currentmarker}{}%
\end{pgfscope}%
\begin{pgfscope}%
\pgfsys@transformshift{2.050019in}{1.979818in}%
\pgfsys@useobject{currentmarker}{}%
\end{pgfscope}%
\begin{pgfscope}%
\pgfsys@transformshift{2.050019in}{1.979818in}%
\pgfsys@useobject{currentmarker}{}%
\end{pgfscope}%
\begin{pgfscope}%
\pgfsys@transformshift{2.050019in}{1.979818in}%
\pgfsys@useobject{currentmarker}{}%
\end{pgfscope}%
\begin{pgfscope}%
\pgfsys@transformshift{2.050019in}{1.979818in}%
\pgfsys@useobject{currentmarker}{}%
\end{pgfscope}%
\begin{pgfscope}%
\pgfsys@transformshift{2.050019in}{1.979818in}%
\pgfsys@useobject{currentmarker}{}%
\end{pgfscope}%
\begin{pgfscope}%
\pgfsys@transformshift{2.050019in}{1.979818in}%
\pgfsys@useobject{currentmarker}{}%
\end{pgfscope}%
\begin{pgfscope}%
\pgfsys@transformshift{2.050019in}{1.979818in}%
\pgfsys@useobject{currentmarker}{}%
\end{pgfscope}%
\begin{pgfscope}%
\pgfsys@transformshift{2.050019in}{1.979818in}%
\pgfsys@useobject{currentmarker}{}%
\end{pgfscope}%
\begin{pgfscope}%
\pgfsys@transformshift{2.050019in}{1.979818in}%
\pgfsys@useobject{currentmarker}{}%
\end{pgfscope}%
\begin{pgfscope}%
\pgfsys@transformshift{2.050019in}{1.979818in}%
\pgfsys@useobject{currentmarker}{}%
\end{pgfscope}%
\begin{pgfscope}%
\pgfsys@transformshift{2.050019in}{1.979818in}%
\pgfsys@useobject{currentmarker}{}%
\end{pgfscope}%
\begin{pgfscope}%
\pgfsys@transformshift{2.050019in}{1.979818in}%
\pgfsys@useobject{currentmarker}{}%
\end{pgfscope}%
\begin{pgfscope}%
\pgfsys@transformshift{2.050019in}{1.979818in}%
\pgfsys@useobject{currentmarker}{}%
\end{pgfscope}%
\begin{pgfscope}%
\pgfsys@transformshift{2.050019in}{1.979818in}%
\pgfsys@useobject{currentmarker}{}%
\end{pgfscope}%
\begin{pgfscope}%
\pgfsys@transformshift{2.050019in}{1.979818in}%
\pgfsys@useobject{currentmarker}{}%
\end{pgfscope}%
\begin{pgfscope}%
\pgfsys@transformshift{2.050019in}{1.979818in}%
\pgfsys@useobject{currentmarker}{}%
\end{pgfscope}%
\begin{pgfscope}%
\pgfsys@transformshift{2.050019in}{1.979818in}%
\pgfsys@useobject{currentmarker}{}%
\end{pgfscope}%
\begin{pgfscope}%
\pgfsys@transformshift{2.050019in}{1.979818in}%
\pgfsys@useobject{currentmarker}{}%
\end{pgfscope}%
\begin{pgfscope}%
\pgfsys@transformshift{2.050019in}{1.979818in}%
\pgfsys@useobject{currentmarker}{}%
\end{pgfscope}%
\begin{pgfscope}%
\pgfsys@transformshift{2.050019in}{1.979818in}%
\pgfsys@useobject{currentmarker}{}%
\end{pgfscope}%
\begin{pgfscope}%
\pgfsys@transformshift{2.050019in}{1.979818in}%
\pgfsys@useobject{currentmarker}{}%
\end{pgfscope}%
\begin{pgfscope}%
\pgfsys@transformshift{2.050019in}{1.979818in}%
\pgfsys@useobject{currentmarker}{}%
\end{pgfscope}%
\begin{pgfscope}%
\pgfsys@transformshift{2.050019in}{1.979818in}%
\pgfsys@useobject{currentmarker}{}%
\end{pgfscope}%
\begin{pgfscope}%
\pgfsys@transformshift{2.050019in}{1.979818in}%
\pgfsys@useobject{currentmarker}{}%
\end{pgfscope}%
\begin{pgfscope}%
\pgfsys@transformshift{2.050019in}{1.979818in}%
\pgfsys@useobject{currentmarker}{}%
\end{pgfscope}%
\begin{pgfscope}%
\pgfsys@transformshift{2.050019in}{1.979818in}%
\pgfsys@useobject{currentmarker}{}%
\end{pgfscope}%
\begin{pgfscope}%
\pgfsys@transformshift{2.050019in}{1.979818in}%
\pgfsys@useobject{currentmarker}{}%
\end{pgfscope}%
\begin{pgfscope}%
\pgfsys@transformshift{2.050019in}{1.979818in}%
\pgfsys@useobject{currentmarker}{}%
\end{pgfscope}%
\begin{pgfscope}%
\pgfsys@transformshift{2.050019in}{1.979818in}%
\pgfsys@useobject{currentmarker}{}%
\end{pgfscope}%
\begin{pgfscope}%
\pgfsys@transformshift{2.050019in}{1.979818in}%
\pgfsys@useobject{currentmarker}{}%
\end{pgfscope}%
\begin{pgfscope}%
\pgfsys@transformshift{2.050019in}{1.979818in}%
\pgfsys@useobject{currentmarker}{}%
\end{pgfscope}%
\begin{pgfscope}%
\pgfsys@transformshift{2.050019in}{1.979818in}%
\pgfsys@useobject{currentmarker}{}%
\end{pgfscope}%
\begin{pgfscope}%
\pgfsys@transformshift{2.050019in}{1.979818in}%
\pgfsys@useobject{currentmarker}{}%
\end{pgfscope}%
\begin{pgfscope}%
\pgfsys@transformshift{2.050019in}{1.979818in}%
\pgfsys@useobject{currentmarker}{}%
\end{pgfscope}%
\begin{pgfscope}%
\pgfsys@transformshift{2.050019in}{1.979818in}%
\pgfsys@useobject{currentmarker}{}%
\end{pgfscope}%
\begin{pgfscope}%
\pgfsys@transformshift{2.050019in}{1.979818in}%
\pgfsys@useobject{currentmarker}{}%
\end{pgfscope}%
\begin{pgfscope}%
\pgfsys@transformshift{2.050019in}{1.979818in}%
\pgfsys@useobject{currentmarker}{}%
\end{pgfscope}%
\begin{pgfscope}%
\pgfsys@transformshift{2.050019in}{1.979818in}%
\pgfsys@useobject{currentmarker}{}%
\end{pgfscope}%
\begin{pgfscope}%
\pgfsys@transformshift{2.050019in}{1.979818in}%
\pgfsys@useobject{currentmarker}{}%
\end{pgfscope}%
\begin{pgfscope}%
\pgfsys@transformshift{2.050019in}{1.979818in}%
\pgfsys@useobject{currentmarker}{}%
\end{pgfscope}%
\begin{pgfscope}%
\pgfsys@transformshift{2.050019in}{1.979818in}%
\pgfsys@useobject{currentmarker}{}%
\end{pgfscope}%
\begin{pgfscope}%
\pgfsys@transformshift{2.050019in}{1.979818in}%
\pgfsys@useobject{currentmarker}{}%
\end{pgfscope}%
\begin{pgfscope}%
\pgfsys@transformshift{2.050019in}{1.979818in}%
\pgfsys@useobject{currentmarker}{}%
\end{pgfscope}%
\begin{pgfscope}%
\pgfsys@transformshift{2.050019in}{1.979818in}%
\pgfsys@useobject{currentmarker}{}%
\end{pgfscope}%
\begin{pgfscope}%
\pgfsys@transformshift{2.050019in}{1.979818in}%
\pgfsys@useobject{currentmarker}{}%
\end{pgfscope}%
\begin{pgfscope}%
\pgfsys@transformshift{2.050019in}{1.979818in}%
\pgfsys@useobject{currentmarker}{}%
\end{pgfscope}%
\begin{pgfscope}%
\pgfsys@transformshift{2.050019in}{1.979818in}%
\pgfsys@useobject{currentmarker}{}%
\end{pgfscope}%
\begin{pgfscope}%
\pgfsys@transformshift{2.050019in}{1.979818in}%
\pgfsys@useobject{currentmarker}{}%
\end{pgfscope}%
\begin{pgfscope}%
\pgfsys@transformshift{2.050019in}{1.979818in}%
\pgfsys@useobject{currentmarker}{}%
\end{pgfscope}%
\begin{pgfscope}%
\pgfsys@transformshift{2.050019in}{1.979818in}%
\pgfsys@useobject{currentmarker}{}%
\end{pgfscope}%
\begin{pgfscope}%
\pgfsys@transformshift{2.050019in}{1.979818in}%
\pgfsys@useobject{currentmarker}{}%
\end{pgfscope}%
\begin{pgfscope}%
\pgfsys@transformshift{2.050019in}{1.979818in}%
\pgfsys@useobject{currentmarker}{}%
\end{pgfscope}%
\begin{pgfscope}%
\pgfsys@transformshift{2.050019in}{1.979818in}%
\pgfsys@useobject{currentmarker}{}%
\end{pgfscope}%
\begin{pgfscope}%
\pgfsys@transformshift{2.050019in}{1.979818in}%
\pgfsys@useobject{currentmarker}{}%
\end{pgfscope}%
\begin{pgfscope}%
\pgfsys@transformshift{2.050019in}{1.979818in}%
\pgfsys@useobject{currentmarker}{}%
\end{pgfscope}%
\begin{pgfscope}%
\pgfsys@transformshift{2.050019in}{1.979818in}%
\pgfsys@useobject{currentmarker}{}%
\end{pgfscope}%
\begin{pgfscope}%
\pgfsys@transformshift{2.050019in}{1.979818in}%
\pgfsys@useobject{currentmarker}{}%
\end{pgfscope}%
\begin{pgfscope}%
\pgfsys@transformshift{2.050019in}{1.979818in}%
\pgfsys@useobject{currentmarker}{}%
\end{pgfscope}%
\begin{pgfscope}%
\pgfsys@transformshift{2.050019in}{1.979818in}%
\pgfsys@useobject{currentmarker}{}%
\end{pgfscope}%
\begin{pgfscope}%
\pgfsys@transformshift{2.050019in}{1.979818in}%
\pgfsys@useobject{currentmarker}{}%
\end{pgfscope}%
\begin{pgfscope}%
\pgfsys@transformshift{2.050019in}{1.979818in}%
\pgfsys@useobject{currentmarker}{}%
\end{pgfscope}%
\begin{pgfscope}%
\pgfsys@transformshift{2.050019in}{1.979818in}%
\pgfsys@useobject{currentmarker}{}%
\end{pgfscope}%
\begin{pgfscope}%
\pgfsys@transformshift{2.050019in}{1.979818in}%
\pgfsys@useobject{currentmarker}{}%
\end{pgfscope}%
\begin{pgfscope}%
\pgfsys@transformshift{2.050019in}{1.979818in}%
\pgfsys@useobject{currentmarker}{}%
\end{pgfscope}%
\begin{pgfscope}%
\pgfsys@transformshift{2.050019in}{1.979818in}%
\pgfsys@useobject{currentmarker}{}%
\end{pgfscope}%
\begin{pgfscope}%
\pgfsys@transformshift{2.050019in}{1.979818in}%
\pgfsys@useobject{currentmarker}{}%
\end{pgfscope}%
\begin{pgfscope}%
\pgfsys@transformshift{2.050019in}{1.979818in}%
\pgfsys@useobject{currentmarker}{}%
\end{pgfscope}%
\begin{pgfscope}%
\pgfsys@transformshift{2.050019in}{1.979818in}%
\pgfsys@useobject{currentmarker}{}%
\end{pgfscope}%
\begin{pgfscope}%
\pgfsys@transformshift{2.050019in}{1.979818in}%
\pgfsys@useobject{currentmarker}{}%
\end{pgfscope}%
\begin{pgfscope}%
\pgfsys@transformshift{2.050019in}{1.979818in}%
\pgfsys@useobject{currentmarker}{}%
\end{pgfscope}%
\begin{pgfscope}%
\pgfsys@transformshift{2.050019in}{1.979818in}%
\pgfsys@useobject{currentmarker}{}%
\end{pgfscope}%
\begin{pgfscope}%
\pgfsys@transformshift{2.050019in}{1.979818in}%
\pgfsys@useobject{currentmarker}{}%
\end{pgfscope}%
\begin{pgfscope}%
\pgfsys@transformshift{2.050019in}{1.979818in}%
\pgfsys@useobject{currentmarker}{}%
\end{pgfscope}%
\begin{pgfscope}%
\pgfsys@transformshift{2.050019in}{1.979818in}%
\pgfsys@useobject{currentmarker}{}%
\end{pgfscope}%
\begin{pgfscope}%
\pgfsys@transformshift{2.050019in}{1.979818in}%
\pgfsys@useobject{currentmarker}{}%
\end{pgfscope}%
\begin{pgfscope}%
\pgfsys@transformshift{2.050019in}{1.979818in}%
\pgfsys@useobject{currentmarker}{}%
\end{pgfscope}%
\begin{pgfscope}%
\pgfsys@transformshift{2.050019in}{1.979818in}%
\pgfsys@useobject{currentmarker}{}%
\end{pgfscope}%
\begin{pgfscope}%
\pgfsys@transformshift{2.050019in}{1.979818in}%
\pgfsys@useobject{currentmarker}{}%
\end{pgfscope}%
\begin{pgfscope}%
\pgfsys@transformshift{2.050019in}{1.979818in}%
\pgfsys@useobject{currentmarker}{}%
\end{pgfscope}%
\begin{pgfscope}%
\pgfsys@transformshift{2.050019in}{1.979818in}%
\pgfsys@useobject{currentmarker}{}%
\end{pgfscope}%
\begin{pgfscope}%
\pgfsys@transformshift{2.050019in}{1.979818in}%
\pgfsys@useobject{currentmarker}{}%
\end{pgfscope}%
\begin{pgfscope}%
\pgfsys@transformshift{2.050019in}{1.979818in}%
\pgfsys@useobject{currentmarker}{}%
\end{pgfscope}%
\begin{pgfscope}%
\pgfsys@transformshift{2.050019in}{1.979818in}%
\pgfsys@useobject{currentmarker}{}%
\end{pgfscope}%
\begin{pgfscope}%
\pgfsys@transformshift{2.050019in}{1.979818in}%
\pgfsys@useobject{currentmarker}{}%
\end{pgfscope}%
\begin{pgfscope}%
\pgfsys@transformshift{2.050019in}{1.979818in}%
\pgfsys@useobject{currentmarker}{}%
\end{pgfscope}%
\begin{pgfscope}%
\pgfsys@transformshift{2.050019in}{1.979818in}%
\pgfsys@useobject{currentmarker}{}%
\end{pgfscope}%
\begin{pgfscope}%
\pgfsys@transformshift{2.050019in}{1.979818in}%
\pgfsys@useobject{currentmarker}{}%
\end{pgfscope}%
\begin{pgfscope}%
\pgfsys@transformshift{2.050019in}{1.979818in}%
\pgfsys@useobject{currentmarker}{}%
\end{pgfscope}%
\begin{pgfscope}%
\pgfsys@transformshift{2.050019in}{1.979818in}%
\pgfsys@useobject{currentmarker}{}%
\end{pgfscope}%
\begin{pgfscope}%
\pgfsys@transformshift{2.050019in}{1.979818in}%
\pgfsys@useobject{currentmarker}{}%
\end{pgfscope}%
\begin{pgfscope}%
\pgfsys@transformshift{2.050019in}{1.979818in}%
\pgfsys@useobject{currentmarker}{}%
\end{pgfscope}%
\begin{pgfscope}%
\pgfsys@transformshift{2.050019in}{1.979818in}%
\pgfsys@useobject{currentmarker}{}%
\end{pgfscope}%
\begin{pgfscope}%
\pgfsys@transformshift{2.050019in}{1.979818in}%
\pgfsys@useobject{currentmarker}{}%
\end{pgfscope}%
\begin{pgfscope}%
\pgfsys@transformshift{2.050019in}{1.979818in}%
\pgfsys@useobject{currentmarker}{}%
\end{pgfscope}%
\begin{pgfscope}%
\pgfsys@transformshift{2.050019in}{1.979818in}%
\pgfsys@useobject{currentmarker}{}%
\end{pgfscope}%
\begin{pgfscope}%
\pgfsys@transformshift{2.050019in}{1.979818in}%
\pgfsys@useobject{currentmarker}{}%
\end{pgfscope}%
\begin{pgfscope}%
\pgfsys@transformshift{2.050019in}{1.979818in}%
\pgfsys@useobject{currentmarker}{}%
\end{pgfscope}%
\begin{pgfscope}%
\pgfsys@transformshift{2.050019in}{1.979818in}%
\pgfsys@useobject{currentmarker}{}%
\end{pgfscope}%
\begin{pgfscope}%
\pgfsys@transformshift{2.050019in}{1.979818in}%
\pgfsys@useobject{currentmarker}{}%
\end{pgfscope}%
\begin{pgfscope}%
\pgfsys@transformshift{2.050019in}{1.979818in}%
\pgfsys@useobject{currentmarker}{}%
\end{pgfscope}%
\begin{pgfscope}%
\pgfsys@transformshift{2.050019in}{1.979818in}%
\pgfsys@useobject{currentmarker}{}%
\end{pgfscope}%
\begin{pgfscope}%
\pgfsys@transformshift{2.050019in}{1.979818in}%
\pgfsys@useobject{currentmarker}{}%
\end{pgfscope}%
\begin{pgfscope}%
\pgfsys@transformshift{2.050019in}{1.979818in}%
\pgfsys@useobject{currentmarker}{}%
\end{pgfscope}%
\begin{pgfscope}%
\pgfsys@transformshift{2.050019in}{1.979818in}%
\pgfsys@useobject{currentmarker}{}%
\end{pgfscope}%
\begin{pgfscope}%
\pgfsys@transformshift{2.050019in}{1.979818in}%
\pgfsys@useobject{currentmarker}{}%
\end{pgfscope}%
\begin{pgfscope}%
\pgfsys@transformshift{2.050019in}{1.979818in}%
\pgfsys@useobject{currentmarker}{}%
\end{pgfscope}%
\begin{pgfscope}%
\pgfsys@transformshift{2.050019in}{1.979818in}%
\pgfsys@useobject{currentmarker}{}%
\end{pgfscope}%
\begin{pgfscope}%
\pgfsys@transformshift{2.050019in}{1.979818in}%
\pgfsys@useobject{currentmarker}{}%
\end{pgfscope}%
\begin{pgfscope}%
\pgfsys@transformshift{2.050019in}{1.979818in}%
\pgfsys@useobject{currentmarker}{}%
\end{pgfscope}%
\begin{pgfscope}%
\pgfsys@transformshift{2.050019in}{1.979818in}%
\pgfsys@useobject{currentmarker}{}%
\end{pgfscope}%
\begin{pgfscope}%
\pgfsys@transformshift{2.050019in}{1.979818in}%
\pgfsys@useobject{currentmarker}{}%
\end{pgfscope}%
\begin{pgfscope}%
\pgfsys@transformshift{2.050019in}{1.979818in}%
\pgfsys@useobject{currentmarker}{}%
\end{pgfscope}%
\begin{pgfscope}%
\pgfsys@transformshift{2.050019in}{1.979818in}%
\pgfsys@useobject{currentmarker}{}%
\end{pgfscope}%
\begin{pgfscope}%
\pgfsys@transformshift{2.050019in}{1.979818in}%
\pgfsys@useobject{currentmarker}{}%
\end{pgfscope}%
\begin{pgfscope}%
\pgfsys@transformshift{2.050019in}{1.979818in}%
\pgfsys@useobject{currentmarker}{}%
\end{pgfscope}%
\begin{pgfscope}%
\pgfsys@transformshift{2.050019in}{1.979818in}%
\pgfsys@useobject{currentmarker}{}%
\end{pgfscope}%
\begin{pgfscope}%
\pgfsys@transformshift{2.050019in}{1.979818in}%
\pgfsys@useobject{currentmarker}{}%
\end{pgfscope}%
\begin{pgfscope}%
\pgfsys@transformshift{2.050019in}{1.979818in}%
\pgfsys@useobject{currentmarker}{}%
\end{pgfscope}%
\begin{pgfscope}%
\pgfsys@transformshift{2.050019in}{1.979818in}%
\pgfsys@useobject{currentmarker}{}%
\end{pgfscope}%
\begin{pgfscope}%
\pgfsys@transformshift{2.050019in}{1.979818in}%
\pgfsys@useobject{currentmarker}{}%
\end{pgfscope}%
\begin{pgfscope}%
\pgfsys@transformshift{2.050019in}{1.979818in}%
\pgfsys@useobject{currentmarker}{}%
\end{pgfscope}%
\begin{pgfscope}%
\pgfsys@transformshift{2.050019in}{1.979818in}%
\pgfsys@useobject{currentmarker}{}%
\end{pgfscope}%
\begin{pgfscope}%
\pgfsys@transformshift{2.050019in}{1.979818in}%
\pgfsys@useobject{currentmarker}{}%
\end{pgfscope}%
\begin{pgfscope}%
\pgfsys@transformshift{2.050019in}{1.979818in}%
\pgfsys@useobject{currentmarker}{}%
\end{pgfscope}%
\begin{pgfscope}%
\pgfsys@transformshift{2.050019in}{1.979818in}%
\pgfsys@useobject{currentmarker}{}%
\end{pgfscope}%
\begin{pgfscope}%
\pgfsys@transformshift{2.050019in}{1.979818in}%
\pgfsys@useobject{currentmarker}{}%
\end{pgfscope}%
\begin{pgfscope}%
\pgfsys@transformshift{2.050019in}{1.979818in}%
\pgfsys@useobject{currentmarker}{}%
\end{pgfscope}%
\begin{pgfscope}%
\pgfsys@transformshift{2.050019in}{1.979818in}%
\pgfsys@useobject{currentmarker}{}%
\end{pgfscope}%
\begin{pgfscope}%
\pgfsys@transformshift{2.050019in}{1.979818in}%
\pgfsys@useobject{currentmarker}{}%
\end{pgfscope}%
\begin{pgfscope}%
\pgfsys@transformshift{2.050019in}{1.979818in}%
\pgfsys@useobject{currentmarker}{}%
\end{pgfscope}%
\begin{pgfscope}%
\pgfsys@transformshift{2.050019in}{1.979818in}%
\pgfsys@useobject{currentmarker}{}%
\end{pgfscope}%
\begin{pgfscope}%
\pgfsys@transformshift{2.050019in}{1.979818in}%
\pgfsys@useobject{currentmarker}{}%
\end{pgfscope}%
\begin{pgfscope}%
\pgfsys@transformshift{2.050019in}{1.979818in}%
\pgfsys@useobject{currentmarker}{}%
\end{pgfscope}%
\begin{pgfscope}%
\pgfsys@transformshift{2.050019in}{1.979818in}%
\pgfsys@useobject{currentmarker}{}%
\end{pgfscope}%
\begin{pgfscope}%
\pgfsys@transformshift{2.050019in}{1.979818in}%
\pgfsys@useobject{currentmarker}{}%
\end{pgfscope}%
\begin{pgfscope}%
\pgfsys@transformshift{2.050019in}{1.979818in}%
\pgfsys@useobject{currentmarker}{}%
\end{pgfscope}%
\begin{pgfscope}%
\pgfsys@transformshift{2.050019in}{1.979818in}%
\pgfsys@useobject{currentmarker}{}%
\end{pgfscope}%
\begin{pgfscope}%
\pgfsys@transformshift{2.050019in}{1.979818in}%
\pgfsys@useobject{currentmarker}{}%
\end{pgfscope}%
\begin{pgfscope}%
\pgfsys@transformshift{2.050019in}{1.979818in}%
\pgfsys@useobject{currentmarker}{}%
\end{pgfscope}%
\begin{pgfscope}%
\pgfsys@transformshift{2.050019in}{1.979818in}%
\pgfsys@useobject{currentmarker}{}%
\end{pgfscope}%
\begin{pgfscope}%
\pgfsys@transformshift{2.050019in}{1.979818in}%
\pgfsys@useobject{currentmarker}{}%
\end{pgfscope}%
\begin{pgfscope}%
\pgfsys@transformshift{2.050019in}{1.979818in}%
\pgfsys@useobject{currentmarker}{}%
\end{pgfscope}%
\begin{pgfscope}%
\pgfsys@transformshift{2.050019in}{1.979818in}%
\pgfsys@useobject{currentmarker}{}%
\end{pgfscope}%
\begin{pgfscope}%
\pgfsys@transformshift{2.050019in}{1.979818in}%
\pgfsys@useobject{currentmarker}{}%
\end{pgfscope}%
\begin{pgfscope}%
\pgfsys@transformshift{2.050019in}{1.979818in}%
\pgfsys@useobject{currentmarker}{}%
\end{pgfscope}%
\begin{pgfscope}%
\pgfsys@transformshift{2.050019in}{1.979818in}%
\pgfsys@useobject{currentmarker}{}%
\end{pgfscope}%
\begin{pgfscope}%
\pgfsys@transformshift{2.050019in}{1.979818in}%
\pgfsys@useobject{currentmarker}{}%
\end{pgfscope}%
\begin{pgfscope}%
\pgfsys@transformshift{2.050019in}{1.979818in}%
\pgfsys@useobject{currentmarker}{}%
\end{pgfscope}%
\begin{pgfscope}%
\pgfsys@transformshift{2.050019in}{1.979818in}%
\pgfsys@useobject{currentmarker}{}%
\end{pgfscope}%
\begin{pgfscope}%
\pgfsys@transformshift{2.050019in}{1.979818in}%
\pgfsys@useobject{currentmarker}{}%
\end{pgfscope}%
\begin{pgfscope}%
\pgfsys@transformshift{2.050019in}{1.979818in}%
\pgfsys@useobject{currentmarker}{}%
\end{pgfscope}%
\begin{pgfscope}%
\pgfsys@transformshift{2.050019in}{1.979818in}%
\pgfsys@useobject{currentmarker}{}%
\end{pgfscope}%
\begin{pgfscope}%
\pgfsys@transformshift{2.050019in}{1.979818in}%
\pgfsys@useobject{currentmarker}{}%
\end{pgfscope}%
\begin{pgfscope}%
\pgfsys@transformshift{2.050019in}{1.979818in}%
\pgfsys@useobject{currentmarker}{}%
\end{pgfscope}%
\begin{pgfscope}%
\pgfsys@transformshift{2.050019in}{1.979818in}%
\pgfsys@useobject{currentmarker}{}%
\end{pgfscope}%
\begin{pgfscope}%
\pgfsys@transformshift{2.050019in}{1.979818in}%
\pgfsys@useobject{currentmarker}{}%
\end{pgfscope}%
\begin{pgfscope}%
\pgfsys@transformshift{2.050019in}{1.979818in}%
\pgfsys@useobject{currentmarker}{}%
\end{pgfscope}%
\begin{pgfscope}%
\pgfsys@transformshift{2.050019in}{1.979818in}%
\pgfsys@useobject{currentmarker}{}%
\end{pgfscope}%
\begin{pgfscope}%
\pgfsys@transformshift{2.050019in}{1.979818in}%
\pgfsys@useobject{currentmarker}{}%
\end{pgfscope}%
\begin{pgfscope}%
\pgfsys@transformshift{2.050019in}{1.979818in}%
\pgfsys@useobject{currentmarker}{}%
\end{pgfscope}%
\begin{pgfscope}%
\pgfsys@transformshift{2.050019in}{1.979818in}%
\pgfsys@useobject{currentmarker}{}%
\end{pgfscope}%
\begin{pgfscope}%
\pgfsys@transformshift{2.050019in}{1.979818in}%
\pgfsys@useobject{currentmarker}{}%
\end{pgfscope}%
\begin{pgfscope}%
\pgfsys@transformshift{2.050019in}{1.979818in}%
\pgfsys@useobject{currentmarker}{}%
\end{pgfscope}%
\begin{pgfscope}%
\pgfsys@transformshift{2.050019in}{1.979818in}%
\pgfsys@useobject{currentmarker}{}%
\end{pgfscope}%
\begin{pgfscope}%
\pgfsys@transformshift{2.050019in}{1.979818in}%
\pgfsys@useobject{currentmarker}{}%
\end{pgfscope}%
\begin{pgfscope}%
\pgfsys@transformshift{2.050019in}{1.979818in}%
\pgfsys@useobject{currentmarker}{}%
\end{pgfscope}%
\begin{pgfscope}%
\pgfsys@transformshift{2.050019in}{1.979818in}%
\pgfsys@useobject{currentmarker}{}%
\end{pgfscope}%
\begin{pgfscope}%
\pgfsys@transformshift{2.050019in}{1.979818in}%
\pgfsys@useobject{currentmarker}{}%
\end{pgfscope}%
\begin{pgfscope}%
\pgfsys@transformshift{2.050019in}{1.979818in}%
\pgfsys@useobject{currentmarker}{}%
\end{pgfscope}%
\begin{pgfscope}%
\pgfsys@transformshift{2.050019in}{1.979818in}%
\pgfsys@useobject{currentmarker}{}%
\end{pgfscope}%
\begin{pgfscope}%
\pgfsys@transformshift{2.050019in}{1.979818in}%
\pgfsys@useobject{currentmarker}{}%
\end{pgfscope}%
\begin{pgfscope}%
\pgfsys@transformshift{2.050019in}{1.979818in}%
\pgfsys@useobject{currentmarker}{}%
\end{pgfscope}%
\begin{pgfscope}%
\pgfsys@transformshift{2.050019in}{1.979818in}%
\pgfsys@useobject{currentmarker}{}%
\end{pgfscope}%
\begin{pgfscope}%
\pgfsys@transformshift{2.050019in}{1.979818in}%
\pgfsys@useobject{currentmarker}{}%
\end{pgfscope}%
\begin{pgfscope}%
\pgfsys@transformshift{2.050019in}{1.979818in}%
\pgfsys@useobject{currentmarker}{}%
\end{pgfscope}%
\begin{pgfscope}%
\pgfsys@transformshift{2.050019in}{1.979818in}%
\pgfsys@useobject{currentmarker}{}%
\end{pgfscope}%
\begin{pgfscope}%
\pgfsys@transformshift{2.050019in}{1.979818in}%
\pgfsys@useobject{currentmarker}{}%
\end{pgfscope}%
\begin{pgfscope}%
\pgfsys@transformshift{2.050019in}{1.979818in}%
\pgfsys@useobject{currentmarker}{}%
\end{pgfscope}%
\begin{pgfscope}%
\pgfsys@transformshift{2.050019in}{1.979818in}%
\pgfsys@useobject{currentmarker}{}%
\end{pgfscope}%
\begin{pgfscope}%
\pgfsys@transformshift{2.050019in}{1.979818in}%
\pgfsys@useobject{currentmarker}{}%
\end{pgfscope}%
\begin{pgfscope}%
\pgfsys@transformshift{2.050019in}{1.979818in}%
\pgfsys@useobject{currentmarker}{}%
\end{pgfscope}%
\begin{pgfscope}%
\pgfsys@transformshift{2.050019in}{1.979818in}%
\pgfsys@useobject{currentmarker}{}%
\end{pgfscope}%
\begin{pgfscope}%
\pgfsys@transformshift{2.050019in}{1.979818in}%
\pgfsys@useobject{currentmarker}{}%
\end{pgfscope}%
\begin{pgfscope}%
\pgfsys@transformshift{2.050019in}{1.979818in}%
\pgfsys@useobject{currentmarker}{}%
\end{pgfscope}%
\begin{pgfscope}%
\pgfsys@transformshift{2.050019in}{1.979818in}%
\pgfsys@useobject{currentmarker}{}%
\end{pgfscope}%
\begin{pgfscope}%
\pgfsys@transformshift{2.050019in}{1.979818in}%
\pgfsys@useobject{currentmarker}{}%
\end{pgfscope}%
\begin{pgfscope}%
\pgfsys@transformshift{2.050019in}{1.979818in}%
\pgfsys@useobject{currentmarker}{}%
\end{pgfscope}%
\begin{pgfscope}%
\pgfsys@transformshift{2.050019in}{1.979818in}%
\pgfsys@useobject{currentmarker}{}%
\end{pgfscope}%
\begin{pgfscope}%
\pgfsys@transformshift{2.050019in}{1.979818in}%
\pgfsys@useobject{currentmarker}{}%
\end{pgfscope}%
\begin{pgfscope}%
\pgfsys@transformshift{2.050019in}{1.979818in}%
\pgfsys@useobject{currentmarker}{}%
\end{pgfscope}%
\begin{pgfscope}%
\pgfsys@transformshift{2.050019in}{1.979818in}%
\pgfsys@useobject{currentmarker}{}%
\end{pgfscope}%
\begin{pgfscope}%
\pgfsys@transformshift{2.050019in}{1.979818in}%
\pgfsys@useobject{currentmarker}{}%
\end{pgfscope}%
\begin{pgfscope}%
\pgfsys@transformshift{2.050019in}{1.979818in}%
\pgfsys@useobject{currentmarker}{}%
\end{pgfscope}%
\begin{pgfscope}%
\pgfsys@transformshift{2.050019in}{1.979818in}%
\pgfsys@useobject{currentmarker}{}%
\end{pgfscope}%
\begin{pgfscope}%
\pgfsys@transformshift{2.050019in}{1.979818in}%
\pgfsys@useobject{currentmarker}{}%
\end{pgfscope}%
\begin{pgfscope}%
\pgfsys@transformshift{2.050019in}{1.979818in}%
\pgfsys@useobject{currentmarker}{}%
\end{pgfscope}%
\begin{pgfscope}%
\pgfsys@transformshift{2.050019in}{1.979818in}%
\pgfsys@useobject{currentmarker}{}%
\end{pgfscope}%
\begin{pgfscope}%
\pgfsys@transformshift{2.050019in}{1.979818in}%
\pgfsys@useobject{currentmarker}{}%
\end{pgfscope}%
\begin{pgfscope}%
\pgfsys@transformshift{2.050019in}{1.979818in}%
\pgfsys@useobject{currentmarker}{}%
\end{pgfscope}%
\begin{pgfscope}%
\pgfsys@transformshift{2.050019in}{1.979818in}%
\pgfsys@useobject{currentmarker}{}%
\end{pgfscope}%
\begin{pgfscope}%
\pgfsys@transformshift{2.050019in}{1.979818in}%
\pgfsys@useobject{currentmarker}{}%
\end{pgfscope}%
\begin{pgfscope}%
\pgfsys@transformshift{2.050019in}{1.979818in}%
\pgfsys@useobject{currentmarker}{}%
\end{pgfscope}%
\begin{pgfscope}%
\pgfsys@transformshift{2.050019in}{1.979818in}%
\pgfsys@useobject{currentmarker}{}%
\end{pgfscope}%
\begin{pgfscope}%
\pgfsys@transformshift{2.050019in}{1.979818in}%
\pgfsys@useobject{currentmarker}{}%
\end{pgfscope}%
\begin{pgfscope}%
\pgfsys@transformshift{2.050019in}{1.979818in}%
\pgfsys@useobject{currentmarker}{}%
\end{pgfscope}%
\begin{pgfscope}%
\pgfsys@transformshift{2.050019in}{1.979818in}%
\pgfsys@useobject{currentmarker}{}%
\end{pgfscope}%
\begin{pgfscope}%
\pgfsys@transformshift{2.050019in}{1.979818in}%
\pgfsys@useobject{currentmarker}{}%
\end{pgfscope}%
\begin{pgfscope}%
\pgfsys@transformshift{2.050019in}{1.979818in}%
\pgfsys@useobject{currentmarker}{}%
\end{pgfscope}%
\begin{pgfscope}%
\pgfsys@transformshift{2.050019in}{1.979818in}%
\pgfsys@useobject{currentmarker}{}%
\end{pgfscope}%
\begin{pgfscope}%
\pgfsys@transformshift{2.050019in}{1.979818in}%
\pgfsys@useobject{currentmarker}{}%
\end{pgfscope}%
\begin{pgfscope}%
\pgfsys@transformshift{2.050019in}{1.979818in}%
\pgfsys@useobject{currentmarker}{}%
\end{pgfscope}%
\begin{pgfscope}%
\pgfsys@transformshift{2.050019in}{1.979818in}%
\pgfsys@useobject{currentmarker}{}%
\end{pgfscope}%
\begin{pgfscope}%
\pgfsys@transformshift{2.050019in}{1.979818in}%
\pgfsys@useobject{currentmarker}{}%
\end{pgfscope}%
\begin{pgfscope}%
\pgfsys@transformshift{2.050019in}{1.979818in}%
\pgfsys@useobject{currentmarker}{}%
\end{pgfscope}%
\begin{pgfscope}%
\pgfsys@transformshift{2.050019in}{1.979818in}%
\pgfsys@useobject{currentmarker}{}%
\end{pgfscope}%
\begin{pgfscope}%
\pgfsys@transformshift{2.050019in}{1.979818in}%
\pgfsys@useobject{currentmarker}{}%
\end{pgfscope}%
\begin{pgfscope}%
\pgfsys@transformshift{2.050019in}{1.979818in}%
\pgfsys@useobject{currentmarker}{}%
\end{pgfscope}%
\begin{pgfscope}%
\pgfsys@transformshift{2.050019in}{1.979818in}%
\pgfsys@useobject{currentmarker}{}%
\end{pgfscope}%
\begin{pgfscope}%
\pgfsys@transformshift{2.050019in}{1.979818in}%
\pgfsys@useobject{currentmarker}{}%
\end{pgfscope}%
\begin{pgfscope}%
\pgfsys@transformshift{2.050019in}{1.979818in}%
\pgfsys@useobject{currentmarker}{}%
\end{pgfscope}%
\begin{pgfscope}%
\pgfsys@transformshift{2.050019in}{1.979818in}%
\pgfsys@useobject{currentmarker}{}%
\end{pgfscope}%
\begin{pgfscope}%
\pgfsys@transformshift{2.050019in}{1.979818in}%
\pgfsys@useobject{currentmarker}{}%
\end{pgfscope}%
\begin{pgfscope}%
\pgfsys@transformshift{2.050019in}{1.979818in}%
\pgfsys@useobject{currentmarker}{}%
\end{pgfscope}%
\begin{pgfscope}%
\pgfsys@transformshift{2.050019in}{1.979818in}%
\pgfsys@useobject{currentmarker}{}%
\end{pgfscope}%
\begin{pgfscope}%
\pgfsys@transformshift{2.050019in}{1.979818in}%
\pgfsys@useobject{currentmarker}{}%
\end{pgfscope}%
\begin{pgfscope}%
\pgfsys@transformshift{2.050019in}{1.979818in}%
\pgfsys@useobject{currentmarker}{}%
\end{pgfscope}%
\begin{pgfscope}%
\pgfsys@transformshift{2.050019in}{1.979818in}%
\pgfsys@useobject{currentmarker}{}%
\end{pgfscope}%
\begin{pgfscope}%
\pgfsys@transformshift{2.050019in}{1.979818in}%
\pgfsys@useobject{currentmarker}{}%
\end{pgfscope}%
\begin{pgfscope}%
\pgfsys@transformshift{2.050019in}{1.979818in}%
\pgfsys@useobject{currentmarker}{}%
\end{pgfscope}%
\begin{pgfscope}%
\pgfsys@transformshift{2.050019in}{1.979818in}%
\pgfsys@useobject{currentmarker}{}%
\end{pgfscope}%
\begin{pgfscope}%
\pgfsys@transformshift{2.050019in}{1.979818in}%
\pgfsys@useobject{currentmarker}{}%
\end{pgfscope}%
\begin{pgfscope}%
\pgfsys@transformshift{2.050019in}{1.979818in}%
\pgfsys@useobject{currentmarker}{}%
\end{pgfscope}%
\begin{pgfscope}%
\pgfsys@transformshift{2.050019in}{1.979818in}%
\pgfsys@useobject{currentmarker}{}%
\end{pgfscope}%
\begin{pgfscope}%
\pgfsys@transformshift{2.050019in}{1.979818in}%
\pgfsys@useobject{currentmarker}{}%
\end{pgfscope}%
\begin{pgfscope}%
\pgfsys@transformshift{2.050019in}{1.979818in}%
\pgfsys@useobject{currentmarker}{}%
\end{pgfscope}%
\begin{pgfscope}%
\pgfsys@transformshift{2.050019in}{1.979818in}%
\pgfsys@useobject{currentmarker}{}%
\end{pgfscope}%
\begin{pgfscope}%
\pgfsys@transformshift{2.050019in}{1.979818in}%
\pgfsys@useobject{currentmarker}{}%
\end{pgfscope}%
\begin{pgfscope}%
\pgfsys@transformshift{2.050019in}{1.979818in}%
\pgfsys@useobject{currentmarker}{}%
\end{pgfscope}%
\begin{pgfscope}%
\pgfsys@transformshift{2.050019in}{1.979818in}%
\pgfsys@useobject{currentmarker}{}%
\end{pgfscope}%
\begin{pgfscope}%
\pgfsys@transformshift{2.050019in}{1.979818in}%
\pgfsys@useobject{currentmarker}{}%
\end{pgfscope}%
\begin{pgfscope}%
\pgfsys@transformshift{2.050019in}{1.979818in}%
\pgfsys@useobject{currentmarker}{}%
\end{pgfscope}%
\begin{pgfscope}%
\pgfsys@transformshift{2.050019in}{1.979818in}%
\pgfsys@useobject{currentmarker}{}%
\end{pgfscope}%
\begin{pgfscope}%
\pgfsys@transformshift{2.050019in}{1.979818in}%
\pgfsys@useobject{currentmarker}{}%
\end{pgfscope}%
\begin{pgfscope}%
\pgfsys@transformshift{2.050019in}{1.979818in}%
\pgfsys@useobject{currentmarker}{}%
\end{pgfscope}%
\begin{pgfscope}%
\pgfsys@transformshift{2.050019in}{1.979818in}%
\pgfsys@useobject{currentmarker}{}%
\end{pgfscope}%
\begin{pgfscope}%
\pgfsys@transformshift{2.050019in}{1.979818in}%
\pgfsys@useobject{currentmarker}{}%
\end{pgfscope}%
\begin{pgfscope}%
\pgfsys@transformshift{2.050019in}{1.979818in}%
\pgfsys@useobject{currentmarker}{}%
\end{pgfscope}%
\begin{pgfscope}%
\pgfsys@transformshift{2.050019in}{1.979818in}%
\pgfsys@useobject{currentmarker}{}%
\end{pgfscope}%
\begin{pgfscope}%
\pgfsys@transformshift{2.050019in}{1.979818in}%
\pgfsys@useobject{currentmarker}{}%
\end{pgfscope}%
\begin{pgfscope}%
\pgfsys@transformshift{2.050019in}{1.979818in}%
\pgfsys@useobject{currentmarker}{}%
\end{pgfscope}%
\begin{pgfscope}%
\pgfsys@transformshift{2.050019in}{1.979818in}%
\pgfsys@useobject{currentmarker}{}%
\end{pgfscope}%
\begin{pgfscope}%
\pgfsys@transformshift{2.050019in}{1.979818in}%
\pgfsys@useobject{currentmarker}{}%
\end{pgfscope}%
\begin{pgfscope}%
\pgfsys@transformshift{2.050019in}{1.979818in}%
\pgfsys@useobject{currentmarker}{}%
\end{pgfscope}%
\begin{pgfscope}%
\pgfsys@transformshift{2.050019in}{1.979818in}%
\pgfsys@useobject{currentmarker}{}%
\end{pgfscope}%
\begin{pgfscope}%
\pgfsys@transformshift{2.050019in}{1.979818in}%
\pgfsys@useobject{currentmarker}{}%
\end{pgfscope}%
\begin{pgfscope}%
\pgfsys@transformshift{2.050019in}{1.979818in}%
\pgfsys@useobject{currentmarker}{}%
\end{pgfscope}%
\begin{pgfscope}%
\pgfsys@transformshift{2.050019in}{1.979818in}%
\pgfsys@useobject{currentmarker}{}%
\end{pgfscope}%
\begin{pgfscope}%
\pgfsys@transformshift{2.050019in}{1.979818in}%
\pgfsys@useobject{currentmarker}{}%
\end{pgfscope}%
\begin{pgfscope}%
\pgfsys@transformshift{2.050019in}{1.979818in}%
\pgfsys@useobject{currentmarker}{}%
\end{pgfscope}%
\begin{pgfscope}%
\pgfsys@transformshift{2.050019in}{1.979818in}%
\pgfsys@useobject{currentmarker}{}%
\end{pgfscope}%
\begin{pgfscope}%
\pgfsys@transformshift{2.050019in}{1.979818in}%
\pgfsys@useobject{currentmarker}{}%
\end{pgfscope}%
\begin{pgfscope}%
\pgfsys@transformshift{2.050019in}{1.979818in}%
\pgfsys@useobject{currentmarker}{}%
\end{pgfscope}%
\begin{pgfscope}%
\pgfsys@transformshift{2.050019in}{1.979818in}%
\pgfsys@useobject{currentmarker}{}%
\end{pgfscope}%
\begin{pgfscope}%
\pgfsys@transformshift{2.050019in}{1.979818in}%
\pgfsys@useobject{currentmarker}{}%
\end{pgfscope}%
\begin{pgfscope}%
\pgfsys@transformshift{2.050019in}{1.979818in}%
\pgfsys@useobject{currentmarker}{}%
\end{pgfscope}%
\begin{pgfscope}%
\pgfsys@transformshift{2.050019in}{1.979818in}%
\pgfsys@useobject{currentmarker}{}%
\end{pgfscope}%
\begin{pgfscope}%
\pgfsys@transformshift{2.050019in}{1.979818in}%
\pgfsys@useobject{currentmarker}{}%
\end{pgfscope}%
\begin{pgfscope}%
\pgfsys@transformshift{2.050019in}{1.979818in}%
\pgfsys@useobject{currentmarker}{}%
\end{pgfscope}%
\begin{pgfscope}%
\pgfsys@transformshift{2.050019in}{1.979818in}%
\pgfsys@useobject{currentmarker}{}%
\end{pgfscope}%
\begin{pgfscope}%
\pgfsys@transformshift{2.050019in}{1.979818in}%
\pgfsys@useobject{currentmarker}{}%
\end{pgfscope}%
\begin{pgfscope}%
\pgfsys@transformshift{2.050019in}{1.979818in}%
\pgfsys@useobject{currentmarker}{}%
\end{pgfscope}%
\begin{pgfscope}%
\pgfsys@transformshift{2.050019in}{1.979818in}%
\pgfsys@useobject{currentmarker}{}%
\end{pgfscope}%
\begin{pgfscope}%
\pgfsys@transformshift{2.050019in}{1.979818in}%
\pgfsys@useobject{currentmarker}{}%
\end{pgfscope}%
\begin{pgfscope}%
\pgfsys@transformshift{2.050019in}{1.979818in}%
\pgfsys@useobject{currentmarker}{}%
\end{pgfscope}%
\begin{pgfscope}%
\pgfsys@transformshift{2.050019in}{1.979818in}%
\pgfsys@useobject{currentmarker}{}%
\end{pgfscope}%
\begin{pgfscope}%
\pgfsys@transformshift{2.050019in}{1.979818in}%
\pgfsys@useobject{currentmarker}{}%
\end{pgfscope}%
\begin{pgfscope}%
\pgfsys@transformshift{2.050019in}{1.979818in}%
\pgfsys@useobject{currentmarker}{}%
\end{pgfscope}%
\begin{pgfscope}%
\pgfsys@transformshift{2.050019in}{1.979818in}%
\pgfsys@useobject{currentmarker}{}%
\end{pgfscope}%
\begin{pgfscope}%
\pgfsys@transformshift{2.050019in}{1.979818in}%
\pgfsys@useobject{currentmarker}{}%
\end{pgfscope}%
\begin{pgfscope}%
\pgfsys@transformshift{2.050019in}{1.979818in}%
\pgfsys@useobject{currentmarker}{}%
\end{pgfscope}%
\begin{pgfscope}%
\pgfsys@transformshift{2.050019in}{1.979818in}%
\pgfsys@useobject{currentmarker}{}%
\end{pgfscope}%
\begin{pgfscope}%
\pgfsys@transformshift{2.050019in}{1.979818in}%
\pgfsys@useobject{currentmarker}{}%
\end{pgfscope}%
\begin{pgfscope}%
\pgfsys@transformshift{2.050019in}{1.979818in}%
\pgfsys@useobject{currentmarker}{}%
\end{pgfscope}%
\begin{pgfscope}%
\pgfsys@transformshift{2.050019in}{1.979818in}%
\pgfsys@useobject{currentmarker}{}%
\end{pgfscope}%
\begin{pgfscope}%
\pgfsys@transformshift{2.050019in}{1.979818in}%
\pgfsys@useobject{currentmarker}{}%
\end{pgfscope}%
\begin{pgfscope}%
\pgfsys@transformshift{2.050019in}{1.979818in}%
\pgfsys@useobject{currentmarker}{}%
\end{pgfscope}%
\begin{pgfscope}%
\pgfsys@transformshift{2.050019in}{1.979818in}%
\pgfsys@useobject{currentmarker}{}%
\end{pgfscope}%
\begin{pgfscope}%
\pgfsys@transformshift{2.050019in}{1.979818in}%
\pgfsys@useobject{currentmarker}{}%
\end{pgfscope}%
\begin{pgfscope}%
\pgfsys@transformshift{2.050019in}{1.979818in}%
\pgfsys@useobject{currentmarker}{}%
\end{pgfscope}%
\begin{pgfscope}%
\pgfsys@transformshift{2.050019in}{1.979818in}%
\pgfsys@useobject{currentmarker}{}%
\end{pgfscope}%
\begin{pgfscope}%
\pgfsys@transformshift{2.050019in}{1.979818in}%
\pgfsys@useobject{currentmarker}{}%
\end{pgfscope}%
\begin{pgfscope}%
\pgfsys@transformshift{2.050019in}{1.979818in}%
\pgfsys@useobject{currentmarker}{}%
\end{pgfscope}%
\begin{pgfscope}%
\pgfsys@transformshift{2.050019in}{1.979818in}%
\pgfsys@useobject{currentmarker}{}%
\end{pgfscope}%
\begin{pgfscope}%
\pgfsys@transformshift{2.050019in}{1.979818in}%
\pgfsys@useobject{currentmarker}{}%
\end{pgfscope}%
\begin{pgfscope}%
\pgfsys@transformshift{2.050019in}{1.979818in}%
\pgfsys@useobject{currentmarker}{}%
\end{pgfscope}%
\begin{pgfscope}%
\pgfsys@transformshift{2.050019in}{1.979818in}%
\pgfsys@useobject{currentmarker}{}%
\end{pgfscope}%
\begin{pgfscope}%
\pgfsys@transformshift{2.050019in}{1.979818in}%
\pgfsys@useobject{currentmarker}{}%
\end{pgfscope}%
\begin{pgfscope}%
\pgfsys@transformshift{2.050019in}{1.979818in}%
\pgfsys@useobject{currentmarker}{}%
\end{pgfscope}%
\begin{pgfscope}%
\pgfsys@transformshift{2.050019in}{1.979818in}%
\pgfsys@useobject{currentmarker}{}%
\end{pgfscope}%
\begin{pgfscope}%
\pgfsys@transformshift{2.050019in}{1.979818in}%
\pgfsys@useobject{currentmarker}{}%
\end{pgfscope}%
\begin{pgfscope}%
\pgfsys@transformshift{2.050019in}{1.979818in}%
\pgfsys@useobject{currentmarker}{}%
\end{pgfscope}%
\begin{pgfscope}%
\pgfsys@transformshift{2.050019in}{1.979818in}%
\pgfsys@useobject{currentmarker}{}%
\end{pgfscope}%
\begin{pgfscope}%
\pgfsys@transformshift{2.050019in}{1.979818in}%
\pgfsys@useobject{currentmarker}{}%
\end{pgfscope}%
\begin{pgfscope}%
\pgfsys@transformshift{2.050019in}{1.979818in}%
\pgfsys@useobject{currentmarker}{}%
\end{pgfscope}%
\begin{pgfscope}%
\pgfsys@transformshift{2.050019in}{1.979818in}%
\pgfsys@useobject{currentmarker}{}%
\end{pgfscope}%
\begin{pgfscope}%
\pgfsys@transformshift{2.050019in}{1.979818in}%
\pgfsys@useobject{currentmarker}{}%
\end{pgfscope}%
\begin{pgfscope}%
\pgfsys@transformshift{2.050019in}{1.979818in}%
\pgfsys@useobject{currentmarker}{}%
\end{pgfscope}%
\begin{pgfscope}%
\pgfsys@transformshift{2.050019in}{1.979818in}%
\pgfsys@useobject{currentmarker}{}%
\end{pgfscope}%
\begin{pgfscope}%
\pgfsys@transformshift{2.050019in}{1.979818in}%
\pgfsys@useobject{currentmarker}{}%
\end{pgfscope}%
\begin{pgfscope}%
\pgfsys@transformshift{2.050019in}{1.979818in}%
\pgfsys@useobject{currentmarker}{}%
\end{pgfscope}%
\begin{pgfscope}%
\pgfsys@transformshift{2.050019in}{1.979818in}%
\pgfsys@useobject{currentmarker}{}%
\end{pgfscope}%
\begin{pgfscope}%
\pgfsys@transformshift{2.050019in}{1.979818in}%
\pgfsys@useobject{currentmarker}{}%
\end{pgfscope}%
\begin{pgfscope}%
\pgfsys@transformshift{2.050019in}{1.979818in}%
\pgfsys@useobject{currentmarker}{}%
\end{pgfscope}%
\begin{pgfscope}%
\pgfsys@transformshift{2.050019in}{1.979818in}%
\pgfsys@useobject{currentmarker}{}%
\end{pgfscope}%
\begin{pgfscope}%
\pgfsys@transformshift{2.050019in}{1.979818in}%
\pgfsys@useobject{currentmarker}{}%
\end{pgfscope}%
\begin{pgfscope}%
\pgfsys@transformshift{2.050019in}{1.979818in}%
\pgfsys@useobject{currentmarker}{}%
\end{pgfscope}%
\begin{pgfscope}%
\pgfsys@transformshift{2.050019in}{1.979818in}%
\pgfsys@useobject{currentmarker}{}%
\end{pgfscope}%
\begin{pgfscope}%
\pgfsys@transformshift{2.050019in}{1.979818in}%
\pgfsys@useobject{currentmarker}{}%
\end{pgfscope}%
\begin{pgfscope}%
\pgfsys@transformshift{2.050019in}{1.979818in}%
\pgfsys@useobject{currentmarker}{}%
\end{pgfscope}%
\begin{pgfscope}%
\pgfsys@transformshift{2.050019in}{1.979818in}%
\pgfsys@useobject{currentmarker}{}%
\end{pgfscope}%
\begin{pgfscope}%
\pgfsys@transformshift{2.050019in}{1.979818in}%
\pgfsys@useobject{currentmarker}{}%
\end{pgfscope}%
\begin{pgfscope}%
\pgfsys@transformshift{2.050019in}{1.979818in}%
\pgfsys@useobject{currentmarker}{}%
\end{pgfscope}%
\begin{pgfscope}%
\pgfsys@transformshift{2.050019in}{1.979818in}%
\pgfsys@useobject{currentmarker}{}%
\end{pgfscope}%
\begin{pgfscope}%
\pgfsys@transformshift{2.050019in}{1.979818in}%
\pgfsys@useobject{currentmarker}{}%
\end{pgfscope}%
\begin{pgfscope}%
\pgfsys@transformshift{2.050019in}{1.979818in}%
\pgfsys@useobject{currentmarker}{}%
\end{pgfscope}%
\begin{pgfscope}%
\pgfsys@transformshift{2.050019in}{1.979818in}%
\pgfsys@useobject{currentmarker}{}%
\end{pgfscope}%
\begin{pgfscope}%
\pgfsys@transformshift{2.050019in}{1.979818in}%
\pgfsys@useobject{currentmarker}{}%
\end{pgfscope}%
\begin{pgfscope}%
\pgfsys@transformshift{2.050019in}{1.979818in}%
\pgfsys@useobject{currentmarker}{}%
\end{pgfscope}%
\begin{pgfscope}%
\pgfsys@transformshift{2.050019in}{1.979818in}%
\pgfsys@useobject{currentmarker}{}%
\end{pgfscope}%
\begin{pgfscope}%
\pgfsys@transformshift{2.050019in}{1.979818in}%
\pgfsys@useobject{currentmarker}{}%
\end{pgfscope}%
\begin{pgfscope}%
\pgfsys@transformshift{2.050019in}{1.979818in}%
\pgfsys@useobject{currentmarker}{}%
\end{pgfscope}%
\begin{pgfscope}%
\pgfsys@transformshift{2.050019in}{1.979818in}%
\pgfsys@useobject{currentmarker}{}%
\end{pgfscope}%
\begin{pgfscope}%
\pgfsys@transformshift{2.050019in}{1.979818in}%
\pgfsys@useobject{currentmarker}{}%
\end{pgfscope}%
\begin{pgfscope}%
\pgfsys@transformshift{2.050019in}{1.979818in}%
\pgfsys@useobject{currentmarker}{}%
\end{pgfscope}%
\begin{pgfscope}%
\pgfsys@transformshift{2.050019in}{1.979818in}%
\pgfsys@useobject{currentmarker}{}%
\end{pgfscope}%
\begin{pgfscope}%
\pgfsys@transformshift{2.050019in}{1.979818in}%
\pgfsys@useobject{currentmarker}{}%
\end{pgfscope}%
\begin{pgfscope}%
\pgfsys@transformshift{2.050019in}{1.979818in}%
\pgfsys@useobject{currentmarker}{}%
\end{pgfscope}%
\begin{pgfscope}%
\pgfsys@transformshift{2.050019in}{1.979818in}%
\pgfsys@useobject{currentmarker}{}%
\end{pgfscope}%
\begin{pgfscope}%
\pgfsys@transformshift{2.050019in}{1.979818in}%
\pgfsys@useobject{currentmarker}{}%
\end{pgfscope}%
\begin{pgfscope}%
\pgfsys@transformshift{2.050019in}{1.979818in}%
\pgfsys@useobject{currentmarker}{}%
\end{pgfscope}%
\begin{pgfscope}%
\pgfsys@transformshift{2.050019in}{1.979818in}%
\pgfsys@useobject{currentmarker}{}%
\end{pgfscope}%
\begin{pgfscope}%
\pgfsys@transformshift{2.050019in}{1.979818in}%
\pgfsys@useobject{currentmarker}{}%
\end{pgfscope}%
\begin{pgfscope}%
\pgfsys@transformshift{2.050019in}{1.979818in}%
\pgfsys@useobject{currentmarker}{}%
\end{pgfscope}%
\begin{pgfscope}%
\pgfsys@transformshift{2.050019in}{1.979818in}%
\pgfsys@useobject{currentmarker}{}%
\end{pgfscope}%
\begin{pgfscope}%
\pgfsys@transformshift{2.050019in}{1.979818in}%
\pgfsys@useobject{currentmarker}{}%
\end{pgfscope}%
\begin{pgfscope}%
\pgfsys@transformshift{2.050019in}{1.979818in}%
\pgfsys@useobject{currentmarker}{}%
\end{pgfscope}%
\begin{pgfscope}%
\pgfsys@transformshift{2.050019in}{1.979818in}%
\pgfsys@useobject{currentmarker}{}%
\end{pgfscope}%
\begin{pgfscope}%
\pgfsys@transformshift{2.050019in}{1.979818in}%
\pgfsys@useobject{currentmarker}{}%
\end{pgfscope}%
\begin{pgfscope}%
\pgfsys@transformshift{2.050019in}{1.979818in}%
\pgfsys@useobject{currentmarker}{}%
\end{pgfscope}%
\begin{pgfscope}%
\pgfsys@transformshift{2.050019in}{1.979818in}%
\pgfsys@useobject{currentmarker}{}%
\end{pgfscope}%
\begin{pgfscope}%
\pgfsys@transformshift{2.050019in}{1.979818in}%
\pgfsys@useobject{currentmarker}{}%
\end{pgfscope}%
\begin{pgfscope}%
\pgfsys@transformshift{2.050019in}{1.979818in}%
\pgfsys@useobject{currentmarker}{}%
\end{pgfscope}%
\begin{pgfscope}%
\pgfsys@transformshift{2.050019in}{1.979818in}%
\pgfsys@useobject{currentmarker}{}%
\end{pgfscope}%
\begin{pgfscope}%
\pgfsys@transformshift{2.050019in}{1.979818in}%
\pgfsys@useobject{currentmarker}{}%
\end{pgfscope}%
\begin{pgfscope}%
\pgfsys@transformshift{2.050019in}{1.979818in}%
\pgfsys@useobject{currentmarker}{}%
\end{pgfscope}%
\begin{pgfscope}%
\pgfsys@transformshift{2.050019in}{1.979818in}%
\pgfsys@useobject{currentmarker}{}%
\end{pgfscope}%
\begin{pgfscope}%
\pgfsys@transformshift{2.050019in}{1.979818in}%
\pgfsys@useobject{currentmarker}{}%
\end{pgfscope}%
\begin{pgfscope}%
\pgfsys@transformshift{2.050019in}{1.979818in}%
\pgfsys@useobject{currentmarker}{}%
\end{pgfscope}%
\begin{pgfscope}%
\pgfsys@transformshift{2.050019in}{1.979818in}%
\pgfsys@useobject{currentmarker}{}%
\end{pgfscope}%
\begin{pgfscope}%
\pgfsys@transformshift{2.050019in}{1.979818in}%
\pgfsys@useobject{currentmarker}{}%
\end{pgfscope}%
\begin{pgfscope}%
\pgfsys@transformshift{2.050019in}{1.979818in}%
\pgfsys@useobject{currentmarker}{}%
\end{pgfscope}%
\begin{pgfscope}%
\pgfsys@transformshift{2.050019in}{1.979818in}%
\pgfsys@useobject{currentmarker}{}%
\end{pgfscope}%
\begin{pgfscope}%
\pgfsys@transformshift{2.050019in}{1.979818in}%
\pgfsys@useobject{currentmarker}{}%
\end{pgfscope}%
\begin{pgfscope}%
\pgfsys@transformshift{2.050019in}{1.979818in}%
\pgfsys@useobject{currentmarker}{}%
\end{pgfscope}%
\begin{pgfscope}%
\pgfsys@transformshift{2.050019in}{1.979818in}%
\pgfsys@useobject{currentmarker}{}%
\end{pgfscope}%
\begin{pgfscope}%
\pgfsys@transformshift{2.050019in}{1.979818in}%
\pgfsys@useobject{currentmarker}{}%
\end{pgfscope}%
\begin{pgfscope}%
\pgfsys@transformshift{2.050019in}{1.979818in}%
\pgfsys@useobject{currentmarker}{}%
\end{pgfscope}%
\begin{pgfscope}%
\pgfsys@transformshift{2.050019in}{1.979818in}%
\pgfsys@useobject{currentmarker}{}%
\end{pgfscope}%
\begin{pgfscope}%
\pgfsys@transformshift{2.050019in}{1.979818in}%
\pgfsys@useobject{currentmarker}{}%
\end{pgfscope}%
\begin{pgfscope}%
\pgfsys@transformshift{2.050019in}{1.979818in}%
\pgfsys@useobject{currentmarker}{}%
\end{pgfscope}%
\begin{pgfscope}%
\pgfsys@transformshift{2.050019in}{1.979818in}%
\pgfsys@useobject{currentmarker}{}%
\end{pgfscope}%
\begin{pgfscope}%
\pgfsys@transformshift{2.050019in}{1.979818in}%
\pgfsys@useobject{currentmarker}{}%
\end{pgfscope}%
\begin{pgfscope}%
\pgfsys@transformshift{2.050019in}{1.979818in}%
\pgfsys@useobject{currentmarker}{}%
\end{pgfscope}%
\begin{pgfscope}%
\pgfsys@transformshift{2.050019in}{1.979818in}%
\pgfsys@useobject{currentmarker}{}%
\end{pgfscope}%
\begin{pgfscope}%
\pgfsys@transformshift{2.050019in}{1.979818in}%
\pgfsys@useobject{currentmarker}{}%
\end{pgfscope}%
\begin{pgfscope}%
\pgfsys@transformshift{2.050019in}{1.979818in}%
\pgfsys@useobject{currentmarker}{}%
\end{pgfscope}%
\begin{pgfscope}%
\pgfsys@transformshift{2.050019in}{1.979818in}%
\pgfsys@useobject{currentmarker}{}%
\end{pgfscope}%
\begin{pgfscope}%
\pgfsys@transformshift{2.050019in}{1.979818in}%
\pgfsys@useobject{currentmarker}{}%
\end{pgfscope}%
\begin{pgfscope}%
\pgfsys@transformshift{2.050019in}{1.979818in}%
\pgfsys@useobject{currentmarker}{}%
\end{pgfscope}%
\begin{pgfscope}%
\pgfsys@transformshift{2.050019in}{1.979818in}%
\pgfsys@useobject{currentmarker}{}%
\end{pgfscope}%
\begin{pgfscope}%
\pgfsys@transformshift{2.050019in}{1.979818in}%
\pgfsys@useobject{currentmarker}{}%
\end{pgfscope}%
\begin{pgfscope}%
\pgfsys@transformshift{2.050019in}{1.979818in}%
\pgfsys@useobject{currentmarker}{}%
\end{pgfscope}%
\begin{pgfscope}%
\pgfsys@transformshift{2.050019in}{1.979818in}%
\pgfsys@useobject{currentmarker}{}%
\end{pgfscope}%
\begin{pgfscope}%
\pgfsys@transformshift{2.050019in}{1.979818in}%
\pgfsys@useobject{currentmarker}{}%
\end{pgfscope}%
\begin{pgfscope}%
\pgfsys@transformshift{2.050019in}{1.979818in}%
\pgfsys@useobject{currentmarker}{}%
\end{pgfscope}%
\begin{pgfscope}%
\pgfsys@transformshift{2.050019in}{1.979818in}%
\pgfsys@useobject{currentmarker}{}%
\end{pgfscope}%
\begin{pgfscope}%
\pgfsys@transformshift{2.050019in}{1.979818in}%
\pgfsys@useobject{currentmarker}{}%
\end{pgfscope}%
\begin{pgfscope}%
\pgfsys@transformshift{2.050019in}{1.979818in}%
\pgfsys@useobject{currentmarker}{}%
\end{pgfscope}%
\begin{pgfscope}%
\pgfsys@transformshift{2.050019in}{1.979818in}%
\pgfsys@useobject{currentmarker}{}%
\end{pgfscope}%
\begin{pgfscope}%
\pgfsys@transformshift{2.050019in}{1.979818in}%
\pgfsys@useobject{currentmarker}{}%
\end{pgfscope}%
\begin{pgfscope}%
\pgfsys@transformshift{2.050019in}{1.979818in}%
\pgfsys@useobject{currentmarker}{}%
\end{pgfscope}%
\begin{pgfscope}%
\pgfsys@transformshift{2.050019in}{1.979818in}%
\pgfsys@useobject{currentmarker}{}%
\end{pgfscope}%
\begin{pgfscope}%
\pgfsys@transformshift{2.050019in}{1.979818in}%
\pgfsys@useobject{currentmarker}{}%
\end{pgfscope}%
\begin{pgfscope}%
\pgfsys@transformshift{2.050019in}{1.979818in}%
\pgfsys@useobject{currentmarker}{}%
\end{pgfscope}%
\begin{pgfscope}%
\pgfsys@transformshift{2.050019in}{1.979818in}%
\pgfsys@useobject{currentmarker}{}%
\end{pgfscope}%
\begin{pgfscope}%
\pgfsys@transformshift{2.050019in}{1.979818in}%
\pgfsys@useobject{currentmarker}{}%
\end{pgfscope}%
\begin{pgfscope}%
\pgfsys@transformshift{2.050019in}{1.979818in}%
\pgfsys@useobject{currentmarker}{}%
\end{pgfscope}%
\begin{pgfscope}%
\pgfsys@transformshift{2.050019in}{1.979818in}%
\pgfsys@useobject{currentmarker}{}%
\end{pgfscope}%
\begin{pgfscope}%
\pgfsys@transformshift{2.050019in}{1.979818in}%
\pgfsys@useobject{currentmarker}{}%
\end{pgfscope}%
\begin{pgfscope}%
\pgfsys@transformshift{2.050019in}{1.979818in}%
\pgfsys@useobject{currentmarker}{}%
\end{pgfscope}%
\begin{pgfscope}%
\pgfsys@transformshift{2.050019in}{1.979818in}%
\pgfsys@useobject{currentmarker}{}%
\end{pgfscope}%
\begin{pgfscope}%
\pgfsys@transformshift{2.050019in}{1.979818in}%
\pgfsys@useobject{currentmarker}{}%
\end{pgfscope}%
\begin{pgfscope}%
\pgfsys@transformshift{2.050019in}{1.979818in}%
\pgfsys@useobject{currentmarker}{}%
\end{pgfscope}%
\begin{pgfscope}%
\pgfsys@transformshift{2.050019in}{1.979818in}%
\pgfsys@useobject{currentmarker}{}%
\end{pgfscope}%
\begin{pgfscope}%
\pgfsys@transformshift{2.050019in}{1.979818in}%
\pgfsys@useobject{currentmarker}{}%
\end{pgfscope}%
\begin{pgfscope}%
\pgfsys@transformshift{2.050019in}{1.979818in}%
\pgfsys@useobject{currentmarker}{}%
\end{pgfscope}%
\begin{pgfscope}%
\pgfsys@transformshift{2.050019in}{1.979818in}%
\pgfsys@useobject{currentmarker}{}%
\end{pgfscope}%
\begin{pgfscope}%
\pgfsys@transformshift{2.050019in}{1.979818in}%
\pgfsys@useobject{currentmarker}{}%
\end{pgfscope}%
\begin{pgfscope}%
\pgfsys@transformshift{2.050019in}{1.979818in}%
\pgfsys@useobject{currentmarker}{}%
\end{pgfscope}%
\begin{pgfscope}%
\pgfsys@transformshift{2.050019in}{1.979818in}%
\pgfsys@useobject{currentmarker}{}%
\end{pgfscope}%
\begin{pgfscope}%
\pgfsys@transformshift{2.050019in}{1.979818in}%
\pgfsys@useobject{currentmarker}{}%
\end{pgfscope}%
\begin{pgfscope}%
\pgfsys@transformshift{2.050019in}{1.979818in}%
\pgfsys@useobject{currentmarker}{}%
\end{pgfscope}%
\begin{pgfscope}%
\pgfsys@transformshift{2.050019in}{1.979818in}%
\pgfsys@useobject{currentmarker}{}%
\end{pgfscope}%
\begin{pgfscope}%
\pgfsys@transformshift{2.050019in}{1.979818in}%
\pgfsys@useobject{currentmarker}{}%
\end{pgfscope}%
\begin{pgfscope}%
\pgfsys@transformshift{2.050019in}{1.979818in}%
\pgfsys@useobject{currentmarker}{}%
\end{pgfscope}%
\begin{pgfscope}%
\pgfsys@transformshift{2.050019in}{1.979818in}%
\pgfsys@useobject{currentmarker}{}%
\end{pgfscope}%
\begin{pgfscope}%
\pgfsys@transformshift{2.050019in}{1.979818in}%
\pgfsys@useobject{currentmarker}{}%
\end{pgfscope}%
\begin{pgfscope}%
\pgfsys@transformshift{2.050019in}{1.979818in}%
\pgfsys@useobject{currentmarker}{}%
\end{pgfscope}%
\begin{pgfscope}%
\pgfsys@transformshift{2.050019in}{1.979818in}%
\pgfsys@useobject{currentmarker}{}%
\end{pgfscope}%
\end{pgfscope}%
\begin{pgfscope}%
\pgfpathrectangle{\pgfqpoint{0.500000in}{0.440000in}}{\pgfqpoint{3.100000in}{3.080000in}}%
\pgfusepath{clip}%
\pgfsetrectcap%
\pgfsetroundjoin%
\pgfsetlinewidth{0.501875pt}%
\definecolor{currentstroke}{rgb}{1.000000,0.498039,0.054902}%
\pgfsetstrokecolor{currentstroke}%
\pgfsetdash{}{0pt}%
\pgfpathmoveto{\pgfqpoint{2.050019in}{1.979818in}}%
\pgfpathlineto{\pgfqpoint{2.050019in}{1.979818in}}%
\pgfusepath{stroke}%
\end{pgfscope}%
\begin{pgfscope}%
\pgfpathrectangle{\pgfqpoint{0.500000in}{0.440000in}}{\pgfqpoint{3.100000in}{3.080000in}}%
\pgfusepath{clip}%
\pgfsetrectcap%
\pgfsetroundjoin%
\pgfsetlinewidth{0.501875pt}%
\definecolor{currentstroke}{rgb}{0.172549,0.627451,0.172549}%
\pgfsetstrokecolor{currentstroke}%
\pgfsetdash{}{0pt}%
\pgfpathmoveto{\pgfqpoint{2.667558in}{1.366237in}}%
\pgfpathlineto{\pgfqpoint{2.602754in}{1.307991in}}%
\pgfpathlineto{\pgfqpoint{2.532411in}{1.256423in}}%
\pgfpathlineto{\pgfqpoint{2.457244in}{1.212047in}}%
\pgfpathlineto{\pgfqpoint{2.378012in}{1.175363in}}%
\pgfpathlineto{\pgfqpoint{2.295501in}{1.146769in}}%
\pgfpathlineto{\pgfqpoint{2.210520in}{1.126566in}}%
\pgfpathlineto{\pgfqpoint{2.123905in}{1.114956in}}%
\pgfpathlineto{\pgfqpoint{2.036516in}{1.112042in}}%
\pgfpathlineto{\pgfqpoint{1.949237in}{1.117830in}}%
\pgfpathlineto{\pgfqpoint{1.862979in}{1.132225in}}%
\pgfpathlineto{\pgfqpoint{1.778640in}{1.155043in}}%
\pgfpathlineto{\pgfqpoint{1.696970in}{1.186067in}}%
\pgfpathlineto{\pgfqpoint{1.618824in}{1.225015in}}%
\pgfpathlineto{\pgfqpoint{1.545020in}{1.271525in}}%
\pgfpathlineto{\pgfqpoint{1.476308in}{1.325158in}}%
\pgfpathlineto{\pgfqpoint{1.413374in}{1.385394in}}%
\pgfpathlineto{\pgfqpoint{1.356835in}{1.451640in}}%
\pgfpathlineto{\pgfqpoint{1.307240in}{1.523222in}}%
\pgfpathlineto{\pgfqpoint{1.265074in}{1.599389in}}%
\pgfpathlineto{\pgfqpoint{1.230755in}{1.679312in}}%
\pgfpathlineto{\pgfqpoint{1.204592in}{1.762183in}}%
\pgfpathlineto{\pgfqpoint{1.186826in}{1.847250in}}%
\pgfpathlineto{\pgfqpoint{1.177690in}{1.933633in}}%
\pgfpathlineto{\pgfqpoint{1.177318in}{2.020461in}}%
\pgfpathlineto{\pgfqpoint{1.185738in}{2.106872in}}%
\pgfpathlineto{\pgfqpoint{1.202878in}{2.192019in}}%
\pgfpathlineto{\pgfqpoint{1.228562in}{2.275061in}}%
\pgfpathlineto{\pgfqpoint{1.262511in}{2.355172in}}%
\pgfpathlineto{\pgfqpoint{1.304344in}{2.431536in}}%
\pgfpathlineto{\pgfqpoint{1.353578in}{2.503347in}}%
\pgfpathlineto{\pgfqpoint{1.409715in}{2.569922in}}%
\pgfpathlineto{\pgfqpoint{1.472246in}{2.630635in}}%
\pgfpathlineto{\pgfqpoint{1.540552in}{2.684824in}}%
\pgfpathlineto{\pgfqpoint{1.613962in}{2.731920in}}%
\pgfpathlineto{\pgfqpoint{1.691760in}{2.771441in}}%
\pgfpathlineto{\pgfqpoint{1.773181in}{2.802997in}}%
\pgfpathlineto{\pgfqpoint{1.857413in}{2.826285in}}%
\pgfpathlineto{\pgfqpoint{1.943597in}{2.841094in}}%
\pgfpathlineto{\pgfqpoint{2.030825in}{2.847301in}}%
\pgfpathlineto{\pgfqpoint{2.118197in}{2.844879in}}%
\pgfpathlineto{\pgfqpoint{2.204930in}{2.833863in}}%
\pgfpathlineto{\pgfqpoint{2.290113in}{2.814320in}}%
\pgfpathlineto{\pgfqpoint{2.372868in}{2.786408in}}%
\pgfpathlineto{\pgfqpoint{2.452361in}{2.750377in}}%
\pgfpathlineto{\pgfqpoint{2.527806in}{2.706568in}}%
\pgfpathlineto{\pgfqpoint{2.598461in}{2.655413in}}%
\pgfpathlineto{\pgfqpoint{2.663633in}{2.597439in}}%
\pgfpathlineto{\pgfqpoint{2.722673in}{2.533263in}}%
\pgfpathlineto{\pgfqpoint{2.774979in}{2.463593in}}%
\pgfpathlineto{\pgfqpoint{2.820034in}{2.389173in}}%
\pgfpathlineto{\pgfqpoint{2.857451in}{2.310655in}}%
\pgfpathlineto{\pgfqpoint{2.886799in}{2.228837in}}%
\pgfpathlineto{\pgfqpoint{2.907742in}{2.144537in}}%
\pgfpathlineto{\pgfqpoint{2.920046in}{2.058589in}}%
\pgfpathlineto{\pgfqpoint{2.923580in}{1.971834in}}%
\pgfpathlineto{\pgfqpoint{2.918314in}{1.885129in}}%
\pgfpathlineto{\pgfqpoint{2.904322in}{1.799341in}}%
\pgfpathlineto{\pgfqpoint{2.881777in}{1.715348in}}%
\pgfpathlineto{\pgfqpoint{2.850958in}{1.634042in}}%
\pgfpathlineto{\pgfqpoint{2.812187in}{1.556203in}}%
\pgfpathlineto{\pgfqpoint{2.765811in}{1.482565in}}%
\pgfpathlineto{\pgfqpoint{2.712277in}{1.413915in}}%
\pgfpathlineto{\pgfqpoint{2.652101in}{1.350963in}}%
\pgfpathlineto{\pgfqpoint{2.585865in}{1.294347in}}%
\pgfpathlineto{\pgfqpoint{2.514219in}{1.244623in}}%
\pgfpathlineto{\pgfqpoint{2.437881in}{1.202277in}}%
\pgfpathlineto{\pgfqpoint{2.357635in}{1.167714in}}%
\pgfpathlineto{\pgfqpoint{2.274335in}{1.141266in}}%
\pgfpathlineto{\pgfqpoint{2.188850in}{1.123171in}}%
\pgfpathlineto{\pgfqpoint{2.101949in}{1.113578in}}%
\pgfpathlineto{\pgfqpoint{2.014532in}{1.112633in}}%
\pgfpathlineto{\pgfqpoint{1.927491in}{1.120390in}}%
\pgfpathlineto{\pgfqpoint{1.841695in}{1.136800in}}%
\pgfpathlineto{\pgfqpoint{1.757989in}{1.161717in}}%
\pgfpathlineto{\pgfqpoint{1.677197in}{1.194893in}}%
\pgfpathlineto{\pgfqpoint{1.600117in}{1.235979in}}%
\pgfpathlineto{\pgfqpoint{1.527526in}{1.284529in}}%
\pgfpathlineto{\pgfqpoint{1.460177in}{1.339993in}}%
\pgfpathlineto{\pgfqpoint{1.398749in}{1.401772in}}%
\pgfpathlineto{\pgfqpoint{1.343781in}{1.469323in}}%
\pgfpathlineto{\pgfqpoint{1.295874in}{1.541971in}}%
\pgfpathlineto{\pgfqpoint{1.255547in}{1.618998in}}%
\pgfpathlineto{\pgfqpoint{1.223221in}{1.699649in}}%
\pgfpathlineto{\pgfqpoint{1.199222in}{1.783133in}}%
\pgfpathlineto{\pgfqpoint{1.183781in}{1.868625in}}%
\pgfpathlineto{\pgfqpoint{1.177031in}{1.955263in}}%
\pgfpathlineto{\pgfqpoint{1.179011in}{2.042149in}}%
\pgfpathlineto{\pgfqpoint{1.189664in}{2.128355in}}%
\pgfpathlineto{\pgfqpoint{1.208863in}{2.213097in}}%
\pgfpathlineto{\pgfqpoint{1.236446in}{2.295530in}}%
\pgfpathlineto{\pgfqpoint{1.272170in}{2.374785in}}%
\pgfpathlineto{\pgfqpoint{1.315710in}{2.450054in}}%
\pgfpathlineto{\pgfqpoint{1.366653in}{2.520582in}}%
\pgfpathlineto{\pgfqpoint{1.424502in}{2.585676in}}%
\pgfpathlineto{\pgfqpoint{1.488673in}{2.644699in}}%
\pgfpathlineto{\pgfqpoint{1.558501in}{2.697071in}}%
\pgfpathlineto{\pgfqpoint{1.633230in}{2.742271in}}%
\pgfpathlineto{\pgfqpoint{1.712038in}{2.779845in}}%
\pgfpathlineto{\pgfqpoint{1.794220in}{2.809480in}}%
\pgfpathlineto{\pgfqpoint{1.878959in}{2.830845in}}%
\pgfpathlineto{\pgfqpoint{1.965396in}{2.843678in}}%
\pgfpathlineto{\pgfqpoint{2.052668in}{2.847817in}}%
\pgfpathlineto{\pgfqpoint{2.139919in}{2.843205in}}%
\pgfpathlineto{\pgfqpoint{2.226289in}{2.829886in}}%
\pgfpathlineto{\pgfqpoint{2.310923in}{2.808007in}}%
\pgfpathlineto{\pgfqpoint{2.392966in}{2.777818in}}%
\pgfpathlineto{\pgfqpoint{2.471566in}{2.739670in}}%
\pgfpathlineto{\pgfqpoint{2.545904in}{2.693998in}}%
\pgfpathlineto{\pgfqpoint{2.615324in}{2.641205in}}%
\pgfpathlineto{\pgfqpoint{2.679094in}{2.581806in}}%
\pgfpathlineto{\pgfqpoint{2.736541in}{2.516380in}}%
\pgfpathlineto{\pgfqpoint{2.787074in}{2.445564in}}%
\pgfpathlineto{\pgfqpoint{2.830189in}{2.370052in}}%
\pgfpathlineto{\pgfqpoint{2.865467in}{2.290593in}}%
\pgfpathlineto{\pgfqpoint{2.892573in}{2.207991in}}%
\pgfpathlineto{\pgfqpoint{2.911255in}{2.123110in}}%
\pgfpathlineto{\pgfqpoint{2.921351in}{2.036850in}}%
\pgfpathlineto{\pgfqpoint{2.922799in}{1.949996in}}%
\pgfpathlineto{\pgfqpoint{2.915548in}{1.863430in}}%
\pgfpathlineto{\pgfqpoint{2.899627in}{1.778048in}}%
\pgfpathlineto{\pgfqpoint{2.875160in}{1.694712in}}%
\pgfpathlineto{\pgfqpoint{2.842370in}{1.614247in}}%
\pgfpathlineto{\pgfqpoint{2.801576in}{1.537444in}}%
\pgfpathlineto{\pgfqpoint{2.753195in}{1.465060in}}%
\pgfpathlineto{\pgfqpoint{2.697742in}{1.397814in}}%
\pgfpathlineto{\pgfqpoint{2.635828in}{1.336391in}}%
\pgfpathlineto{\pgfqpoint{2.568147in}{1.281430in}}%
\pgfpathlineto{\pgfqpoint{2.495306in}{1.233401in}}%
\pgfpathlineto{\pgfqpoint{2.418017in}{1.192815in}}%
\pgfpathlineto{\pgfqpoint{2.337055in}{1.160126in}}%
\pgfpathlineto{\pgfqpoint{2.253219in}{1.135688in}}%
\pgfpathlineto{\pgfqpoint{2.167332in}{1.119756in}}%
\pgfpathlineto{\pgfqpoint{2.080237in}{1.112486in}}%
\pgfpathlineto{\pgfqpoint{1.992805in}{1.113933in}}%
\pgfpathlineto{\pgfqpoint{1.905926in}{1.124056in}}%
\pgfpathlineto{\pgfqpoint{1.820516in}{1.142712in}}%
\pgfpathlineto{\pgfqpoint{1.737439in}{1.169679in}}%
\pgfpathlineto{\pgfqpoint{1.657440in}{1.204715in}}%
\pgfpathlineto{\pgfqpoint{1.581372in}{1.247495in}}%
\pgfpathlineto{\pgfqpoint{1.510025in}{1.297617in}}%
\pgfpathlineto{\pgfqpoint{1.444122in}{1.354605in}}%
\pgfpathlineto{\pgfqpoint{1.384318in}{1.417903in}}%
\pgfpathlineto{\pgfqpoint{1.331196in}{1.486883in}}%
\pgfpathlineto{\pgfqpoint{1.285274in}{1.560840in}}%
\pgfpathlineto{\pgfqpoint{1.246998in}{1.638992in}}%
\pgfpathlineto{\pgfqpoint{1.216748in}{1.720484in}}%
\pgfpathlineto{\pgfqpoint{1.194773in}{1.804548in}}%
\pgfpathlineto{\pgfqpoint{1.181308in}{1.890378in}}%
\pgfpathlineto{\pgfqpoint{1.176541in}{1.977088in}}%
\pgfpathlineto{\pgfqpoint{1.180556in}{2.063808in}}%
\pgfpathlineto{\pgfqpoint{1.193336in}{2.149680in}}%
\pgfpathlineto{\pgfqpoint{1.214761in}{2.233860in}}%
\pgfpathlineto{\pgfqpoint{1.244607in}{2.315518in}}%
\pgfpathlineto{\pgfqpoint{1.282551in}{2.393839in}}%
\pgfpathlineto{\pgfqpoint{1.328165in}{2.468018in}}%
\pgfpathlineto{\pgfqpoint{1.380924in}{2.537271in}}%
\pgfpathlineto{\pgfqpoint{1.440333in}{2.600979in}}%
\pgfpathlineto{\pgfqpoint{1.505832in}{2.658500in}}%
\pgfpathlineto{\pgfqpoint{1.576770in}{2.709210in}}%
\pgfpathlineto{\pgfqpoint{1.652454in}{2.752578in}}%
\pgfpathlineto{\pgfqpoint{1.732143in}{2.788163in}}%
\pgfpathlineto{\pgfqpoint{1.815052in}{2.815616in}}%
\pgfpathlineto{\pgfqpoint{1.900352in}{2.834679in}}%
\pgfpathlineto{\pgfqpoint{1.987167in}{2.845183in}}%
\pgfpathlineto{\pgfqpoint{2.074576in}{2.847052in}}%
\pgfpathlineto{\pgfqpoint{2.161711in}{2.840302in}}%
\pgfpathlineto{\pgfqpoint{2.247771in}{2.824995in}}%
\pgfpathlineto{\pgfqpoint{2.331845in}{2.801241in}}%
\pgfpathlineto{\pgfqpoint{2.413069in}{2.769244in}}%
\pgfpathlineto{\pgfqpoint{2.490629in}{2.729295in}}%
\pgfpathlineto{\pgfqpoint{2.563758in}{2.681777in}}%
\pgfpathlineto{\pgfqpoint{2.631740in}{2.627164in}}%
\pgfpathlineto{\pgfqpoint{2.693905in}{2.566020in}}%
\pgfpathlineto{\pgfqpoint{2.749633in}{2.499000in}}%
\pgfpathlineto{\pgfqpoint{2.798355in}{2.426849in}}%
\pgfpathlineto{\pgfqpoint{2.839615in}{2.350295in}}%
\pgfpathlineto{\pgfqpoint{2.873032in}{2.270028in}}%
\pgfpathlineto{\pgfqpoint{2.898214in}{2.186868in}}%
\pgfpathlineto{\pgfqpoint{2.914871in}{2.101647in}}%
\pgfpathlineto{\pgfqpoint{2.922813in}{2.015204in}}%
\pgfpathlineto{\pgfqpoint{2.921957in}{1.928388in}}%
\pgfpathlineto{\pgfqpoint{2.912319in}{1.842056in}}%
\pgfpathlineto{\pgfqpoint{2.894019in}{1.757074in}}%
\pgfpathlineto{\pgfqpoint{2.867279in}{1.674318in}}%
\pgfpathlineto{\pgfqpoint{2.832423in}{1.594668in}}%
\pgfpathlineto{\pgfqpoint{2.789788in}{1.518855in}}%
\pgfpathlineto{\pgfqpoint{2.739767in}{1.447629in}}%
\pgfpathlineto{\pgfqpoint{2.682846in}{1.381748in}}%
\pgfpathlineto{\pgfqpoint{2.619574in}{1.321890in}}%
\pgfpathlineto{\pgfqpoint{2.550567in}{1.268656in}}%
\pgfpathlineto{\pgfqpoint{2.476504in}{1.222571in}}%
\pgfpathlineto{\pgfqpoint{2.398132in}{1.184079in}}%
\pgfpathlineto{\pgfqpoint{2.316260in}{1.153550in}}%
\pgfpathlineto{\pgfqpoint{2.231766in}{1.131272in}}%
\pgfpathlineto{\pgfqpoint{2.145484in}{1.117430in}}%
\pgfpathlineto{\pgfqpoint{2.058220in}{1.112159in}}%
\pgfpathlineto{\pgfqpoint{1.970886in}{1.115564in}}%
\pgfpathlineto{\pgfqpoint{1.884366in}{1.127651in}}%
\pgfpathlineto{\pgfqpoint{1.799520in}{1.148325in}}%
\pgfpathlineto{\pgfqpoint{1.717185in}{1.177392in}}%
\pgfpathlineto{\pgfqpoint{1.638168in}{1.214559in}}%
\pgfpathlineto{\pgfqpoint{1.563256in}{1.259432in}}%
\pgfpathlineto{\pgfqpoint{1.493206in}{1.311520in}}%
\pgfpathlineto{\pgfqpoint{1.428753in}{1.370228in}}%
\pgfpathlineto{\pgfqpoint{1.370504in}{1.434967in}}%
\pgfpathlineto{\pgfqpoint{1.319007in}{1.505152in}}%
\pgfpathlineto{\pgfqpoint{1.274832in}{1.580077in}}%
\pgfpathlineto{\pgfqpoint{1.238454in}{1.659002in}}%
\pgfpathlineto{\pgfqpoint{1.210252in}{1.741155in}}%
\pgfpathlineto{\pgfqpoint{1.190506in}{1.825729in}}%
\pgfpathlineto{\pgfqpoint{1.179401in}{1.911884in}}%
\pgfpathlineto{\pgfqpoint{1.177026in}{1.998746in}}%
\pgfpathlineto{\pgfqpoint{1.183371in}{2.085408in}}%
\pgfpathlineto{\pgfqpoint{1.198334in}{2.170958in}}%
\pgfpathlineto{\pgfqpoint{1.221764in}{2.254639in}}%
\pgfpathlineto{\pgfqpoint{1.253457in}{2.335580in}}%
\pgfpathlineto{\pgfqpoint{1.293130in}{2.412937in}}%
\pgfpathlineto{\pgfqpoint{1.340414in}{2.485924in}}%
\pgfpathlineto{\pgfqpoint{1.394858in}{2.553812in}}%
\pgfpathlineto{\pgfqpoint{1.455924in}{2.615936in}}%
\pgfpathlineto{\pgfqpoint{1.522994in}{2.671688in}}%
\pgfpathlineto{\pgfqpoint{1.595364in}{2.720520in}}%
\pgfpathlineto{\pgfqpoint{1.672245in}{2.761946in}}%
\pgfpathlineto{\pgfqpoint{1.752821in}{2.795564in}}%
\pgfpathlineto{\pgfqpoint{1.836382in}{2.821088in}}%
\pgfpathlineto{\pgfqpoint{1.922073in}{2.838210in}}%
\pgfpathlineto{\pgfqpoint{2.009029in}{2.846716in}}%
\pgfpathlineto{\pgfqpoint{2.096386in}{2.846491in}}%
\pgfpathlineto{\pgfqpoint{2.183287in}{2.837527in}}%
\pgfpathlineto{\pgfqpoint{2.268876in}{2.819913in}}%
\pgfpathlineto{\pgfqpoint{2.352302in}{2.793846in}}%
\pgfpathlineto{\pgfqpoint{2.432717in}{2.759620in}}%
\pgfpathlineto{\pgfqpoint{2.509277in}{2.717637in}}%
\pgfpathlineto{\pgfqpoint{2.581219in}{2.668345in}}%
\pgfpathlineto{\pgfqpoint{2.647891in}{2.612179in}}%
\pgfpathlineto{\pgfqpoint{2.708576in}{2.549694in}}%
\pgfpathlineto{\pgfqpoint{2.762635in}{2.481501in}}%
\pgfpathlineto{\pgfqpoint{2.809519in}{2.408263in}}%
\pgfpathlineto{\pgfqpoint{2.848761in}{2.330699in}}%
\pgfpathlineto{\pgfqpoint{2.879982in}{2.249582in}}%
\pgfpathlineto{\pgfqpoint{2.902890in}{2.165740in}}%
\pgfpathlineto{\pgfqpoint{2.917277in}{2.080053in}}%
\pgfpathlineto{\pgfqpoint{2.923029in}{1.993415in}}%
\pgfpathlineto{\pgfqpoint{2.920113in}{1.906604in}}%
\pgfpathlineto{\pgfqpoint{2.908514in}{1.820524in}}%
\pgfpathlineto{\pgfqpoint{2.888306in}{1.736061in}}%
\pgfpathlineto{\pgfqpoint{2.859660in}{1.654065in}}%
\pgfpathlineto{\pgfqpoint{2.822843in}{1.575345in}}%
\pgfpathlineto{\pgfqpoint{2.778218in}{1.500676in}}%
\pgfpathlineto{\pgfqpoint{2.726245in}{1.430792in}}%
\pgfpathlineto{\pgfqpoint{2.667479in}{1.366392in}}%
\pgfpathlineto{\pgfqpoint{2.602574in}{1.308134in}}%
\pgfpathlineto{\pgfqpoint{2.532234in}{1.256609in}}%
\pgfpathlineto{\pgfqpoint{2.457076in}{1.212256in}}%
\pgfpathlineto{\pgfqpoint{2.377855in}{1.175571in}}%
\pgfpathlineto{\pgfqpoint{2.295362in}{1.146963in}}%
\pgfpathlineto{\pgfqpoint{2.210410in}{1.126743in}}%
\pgfpathlineto{\pgfqpoint{2.123832in}{1.115119in}}%
\pgfpathlineto{\pgfqpoint{2.036481in}{1.112202in}}%
\pgfpathlineto{\pgfqpoint{1.949233in}{1.118002in}}%
\pgfpathlineto{\pgfqpoint{1.862981in}{1.132429in}}%
\pgfpathlineto{\pgfqpoint{1.778642in}{1.155293in}}%
\pgfpathlineto{\pgfqpoint{1.697029in}{1.186344in}}%
\pgfpathlineto{\pgfqpoint{1.618906in}{1.225307in}}%
\pgfpathlineto{\pgfqpoint{1.545107in}{1.271813in}}%
\pgfpathlineto{\pgfqpoint{1.476397in}{1.325423in}}%
\pgfpathlineto{\pgfqpoint{1.413468in}{1.385621in}}%
\pgfpathlineto{\pgfqpoint{1.356942in}{1.451821in}}%
\pgfpathlineto{\pgfqpoint{1.307371in}{1.523359in}}%
\pgfpathlineto{\pgfqpoint{1.265234in}{1.599500in}}%
\pgfpathlineto{\pgfqpoint{1.230943in}{1.679434in}}%
\pgfpathlineto{\pgfqpoint{1.204827in}{1.762296in}}%
\pgfpathlineto{\pgfqpoint{1.187097in}{1.847343in}}%
\pgfpathlineto{\pgfqpoint{1.177970in}{1.933718in}}%
\pgfpathlineto{\pgfqpoint{1.177585in}{2.020536in}}%
\pgfpathlineto{\pgfqpoint{1.185978in}{2.106930in}}%
\pgfpathlineto{\pgfqpoint{1.203086in}{2.192047in}}%
\pgfpathlineto{\pgfqpoint{1.228741in}{2.275051in}}%
\pgfpathlineto{\pgfqpoint{1.262674in}{2.355122in}}%
\pgfpathlineto{\pgfqpoint{1.304514in}{2.431454in}}%
\pgfpathlineto{\pgfqpoint{1.353787in}{2.503260in}}%
\pgfpathlineto{\pgfqpoint{1.409940in}{2.569793in}}%
\pgfpathlineto{\pgfqpoint{1.472472in}{2.630472in}}%
\pgfpathlineto{\pgfqpoint{1.540772in}{2.684654in}}%
\pgfpathlineto{\pgfqpoint{1.614165in}{2.731757in}}%
\pgfpathlineto{\pgfqpoint{1.691932in}{2.771288in}}%
\pgfpathlineto{\pgfqpoint{1.773313in}{2.802848in}}%
\pgfpathlineto{\pgfqpoint{1.857503in}{2.826131in}}%
\pgfpathlineto{\pgfqpoint{1.943656in}{2.840922in}}%
\pgfpathlineto{\pgfqpoint{2.030881in}{2.847098in}}%
\pgfpathlineto{\pgfqpoint{2.118248in}{2.844628in}}%
\pgfpathlineto{\pgfqpoint{2.204935in}{2.833570in}}%
\pgfpathlineto{\pgfqpoint{2.290104in}{2.814010in}}%
\pgfpathlineto{\pgfqpoint{2.372856in}{2.786106in}}%
\pgfpathlineto{\pgfqpoint{2.452345in}{2.750103in}}%
\pgfpathlineto{\pgfqpoint{2.527777in}{2.706335in}}%
\pgfpathlineto{\pgfqpoint{2.598408in}{2.655228in}}%
\pgfpathlineto{\pgfqpoint{2.663545in}{2.597294in}}%
\pgfpathlineto{\pgfqpoint{2.722549in}{2.533136in}}%
\pgfpathlineto{\pgfqpoint{2.774827in}{2.463446in}}%
\pgfpathlineto{\pgfqpoint{2.819845in}{2.389001in}}%
\pgfpathlineto{\pgfqpoint{2.857213in}{2.310498in}}%
\pgfpathlineto{\pgfqpoint{2.886546in}{2.228685in}}%
\pgfpathlineto{\pgfqpoint{2.907498in}{2.144394in}}%
\pgfpathlineto{\pgfqpoint{2.919826in}{2.058466in}}%
\pgfpathlineto{\pgfqpoint{2.923386in}{1.971745in}}%
\pgfpathlineto{\pgfqpoint{2.918142in}{1.885084in}}%
\pgfpathlineto{\pgfqpoint{2.904157in}{1.799340in}}%
\pgfpathlineto{\pgfqpoint{2.881598in}{1.715378in}}%
\pgfpathlineto{\pgfqpoint{2.850735in}{1.634069in}}%
\pgfpathlineto{\pgfqpoint{2.811931in}{1.556270in}}%
\pgfpathlineto{\pgfqpoint{2.765540in}{1.482674in}}%
\pgfpathlineto{\pgfqpoint{2.712002in}{1.414041in}}%
\pgfpathlineto{\pgfqpoint{2.651837in}{1.351094in}}%
\pgfpathlineto{\pgfqpoint{2.585629in}{1.294480in}}%
\pgfpathlineto{\pgfqpoint{2.514023in}{1.244762in}}%
\pgfpathlineto{\pgfqpoint{2.437731in}{1.202428in}}%
\pgfpathlineto{\pgfqpoint{2.357523in}{1.167886in}}%
\pgfpathlineto{\pgfqpoint{2.274236in}{1.141463in}}%
\pgfpathlineto{\pgfqpoint{2.188761in}{1.123405in}}%
\pgfpathlineto{\pgfqpoint{2.101888in}{1.113847in}}%
\pgfpathlineto{\pgfqpoint{2.014484in}{1.112913in}}%
\pgfpathlineto{\pgfqpoint{1.927453in}{1.120662in}}%
\pgfpathlineto{\pgfqpoint{1.841671in}{1.137050in}}%
\pgfpathlineto{\pgfqpoint{1.757989in}{1.161936in}}%
\pgfpathlineto{\pgfqpoint{1.677228in}{1.195081in}}%
\pgfpathlineto{\pgfqpoint{1.600185in}{1.236145in}}%
\pgfpathlineto{\pgfqpoint{1.527627in}{1.284689in}}%
\pgfpathlineto{\pgfqpoint{1.460295in}{1.340175in}}%
\pgfpathlineto{\pgfqpoint{1.398896in}{1.401974in}}%
\pgfpathlineto{\pgfqpoint{1.343966in}{1.469518in}}%
\pgfpathlineto{\pgfqpoint{1.296072in}{1.542160in}}%
\pgfpathlineto{\pgfqpoint{1.255742in}{1.619173in}}%
\pgfpathlineto{\pgfqpoint{1.223406in}{1.699799in}}%
\pgfpathlineto{\pgfqpoint{1.199398in}{1.783248in}}%
\pgfpathlineto{\pgfqpoint{1.183954in}{1.868700in}}%
\pgfpathlineto{\pgfqpoint{1.177212in}{1.955302in}}%
\pgfpathlineto{\pgfqpoint{1.179213in}{2.042168in}}%
\pgfpathlineto{\pgfqpoint{1.189901in}{2.128383in}}%
\pgfpathlineto{\pgfqpoint{1.209134in}{2.213074in}}%
\pgfpathlineto{\pgfqpoint{1.236734in}{2.295480in}}%
\pgfpathlineto{\pgfqpoint{1.272459in}{2.374725in}}%
\pgfpathlineto{\pgfqpoint{1.315982in}{2.449988in}}%
\pgfpathlineto{\pgfqpoint{1.366893in}{2.520509in}}%
\pgfpathlineto{\pgfqpoint{1.424701in}{2.585587in}}%
\pgfpathlineto{\pgfqpoint{1.488832in}{2.644586in}}%
\pgfpathlineto{\pgfqpoint{1.558629in}{2.696930in}}%
\pgfpathlineto{\pgfqpoint{1.633353in}{2.742102in}}%
\pgfpathlineto{\pgfqpoint{1.712183in}{2.779651in}}%
\pgfpathlineto{\pgfqpoint{1.794339in}{2.809240in}}%
\pgfpathlineto{\pgfqpoint{1.879067in}{2.830586in}}%
\pgfpathlineto{\pgfqpoint{1.965493in}{2.843421in}}%
\pgfpathlineto{\pgfqpoint{2.052749in}{2.847576in}}%
\pgfpathlineto{\pgfqpoint{2.139973in}{2.842987in}}%
\pgfpathlineto{\pgfqpoint{2.226308in}{2.829692in}}%
\pgfpathlineto{\pgfqpoint{2.310902in}{2.807829in}}%
\pgfpathlineto{\pgfqpoint{2.392911in}{2.777641in}}%
\pgfpathlineto{\pgfqpoint{2.471494in}{2.739472in}}%
\pgfpathlineto{\pgfqpoint{2.545817in}{2.693768in}}%
\pgfpathlineto{\pgfqpoint{2.615195in}{2.640975in}}%
\pgfpathlineto{\pgfqpoint{2.678950in}{2.581578in}}%
\pgfpathlineto{\pgfqpoint{2.736394in}{2.516165in}}%
\pgfpathlineto{\pgfqpoint{2.786931in}{2.445375in}}%
\pgfpathlineto{\pgfqpoint{2.830046in}{2.369897in}}%
\pgfpathlineto{\pgfqpoint{2.865317in}{2.290475in}}%
\pgfpathlineto{\pgfqpoint{2.892406in}{2.207905in}}%
\pgfpathlineto{\pgfqpoint{2.911062in}{2.123035in}}%
\pgfpathlineto{\pgfqpoint{2.921122in}{2.036765in}}%
\pgfpathlineto{\pgfqpoint{2.922522in}{1.949955in}}%
\pgfpathlineto{\pgfqpoint{2.915250in}{1.863407in}}%
\pgfpathlineto{\pgfqpoint{2.899331in}{1.778031in}}%
\pgfpathlineto{\pgfqpoint{2.874886in}{1.694702in}}%
\pgfpathlineto{\pgfqpoint{2.842131in}{1.614252in}}%
\pgfpathlineto{\pgfqpoint{2.801378in}{1.537475in}}%
\pgfpathlineto{\pgfqpoint{2.753034in}{1.465125in}}%
\pgfpathlineto{\pgfqpoint{2.697600in}{1.397916in}}%
\pgfpathlineto{\pgfqpoint{2.635675in}{1.336521in}}%
\pgfpathlineto{\pgfqpoint{2.567952in}{1.281576in}}%
\pgfpathlineto{\pgfqpoint{2.495123in}{1.233603in}}%
\pgfpathlineto{\pgfqpoint{2.417839in}{1.193040in}}%
\pgfpathlineto{\pgfqpoint{2.336887in}{1.160349in}}%
\pgfpathlineto{\pgfqpoint{2.253070in}{1.135896in}}%
\pgfpathlineto{\pgfqpoint{2.167214in}{1.119945in}}%
\pgfpathlineto{\pgfqpoint{2.080162in}{1.112659in}}%
\pgfpathlineto{\pgfqpoint{1.992773in}{1.114102in}}%
\pgfpathlineto{\pgfqpoint{1.905927in}{1.124237in}}%
\pgfpathlineto{\pgfqpoint{1.820522in}{1.142926in}}%
\pgfpathlineto{\pgfqpoint{1.737467in}{1.169934in}}%
\pgfpathlineto{\pgfqpoint{1.657515in}{1.204992in}}%
\pgfpathlineto{\pgfqpoint{1.581465in}{1.247782in}}%
\pgfpathlineto{\pgfqpoint{1.510124in}{1.297897in}}%
\pgfpathlineto{\pgfqpoint{1.444224in}{1.354859in}}%
\pgfpathlineto{\pgfqpoint{1.384426in}{1.418118in}}%
\pgfpathlineto{\pgfqpoint{1.331320in}{1.487053in}}%
\pgfpathlineto{\pgfqpoint{1.285421in}{1.560969in}}%
\pgfpathlineto{\pgfqpoint{1.247173in}{1.639100in}}%
\pgfpathlineto{\pgfqpoint{1.216948in}{1.720607in}}%
\pgfpathlineto{\pgfqpoint{1.195022in}{1.804644in}}%
\pgfpathlineto{\pgfqpoint{1.181582in}{1.890457in}}%
\pgfpathlineto{\pgfqpoint{1.176816in}{1.977158in}}%
\pgfpathlineto{\pgfqpoint{1.180815in}{2.063865in}}%
\pgfpathlineto{\pgfqpoint{1.193569in}{2.149717in}}%
\pgfpathlineto{\pgfqpoint{1.214963in}{2.233867in}}%
\pgfpathlineto{\pgfqpoint{1.244785in}{2.315488in}}%
\pgfpathlineto{\pgfqpoint{1.282717in}{2.393771in}}%
\pgfpathlineto{\pgfqpoint{1.328343in}{2.467924in}}%
\pgfpathlineto{\pgfqpoint{1.381141in}{2.537174in}}%
\pgfpathlineto{\pgfqpoint{1.440550in}{2.600831in}}%
\pgfpathlineto{\pgfqpoint{1.506046in}{2.658327in}}%
\pgfpathlineto{\pgfqpoint{1.576975in}{2.709034in}}%
\pgfpathlineto{\pgfqpoint{1.652638in}{2.752408in}}%
\pgfpathlineto{\pgfqpoint{1.732295in}{2.788001in}}%
\pgfpathlineto{\pgfqpoint{1.815165in}{2.815457in}}%
\pgfpathlineto{\pgfqpoint{1.900425in}{2.834513in}}%
\pgfpathlineto{\pgfqpoint{1.987214in}{2.844998in}}%
\pgfpathlineto{\pgfqpoint{2.074627in}{2.846835in}}%
\pgfpathlineto{\pgfqpoint{2.161735in}{2.840042in}}%
\pgfpathlineto{\pgfqpoint{2.247759in}{2.824704in}}%
\pgfpathlineto{\pgfqpoint{2.331821in}{2.800941in}}%
\pgfpathlineto{\pgfqpoint{2.413040in}{2.768956in}}%
\pgfpathlineto{\pgfqpoint{2.490594in}{2.729036in}}%
\pgfpathlineto{\pgfqpoint{2.563708in}{2.681558in}}%
\pgfpathlineto{\pgfqpoint{2.631664in}{2.626988in}}%
\pgfpathlineto{\pgfqpoint{2.693796in}{2.565879in}}%
\pgfpathlineto{\pgfqpoint{2.749492in}{2.498871in}}%
\pgfpathlineto{\pgfqpoint{2.798192in}{2.426694in}}%
\pgfpathlineto{\pgfqpoint{2.839405in}{2.350140in}}%
\pgfpathlineto{\pgfqpoint{2.872787in}{2.269885in}}%
\pgfpathlineto{\pgfqpoint{2.897962in}{2.186734in}}%
\pgfpathlineto{\pgfqpoint{2.914630in}{2.101525in}}%
\pgfpathlineto{\pgfqpoint{2.922594in}{2.015105in}}%
\pgfpathlineto{\pgfqpoint{2.921762in}{1.928324in}}%
\pgfpathlineto{\pgfqpoint{2.912142in}{1.842033in}}%
\pgfpathlineto{\pgfqpoint{2.893846in}{1.757092in}}%
\pgfpathlineto{\pgfqpoint{2.867089in}{1.674361in}}%
\pgfpathlineto{\pgfqpoint{2.832189in}{1.594706in}}%
\pgfpathlineto{\pgfqpoint{2.789538in}{1.518945in}}%
\pgfpathlineto{\pgfqpoint{2.739509in}{1.447751in}}%
\pgfpathlineto{\pgfqpoint{2.682590in}{1.381883in}}%
\pgfpathlineto{\pgfqpoint{2.619333in}{1.322030in}}%
\pgfpathlineto{\pgfqpoint{2.550355in}{1.268800in}}%
\pgfpathlineto{\pgfqpoint{2.476332in}{1.222722in}}%
\pgfpathlineto{\pgfqpoint{2.398000in}{1.184244in}}%
\pgfpathlineto{\pgfqpoint{2.316160in}{1.153735in}}%
\pgfpathlineto{\pgfqpoint{2.231670in}{1.131480in}}%
\pgfpathlineto{\pgfqpoint{2.145414in}{1.117677in}}%
\pgfpathlineto{\pgfqpoint{2.058173in}{1.112430in}}%
\pgfpathlineto{\pgfqpoint{1.970851in}{1.115839in}}%
\pgfpathlineto{\pgfqpoint{1.884342in}{1.127914in}}%
\pgfpathlineto{\pgfqpoint{1.799514in}{1.148564in}}%
\pgfpathlineto{\pgfqpoint{1.717203in}{1.177603in}}%
\pgfpathlineto{\pgfqpoint{1.638219in}{1.214742in}}%
\pgfpathlineto{\pgfqpoint{1.563341in}{1.259597in}}%
\pgfpathlineto{\pgfqpoint{1.493320in}{1.311684in}}%
\pgfpathlineto{\pgfqpoint{1.428878in}{1.370419in}}%
\pgfpathlineto{\pgfqpoint{1.370671in}{1.435160in}}%
\pgfpathlineto{\pgfqpoint{1.319202in}{1.505338in}}%
\pgfpathlineto{\pgfqpoint{1.275034in}{1.580253in}}%
\pgfpathlineto{\pgfqpoint{1.238652in}{1.659161in}}%
\pgfpathlineto{\pgfqpoint{1.210441in}{1.741286in}}%
\pgfpathlineto{\pgfqpoint{1.190688in}{1.825824in}}%
\pgfpathlineto{\pgfqpoint{1.179582in}{1.911941in}}%
\pgfpathlineto{\pgfqpoint{1.177217in}{1.998771in}}%
\pgfpathlineto{\pgfqpoint{1.183584in}{2.085420in}}%
\pgfpathlineto{\pgfqpoint{1.198580in}{2.170963in}}%
\pgfpathlineto{\pgfqpoint{1.222034in}{2.254598in}}%
\pgfpathlineto{\pgfqpoint{1.253738in}{2.335518in}}%
\pgfpathlineto{\pgfqpoint{1.293406in}{2.412866in}}%
\pgfpathlineto{\pgfqpoint{1.340670in}{2.485845in}}%
\pgfpathlineto{\pgfqpoint{1.395081in}{2.553723in}}%
\pgfpathlineto{\pgfqpoint{1.456109in}{2.615830in}}%
\pgfpathlineto{\pgfqpoint{1.523142in}{2.671558in}}%
\pgfpathlineto{\pgfqpoint{1.595487in}{2.720363in}}%
\pgfpathlineto{\pgfqpoint{1.672371in}{2.761765in}}%
\pgfpathlineto{\pgfqpoint{1.752948in}{2.795350in}}%
\pgfpathlineto{\pgfqpoint{1.836487in}{2.820840in}}%
\pgfpathlineto{\pgfqpoint{1.922166in}{2.837952in}}%
\pgfpathlineto{\pgfqpoint{2.009109in}{2.846464in}}%
\pgfpathlineto{\pgfqpoint{2.096447in}{2.846256in}}%
\pgfpathlineto{\pgfqpoint{2.183320in}{2.837313in}}%
\pgfpathlineto{\pgfqpoint{2.268874in}{2.819720in}}%
\pgfpathlineto{\pgfqpoint{2.352263in}{2.793664in}}%
\pgfpathlineto{\pgfqpoint{2.432648in}{2.759436in}}%
\pgfpathlineto{\pgfqpoint{2.509197in}{2.717428in}}%
\pgfpathlineto{\pgfqpoint{2.581108in}{2.668121in}}%
\pgfpathlineto{\pgfqpoint{2.647749in}{2.611957in}}%
\pgfpathlineto{\pgfqpoint{2.708422in}{2.549480in}}%
\pgfpathlineto{\pgfqpoint{2.762481in}{2.481303in}}%
\pgfpathlineto{\pgfqpoint{2.809366in}{2.408091in}}%
\pgfpathlineto{\pgfqpoint{2.848606in}{2.330562in}}%
\pgfpathlineto{\pgfqpoint{2.879819in}{2.249481in}}%
\pgfpathlineto{\pgfqpoint{2.902709in}{2.165666in}}%
\pgfpathlineto{\pgfqpoint{2.917070in}{2.079985in}}%
\pgfpathlineto{\pgfqpoint{2.922785in}{1.993355in}}%
\pgfpathlineto{\pgfqpoint{2.919834in}{1.906580in}}%
\pgfpathlineto{\pgfqpoint{2.908222in}{1.820515in}}%
\pgfpathlineto{\pgfqpoint{2.888021in}{1.736060in}}%
\pgfpathlineto{\pgfqpoint{2.859399in}{1.654074in}}%
\pgfpathlineto{\pgfqpoint{2.822616in}{1.575371in}}%
\pgfpathlineto{\pgfqpoint{2.778028in}{1.500729in}}%
\pgfpathlineto{\pgfqpoint{2.726086in}{1.430878in}}%
\pgfpathlineto{\pgfqpoint{2.667335in}{1.366511in}}%
\pgfpathlineto{\pgfqpoint{2.602414in}{1.308277in}}%
\pgfpathlineto{\pgfqpoint{2.532053in}{1.256780in}}%
\pgfpathlineto{\pgfqpoint{2.456908in}{1.212469in}}%
\pgfpathlineto{\pgfqpoint{2.377694in}{1.175799in}}%
\pgfpathlineto{\pgfqpoint{2.295214in}{1.147187in}}%
\pgfpathlineto{\pgfqpoint{2.210284in}{1.126952in}}%
\pgfpathlineto{\pgfqpoint{2.123739in}{1.115311in}}%
\pgfpathlineto{\pgfqpoint{2.036429in}{1.112382in}}%
\pgfpathlineto{\pgfqpoint{1.949220in}{1.118180in}}%
\pgfpathlineto{\pgfqpoint{1.862996in}{1.132621in}}%
\pgfpathlineto{\pgfqpoint{1.778655in}{1.155519in}}%
\pgfpathlineto{\pgfqpoint{1.697082in}{1.186598in}}%
\pgfpathlineto{\pgfqpoint{1.618995in}{1.225575in}}%
\pgfpathlineto{\pgfqpoint{1.545212in}{1.272084in}}%
\pgfpathlineto{\pgfqpoint{1.476509in}{1.325682in}}%
\pgfpathlineto{\pgfqpoint{1.413585in}{1.385854in}}%
\pgfpathlineto{\pgfqpoint{1.357068in}{1.452015in}}%
\pgfpathlineto{\pgfqpoint{1.307513in}{1.523512in}}%
\pgfpathlineto{\pgfqpoint{1.265399in}{1.599618in}}%
\pgfpathlineto{\pgfqpoint{1.231132in}{1.679538in}}%
\pgfpathlineto{\pgfqpoint{1.205045in}{1.762404in}}%
\pgfpathlineto{\pgfqpoint{1.187353in}{1.847422in}}%
\pgfpathlineto{\pgfqpoint{1.178241in}{1.933782in}}%
\pgfpathlineto{\pgfqpoint{1.177851in}{2.020589in}}%
\pgfpathlineto{\pgfqpoint{1.186228in}{2.106967in}}%
\pgfpathlineto{\pgfqpoint{1.203311in}{2.192062in}}%
\pgfpathlineto{\pgfqpoint{1.228940in}{2.275035in}}%
\pgfpathlineto{\pgfqpoint{1.262853in}{2.355070in}}%
\pgfpathlineto{\pgfqpoint{1.304686in}{2.431369in}}%
\pgfpathlineto{\pgfqpoint{1.353974in}{2.503155in}}%
\pgfpathlineto{\pgfqpoint{1.410151in}{2.569671in}}%
\pgfpathlineto{\pgfqpoint{1.472678in}{2.630309in}}%
\pgfpathlineto{\pgfqpoint{1.540971in}{2.684475in}}%
\pgfpathlineto{\pgfqpoint{1.614351in}{2.731576in}}%
\pgfpathlineto{\pgfqpoint{1.692095in}{2.771111in}}%
\pgfpathlineto{\pgfqpoint{1.773443in}{2.802677in}}%
\pgfpathlineto{\pgfqpoint{1.857595in}{2.825960in}}%
\pgfpathlineto{\pgfqpoint{1.943713in}{2.840742in}}%
\pgfpathlineto{\pgfqpoint{2.030917in}{2.846899in}}%
\pgfpathlineto{\pgfqpoint{2.118291in}{2.844400in}}%
\pgfpathlineto{\pgfqpoint{2.204935in}{2.833306in}}%
\pgfpathlineto{\pgfqpoint{2.290076in}{2.813726in}}%
\pgfpathlineto{\pgfqpoint{2.372817in}{2.785820in}}%
\pgfpathlineto{\pgfqpoint{2.452299in}{2.749832in}}%
\pgfpathlineto{\pgfqpoint{2.527720in}{2.706094in}}%
\pgfpathlineto{\pgfqpoint{2.598334in}{2.655024in}}%
\pgfpathlineto{\pgfqpoint{2.663446in}{2.597128in}}%
\pgfpathlineto{\pgfqpoint{2.722419in}{2.532999in}}%
\pgfpathlineto{\pgfqpoint{2.774670in}{2.463315in}}%
\pgfpathlineto{\pgfqpoint{2.819669in}{2.388844in}}%
\pgfpathlineto{\pgfqpoint{2.856989in}{2.310360in}}%
\pgfpathlineto{\pgfqpoint{2.886299in}{2.228559in}}%
\pgfpathlineto{\pgfqpoint{2.907249in}{2.144279in}}%
\pgfpathlineto{\pgfqpoint{2.919589in}{2.058367in}}%
\pgfpathlineto{\pgfqpoint{2.923168in}{1.971672in}}%
\pgfpathlineto{\pgfqpoint{2.917944in}{1.885046in}}%
\pgfpathlineto{\pgfqpoint{2.903972in}{1.799341in}}%
\pgfpathlineto{\pgfqpoint{2.881414in}{1.715414in}}%
\pgfpathlineto{\pgfqpoint{2.850533in}{1.634124in}}%
\pgfpathlineto{\pgfqpoint{2.811696in}{1.556331in}}%
\pgfpathlineto{\pgfqpoint{2.765300in}{1.482781in}}%
\pgfpathlineto{\pgfqpoint{2.711761in}{1.414169in}}%
\pgfpathlineto{\pgfqpoint{2.651606in}{1.351230in}}%
\pgfpathlineto{\pgfqpoint{2.585419in}{1.294618in}}%
\pgfpathlineto{\pgfqpoint{2.513845in}{1.244905in}}%
\pgfpathlineto{\pgfqpoint{2.437589in}{1.202581in}}%
\pgfpathlineto{\pgfqpoint{2.357414in}{1.168055in}}%
\pgfpathlineto{\pgfqpoint{2.274145in}{1.141654in}}%
\pgfpathlineto{\pgfqpoint{2.188665in}{1.123623in}}%
\pgfpathlineto{\pgfqpoint{2.101829in}{1.114107in}}%
\pgfpathlineto{\pgfqpoint{2.014443in}{1.113192in}}%
\pgfpathlineto{\pgfqpoint{1.927422in}{1.120939in}}%
\pgfpathlineto{\pgfqpoint{1.841652in}{1.137310in}}%
\pgfpathlineto{\pgfqpoint{1.757988in}{1.162169in}}%
\pgfpathlineto{\pgfqpoint{1.677255in}{1.195283in}}%
\pgfpathlineto{\pgfqpoint{1.600246in}{1.236319in}}%
\pgfpathlineto{\pgfqpoint{1.527722in}{1.284848in}}%
\pgfpathlineto{\pgfqpoint{1.460416in}{1.340341in}}%
\pgfpathlineto{\pgfqpoint{1.399028in}{1.402173in}}%
\pgfpathlineto{\pgfqpoint{1.344148in}{1.469708in}}%
\pgfpathlineto{\pgfqpoint{1.296275in}{1.542343in}}%
\pgfpathlineto{\pgfqpoint{1.255948in}{1.619345in}}%
\pgfpathlineto{\pgfqpoint{1.223604in}{1.699951in}}%
\pgfpathlineto{\pgfqpoint{1.199585in}{1.783370in}}%
\pgfpathlineto{\pgfqpoint{1.184133in}{1.868784in}}%
\pgfpathlineto{\pgfqpoint{1.177392in}{1.955346in}}%
\pgfpathlineto{\pgfqpoint{1.179406in}{2.042183in}}%
\pgfpathlineto{\pgfqpoint{1.190121in}{2.128394in}}%
\pgfpathlineto{\pgfqpoint{1.209388in}{2.213061in}}%
\pgfpathlineto{\pgfqpoint{1.237011in}{2.295427in}}%
\pgfpathlineto{\pgfqpoint{1.272743in}{2.374656in}}%
\pgfpathlineto{\pgfqpoint{1.316257in}{2.449912in}}%
\pgfpathlineto{\pgfqpoint{1.367143in}{2.520426in}}%
\pgfpathlineto{\pgfqpoint{1.424915in}{2.585493in}}%
\pgfpathlineto{\pgfqpoint{1.489005in}{2.644473in}}%
\pgfpathlineto{\pgfqpoint{1.558766in}{2.696792in}}%
\pgfpathlineto{\pgfqpoint{1.633471in}{2.741937in}}%
\pgfpathlineto{\pgfqpoint{1.712315in}{2.779463in}}%
\pgfpathlineto{\pgfqpoint{1.794455in}{2.809010in}}%
\pgfpathlineto{\pgfqpoint{1.879165in}{2.830330in}}%
\pgfpathlineto{\pgfqpoint{1.965580in}{2.843160in}}%
\pgfpathlineto{\pgfqpoint{2.052822in}{2.847326in}}%
\pgfpathlineto{\pgfqpoint{2.140025in}{2.842758in}}%
\pgfpathlineto{\pgfqpoint{2.226329in}{2.829486in}}%
\pgfpathlineto{\pgfqpoint{2.310887in}{2.807643in}}%
\pgfpathlineto{\pgfqpoint{2.392859in}{2.777464in}}%
\pgfpathlineto{\pgfqpoint{2.471415in}{2.739286in}}%
\pgfpathlineto{\pgfqpoint{2.545736in}{2.693550in}}%
\pgfpathlineto{\pgfqpoint{2.615069in}{2.640752in}}%
\pgfpathlineto{\pgfqpoint{2.678798in}{2.581358in}}%
\pgfpathlineto{\pgfqpoint{2.736236in}{2.515955in}}%
\pgfpathlineto{\pgfqpoint{2.786774in}{2.445183in}}%
\pgfpathlineto{\pgfqpoint{2.829893in}{2.369735in}}%
\pgfpathlineto{\pgfqpoint{2.865161in}{2.290350in}}%
\pgfpathlineto{\pgfqpoint{2.892239in}{2.207815in}}%
\pgfpathlineto{\pgfqpoint{2.910875in}{2.122969in}}%
\pgfpathlineto{\pgfqpoint{2.920907in}{2.036696in}}%
\pgfpathlineto{\pgfqpoint{2.922265in}{1.949912in}}%
\pgfpathlineto{\pgfqpoint{2.914963in}{1.863394in}}%
\pgfpathlineto{\pgfqpoint{2.899036in}{1.778030in}}%
\pgfpathlineto{\pgfqpoint{2.874603in}{1.694707in}}%
\pgfpathlineto{\pgfqpoint{2.841876in}{1.614267in}}%
\pgfpathlineto{\pgfqpoint{2.801160in}{1.537509in}}%
\pgfpathlineto{\pgfqpoint{2.752854in}{1.465188in}}%
\pgfpathlineto{\pgfqpoint{2.697450in}{1.398013in}}%
\pgfpathlineto{\pgfqpoint{2.635534in}{1.336650in}}%
\pgfpathlineto{\pgfqpoint{2.567784in}{1.281723in}}%
\pgfpathlineto{\pgfqpoint{2.494950in}{1.233793in}}%
\pgfpathlineto{\pgfqpoint{2.417676in}{1.193264in}}%
\pgfpathlineto{\pgfqpoint{2.336731in}{1.160582in}}%
\pgfpathlineto{\pgfqpoint{2.252929in}{1.136120in}}%
\pgfpathlineto{\pgfqpoint{2.167098in}{1.120151in}}%
\pgfpathlineto{\pgfqpoint{2.080081in}{1.112848in}}%
\pgfpathlineto{\pgfqpoint{1.992734in}{1.114279in}}%
\pgfpathlineto{\pgfqpoint{1.905927in}{1.124415in}}%
\pgfpathlineto{\pgfqpoint{1.820544in}{1.143123in}}%
\pgfpathlineto{\pgfqpoint{1.737483in}{1.170170in}}%
\pgfpathlineto{\pgfqpoint{1.657584in}{1.205250in}}%
\pgfpathlineto{\pgfqpoint{1.581566in}{1.248052in}}%
\pgfpathlineto{\pgfqpoint{1.510237in}{1.298166in}}%
\pgfpathlineto{\pgfqpoint{1.444342in}{1.355112in}}%
\pgfpathlineto{\pgfqpoint{1.384550in}{1.418341in}}%
\pgfpathlineto{\pgfqpoint{1.331454in}{1.487236in}}%
\pgfpathlineto{\pgfqpoint{1.285572in}{1.561110in}}%
\pgfpathlineto{\pgfqpoint{1.247347in}{1.639210in}}%
\pgfpathlineto{\pgfqpoint{1.217146in}{1.720711in}}%
\pgfpathlineto{\pgfqpoint{1.195256in}{1.804737in}}%
\pgfpathlineto{\pgfqpoint{1.181846in}{1.890526in}}%
\pgfpathlineto{\pgfqpoint{1.177089in}{1.977214in}}%
\pgfpathlineto{\pgfqpoint{1.181080in}{2.063910in}}%
\pgfpathlineto{\pgfqpoint{1.193813in}{2.149744in}}%
\pgfpathlineto{\pgfqpoint{1.215181in}{2.233869in}}%
\pgfpathlineto{\pgfqpoint{1.244976in}{2.315458in}}%
\pgfpathlineto{\pgfqpoint{1.282891in}{2.393705in}}%
\pgfpathlineto{\pgfqpoint{1.328514in}{2.467828in}}%
\pgfpathlineto{\pgfqpoint{1.381335in}{2.537064in}}%
\pgfpathlineto{\pgfqpoint{1.440757in}{2.600689in}}%
\pgfpathlineto{\pgfqpoint{1.506247in}{2.658153in}}%
\pgfpathlineto{\pgfqpoint{1.577168in}{2.708848in}}%
\pgfpathlineto{\pgfqpoint{1.652815in}{2.752224in}}%
\pgfpathlineto{\pgfqpoint{1.732446in}{2.787824in}}%
\pgfpathlineto{\pgfqpoint{1.815281in}{2.815285in}}%
\pgfpathlineto{\pgfqpoint{1.900503in}{2.834339in}}%
\pgfpathlineto{\pgfqpoint{1.987259in}{2.844813in}}%
\pgfpathlineto{\pgfqpoint{2.074657in}{2.846629in}}%
\pgfpathlineto{\pgfqpoint{2.161767in}{2.839802in}}%
\pgfpathlineto{\pgfqpoint{2.247744in}{2.824433in}}%
\pgfpathlineto{\pgfqpoint{2.331783in}{2.800656in}}%
\pgfpathlineto{\pgfqpoint{2.412993in}{2.768673in}}%
\pgfpathlineto{\pgfqpoint{2.490540in}{2.728773in}}%
\pgfpathlineto{\pgfqpoint{2.563642in}{2.681327in}}%
\pgfpathlineto{\pgfqpoint{2.631579in}{2.626796in}}%
\pgfpathlineto{\pgfqpoint{2.693685in}{2.565724in}}%
\pgfpathlineto{\pgfqpoint{2.749351in}{2.498740in}}%
\pgfpathlineto{\pgfqpoint{2.798025in}{2.426562in}}%
\pgfpathlineto{\pgfqpoint{2.839211in}{2.349990in}}%
\pgfpathlineto{\pgfqpoint{2.872551in}{2.269757in}}%
\pgfpathlineto{\pgfqpoint{2.897709in}{2.186617in}}%
\pgfpathlineto{\pgfqpoint{2.914380in}{2.101420in}}%
\pgfpathlineto{\pgfqpoint{2.922360in}{2.015019in}}%
\pgfpathlineto{\pgfqpoint{2.921548in}{1.928265in}}%
\pgfpathlineto{\pgfqpoint{2.911947in}{1.842010in}}%
\pgfpathlineto{\pgfqpoint{2.893663in}{1.757108in}}%
\pgfpathlineto{\pgfqpoint{2.866903in}{1.674409in}}%
\pgfpathlineto{\pgfqpoint{2.831980in}{1.594767in}}%
\pgfpathlineto{\pgfqpoint{2.789304in}{1.519026in}}%
\pgfpathlineto{\pgfqpoint{2.739273in}{1.447868in}}%
\pgfpathlineto{\pgfqpoint{2.682358in}{1.382013in}}%
\pgfpathlineto{\pgfqpoint{2.619117in}{1.322162in}}%
\pgfpathlineto{\pgfqpoint{2.550165in}{1.268932in}}%
\pgfpathlineto{\pgfqpoint{2.476175in}{1.222858in}}%
\pgfpathlineto{\pgfqpoint{2.397879in}{1.184392in}}%
\pgfpathlineto{\pgfqpoint{2.316065in}{1.153902in}}%
\pgfpathlineto{\pgfqpoint{2.231579in}{1.131675in}}%
\pgfpathlineto{\pgfqpoint{2.145326in}{1.117912in}}%
\pgfpathlineto{\pgfqpoint{2.058120in}{1.112707in}}%
\pgfpathlineto{\pgfqpoint{1.970811in}{1.116132in}}%
\pgfpathlineto{\pgfqpoint{1.884309in}{1.128199in}}%
\pgfpathlineto{\pgfqpoint{1.799491in}{1.148825in}}%
\pgfpathlineto{\pgfqpoint{1.717200in}{1.177828in}}%
\pgfpathlineto{\pgfqpoint{1.638246in}{1.214930in}}%
\pgfpathlineto{\pgfqpoint{1.563406in}{1.259756in}}%
\pgfpathlineto{\pgfqpoint{1.493421in}{1.311832in}}%
\pgfpathlineto{\pgfqpoint{1.429001in}{1.370589in}}%
\pgfpathlineto{\pgfqpoint{1.370819in}{1.435360in}}%
\pgfpathlineto{\pgfqpoint{1.319398in}{1.505530in}}%
\pgfpathlineto{\pgfqpoint{1.275246in}{1.580441in}}%
\pgfpathlineto{\pgfqpoint{1.238860in}{1.659337in}}%
\pgfpathlineto{\pgfqpoint{1.210634in}{1.741440in}}%
\pgfpathlineto{\pgfqpoint{1.190865in}{1.825943in}}%
\pgfpathlineto{\pgfqpoint{1.179750in}{1.912016in}}%
\pgfpathlineto{\pgfqpoint{1.177388in}{1.998805in}}%
\pgfpathlineto{\pgfqpoint{1.183775in}{2.085429in}}%
\pgfpathlineto{\pgfqpoint{1.198811in}{2.170983in}}%
\pgfpathlineto{\pgfqpoint{1.222302in}{2.254571in}}%
\pgfpathlineto{\pgfqpoint{1.254029in}{2.335457in}}%
\pgfpathlineto{\pgfqpoint{1.293702in}{2.412792in}}%
\pgfpathlineto{\pgfqpoint{1.340952in}{2.485768in}}%
\pgfpathlineto{\pgfqpoint{1.395332in}{2.553642in}}%
\pgfpathlineto{\pgfqpoint{1.456317in}{2.615737in}}%
\pgfpathlineto{\pgfqpoint{1.523304in}{2.671445in}}%
\pgfpathlineto{\pgfqpoint{1.595613in}{2.720223in}}%
\pgfpathlineto{\pgfqpoint{1.672486in}{2.761595in}}%
\pgfpathlineto{\pgfqpoint{1.753086in}{2.795153in}}%
\pgfpathlineto{\pgfqpoint{1.836597in}{2.820593in}}%
\pgfpathlineto{\pgfqpoint{1.922263in}{2.837683in}}%
\pgfpathlineto{\pgfqpoint{2.009197in}{2.846197in}}%
\pgfpathlineto{\pgfqpoint{2.096522in}{2.846007in}}%
\pgfpathlineto{\pgfqpoint{2.183371in}{2.837091in}}%
\pgfpathlineto{\pgfqpoint{2.268890in}{2.819526in}}%
\pgfpathlineto{\pgfqpoint{2.352239in}{2.793491in}}%
\pgfpathlineto{\pgfqpoint{2.432587in}{2.759268in}}%
\pgfpathlineto{\pgfqpoint{2.509115in}{2.717241in}}%
\pgfpathlineto{\pgfqpoint{2.581018in}{2.667895in}}%
\pgfpathlineto{\pgfqpoint{2.647609in}{2.611730in}}%
\pgfpathlineto{\pgfqpoint{2.708264in}{2.549254in}}%
\pgfpathlineto{\pgfqpoint{2.762322in}{2.481088in}}%
\pgfpathlineto{\pgfqpoint{2.809213in}{2.407901in}}%
\pgfpathlineto{\pgfqpoint{2.848459in}{2.330406in}}%
\pgfpathlineto{\pgfqpoint{2.879669in}{2.249365in}}%
\pgfpathlineto{\pgfqpoint{2.902546in}{2.165586in}}%
\pgfpathlineto{\pgfqpoint{2.916883in}{2.079924in}}%
\pgfpathlineto{\pgfqpoint{2.922561in}{1.993278in}}%
\pgfpathlineto{\pgfqpoint{2.919560in}{1.906548in}}%
\pgfpathlineto{\pgfqpoint{2.907919in}{1.820508in}}%
\pgfpathlineto{\pgfqpoint{2.887715in}{1.736060in}}%
\pgfpathlineto{\pgfqpoint{2.859111in}{1.654077in}}%
\pgfpathlineto{\pgfqpoint{2.822364in}{1.575384in}}%
\pgfpathlineto{\pgfqpoint{2.777820in}{1.500762in}}%
\pgfpathlineto{\pgfqpoint{2.725921in}{1.430943in}}%
\pgfpathlineto{\pgfqpoint{2.667198in}{1.366613in}}%
\pgfpathlineto{\pgfqpoint{2.602276in}{1.308411in}}%
\pgfpathlineto{\pgfqpoint{2.531870in}{1.256929in}}%
\pgfpathlineto{\pgfqpoint{2.456735in}{1.212677in}}%
\pgfpathlineto{\pgfqpoint{2.377529in}{1.176036in}}%
\pgfpathlineto{\pgfqpoint{2.295055in}{1.147426in}}%
\pgfpathlineto{\pgfqpoint{2.210141in}{1.127175in}}%
\pgfpathlineto{\pgfqpoint{2.123624in}{1.115511in}}%
\pgfpathlineto{\pgfqpoint{2.036354in}{1.112561in}}%
\pgfpathlineto{\pgfqpoint{1.949190in}{1.118347in}}%
\pgfpathlineto{\pgfqpoint{1.863005in}{1.132793in}}%
\pgfpathlineto{\pgfqpoint{1.778680in}{1.155719in}}%
\pgfpathlineto{\pgfqpoint{1.697107in}{1.186845in}}%
\pgfpathlineto{\pgfqpoint{1.619078in}{1.225843in}}%
\pgfpathlineto{\pgfqpoint{1.545322in}{1.272363in}}%
\pgfpathlineto{\pgfqpoint{1.476626in}{1.325957in}}%
\pgfpathlineto{\pgfqpoint{1.413705in}{1.386109in}}%
\pgfpathlineto{\pgfqpoint{1.357193in}{1.452235in}}%
\pgfpathlineto{\pgfqpoint{1.307648in}{1.523687in}}%
\pgfpathlineto{\pgfqpoint{1.265553in}{1.599750in}}%
\pgfpathlineto{\pgfqpoint{1.231311in}{1.679642in}}%
\pgfpathlineto{\pgfqpoint{1.205250in}{1.762516in}}%
\pgfpathlineto{\pgfqpoint{1.187603in}{1.847505in}}%
\pgfpathlineto{\pgfqpoint{1.178517in}{1.933846in}}%
\pgfpathlineto{\pgfqpoint{1.178130in}{2.020643in}}%
\pgfpathlineto{\pgfqpoint{1.186492in}{2.107010in}}%
\pgfpathlineto{\pgfqpoint{1.203548in}{2.192086in}}%
\pgfpathlineto{\pgfqpoint{1.229146in}{2.275032in}}%
\pgfpathlineto{\pgfqpoint{1.263032in}{2.355032in}}%
\pgfpathlineto{\pgfqpoint{1.304849in}{2.431294in}}%
\pgfpathlineto{\pgfqpoint{1.354143in}{2.503052in}}%
\pgfpathlineto{\pgfqpoint{1.410358in}{2.569562in}}%
\pgfpathlineto{\pgfqpoint{1.472884in}{2.630151in}}%
\pgfpathlineto{\pgfqpoint{1.541174in}{2.684290in}}%
\pgfpathlineto{\pgfqpoint{1.614545in}{2.731386in}}%
\pgfpathlineto{\pgfqpoint{1.692270in}{2.770928in}}%
\pgfpathlineto{\pgfqpoint{1.773588in}{2.802503in}}%
\pgfpathlineto{\pgfqpoint{1.857701in}{2.825792in}}%
\pgfpathlineto{\pgfqpoint{1.943779in}{2.840572in}}%
\pgfpathlineto{\pgfqpoint{2.030953in}{2.846714in}}%
\pgfpathlineto{\pgfqpoint{2.118321in}{2.844187in}}%
\pgfpathlineto{\pgfqpoint{2.204950in}{2.833053in}}%
\pgfpathlineto{\pgfqpoint{2.290049in}{2.813444in}}%
\pgfpathlineto{\pgfqpoint{2.372773in}{2.785526in}}%
\pgfpathlineto{\pgfqpoint{2.452249in}{2.749546in}}%
\pgfpathlineto{\pgfqpoint{2.527665in}{2.705833in}}%
\pgfpathlineto{\pgfqpoint{2.598266in}{2.654801in}}%
\pgfpathlineto{\pgfqpoint{2.663358in}{2.596947in}}%
\pgfpathlineto{\pgfqpoint{2.722302in}{2.532855in}}%
\pgfpathlineto{\pgfqpoint{2.774522in}{2.463191in}}%
\pgfpathlineto{\pgfqpoint{2.819497in}{2.388706in}}%
\pgfpathlineto{\pgfqpoint{2.856776in}{2.310219in}}%
\pgfpathlineto{\pgfqpoint{2.886049in}{2.228435in}}%
\pgfpathlineto{\pgfqpoint{2.906990in}{2.144165in}}%
\pgfpathlineto{\pgfqpoint{2.919339in}{2.058264in}}%
\pgfpathlineto{\pgfqpoint{2.922939in}{1.971590in}}%
\pgfpathlineto{\pgfqpoint{2.917739in}{1.884994in}}%
\pgfpathlineto{\pgfqpoint{2.903787in}{1.799329in}}%
\pgfpathlineto{\pgfqpoint{2.881238in}{1.715442in}}%
\pgfpathlineto{\pgfqpoint{2.850348in}{1.634181in}}%
\pgfpathlineto{\pgfqpoint{2.811475in}{1.556389in}}%
\pgfpathlineto{\pgfqpoint{2.765063in}{1.482879in}}%
\pgfpathlineto{\pgfqpoint{2.711522in}{1.414296in}}%
\pgfpathlineto{\pgfqpoint{2.651373in}{1.351365in}}%
\pgfpathlineto{\pgfqpoint{2.585205in}{1.294751in}}%
\pgfpathlineto{\pgfqpoint{2.513661in}{1.245036in}}%
\pgfpathlineto{\pgfqpoint{2.437442in}{1.202715in}}%
\pgfpathlineto{\pgfqpoint{2.357304in}{1.168202in}}%
\pgfpathlineto{\pgfqpoint{2.274057in}{1.141824in}}%
\pgfpathlineto{\pgfqpoint{2.188571in}{1.123825in}}%
\pgfpathlineto{\pgfqpoint{2.101755in}{1.114359in}}%
\pgfpathlineto{\pgfqpoint{2.014398in}{1.113483in}}%
\pgfpathlineto{\pgfqpoint{1.927386in}{1.121240in}}%
\pgfpathlineto{\pgfqpoint{1.841620in}{1.137598in}}%
\pgfpathlineto{\pgfqpoint{1.757967in}{1.162426in}}%
\pgfpathlineto{\pgfqpoint{1.677255in}{1.195499in}}%
\pgfpathlineto{\pgfqpoint{1.600279in}{1.236495in}}%
\pgfpathlineto{\pgfqpoint{1.527796in}{1.284995in}}%
\pgfpathlineto{\pgfqpoint{1.460525in}{1.340484in}}%
\pgfpathlineto{\pgfqpoint{1.399153in}{1.402353in}}%
\pgfpathlineto{\pgfqpoint{1.344316in}{1.469906in}}%
\pgfpathlineto{\pgfqpoint{1.296485in}{1.542535in}}%
\pgfpathlineto{\pgfqpoint{1.256167in}{1.619533in}}%
\pgfpathlineto{\pgfqpoint{1.223812in}{1.700126in}}%
\pgfpathlineto{\pgfqpoint{1.199775in}{1.783519in}}%
\pgfpathlineto{\pgfqpoint{1.184304in}{1.868894in}}%
\pgfpathlineto{\pgfqpoint{1.177554in}{1.955410in}}%
\pgfpathlineto{\pgfqpoint{1.179574in}{2.042207in}}%
\pgfpathlineto{\pgfqpoint{1.190316in}{2.128399in}}%
\pgfpathlineto{\pgfqpoint{1.209631in}{2.213078in}}%
\pgfpathlineto{\pgfqpoint{1.237289in}{2.295385in}}%
\pgfpathlineto{\pgfqpoint{1.273042in}{2.374584in}}%
\pgfpathlineto{\pgfqpoint{1.316558in}{2.449832in}}%
\pgfpathlineto{\pgfqpoint{1.367426in}{2.520345in}}%
\pgfpathlineto{\pgfqpoint{1.425162in}{2.585408in}}%
\pgfpathlineto{\pgfqpoint{1.489204in}{2.644377in}}%
\pgfpathlineto{\pgfqpoint{1.558918in}{2.696673in}}%
\pgfpathlineto{\pgfqpoint{1.633589in}{2.741789in}}%
\pgfpathlineto{\pgfqpoint{1.712432in}{2.779286in}}%
\pgfpathlineto{\pgfqpoint{1.794583in}{2.808794in}}%
\pgfpathlineto{\pgfqpoint{1.879267in}{2.830068in}}%
\pgfpathlineto{\pgfqpoint{1.965672in}{2.842883in}}%
\pgfpathlineto{\pgfqpoint{2.052907in}{2.847057in}}%
\pgfpathlineto{\pgfqpoint{2.140094in}{2.842512in}}%
\pgfpathlineto{\pgfqpoint{2.226372in}{2.829271in}}%
\pgfpathlineto{\pgfqpoint{2.310893in}{2.807458in}}%
\pgfpathlineto{\pgfqpoint{2.392823in}{2.777299in}}%
\pgfpathlineto{\pgfqpoint{2.471344in}{2.739122in}}%
\pgfpathlineto{\pgfqpoint{2.545651in}{2.693355in}}%
\pgfpathlineto{\pgfqpoint{2.614958in}{2.640526in}}%
\pgfpathlineto{\pgfqpoint{2.678645in}{2.581130in}}%
\pgfpathlineto{\pgfqpoint{2.736070in}{2.515730in}}%
\pgfpathlineto{\pgfqpoint{2.786612in}{2.444973in}}%
\pgfpathlineto{\pgfqpoint{2.829740in}{2.369552in}}%
\pgfpathlineto{\pgfqpoint{2.865015in}{2.290204in}}%
\pgfpathlineto{\pgfqpoint{2.892090in}{2.207712in}}%
\pgfpathlineto{\pgfqpoint{2.910711in}{2.122901in}}%
\pgfpathlineto{\pgfqpoint{2.920714in}{2.036639in}}%
\pgfpathlineto{\pgfqpoint{2.922028in}{1.949842in}}%
\pgfpathlineto{\pgfqpoint{2.914675in}{1.863375in}}%
\pgfpathlineto{\pgfqpoint{2.898724in}{1.778031in}}%
\pgfpathlineto{\pgfqpoint{2.874293in}{1.694712in}}%
\pgfpathlineto{\pgfqpoint{2.841590in}{1.614275in}}%
\pgfpathlineto{\pgfqpoint{2.800915in}{1.537527in}}%
\pgfpathlineto{\pgfqpoint{2.752657in}{1.465228in}}%
\pgfpathlineto{\pgfqpoint{2.697296in}{1.398086in}}%
\pgfpathlineto{\pgfqpoint{2.635404in}{1.336761in}}%
\pgfpathlineto{\pgfqpoint{2.567643in}{1.281865in}}%
\pgfpathlineto{\pgfqpoint{2.494765in}{1.233958in}}%
\pgfpathlineto{\pgfqpoint{2.417510in}{1.193488in}}%
\pgfpathlineto{\pgfqpoint{2.336570in}{1.160829in}}%
\pgfpathlineto{\pgfqpoint{2.252775in}{1.136363in}}%
\pgfpathlineto{\pgfqpoint{2.166961in}{1.120374in}}%
\pgfpathlineto{\pgfqpoint{2.079976in}{1.113044in}}%
\pgfpathlineto{\pgfqpoint{1.992671in}{1.114452in}}%
\pgfpathlineto{\pgfqpoint{1.905911in}{1.124578in}}%
\pgfpathlineto{\pgfqpoint{1.820564in}{1.143295in}}%
\pgfpathlineto{\pgfqpoint{1.737509in}{1.170379in}}%
\pgfpathlineto{\pgfqpoint{1.657630in}{1.205502in}}%
\pgfpathlineto{\pgfqpoint{1.581663in}{1.248322in}}%
\pgfpathlineto{\pgfqpoint{1.510356in}{1.298445in}}%
\pgfpathlineto{\pgfqpoint{1.444467in}{1.355385in}}%
\pgfpathlineto{\pgfqpoint{1.384676in}{1.418590in}}%
\pgfpathlineto{\pgfqpoint{1.331585in}{1.487446in}}%
\pgfpathlineto{\pgfqpoint{1.285714in}{1.561274in}}%
\pgfpathlineto{\pgfqpoint{1.247509in}{1.639332in}}%
\pgfpathlineto{\pgfqpoint{1.217334in}{1.720812in}}%
\pgfpathlineto{\pgfqpoint{1.195474in}{1.804843in}}%
\pgfpathlineto{\pgfqpoint{1.182109in}{1.890597in}}%
\pgfpathlineto{\pgfqpoint{1.177371in}{1.977270in}}%
\pgfpathlineto{\pgfqpoint{1.181358in}{2.063957in}}%
\pgfpathlineto{\pgfqpoint{1.194071in}{2.149779in}}%
\pgfpathlineto{\pgfqpoint{1.215410in}{2.233884in}}%
\pgfpathlineto{\pgfqpoint{1.245173in}{2.315443in}}%
\pgfpathlineto{\pgfqpoint{1.283061in}{2.393653in}}%
\pgfpathlineto{\pgfqpoint{1.328673in}{2.467740in}}%
\pgfpathlineto{\pgfqpoint{1.381509in}{2.536953in}}%
\pgfpathlineto{\pgfqpoint{1.440967in}{2.600566in}}%
\pgfpathlineto{\pgfqpoint{1.506450in}{2.657980in}}%
\pgfpathlineto{\pgfqpoint{1.577367in}{2.708656in}}%
\pgfpathlineto{\pgfqpoint{1.653002in}{2.752032in}}%
\pgfpathlineto{\pgfqpoint{1.732611in}{2.787640in}}%
\pgfpathlineto{\pgfqpoint{1.815413in}{2.815112in}}%
\pgfpathlineto{\pgfqpoint{1.900595in}{2.834171in}}%
\pgfpathlineto{\pgfqpoint{1.987312in}{2.844640in}}%
\pgfpathlineto{\pgfqpoint{2.074683in}{2.846438in}}%
\pgfpathlineto{\pgfqpoint{2.161798in}{2.839579in}}%
\pgfpathlineto{\pgfqpoint{2.247739in}{2.824170in}}%
\pgfpathlineto{\pgfqpoint{2.331744in}{2.800369in}}%
\pgfpathlineto{\pgfqpoint{2.412941in}{2.768380in}}%
\pgfpathlineto{\pgfqpoint{2.490483in}{2.728492in}}%
\pgfpathlineto{\pgfqpoint{2.563579in}{2.681075in}}%
\pgfpathlineto{\pgfqpoint{2.631503in}{2.626584in}}%
\pgfpathlineto{\pgfqpoint{2.693586in}{2.565555in}}%
\pgfpathlineto{\pgfqpoint{2.749222in}{2.498606in}}%
\pgfpathlineto{\pgfqpoint{2.797866in}{2.426441in}}%
\pgfpathlineto{\pgfqpoint{2.839031in}{2.349843in}}%
\pgfpathlineto{\pgfqpoint{2.872321in}{2.269628in}}%
\pgfpathlineto{\pgfqpoint{2.897452in}{2.186501in}}%
\pgfpathlineto{\pgfqpoint{2.914119in}{2.101313in}}%
\pgfpathlineto{\pgfqpoint{2.922113in}{2.014925in}}%
\pgfpathlineto{\pgfqpoint{2.921324in}{1.928195in}}%
\pgfpathlineto{\pgfqpoint{2.911748in}{1.841974in}}%
\pgfpathlineto{\pgfqpoint{2.893483in}{1.757111in}}%
\pgfpathlineto{\pgfqpoint{2.866728in}{1.674451in}}%
\pgfpathlineto{\pgfqpoint{2.831790in}{1.594832in}}%
\pgfpathlineto{\pgfqpoint{2.789073in}{1.519091in}}%
\pgfpathlineto{\pgfqpoint{2.739036in}{1.447983in}}%
\pgfpathlineto{\pgfqpoint{2.682122in}{1.382150in}}%
\pgfpathlineto{\pgfqpoint{2.618890in}{1.322302in}}%
\pgfpathlineto{\pgfqpoint{2.549960in}{1.269069in}}%
\pgfpathlineto{\pgfqpoint{2.476004in}{1.222993in}}%
\pgfpathlineto{\pgfqpoint{2.397746in}{1.184532in}}%
\pgfpathlineto{\pgfqpoint{2.315967in}{1.154057in}}%
\pgfpathlineto{\pgfqpoint{2.231499in}{1.131854in}}%
\pgfpathlineto{\pgfqpoint{2.145231in}{1.118124in}}%
\pgfpathlineto{\pgfqpoint{2.058062in}{1.112973in}}%
\pgfpathlineto{\pgfqpoint{1.970777in}{1.116429in}}%
\pgfpathlineto{\pgfqpoint{1.884281in}{1.128500in}}%
\pgfpathlineto{\pgfqpoint{1.799468in}{1.149108in}}%
\pgfpathlineto{\pgfqpoint{1.717190in}{1.178077in}}%
\pgfpathlineto{\pgfqpoint{1.638260in}{1.215137in}}%
\pgfpathlineto{\pgfqpoint{1.563454in}{1.259923in}}%
\pgfpathlineto{\pgfqpoint{1.493508in}{1.311974in}}%
\pgfpathlineto{\pgfqpoint{1.429120in}{1.370734in}}%
\pgfpathlineto{\pgfqpoint{1.370951in}{1.435552in}}%
\pgfpathlineto{\pgfqpoint{1.319586in}{1.505721in}}%
\pgfpathlineto{\pgfqpoint{1.275466in}{1.580627in}}%
\pgfpathlineto{\pgfqpoint{1.239083in}{1.659517in}}%
\pgfpathlineto{\pgfqpoint{1.210844in}{1.741604in}}%
\pgfpathlineto{\pgfqpoint{1.191055in}{1.826078in}}%
\pgfpathlineto{\pgfqpoint{1.179922in}{1.912110in}}%
\pgfpathlineto{\pgfqpoint{1.177551in}{1.998854in}}%
\pgfpathlineto{\pgfqpoint{1.183947in}{2.085440in}}%
\pgfpathlineto{\pgfqpoint{1.199014in}{2.170981in}}%
\pgfpathlineto{\pgfqpoint{1.222554in}{2.254570in}}%
\pgfpathlineto{\pgfqpoint{1.254310in}{2.335398in}}%
\pgfpathlineto{\pgfqpoint{1.293999in}{2.412709in}}%
\pgfpathlineto{\pgfqpoint{1.341246in}{2.485679in}}%
\pgfpathlineto{\pgfqpoint{1.395604in}{2.553551in}}%
\pgfpathlineto{\pgfqpoint{1.456551in}{2.615642in}}%
\pgfpathlineto{\pgfqpoint{1.523491in}{2.671336in}}%
\pgfpathlineto{\pgfqpoint{1.595754in}{2.720091in}}%
\pgfpathlineto{\pgfqpoint{1.672598in}{2.761435in}}%
\pgfpathlineto{\pgfqpoint{1.753207in}{2.794965in}}%
\pgfpathlineto{\pgfqpoint{1.836709in}{2.820359in}}%
\pgfpathlineto{\pgfqpoint{1.922354in}{2.837414in}}%
\pgfpathlineto{\pgfqpoint{2.009279in}{2.845919in}}%
\pgfpathlineto{\pgfqpoint{2.096594in}{2.845742in}}%
\pgfpathlineto{\pgfqpoint{2.183425in}{2.836852in}}%
\pgfpathlineto{\pgfqpoint{2.268916in}{2.819318in}}%
\pgfpathlineto{\pgfqpoint{2.352227in}{2.793312in}}%
\pgfpathlineto{\pgfqpoint{2.432534in}{2.759106in}}%
\pgfpathlineto{\pgfqpoint{2.509031in}{2.717073in}}%
\pgfpathlineto{\pgfqpoint{2.580928in}{2.667688in}}%
\pgfpathlineto{\pgfqpoint{2.647477in}{2.611508in}}%
\pgfpathlineto{\pgfqpoint{2.708098in}{2.549033in}}%
\pgfpathlineto{\pgfqpoint{2.762149in}{2.480873in}}%
\pgfpathlineto{\pgfqpoint{2.809046in}{2.407702in}}%
\pgfpathlineto{\pgfqpoint{2.848301in}{2.330237in}}%
\pgfpathlineto{\pgfqpoint{2.879517in}{2.249236in}}%
\pgfpathlineto{\pgfqpoint{2.902390in}{2.165499in}}%
\pgfpathlineto{\pgfqpoint{2.916709in}{2.079867in}}%
\pgfpathlineto{\pgfqpoint{2.922357in}{1.993226in}}%
\pgfpathlineto{\pgfqpoint{2.919308in}{1.906500in}}%
\pgfpathlineto{\pgfqpoint{2.907583in}{1.820553in}}%
\pgfpathlineto{\pgfqpoint{2.887271in}{1.736201in}}%
\pgfpathlineto{\pgfqpoint{2.887271in}{1.736201in}}%
\pgfusepath{stroke}%
\end{pgfscope}%
\begin{pgfscope}%
\pgfpathrectangle{\pgfqpoint{0.500000in}{0.440000in}}{\pgfqpoint{3.100000in}{3.080000in}}%
\pgfusepath{clip}%
\pgfsetrectcap%
\pgfsetroundjoin%
\pgfsetlinewidth{0.501875pt}%
\definecolor{currentstroke}{rgb}{0.839216,0.152941,0.156863}%
\pgfsetstrokecolor{currentstroke}%
\pgfsetdash{}{0pt}%
\pgfpathmoveto{\pgfqpoint{2.667558in}{3.206981in}}%
\pgfpathlineto{\pgfqpoint{2.787892in}{3.139521in}}%
\pgfpathlineto{\pgfqpoint{2.900910in}{3.060464in}}%
\pgfpathlineto{\pgfqpoint{3.005457in}{2.970611in}}%
\pgfpathlineto{\pgfqpoint{3.100417in}{2.870850in}}%
\pgfpathlineto{\pgfqpoint{3.184807in}{2.762154in}}%
\pgfpathlineto{\pgfqpoint{3.257776in}{2.645587in}}%
\pgfpathlineto{\pgfqpoint{3.318607in}{2.522298in}}%
\pgfpathlineto{\pgfqpoint{3.366716in}{2.393527in}}%
\pgfpathlineto{\pgfqpoint{3.401653in}{2.260601in}}%
\pgfpathlineto{\pgfqpoint{3.423100in}{2.124933in}}%
\pgfpathlineto{\pgfqpoint{3.430912in}{1.987805in}}%
\pgfpathlineto{\pgfqpoint{3.424981in}{1.850564in}}%
\pgfpathlineto{\pgfqpoint{3.405290in}{1.714639in}}%
\pgfpathlineto{\pgfqpoint{3.371979in}{1.581409in}}%
\pgfpathlineto{\pgfqpoint{3.325339in}{1.452197in}}%
\pgfpathlineto{\pgfqpoint{3.265821in}{1.328276in}}%
\pgfpathlineto{\pgfqpoint{3.194026in}{1.210862in}}%
\pgfpathlineto{\pgfqpoint{3.110713in}{1.101122in}}%
\pgfpathlineto{\pgfqpoint{3.016794in}{1.000167in}}%
\pgfpathlineto{\pgfqpoint{2.913332in}{0.909054in}}%
\pgfpathlineto{\pgfqpoint{2.801308in}{0.828594in}}%
\pgfpathlineto{\pgfqpoint{2.681774in}{0.759592in}}%
\pgfpathlineto{\pgfqpoint{2.555936in}{0.702821in}}%
\pgfpathlineto{\pgfqpoint{2.425036in}{0.658898in}}%
\pgfpathlineto{\pgfqpoint{2.290358in}{0.628283in}}%
\pgfpathlineto{\pgfqpoint{2.153226in}{0.611283in}}%
\pgfpathlineto{\pgfqpoint{2.015003in}{0.608044in}}%
\pgfpathlineto{\pgfqpoint{1.877092in}{0.618559in}}%
\pgfpathlineto{\pgfqpoint{1.740934in}{0.642664in}}%
\pgfpathlineto{\pgfqpoint{1.607951in}{0.680050in}}%
\pgfpathlineto{\pgfqpoint{1.479316in}{0.730370in}}%
\pgfpathlineto{\pgfqpoint{1.356379in}{0.793164in}}%
\pgfpathlineto{\pgfqpoint{1.240428in}{0.867847in}}%
\pgfpathlineto{\pgfqpoint{1.132640in}{0.953713in}}%
\pgfpathlineto{\pgfqpoint{1.034090in}{1.049930in}}%
\pgfpathlineto{\pgfqpoint{0.945745in}{1.155549in}}%
\pgfpathlineto{\pgfqpoint{0.868463in}{1.269495in}}%
\pgfpathlineto{\pgfqpoint{0.802999in}{1.390574in}}%
\pgfpathlineto{\pgfqpoint{0.749999in}{1.517468in}}%
\pgfpathlineto{\pgfqpoint{0.709940in}{1.648897in}}%
\pgfpathlineto{\pgfqpoint{0.683192in}{1.783667in}}%
\pgfpathlineto{\pgfqpoint{0.670111in}{1.920383in}}%
\pgfpathlineto{\pgfqpoint{0.670892in}{2.057666in}}%
\pgfpathlineto{\pgfqpoint{0.685569in}{2.194154in}}%
\pgfpathlineto{\pgfqpoint{0.714012in}{2.328505in}}%
\pgfpathlineto{\pgfqpoint{0.755930in}{2.459398in}}%
\pgfpathlineto{\pgfqpoint{0.810869in}{2.585528in}}%
\pgfpathlineto{\pgfqpoint{0.878213in}{2.705611in}}%
\pgfpathlineto{\pgfqpoint{0.957185in}{2.818379in}}%
\pgfpathlineto{\pgfqpoint{1.046989in}{2.922763in}}%
\pgfpathlineto{\pgfqpoint{1.146811in}{3.017778in}}%
\pgfpathlineto{\pgfqpoint{1.255658in}{3.102390in}}%
\pgfpathlineto{\pgfqpoint{1.372464in}{3.175707in}}%
\pgfpathlineto{\pgfqpoint{1.496088in}{3.236979in}}%
\pgfpathlineto{\pgfqpoint{1.625316in}{3.285601in}}%
\pgfpathlineto{\pgfqpoint{1.758858in}{3.321108in}}%
\pgfpathlineto{\pgfqpoint{1.895352in}{3.343178in}}%
\pgfpathlineto{\pgfqpoint{2.033361in}{3.351633in}}%
\pgfpathlineto{\pgfqpoint{2.171461in}{3.346442in}}%
\pgfpathlineto{\pgfqpoint{2.308413in}{3.327673in}}%
\pgfpathlineto{\pgfqpoint{2.442782in}{3.295448in}}%
\pgfpathlineto{\pgfqpoint{2.573180in}{3.250030in}}%
\pgfpathlineto{\pgfqpoint{2.698292in}{3.191825in}}%
\pgfpathlineto{\pgfqpoint{2.816881in}{3.121386in}}%
\pgfpathlineto{\pgfqpoint{2.927782in}{3.039409in}}%
\pgfpathlineto{\pgfqpoint{3.029906in}{2.946736in}}%
\pgfpathlineto{\pgfqpoint{3.122236in}{2.844352in}}%
\pgfpathlineto{\pgfqpoint{3.203833in}{2.733389in}}%
\pgfpathlineto{\pgfqpoint{3.273894in}{2.615027in}}%
\pgfpathlineto{\pgfqpoint{3.331812in}{2.490305in}}%
\pgfpathlineto{\pgfqpoint{3.376922in}{2.360491in}}%
\pgfpathlineto{\pgfqpoint{3.408704in}{2.226885in}}%
\pgfpathlineto{\pgfqpoint{3.426802in}{2.090806in}}%
\pgfpathlineto{\pgfqpoint{3.431021in}{1.953589in}}%
\pgfpathlineto{\pgfqpoint{3.421328in}{1.816588in}}%
\pgfpathlineto{\pgfqpoint{3.397851in}{1.681172in}}%
\pgfpathlineto{\pgfqpoint{3.360881in}{1.548727in}}%
\pgfpathlineto{\pgfqpoint{3.310871in}{1.420658in}}%
\pgfpathlineto{\pgfqpoint{3.248343in}{1.298199in}}%
\pgfpathlineto{\pgfqpoint{3.173858in}{1.182499in}}%
\pgfpathlineto{\pgfqpoint{3.088135in}{1.074797in}}%
\pgfpathlineto{\pgfqpoint{2.992000in}{0.976208in}}%
\pgfpathlineto{\pgfqpoint{2.886385in}{0.887727in}}%
\pgfpathlineto{\pgfqpoint{2.772327in}{0.810229in}}%
\pgfpathlineto{\pgfqpoint{2.650967in}{0.744466in}}%
\pgfpathlineto{\pgfqpoint{2.523554in}{0.691070in}}%
\pgfpathlineto{\pgfqpoint{2.391442in}{0.650551in}}%
\pgfpathlineto{\pgfqpoint{2.256002in}{0.623270in}}%
\pgfpathlineto{\pgfqpoint{2.118458in}{0.609455in}}%
\pgfpathlineto{\pgfqpoint{1.980236in}{0.609322in}}%
\pgfpathlineto{\pgfqpoint{1.842745in}{0.622942in}}%
\pgfpathlineto{\pgfqpoint{1.707359in}{0.650225in}}%
\pgfpathlineto{\pgfqpoint{1.575411in}{0.690926in}}%
\pgfpathlineto{\pgfqpoint{1.448198in}{0.744638in}}%
\pgfpathlineto{\pgfqpoint{1.326979in}{0.810799in}}%
\pgfpathlineto{\pgfqpoint{1.212975in}{0.888687in}}%
\pgfpathlineto{\pgfqpoint{1.107370in}{0.977423in}}%
\pgfpathlineto{\pgfqpoint{1.011224in}{1.076052in}}%
\pgfpathlineto{\pgfqpoint{0.925381in}{1.183706in}}%
\pgfpathlineto{\pgfqpoint{0.850785in}{1.299311in}}%
\pgfpathlineto{\pgfqpoint{0.788242in}{1.421720in}}%
\pgfpathlineto{\pgfqpoint{0.738407in}{1.549735in}}%
\pgfpathlineto{\pgfqpoint{0.701783in}{1.682101in}}%
\pgfpathlineto{\pgfqpoint{0.678719in}{1.817509in}}%
\pgfpathlineto{\pgfqpoint{0.669415in}{1.954592in}}%
\pgfpathlineto{\pgfqpoint{0.673916in}{2.091931in}}%
\pgfpathlineto{\pgfqpoint{0.692117in}{2.228059in}}%
\pgfpathlineto{\pgfqpoint{0.723807in}{2.361740in}}%
\pgfpathlineto{\pgfqpoint{0.768716in}{2.491638in}}%
\pgfpathlineto{\pgfqpoint{0.826448in}{2.616385in}}%
\pgfpathlineto{\pgfqpoint{0.896475in}{2.734707in}}%
\pgfpathlineto{\pgfqpoint{0.978131in}{2.845420in}}%
\pgfpathlineto{\pgfqpoint{1.070620in}{2.947435in}}%
\pgfpathlineto{\pgfqpoint{1.173010in}{3.039752in}}%
\pgfpathlineto{\pgfqpoint{1.284234in}{3.121467in}}%
\pgfpathlineto{\pgfqpoint{1.403093in}{3.191767in}}%
\pgfpathlineto{\pgfqpoint{1.528281in}{3.249947in}}%
\pgfpathlineto{\pgfqpoint{1.658679in}{3.295524in}}%
\pgfpathlineto{\pgfqpoint{1.792991in}{3.327985in}}%
\pgfpathlineto{\pgfqpoint{1.929850in}{3.346927in}}%
\pgfpathlineto{\pgfqpoint{2.067896in}{3.352110in}}%
\pgfpathlineto{\pgfqpoint{2.205768in}{3.343457in}}%
\pgfpathlineto{\pgfqpoint{2.342113in}{3.321053in}}%
\pgfpathlineto{\pgfqpoint{2.475579in}{3.285142in}}%
\pgfpathlineto{\pgfqpoint{2.604817in}{3.236135in}}%
\pgfpathlineto{\pgfqpoint{2.728483in}{3.174600in}}%
\pgfpathlineto{\pgfqpoint{2.845293in}{3.101235in}}%
\pgfpathlineto{\pgfqpoint{2.954218in}{3.016689in}}%
\pgfpathlineto{\pgfqpoint{3.054104in}{2.921784in}}%
\pgfpathlineto{\pgfqpoint{3.143895in}{2.817449in}}%
\pgfpathlineto{\pgfqpoint{3.222672in}{2.704701in}}%
\pgfpathlineto{\pgfqpoint{3.289646in}{2.584641in}}%
\pgfpathlineto{\pgfqpoint{3.344168in}{2.458463in}}%
\pgfpathlineto{\pgfqpoint{3.385719in}{2.327445in}}%
\pgfpathlineto{\pgfqpoint{3.413917in}{2.192952in}}%
\pgfpathlineto{\pgfqpoint{3.428519in}{2.056417in}}%
\pgfpathlineto{\pgfqpoint{3.429440in}{1.919076in}}%
\pgfpathlineto{\pgfqpoint{3.416611in}{1.782325in}}%
\pgfpathlineto{\pgfqpoint{3.390092in}{1.647583in}}%
\pgfpathlineto{\pgfqpoint{3.350093in}{1.516207in}}%
\pgfpathlineto{\pgfqpoint{3.296981in}{1.389500in}}%
\pgfpathlineto{\pgfqpoint{3.231272in}{1.268708in}}%
\pgfpathlineto{\pgfqpoint{3.153638in}{1.155019in}}%
\pgfpathlineto{\pgfqpoint{3.064904in}{1.049565in}}%
\pgfpathlineto{\pgfqpoint{2.966048in}{0.953419in}}%
\pgfpathlineto{\pgfqpoint{2.858176in}{0.867577in}}%
\pgfpathlineto{\pgfqpoint{2.742255in}{0.792775in}}%
\pgfpathlineto{\pgfqpoint{2.619419in}{0.729811in}}%
\pgfpathlineto{\pgfqpoint{2.490901in}{0.679389in}}%
\pgfpathlineto{\pgfqpoint{2.357969in}{0.642057in}}%
\pgfpathlineto{\pgfqpoint{2.221927in}{0.618207in}}%
\pgfpathlineto{\pgfqpoint{2.084112in}{0.608069in}}%
\pgfpathlineto{\pgfqpoint{1.945901in}{0.611720in}}%
\pgfpathlineto{\pgfqpoint{1.808702in}{0.629079in}}%
\pgfpathlineto{\pgfqpoint{1.673961in}{0.659904in}}%
\pgfpathlineto{\pgfqpoint{1.543039in}{0.703834in}}%
\pgfpathlineto{\pgfqpoint{1.417112in}{0.760474in}}%
\pgfpathlineto{\pgfqpoint{1.297522in}{0.829296in}}%
\pgfpathlineto{\pgfqpoint{1.185513in}{0.909654in}}%
\pgfpathlineto{\pgfqpoint{1.082219in}{1.000780in}}%
\pgfpathlineto{\pgfqpoint{0.988666in}{1.101789in}}%
\pgfpathlineto{\pgfqpoint{0.905767in}{1.211675in}}%
\pgfpathlineto{\pgfqpoint{0.834329in}{1.329316in}}%
\pgfpathlineto{\pgfqpoint{0.775046in}{1.453469in}}%
\pgfpathlineto{\pgfqpoint{0.728503in}{1.582778in}}%
\pgfpathlineto{\pgfqpoint{0.695083in}{1.716029in}}%
\pgfpathlineto{\pgfqpoint{0.675148in}{1.851937in}}%
\pgfpathlineto{\pgfqpoint{0.668979in}{1.989101in}}%
\pgfpathlineto{\pgfqpoint{0.676696in}{2.126144in}}%
\pgfpathlineto{\pgfqpoint{0.698258in}{2.261711in}}%
\pgfpathlineto{\pgfqpoint{0.733460in}{2.394470in}}%
\pgfpathlineto{\pgfqpoint{0.781938in}{2.523111in}}%
\pgfpathlineto{\pgfqpoint{0.843165in}{2.646348in}}%
\pgfpathlineto{\pgfqpoint{0.916451in}{2.762917in}}%
\pgfpathlineto{\pgfqpoint{1.000954in}{2.871586in}}%
\pgfpathlineto{\pgfqpoint{1.095886in}{2.971388in}}%
\pgfpathlineto{\pgfqpoint{1.200348in}{3.061310in}}%
\pgfpathlineto{\pgfqpoint{1.313302in}{3.140379in}}%
\pgfpathlineto{\pgfqpoint{1.433644in}{3.207766in}}%
\pgfpathlineto{\pgfqpoint{1.560197in}{3.262785in}}%
\pgfpathlineto{\pgfqpoint{1.691713in}{3.304898in}}%
\pgfpathlineto{\pgfqpoint{1.826877in}{3.333707in}}%
\pgfpathlineto{\pgfqpoint{1.964301in}{3.348962in}}%
\pgfpathlineto{\pgfqpoint{2.102527in}{3.350556in}}%
\pgfpathlineto{\pgfqpoint{2.240183in}{3.338529in}}%
\pgfpathlineto{\pgfqpoint{2.376004in}{3.312987in}}%
\pgfpathlineto{\pgfqpoint{2.508552in}{3.274123in}}%
\pgfpathlineto{\pgfqpoint{2.636464in}{3.222268in}}%
\pgfpathlineto{\pgfqpoint{2.758458in}{3.157898in}}%
\pgfpathlineto{\pgfqpoint{2.873328in}{3.081631in}}%
\pgfpathlineto{\pgfqpoint{2.979947in}{2.994227in}}%
\pgfpathlineto{\pgfqpoint{3.077267in}{2.896588in}}%
\pgfpathlineto{\pgfqpoint{3.164318in}{2.789761in}}%
\pgfpathlineto{\pgfqpoint{3.240210in}{2.674932in}}%
\pgfpathlineto{\pgfqpoint{3.304238in}{2.553256in}}%
\pgfpathlineto{\pgfqpoint{3.355806in}{2.425832in}}%
\pgfpathlineto{\pgfqpoint{3.394309in}{2.293963in}}%
\pgfpathlineto{\pgfqpoint{3.419298in}{2.158967in}}%
\pgfpathlineto{\pgfqpoint{3.430491in}{2.022175in}}%
\pgfpathlineto{\pgfqpoint{3.427767in}{1.884928in}}%
\pgfpathlineto{\pgfqpoint{3.411167in}{1.748583in}}%
\pgfpathlineto{\pgfqpoint{3.380892in}{1.614508in}}%
\pgfpathlineto{\pgfqpoint{3.337311in}{1.484085in}}%
\pgfpathlineto{\pgfqpoint{3.280946in}{1.358700in}}%
\pgfpathlineto{\pgfqpoint{3.212342in}{1.239501in}}%
\pgfpathlineto{\pgfqpoint{3.132134in}{1.127672in}}%
\pgfpathlineto{\pgfqpoint{3.041099in}{1.024401in}}%
\pgfpathlineto{\pgfqpoint{2.940119in}{0.930751in}}%
\pgfpathlineto{\pgfqpoint{2.830175in}{0.847663in}}%
\pgfpathlineto{\pgfqpoint{2.712353in}{0.775954in}}%
\pgfpathlineto{\pgfqpoint{2.587838in}{0.716317in}}%
\pgfpathlineto{\pgfqpoint{2.457919in}{0.669322in}}%
\pgfpathlineto{\pgfqpoint{2.323986in}{0.635414in}}%
\pgfpathlineto{\pgfqpoint{2.187351in}{0.614868in}}%
\pgfpathlineto{\pgfqpoint{2.049302in}{0.607890in}}%
\pgfpathlineto{\pgfqpoint{1.911279in}{0.614633in}}%
\pgfpathlineto{\pgfqpoint{1.774681in}{0.635092in}}%
\pgfpathlineto{\pgfqpoint{1.640865in}{0.669102in}}%
\pgfpathlineto{\pgfqpoint{1.511148in}{0.716344in}}%
\pgfpathlineto{\pgfqpoint{1.386805in}{0.776340in}}%
\pgfpathlineto{\pgfqpoint{1.269072in}{0.848454in}}%
\pgfpathlineto{\pgfqpoint{1.159141in}{0.931895in}}%
\pgfpathlineto{\pgfqpoint{1.058167in}{1.025711in}}%
\pgfpathlineto{\pgfqpoint{0.967091in}{1.128972in}}%
\pgfpathlineto{\pgfqpoint{0.886778in}{1.240737in}}%
\pgfpathlineto{\pgfqpoint{0.818118in}{1.359882in}}%
\pgfpathlineto{\pgfqpoint{0.761851in}{1.485231in}}%
\pgfpathlineto{\pgfqpoint{0.718560in}{1.615557in}}%
\pgfpathlineto{\pgfqpoint{0.688677in}{1.749580in}}%
\pgfpathlineto{\pgfqpoint{0.672480in}{1.885969in}}%
\pgfpathlineto{\pgfqpoint{0.670095in}{2.023339in}}%
\pgfpathlineto{\pgfqpoint{0.681492in}{2.160254in}}%
\pgfpathlineto{\pgfqpoint{0.706495in}{2.295278in}}%
\pgfpathlineto{\pgfqpoint{0.744855in}{2.427215in}}%
\pgfpathlineto{\pgfqpoint{0.796239in}{2.554692in}}%
\pgfpathlineto{\pgfqpoint{0.860185in}{2.676376in}}%
\pgfpathlineto{\pgfqpoint{0.936096in}{2.791030in}}%
\pgfpathlineto{\pgfqpoint{1.023247in}{2.897515in}}%
\pgfpathlineto{\pgfqpoint{1.120778in}{2.994783in}}%
\pgfpathlineto{\pgfqpoint{1.227698in}{3.081885in}}%
\pgfpathlineto{\pgfqpoint{1.342885in}{3.157965in}}%
\pgfpathlineto{\pgfqpoint{1.465084in}{3.222264in}}%
\pgfpathlineto{\pgfqpoint{1.593005in}{3.274165in}}%
\pgfpathlineto{\pgfqpoint{1.725519in}{3.313221in}}%
\pgfpathlineto{\pgfqpoint{1.861267in}{3.338959in}}%
\pgfpathlineto{\pgfqpoint{1.998879in}{3.351051in}}%
\pgfpathlineto{\pgfqpoint{2.136990in}{3.349333in}}%
\pgfpathlineto{\pgfqpoint{2.274242in}{3.333802in}}%
\pgfpathlineto{\pgfqpoint{2.409286in}{3.304619in}}%
\pgfpathlineto{\pgfqpoint{2.540776in}{3.262105in}}%
\pgfpathlineto{\pgfqpoint{2.667376in}{3.206743in}}%
\pgfpathlineto{\pgfqpoint{2.787755in}{3.139179in}}%
\pgfpathlineto{\pgfqpoint{2.900721in}{3.060129in}}%
\pgfpathlineto{\pgfqpoint{3.005248in}{2.970294in}}%
\pgfpathlineto{\pgfqpoint{3.100206in}{2.870560in}}%
\pgfpathlineto{\pgfqpoint{3.184598in}{2.761902in}}%
\pgfpathlineto{\pgfqpoint{3.257565in}{2.645379in}}%
\pgfpathlineto{\pgfqpoint{3.318384in}{2.522135in}}%
\pgfpathlineto{\pgfqpoint{3.366466in}{2.393399in}}%
\pgfpathlineto{\pgfqpoint{3.401362in}{2.260484in}}%
\pgfpathlineto{\pgfqpoint{3.422757in}{2.124788in}}%
\pgfpathlineto{\pgfqpoint{3.430488in}{1.987721in}}%
\pgfpathlineto{\pgfqpoint{3.424512in}{1.850515in}}%
\pgfpathlineto{\pgfqpoint{3.404816in}{1.714602in}}%
\pgfpathlineto{\pgfqpoint{3.371533in}{1.581380in}}%
\pgfpathlineto{\pgfqpoint{3.324947in}{1.452189in}}%
\pgfpathlineto{\pgfqpoint{3.265494in}{1.328305in}}%
\pgfpathlineto{\pgfqpoint{3.193760in}{1.210945in}}%
\pgfpathlineto{\pgfqpoint{3.110486in}{1.101264in}}%
\pgfpathlineto{\pgfqpoint{3.016560in}{1.000359in}}%
\pgfpathlineto{\pgfqpoint{2.913028in}{0.909264in}}%
\pgfpathlineto{\pgfqpoint{2.801008in}{0.828894in}}%
\pgfpathlineto{\pgfqpoint{2.681483in}{0.759939in}}%
\pgfpathlineto{\pgfqpoint{2.555655in}{0.703172in}}%
\pgfpathlineto{\pgfqpoint{2.424781in}{0.659228in}}%
\pgfpathlineto{\pgfqpoint{2.290151in}{0.628583in}}%
\pgfpathlineto{\pgfqpoint{2.153084in}{0.611556in}}%
\pgfpathlineto{\pgfqpoint{2.014932in}{0.608306in}}%
\pgfpathlineto{\pgfqpoint{1.877079in}{0.618835in}}%
\pgfpathlineto{\pgfqpoint{1.740939in}{0.642987in}}%
\pgfpathlineto{\pgfqpoint{1.607960in}{0.680446in}}%
\pgfpathlineto{\pgfqpoint{1.479414in}{0.730809in}}%
\pgfpathlineto{\pgfqpoint{1.356515in}{0.793626in}}%
\pgfpathlineto{\pgfqpoint{1.240574in}{0.868304in}}%
\pgfpathlineto{\pgfqpoint{1.132789in}{0.954134in}}%
\pgfpathlineto{\pgfqpoint{1.034248in}{1.050292in}}%
\pgfpathlineto{\pgfqpoint{0.945924in}{1.155837in}}%
\pgfpathlineto{\pgfqpoint{0.868678in}{1.269714in}}%
\pgfpathlineto{\pgfqpoint{0.803259in}{1.390749in}}%
\pgfpathlineto{\pgfqpoint{0.750302in}{1.517656in}}%
\pgfpathlineto{\pgfqpoint{0.710316in}{1.649067in}}%
\pgfpathlineto{\pgfqpoint{0.683621in}{1.783805in}}%
\pgfpathlineto{\pgfqpoint{0.670551in}{1.920507in}}%
\pgfpathlineto{\pgfqpoint{0.671312in}{2.057773in}}%
\pgfpathlineto{\pgfqpoint{0.685948in}{2.194233in}}%
\pgfpathlineto{\pgfqpoint{0.714342in}{2.328539in}}%
\pgfpathlineto{\pgfqpoint{0.756216in}{2.459373in}}%
\pgfpathlineto{\pgfqpoint{0.811130in}{2.585441in}}%
\pgfpathlineto{\pgfqpoint{0.878484in}{2.705475in}}%
\pgfpathlineto{\pgfqpoint{0.957516in}{2.818235in}}%
\pgfpathlineto{\pgfqpoint{1.047339in}{2.922551in}}%
\pgfpathlineto{\pgfqpoint{1.147160in}{3.017516in}}%
\pgfpathlineto{\pgfqpoint{1.255997in}{3.102117in}}%
\pgfpathlineto{\pgfqpoint{1.372775in}{3.175442in}}%
\pgfpathlineto{\pgfqpoint{1.496351in}{3.236729in}}%
\pgfpathlineto{\pgfqpoint{1.625516in}{3.285358in}}%
\pgfpathlineto{\pgfqpoint{1.758993in}{3.320857in}}%
\pgfpathlineto{\pgfqpoint{1.895439in}{3.342900in}}%
\pgfpathlineto{\pgfqpoint{2.033442in}{3.351307in}}%
\pgfpathlineto{\pgfqpoint{2.171529in}{3.346045in}}%
\pgfpathlineto{\pgfqpoint{2.308413in}{3.327215in}}%
\pgfpathlineto{\pgfqpoint{2.442758in}{3.294966in}}%
\pgfpathlineto{\pgfqpoint{2.573148in}{3.249559in}}%
\pgfpathlineto{\pgfqpoint{2.698253in}{3.191397in}}%
\pgfpathlineto{\pgfqpoint{2.816821in}{3.121022in}}%
\pgfpathlineto{\pgfqpoint{2.927684in}{3.039117in}}%
\pgfpathlineto{\pgfqpoint{3.029756in}{2.946505in}}%
\pgfpathlineto{\pgfqpoint{3.122031in}{2.844151in}}%
\pgfpathlineto{\pgfqpoint{3.203587in}{2.733157in}}%
\pgfpathlineto{\pgfqpoint{3.273589in}{2.614764in}}%
\pgfpathlineto{\pgfqpoint{3.331435in}{2.490066in}}%
\pgfpathlineto{\pgfqpoint{3.376523in}{2.360262in}}%
\pgfpathlineto{\pgfqpoint{3.408319in}{2.226671in}}%
\pgfpathlineto{\pgfqpoint{3.426451in}{2.090625in}}%
\pgfpathlineto{\pgfqpoint{3.430709in}{1.953461in}}%
\pgfpathlineto{\pgfqpoint{3.421048in}{1.816527in}}%
\pgfpathlineto{\pgfqpoint{3.397583in}{1.681179in}}%
\pgfpathlineto{\pgfqpoint{3.360593in}{1.548783in}}%
\pgfpathlineto{\pgfqpoint{3.310518in}{1.420711in}}%
\pgfpathlineto{\pgfqpoint{3.247945in}{1.298314in}}%
\pgfpathlineto{\pgfqpoint{3.173439in}{1.182678in}}%
\pgfpathlineto{\pgfqpoint{3.087712in}{1.075002in}}%
\pgfpathlineto{\pgfqpoint{2.991595in}{0.976422in}}%
\pgfpathlineto{\pgfqpoint{2.886023in}{0.887946in}}%
\pgfpathlineto{\pgfqpoint{2.772027in}{0.810458in}}%
\pgfpathlineto{\pgfqpoint{2.650737in}{0.744716in}}%
\pgfpathlineto{\pgfqpoint{2.523382in}{0.691350in}}%
\pgfpathlineto{\pgfqpoint{2.391290in}{0.650868in}}%
\pgfpathlineto{\pgfqpoint{2.255872in}{0.623644in}}%
\pgfpathlineto{\pgfqpoint{2.118370in}{0.609878in}}%
\pgfpathlineto{\pgfqpoint{1.980169in}{0.609760in}}%
\pgfpathlineto{\pgfqpoint{1.842696in}{0.623366in}}%
\pgfpathlineto{\pgfqpoint{1.707333in}{0.650616in}}%
\pgfpathlineto{\pgfqpoint{1.575422in}{0.691271in}}%
\pgfpathlineto{\pgfqpoint{1.448259in}{0.744938in}}%
\pgfpathlineto{\pgfqpoint{1.327096in}{0.811064in}}%
\pgfpathlineto{\pgfqpoint{1.213142in}{0.888944in}}%
\pgfpathlineto{\pgfqpoint{1.107563in}{0.977712in}}%
\pgfpathlineto{\pgfqpoint{1.011464in}{1.076364in}}%
\pgfpathlineto{\pgfqpoint{0.925678in}{1.184008in}}%
\pgfpathlineto{\pgfqpoint{0.851101in}{1.299599in}}%
\pgfpathlineto{\pgfqpoint{0.788554in}{1.421985in}}%
\pgfpathlineto{\pgfqpoint{0.738706in}{1.549961in}}%
\pgfpathlineto{\pgfqpoint{0.702069in}{1.682273in}}%
\pgfpathlineto{\pgfqpoint{0.679001in}{1.817619in}}%
\pgfpathlineto{\pgfqpoint{0.669708in}{1.954647in}}%
\pgfpathlineto{\pgfqpoint{0.674240in}{2.091955in}}%
\pgfpathlineto{\pgfqpoint{0.692493in}{2.228093in}}%
\pgfpathlineto{\pgfqpoint{0.724230in}{2.361695in}}%
\pgfpathlineto{\pgfqpoint{0.769164in}{2.491552in}}%
\pgfpathlineto{\pgfqpoint{0.826895in}{2.616282in}}%
\pgfpathlineto{\pgfqpoint{0.896894in}{2.734594in}}%
\pgfpathlineto{\pgfqpoint{0.978503in}{2.845293in}}%
\pgfpathlineto{\pgfqpoint{1.070930in}{2.947282in}}%
\pgfpathlineto{\pgfqpoint{1.173258in}{3.039563in}}%
\pgfpathlineto{\pgfqpoint{1.284436in}{3.121235in}}%
\pgfpathlineto{\pgfqpoint{1.403287in}{3.191493in}}%
\pgfpathlineto{\pgfqpoint{1.528500in}{3.249633in}}%
\pgfpathlineto{\pgfqpoint{1.658859in}{3.295143in}}%
\pgfpathlineto{\pgfqpoint{1.793150in}{3.327577in}}%
\pgfpathlineto{\pgfqpoint{1.929992in}{3.346523in}}%
\pgfpathlineto{\pgfqpoint{2.068010in}{3.351731in}}%
\pgfpathlineto{\pgfqpoint{2.205841in}{3.343112in}}%
\pgfpathlineto{\pgfqpoint{2.342131in}{3.320742in}}%
\pgfpathlineto{\pgfqpoint{2.475537in}{3.284855in}}%
\pgfpathlineto{\pgfqpoint{2.604722in}{3.235849in}}%
\pgfpathlineto{\pgfqpoint{2.728362in}{3.174284in}}%
\pgfpathlineto{\pgfqpoint{2.845145in}{3.100877in}}%
\pgfpathlineto{\pgfqpoint{2.954007in}{3.016331in}}%
\pgfpathlineto{\pgfqpoint{3.053869in}{2.921434in}}%
\pgfpathlineto{\pgfqpoint{3.143657in}{2.817121in}}%
\pgfpathlineto{\pgfqpoint{3.222436in}{2.704411in}}%
\pgfpathlineto{\pgfqpoint{3.289410in}{2.584405in}}%
\pgfpathlineto{\pgfqpoint{3.343920in}{2.458284in}}%
\pgfpathlineto{\pgfqpoint{3.385446in}{2.327314in}}%
\pgfpathlineto{\pgfqpoint{3.413605in}{2.192841in}}%
\pgfpathlineto{\pgfqpoint{3.428154in}{2.056294in}}%
\pgfpathlineto{\pgfqpoint{3.429006in}{1.919020in}}%
\pgfpathlineto{\pgfqpoint{3.416147in}{1.782299in}}%
\pgfpathlineto{\pgfqpoint{3.389632in}{1.647568in}}%
\pgfpathlineto{\pgfqpoint{3.349667in}{1.516205in}}%
\pgfpathlineto{\pgfqpoint{3.296607in}{1.389523in}}%
\pgfpathlineto{\pgfqpoint{3.230959in}{1.268771in}}%
\pgfpathlineto{\pgfqpoint{3.153380in}{1.155135in}}%
\pgfpathlineto{\pgfqpoint{3.064676in}{1.049736in}}%
\pgfpathlineto{\pgfqpoint{2.965805in}{0.953632in}}%
\pgfpathlineto{\pgfqpoint{2.857876in}{0.867817in}}%
\pgfpathlineto{\pgfqpoint{2.741976in}{0.793098in}}%
\pgfpathlineto{\pgfqpoint{2.619151in}{0.730166in}}%
\pgfpathlineto{\pgfqpoint{2.490649in}{0.679743in}}%
\pgfpathlineto{\pgfqpoint{2.357749in}{0.642390in}}%
\pgfpathlineto{\pgfqpoint{2.221756in}{0.618511in}}%
\pgfpathlineto{\pgfqpoint{2.084005in}{0.608351in}}%
\pgfpathlineto{\pgfqpoint{1.945860in}{0.611996in}}%
\pgfpathlineto{\pgfqpoint{1.808711in}{0.629372in}}%
\pgfpathlineto{\pgfqpoint{1.673979in}{0.660247in}}%
\pgfpathlineto{\pgfqpoint{1.543095in}{0.704233in}}%
\pgfpathlineto{\pgfqpoint{1.417238in}{0.760903in}}%
\pgfpathlineto{\pgfqpoint{1.297677in}{0.829737in}}%
\pgfpathlineto{\pgfqpoint{1.185678in}{0.910081in}}%
\pgfpathlineto{\pgfqpoint{1.082391in}{1.001168in}}%
\pgfpathlineto{\pgfqpoint{0.988850in}{1.102118in}}%
\pgfpathlineto{\pgfqpoint{0.905975in}{1.211937in}}%
\pgfpathlineto{\pgfqpoint{0.834572in}{1.329517in}}%
\pgfpathlineto{\pgfqpoint{0.775331in}{1.453638in}}%
\pgfpathlineto{\pgfqpoint{0.728826in}{1.582966in}}%
\pgfpathlineto{\pgfqpoint{0.695477in}{1.716171in}}%
\pgfpathlineto{\pgfqpoint{0.675576in}{1.852052in}}%
\pgfpathlineto{\pgfqpoint{0.669408in}{1.989200in}}%
\pgfpathlineto{\pgfqpoint{0.677102in}{2.126222in}}%
\pgfpathlineto{\pgfqpoint{0.698624in}{2.261755in}}%
\pgfpathlineto{\pgfqpoint{0.733783in}{2.394467in}}%
\pgfpathlineto{\pgfqpoint{0.782224in}{2.523051in}}%
\pgfpathlineto{\pgfqpoint{0.843434in}{2.646231in}}%
\pgfpathlineto{\pgfqpoint{0.916735in}{2.762761in}}%
\pgfpathlineto{\pgfqpoint{1.001294in}{2.871422in}}%
\pgfpathlineto{\pgfqpoint{1.096221in}{2.971146in}}%
\pgfpathlineto{\pgfqpoint{1.200675in}{3.061033in}}%
\pgfpathlineto{\pgfqpoint{1.313613in}{3.140096in}}%
\pgfpathlineto{\pgfqpoint{1.433922in}{3.207491in}}%
\pgfpathlineto{\pgfqpoint{1.560425in}{3.262520in}}%
\pgfpathlineto{\pgfqpoint{1.691881in}{3.304636in}}%
\pgfpathlineto{\pgfqpoint{1.826985in}{3.333434in}}%
\pgfpathlineto{\pgfqpoint{1.964369in}{3.348661in}}%
\pgfpathlineto{\pgfqpoint{2.102597in}{3.350209in}}%
\pgfpathlineto{\pgfqpoint{2.240209in}{3.338119in}}%
\pgfpathlineto{\pgfqpoint{2.375977in}{3.312535in}}%
\pgfpathlineto{\pgfqpoint{2.508503in}{3.273660in}}%
\pgfpathlineto{\pgfqpoint{2.636406in}{3.221824in}}%
\pgfpathlineto{\pgfqpoint{2.758387in}{3.157498in}}%
\pgfpathlineto{\pgfqpoint{2.873232in}{3.081291in}}%
\pgfpathlineto{\pgfqpoint{2.979812in}{2.993951in}}%
\pgfpathlineto{\pgfqpoint{3.077083in}{2.896364in}}%
\pgfpathlineto{\pgfqpoint{3.164085in}{2.789556in}}%
\pgfpathlineto{\pgfqpoint{3.239945in}{2.674691in}}%
\pgfpathlineto{\pgfqpoint{3.303902in}{2.553022in}}%
\pgfpathlineto{\pgfqpoint{3.355420in}{2.425618in}}%
\pgfpathlineto{\pgfqpoint{3.393912in}{2.293764in}}%
\pgfpathlineto{\pgfqpoint{3.418918in}{2.158790in}}%
\pgfpathlineto{\pgfqpoint{3.430143in}{2.022034in}}%
\pgfpathlineto{\pgfqpoint{3.427452in}{1.884841in}}%
\pgfpathlineto{\pgfqpoint{3.410877in}{1.748560in}}%
\pgfpathlineto{\pgfqpoint{3.380609in}{1.614546in}}%
\pgfpathlineto{\pgfqpoint{3.337003in}{1.484161in}}%
\pgfpathlineto{\pgfqpoint{3.280578in}{1.358771in}}%
\pgfpathlineto{\pgfqpoint{3.211955in}{1.239651in}}%
\pgfpathlineto{\pgfqpoint{3.131738in}{1.127869in}}%
\pgfpathlineto{\pgfqpoint{3.040710in}{1.024617in}}%
\pgfpathlineto{\pgfqpoint{2.939756in}{0.930975in}}%
\pgfpathlineto{\pgfqpoint{2.829858in}{0.847894in}}%
\pgfpathlineto{\pgfqpoint{2.712095in}{0.776198in}}%
\pgfpathlineto{\pgfqpoint{2.587640in}{0.716584in}}%
\pgfpathlineto{\pgfqpoint{2.457765in}{0.669620in}}%
\pgfpathlineto{\pgfqpoint{2.323837in}{0.635747in}}%
\pgfpathlineto{\pgfqpoint{2.187247in}{0.615260in}}%
\pgfpathlineto{\pgfqpoint{2.049233in}{0.608318in}}%
\pgfpathlineto{\pgfqpoint{1.911230in}{0.615067in}}%
\pgfpathlineto{\pgfqpoint{1.774651in}{0.635506in}}%
\pgfpathlineto{\pgfqpoint{1.640862in}{0.669480in}}%
\pgfpathlineto{\pgfqpoint{1.511185in}{0.716677in}}%
\pgfpathlineto{\pgfqpoint{1.386893in}{0.776630in}}%
\pgfpathlineto{\pgfqpoint{1.269213in}{0.848716in}}%
\pgfpathlineto{\pgfqpoint{1.159327in}{0.932156in}}%
\pgfpathlineto{\pgfqpoint{1.058369in}{1.026014in}}%
\pgfpathlineto{\pgfqpoint{0.967360in}{1.129271in}}%
\pgfpathlineto{\pgfqpoint{0.887089in}{1.241024in}}%
\pgfpathlineto{\pgfqpoint{0.818440in}{1.360152in}}%
\pgfpathlineto{\pgfqpoint{0.762167in}{1.485473in}}%
\pgfpathlineto{\pgfqpoint{0.718862in}{1.615756in}}%
\pgfpathlineto{\pgfqpoint{0.688969in}{1.749723in}}%
\pgfpathlineto{\pgfqpoint{0.672772in}{1.886052in}}%
\pgfpathlineto{\pgfqpoint{0.670401in}{2.023373in}}%
\pgfpathlineto{\pgfqpoint{0.681832in}{2.160268in}}%
\pgfpathlineto{\pgfqpoint{0.706885in}{2.295277in}}%
\pgfpathlineto{\pgfqpoint{0.745280in}{2.427145in}}%
\pgfpathlineto{\pgfqpoint{0.796679in}{2.554588in}}%
\pgfpathlineto{\pgfqpoint{0.860615in}{2.676257in}}%
\pgfpathlineto{\pgfqpoint{0.936494in}{2.790899in}}%
\pgfpathlineto{\pgfqpoint{1.023594in}{2.897366in}}%
\pgfpathlineto{\pgfqpoint{1.121065in}{2.994607in}}%
\pgfpathlineto{\pgfqpoint{1.227928in}{3.081672in}}%
\pgfpathlineto{\pgfqpoint{1.343078in}{3.157711in}}%
\pgfpathlineto{\pgfqpoint{1.465281in}{3.221973in}}%
\pgfpathlineto{\pgfqpoint{1.593198in}{3.273821in}}%
\pgfpathlineto{\pgfqpoint{1.725679in}{3.312829in}}%
\pgfpathlineto{\pgfqpoint{1.861408in}{3.338551in}}%
\pgfpathlineto{\pgfqpoint{1.998998in}{3.350653in}}%
\pgfpathlineto{\pgfqpoint{2.137077in}{3.348962in}}%
\pgfpathlineto{\pgfqpoint{2.274285in}{3.333464in}}%
\pgfpathlineto{\pgfqpoint{2.409273in}{3.304312in}}%
\pgfpathlineto{\pgfqpoint{2.540706in}{3.261816in}}%
\pgfpathlineto{\pgfqpoint{2.667260in}{3.206450in}}%
\pgfpathlineto{\pgfqpoint{2.787623in}{3.138848in}}%
\pgfpathlineto{\pgfqpoint{2.900539in}{3.059779in}}%
\pgfpathlineto{\pgfqpoint{3.005019in}{2.969948in}}%
\pgfpathlineto{\pgfqpoint{3.099960in}{2.870226in}}%
\pgfpathlineto{\pgfqpoint{3.184350in}{2.761595in}}%
\pgfpathlineto{\pgfqpoint{3.257319in}{2.645115in}}%
\pgfpathlineto{\pgfqpoint{3.318133in}{2.521925in}}%
\pgfpathlineto{\pgfqpoint{3.366201in}{2.393244in}}%
\pgfpathlineto{\pgfqpoint{3.401070in}{2.260371in}}%
\pgfpathlineto{\pgfqpoint{3.422427in}{2.124685in}}%
\pgfpathlineto{\pgfqpoint{3.430100in}{1.987635in}}%
\pgfpathlineto{\pgfqpoint{3.424072in}{1.850484in}}%
\pgfpathlineto{\pgfqpoint{3.404358in}{1.714595in}}%
\pgfpathlineto{\pgfqpoint{3.371088in}{1.581387in}}%
\pgfpathlineto{\pgfqpoint{3.324539in}{1.452212in}}%
\pgfpathlineto{\pgfqpoint{3.265138in}{1.328356in}}%
\pgfpathlineto{\pgfqpoint{3.193462in}{1.211038in}}%
\pgfpathlineto{\pgfqpoint{3.110235in}{1.101409in}}%
\pgfpathlineto{\pgfqpoint{3.016331in}{1.000555in}}%
\pgfpathlineto{\pgfqpoint{2.912775in}{0.909494in}}%
\pgfpathlineto{\pgfqpoint{2.800729in}{0.829171in}}%
\pgfpathlineto{\pgfqpoint{2.681224in}{0.760279in}}%
\pgfpathlineto{\pgfqpoint{2.555409in}{0.703533in}}%
\pgfpathlineto{\pgfqpoint{2.424556in}{0.659583in}}%
\pgfpathlineto{\pgfqpoint{2.289961in}{0.628916in}}%
\pgfpathlineto{\pgfqpoint{2.152946in}{0.611863in}}%
\pgfpathlineto{\pgfqpoint{2.014857in}{0.608595in}}%
\pgfpathlineto{\pgfqpoint{1.877065in}{0.619122in}}%
\pgfpathlineto{\pgfqpoint{1.740968in}{0.643295in}}%
\pgfpathlineto{\pgfqpoint{1.607987in}{0.680806in}}%
\pgfpathlineto{\pgfqpoint{1.479507in}{0.731208in}}%
\pgfpathlineto{\pgfqpoint{1.356662in}{0.794045in}}%
\pgfpathlineto{\pgfqpoint{1.240745in}{0.868725in}}%
\pgfpathlineto{\pgfqpoint{1.132972in}{0.954535in}}%
\pgfpathlineto{\pgfqpoint{1.034440in}{1.050651in}}%
\pgfpathlineto{\pgfqpoint{0.946132in}{1.156138in}}%
\pgfpathlineto{\pgfqpoint{0.868911in}{1.269951in}}%
\pgfpathlineto{\pgfqpoint{0.803526in}{1.390934in}}%
\pgfpathlineto{\pgfqpoint{0.750607in}{1.517819in}}%
\pgfpathlineto{\pgfqpoint{0.710666in}{1.649230in}}%
\pgfpathlineto{\pgfqpoint{0.684026in}{1.783923in}}%
\pgfpathlineto{\pgfqpoint{0.670977in}{1.920601in}}%
\pgfpathlineto{\pgfqpoint{0.671731in}{2.057848in}}%
\pgfpathlineto{\pgfqpoint{0.686340in}{2.194282in}}%
\pgfpathlineto{\pgfqpoint{0.714697in}{2.328552in}}%
\pgfpathlineto{\pgfqpoint{0.756531in}{2.459338in}}%
\pgfpathlineto{\pgfqpoint{0.811415in}{2.585351in}}%
\pgfpathlineto{\pgfqpoint{0.878758in}{2.705334in}}%
\pgfpathlineto{\pgfqpoint{0.957812in}{2.818064in}}%
\pgfpathlineto{\pgfqpoint{1.047669in}{2.922349in}}%
\pgfpathlineto{\pgfqpoint{1.147479in}{3.017253in}}%
\pgfpathlineto{\pgfqpoint{1.256305in}{3.101829in}}%
\pgfpathlineto{\pgfqpoint{1.373060in}{3.175152in}}%
\pgfpathlineto{\pgfqpoint{1.496599in}{3.236445in}}%
\pgfpathlineto{\pgfqpoint{1.625713in}{3.285081in}}%
\pgfpathlineto{\pgfqpoint{1.759132in}{3.320579in}}%
\pgfpathlineto{\pgfqpoint{1.895523in}{3.342608in}}%
\pgfpathlineto{\pgfqpoint{2.033494in}{3.350987in}}%
\pgfpathlineto{\pgfqpoint{2.171590in}{3.345680in}}%
\pgfpathlineto{\pgfqpoint{2.308404in}{3.326799in}}%
\pgfpathlineto{\pgfqpoint{2.442707in}{3.294522in}}%
\pgfpathlineto{\pgfqpoint{2.573080in}{3.249114in}}%
\pgfpathlineto{\pgfqpoint{2.698172in}{3.190977in}}%
\pgfpathlineto{\pgfqpoint{2.816723in}{3.120647in}}%
\pgfpathlineto{\pgfqpoint{2.927558in}{3.038799in}}%
\pgfpathlineto{\pgfqpoint{3.029589in}{2.946245in}}%
\pgfpathlineto{\pgfqpoint{3.121818in}{2.843934in}}%
\pgfpathlineto{\pgfqpoint{3.203332in}{2.732951in}}%
\pgfpathlineto{\pgfqpoint{3.273304in}{2.614519in}}%
\pgfpathlineto{\pgfqpoint{3.331077in}{2.489855in}}%
\pgfpathlineto{\pgfqpoint{3.376132in}{2.360071in}}%
\pgfpathlineto{\pgfqpoint{3.407926in}{2.226499in}}%
\pgfpathlineto{\pgfqpoint{3.426076in}{2.090480in}}%
\pgfpathlineto{\pgfqpoint{3.430364in}{1.953357in}}%
\pgfpathlineto{\pgfqpoint{3.420732in}{1.816476in}}%
\pgfpathlineto{\pgfqpoint{3.397287in}{1.681189in}}%
\pgfpathlineto{\pgfqpoint{3.360297in}{1.548846in}}%
\pgfpathlineto{\pgfqpoint{3.310195in}{1.420804in}}%
\pgfpathlineto{\pgfqpoint{3.247576in}{1.298421in}}%
\pgfpathlineto{\pgfqpoint{3.173065in}{1.182852in}}%
\pgfpathlineto{\pgfqpoint{3.087340in}{1.075207in}}%
\pgfpathlineto{\pgfqpoint{2.991241in}{0.976638in}}%
\pgfpathlineto{\pgfqpoint{2.885704in}{0.888166in}}%
\pgfpathlineto{\pgfqpoint{2.771757in}{0.810685in}}%
\pgfpathlineto{\pgfqpoint{2.650523in}{0.744959in}}%
\pgfpathlineto{\pgfqpoint{2.523217in}{0.691620in}}%
\pgfpathlineto{\pgfqpoint{2.391149in}{0.651173in}}%
\pgfpathlineto{\pgfqpoint{2.255724in}{0.623994in}}%
\pgfpathlineto{\pgfqpoint{2.118282in}{0.610292in}}%
\pgfpathlineto{\pgfqpoint{1.980108in}{0.610203in}}%
\pgfpathlineto{\pgfqpoint{1.842650in}{0.623806in}}%
\pgfpathlineto{\pgfqpoint{1.707306in}{0.651027in}}%
\pgfpathlineto{\pgfqpoint{1.575424in}{0.691637in}}%
\pgfpathlineto{\pgfqpoint{1.448305in}{0.745253in}}%
\pgfpathlineto{\pgfqpoint{1.327196in}{0.811335in}}%
\pgfpathlineto{\pgfqpoint{1.213297in}{0.889192in}}%
\pgfpathlineto{\pgfqpoint{1.107758in}{0.977974in}}%
\pgfpathlineto{\pgfqpoint{1.011678in}{1.076678in}}%
\pgfpathlineto{\pgfqpoint{0.925970in}{1.184305in}}%
\pgfpathlineto{\pgfqpoint{0.851425in}{1.299886in}}%
\pgfpathlineto{\pgfqpoint{0.788881in}{1.422255in}}%
\pgfpathlineto{\pgfqpoint{0.739019in}{1.550198in}}%
\pgfpathlineto{\pgfqpoint{0.702364in}{1.682461in}}%
\pgfpathlineto{\pgfqpoint{0.679284in}{1.817746in}}%
\pgfpathlineto{\pgfqpoint{0.669993in}{1.954712in}}%
\pgfpathlineto{\pgfqpoint{0.674546in}{2.091975in}}%
\pgfpathlineto{\pgfqpoint{0.692844in}{2.228108in}}%
\pgfpathlineto{\pgfqpoint{0.724635in}{2.361667in}}%
\pgfpathlineto{\pgfqpoint{0.769603in}{2.491464in}}%
\pgfpathlineto{\pgfqpoint{0.827344in}{2.616168in}}%
\pgfpathlineto{\pgfqpoint{0.897328in}{2.734469in}}%
\pgfpathlineto{\pgfqpoint{0.978896in}{2.845158in}}%
\pgfpathlineto{\pgfqpoint{1.071265in}{2.947130in}}%
\pgfpathlineto{\pgfqpoint{1.173528in}{3.039381in}}%
\pgfpathlineto{\pgfqpoint{1.284649in}{3.121013in}}%
\pgfpathlineto{\pgfqpoint{1.403471in}{3.191228in}}%
\pgfpathlineto{\pgfqpoint{1.528709in}{3.249333in}}%
\pgfpathlineto{\pgfqpoint{1.659037in}{3.294775in}}%
\pgfpathlineto{\pgfqpoint{1.793302in}{3.327169in}}%
\pgfpathlineto{\pgfqpoint{1.930126in}{3.346109in}}%
\pgfpathlineto{\pgfqpoint{2.068122in}{3.351335in}}%
\pgfpathlineto{\pgfqpoint{2.205919in}{3.342750in}}%
\pgfpathlineto{\pgfqpoint{2.342160in}{3.320418in}}%
\pgfpathlineto{\pgfqpoint{2.475507in}{3.284562in}}%
\pgfpathlineto{\pgfqpoint{2.604634in}{3.235570in}}%
\pgfpathlineto{\pgfqpoint{2.728234in}{3.173990in}}%
\pgfpathlineto{\pgfqpoint{2.845013in}{3.100530in}}%
\pgfpathlineto{\pgfqpoint{2.953801in}{3.015980in}}%
\pgfpathlineto{\pgfqpoint{3.053626in}{2.921087in}}%
\pgfpathlineto{\pgfqpoint{3.143405in}{2.816790in}}%
\pgfpathlineto{\pgfqpoint{3.222187in}{2.704112in}}%
\pgfpathlineto{\pgfqpoint{3.289166in}{2.584154in}}%
\pgfpathlineto{\pgfqpoint{3.343672in}{2.458091in}}%
\pgfpathlineto{\pgfqpoint{3.385180in}{2.327177in}}%
\pgfpathlineto{\pgfqpoint{3.413308in}{2.192740in}}%
\pgfpathlineto{\pgfqpoint{3.427811in}{2.056186in}}%
\pgfpathlineto{\pgfqpoint{3.428595in}{1.918958in}}%
\pgfpathlineto{\pgfqpoint{3.415690in}{1.782282in}}%
\pgfpathlineto{\pgfqpoint{3.389165in}{1.647569in}}%
\pgfpathlineto{\pgfqpoint{3.349220in}{1.516216in}}%
\pgfpathlineto{\pgfqpoint{3.296206in}{1.389551in}}%
\pgfpathlineto{\pgfqpoint{3.230618in}{1.268830in}}%
\pgfpathlineto{\pgfqpoint{3.153100in}{1.155239in}}%
\pgfpathlineto{\pgfqpoint{3.064442in}{1.049894in}}%
\pgfpathlineto{\pgfqpoint{2.965583in}{0.953840in}}%
\pgfpathlineto{\pgfqpoint{2.857608in}{0.868053in}}%
\pgfpathlineto{\pgfqpoint{2.741705in}{0.793404in}}%
\pgfpathlineto{\pgfqpoint{2.618896in}{0.730525in}}%
\pgfpathlineto{\pgfqpoint{2.490407in}{0.680113in}}%
\pgfpathlineto{\pgfqpoint{2.357529in}{0.642745in}}%
\pgfpathlineto{\pgfqpoint{2.221576in}{0.618838in}}%
\pgfpathlineto{\pgfqpoint{2.083883in}{0.608650in}}%
\pgfpathlineto{\pgfqpoint{1.945804in}{0.612276in}}%
\pgfpathlineto{\pgfqpoint{1.808716in}{0.629654in}}%
\pgfpathlineto{\pgfqpoint{1.674017in}{0.660560in}}%
\pgfpathlineto{\pgfqpoint{1.543125in}{0.704609in}}%
\pgfpathlineto{\pgfqpoint{1.417354in}{0.761311in}}%
\pgfpathlineto{\pgfqpoint{1.297841in}{0.830163in}}%
\pgfpathlineto{\pgfqpoint{1.185861in}{0.910505in}}%
\pgfpathlineto{\pgfqpoint{1.082581in}{1.001566in}}%
\pgfpathlineto{\pgfqpoint{0.989049in}{1.102467in}}%
\pgfpathlineto{\pgfqpoint{0.906191in}{1.212222in}}%
\pgfpathlineto{\pgfqpoint{0.834815in}{1.329736in}}%
\pgfpathlineto{\pgfqpoint{0.775609in}{1.453809in}}%
\pgfpathlineto{\pgfqpoint{0.729143in}{1.583129in}}%
\pgfpathlineto{\pgfqpoint{0.695852in}{1.716312in}}%
\pgfpathlineto{\pgfqpoint{0.675997in}{1.852157in}}%
\pgfpathlineto{\pgfqpoint{0.669841in}{1.989284in}}%
\pgfpathlineto{\pgfqpoint{0.677519in}{2.126288in}}%
\pgfpathlineto{\pgfqpoint{0.699008in}{2.261794in}}%
\pgfpathlineto{\pgfqpoint{0.734124in}{2.394465in}}%
\pgfpathlineto{\pgfqpoint{0.782525in}{2.522998in}}%
\pgfpathlineto{\pgfqpoint{0.843706in}{2.646122in}}%
\pgfpathlineto{\pgfqpoint{0.917006in}{2.762605in}}%
\pgfpathlineto{\pgfqpoint{1.001602in}{2.871246in}}%
\pgfpathlineto{\pgfqpoint{1.096545in}{2.970916in}}%
\pgfpathlineto{\pgfqpoint{1.200991in}{3.060753in}}%
\pgfpathlineto{\pgfqpoint{1.313916in}{3.139800in}}%
\pgfpathlineto{\pgfqpoint{1.434198in}{3.207198in}}%
\pgfpathlineto{\pgfqpoint{1.560659in}{3.262238in}}%
\pgfpathlineto{\pgfqpoint{1.692059in}{3.304361in}}%
\pgfpathlineto{\pgfqpoint{1.827103in}{3.333157in}}%
\pgfpathlineto{\pgfqpoint{1.964435in}{3.348366in}}%
\pgfpathlineto{\pgfqpoint{2.102642in}{3.349880in}}%
\pgfpathlineto{\pgfqpoint{2.240252in}{3.337736in}}%
\pgfpathlineto{\pgfqpoint{2.375946in}{3.312105in}}%
\pgfpathlineto{\pgfqpoint{2.508440in}{3.273209in}}%
\pgfpathlineto{\pgfqpoint{2.636328in}{3.221379in}}%
\pgfpathlineto{\pgfqpoint{2.758298in}{3.157085in}}%
\pgfpathlineto{\pgfqpoint{2.873124in}{3.080930in}}%
\pgfpathlineto{\pgfqpoint{2.979673in}{2.993651in}}%
\pgfpathlineto{\pgfqpoint{3.076902in}{2.896122in}}%
\pgfpathlineto{\pgfqpoint{3.163858in}{2.789351in}}%
\pgfpathlineto{\pgfqpoint{3.239677in}{2.674482in}}%
\pgfpathlineto{\pgfqpoint{3.303588in}{2.552790in}}%
\pgfpathlineto{\pgfqpoint{3.355043in}{2.425420in}}%
\pgfpathlineto{\pgfqpoint{3.393511in}{2.293583in}}%
\pgfpathlineto{\pgfqpoint{3.418524in}{2.158628in}}%
\pgfpathlineto{\pgfqpoint{3.429773in}{2.021902in}}%
\pgfpathlineto{\pgfqpoint{3.427116in}{1.884754in}}%
\pgfpathlineto{\pgfqpoint{3.410571in}{1.748530in}}%
\pgfpathlineto{\pgfqpoint{3.380320in}{1.614577in}}%
\pgfpathlineto{\pgfqpoint{3.336708in}{1.484242in}}%
\pgfpathlineto{\pgfqpoint{3.280243in}{1.358870in}}%
\pgfpathlineto{\pgfqpoint{3.211585in}{1.239788in}}%
\pgfpathlineto{\pgfqpoint{3.131366in}{1.128059in}}%
\pgfpathlineto{\pgfqpoint{3.040346in}{1.024826in}}%
\pgfpathlineto{\pgfqpoint{2.939417in}{0.931187in}}%
\pgfpathlineto{\pgfqpoint{2.829561in}{0.848106in}}%
\pgfpathlineto{\pgfqpoint{2.711852in}{0.776416in}}%
\pgfpathlineto{\pgfqpoint{2.587453in}{0.716820in}}%
\pgfpathlineto{\pgfqpoint{2.457619in}{0.669888in}}%
\pgfpathlineto{\pgfqpoint{2.323696in}{0.636059in}}%
\pgfpathlineto{\pgfqpoint{2.187115in}{0.615637in}}%
\pgfpathlineto{\pgfqpoint{2.049154in}{0.608759in}}%
\pgfpathlineto{\pgfqpoint{1.911171in}{0.615530in}}%
\pgfpathlineto{\pgfqpoint{1.774602in}{0.635955in}}%
\pgfpathlineto{\pgfqpoint{1.640831in}{0.669889in}}%
\pgfpathlineto{\pgfqpoint{1.511186in}{0.717030in}}%
\pgfpathlineto{\pgfqpoint{1.386942in}{0.776924in}}%
\pgfpathlineto{\pgfqpoint{1.269322in}{0.848964in}}%
\pgfpathlineto{\pgfqpoint{1.159492in}{0.932389in}}%
\pgfpathlineto{\pgfqpoint{1.058567in}{1.026284in}}%
\pgfpathlineto{\pgfqpoint{0.967603in}{1.129585in}}%
\pgfpathlineto{\pgfqpoint{0.887404in}{1.241326in}}%
\pgfpathlineto{\pgfqpoint{0.818778in}{1.360446in}}%
\pgfpathlineto{\pgfqpoint{0.762496in}{1.485748in}}%
\pgfpathlineto{\pgfqpoint{0.719169in}{1.615993in}}%
\pgfpathlineto{\pgfqpoint{0.689250in}{1.749904in}}%
\pgfpathlineto{\pgfqpoint{0.673039in}{1.886166in}}%
\pgfpathlineto{\pgfqpoint{0.670673in}{2.023422in}}%
\pgfpathlineto{\pgfqpoint{0.682136in}{2.160279in}}%
\pgfpathlineto{\pgfqpoint{0.707252in}{2.295305in}}%
\pgfpathlineto{\pgfqpoint{0.745705in}{2.427095in}}%
\pgfpathlineto{\pgfqpoint{0.797138in}{2.554485in}}%
\pgfpathlineto{\pgfqpoint{0.861081in}{2.676136in}}%
\pgfpathlineto{\pgfqpoint{0.936937in}{2.790773in}}%
\pgfpathlineto{\pgfqpoint{1.023987in}{2.897233in}}%
\pgfpathlineto{\pgfqpoint{1.121390in}{2.994456in}}%
\pgfpathlineto{\pgfqpoint{1.228181in}{3.081488in}}%
\pgfpathlineto{\pgfqpoint{1.343275in}{3.157483in}}%
\pgfpathlineto{\pgfqpoint{1.465462in}{3.221700in}}%
\pgfpathlineto{\pgfqpoint{1.593413in}{3.273504in}}%
\pgfpathlineto{\pgfqpoint{1.725848in}{3.312434in}}%
\pgfpathlineto{\pgfqpoint{1.861556in}{3.338125in}}%
\pgfpathlineto{\pgfqpoint{1.999133in}{3.350231in}}%
\pgfpathlineto{\pgfqpoint{2.137189in}{3.348569in}}%
\pgfpathlineto{\pgfqpoint{2.274359in}{3.333115in}}%
\pgfpathlineto{\pgfqpoint{2.409292in}{3.304006in}}%
\pgfpathlineto{\pgfqpoint{2.540662in}{3.261543in}}%
\pgfpathlineto{\pgfqpoint{2.667157in}{3.206184in}}%
\pgfpathlineto{\pgfqpoint{2.787489in}{3.138550in}}%
\pgfpathlineto{\pgfqpoint{2.900388in}{3.059423in}}%
\pgfpathlineto{\pgfqpoint{3.004791in}{2.969591in}}%
\pgfpathlineto{\pgfqpoint{3.099705in}{2.869873in}}%
\pgfpathlineto{\pgfqpoint{3.184095in}{2.761261in}}%
\pgfpathlineto{\pgfqpoint{3.257074in}{2.644819in}}%
\pgfpathlineto{\pgfqpoint{3.317897in}{2.521685in}}%
\pgfpathlineto{\pgfqpoint{3.365961in}{2.393067in}}%
\pgfpathlineto{\pgfqpoint{3.400809in}{2.260251in}}%
\pgfpathlineto{\pgfqpoint{3.422126in}{2.124591in}}%
\pgfpathlineto{\pgfqpoint{3.429742in}{1.987518in}}%
\pgfpathlineto{\pgfqpoint{3.423636in}{1.850440in}}%
\pgfpathlineto{\pgfqpoint{3.403880in}{1.714589in}}%
\pgfpathlineto{\pgfqpoint{3.370606in}{1.581391in}}%
\pgfpathlineto{\pgfqpoint{3.324088in}{1.452222in}}%
\pgfpathlineto{\pgfqpoint{3.264744in}{1.328383in}}%
\pgfpathlineto{\pgfqpoint{3.193137in}{1.211098in}}%
\pgfpathlineto{\pgfqpoint{3.109976in}{1.101518in}}%
\pgfpathlineto{\pgfqpoint{3.016116in}{1.000722in}}%
\pgfpathlineto{\pgfqpoint{2.912556in}{0.909711in}}%
\pgfpathlineto{\pgfqpoint{2.800441in}{0.829413in}}%
\pgfpathlineto{\pgfqpoint{2.680956in}{0.760614in}}%
\pgfpathlineto{\pgfqpoint{2.555153in}{0.703912in}}%
\pgfpathlineto{\pgfqpoint{2.424311in}{0.659962in}}%
\pgfpathlineto{\pgfqpoint{2.289741in}{0.629270in}}%
\pgfpathlineto{\pgfqpoint{2.152771in}{0.612181in}}%
\pgfpathlineto{\pgfqpoint{2.014746in}{0.608879in}}%
\pgfpathlineto{\pgfqpoint{1.877026in}{0.619387in}}%
\pgfpathlineto{\pgfqpoint{1.740989in}{0.643569in}}%
\pgfpathlineto{\pgfqpoint{1.608029in}{0.681126in}}%
\pgfpathlineto{\pgfqpoint{1.479557in}{0.731599in}}%
\pgfpathlineto{\pgfqpoint{1.356801in}{0.794466in}}%
\pgfpathlineto{\pgfqpoint{1.240923in}{0.869163in}}%
\pgfpathlineto{\pgfqpoint{1.133162in}{0.954966in}}%
\pgfpathlineto{\pgfqpoint{1.034635in}{1.051049in}}%
\pgfpathlineto{\pgfqpoint{0.946333in}{1.156479in}}%
\pgfpathlineto{\pgfqpoint{0.869130in}{1.270222in}}%
\pgfpathlineto{\pgfqpoint{0.803775in}{1.391137in}}%
\pgfpathlineto{\pgfqpoint{0.750895in}{1.517981in}}%
\pgfpathlineto{\pgfqpoint{0.710994in}{1.649404in}}%
\pgfpathlineto{\pgfqpoint{0.684426in}{1.784048in}}%
\pgfpathlineto{\pgfqpoint{0.671414in}{1.920697in}}%
\pgfpathlineto{\pgfqpoint{0.672170in}{2.057928in}}%
\pgfpathlineto{\pgfqpoint{0.686755in}{2.194344in}}%
\pgfpathlineto{\pgfqpoint{0.715070in}{2.328584in}}%
\pgfpathlineto{\pgfqpoint{0.756856in}{2.459325in}}%
\pgfpathlineto{\pgfqpoint{0.811697in}{2.585282in}}%
\pgfpathlineto{\pgfqpoint{0.879017in}{2.705209in}}%
\pgfpathlineto{\pgfqpoint{0.958081in}{2.817896in}}%
\pgfpathlineto{\pgfqpoint{1.047997in}{2.922172in}}%
\pgfpathlineto{\pgfqpoint{1.147803in}{3.016997in}}%
\pgfpathlineto{\pgfqpoint{1.256621in}{3.101534in}}%
\pgfpathlineto{\pgfqpoint{1.373362in}{3.174851in}}%
\pgfpathlineto{\pgfqpoint{1.496870in}{3.236154in}}%
\pgfpathlineto{\pgfqpoint{1.625935in}{3.284804in}}%
\pgfpathlineto{\pgfqpoint{1.759292in}{3.320311in}}%
\pgfpathlineto{\pgfqpoint{1.895621in}{3.342336in}}%
\pgfpathlineto{\pgfqpoint{2.033546in}{3.350691in}}%
\pgfpathlineto{\pgfqpoint{2.171636in}{3.345340in}}%
\pgfpathlineto{\pgfqpoint{2.308419in}{3.326396in}}%
\pgfpathlineto{\pgfqpoint{2.442659in}{3.294075in}}%
\pgfpathlineto{\pgfqpoint{2.573005in}{3.248652in}}%
\pgfpathlineto{\pgfqpoint{2.698088in}{3.190529in}}%
\pgfpathlineto{\pgfqpoint{2.816629in}{3.120240in}}%
\pgfpathlineto{\pgfqpoint{2.927445in}{3.038451in}}%
\pgfpathlineto{\pgfqpoint{3.029443in}{2.945964in}}%
\pgfpathlineto{\pgfqpoint{3.121627in}{2.843710in}}%
\pgfpathlineto{\pgfqpoint{3.203092in}{2.732756in}}%
\pgfpathlineto{\pgfqpoint{3.273027in}{2.614300in}}%
\pgfpathlineto{\pgfqpoint{3.330734in}{2.489638in}}%
\pgfpathlineto{\pgfqpoint{3.375735in}{2.359880in}}%
\pgfpathlineto{\pgfqpoint{3.407515in}{2.226323in}}%
\pgfpathlineto{\pgfqpoint{3.425682in}{2.090323in}}%
\pgfpathlineto{\pgfqpoint{3.430002in}{1.953234in}}%
\pgfpathlineto{\pgfqpoint{3.420408in}{1.816404in}}%
\pgfpathlineto{\pgfqpoint{3.396994in}{1.681178in}}%
\pgfpathlineto{\pgfqpoint{3.360018in}{1.548897in}}%
\pgfpathlineto{\pgfqpoint{3.309900in}{1.420899in}}%
\pgfpathlineto{\pgfqpoint{3.247223in}{1.298516in}}%
\pgfpathlineto{\pgfqpoint{3.172692in}{1.183014in}}%
\pgfpathlineto{\pgfqpoint{3.086965in}{1.075413in}}%
\pgfpathlineto{\pgfqpoint{2.990878in}{0.976854in}}%
\pgfpathlineto{\pgfqpoint{2.885372in}{0.888380in}}%
\pgfpathlineto{\pgfqpoint{2.771473in}{0.810896in}}%
\pgfpathlineto{\pgfqpoint{2.650297in}{0.745176in}}%
\pgfpathlineto{\pgfqpoint{2.523048in}{0.691859in}}%
\pgfpathlineto{\pgfqpoint{2.391015in}{0.651449in}}%
\pgfpathlineto{\pgfqpoint{2.255577in}{0.624317in}}%
\pgfpathlineto{\pgfqpoint{2.118172in}{0.610696in}}%
\pgfpathlineto{\pgfqpoint{1.980043in}{0.610663in}}%
\pgfpathlineto{\pgfqpoint{1.842599in}{0.624280in}}%
\pgfpathlineto{\pgfqpoint{1.707262in}{0.651478in}}%
\pgfpathlineto{\pgfqpoint{1.575398in}{0.692039in}}%
\pgfpathlineto{\pgfqpoint{1.448314in}{0.745591in}}%
\pgfpathlineto{\pgfqpoint{1.327258in}{0.811611in}}%
\pgfpathlineto{\pgfqpoint{1.213421in}{0.889424in}}%
\pgfpathlineto{\pgfqpoint{1.107937in}{0.978202in}}%
\pgfpathlineto{\pgfqpoint{1.011880in}{1.076966in}}%
\pgfpathlineto{\pgfqpoint{0.926245in}{1.184613in}}%
\pgfpathlineto{\pgfqpoint{0.851761in}{1.300186in}}%
\pgfpathlineto{\pgfqpoint{0.789229in}{1.422546in}}%
\pgfpathlineto{\pgfqpoint{0.739350in}{1.550467in}}%
\pgfpathlineto{\pgfqpoint{0.702665in}{1.682688in}}%
\pgfpathlineto{\pgfqpoint{0.679558in}{1.817911in}}%
\pgfpathlineto{\pgfqpoint{0.670253in}{1.954806in}}%
\pgfpathlineto{\pgfqpoint{0.674816in}{2.092006in}}%
\pgfpathlineto{\pgfqpoint{0.693156in}{2.228111in}}%
\pgfpathlineto{\pgfqpoint{0.725022in}{2.361686in}}%
\pgfpathlineto{\pgfqpoint{0.770042in}{2.491389in}}%
\pgfpathlineto{\pgfqpoint{0.827814in}{2.616049in}}%
\pgfpathlineto{\pgfqpoint{0.897798in}{2.734338in}}%
\pgfpathlineto{\pgfqpoint{0.979336in}{2.845025in}}%
\pgfpathlineto{\pgfqpoint{1.071648in}{2.946989in}}%
\pgfpathlineto{\pgfqpoint{1.173836in}{3.039221in}}%
\pgfpathlineto{\pgfqpoint{1.284884in}{3.120818in}}%
\pgfpathlineto{\pgfqpoint{1.403655in}{3.190988in}}%
\pgfpathlineto{\pgfqpoint{1.528894in}{3.249047in}}%
\pgfpathlineto{\pgfqpoint{1.659232in}{3.294426in}}%
\pgfpathlineto{\pgfqpoint{1.793458in}{3.326753in}}%
\pgfpathlineto{\pgfqpoint{1.930266in}{3.345672in}}%
\pgfpathlineto{\pgfqpoint{2.068248in}{3.350912in}}%
\pgfpathlineto{\pgfqpoint{2.206020in}{3.342365in}}%
\pgfpathlineto{\pgfqpoint{2.342219in}{3.320081in}}%
\pgfpathlineto{\pgfqpoint{2.475507in}{3.284271in}}%
\pgfpathlineto{\pgfqpoint{2.604569in}{3.235309in}}%
\pgfpathlineto{\pgfqpoint{2.728115in}{3.173727in}}%
\pgfpathlineto{\pgfqpoint{2.844875in}{3.100217in}}%
\pgfpathlineto{\pgfqpoint{2.953617in}{3.015625in}}%
\pgfpathlineto{\pgfqpoint{3.053378in}{2.920731in}}%
\pgfpathlineto{\pgfqpoint{3.143139in}{2.816440in}}%
\pgfpathlineto{\pgfqpoint{3.221928in}{2.703786in}}%
\pgfpathlineto{\pgfqpoint{3.288921in}{2.583871in}}%
\pgfpathlineto{\pgfqpoint{3.343436in}{2.457869in}}%
\pgfpathlineto{\pgfqpoint{3.384939in}{2.327021in}}%
\pgfpathlineto{\pgfqpoint{3.413041in}{2.192637in}}%
\pgfpathlineto{\pgfqpoint{3.427499in}{2.056099in}}%
\pgfpathlineto{\pgfqpoint{3.428214in}{1.918856in}}%
\pgfpathlineto{\pgfqpoint{3.415235in}{1.782261in}}%
\pgfpathlineto{\pgfqpoint{3.388675in}{1.647577in}}%
\pgfpathlineto{\pgfqpoint{3.348736in}{1.516231in}}%
\pgfpathlineto{\pgfqpoint{3.295761in}{1.389571in}}%
\pgfpathlineto{\pgfqpoint{3.230236in}{1.268867in}}%
\pgfpathlineto{\pgfqpoint{3.152791in}{1.155312in}}%
\pgfpathlineto{\pgfqpoint{3.064200in}{1.050019in}}%
\pgfpathlineto{\pgfqpoint{2.965378in}{0.954024in}}%
\pgfpathlineto{\pgfqpoint{2.857384in}{0.868283in}}%
\pgfpathlineto{\pgfqpoint{2.741420in}{0.793675in}}%
\pgfpathlineto{\pgfqpoint{2.618639in}{0.730884in}}%
\pgfpathlineto{\pgfqpoint{2.490159in}{0.680504in}}%
\pgfpathlineto{\pgfqpoint{2.357293in}{0.643128in}}%
\pgfpathlineto{\pgfqpoint{2.221370in}{0.619190in}}%
\pgfpathlineto{\pgfqpoint{2.083727in}{0.608961in}}%
\pgfpathlineto{\pgfqpoint{1.945715in}{0.612552in}}%
\pgfpathlineto{\pgfqpoint{1.808700in}{0.629915in}}%
\pgfpathlineto{\pgfqpoint{1.674056in}{0.660836in}}%
\pgfpathlineto{\pgfqpoint{1.543171in}{0.704944in}}%
\pgfpathlineto{\pgfqpoint{1.417438in}{0.761708in}}%
\pgfpathlineto{\pgfqpoint{1.298001in}{0.830587in}}%
\pgfpathlineto{\pgfqpoint{1.186054in}{0.910940in}}%
\pgfpathlineto{\pgfqpoint{1.082784in}{1.001989in}}%
\pgfpathlineto{\pgfqpoint{0.989254in}{1.102851in}}%
\pgfpathlineto{\pgfqpoint{0.906404in}{1.212544in}}%
\pgfpathlineto{\pgfqpoint{0.835047in}{1.329987in}}%
\pgfpathlineto{\pgfqpoint{0.775873in}{1.453996in}}%
\pgfpathlineto{\pgfqpoint{0.729445in}{1.583285in}}%
\pgfpathlineto{\pgfqpoint{0.696205in}{1.716471in}}%
\pgfpathlineto{\pgfqpoint{0.676415in}{1.852263in}}%
\pgfpathlineto{\pgfqpoint{0.670285in}{1.989367in}}%
\pgfpathlineto{\pgfqpoint{0.677956in}{2.126356in}}%
\pgfpathlineto{\pgfqpoint{0.699414in}{2.261842in}}%
\pgfpathlineto{\pgfqpoint{0.734484in}{2.394479in}}%
\pgfpathlineto{\pgfqpoint{0.782835in}{2.522964in}}%
\pgfpathlineto{\pgfqpoint{0.843976in}{2.646031in}}%
\pgfpathlineto{\pgfqpoint{0.917260in}{2.762459in}}%
\pgfpathlineto{\pgfqpoint{1.001881in}{2.871066in}}%
\pgfpathlineto{\pgfqpoint{1.096873in}{2.970710in}}%
\pgfpathlineto{\pgfqpoint{1.201308in}{3.060474in}}%
\pgfpathlineto{\pgfqpoint{1.314224in}{3.139494in}}%
\pgfpathlineto{\pgfqpoint{1.434487in}{3.206892in}}%
\pgfpathlineto{\pgfqpoint{1.560911in}{3.261946in}}%
\pgfpathlineto{\pgfqpoint{1.692259in}{3.304084in}}%
\pgfpathlineto{\pgfqpoint{1.827240in}{3.332887in}}%
\pgfpathlineto{\pgfqpoint{1.964511in}{3.348088in}}%
\pgfpathlineto{\pgfqpoint{2.102679in}{3.349573in}}%
\pgfpathlineto{\pgfqpoint{2.240298in}{3.337378in}}%
\pgfpathlineto{\pgfqpoint{2.375929in}{3.311689in}}%
\pgfpathlineto{\pgfqpoint{2.508371in}{3.272757in}}%
\pgfpathlineto{\pgfqpoint{2.636240in}{3.220920in}}%
\pgfpathlineto{\pgfqpoint{2.758201in}{3.156648in}}%
\pgfpathlineto{\pgfqpoint{2.873016in}{3.080539in}}%
\pgfpathlineto{\pgfqpoint{2.979543in}{2.993323in}}%
\pgfpathlineto{\pgfqpoint{3.076737in}{2.895860in}}%
\pgfpathlineto{\pgfqpoint{3.163646in}{2.789142in}}%
\pgfpathlineto{\pgfqpoint{3.239419in}{2.674291in}}%
\pgfpathlineto{\pgfqpoint{3.303298in}{2.552559in}}%
\pgfpathlineto{\pgfqpoint{3.354674in}{2.425221in}}%
\pgfpathlineto{\pgfqpoint{3.393103in}{2.293406in}}%
\pgfpathlineto{\pgfqpoint{3.418112in}{2.158466in}}%
\pgfpathlineto{\pgfqpoint{3.429384in}{2.021764in}}%
\pgfpathlineto{\pgfqpoint{3.426763in}{1.884653in}}%
\pgfpathlineto{\pgfqpoint{3.410256in}{1.748483in}}%
\pgfpathlineto{\pgfqpoint{3.380033in}{1.614592in}}%
\pgfpathlineto{\pgfqpoint{3.336429in}{1.484315in}}%
\pgfpathlineto{\pgfqpoint{3.279938in}{1.358978in}}%
\pgfpathlineto{\pgfqpoint{3.211220in}{1.239899in}}%
\pgfpathlineto{\pgfqpoint{3.130996in}{1.128247in}}%
\pgfpathlineto{\pgfqpoint{3.039978in}{1.025047in}}%
\pgfpathlineto{\pgfqpoint{2.939066in}{0.931413in}}%
\pgfpathlineto{\pgfqpoint{2.829247in}{0.848327in}}%
\pgfpathlineto{\pgfqpoint{2.711589in}{0.776636in}}%
\pgfpathlineto{\pgfqpoint{2.587250in}{0.717049in}}%
\pgfpathlineto{\pgfqpoint{2.457470in}{0.670141in}}%
\pgfpathlineto{\pgfqpoint{2.323573in}{0.636350in}}%
\pgfpathlineto{\pgfqpoint{2.186969in}{0.615979in}}%
\pgfpathlineto{\pgfqpoint{2.049073in}{0.609182in}}%
\pgfpathlineto{\pgfqpoint{1.911124in}{0.615997in}}%
\pgfpathlineto{\pgfqpoint{1.774567in}{0.636426in}}%
\pgfpathlineto{\pgfqpoint{1.640804in}{0.670330in}}%
\pgfpathlineto{\pgfqpoint{1.511180in}{0.717418in}}%
\pgfpathlineto{\pgfqpoint{1.386974in}{0.777248in}}%
\pgfpathlineto{\pgfqpoint{1.269408in}{0.849227in}}%
\pgfpathlineto{\pgfqpoint{1.159639in}{0.932615in}}%
\pgfpathlineto{\pgfqpoint{1.058764in}{1.026516in}}%
\pgfpathlineto{\pgfqpoint{0.967817in}{1.129889in}}%
\pgfpathlineto{\pgfqpoint{0.887708in}{1.241620in}}%
\pgfpathlineto{\pgfqpoint{0.819129in}{1.360733in}}%
\pgfpathlineto{\pgfqpoint{0.762850in}{1.486024in}}%
\pgfpathlineto{\pgfqpoint{0.719501in}{1.616242in}}%
\pgfpathlineto{\pgfqpoint{0.689552in}{1.750107in}}%
\pgfpathlineto{\pgfqpoint{0.673314in}{1.886303in}}%
\pgfpathlineto{\pgfqpoint{0.670938in}{2.023489in}}%
\pgfpathlineto{\pgfqpoint{0.682415in}{2.160289in}}%
\pgfpathlineto{\pgfqpoint{0.707580in}{2.295298in}}%
\pgfpathlineto{\pgfqpoint{0.746104in}{2.427080in}}%
\pgfpathlineto{\pgfqpoint{0.797579in}{2.554384in}}%
\pgfpathlineto{\pgfqpoint{0.861544in}{2.675999in}}%
\pgfpathlineto{\pgfqpoint{0.937393in}{2.790626in}}%
\pgfpathlineto{\pgfqpoint{1.024407in}{2.897083in}}%
\pgfpathlineto{\pgfqpoint{1.121750in}{2.994297in}}%
\pgfpathlineto{\pgfqpoint{1.228467in}{3.081307in}}%
\pgfpathlineto{\pgfqpoint{1.343492in}{3.157266in}}%
\pgfpathlineto{\pgfqpoint{1.465637in}{3.221439in}}%
\pgfpathlineto{\pgfqpoint{1.593603in}{3.273201in}}%
\pgfpathlineto{\pgfqpoint{1.726005in}{3.312038in}}%
\pgfpathlineto{\pgfqpoint{1.861618in}{3.337542in}}%
\pgfpathlineto{\pgfqpoint{1.999118in}{3.349452in}}%
\pgfpathlineto{\pgfqpoint{1.999118in}{3.349452in}}%
\pgfusepath{stroke}%
\end{pgfscope}%
\begin{pgfscope}%
\pgfpathrectangle{\pgfqpoint{0.500000in}{0.440000in}}{\pgfqpoint{3.100000in}{3.080000in}}%
\pgfusepath{clip}%
\pgfsetrectcap%
\pgfsetroundjoin%
\pgfsetlinewidth{0.501875pt}%
\definecolor{currentstroke}{rgb}{0.580392,0.403922,0.741176}%
\pgfsetstrokecolor{currentstroke}%
\pgfsetdash{}{0pt}%
\pgfpathmoveto{\pgfqpoint{1.895634in}{3.206981in}}%
\pgfpathlineto{\pgfqpoint{2.019834in}{3.216169in}}%
\pgfpathlineto{\pgfqpoint{2.144356in}{3.212993in}}%
\pgfpathlineto{\pgfqpoint{2.267929in}{3.197475in}}%
\pgfpathlineto{\pgfqpoint{2.389304in}{3.169755in}}%
\pgfpathlineto{\pgfqpoint{2.507271in}{3.130101in}}%
\pgfpathlineto{\pgfqpoint{2.620661in}{3.078915in}}%
\pgfpathlineto{\pgfqpoint{2.728346in}{3.016726in}}%
\pgfpathlineto{\pgfqpoint{2.829268in}{2.944182in}}%
\pgfpathlineto{\pgfqpoint{2.922464in}{2.862005in}}%
\pgfpathlineto{\pgfqpoint{3.006931in}{2.771012in}}%
\pgfpathlineto{\pgfqpoint{3.081776in}{2.672098in}}%
\pgfpathlineto{\pgfqpoint{3.146236in}{2.566229in}}%
\pgfpathlineto{\pgfqpoint{3.199667in}{2.454446in}}%
\pgfpathlineto{\pgfqpoint{3.241554in}{2.337859in}}%
\pgfpathlineto{\pgfqpoint{3.271505in}{2.217654in}}%
\pgfpathlineto{\pgfqpoint{3.289251in}{2.095089in}}%
\pgfpathlineto{\pgfqpoint{3.294661in}{1.971431in}}%
\pgfpathlineto{\pgfqpoint{3.287710in}{1.847802in}}%
\pgfpathlineto{\pgfqpoint{3.268406in}{1.725492in}}%
\pgfpathlineto{\pgfqpoint{3.236884in}{1.605761in}}%
\pgfpathlineto{\pgfqpoint{3.193414in}{1.489813in}}%
\pgfpathlineto{\pgfqpoint{3.138403in}{1.378795in}}%
\pgfpathlineto{\pgfqpoint{3.072397in}{1.273796in}}%
\pgfpathlineto{\pgfqpoint{2.996073in}{1.175850in}}%
\pgfpathlineto{\pgfqpoint{2.910248in}{1.085936in}}%
\pgfpathlineto{\pgfqpoint{2.815874in}{1.004973in}}%
\pgfpathlineto{\pgfqpoint{2.713970in}{0.933773in}}%
\pgfpathlineto{\pgfqpoint{2.605428in}{0.872950in}}%
\pgfpathlineto{\pgfqpoint{2.491343in}{0.823185in}}%
\pgfpathlineto{\pgfqpoint{2.372855in}{0.785037in}}%
\pgfpathlineto{\pgfqpoint{2.251131in}{0.758922in}}%
\pgfpathlineto{\pgfqpoint{2.127365in}{0.745111in}}%
\pgfpathlineto{\pgfqpoint{2.002778in}{0.743734in}}%
\pgfpathlineto{\pgfqpoint{1.878617in}{0.754776in}}%
\pgfpathlineto{\pgfqpoint{1.756155in}{0.778080in}}%
\pgfpathlineto{\pgfqpoint{1.636695in}{0.813347in}}%
\pgfpathlineto{\pgfqpoint{1.521379in}{0.860196in}}%
\pgfpathlineto{\pgfqpoint{1.411297in}{0.918210in}}%
\pgfpathlineto{\pgfqpoint{1.307622in}{0.986840in}}%
\pgfpathlineto{\pgfqpoint{1.211428in}{1.065432in}}%
\pgfpathlineto{\pgfqpoint{1.123684in}{1.153230in}}%
\pgfpathlineto{\pgfqpoint{1.045258in}{1.249376in}}%
\pgfpathlineto{\pgfqpoint{0.976913in}{1.352908in}}%
\pgfpathlineto{\pgfqpoint{0.919312in}{1.462759in}}%
\pgfpathlineto{\pgfqpoint{0.873012in}{1.577761in}}%
\pgfpathlineto{\pgfqpoint{0.838456in}{1.696681in}}%
\pgfpathlineto{\pgfqpoint{0.815919in}{1.818453in}}%
\pgfpathlineto{\pgfqpoint{0.805690in}{1.941841in}}%
\pgfpathlineto{\pgfqpoint{0.807939in}{2.065583in}}%
\pgfpathlineto{\pgfqpoint{0.822688in}{2.188442in}}%
\pgfpathlineto{\pgfqpoint{0.849817in}{2.309206in}}%
\pgfpathlineto{\pgfqpoint{0.889060in}{2.426689in}}%
\pgfpathlineto{\pgfqpoint{0.940003in}{2.539729in}}%
\pgfpathlineto{\pgfqpoint{1.002092in}{2.647189in}}%
\pgfpathlineto{\pgfqpoint{1.074623in}{2.747959in}}%
\pgfpathlineto{\pgfqpoint{1.156788in}{2.840996in}}%
\pgfpathlineto{\pgfqpoint{1.247860in}{2.925492in}}%
\pgfpathlineto{\pgfqpoint{1.346943in}{3.000541in}}%
\pgfpathlineto{\pgfqpoint{1.453057in}{3.065335in}}%
\pgfpathlineto{\pgfqpoint{1.565164in}{3.119198in}}%
\pgfpathlineto{\pgfqpoint{1.682166in}{3.161587in}}%
\pgfpathlineto{\pgfqpoint{1.802906in}{3.192090in}}%
\pgfpathlineto{\pgfqpoint{1.926170in}{3.210430in}}%
\pgfpathlineto{\pgfqpoint{2.050683in}{3.216459in}}%
\pgfpathlineto{\pgfqpoint{2.175114in}{3.210164in}}%
\pgfpathlineto{\pgfqpoint{2.298306in}{3.191651in}}%
\pgfpathlineto{\pgfqpoint{2.419056in}{3.161067in}}%
\pgfpathlineto{\pgfqpoint{2.536092in}{3.118666in}}%
\pgfpathlineto{\pgfqpoint{2.648216in}{3.064823in}}%
\pgfpathlineto{\pgfqpoint{2.754309in}{3.000044in}}%
\pgfpathlineto{\pgfqpoint{2.853324in}{2.924957in}}%
\pgfpathlineto{\pgfqpoint{2.944292in}{2.840316in}}%
\pgfpathlineto{\pgfqpoint{3.026318in}{2.747003in}}%
\pgfpathlineto{\pgfqpoint{3.098581in}{2.646024in}}%
\pgfpathlineto{\pgfqpoint{3.160344in}{2.538503in}}%
\pgfpathlineto{\pgfqpoint{3.211077in}{2.425433in}}%
\pgfpathlineto{\pgfqpoint{3.250250in}{2.307903in}}%
\pgfpathlineto{\pgfqpoint{3.277396in}{2.187109in}}%
\pgfpathlineto{\pgfqpoint{3.292194in}{2.064255in}}%
\pgfpathlineto{\pgfqpoint{3.294471in}{1.940548in}}%
\pgfpathlineto{\pgfqpoint{3.284202in}{1.817203in}}%
\pgfpathlineto{\pgfqpoint{3.261506in}{1.695444in}}%
\pgfpathlineto{\pgfqpoint{3.226650in}{1.576497in}}%
\pgfpathlineto{\pgfqpoint{3.180048in}{1.461598in}}%
\pgfpathlineto{\pgfqpoint{3.122244in}{1.351955in}}%
\pgfpathlineto{\pgfqpoint{3.053758in}{1.248542in}}%
\pgfpathlineto{\pgfqpoint{2.975244in}{1.152433in}}%
\pgfpathlineto{\pgfqpoint{2.887468in}{1.064642in}}%
\pgfpathlineto{\pgfqpoint{2.791282in}{0.986065in}}%
\pgfpathlineto{\pgfqpoint{2.687627in}{0.917488in}}%
\pgfpathlineto{\pgfqpoint{2.577530in}{0.859581in}}%
\pgfpathlineto{\pgfqpoint{2.462109in}{0.812899in}}%
\pgfpathlineto{\pgfqpoint{2.342567in}{0.777884in}}%
\pgfpathlineto{\pgfqpoint{2.220176in}{0.754856in}}%
\pgfpathlineto{\pgfqpoint{2.096054in}{0.743985in}}%
\pgfpathlineto{\pgfqpoint{1.971460in}{0.745429in}}%
\pgfpathlineto{\pgfqpoint{1.847679in}{0.759241in}}%
\pgfpathlineto{\pgfqpoint{1.725956in}{0.785331in}}%
\pgfpathlineto{\pgfqpoint{1.607495in}{0.823470in}}%
\pgfpathlineto{\pgfqpoint{1.493462in}{0.873285in}}%
\pgfpathlineto{\pgfqpoint{1.384978in}{0.934265in}}%
\pgfpathlineto{\pgfqpoint{1.283128in}{1.005757in}}%
\pgfpathlineto{\pgfqpoint{1.188954in}{1.086967in}}%
\pgfpathlineto{\pgfqpoint{1.103438in}{1.176980in}}%
\pgfpathlineto{\pgfqpoint{1.027318in}{1.274987in}}%
\pgfpathlineto{\pgfqpoint{0.961399in}{1.380034in}}%
\pgfpathlineto{\pgfqpoint{0.906405in}{1.491071in}}%
\pgfpathlineto{\pgfqpoint{0.862923in}{1.607007in}}%
\pgfpathlineto{\pgfqpoint{0.831402in}{1.726705in}}%
\pgfpathlineto{\pgfqpoint{0.812148in}{1.848987in}}%
\pgfpathlineto{\pgfqpoint{0.805330in}{1.972631in}}%
\pgfpathlineto{\pgfqpoint{0.810976in}{2.096370in}}%
\pgfpathlineto{\pgfqpoint{0.828978in}{2.218896in}}%
\pgfpathlineto{\pgfqpoint{0.859106in}{2.338990in}}%
\pgfpathlineto{\pgfqpoint{0.901092in}{2.455557in}}%
\pgfpathlineto{\pgfqpoint{0.954559in}{2.567356in}}%
\pgfpathlineto{\pgfqpoint{1.019014in}{2.673230in}}%
\pgfpathlineto{\pgfqpoint{1.093850in}{2.772111in}}%
\pgfpathlineto{\pgfqpoint{1.178342in}{2.863019in}}%
\pgfpathlineto{\pgfqpoint{1.271647in}{2.945065in}}%
\pgfpathlineto{\pgfqpoint{1.372809in}{3.017448in}}%
\pgfpathlineto{\pgfqpoint{1.480754in}{3.079455in}}%
\pgfpathlineto{\pgfqpoint{1.594292in}{3.130465in}}%
\pgfpathlineto{\pgfqpoint{1.712335in}{3.170038in}}%
\pgfpathlineto{\pgfqpoint{1.833772in}{3.197773in}}%
\pgfpathlineto{\pgfqpoint{1.957354in}{3.213316in}}%
\pgfpathlineto{\pgfqpoint{2.081841in}{3.216457in}}%
\pgfpathlineto{\pgfqpoint{2.206003in}{3.207135in}}%
\pgfpathlineto{\pgfqpoint{2.328622in}{3.185432in}}%
\pgfpathlineto{\pgfqpoint{2.448487in}{3.151577in}}%
\pgfpathlineto{\pgfqpoint{2.564398in}{3.105944in}}%
\pgfpathlineto{\pgfqpoint{2.675164in}{3.049055in}}%
\pgfpathlineto{\pgfqpoint{2.779612in}{2.981570in}}%
\pgfpathlineto{\pgfqpoint{2.876799in}{2.904128in}}%
\pgfpathlineto{\pgfqpoint{2.965749in}{2.817454in}}%
\pgfpathlineto{\pgfqpoint{3.045509in}{2.722404in}}%
\pgfpathlineto{\pgfqpoint{3.115247in}{2.619906in}}%
\pgfpathlineto{\pgfqpoint{3.174259in}{2.510961in}}%
\pgfpathlineto{\pgfqpoint{3.221966in}{2.396642in}}%
\pgfpathlineto{\pgfqpoint{3.257912in}{2.278098in}}%
\pgfpathlineto{\pgfqpoint{3.281770in}{2.156548in}}%
\pgfpathlineto{\pgfqpoint{3.293335in}{2.033285in}}%
\pgfpathlineto{\pgfqpoint{3.292545in}{1.909515in}}%
\pgfpathlineto{\pgfqpoint{3.279399in}{1.786399in}}%
\pgfpathlineto{\pgfqpoint{3.253961in}{1.665233in}}%
\pgfpathlineto{\pgfqpoint{3.216435in}{1.547255in}}%
\pgfpathlineto{\pgfqpoint{3.167154in}{1.433643in}}%
\pgfpathlineto{\pgfqpoint{3.106593in}{1.325518in}}%
\pgfpathlineto{\pgfqpoint{3.035355in}{1.223940in}}%
\pgfpathlineto{\pgfqpoint{2.954185in}{1.129914in}}%
\pgfpathlineto{\pgfqpoint{2.863957in}{1.044382in}}%
\pgfpathlineto{\pgfqpoint{2.765684in}{0.968229in}}%
\pgfpathlineto{\pgfqpoint{2.660346in}{0.902164in}}%
\pgfpathlineto{\pgfqpoint{2.548899in}{0.846806in}}%
\pgfpathlineto{\pgfqpoint{2.432477in}{0.802791in}}%
\pgfpathlineto{\pgfqpoint{2.312237in}{0.770610in}}%
\pgfpathlineto{\pgfqpoint{2.189361in}{0.750613in}}%
\pgfpathlineto{\pgfqpoint{2.065056in}{0.743003in}}%
\pgfpathlineto{\pgfqpoint{1.940552in}{0.747843in}}%
\pgfpathlineto{\pgfqpoint{1.817102in}{0.765049in}}%
\pgfpathlineto{\pgfqpoint{1.695986in}{0.794395in}}%
\pgfpathlineto{\pgfqpoint{1.578490in}{0.835518in}}%
\pgfpathlineto{\pgfqpoint{1.465668in}{0.888022in}}%
\pgfpathlineto{\pgfqpoint{1.358664in}{0.951422in}}%
\pgfpathlineto{\pgfqpoint{1.258610in}{1.025115in}}%
\pgfpathlineto{\pgfqpoint{1.166531in}{1.108396in}}%
\pgfpathlineto{\pgfqpoint{1.083351in}{1.200460in}}%
\pgfpathlineto{\pgfqpoint{1.009886in}{1.300399in}}%
\pgfpathlineto{\pgfqpoint{0.946850in}{1.407204in}}%
\pgfpathlineto{\pgfqpoint{0.894852in}{1.519767in}}%
\pgfpathlineto{\pgfqpoint{0.854395in}{1.636877in}}%
\pgfpathlineto{\pgfqpoint{0.825839in}{1.757342in}}%
\pgfpathlineto{\pgfqpoint{0.809439in}{1.880068in}}%
\pgfpathlineto{\pgfqpoint{0.805438in}{2.003779in}}%
\pgfpathlineto{\pgfqpoint{0.813936in}{2.127222in}}%
\pgfpathlineto{\pgfqpoint{0.834885in}{2.249169in}}%
\pgfpathlineto{\pgfqpoint{0.868097in}{2.368419in}}%
\pgfpathlineto{\pgfqpoint{0.913236in}{2.483801in}}%
\pgfpathlineto{\pgfqpoint{0.969822in}{2.594169in}}%
\pgfpathlineto{\pgfqpoint{1.037231in}{2.698404in}}%
\pgfpathlineto{\pgfqpoint{1.114693in}{2.795414in}}%
\pgfpathlineto{\pgfqpoint{1.201406in}{2.884258in}}%
\pgfpathlineto{\pgfqpoint{1.296597in}{2.964124in}}%
\pgfpathlineto{\pgfqpoint{1.399313in}{3.034133in}}%
\pgfpathlineto{\pgfqpoint{1.508543in}{3.093539in}}%
\pgfpathlineto{\pgfqpoint{1.623218in}{3.141726in}}%
\pgfpathlineto{\pgfqpoint{1.742214in}{3.178216in}}%
\pgfpathlineto{\pgfqpoint{1.864347in}{3.202660in}}%
\pgfpathlineto{\pgfqpoint{1.988379in}{3.214845in}}%
\pgfpathlineto{\pgfqpoint{2.113014in}{3.214691in}}%
\pgfpathlineto{\pgfqpoint{2.236938in}{3.202249in}}%
\pgfpathlineto{\pgfqpoint{2.359049in}{3.177662in}}%
\pgfpathlineto{\pgfqpoint{2.478084in}{3.141126in}}%
\pgfpathlineto{\pgfqpoint{2.592802in}{3.092956in}}%
\pgfpathlineto{\pgfqpoint{2.702038in}{3.033591in}}%
\pgfpathlineto{\pgfqpoint{2.804707in}{2.963593in}}%
\pgfpathlineto{\pgfqpoint{2.899800in}{2.883651in}}%
\pgfpathlineto{\pgfqpoint{2.986384in}{2.794578in}}%
\pgfpathlineto{\pgfqpoint{3.063607in}{2.697310in}}%
\pgfpathlineto{\pgfqpoint{3.130691in}{2.592908in}}%
\pgfpathlineto{\pgfqpoint{3.186971in}{2.482503in}}%
\pgfpathlineto{\pgfqpoint{3.231971in}{2.367065in}}%
\pgfpathlineto{\pgfqpoint{3.265174in}{2.247762in}}%
\pgfpathlineto{\pgfqpoint{3.286180in}{2.125798in}}%
\pgfpathlineto{\pgfqpoint{3.294741in}{2.002382in}}%
\pgfpathlineto{\pgfqpoint{3.290751in}{1.878726in}}%
\pgfpathlineto{\pgfqpoint{3.274254in}{1.756046in}}%
\pgfpathlineto{\pgfqpoint{3.245440in}{1.635564in}}%
\pgfpathlineto{\pgfqpoint{3.204644in}{1.518505in}}%
\pgfpathlineto{\pgfqpoint{3.152352in}{1.406097in}}%
\pgfpathlineto{\pgfqpoint{3.089134in}{1.299474in}}%
\pgfpathlineto{\pgfqpoint{3.015545in}{1.199596in}}%
\pgfpathlineto{\pgfqpoint{2.932305in}{1.107534in}}%
\pgfpathlineto{\pgfqpoint{2.840226in}{1.024247in}}%
\pgfpathlineto{\pgfqpoint{2.740202in}{0.950582in}}%
\pgfpathlineto{\pgfqpoint{2.633215in}{0.887269in}}%
\pgfpathlineto{\pgfqpoint{2.520333in}{0.834922in}}%
\pgfpathlineto{\pgfqpoint{2.402710in}{0.794041in}}%
\pgfpathlineto{\pgfqpoint{2.281583in}{0.765007in}}%
\pgfpathlineto{\pgfqpoint{2.158208in}{0.748068in}}%
\pgfpathlineto{\pgfqpoint{2.033703in}{0.743358in}}%
\pgfpathlineto{\pgfqpoint{1.909365in}{0.750992in}}%
\pgfpathlineto{\pgfqpoint{1.786467in}{0.770954in}}%
\pgfpathlineto{\pgfqpoint{1.666239in}{0.803087in}}%
\pgfpathlineto{\pgfqpoint{1.549866in}{0.847097in}}%
\pgfpathlineto{\pgfqpoint{1.438493in}{0.902546in}}%
\pgfpathlineto{\pgfqpoint{1.333219in}{0.968858in}}%
\pgfpathlineto{\pgfqpoint{1.235100in}{1.045316in}}%
\pgfpathlineto{\pgfqpoint{1.145151in}{1.131065in}}%
\pgfpathlineto{\pgfqpoint{1.064276in}{1.225176in}}%
\pgfpathlineto{\pgfqpoint{0.993178in}{1.326827in}}%
\pgfpathlineto{\pgfqpoint{0.932644in}{1.434997in}}%
\pgfpathlineto{\pgfqpoint{0.883337in}{1.548610in}}%
\pgfpathlineto{\pgfqpoint{0.845780in}{1.666552in}}%
\pgfpathlineto{\pgfqpoint{0.820356in}{1.787667in}}%
\pgfpathlineto{\pgfqpoint{0.807306in}{1.910756in}}%
\pgfpathlineto{\pgfqpoint{0.806731in}{2.034585in}}%
\pgfpathlineto{\pgfqpoint{0.818595in}{2.157874in}}%
\pgfpathlineto{\pgfqpoint{0.842717in}{2.279308in}}%
\pgfpathlineto{\pgfqpoint{0.878832in}{2.397762in}}%
\pgfpathlineto{\pgfqpoint{0.926621in}{2.512079in}}%
\pgfpathlineto{\pgfqpoint{0.985648in}{2.621048in}}%
\pgfpathlineto{\pgfqpoint{1.055363in}{2.723550in}}%
\pgfpathlineto{\pgfqpoint{1.135101in}{2.818555in}}%
\pgfpathlineto{\pgfqpoint{1.224082in}{2.905126in}}%
\pgfpathlineto{\pgfqpoint{1.321415in}{2.982418in}}%
\pgfpathlineto{\pgfqpoint{1.426091in}{3.049677in}}%
\pgfpathlineto{\pgfqpoint{1.536989in}{3.106240in}}%
\pgfpathlineto{\pgfqpoint{1.652899in}{3.151549in}}%
\pgfpathlineto{\pgfqpoint{1.772780in}{3.185231in}}%
\pgfpathlineto{\pgfqpoint{1.895437in}{3.206897in}}%
\pgfpathlineto{\pgfqpoint{2.019617in}{3.216262in}}%
\pgfpathlineto{\pgfqpoint{2.144081in}{3.213184in}}%
\pgfpathlineto{\pgfqpoint{2.267604in}{3.197669in}}%
\pgfpathlineto{\pgfqpoint{2.388971in}{3.169868in}}%
\pgfpathlineto{\pgfqpoint{2.506981in}{3.130077in}}%
\pgfpathlineto{\pgfqpoint{2.620446in}{3.078740in}}%
\pgfpathlineto{\pgfqpoint{2.728189in}{3.016445in}}%
\pgfpathlineto{\pgfqpoint{2.829089in}{2.943895in}}%
\pgfpathlineto{\pgfqpoint{2.922261in}{2.861735in}}%
\pgfpathlineto{\pgfqpoint{3.006720in}{2.770766in}}%
\pgfpathlineto{\pgfqpoint{3.081565in}{2.671884in}}%
\pgfpathlineto{\pgfqpoint{3.146023in}{2.566052in}}%
\pgfpathlineto{\pgfqpoint{3.199447in}{2.454306in}}%
\pgfpathlineto{\pgfqpoint{3.241317in}{2.337753in}}%
\pgfpathlineto{\pgfqpoint{3.271241in}{2.217570in}}%
\pgfpathlineto{\pgfqpoint{3.288951in}{2.095002in}}%
\pgfpathlineto{\pgfqpoint{3.294310in}{1.971360in}}%
\pgfpathlineto{\pgfqpoint{3.287315in}{1.847780in}}%
\pgfpathlineto{\pgfqpoint{3.267996in}{1.725492in}}%
\pgfpathlineto{\pgfqpoint{3.236485in}{1.605773in}}%
\pgfpathlineto{\pgfqpoint{3.193048in}{1.489840in}}%
\pgfpathlineto{\pgfqpoint{3.138085in}{1.378846in}}%
\pgfpathlineto{\pgfqpoint{3.072130in}{1.273884in}}%
\pgfpathlineto{\pgfqpoint{2.995849in}{1.175984in}}%
\pgfpathlineto{\pgfqpoint{2.910043in}{1.086115in}}%
\pgfpathlineto{\pgfqpoint{2.815648in}{1.005183in}}%
\pgfpathlineto{\pgfqpoint{2.713721in}{0.934027in}}%
\pgfpathlineto{\pgfqpoint{2.605198in}{0.873259in}}%
\pgfpathlineto{\pgfqpoint{2.491126in}{0.823513in}}%
\pgfpathlineto{\pgfqpoint{2.372656in}{0.785359in}}%
\pgfpathlineto{\pgfqpoint{2.250964in}{0.759224in}}%
\pgfpathlineto{\pgfqpoint{2.127245in}{0.745391in}}%
\pgfpathlineto{\pgfqpoint{2.002714in}{0.743996in}}%
\pgfpathlineto{\pgfqpoint{1.878608in}{0.755036in}}%
\pgfpathlineto{\pgfqpoint{1.756185in}{0.778358in}}%
\pgfpathlineto{\pgfqpoint{1.636723in}{0.813670in}}%
\pgfpathlineto{\pgfqpoint{1.521467in}{0.860554in}}%
\pgfpathlineto{\pgfqpoint{1.411434in}{0.918586in}}%
\pgfpathlineto{\pgfqpoint{1.307782in}{0.987217in}}%
\pgfpathlineto{\pgfqpoint{1.211597in}{1.065790in}}%
\pgfpathlineto{\pgfqpoint{1.123862in}{1.153551in}}%
\pgfpathlineto{\pgfqpoint{1.045450in}{1.249645in}}%
\pgfpathlineto{\pgfqpoint{0.977127in}{1.353120in}}%
\pgfpathlineto{\pgfqpoint{0.919556in}{1.462923in}}%
\pgfpathlineto{\pgfqpoint{0.873290in}{1.577906in}}%
\pgfpathlineto{\pgfqpoint{0.838774in}{1.696822in}}%
\pgfpathlineto{\pgfqpoint{0.816285in}{1.818555in}}%
\pgfpathlineto{\pgfqpoint{0.806074in}{1.941921in}}%
\pgfpathlineto{\pgfqpoint{0.808316in}{2.065645in}}%
\pgfpathlineto{\pgfqpoint{0.823042in}{2.188481in}}%
\pgfpathlineto{\pgfqpoint{0.850137in}{2.309212in}}%
\pgfpathlineto{\pgfqpoint{0.889344in}{2.426652in}}%
\pgfpathlineto{\pgfqpoint{0.940260in}{2.539643in}}%
\pgfpathlineto{\pgfqpoint{1.002339in}{2.647059in}}%
\pgfpathlineto{\pgfqpoint{1.074890in}{2.747801in}}%
\pgfpathlineto{\pgfqpoint{1.157082in}{2.840809in}}%
\pgfpathlineto{\pgfqpoint{1.248144in}{2.925251in}}%
\pgfpathlineto{\pgfqpoint{1.347216in}{3.000279in}}%
\pgfpathlineto{\pgfqpoint{1.453310in}{3.065070in}}%
\pgfpathlineto{\pgfqpoint{1.565383in}{3.118940in}}%
\pgfpathlineto{\pgfqpoint{1.682339in}{3.161334in}}%
\pgfpathlineto{\pgfqpoint{1.803027in}{3.191837in}}%
\pgfpathlineto{\pgfqpoint{1.926243in}{3.210165in}}%
\pgfpathlineto{\pgfqpoint{2.050726in}{3.216169in}}%
\pgfpathlineto{\pgfqpoint{2.175164in}{3.209834in}}%
\pgfpathlineto{\pgfqpoint{2.298294in}{3.191276in}}%
\pgfpathlineto{\pgfqpoint{2.419006in}{3.160669in}}%
\pgfpathlineto{\pgfqpoint{2.536024in}{3.118267in}}%
\pgfpathlineto{\pgfqpoint{2.648137in}{3.064447in}}%
\pgfpathlineto{\pgfqpoint{2.754214in}{2.999708in}}%
\pgfpathlineto{\pgfqpoint{2.853205in}{2.924672in}}%
\pgfpathlineto{\pgfqpoint{2.944137in}{2.840083in}}%
\pgfpathlineto{\pgfqpoint{3.026122in}{2.746808in}}%
\pgfpathlineto{\pgfqpoint{3.098347in}{2.645839in}}%
\pgfpathlineto{\pgfqpoint{3.160083in}{2.538287in}}%
\pgfpathlineto{\pgfqpoint{3.210753in}{2.425248in}}%
\pgfpathlineto{\pgfqpoint{3.249897in}{2.307737in}}%
\pgfpathlineto{\pgfqpoint{3.277041in}{2.186960in}}%
\pgfpathlineto{\pgfqpoint{3.291855in}{2.064130in}}%
\pgfpathlineto{\pgfqpoint{3.294158in}{1.940459in}}%
\pgfpathlineto{\pgfqpoint{3.283915in}{1.817163in}}%
\pgfpathlineto{\pgfqpoint{3.261237in}{1.695457in}}%
\pgfpathlineto{\pgfqpoint{3.226383in}{1.576559in}}%
\pgfpathlineto{\pgfqpoint{3.179757in}{1.461686in}}%
\pgfpathlineto{\pgfqpoint{3.121913in}{1.352057in}}%
\pgfpathlineto{\pgfqpoint{3.053424in}{1.248702in}}%
\pgfpathlineto{\pgfqpoint{2.974916in}{1.152617in}}%
\pgfpathlineto{\pgfqpoint{2.887158in}{1.064832in}}%
\pgfpathlineto{\pgfqpoint{2.791005in}{0.986257in}}%
\pgfpathlineto{\pgfqpoint{2.687395in}{0.917686in}}%
\pgfpathlineto{\pgfqpoint{2.577347in}{0.859793in}}%
\pgfpathlineto{\pgfqpoint{2.461967in}{0.813136in}}%
\pgfpathlineto{\pgfqpoint{2.342441in}{0.778156in}}%
\pgfpathlineto{\pgfqpoint{2.220039in}{0.755174in}}%
\pgfpathlineto{\pgfqpoint{2.095974in}{0.744366in}}%
\pgfpathlineto{\pgfqpoint{1.971403in}{0.745837in}}%
\pgfpathlineto{\pgfqpoint{1.847634in}{0.759645in}}%
\pgfpathlineto{\pgfqpoint{1.725925in}{0.785707in}}%
\pgfpathlineto{\pgfqpoint{1.607491in}{0.823801in}}%
\pgfpathlineto{\pgfqpoint{1.493497in}{0.873566in}}%
\pgfpathlineto{\pgfqpoint{1.385064in}{0.934503in}}%
\pgfpathlineto{\pgfqpoint{1.283266in}{1.005974in}}%
\pgfpathlineto{\pgfqpoint{1.189128in}{1.087200in}}%
\pgfpathlineto{\pgfqpoint{1.103632in}{1.177266in}}%
\pgfpathlineto{\pgfqpoint{1.027586in}{1.275259in}}%
\pgfpathlineto{\pgfqpoint{0.961696in}{1.380298in}}%
\pgfpathlineto{\pgfqpoint{0.906702in}{1.491321in}}%
\pgfpathlineto{\pgfqpoint{0.863204in}{1.607227in}}%
\pgfpathlineto{\pgfqpoint{0.831663in}{1.726880in}}%
\pgfpathlineto{\pgfqpoint{0.812395in}{1.849104in}}%
\pgfpathlineto{\pgfqpoint{0.805578in}{1.972690in}}%
\pgfpathlineto{\pgfqpoint{0.811246in}{2.096388in}}%
\pgfpathlineto{\pgfqpoint{0.829293in}{2.218913in}}%
\pgfpathlineto{\pgfqpoint{0.859476in}{2.338965in}}%
\pgfpathlineto{\pgfqpoint{0.901496in}{2.455476in}}%
\pgfpathlineto{\pgfqpoint{0.954974in}{2.567252in}}%
\pgfpathlineto{\pgfqpoint{1.019415in}{2.673119in}}%
\pgfpathlineto{\pgfqpoint{1.094212in}{2.771993in}}%
\pgfpathlineto{\pgfqpoint{1.178648in}{2.862886in}}%
\pgfpathlineto{\pgfqpoint{1.271891in}{2.944906in}}%
\pgfpathlineto{\pgfqpoint{1.372999in}{3.017251in}}%
\pgfpathlineto{\pgfqpoint{1.480918in}{3.079218in}}%
\pgfpathlineto{\pgfqpoint{1.594484in}{3.130195in}}%
\pgfpathlineto{\pgfqpoint{1.712496in}{3.169700in}}%
\pgfpathlineto{\pgfqpoint{1.833911in}{3.197398in}}%
\pgfpathlineto{\pgfqpoint{1.957478in}{3.212936in}}%
\pgfpathlineto{\pgfqpoint{2.081947in}{3.216098in}}%
\pgfpathlineto{\pgfqpoint{2.206078in}{3.206810in}}%
\pgfpathlineto{\pgfqpoint{2.328652in}{3.185145in}}%
\pgfpathlineto{\pgfqpoint{2.448462in}{3.151320in}}%
\pgfpathlineto{\pgfqpoint{2.564319in}{3.105700in}}%
\pgfpathlineto{\pgfqpoint{2.675049in}{3.048794in}}%
\pgfpathlineto{\pgfqpoint{2.779495in}{2.981254in}}%
\pgfpathlineto{\pgfqpoint{2.876610in}{2.903808in}}%
\pgfpathlineto{\pgfqpoint{2.965527in}{2.817136in}}%
\pgfpathlineto{\pgfqpoint{3.045280in}{2.722100in}}%
\pgfpathlineto{\pgfqpoint{3.115025in}{2.619631in}}%
\pgfpathlineto{\pgfqpoint{3.174043in}{2.510732in}}%
\pgfpathlineto{\pgfqpoint{3.221748in}{2.396468in}}%
\pgfpathlineto{\pgfqpoint{3.257680in}{2.277976in}}%
\pgfpathlineto{\pgfqpoint{3.281507in}{2.156458in}}%
\pgfpathlineto{\pgfqpoint{3.293027in}{2.033185in}}%
\pgfpathlineto{\pgfqpoint{3.292170in}{1.909459in}}%
\pgfpathlineto{\pgfqpoint{3.278979in}{1.786385in}}%
\pgfpathlineto{\pgfqpoint{3.253531in}{1.665234in}}%
\pgfpathlineto{\pgfqpoint{3.216024in}{1.547262in}}%
\pgfpathlineto{\pgfqpoint{3.166788in}{1.433663in}}%
\pgfpathlineto{\pgfqpoint{3.106284in}{1.325565in}}%
\pgfpathlineto{\pgfqpoint{3.035107in}{1.224029in}}%
\pgfpathlineto{\pgfqpoint{2.953980in}{1.130052in}}%
\pgfpathlineto{\pgfqpoint{2.863761in}{1.044568in}}%
\pgfpathlineto{\pgfqpoint{2.765441in}{0.968441in}}%
\pgfpathlineto{\pgfqpoint{2.660099in}{0.902444in}}%
\pgfpathlineto{\pgfqpoint{2.548666in}{0.847136in}}%
\pgfpathlineto{\pgfqpoint{2.432253in}{0.803130in}}%
\pgfpathlineto{\pgfqpoint{2.312033in}{0.770933in}}%
\pgfpathlineto{\pgfqpoint{2.189194in}{0.750906in}}%
\pgfpathlineto{\pgfqpoint{2.064942in}{0.743267in}}%
\pgfpathlineto{\pgfqpoint{1.940500in}{0.748088in}}%
\pgfpathlineto{\pgfqpoint{1.817107in}{0.765296in}}%
\pgfpathlineto{\pgfqpoint{1.696021in}{0.794673in}}%
\pgfpathlineto{\pgfqpoint{1.578516in}{0.835859in}}%
\pgfpathlineto{\pgfqpoint{1.465775in}{0.888394in}}%
\pgfpathlineto{\pgfqpoint{1.358815in}{0.951812in}}%
\pgfpathlineto{\pgfqpoint{1.258776in}{1.025504in}}%
\pgfpathlineto{\pgfqpoint{1.166703in}{1.108761in}}%
\pgfpathlineto{\pgfqpoint{1.083529in}{1.200779in}}%
\pgfpathlineto{\pgfqpoint{1.010078in}{1.300658in}}%
\pgfpathlineto{\pgfqpoint{0.947066in}{1.407402in}}%
\pgfpathlineto{\pgfqpoint{0.895101in}{1.519921in}}%
\pgfpathlineto{\pgfqpoint{0.854681in}{1.637026in}}%
\pgfpathlineto{\pgfqpoint{0.826181in}{1.757469in}}%
\pgfpathlineto{\pgfqpoint{0.809824in}{1.880162in}}%
\pgfpathlineto{\pgfqpoint{0.805833in}{2.003856in}}%
\pgfpathlineto{\pgfqpoint{0.814315in}{2.127284in}}%
\pgfpathlineto{\pgfqpoint{0.835232in}{2.249206in}}%
\pgfpathlineto{\pgfqpoint{0.868402in}{2.368420in}}%
\pgfpathlineto{\pgfqpoint{0.913502in}{2.483755in}}%
\pgfpathlineto{\pgfqpoint{0.970062in}{2.594070in}}%
\pgfpathlineto{\pgfqpoint{1.037471in}{2.698262in}}%
\pgfpathlineto{\pgfqpoint{1.114973in}{2.795256in}}%
\pgfpathlineto{\pgfqpoint{1.201700in}{2.884047in}}%
\pgfpathlineto{\pgfqpoint{1.296884in}{2.963868in}}%
\pgfpathlineto{\pgfqpoint{1.399590in}{3.033864in}}%
\pgfpathlineto{\pgfqpoint{1.508796in}{3.093275in}}%
\pgfpathlineto{\pgfqpoint{1.623431in}{3.141474in}}%
\pgfpathlineto{\pgfqpoint{1.742375in}{3.177973in}}%
\pgfpathlineto{\pgfqpoint{1.864451in}{3.202415in}}%
\pgfpathlineto{\pgfqpoint{1.988437in}{3.214584in}}%
\pgfpathlineto{\pgfqpoint{2.113054in}{3.214395in}}%
\pgfpathlineto{\pgfqpoint{2.236977in}{3.201901in}}%
\pgfpathlineto{\pgfqpoint{2.359019in}{3.177270in}}%
\pgfpathlineto{\pgfqpoint{2.478025in}{3.140713in}}%
\pgfpathlineto{\pgfqpoint{2.592731in}{3.092549in}}%
\pgfpathlineto{\pgfqpoint{2.701959in}{3.033214in}}%
\pgfpathlineto{\pgfqpoint{2.804612in}{2.963265in}}%
\pgfpathlineto{\pgfqpoint{2.899677in}{2.883382in}}%
\pgfpathlineto{\pgfqpoint{2.986223in}{2.794363in}}%
\pgfpathlineto{\pgfqpoint{3.063402in}{2.697129in}}%
\pgfpathlineto{\pgfqpoint{3.130450in}{2.592720in}}%
\pgfpathlineto{\pgfqpoint{3.186684in}{2.482295in}}%
\pgfpathlineto{\pgfqpoint{3.231626in}{2.366886in}}%
\pgfpathlineto{\pgfqpoint{3.264808in}{2.247598in}}%
\pgfpathlineto{\pgfqpoint{3.285822in}{2.125650in}}%
\pgfpathlineto{\pgfqpoint{3.294407in}{2.002261in}}%
\pgfpathlineto{\pgfqpoint{3.290450in}{1.878645in}}%
\pgfpathlineto{\pgfqpoint{3.273982in}{1.756019in}}%
\pgfpathlineto{\pgfqpoint{3.245184in}{1.635594in}}%
\pgfpathlineto{\pgfqpoint{3.204383in}{1.518579in}}%
\pgfpathlineto{\pgfqpoint{3.152051in}{1.406185in}}%
\pgfpathlineto{\pgfqpoint{3.088798in}{1.299599in}}%
\pgfpathlineto{\pgfqpoint{3.015206in}{1.199771in}}%
\pgfpathlineto{\pgfqpoint{2.931974in}{1.107724in}}%
\pgfpathlineto{\pgfqpoint{2.839917in}{1.024438in}}%
\pgfpathlineto{\pgfqpoint{2.739932in}{0.950771in}}%
\pgfpathlineto{\pgfqpoint{2.632996in}{0.887462in}}%
\pgfpathlineto{\pgfqpoint{2.520166in}{0.835132in}}%
\pgfpathlineto{\pgfqpoint{2.402580in}{0.794280in}}%
\pgfpathlineto{\pgfqpoint{2.281456in}{0.765286in}}%
\pgfpathlineto{\pgfqpoint{2.158089in}{0.748411in}}%
\pgfpathlineto{\pgfqpoint{2.033634in}{0.743761in}}%
\pgfpathlineto{\pgfqpoint{1.909313in}{0.751416in}}%
\pgfpathlineto{\pgfqpoint{1.786423in}{0.771365in}}%
\pgfpathlineto{\pgfqpoint{1.666208in}{0.803460in}}%
\pgfpathlineto{\pgfqpoint{1.549864in}{0.847416in}}%
\pgfpathlineto{\pgfqpoint{1.438534in}{0.902808in}}%
\pgfpathlineto{\pgfqpoint{1.333315in}{0.969077in}}%
\pgfpathlineto{\pgfqpoint{1.235249in}{1.045521in}}%
\pgfpathlineto{\pgfqpoint{1.145330in}{1.131306in}}%
\pgfpathlineto{\pgfqpoint{1.064497in}{1.225461in}}%
\pgfpathlineto{\pgfqpoint{0.993467in}{1.327100in}}%
\pgfpathlineto{\pgfqpoint{0.932953in}{1.435264in}}%
\pgfpathlineto{\pgfqpoint{0.883637in}{1.548862in}}%
\pgfpathlineto{\pgfqpoint{0.846057in}{1.666770in}}%
\pgfpathlineto{\pgfqpoint{0.820607in}{1.787832in}}%
\pgfpathlineto{\pgfqpoint{0.807542in}{1.910859in}}%
\pgfpathlineto{\pgfqpoint{0.806972in}{2.034628in}}%
\pgfpathlineto{\pgfqpoint{0.818865in}{2.157882in}}%
\pgfpathlineto{\pgfqpoint{0.843048in}{2.279333in}}%
\pgfpathlineto{\pgfqpoint{0.879218in}{2.397715in}}%
\pgfpathlineto{\pgfqpoint{0.927040in}{2.511983in}}%
\pgfpathlineto{\pgfqpoint{0.986074in}{2.620936in}}%
\pgfpathlineto{\pgfqpoint{1.055768in}{2.723434in}}%
\pgfpathlineto{\pgfqpoint{1.135459in}{2.818434in}}%
\pgfpathlineto{\pgfqpoint{1.224377in}{2.904990in}}%
\pgfpathlineto{\pgfqpoint{1.321643in}{2.982254in}}%
\pgfpathlineto{\pgfqpoint{1.426267in}{3.049473in}}%
\pgfpathlineto{\pgfqpoint{1.537152in}{3.105994in}}%
\pgfpathlineto{\pgfqpoint{1.653092in}{3.151260in}}%
\pgfpathlineto{\pgfqpoint{1.772931in}{3.184870in}}%
\pgfpathlineto{\pgfqpoint{1.895570in}{3.206508in}}%
\pgfpathlineto{\pgfqpoint{2.019738in}{3.215877in}}%
\pgfpathlineto{\pgfqpoint{2.144183in}{3.212828in}}%
\pgfpathlineto{\pgfqpoint{2.267671in}{3.197354in}}%
\pgfpathlineto{\pgfqpoint{2.388989in}{3.169595in}}%
\pgfpathlineto{\pgfqpoint{2.506941in}{3.129836in}}%
\pgfpathlineto{\pgfqpoint{2.620352in}{3.078505in}}%
\pgfpathlineto{\pgfqpoint{2.728068in}{3.016178in}}%
\pgfpathlineto{\pgfqpoint{2.828950in}{2.943572in}}%
\pgfpathlineto{\pgfqpoint{2.922052in}{2.861412in}}%
\pgfpathlineto{\pgfqpoint{3.006486in}{2.770447in}}%
\pgfpathlineto{\pgfqpoint{3.081333in}{2.671580in}}%
\pgfpathlineto{\pgfqpoint{3.145802in}{2.565783in}}%
\pgfpathlineto{\pgfqpoint{3.199235in}{2.454089in}}%
\pgfpathlineto{\pgfqpoint{3.241103in}{2.337595in}}%
\pgfpathlineto{\pgfqpoint{3.271008in}{2.217464in}}%
\pgfpathlineto{\pgfqpoint{3.288682in}{2.094920in}}%
\pgfpathlineto{\pgfqpoint{3.293986in}{1.971255in}}%
\pgfpathlineto{\pgfqpoint{3.286918in}{1.847743in}}%
\pgfpathlineto{\pgfqpoint{3.267559in}{1.725489in}}%
\pgfpathlineto{\pgfqpoint{3.236045in}{1.605780in}}%
\pgfpathlineto{\pgfqpoint{3.192636in}{1.489850in}}%
\pgfpathlineto{\pgfqpoint{3.137726in}{1.378870in}}%
\pgfpathlineto{\pgfqpoint{3.071836in}{1.273937in}}%
\pgfpathlineto{\pgfqpoint{2.995617in}{1.176082in}}%
\pgfpathlineto{\pgfqpoint{2.909853in}{1.086265in}}%
\pgfpathlineto{\pgfqpoint{2.815454in}{1.005379in}}%
\pgfpathlineto{\pgfqpoint{2.713462in}{0.934247in}}%
\pgfpathlineto{\pgfqpoint{2.604958in}{0.873565in}}%
\pgfpathlineto{\pgfqpoint{2.490895in}{0.823859in}}%
\pgfpathlineto{\pgfqpoint{2.372435in}{0.785706in}}%
\pgfpathlineto{\pgfqpoint{2.250765in}{0.759546in}}%
\pgfpathlineto{\pgfqpoint{2.127087in}{0.745677in}}%
\pgfpathlineto{\pgfqpoint{2.002614in}{0.744250in}}%
\pgfpathlineto{\pgfqpoint{1.878574in}{0.755271in}}%
\pgfpathlineto{\pgfqpoint{1.756206in}{0.778601in}}%
\pgfpathlineto{\pgfqpoint{1.636764in}{0.813956in}}%
\pgfpathlineto{\pgfqpoint{1.521512in}{0.860907in}}%
\pgfpathlineto{\pgfqpoint{1.411562in}{0.918966in}}%
\pgfpathlineto{\pgfqpoint{1.307946in}{0.987612in}}%
\pgfpathlineto{\pgfqpoint{1.211773in}{1.066181in}}%
\pgfpathlineto{\pgfqpoint{1.124041in}{1.153912in}}%
\pgfpathlineto{\pgfqpoint{1.045634in}{1.249954in}}%
\pgfpathlineto{\pgfqpoint{0.977327in}{1.353364in}}%
\pgfpathlineto{\pgfqpoint{0.919782in}{1.463106in}}%
\pgfpathlineto{\pgfqpoint{0.873550in}{1.578051in}}%
\pgfpathlineto{\pgfqpoint{0.839072in}{1.696978in}}%
\pgfpathlineto{\pgfqpoint{0.816649in}{1.818664in}}%
\pgfpathlineto{\pgfqpoint{0.806471in}{1.942005in}}%
\pgfpathlineto{\pgfqpoint{0.808715in}{2.065715in}}%
\pgfpathlineto{\pgfqpoint{0.823417in}{2.188535in}}%
\pgfpathlineto{\pgfqpoint{0.850473in}{2.309240in}}%
\pgfpathlineto{\pgfqpoint{0.889634in}{2.426639in}}%
\pgfpathlineto{\pgfqpoint{0.940511in}{2.539580in}}%
\pgfpathlineto{\pgfqpoint{1.002569in}{2.646943in}}%
\pgfpathlineto{\pgfqpoint{1.075131in}{2.747648in}}%
\pgfpathlineto{\pgfqpoint{1.157378in}{2.840646in}}%
\pgfpathlineto{\pgfqpoint{1.248435in}{2.925016in}}%
\pgfpathlineto{\pgfqpoint{1.347501in}{3.000009in}}%
\pgfpathlineto{\pgfqpoint{1.453581in}{3.064795in}}%
\pgfpathlineto{\pgfqpoint{1.565626in}{3.118675in}}%
\pgfpathlineto{\pgfqpoint{1.682537in}{3.161085in}}%
\pgfpathlineto{\pgfqpoint{1.803170in}{3.191596in}}%
\pgfpathlineto{\pgfqpoint{1.926328in}{3.209920in}}%
\pgfpathlineto{\pgfqpoint{2.050770in}{3.215902in}}%
\pgfpathlineto{\pgfqpoint{2.175203in}{3.209527in}}%
\pgfpathlineto{\pgfqpoint{2.298303in}{3.190912in}}%
\pgfpathlineto{\pgfqpoint{2.418958in}{3.160265in}}%
\pgfpathlineto{\pgfqpoint{2.535954in}{3.117850in}}%
\pgfpathlineto{\pgfqpoint{2.648058in}{3.064042in}}%
\pgfpathlineto{\pgfqpoint{2.754127in}{2.999340in}}%
\pgfpathlineto{\pgfqpoint{2.853100in}{2.924358in}}%
\pgfpathlineto{\pgfqpoint{2.944003in}{2.839830in}}%
\pgfpathlineto{\pgfqpoint{3.025947in}{2.746609in}}%
\pgfpathlineto{\pgfqpoint{3.098129in}{2.645665in}}%
\pgfpathlineto{\pgfqpoint{3.159831in}{2.538089in}}%
\pgfpathlineto{\pgfqpoint{3.210440in}{2.425056in}}%
\pgfpathlineto{\pgfqpoint{3.249536in}{2.307568in}}%
\pgfpathlineto{\pgfqpoint{3.276668in}{2.186804in}}%
\pgfpathlineto{\pgfqpoint{3.291498in}{2.063991in}}%
\pgfpathlineto{\pgfqpoint{3.293832in}{1.940351in}}%
\pgfpathlineto{\pgfqpoint{3.283624in}{1.817101in}}%
\pgfpathlineto{\pgfqpoint{3.260975in}{1.695451in}}%
\pgfpathlineto{\pgfqpoint{3.226133in}{1.576608in}}%
\pgfpathlineto{\pgfqpoint{3.179492in}{1.461774in}}%
\pgfpathlineto{\pgfqpoint{3.121595in}{1.352145in}}%
\pgfpathlineto{\pgfqpoint{3.053089in}{1.248852in}}%
\pgfpathlineto{\pgfqpoint{2.974578in}{1.152807in}}%
\pgfpathlineto{\pgfqpoint{2.886832in}{1.065031in}}%
\pgfpathlineto{\pgfqpoint{2.790707in}{0.986453in}}%
\pgfpathlineto{\pgfqpoint{2.687140in}{0.917878in}}%
\pgfpathlineto{\pgfqpoint{2.577147in}{0.859990in}}%
\pgfpathlineto{\pgfqpoint{2.461818in}{0.813352in}}%
\pgfpathlineto{\pgfqpoint{2.342323in}{0.778405in}}%
\pgfpathlineto{\pgfqpoint{2.219910in}{0.755467in}}%
\pgfpathlineto{\pgfqpoint{2.095879in}{0.744731in}}%
\pgfpathlineto{\pgfqpoint{1.971349in}{0.746254in}}%
\pgfpathlineto{\pgfqpoint{1.847592in}{0.760074in}}%
\pgfpathlineto{\pgfqpoint{1.725890in}{0.786116in}}%
\pgfpathlineto{\pgfqpoint{1.607471in}{0.824164in}}%
\pgfpathlineto{\pgfqpoint{1.493508in}{0.873872in}}%
\pgfpathlineto{\pgfqpoint{1.385122in}{0.934751in}}%
\pgfpathlineto{\pgfqpoint{1.283379in}{1.006182in}}%
\pgfpathlineto{\pgfqpoint{1.189292in}{1.087404in}}%
\pgfpathlineto{\pgfqpoint{1.103817in}{1.177523in}}%
\pgfpathlineto{\pgfqpoint{1.027838in}{1.275535in}}%
\pgfpathlineto{\pgfqpoint{0.962003in}{1.380565in}}%
\pgfpathlineto{\pgfqpoint{0.907020in}{1.491581in}}%
\pgfpathlineto{\pgfqpoint{0.863506in}{1.607468in}}%
\pgfpathlineto{\pgfqpoint{0.831937in}{1.727082in}}%
\pgfpathlineto{\pgfqpoint{0.812644in}{1.849251in}}%
\pgfpathlineto{\pgfqpoint{0.805812in}{1.972771in}}%
\pgfpathlineto{\pgfqpoint{0.811489in}{2.096412in}}%
\pgfpathlineto{\pgfqpoint{0.829573in}{2.218912in}}%
\pgfpathlineto{\pgfqpoint{0.859824in}{2.338979in}}%
\pgfpathlineto{\pgfqpoint{0.901890in}{2.455404in}}%
\pgfpathlineto{\pgfqpoint{0.955395in}{2.567140in}}%
\pgfpathlineto{\pgfqpoint{1.019837in}{2.672995in}}%
\pgfpathlineto{\pgfqpoint{1.094608in}{2.771866in}}%
\pgfpathlineto{\pgfqpoint{1.178992in}{2.862754in}}%
\pgfpathlineto{\pgfqpoint{1.272169in}{2.944757in}}%
\pgfpathlineto{\pgfqpoint{1.373210in}{3.017072in}}%
\pgfpathlineto{\pgfqpoint{1.481083in}{3.078999in}}%
\pgfpathlineto{\pgfqpoint{1.594648in}{3.129935in}}%
\pgfpathlineto{\pgfqpoint{1.712667in}{3.169383in}}%
\pgfpathlineto{\pgfqpoint{1.834046in}{3.197021in}}%
\pgfpathlineto{\pgfqpoint{1.957599in}{3.212541in}}%
\pgfpathlineto{\pgfqpoint{2.082055in}{3.215716in}}%
\pgfpathlineto{\pgfqpoint{2.206164in}{3.206462in}}%
\pgfpathlineto{\pgfqpoint{2.328700in}{3.184840in}}%
\pgfpathlineto{\pgfqpoint{2.448457in}{3.151058in}}%
\pgfpathlineto{\pgfqpoint{2.564256in}{3.105465in}}%
\pgfpathlineto{\pgfqpoint{2.674938in}{3.048557in}}%
\pgfpathlineto{\pgfqpoint{2.779366in}{2.980974in}}%
\pgfpathlineto{\pgfqpoint{2.876439in}{2.903490in}}%
\pgfpathlineto{\pgfqpoint{2.965299in}{2.816819in}}%
\pgfpathlineto{\pgfqpoint{3.045036in}{2.721788in}}%
\pgfpathlineto{\pgfqpoint{3.114787in}{2.619341in}}%
\pgfpathlineto{\pgfqpoint{3.173818in}{2.510480in}}%
\pgfpathlineto{\pgfqpoint{3.221532in}{2.396271in}}%
\pgfpathlineto{\pgfqpoint{3.257460in}{2.277839in}}%
\pgfpathlineto{\pgfqpoint{3.281265in}{2.156370in}}%
\pgfpathlineto{\pgfqpoint{3.292745in}{2.033111in}}%
\pgfpathlineto{\pgfqpoint{3.291826in}{1.909372in}}%
\pgfpathlineto{\pgfqpoint{3.278570in}{1.786371in}}%
\pgfpathlineto{\pgfqpoint{3.253091in}{1.665247in}}%
\pgfpathlineto{\pgfqpoint{3.215588in}{1.547282in}}%
\pgfpathlineto{\pgfqpoint{3.166387in}{1.433688in}}%
\pgfpathlineto{\pgfqpoint{3.105939in}{1.325604in}}%
\pgfpathlineto{\pgfqpoint{3.034828in}{1.224100in}}%
\pgfpathlineto{\pgfqpoint{2.953761in}{1.130170in}}%
\pgfpathlineto{\pgfqpoint{2.863577in}{1.044736in}}%
\pgfpathlineto{\pgfqpoint{2.765241in}{0.968651in}}%
\pgfpathlineto{\pgfqpoint{2.659847in}{0.902692in}}%
\pgfpathlineto{\pgfqpoint{2.548440in}{0.847463in}}%
\pgfpathlineto{\pgfqpoint{2.432036in}{0.803486in}}%
\pgfpathlineto{\pgfqpoint{2.311827in}{0.771281in}}%
\pgfpathlineto{\pgfqpoint{2.189014in}{0.751226in}}%
\pgfpathlineto{\pgfqpoint{2.064807in}{0.743551in}}%
\pgfpathlineto{\pgfqpoint{1.940424in}{0.748339in}}%
\pgfpathlineto{\pgfqpoint{1.817096in}{0.765532in}}%
\pgfpathlineto{\pgfqpoint{1.696060in}{0.794922in}}%
\pgfpathlineto{\pgfqpoint{1.578564in}{0.836159in}}%
\pgfpathlineto{\pgfqpoint{1.465856in}{0.888749in}}%
\pgfpathlineto{\pgfqpoint{1.358964in}{0.952189in}}%
\pgfpathlineto{\pgfqpoint{1.258955in}{1.025891in}}%
\pgfpathlineto{\pgfqpoint{1.166891in}{1.109137in}}%
\pgfpathlineto{\pgfqpoint{1.083720in}{1.201120in}}%
\pgfpathlineto{\pgfqpoint{1.010276in}{1.300945in}}%
\pgfpathlineto{\pgfqpoint{0.947281in}{1.407626in}}%
\pgfpathlineto{\pgfqpoint{0.895344in}{1.520087in}}%
\pgfpathlineto{\pgfqpoint{0.854957in}{1.637165in}}%
\pgfpathlineto{\pgfqpoint{0.826503in}{1.757605in}}%
\pgfpathlineto{\pgfqpoint{0.810202in}{1.880252in}}%
\pgfpathlineto{\pgfqpoint{0.806233in}{2.003925in}}%
\pgfpathlineto{\pgfqpoint{0.814707in}{2.127338in}}%
\pgfpathlineto{\pgfqpoint{0.835596in}{2.249242in}}%
\pgfpathlineto{\pgfqpoint{0.868725in}{2.368426in}}%
\pgfpathlineto{\pgfqpoint{0.913781in}{2.483718in}}%
\pgfpathlineto{\pgfqpoint{0.970306in}{2.593983in}}%
\pgfpathlineto{\pgfqpoint{1.037701in}{2.698126in}}%
\pgfpathlineto{\pgfqpoint{1.115224in}{2.795090in}}%
\pgfpathlineto{\pgfqpoint{1.201992in}{2.883855in}}%
\pgfpathlineto{\pgfqpoint{1.297165in}{2.963612in}}%
\pgfpathlineto{\pgfqpoint{1.399861in}{3.033584in}}%
\pgfpathlineto{\pgfqpoint{1.509049in}{3.092996in}}%
\pgfpathlineto{\pgfqpoint{1.623653in}{3.141207in}}%
\pgfpathlineto{\pgfqpoint{1.742549in}{3.177719in}}%
\pgfpathlineto{\pgfqpoint{1.864569in}{3.202168in}}%
\pgfpathlineto{\pgfqpoint{1.988500in}{3.214329in}}%
\pgfpathlineto{\pgfqpoint{2.113083in}{3.214116in}}%
\pgfpathlineto{\pgfqpoint{2.237013in}{3.201577in}}%
\pgfpathlineto{\pgfqpoint{2.358998in}{3.176896in}}%
\pgfpathlineto{\pgfqpoint{2.477957in}{3.140310in}}%
\pgfpathlineto{\pgfqpoint{2.592645in}{3.092140in}}%
\pgfpathlineto{\pgfqpoint{2.701865in}{3.032825in}}%
\pgfpathlineto{\pgfqpoint{2.804508in}{2.962917in}}%
\pgfpathlineto{\pgfqpoint{2.899553in}{2.883089in}}%
\pgfpathlineto{\pgfqpoint{2.986067in}{2.794128in}}%
\pgfpathlineto{\pgfqpoint{3.063206in}{2.696941in}}%
\pgfpathlineto{\pgfqpoint{3.130213in}{2.592548in}}%
\pgfpathlineto{\pgfqpoint{3.186418in}{2.482090in}}%
\pgfpathlineto{\pgfqpoint{3.231291in}{2.366714in}}%
\pgfpathlineto{\pgfqpoint{3.264440in}{2.247445in}}%
\pgfpathlineto{\pgfqpoint{3.285452in}{2.125512in}}%
\pgfpathlineto{\pgfqpoint{3.294057in}{2.002144in}}%
\pgfpathlineto{\pgfqpoint{3.290131in}{1.878563in}}%
\pgfpathlineto{\pgfqpoint{3.273696in}{1.755984in}}%
\pgfpathlineto{\pgfqpoint{3.244923in}{1.635613in}}%
\pgfpathlineto{\pgfqpoint{3.204128in}{1.518651in}}%
\pgfpathlineto{\pgfqpoint{3.151775in}{1.406287in}}%
\pgfpathlineto{\pgfqpoint{3.088472in}{1.299707in}}%
\pgfpathlineto{\pgfqpoint{3.014877in}{1.199946in}}%
\pgfpathlineto{\pgfqpoint{2.931648in}{1.107928in}}%
\pgfpathlineto{\pgfqpoint{2.839607in}{1.024646in}}%
\pgfpathlineto{\pgfqpoint{2.739655in}{0.950976in}}%
\pgfpathlineto{\pgfqpoint{2.632765in}{0.887666in}}%
\pgfpathlineto{\pgfqpoint{2.519988in}{0.835344in}}%
\pgfpathlineto{\pgfqpoint{2.402450in}{0.794513in}}%
\pgfpathlineto{\pgfqpoint{2.281350in}{0.765553in}}%
\pgfpathlineto{\pgfqpoint{2.157965in}{0.748722in}}%
\pgfpathlineto{\pgfqpoint{2.033567in}{0.744141in}}%
\pgfpathlineto{\pgfqpoint{1.909278in}{0.751833in}}%
\pgfpathlineto{\pgfqpoint{1.786399in}{0.771786in}}%
\pgfpathlineto{\pgfqpoint{1.666193in}{0.803855in}}%
\pgfpathlineto{\pgfqpoint{1.549868in}{0.847764in}}%
\pgfpathlineto{\pgfqpoint{1.438573in}{0.903100in}}%
\pgfpathlineto{\pgfqpoint{1.333400in}{0.969315in}}%
\pgfpathlineto{\pgfqpoint{1.235387in}{1.045727in}}%
\pgfpathlineto{\pgfqpoint{1.145512in}{1.131516in}}%
\pgfpathlineto{\pgfqpoint{1.064697in}{1.225730in}}%
\pgfpathlineto{\pgfqpoint{0.993745in}{1.327360in}}%
\pgfpathlineto{\pgfqpoint{0.933271in}{1.435514in}}%
\pgfpathlineto{\pgfqpoint{0.883959in}{1.549101in}}%
\pgfpathlineto{\pgfqpoint{0.846360in}{1.666985in}}%
\pgfpathlineto{\pgfqpoint{0.820884in}{1.788006in}}%
\pgfpathlineto{\pgfqpoint{0.807796in}{1.910976in}}%
\pgfpathlineto{\pgfqpoint{0.807215in}{2.034683in}}%
\pgfpathlineto{\pgfqpoint{0.819121in}{2.157886in}}%
\pgfpathlineto{\pgfqpoint{0.843344in}{2.279321in}}%
\pgfpathlineto{\pgfqpoint{0.879575in}{2.397694in}}%
\pgfpathlineto{\pgfqpoint{0.927431in}{2.511886in}}%
\pgfpathlineto{\pgfqpoint{0.986483in}{2.620807in}}%
\pgfpathlineto{\pgfqpoint{1.056170in}{2.723294in}}%
\pgfpathlineto{\pgfqpoint{1.135830in}{2.818290in}}%
\pgfpathlineto{\pgfqpoint{1.224696in}{2.904837in}}%
\pgfpathlineto{\pgfqpoint{1.321898in}{2.982081in}}%
\pgfpathlineto{\pgfqpoint{1.426461in}{3.049270in}}%
\pgfpathlineto{\pgfqpoint{1.537308in}{3.105753in}}%
\pgfpathlineto{\pgfqpoint{1.653259in}{3.150982in}}%
\pgfpathlineto{\pgfqpoint{1.773075in}{3.184530in}}%
\pgfpathlineto{\pgfqpoint{1.895685in}{3.206125in}}%
\pgfpathlineto{\pgfqpoint{2.019839in}{3.215487in}}%
\pgfpathlineto{\pgfqpoint{2.144268in}{3.212456in}}%
\pgfpathlineto{\pgfqpoint{2.267729in}{3.197017in}}%
\pgfpathlineto{\pgfqpoint{2.389006in}{3.169299in}}%
\pgfpathlineto{\pgfqpoint{2.506906in}{3.129577in}}%
\pgfpathlineto{\pgfqpoint{2.620263in}{3.078267in}}%
\pgfpathlineto{\pgfqpoint{2.727937in}{3.015932in}}%
\pgfpathlineto{\pgfqpoint{2.828812in}{2.943276in}}%
\pgfpathlineto{\pgfqpoint{2.921850in}{2.861106in}}%
\pgfpathlineto{\pgfqpoint{3.006243in}{2.770144in}}%
\pgfpathlineto{\pgfqpoint{3.081080in}{2.671289in}}%
\pgfpathlineto{\pgfqpoint{3.145556in}{2.565518in}}%
\pgfpathlineto{\pgfqpoint{3.199000in}{2.453866in}}%
\pgfpathlineto{\pgfqpoint{3.240874in}{2.337426in}}%
\pgfpathlineto{\pgfqpoint{3.270771in}{2.217351in}}%
\pgfpathlineto{\pgfqpoint{3.288421in}{2.094849in}}%
\pgfpathlineto{\pgfqpoint{3.293685in}{1.971190in}}%
\pgfpathlineto{\pgfqpoint{3.286558in}{1.847695in}}%
\pgfpathlineto{\pgfqpoint{3.267150in}{1.725500in}}%
\pgfpathlineto{\pgfqpoint{3.235616in}{1.605813in}}%
\pgfpathlineto{\pgfqpoint{3.192219in}{1.489891in}}%
\pgfpathlineto{\pgfqpoint{3.137346in}{1.378919in}}%
\pgfpathlineto{\pgfqpoint{3.071511in}{1.274006in}}%
\pgfpathlineto{\pgfqpoint{2.995354in}{1.176183in}}%
\pgfpathlineto{\pgfqpoint{2.909643in}{1.086411in}}%
\pgfpathlineto{\pgfqpoint{2.815269in}{1.005572in}}%
\pgfpathlineto{\pgfqpoint{2.713254in}{0.934476in}}%
\pgfpathlineto{\pgfqpoint{2.604731in}{0.873847in}}%
\pgfpathlineto{\pgfqpoint{2.490690in}{0.824199in}}%
\pgfpathlineto{\pgfqpoint{2.372241in}{0.786063in}}%
\pgfpathlineto{\pgfqpoint{2.250588in}{0.759892in}}%
\pgfpathlineto{\pgfqpoint{2.126940in}{0.745994in}}%
\pgfpathlineto{\pgfqpoint{2.002513in}{0.744534in}}%
\pgfpathlineto{\pgfqpoint{1.878532in}{0.755528in}}%
\pgfpathlineto{\pgfqpoint{1.756223in}{0.778848in}}%
\pgfpathlineto{\pgfqpoint{1.636822in}{0.814220in}}%
\pgfpathlineto{\pgfqpoint{1.521571in}{0.861224in}}%
\pgfpathlineto{\pgfqpoint{1.411677in}{0.919316in}}%
\pgfpathlineto{\pgfqpoint{1.308115in}{0.987975in}}%
\pgfpathlineto{\pgfqpoint{1.211966in}{1.066545in}}%
\pgfpathlineto{\pgfqpoint{1.124243in}{1.154258in}}%
\pgfpathlineto{\pgfqpoint{1.045841in}{1.250262in}}%
\pgfpathlineto{\pgfqpoint{0.977544in}{1.353619in}}%
\pgfpathlineto{\pgfqpoint{0.920017in}{1.463303in}}%
\pgfpathlineto{\pgfqpoint{0.873812in}{1.578200in}}%
\pgfpathlineto{\pgfqpoint{0.839366in}{1.697110in}}%
\pgfpathlineto{\pgfqpoint{0.816992in}{1.818770in}}%
\pgfpathlineto{\pgfqpoint{0.806854in}{1.942074in}}%
\pgfpathlineto{\pgfqpoint{0.809107in}{2.065765in}}%
\pgfpathlineto{\pgfqpoint{0.823797in}{2.188569in}}%
\pgfpathlineto{\pgfqpoint{0.850823in}{2.309250in}}%
\pgfpathlineto{\pgfqpoint{0.889946in}{2.426616in}}%
\pgfpathlineto{\pgfqpoint{0.940783in}{2.539514in}}%
\pgfpathlineto{\pgfqpoint{1.002812in}{2.646829in}}%
\pgfpathlineto{\pgfqpoint{1.075367in}{2.747491in}}%
\pgfpathlineto{\pgfqpoint{1.157641in}{2.840467in}}%
\pgfpathlineto{\pgfqpoint{1.248711in}{2.924792in}}%
\pgfpathlineto{\pgfqpoint{1.347765in}{2.999739in}}%
\pgfpathlineto{\pgfqpoint{1.453832in}{3.064510in}}%
\pgfpathlineto{\pgfqpoint{1.565854in}{3.118393in}}%
\pgfpathlineto{\pgfqpoint{1.682730in}{3.160813in}}%
\pgfpathlineto{\pgfqpoint{1.803314in}{3.191334in}}%
\pgfpathlineto{\pgfqpoint{1.926418in}{3.209661in}}%
\pgfpathlineto{\pgfqpoint{2.050811in}{3.215633in}}%
\pgfpathlineto{\pgfqpoint{2.175219in}{3.209231in}}%
\pgfpathlineto{\pgfqpoint{2.298322in}{3.190573in}}%
\pgfpathlineto{\pgfqpoint{2.418908in}{3.159890in}}%
\pgfpathlineto{\pgfqpoint{2.535868in}{3.117457in}}%
\pgfpathlineto{\pgfqpoint{2.647956in}{3.063653in}}%
\pgfpathlineto{\pgfqpoint{2.754014in}{2.998975in}}%
\pgfpathlineto{\pgfqpoint{2.852973in}{2.924035in}}%
\pgfpathlineto{\pgfqpoint{2.943853in}{2.839560in}}%
\pgfpathlineto{\pgfqpoint{3.025765in}{2.746390in}}%
\pgfpathlineto{\pgfqpoint{3.097909in}{2.645485in}}%
\pgfpathlineto{\pgfqpoint{3.159575in}{2.537917in}}%
\pgfpathlineto{\pgfqpoint{3.210146in}{2.424871in}}%
\pgfpathlineto{\pgfqpoint{3.249189in}{2.307417in}}%
\pgfpathlineto{\pgfqpoint{3.276301in}{2.186672in}}%
\pgfpathlineto{\pgfqpoint{3.291135in}{2.063877in}}%
\pgfpathlineto{\pgfqpoint{3.293490in}{1.940263in}}%
\pgfpathlineto{\pgfqpoint{3.283311in}{1.817050in}}%
\pgfpathlineto{\pgfqpoint{3.260690in}{1.695447in}}%
\pgfpathlineto{\pgfqpoint{3.225868in}{1.576656in}}%
\pgfpathlineto{\pgfqpoint{3.179228in}{1.461867in}}%
\pgfpathlineto{\pgfqpoint{3.121305in}{1.352261in}}%
\pgfpathlineto{\pgfqpoint{3.052770in}{1.248997in}}%
\pgfpathlineto{\pgfqpoint{2.974262in}{1.153002in}}%
\pgfpathlineto{\pgfqpoint{2.886524in}{1.065247in}}%
\pgfpathlineto{\pgfqpoint{2.790420in}{0.986673in}}%
\pgfpathlineto{\pgfqpoint{2.686889in}{0.918097in}}%
\pgfpathlineto{\pgfqpoint{2.576942in}{0.860212in}}%
\pgfpathlineto{\pgfqpoint{2.461664in}{0.813585in}}%
\pgfpathlineto{\pgfqpoint{2.342211in}{0.778659in}}%
\pgfpathlineto{\pgfqpoint{2.219816in}{0.755754in}}%
\pgfpathlineto{\pgfqpoint{2.095781in}{0.745063in}}%
\pgfpathlineto{\pgfqpoint{1.971310in}{0.746640in}}%
\pgfpathlineto{\pgfqpoint{1.847579in}{0.760483in}}%
\pgfpathlineto{\pgfqpoint{1.725890in}{0.786519in}}%
\pgfpathlineto{\pgfqpoint{1.607484in}{0.824539in}}%
\pgfpathlineto{\pgfqpoint{1.493545in}{0.874201in}}%
\pgfpathlineto{\pgfqpoint{1.385195in}{0.935030in}}%
\pgfpathlineto{\pgfqpoint{1.283497in}{1.006415in}}%
\pgfpathlineto{\pgfqpoint{1.189457in}{1.087612in}}%
\pgfpathlineto{\pgfqpoint{1.104019in}{1.177742in}}%
\pgfpathlineto{\pgfqpoint{1.028068in}{1.275794in}}%
\pgfpathlineto{\pgfqpoint{0.962300in}{1.380803in}}%
\pgfpathlineto{\pgfqpoint{0.907343in}{1.491807in}}%
\pgfpathlineto{\pgfqpoint{0.863829in}{1.607678in}}%
\pgfpathlineto{\pgfqpoint{0.832242in}{1.727263in}}%
\pgfpathlineto{\pgfqpoint{0.812926in}{1.849389in}}%
\pgfpathlineto{\pgfqpoint{0.806077in}{1.972854in}}%
\pgfpathlineto{\pgfqpoint{0.811748in}{2.096439in}}%
\pgfpathlineto{\pgfqpoint{0.829847in}{2.218895in}}%
\pgfpathlineto{\pgfqpoint{0.860138in}{2.338954in}}%
\pgfpathlineto{\pgfqpoint{0.902247in}{2.455341in}}%
\pgfpathlineto{\pgfqpoint{0.955774in}{2.567019in}}%
\pgfpathlineto{\pgfqpoint{1.020223in}{2.672848in}}%
\pgfpathlineto{\pgfqpoint{1.094980in}{2.771708in}}%
\pgfpathlineto{\pgfqpoint{1.179330in}{2.862588in}}%
\pgfpathlineto{\pgfqpoint{1.272456in}{2.944578in}}%
\pgfpathlineto{\pgfqpoint{1.373440in}{3.016872in}}%
\pgfpathlineto{\pgfqpoint{1.481261in}{3.078769in}}%
\pgfpathlineto{\pgfqpoint{1.594798in}{3.129672in}}%
\pgfpathlineto{\pgfqpoint{1.712829in}{3.169086in}}%
\pgfpathlineto{\pgfqpoint{1.834163in}{3.196668in}}%
\pgfpathlineto{\pgfqpoint{1.957692in}{3.212162in}}%
\pgfpathlineto{\pgfqpoint{2.082131in}{3.215337in}}%
\pgfpathlineto{\pgfqpoint{2.206219in}{3.206103in}}%
\pgfpathlineto{\pgfqpoint{2.328724in}{3.184514in}}%
\pgfpathlineto{\pgfqpoint{2.448440in}{3.150768in}}%
\pgfpathlineto{\pgfqpoint{2.564189in}{3.105205in}}%
\pgfpathlineto{\pgfqpoint{2.674822in}{3.048312in}}%
\pgfpathlineto{\pgfqpoint{2.779217in}{2.980716in}}%
\pgfpathlineto{\pgfqpoint{2.876277in}{2.903191in}}%
\pgfpathlineto{\pgfqpoint{2.965073in}{2.816529in}}%
\pgfpathlineto{\pgfqpoint{3.044781in}{2.721506in}}%
\pgfpathlineto{\pgfqpoint{3.114525in}{2.619076in}}%
\pgfpathlineto{\pgfqpoint{3.173562in}{2.510246in}}%
\pgfpathlineto{\pgfqpoint{3.221283in}{2.396080in}}%
\pgfpathlineto{\pgfqpoint{3.257211in}{2.277700in}}%
\pgfpathlineto{\pgfqpoint{3.281006in}{2.156281in}}%
\pgfpathlineto{\pgfqpoint{3.292463in}{2.033058in}}%
\pgfpathlineto{\pgfqpoint{3.291507in}{1.909318in}}%
\pgfpathlineto{\pgfqpoint{3.278200in}{1.786362in}}%
\pgfpathlineto{\pgfqpoint{3.252690in}{1.665282in}}%
\pgfpathlineto{\pgfqpoint{3.215181in}{1.547336in}}%
\pgfpathlineto{\pgfqpoint{3.165997in}{1.433753in}}%
\pgfpathlineto{\pgfqpoint{3.105587in}{1.325684in}}%
\pgfpathlineto{\pgfqpoint{3.034526in}{1.224203in}}%
\pgfpathlineto{\pgfqpoint{2.953513in}{1.130306in}}%
\pgfpathlineto{\pgfqpoint{2.863373in}{1.044914in}}%
\pgfpathlineto{\pgfqpoint{2.765054in}{0.968869in}}%
\pgfpathlineto{\pgfqpoint{2.659631in}{0.902937in}}%
\pgfpathlineto{\pgfqpoint{2.548239in}{0.847769in}}%
\pgfpathlineto{\pgfqpoint{2.431856in}{0.803830in}}%
\pgfpathlineto{\pgfqpoint{2.311662in}{0.771634in}}%
\pgfpathlineto{\pgfqpoint{2.188871in}{0.751566in}}%
\pgfpathlineto{\pgfqpoint{2.064698in}{0.743865in}}%
\pgfpathlineto{\pgfqpoint{1.940362in}{0.748627in}}%
\pgfpathlineto{\pgfqpoint{1.817089in}{0.765799in}}%
\pgfpathlineto{\pgfqpoint{1.696105in}{0.795184in}}%
\pgfpathlineto{\pgfqpoint{1.578642in}{0.836440in}}%
\pgfpathlineto{\pgfqpoint{1.465934in}{0.889077in}}%
\pgfpathlineto{\pgfqpoint{1.359109in}{0.952526in}}%
\pgfpathlineto{\pgfqpoint{1.259137in}{1.026230in}}%
\pgfpathlineto{\pgfqpoint{1.167088in}{1.109465in}}%
\pgfpathlineto{\pgfqpoint{1.083922in}{1.201422in}}%
\pgfpathlineto{\pgfqpoint{1.010485in}{1.301206in}}%
\pgfpathlineto{\pgfqpoint{0.947502in}{1.407837in}}%
\pgfpathlineto{\pgfqpoint{0.895586in}{1.520250in}}%
\pgfpathlineto{\pgfqpoint{0.855228in}{1.637295in}}%
\pgfpathlineto{\pgfqpoint{0.826807in}{1.757739in}}%
\pgfpathlineto{\pgfqpoint{0.810563in}{1.880340in}}%
\pgfpathlineto{\pgfqpoint{0.806625in}{2.003983in}}%
\pgfpathlineto{\pgfqpoint{0.815103in}{2.127380in}}%
\pgfpathlineto{\pgfqpoint{0.835972in}{2.249267in}}%
\pgfpathlineto{\pgfqpoint{0.869066in}{2.368426in}}%
\pgfpathlineto{\pgfqpoint{0.914079in}{2.483682in}}%
\pgfpathlineto{\pgfqpoint{0.970565in}{2.593901in}}%
\pgfpathlineto{\pgfqpoint{1.037934in}{2.697996in}}%
\pgfpathlineto{\pgfqpoint{1.115459in}{2.794920in}}%
\pgfpathlineto{\pgfqpoint{1.202269in}{2.883672in}}%
\pgfpathlineto{\pgfqpoint{1.297434in}{2.963365in}}%
\pgfpathlineto{\pgfqpoint{1.400120in}{3.033301in}}%
\pgfpathlineto{\pgfqpoint{1.509294in}{3.092704in}}%
\pgfpathlineto{\pgfqpoint{1.623871in}{3.140924in}}%
\pgfpathlineto{\pgfqpoint{1.742726in}{3.177449in}}%
\pgfpathlineto{\pgfqpoint{1.864695in}{3.201909in}}%
\pgfpathlineto{\pgfqpoint{1.988571in}{3.214071in}}%
\pgfpathlineto{\pgfqpoint{2.113109in}{3.213843in}}%
\pgfpathlineto{\pgfqpoint{2.237023in}{3.201272in}}%
\pgfpathlineto{\pgfqpoint{2.358989in}{3.176544in}}%
\pgfpathlineto{\pgfqpoint{2.477887in}{3.139927in}}%
\pgfpathlineto{\pgfqpoint{2.592547in}{3.091745in}}%
\pgfpathlineto{\pgfqpoint{2.701753in}{3.032438in}}%
\pgfpathlineto{\pgfqpoint{2.804386in}{2.962561in}}%
\pgfpathlineto{\pgfqpoint{2.899416in}{2.882779in}}%
\pgfpathlineto{\pgfqpoint{2.985906in}{2.793874in}}%
\pgfpathlineto{\pgfqpoint{3.063011in}{2.696738in}}%
\pgfpathlineto{\pgfqpoint{3.129979in}{2.592378in}}%
\pgfpathlineto{\pgfqpoint{3.186151in}{2.481913in}}%
\pgfpathlineto{\pgfqpoint{3.230973in}{2.366546in}}%
\pgfpathlineto{\pgfqpoint{3.264079in}{2.247306in}}%
\pgfpathlineto{\pgfqpoint{3.285080in}{2.125390in}}%
\pgfpathlineto{\pgfqpoint{3.293696in}{2.002041in}}%
\pgfpathlineto{\pgfqpoint{3.289795in}{1.878488in}}%
\pgfpathlineto{\pgfqpoint{3.273392in}{1.755950in}}%
\pgfpathlineto{\pgfqpoint{3.244648in}{1.635629in}}%
\pgfpathlineto{\pgfqpoint{3.203870in}{1.518718in}}%
\pgfpathlineto{\pgfqpoint{3.151510in}{1.406395in}}%
\pgfpathlineto{\pgfqpoint{3.088170in}{1.299826in}}%
\pgfpathlineto{\pgfqpoint{3.014561in}{1.200116in}}%
\pgfpathlineto{\pgfqpoint{2.931335in}{1.108138in}}%
\pgfpathlineto{\pgfqpoint{2.839306in}{1.024871in}}%
\pgfpathlineto{\pgfqpoint{2.739379in}{0.951200in}}%
\pgfpathlineto{\pgfqpoint{2.632529in}{0.887888in}}%
\pgfpathlineto{\pgfqpoint{2.519802in}{0.835569in}}%
\pgfpathlineto{\pgfqpoint{2.402314in}{0.794750in}}%
\pgfpathlineto{\pgfqpoint{2.281252in}{0.765815in}}%
\pgfpathlineto{\pgfqpoint{2.157874in}{0.749020in}}%
\pgfpathlineto{\pgfqpoint{2.033496in}{0.744493in}}%
\pgfpathlineto{\pgfqpoint{1.909257in}{0.752230in}}%
\pgfpathlineto{\pgfqpoint{1.786398in}{0.772198in}}%
\pgfpathlineto{\pgfqpoint{1.666204in}{0.804255in}}%
\pgfpathlineto{\pgfqpoint{1.549893in}{0.848129in}}%
\pgfpathlineto{\pgfqpoint{1.438623in}{0.903417in}}%
\pgfpathlineto{\pgfqpoint{1.333488in}{0.969580in}}%
\pgfpathlineto{\pgfqpoint{1.235522in}{1.045948in}}%
\pgfpathlineto{\pgfqpoint{1.145693in}{1.131718in}}%
\pgfpathlineto{\pgfqpoint{1.064908in}{1.225956in}}%
\pgfpathlineto{\pgfqpoint{0.994003in}{1.327606in}}%
\pgfpathlineto{\pgfqpoint{0.933585in}{1.435743in}}%
\pgfpathlineto{\pgfqpoint{0.884290in}{1.549318in}}%
\pgfpathlineto{\pgfqpoint{0.846684in}{1.667183in}}%
\pgfpathlineto{\pgfqpoint{0.821187in}{1.788172in}}%
\pgfpathlineto{\pgfqpoint{0.808074in}{1.911095in}}%
\pgfpathlineto{\pgfqpoint{0.807477in}{2.034745in}}%
\pgfpathlineto{\pgfqpoint{0.819380in}{2.157893in}}%
\pgfpathlineto{\pgfqpoint{0.843623in}{2.279290in}}%
\pgfpathlineto{\pgfqpoint{0.879903in}{2.397667in}}%
\pgfpathlineto{\pgfqpoint{0.927802in}{2.511777in}}%
\pgfpathlineto{\pgfqpoint{0.986919in}{2.620556in}}%
\pgfpathlineto{\pgfqpoint{1.056679in}{2.722911in}}%
\pgfpathlineto{\pgfqpoint{1.136389in}{2.817814in}}%
\pgfpathlineto{\pgfqpoint{1.225254in}{2.904325in}}%
\pgfpathlineto{\pgfqpoint{1.322374in}{2.981585in}}%
\pgfpathlineto{\pgfqpoint{1.322374in}{2.981585in}}%
\pgfusepath{stroke}%
\end{pgfscope}%
\begin{pgfscope}%
\pgfpathrectangle{\pgfqpoint{0.500000in}{0.440000in}}{\pgfqpoint{3.100000in}{3.080000in}}%
\pgfusepath{clip}%
\pgfsetrectcap%
\pgfsetroundjoin%
\pgfsetlinewidth{0.501875pt}%
\definecolor{currentstroke}{rgb}{0.549020,0.337255,0.294118}%
\pgfsetstrokecolor{currentstroke}%
\pgfsetdash{}{0pt}%
\pgfpathmoveto{\pgfqpoint{1.586864in}{2.593400in}}%
\pgfpathlineto{\pgfqpoint{1.650895in}{2.636316in}}%
\pgfpathlineto{\pgfqpoint{1.718933in}{2.672700in}}%
\pgfpathlineto{\pgfqpoint{1.790287in}{2.702186in}}%
\pgfpathlineto{\pgfqpoint{1.864230in}{2.724436in}}%
\pgfpathlineto{\pgfqpoint{1.940025in}{2.739200in}}%
\pgfpathlineto{\pgfqpoint{2.016926in}{2.746318in}}%
\pgfpathlineto{\pgfqpoint{2.094175in}{2.745718in}}%
\pgfpathlineto{\pgfqpoint{2.171005in}{2.737417in}}%
\pgfpathlineto{\pgfqpoint{2.246641in}{2.721523in}}%
\pgfpathlineto{\pgfqpoint{2.320295in}{2.698230in}}%
\pgfpathlineto{\pgfqpoint{2.391223in}{2.667802in}}%
\pgfpathlineto{\pgfqpoint{2.458777in}{2.630514in}}%
\pgfpathlineto{\pgfqpoint{2.522239in}{2.586723in}}%
\pgfpathlineto{\pgfqpoint{2.580947in}{2.536849in}}%
\pgfpathlineto{\pgfqpoint{2.634306in}{2.481373in}}%
\pgfpathlineto{\pgfqpoint{2.681787in}{2.420838in}}%
\pgfpathlineto{\pgfqpoint{2.722928in}{2.355847in}}%
\pgfpathlineto{\pgfqpoint{2.757329in}{2.287066in}}%
\pgfpathlineto{\pgfqpoint{2.784661in}{2.215219in}}%
\pgfpathlineto{\pgfqpoint{2.804664in}{2.141073in}}%
\pgfpathlineto{\pgfqpoint{2.817181in}{2.065299in}}%
\pgfpathlineto{\pgfqpoint{2.822048in}{1.988668in}}%
\pgfpathlineto{\pgfqpoint{2.819174in}{1.911965in}}%
\pgfpathlineto{\pgfqpoint{2.808560in}{1.835957in}}%
\pgfpathlineto{\pgfqpoint{2.790295in}{1.761395in}}%
\pgfpathlineto{\pgfqpoint{2.764557in}{1.689011in}}%
\pgfpathlineto{\pgfqpoint{2.731615in}{1.619520in}}%
\pgfpathlineto{\pgfqpoint{2.691829in}{1.553620in}}%
\pgfpathlineto{\pgfqpoint{2.645646in}{1.491991in}}%
\pgfpathlineto{\pgfqpoint{2.593582in}{1.435272in}}%
\pgfpathlineto{\pgfqpoint{2.536099in}{1.383955in}}%
\pgfpathlineto{\pgfqpoint{2.473762in}{1.338588in}}%
\pgfpathlineto{\pgfqpoint{2.407188in}{1.299663in}}%
\pgfpathlineto{\pgfqpoint{2.337030in}{1.267586in}}%
\pgfpathlineto{\pgfqpoint{2.263974in}{1.242682in}}%
\pgfpathlineto{\pgfqpoint{2.188742in}{1.225194in}}%
\pgfpathlineto{\pgfqpoint{2.112090in}{1.215279in}}%
\pgfpathlineto{\pgfqpoint{2.034809in}{1.213014in}}%
\pgfpathlineto{\pgfqpoint{1.957724in}{1.218391in}}%
\pgfpathlineto{\pgfqpoint{1.881560in}{1.231330in}}%
\pgfpathlineto{\pgfqpoint{1.807053in}{1.251724in}}%
\pgfpathlineto{\pgfqpoint{1.734988in}{1.279399in}}%
\pgfpathlineto{\pgfqpoint{1.666106in}{1.314109in}}%
\pgfpathlineto{\pgfqpoint{1.601094in}{1.355528in}}%
\pgfpathlineto{\pgfqpoint{1.540596in}{1.403255in}}%
\pgfpathlineto{\pgfqpoint{1.485203in}{1.456810in}}%
\pgfpathlineto{\pgfqpoint{1.435459in}{1.515638in}}%
\pgfpathlineto{\pgfqpoint{1.391860in}{1.579106in}}%
\pgfpathlineto{\pgfqpoint{1.354850in}{1.646510in}}%
\pgfpathlineto{\pgfqpoint{1.324747in}{1.717221in}}%
\pgfpathlineto{\pgfqpoint{1.301863in}{1.790561in}}%
\pgfpathlineto{\pgfqpoint{1.286475in}{1.865784in}}%
\pgfpathlineto{\pgfqpoint{1.278766in}{1.942138in}}%
\pgfpathlineto{\pgfqpoint{1.278832in}{2.018872in}}%
\pgfpathlineto{\pgfqpoint{1.286673in}{2.095232in}}%
\pgfpathlineto{\pgfqpoint{1.302203in}{2.170463in}}%
\pgfpathlineto{\pgfqpoint{1.325241in}{2.243807in}}%
\pgfpathlineto{\pgfqpoint{1.355516in}{2.314504in}}%
\pgfpathlineto{\pgfqpoint{1.392675in}{2.381808in}}%
\pgfpathlineto{\pgfqpoint{1.436381in}{2.445122in}}%
\pgfpathlineto{\pgfqpoint{1.486216in}{2.503790in}}%
\pgfpathlineto{\pgfqpoint{1.541693in}{2.557194in}}%
\pgfpathlineto{\pgfqpoint{1.602274in}{2.604784in}}%
\pgfpathlineto{\pgfqpoint{1.667366in}{2.646084in}}%
\pgfpathlineto{\pgfqpoint{1.736326in}{2.680690in}}%
\pgfpathlineto{\pgfqpoint{1.808454in}{2.708271in}}%
\pgfpathlineto{\pgfqpoint{1.883002in}{2.728567in}}%
\pgfpathlineto{\pgfqpoint{1.959174in}{2.741394in}}%
\pgfpathlineto{\pgfqpoint{2.036266in}{2.746662in}}%
\pgfpathlineto{\pgfqpoint{2.113508in}{2.744292in}}%
\pgfpathlineto{\pgfqpoint{2.190100in}{2.734267in}}%
\pgfpathlineto{\pgfqpoint{2.265267in}{2.716655in}}%
\pgfpathlineto{\pgfqpoint{2.338266in}{2.691613in}}%
\pgfpathlineto{\pgfqpoint{2.408378in}{2.659382in}}%
\pgfpathlineto{\pgfqpoint{2.474913in}{2.620291in}}%
\pgfpathlineto{\pgfqpoint{2.537209in}{2.574757in}}%
\pgfpathlineto{\pgfqpoint{2.594631in}{2.523282in}}%
\pgfpathlineto{\pgfqpoint{2.646580in}{2.466448in}}%
\pgfpathlineto{\pgfqpoint{2.692601in}{2.404775in}}%
\pgfpathlineto{\pgfqpoint{2.732220in}{2.338857in}}%
\pgfpathlineto{\pgfqpoint{2.764995in}{2.269355in}}%
\pgfpathlineto{\pgfqpoint{2.790575in}{2.196953in}}%
\pgfpathlineto{\pgfqpoint{2.808693in}{2.122362in}}%
\pgfpathlineto{\pgfqpoint{2.819172in}{2.046314in}}%
\pgfpathlineto{\pgfqpoint{2.821919in}{1.969568in}}%
\pgfpathlineto{\pgfqpoint{2.816933in}{1.892907in}}%
\pgfpathlineto{\pgfqpoint{2.804295in}{1.817136in}}%
\pgfpathlineto{\pgfqpoint{2.784166in}{1.743030in}}%
\pgfpathlineto{\pgfqpoint{2.756729in}{1.671250in}}%
\pgfpathlineto{\pgfqpoint{2.722234in}{1.602556in}}%
\pgfpathlineto{\pgfqpoint{2.680999in}{1.537665in}}%
\pgfpathlineto{\pgfqpoint{2.633415in}{1.477233in}}%
\pgfpathlineto{\pgfqpoint{2.579942in}{1.421860in}}%
\pgfpathlineto{\pgfqpoint{2.521111in}{1.372088in}}%
\pgfpathlineto{\pgfqpoint{2.457523in}{1.328403in}}%
\pgfpathlineto{\pgfqpoint{2.389848in}{1.291232in}}%
\pgfpathlineto{\pgfqpoint{2.318829in}{1.260946in}}%
\pgfpathlineto{\pgfqpoint{2.245161in}{1.237809in}}%
\pgfpathlineto{\pgfqpoint{2.169528in}{1.222048in}}%
\pgfpathlineto{\pgfqpoint{2.092710in}{1.213869in}}%
\pgfpathlineto{\pgfqpoint{2.015479in}{1.213387in}}%
\pgfpathlineto{\pgfqpoint{1.938601in}{1.220629in}}%
\pgfpathlineto{\pgfqpoint{1.862830in}{1.235530in}}%
\pgfpathlineto{\pgfqpoint{1.788912in}{1.257937in}}%
\pgfpathlineto{\pgfqpoint{1.717587in}{1.287602in}}%
\pgfpathlineto{\pgfqpoint{1.649584in}{1.324192in}}%
\pgfpathlineto{\pgfqpoint{1.585623in}{1.367282in}}%
\pgfpathlineto{\pgfqpoint{1.526293in}{1.416452in}}%
\pgfpathlineto{\pgfqpoint{1.472173in}{1.471250in}}%
\pgfpathlineto{\pgfqpoint{1.423850in}{1.531130in}}%
\pgfpathlineto{\pgfqpoint{1.381828in}{1.595509in}}%
\pgfpathlineto{\pgfqpoint{1.346536in}{1.663757in}}%
\pgfpathlineto{\pgfqpoint{1.318321in}{1.735203in}}%
\pgfpathlineto{\pgfqpoint{1.297451in}{1.809132in}}%
\pgfpathlineto{\pgfqpoint{1.284117in}{1.884786in}}%
\pgfpathlineto{\pgfqpoint{1.278428in}{1.961365in}}%
\pgfpathlineto{\pgfqpoint{1.280411in}{2.038097in}}%
\pgfpathlineto{\pgfqpoint{1.290043in}{2.114282in}}%
\pgfpathlineto{\pgfqpoint{1.307268in}{2.189117in}}%
\pgfpathlineto{\pgfqpoint{1.331946in}{2.261833in}}%
\pgfpathlineto{\pgfqpoint{1.363857in}{2.331702in}}%
\pgfpathlineto{\pgfqpoint{1.402695in}{2.398032in}}%
\pgfpathlineto{\pgfqpoint{1.448076in}{2.460174in}}%
\pgfpathlineto{\pgfqpoint{1.499529in}{2.517514in}}%
\pgfpathlineto{\pgfqpoint{1.556504in}{2.569480in}}%
\pgfpathlineto{\pgfqpoint{1.618366in}{2.615540in}}%
\pgfpathlineto{\pgfqpoint{1.684476in}{2.655251in}}%
\pgfpathlineto{\pgfqpoint{1.754247in}{2.688257in}}%
\pgfpathlineto{\pgfqpoint{1.826967in}{2.714177in}}%
\pgfpathlineto{\pgfqpoint{1.901912in}{2.732717in}}%
\pgfpathlineto{\pgfqpoint{1.978341in}{2.743672in}}%
\pgfpathlineto{\pgfqpoint{2.055507in}{2.746928in}}%
\pgfpathlineto{\pgfqpoint{2.132646in}{2.742460in}}%
\pgfpathlineto{\pgfqpoint{2.208985in}{2.730330in}}%
\pgfpathlineto{\pgfqpoint{2.283738in}{2.710693in}}%
\pgfpathlineto{\pgfqpoint{2.356108in}{2.683792in}}%
\pgfpathlineto{\pgfqpoint{2.425427in}{2.649895in}}%
\pgfpathlineto{\pgfqpoint{2.491016in}{2.609311in}}%
\pgfpathlineto{\pgfqpoint{2.552176in}{2.562429in}}%
\pgfpathlineto{\pgfqpoint{2.608278in}{2.509701in}}%
\pgfpathlineto{\pgfqpoint{2.658757in}{2.451637in}}%
\pgfpathlineto{\pgfqpoint{2.703116in}{2.388806in}}%
\pgfpathlineto{\pgfqpoint{2.740924in}{2.321842in}}%
\pgfpathlineto{\pgfqpoint{2.771818in}{2.251436in}}%
\pgfpathlineto{\pgfqpoint{2.795500in}{2.178339in}}%
\pgfpathlineto{\pgfqpoint{2.811760in}{2.103305in}}%
\pgfpathlineto{\pgfqpoint{2.820457in}{2.027012in}}%
\pgfpathlineto{\pgfqpoint{2.821458in}{1.950252in}}%
\pgfpathlineto{\pgfqpoint{2.814715in}{1.873807in}}%
\pgfpathlineto{\pgfqpoint{2.800270in}{1.798438in}}%
\pgfpathlineto{\pgfqpoint{2.778253in}{1.724886in}}%
\pgfpathlineto{\pgfqpoint{2.748886in}{1.653876in}}%
\pgfpathlineto{\pgfqpoint{2.712479in}{1.586111in}}%
\pgfpathlineto{\pgfqpoint{2.669430in}{1.522275in}}%
\pgfpathlineto{\pgfqpoint{2.620228in}{1.463033in}}%
\pgfpathlineto{\pgfqpoint{2.565393in}{1.408973in}}%
\pgfpathlineto{\pgfqpoint{2.505410in}{1.360576in}}%
\pgfpathlineto{\pgfqpoint{2.440879in}{1.318376in}}%
\pgfpathlineto{\pgfqpoint{2.372437in}{1.282826in}}%
\pgfpathlineto{\pgfqpoint{2.300756in}{1.254296in}}%
\pgfpathlineto{\pgfqpoint{2.226536in}{1.233073in}}%
\pgfpathlineto{\pgfqpoint{2.150515in}{1.219358in}}%
\pgfpathlineto{\pgfqpoint{2.073461in}{1.213270in}}%
\pgfpathlineto{\pgfqpoint{1.996175in}{1.214845in}}%
\pgfpathlineto{\pgfqpoint{1.919479in}{1.224033in}}%
\pgfpathlineto{\pgfqpoint{1.844060in}{1.240729in}}%
\pgfpathlineto{\pgfqpoint{1.770686in}{1.264795in}}%
\pgfpathlineto{\pgfqpoint{1.700126in}{1.296020in}}%
\pgfpathlineto{\pgfqpoint{1.633098in}{1.334119in}}%
\pgfpathlineto{\pgfqpoint{1.570272in}{1.378730in}}%
\pgfpathlineto{\pgfqpoint{1.512265in}{1.429416in}}%
\pgfpathlineto{\pgfqpoint{1.459644in}{1.485666in}}%
\pgfpathlineto{\pgfqpoint{1.412927in}{1.546889in}}%
\pgfpathlineto{\pgfqpoint{1.372582in}{1.612424in}}%
\pgfpathlineto{\pgfqpoint{1.339011in}{1.681553in}}%
\pgfpathlineto{\pgfqpoint{1.312493in}{1.753665in}}%
\pgfpathlineto{\pgfqpoint{1.293332in}{1.828036in}}%
\pgfpathlineto{\pgfqpoint{1.281762in}{1.903913in}}%
\pgfpathlineto{\pgfqpoint{1.277926in}{1.980542in}}%
\pgfpathlineto{\pgfqpoint{1.281874in}{2.057169in}}%
\pgfpathlineto{\pgfqpoint{1.293568in}{2.133042in}}%
\pgfpathlineto{\pgfqpoint{1.312876in}{2.207406in}}%
\pgfpathlineto{\pgfqpoint{1.339577in}{2.279509in}}%
\pgfpathlineto{\pgfqpoint{1.373356in}{2.348597in}}%
\pgfpathlineto{\pgfqpoint{1.413839in}{2.413962in}}%
\pgfpathlineto{\pgfqpoint{1.460667in}{2.475022in}}%
\pgfpathlineto{\pgfqpoint{1.513385in}{2.531128in}}%
\pgfpathlineto{\pgfqpoint{1.571476in}{2.581689in}}%
\pgfpathlineto{\pgfqpoint{1.634377in}{2.626190in}}%
\pgfpathlineto{\pgfqpoint{1.701470in}{2.664187in}}%
\pgfpathlineto{\pgfqpoint{1.772089in}{2.695311in}}%
\pgfpathlineto{\pgfqpoint{1.845514in}{2.719268in}}%
\pgfpathlineto{\pgfqpoint{1.920978in}{2.735834in}}%
\pgfpathlineto{\pgfqpoint{1.997684in}{2.744867in}}%
\pgfpathlineto{\pgfqpoint{2.074941in}{2.746307in}}%
\pgfpathlineto{\pgfqpoint{2.151952in}{2.740102in}}%
\pgfpathlineto{\pgfqpoint{2.227924in}{2.726275in}}%
\pgfpathlineto{\pgfqpoint{2.302094in}{2.704937in}}%
\pgfpathlineto{\pgfqpoint{2.373727in}{2.676282in}}%
\pgfpathlineto{\pgfqpoint{2.442120in}{2.640593in}}%
\pgfpathlineto{\pgfqpoint{2.506599in}{2.598235in}}%
\pgfpathlineto{\pgfqpoint{2.566520in}{2.549664in}}%
\pgfpathlineto{\pgfqpoint{2.621270in}{2.495417in}}%
\pgfpathlineto{\pgfqpoint{2.670285in}{2.436094in}}%
\pgfpathlineto{\pgfqpoint{2.713148in}{2.372220in}}%
\pgfpathlineto{\pgfqpoint{2.749391in}{2.304428in}}%
\pgfpathlineto{\pgfqpoint{2.778612in}{2.233394in}}%
\pgfpathlineto{\pgfqpoint{2.800497in}{2.159819in}}%
\pgfpathlineto{\pgfqpoint{2.814820in}{2.084424in}}%
\pgfpathlineto{\pgfqpoint{2.821443in}{2.007951in}}%
\pgfpathlineto{\pgfqpoint{2.820315in}{1.931166in}}%
\pgfpathlineto{\pgfqpoint{2.811475in}{1.854855in}}%
\pgfpathlineto{\pgfqpoint{2.795048in}{1.779829in}}%
\pgfpathlineto{\pgfqpoint{2.771223in}{1.706819in}}%
\pgfpathlineto{\pgfqpoint{2.740213in}{1.636501in}}%
\pgfpathlineto{\pgfqpoint{2.702302in}{1.569627in}}%
\pgfpathlineto{\pgfqpoint{2.657848in}{1.506887in}}%
\pgfpathlineto{\pgfqpoint{2.607274in}{1.448915in}}%
\pgfpathlineto{\pgfqpoint{2.551073in}{1.396284in}}%
\pgfpathlineto{\pgfqpoint{2.489805in}{1.349508in}}%
\pgfpathlineto{\pgfqpoint{2.424099in}{1.309044in}}%
\pgfpathlineto{\pgfqpoint{2.354655in}{1.275285in}}%
\pgfpathlineto{\pgfqpoint{2.282232in}{1.248567in}}%
\pgfpathlineto{\pgfqpoint{2.207495in}{1.229107in}}%
\pgfpathlineto{\pgfqpoint{2.131176in}{1.217123in}}%
\pgfpathlineto{\pgfqpoint{2.054059in}{1.212780in}}%
\pgfpathlineto{\pgfqpoint{1.976917in}{1.216152in}}%
\pgfpathlineto{\pgfqpoint{1.900510in}{1.227224in}}%
\pgfpathlineto{\pgfqpoint{1.825589in}{1.245890in}}%
\pgfpathlineto{\pgfqpoint{1.752895in}{1.271953in}}%
\pgfpathlineto{\pgfqpoint{1.683156in}{1.305129in}}%
\pgfpathlineto{\pgfqpoint{1.617090in}{1.345041in}}%
\pgfpathlineto{\pgfqpoint{1.555394in}{1.391232in}}%
\pgfpathlineto{\pgfqpoint{1.498610in}{1.443282in}}%
\pgfpathlineto{\pgfqpoint{1.447326in}{1.500691in}}%
\pgfpathlineto{\pgfqpoint{1.402093in}{1.562890in}}%
\pgfpathlineto{\pgfqpoint{1.363384in}{1.629273in}}%
\pgfpathlineto{\pgfqpoint{1.331589in}{1.699189in}}%
\pgfpathlineto{\pgfqpoint{1.307019in}{1.771949in}}%
\pgfpathlineto{\pgfqpoint{1.289906in}{1.846823in}}%
\pgfpathlineto{\pgfqpoint{1.280398in}{1.923038in}}%
\pgfpathlineto{\pgfqpoint{1.278568in}{1.999782in}}%
\pgfpathlineto{\pgfqpoint{1.284400in}{2.076318in}}%
\pgfpathlineto{\pgfqpoint{1.297852in}{2.151918in}}%
\pgfpathlineto{\pgfqpoint{1.318825in}{2.225786in}}%
\pgfpathlineto{\pgfqpoint{1.347141in}{2.297167in}}%
\pgfpathlineto{\pgfqpoint{1.382540in}{2.365345in}}%
\pgfpathlineto{\pgfqpoint{1.424682in}{2.429650in}}%
\pgfpathlineto{\pgfqpoint{1.473143in}{2.489449in}}%
\pgfpathlineto{\pgfqpoint{1.527421in}{2.544155in}}%
\pgfpathlineto{\pgfqpoint{1.586930in}{2.593219in}}%
\pgfpathlineto{\pgfqpoint{1.651003in}{2.636135in}}%
\pgfpathlineto{\pgfqpoint{1.719023in}{2.672519in}}%
\pgfpathlineto{\pgfqpoint{1.790356in}{2.702014in}}%
\pgfpathlineto{\pgfqpoint{1.864277in}{2.724278in}}%
\pgfpathlineto{\pgfqpoint{1.940050in}{2.739055in}}%
\pgfpathlineto{\pgfqpoint{2.016928in}{2.746181in}}%
\pgfpathlineto{\pgfqpoint{2.094157in}{2.745582in}}%
\pgfpathlineto{\pgfqpoint{2.170972in}{2.737274in}}%
\pgfpathlineto{\pgfqpoint{2.246601in}{2.721361in}}%
\pgfpathlineto{\pgfqpoint{2.320260in}{2.698039in}}%
\pgfpathlineto{\pgfqpoint{2.391168in}{2.667589in}}%
\pgfpathlineto{\pgfqpoint{2.458691in}{2.630294in}}%
\pgfpathlineto{\pgfqpoint{2.522139in}{2.586505in}}%
\pgfpathlineto{\pgfqpoint{2.580843in}{2.536643in}}%
\pgfpathlineto{\pgfqpoint{2.634199in}{2.481190in}}%
\pgfpathlineto{\pgfqpoint{2.681674in}{2.420686in}}%
\pgfpathlineto{\pgfqpoint{2.722802in}{2.355727in}}%
\pgfpathlineto{\pgfqpoint{2.757185in}{2.286970in}}%
\pgfpathlineto{\pgfqpoint{2.784494in}{2.215130in}}%
\pgfpathlineto{\pgfqpoint{2.804468in}{2.140979in}}%
\pgfpathlineto{\pgfqpoint{2.816947in}{2.065233in}}%
\pgfpathlineto{\pgfqpoint{2.821799in}{1.988615in}}%
\pgfpathlineto{\pgfqpoint{2.818930in}{1.911920in}}%
\pgfpathlineto{\pgfqpoint{2.808334in}{1.835923in}}%
\pgfpathlineto{\pgfqpoint{2.790095in}{1.761381in}}%
\pgfpathlineto{\pgfqpoint{2.764385in}{1.689024in}}%
\pgfpathlineto{\pgfqpoint{2.731465in}{1.619566in}}%
\pgfpathlineto{\pgfqpoint{2.691686in}{1.553697in}}%
\pgfpathlineto{\pgfqpoint{2.645487in}{1.492087in}}%
\pgfpathlineto{\pgfqpoint{2.593395in}{1.435384in}}%
\pgfpathlineto{\pgfqpoint{2.535917in}{1.384107in}}%
\pgfpathlineto{\pgfqpoint{2.473584in}{1.338755in}}%
\pgfpathlineto{\pgfqpoint{2.407021in}{1.299829in}}%
\pgfpathlineto{\pgfqpoint{2.336884in}{1.267744in}}%
\pgfpathlineto{\pgfqpoint{2.263858in}{1.242832in}}%
\pgfpathlineto{\pgfqpoint{2.188662in}{1.225341in}}%
\pgfpathlineto{\pgfqpoint{2.112044in}{1.215432in}}%
\pgfpathlineto{\pgfqpoint{2.034783in}{1.213184in}}%
\pgfpathlineto{\pgfqpoint{1.957689in}{1.218592in}}%
\pgfpathlineto{\pgfqpoint{1.881565in}{1.231566in}}%
\pgfpathlineto{\pgfqpoint{1.807084in}{1.251980in}}%
\pgfpathlineto{\pgfqpoint{1.735029in}{1.279660in}}%
\pgfpathlineto{\pgfqpoint{1.666151in}{1.314357in}}%
\pgfpathlineto{\pgfqpoint{1.601146in}{1.355748in}}%
\pgfpathlineto{\pgfqpoint{1.540661in}{1.403438in}}%
\pgfpathlineto{\pgfqpoint{1.485289in}{1.456955in}}%
\pgfpathlineto{\pgfqpoint{1.435573in}{1.515755in}}%
\pgfpathlineto{\pgfqpoint{1.392000in}{1.579216in}}%
\pgfpathlineto{\pgfqpoint{1.355009in}{1.646646in}}%
\pgfpathlineto{\pgfqpoint{1.324952in}{1.717338in}}%
\pgfpathlineto{\pgfqpoint{1.302090in}{1.790668in}}%
\pgfpathlineto{\pgfqpoint{1.286703in}{1.865883in}}%
\pgfpathlineto{\pgfqpoint{1.278981in}{1.942224in}}%
\pgfpathlineto{\pgfqpoint{1.279026in}{2.018936in}}%
\pgfpathlineto{\pgfqpoint{1.286846in}{2.095265in}}%
\pgfpathlineto{\pgfqpoint{1.302361in}{2.170460in}}%
\pgfpathlineto{\pgfqpoint{1.325396in}{2.243770in}}%
\pgfpathlineto{\pgfqpoint{1.355689in}{2.314450in}}%
\pgfpathlineto{\pgfqpoint{1.392883in}{2.381752in}}%
\pgfpathlineto{\pgfqpoint{1.436592in}{2.445022in}}%
\pgfpathlineto{\pgfqpoint{1.486427in}{2.503674in}}%
\pgfpathlineto{\pgfqpoint{1.541895in}{2.557075in}}%
\pgfpathlineto{\pgfqpoint{1.602455in}{2.604668in}}%
\pgfpathlineto{\pgfqpoint{1.667517in}{2.645968in}}%
\pgfpathlineto{\pgfqpoint{1.736442in}{2.680569in}}%
\pgfpathlineto{\pgfqpoint{1.808542in}{2.708134in}}%
\pgfpathlineto{\pgfqpoint{1.883078in}{2.728407in}}%
\pgfpathlineto{\pgfqpoint{1.959264in}{2.741203in}}%
\pgfpathlineto{\pgfqpoint{2.036319in}{2.746422in}}%
\pgfpathlineto{\pgfqpoint{2.113544in}{2.744028in}}%
\pgfpathlineto{\pgfqpoint{2.190131in}{2.734002in}}%
\pgfpathlineto{\pgfqpoint{2.265294in}{2.716409in}}%
\pgfpathlineto{\pgfqpoint{2.338280in}{2.691398in}}%
\pgfpathlineto{\pgfqpoint{2.408368in}{2.659205in}}%
\pgfpathlineto{\pgfqpoint{2.474871in}{2.620148in}}%
\pgfpathlineto{\pgfqpoint{2.537132in}{2.574633in}}%
\pgfpathlineto{\pgfqpoint{2.594527in}{2.523149in}}%
\pgfpathlineto{\pgfqpoint{2.646466in}{2.466270in}}%
\pgfpathlineto{\pgfqpoint{2.692433in}{2.404602in}}%
\pgfpathlineto{\pgfqpoint{2.732027in}{2.338687in}}%
\pgfpathlineto{\pgfqpoint{2.764803in}{2.269191in}}%
\pgfpathlineto{\pgfqpoint{2.790396in}{2.196805in}}%
\pgfpathlineto{\pgfqpoint{2.808533in}{2.122241in}}%
\pgfpathlineto{\pgfqpoint{2.819025in}{2.046231in}}%
\pgfpathlineto{\pgfqpoint{2.821777in}{1.969526in}}%
\pgfpathlineto{\pgfqpoint{2.816778in}{1.892895in}}%
\pgfpathlineto{\pgfqpoint{2.804109in}{1.817129in}}%
\pgfpathlineto{\pgfqpoint{2.783937in}{1.743035in}}%
\pgfpathlineto{\pgfqpoint{2.756474in}{1.671300in}}%
\pgfpathlineto{\pgfqpoint{2.721966in}{1.602624in}}%
\pgfpathlineto{\pgfqpoint{2.680736in}{1.537736in}}%
\pgfpathlineto{\pgfqpoint{2.633175in}{1.477305in}}%
\pgfpathlineto{\pgfqpoint{2.579739in}{1.421937in}}%
\pgfpathlineto{\pgfqpoint{2.520951in}{1.372180in}}%
\pgfpathlineto{\pgfqpoint{2.457402in}{1.328517in}}%
\pgfpathlineto{\pgfqpoint{2.389749in}{1.291373in}}%
\pgfpathlineto{\pgfqpoint{2.318713in}{1.261111in}}%
\pgfpathlineto{\pgfqpoint{2.245059in}{1.238020in}}%
\pgfpathlineto{\pgfqpoint{2.169442in}{1.222288in}}%
\pgfpathlineto{\pgfqpoint{2.092630in}{1.214113in}}%
\pgfpathlineto{\pgfqpoint{2.015409in}{1.213618in}}%
\pgfpathlineto{\pgfqpoint{1.938547in}{1.220836in}}%
\pgfpathlineto{\pgfqpoint{1.862803in}{1.235709in}}%
\pgfpathlineto{\pgfqpoint{1.788921in}{1.258090in}}%
\pgfpathlineto{\pgfqpoint{1.717632in}{1.287744in}}%
\pgfpathlineto{\pgfqpoint{1.649655in}{1.324344in}}%
\pgfpathlineto{\pgfqpoint{1.585695in}{1.367473in}}%
\pgfpathlineto{\pgfqpoint{1.526409in}{1.416653in}}%
\pgfpathlineto{\pgfqpoint{1.472316in}{1.471451in}}%
\pgfpathlineto{\pgfqpoint{1.423997in}{1.531326in}}%
\pgfpathlineto{\pgfqpoint{1.381968in}{1.595688in}}%
\pgfpathlineto{\pgfqpoint{1.346668in}{1.663908in}}%
\pgfpathlineto{\pgfqpoint{1.318449in}{1.735317in}}%
\pgfpathlineto{\pgfqpoint{1.297586in}{1.809209in}}%
\pgfpathlineto{\pgfqpoint{1.284270in}{1.884839in}}%
\pgfpathlineto{\pgfqpoint{1.278612in}{1.961421in}}%
\pgfpathlineto{\pgfqpoint{1.280641in}{2.038138in}}%
\pgfpathlineto{\pgfqpoint{1.290309in}{2.114289in}}%
\pgfpathlineto{\pgfqpoint{1.307546in}{2.189114in}}%
\pgfpathlineto{\pgfqpoint{1.332215in}{2.261829in}}%
\pgfpathlineto{\pgfqpoint{1.364098in}{2.331693in}}%
\pgfpathlineto{\pgfqpoint{1.402897in}{2.398011in}}%
\pgfpathlineto{\pgfqpoint{1.448235in}{2.460128in}}%
\pgfpathlineto{\pgfqpoint{1.499654in}{2.517437in}}%
\pgfpathlineto{\pgfqpoint{1.556617in}{2.569371in}}%
\pgfpathlineto{\pgfqpoint{1.618506in}{2.615409in}}%
\pgfpathlineto{\pgfqpoint{1.684631in}{2.655080in}}%
\pgfpathlineto{\pgfqpoint{1.754392in}{2.688046in}}%
\pgfpathlineto{\pgfqpoint{1.827108in}{2.713957in}}%
\pgfpathlineto{\pgfqpoint{1.902044in}{2.732507in}}%
\pgfpathlineto{\pgfqpoint{1.978454in}{2.743484in}}%
\pgfpathlineto{\pgfqpoint{2.055587in}{2.746764in}}%
\pgfpathlineto{\pgfqpoint{2.132686in}{2.742312in}}%
\pgfpathlineto{\pgfqpoint{2.208985in}{2.730186in}}%
\pgfpathlineto{\pgfqpoint{2.283713in}{2.710533in}}%
\pgfpathlineto{\pgfqpoint{2.356092in}{2.683588in}}%
\pgfpathlineto{\pgfqpoint{2.425365in}{2.649665in}}%
\pgfpathlineto{\pgfqpoint{2.490919in}{2.609067in}}%
\pgfpathlineto{\pgfqpoint{2.552068in}{2.562182in}}%
\pgfpathlineto{\pgfqpoint{2.608167in}{2.509466in}}%
\pgfpathlineto{\pgfqpoint{2.658644in}{2.451428in}}%
\pgfpathlineto{\pgfqpoint{2.702997in}{2.388636in}}%
\pgfpathlineto{\pgfqpoint{2.740792in}{2.321713in}}%
\pgfpathlineto{\pgfqpoint{2.771666in}{2.251338in}}%
\pgfpathlineto{\pgfqpoint{2.795325in}{2.178249in}}%
\pgfpathlineto{\pgfqpoint{2.811548in}{2.103226in}}%
\pgfpathlineto{\pgfqpoint{2.820215in}{2.026956in}}%
\pgfpathlineto{\pgfqpoint{2.821207in}{1.950206in}}%
\pgfpathlineto{\pgfqpoint{2.814473in}{1.873769in}}%
\pgfpathlineto{\pgfqpoint{2.800050in}{1.798412in}}%
\pgfpathlineto{\pgfqpoint{2.778062in}{1.724882in}}%
\pgfpathlineto{\pgfqpoint{2.748723in}{1.653901in}}%
\pgfpathlineto{\pgfqpoint{2.712334in}{1.586169in}}%
\pgfpathlineto{\pgfqpoint{2.669287in}{1.522362in}}%
\pgfpathlineto{\pgfqpoint{2.620061in}{1.463134in}}%
\pgfpathlineto{\pgfqpoint{2.565210in}{1.409103in}}%
\pgfpathlineto{\pgfqpoint{2.505231in}{1.360739in}}%
\pgfpathlineto{\pgfqpoint{2.440706in}{1.318548in}}%
\pgfpathlineto{\pgfqpoint{2.372278in}{1.282994in}}%
\pgfpathlineto{\pgfqpoint{2.300620in}{1.254455in}}%
\pgfpathlineto{\pgfqpoint{2.226433in}{1.233223in}}%
\pgfpathlineto{\pgfqpoint{2.150448in}{1.219507in}}%
\pgfpathlineto{\pgfqpoint{2.073426in}{1.213427in}}%
\pgfpathlineto{\pgfqpoint{1.996154in}{1.215023in}}%
\pgfpathlineto{\pgfqpoint{1.919451in}{1.224245in}}%
\pgfpathlineto{\pgfqpoint{1.844080in}{1.240972in}}%
\pgfpathlineto{\pgfqpoint{1.770727in}{1.265053in}}%
\pgfpathlineto{\pgfqpoint{1.700174in}{1.296278in}}%
\pgfpathlineto{\pgfqpoint{1.633151in}{1.334359in}}%
\pgfpathlineto{\pgfqpoint{1.570332in}{1.378940in}}%
\pgfpathlineto{\pgfqpoint{1.512339in}{1.429589in}}%
\pgfpathlineto{\pgfqpoint{1.459741in}{1.485802in}}%
\pgfpathlineto{\pgfqpoint{1.413052in}{1.547000in}}%
\pgfpathlineto{\pgfqpoint{1.372731in}{1.612535in}}%
\pgfpathlineto{\pgfqpoint{1.339186in}{1.681682in}}%
\pgfpathlineto{\pgfqpoint{1.312709in}{1.753773in}}%
\pgfpathlineto{\pgfqpoint{1.293564in}{1.828135in}}%
\pgfpathlineto{\pgfqpoint{1.281990in}{1.904004in}}%
\pgfpathlineto{\pgfqpoint{1.278137in}{1.980618in}}%
\pgfpathlineto{\pgfqpoint{1.282064in}{2.057221in}}%
\pgfpathlineto{\pgfqpoint{1.293736in}{2.133061in}}%
\pgfpathlineto{\pgfqpoint{1.313031in}{2.207389in}}%
\pgfpathlineto{\pgfqpoint{1.339733in}{2.279462in}}%
\pgfpathlineto{\pgfqpoint{1.373536in}{2.348538in}}%
\pgfpathlineto{\pgfqpoint{1.414046in}{2.413885in}}%
\pgfpathlineto{\pgfqpoint{1.460877in}{2.474910in}}%
\pgfpathlineto{\pgfqpoint{1.513591in}{2.531004in}}%
\pgfpathlineto{\pgfqpoint{1.571670in}{2.581566in}}%
\pgfpathlineto{\pgfqpoint{1.634547in}{2.626070in}}%
\pgfpathlineto{\pgfqpoint{1.701609in}{2.664067in}}%
\pgfpathlineto{\pgfqpoint{1.772193in}{2.695184in}}%
\pgfpathlineto{\pgfqpoint{1.845592in}{2.719123in}}%
\pgfpathlineto{\pgfqpoint{1.921049in}{2.735664in}}%
\pgfpathlineto{\pgfqpoint{1.997762in}{2.744661in}}%
\pgfpathlineto{\pgfqpoint{2.074980in}{2.746058in}}%
\pgfpathlineto{\pgfqpoint{2.151978in}{2.739835in}}%
\pgfpathlineto{\pgfqpoint{2.227946in}{2.726013in}}%
\pgfpathlineto{\pgfqpoint{2.302110in}{2.704697in}}%
\pgfpathlineto{\pgfqpoint{2.373729in}{2.676076in}}%
\pgfpathlineto{\pgfqpoint{2.442097in}{2.640424in}}%
\pgfpathlineto{\pgfqpoint{2.506544in}{2.598099in}}%
\pgfpathlineto{\pgfqpoint{2.566431in}{2.549541in}}%
\pgfpathlineto{\pgfqpoint{2.621157in}{2.495278in}}%
\pgfpathlineto{\pgfqpoint{2.670154in}{2.435919in}}%
\pgfpathlineto{\pgfqpoint{2.712967in}{2.372054in}}%
\pgfpathlineto{\pgfqpoint{2.749192in}{2.304265in}}%
\pgfpathlineto{\pgfqpoint{2.778417in}{2.233239in}}%
\pgfpathlineto{\pgfqpoint{2.800318in}{2.159683in}}%
\pgfpathlineto{\pgfqpoint{2.814659in}{2.084317in}}%
\pgfpathlineto{\pgfqpoint{2.821295in}{2.007883in}}%
\pgfpathlineto{\pgfqpoint{2.820170in}{1.931138in}}%
\pgfpathlineto{\pgfqpoint{2.811315in}{1.854854in}}%
\pgfpathlineto{\pgfqpoint{2.794853in}{1.779824in}}%
\pgfpathlineto{\pgfqpoint{2.770990in}{1.706844in}}%
\pgfpathlineto{\pgfqpoint{2.739957in}{1.636565in}}%
\pgfpathlineto{\pgfqpoint{2.702039in}{1.569704in}}%
\pgfpathlineto{\pgfqpoint{2.657594in}{1.506967in}}%
\pgfpathlineto{\pgfqpoint{2.607045in}{1.448996in}}%
\pgfpathlineto{\pgfqpoint{2.550882in}{1.396372in}}%
\pgfpathlineto{\pgfqpoint{2.489656in}{1.349612in}}%
\pgfpathlineto{\pgfqpoint{2.423987in}{1.309170in}}%
\pgfpathlineto{\pgfqpoint{2.354558in}{1.275437in}}%
\pgfpathlineto{\pgfqpoint{2.282116in}{1.248744in}}%
\pgfpathlineto{\pgfqpoint{2.207406in}{1.229330in}}%
\pgfpathlineto{\pgfqpoint{2.131100in}{1.217367in}}%
\pgfpathlineto{\pgfqpoint{2.053990in}{1.213023in}}%
\pgfpathlineto{\pgfqpoint{1.976858in}{1.216379in}}%
\pgfpathlineto{\pgfqpoint{1.900471in}{1.227425in}}%
\pgfpathlineto{\pgfqpoint{1.825579in}{1.246063in}}%
\pgfpathlineto{\pgfqpoint{1.752920in}{1.272104in}}%
\pgfpathlineto{\pgfqpoint{1.683215in}{1.305272in}}%
\pgfpathlineto{\pgfqpoint{1.617170in}{1.345199in}}%
\pgfpathlineto{\pgfqpoint{1.555478in}{1.391429in}}%
\pgfpathlineto{\pgfqpoint{1.498742in}{1.443478in}}%
\pgfpathlineto{\pgfqpoint{1.447477in}{1.500885in}}%
\pgfpathlineto{\pgfqpoint{1.402245in}{1.563076in}}%
\pgfpathlineto{\pgfqpoint{1.363528in}{1.629439in}}%
\pgfpathlineto{\pgfqpoint{1.331725in}{1.699325in}}%
\pgfpathlineto{\pgfqpoint{1.307154in}{1.772049in}}%
\pgfpathlineto{\pgfqpoint{1.290049in}{1.846888in}}%
\pgfpathlineto{\pgfqpoint{1.280561in}{1.923082in}}%
\pgfpathlineto{\pgfqpoint{1.278762in}{1.999836in}}%
\pgfpathlineto{\pgfqpoint{1.284639in}{2.076340in}}%
\pgfpathlineto{\pgfqpoint{1.298119in}{2.151912in}}%
\pgfpathlineto{\pgfqpoint{1.319098in}{2.225772in}}%
\pgfpathlineto{\pgfqpoint{1.347401in}{2.297150in}}%
\pgfpathlineto{\pgfqpoint{1.382771in}{2.365323in}}%
\pgfpathlineto{\pgfqpoint{1.424873in}{2.429613in}}%
\pgfpathlineto{\pgfqpoint{1.473294in}{2.489388in}}%
\pgfpathlineto{\pgfqpoint{1.527542in}{2.544063in}}%
\pgfpathlineto{\pgfqpoint{1.587044in}{2.593097in}}%
\pgfpathlineto{\pgfqpoint{1.651151in}{2.635996in}}%
\pgfpathlineto{\pgfqpoint{1.719166in}{2.672331in}}%
\pgfpathlineto{\pgfqpoint{1.790490in}{2.701797in}}%
\pgfpathlineto{\pgfqpoint{1.864406in}{2.724056in}}%
\pgfpathlineto{\pgfqpoint{1.940167in}{2.738846in}}%
\pgfpathlineto{\pgfqpoint{2.017023in}{2.745994in}}%
\pgfpathlineto{\pgfqpoint{2.094218in}{2.745418in}}%
\pgfpathlineto{\pgfqpoint{2.170994in}{2.737124in}}%
\pgfpathlineto{\pgfqpoint{2.246587in}{2.721213in}}%
\pgfpathlineto{\pgfqpoint{2.320226in}{2.697871in}}%
\pgfpathlineto{\pgfqpoint{2.391140in}{2.667378in}}%
\pgfpathlineto{\pgfqpoint{2.458611in}{2.630068in}}%
\pgfpathlineto{\pgfqpoint{2.522032in}{2.586269in}}%
\pgfpathlineto{\pgfqpoint{2.580725in}{2.536408in}}%
\pgfpathlineto{\pgfqpoint{2.634079in}{2.480970in}}%
\pgfpathlineto{\pgfqpoint{2.681551in}{2.420494in}}%
\pgfpathlineto{\pgfqpoint{2.722671in}{2.355572in}}%
\pgfpathlineto{\pgfqpoint{2.757041in}{2.286853in}}%
\pgfpathlineto{\pgfqpoint{2.784330in}{2.215040in}}%
\pgfpathlineto{\pgfqpoint{2.804283in}{2.140889in}}%
\pgfpathlineto{\pgfqpoint{2.816725in}{2.065170in}}%
\pgfpathlineto{\pgfqpoint{2.821556in}{1.988570in}}%
\pgfpathlineto{\pgfqpoint{2.818684in}{1.911884in}}%
\pgfpathlineto{\pgfqpoint{2.808100in}{1.835898in}}%
\pgfpathlineto{\pgfqpoint{2.789883in}{1.761371in}}%
\pgfpathlineto{\pgfqpoint{2.764200in}{1.689037in}}%
\pgfpathlineto{\pgfqpoint{2.731305in}{1.619609in}}%
\pgfpathlineto{\pgfqpoint{2.691541in}{1.553771in}}%
\pgfpathlineto{\pgfqpoint{2.645340in}{1.492185in}}%
\pgfpathlineto{\pgfqpoint{2.593219in}{1.435490in}}%
\pgfpathlineto{\pgfqpoint{2.535743in}{1.384254in}}%
\pgfpathlineto{\pgfqpoint{2.473416in}{1.338926in}}%
\pgfpathlineto{\pgfqpoint{2.406861in}{1.300004in}}%
\pgfpathlineto{\pgfqpoint{2.336739in}{1.267914in}}%
\pgfpathlineto{\pgfqpoint{2.263739in}{1.242994in}}%
\pgfpathlineto{\pgfqpoint{2.188576in}{1.225496in}}%
\pgfpathlineto{\pgfqpoint{2.111993in}{1.215588in}}%
\pgfpathlineto{\pgfqpoint{2.034759in}{1.213351in}}%
\pgfpathlineto{\pgfqpoint{1.957674in}{1.218780in}}%
\pgfpathlineto{\pgfqpoint{1.881561in}{1.231786in}}%
\pgfpathlineto{\pgfqpoint{1.807119in}{1.252223in}}%
\pgfpathlineto{\pgfqpoint{1.735081in}{1.279911in}}%
\pgfpathlineto{\pgfqpoint{1.666210in}{1.314602in}}%
\pgfpathlineto{\pgfqpoint{1.601210in}{1.355974in}}%
\pgfpathlineto{\pgfqpoint{1.540735in}{1.403634in}}%
\pgfpathlineto{\pgfqpoint{1.485380in}{1.457115in}}%
\pgfpathlineto{\pgfqpoint{1.435686in}{1.515882in}}%
\pgfpathlineto{\pgfqpoint{1.392138in}{1.579324in}}%
\pgfpathlineto{\pgfqpoint{1.355169in}{1.646760in}}%
\pgfpathlineto{\pgfqpoint{1.325144in}{1.717451in}}%
\pgfpathlineto{\pgfqpoint{1.302312in}{1.790764in}}%
\pgfpathlineto{\pgfqpoint{1.286933in}{1.865969in}}%
\pgfpathlineto{\pgfqpoint{1.279205in}{1.942300in}}%
\pgfpathlineto{\pgfqpoint{1.279233in}{2.018994in}}%
\pgfpathlineto{\pgfqpoint{1.287033in}{2.095297in}}%
\pgfpathlineto{\pgfqpoint{1.302529in}{2.170460in}}%
\pgfpathlineto{\pgfqpoint{1.325555in}{2.243737in}}%
\pgfpathlineto{\pgfqpoint{1.355852in}{2.314390in}}%
\pgfpathlineto{\pgfqpoint{1.393073in}{2.381687in}}%
\pgfpathlineto{\pgfqpoint{1.436794in}{2.444925in}}%
\pgfpathlineto{\pgfqpoint{1.486628in}{2.503551in}}%
\pgfpathlineto{\pgfqpoint{1.542090in}{2.556944in}}%
\pgfpathlineto{\pgfqpoint{1.602634in}{2.604537in}}%
\pgfpathlineto{\pgfqpoint{1.667671in}{2.645839in}}%
\pgfpathlineto{\pgfqpoint{1.736565in}{2.680437in}}%
\pgfpathlineto{\pgfqpoint{1.808632in}{2.707994in}}%
\pgfpathlineto{\pgfqpoint{1.883146in}{2.728250in}}%
\pgfpathlineto{\pgfqpoint{1.959331in}{2.741021in}}%
\pgfpathlineto{\pgfqpoint{2.036373in}{2.746203in}}%
\pgfpathlineto{\pgfqpoint{2.113569in}{2.743778in}}%
\pgfpathlineto{\pgfqpoint{2.190144in}{2.733741in}}%
\pgfpathlineto{\pgfqpoint{2.265301in}{2.716157in}}%
\pgfpathlineto{\pgfqpoint{2.338278in}{2.691170in}}%
\pgfpathlineto{\pgfqpoint{2.408350in}{2.659009in}}%
\pgfpathlineto{\pgfqpoint{2.474827in}{2.619986in}}%
\pgfpathlineto{\pgfqpoint{2.537057in}{2.574498in}}%
\pgfpathlineto{\pgfqpoint{2.594422in}{2.523023in}}%
\pgfpathlineto{\pgfqpoint{2.646342in}{2.466124in}}%
\pgfpathlineto{\pgfqpoint{2.692279in}{2.404440in}}%
\pgfpathlineto{\pgfqpoint{2.731837in}{2.338534in}}%
\pgfpathlineto{\pgfqpoint{2.764601in}{2.269043in}}%
\pgfpathlineto{\pgfqpoint{2.790201in}{2.196668in}}%
\pgfpathlineto{\pgfqpoint{2.808351in}{2.122125in}}%
\pgfpathlineto{\pgfqpoint{2.818860in}{2.046146in}}%
\pgfpathlineto{\pgfqpoint{2.821622in}{1.969478in}}%
\pgfpathlineto{\pgfqpoint{2.816623in}{1.892883in}}%
\pgfpathlineto{\pgfqpoint{2.803938in}{1.817139in}}%
\pgfpathlineto{\pgfqpoint{2.783731in}{1.743038in}}%
\pgfpathlineto{\pgfqpoint{2.756242in}{1.671347in}}%
\pgfpathlineto{\pgfqpoint{2.721719in}{1.602701in}}%
\pgfpathlineto{\pgfqpoint{2.680487in}{1.537825in}}%
\pgfpathlineto{\pgfqpoint{2.632939in}{1.477397in}}%
\pgfpathlineto{\pgfqpoint{2.579529in}{1.422034in}}%
\pgfpathlineto{\pgfqpoint{2.520778in}{1.372285in}}%
\pgfpathlineto{\pgfqpoint{2.457268in}{1.328638in}}%
\pgfpathlineto{\pgfqpoint{2.389644in}{1.291515in}}%
\pgfpathlineto{\pgfqpoint{2.318618in}{1.261276in}}%
\pgfpathlineto{\pgfqpoint{2.244961in}{1.238213in}}%
\pgfpathlineto{\pgfqpoint{2.169368in}{1.222516in}}%
\pgfpathlineto{\pgfqpoint{2.092568in}{1.214354in}}%
\pgfpathlineto{\pgfqpoint{2.015356in}{1.213854in}}%
\pgfpathlineto{\pgfqpoint{1.938508in}{1.221054in}}%
\pgfpathlineto{\pgfqpoint{1.862785in}{1.235904in}}%
\pgfpathlineto{\pgfqpoint{1.788932in}{1.258261in}}%
\pgfpathlineto{\pgfqpoint{1.717677in}{1.287897in}}%
\pgfpathlineto{\pgfqpoint{1.649729in}{1.324493in}}%
\pgfpathlineto{\pgfqpoint{1.585785in}{1.367640in}}%
\pgfpathlineto{\pgfqpoint{1.526518in}{1.416842in}}%
\pgfpathlineto{\pgfqpoint{1.472461in}{1.471637in}}%
\pgfpathlineto{\pgfqpoint{1.424155in}{1.531507in}}%
\pgfpathlineto{\pgfqpoint{1.382126in}{1.595857in}}%
\pgfpathlineto{\pgfqpoint{1.346819in}{1.664055in}}%
\pgfpathlineto{\pgfqpoint{1.318596in}{1.735434in}}%
\pgfpathlineto{\pgfqpoint{1.297733in}{1.809291in}}%
\pgfpathlineto{\pgfqpoint{1.284427in}{1.884890in}}%
\pgfpathlineto{\pgfqpoint{1.278788in}{1.961457in}}%
\pgfpathlineto{\pgfqpoint{1.280847in}{2.038183in}}%
\pgfpathlineto{\pgfqpoint{1.290550in}{2.114290in}}%
\pgfpathlineto{\pgfqpoint{1.307806in}{2.189093in}}%
\pgfpathlineto{\pgfqpoint{1.332475in}{2.261800in}}%
\pgfpathlineto{\pgfqpoint{1.364342in}{2.331660in}}%
\pgfpathlineto{\pgfqpoint{1.403112in}{2.397969in}}%
\pgfpathlineto{\pgfqpoint{1.448414in}{2.460070in}}%
\pgfpathlineto{\pgfqpoint{1.499798in}{2.517354in}}%
\pgfpathlineto{\pgfqpoint{1.556736in}{2.569260in}}%
\pgfpathlineto{\pgfqpoint{1.618624in}{2.615273in}}%
\pgfpathlineto{\pgfqpoint{1.684777in}{2.654926in}}%
\pgfpathlineto{\pgfqpoint{1.754519in}{2.687847in}}%
\pgfpathlineto{\pgfqpoint{1.827226in}{2.713738in}}%
\pgfpathlineto{\pgfqpoint{1.902153in}{2.732289in}}%
\pgfpathlineto{\pgfqpoint{1.978548in}{2.743279in}}%
\pgfpathlineto{\pgfqpoint{2.055658in}{2.746577in}}%
\pgfpathlineto{\pgfqpoint{2.132724in}{2.742144in}}%
\pgfpathlineto{\pgfqpoint{2.208986in}{2.730029in}}%
\pgfpathlineto{\pgfqpoint{2.283683in}{2.710374in}}%
\pgfpathlineto{\pgfqpoint{2.356048in}{2.683409in}}%
\pgfpathlineto{\pgfqpoint{2.425312in}{2.649454in}}%
\pgfpathlineto{\pgfqpoint{2.490824in}{2.608849in}}%
\pgfpathlineto{\pgfqpoint{2.551953in}{2.561962in}}%
\pgfpathlineto{\pgfqpoint{2.608045in}{2.509254in}}%
\pgfpathlineto{\pgfqpoint{2.658520in}{2.451235in}}%
\pgfpathlineto{\pgfqpoint{2.702869in}{2.388472in}}%
\pgfpathlineto{\pgfqpoint{2.740655in}{2.321583in}}%
\pgfpathlineto{\pgfqpoint{2.771514in}{2.251239in}}%
\pgfpathlineto{\pgfqpoint{2.795153in}{2.178167in}}%
\pgfpathlineto{\pgfqpoint{2.811351in}{2.103143in}}%
\pgfpathlineto{\pgfqpoint{2.819983in}{2.026904in}}%
\pgfpathlineto{\pgfqpoint{2.820961in}{1.950170in}}%
\pgfpathlineto{\pgfqpoint{2.814230in}{1.873742in}}%
\pgfpathlineto{\pgfqpoint{2.799822in}{1.798396in}}%
\pgfpathlineto{\pgfqpoint{2.777858in}{1.724884in}}%
\pgfpathlineto{\pgfqpoint{2.748545in}{1.653928in}}%
\pgfpathlineto{\pgfqpoint{2.712180in}{1.586225in}}%
\pgfpathlineto{\pgfqpoint{2.669145in}{1.522448in}}%
\pgfpathlineto{\pgfqpoint{2.619910in}{1.463240in}}%
\pgfpathlineto{\pgfqpoint{2.565033in}{1.409222in}}%
\pgfpathlineto{\pgfqpoint{2.505063in}{1.360898in}}%
\pgfpathlineto{\pgfqpoint{2.440544in}{1.318725in}}%
\pgfpathlineto{\pgfqpoint{2.372126in}{1.283172in}}%
\pgfpathlineto{\pgfqpoint{2.300487in}{1.254626in}}%
\pgfpathlineto{\pgfqpoint{2.226328in}{1.233386in}}%
\pgfpathlineto{\pgfqpoint{2.150377in}{1.219664in}}%
\pgfpathlineto{\pgfqpoint{2.073388in}{1.213587in}}%
\pgfpathlineto{\pgfqpoint{1.996140in}{1.215195in}}%
\pgfpathlineto{\pgfqpoint{1.919439in}{1.224441in}}%
\pgfpathlineto{\pgfqpoint{1.844095in}{1.241197in}}%
\pgfpathlineto{\pgfqpoint{1.770774in}{1.265296in}}%
\pgfpathlineto{\pgfqpoint{1.700235in}{1.296525in}}%
\pgfpathlineto{\pgfqpoint{1.633218in}{1.334597in}}%
\pgfpathlineto{\pgfqpoint{1.570405in}{1.379156in}}%
\pgfpathlineto{\pgfqpoint{1.512424in}{1.429773in}}%
\pgfpathlineto{\pgfqpoint{1.459843in}{1.485951in}}%
\pgfpathlineto{\pgfqpoint{1.413176in}{1.547119in}}%
\pgfpathlineto{\pgfqpoint{1.372880in}{1.612640in}}%
\pgfpathlineto{\pgfqpoint{1.339353in}{1.681801in}}%
\pgfpathlineto{\pgfqpoint{1.312915in}{1.753874in}}%
\pgfpathlineto{\pgfqpoint{1.293791in}{1.828222in}}%
\pgfpathlineto{\pgfqpoint{1.282220in}{1.904080in}}%
\pgfpathlineto{\pgfqpoint{1.278357in}{1.980682in}}%
\pgfpathlineto{\pgfqpoint{1.282266in}{2.057265in}}%
\pgfpathlineto{\pgfqpoint{1.293918in}{2.133078in}}%
\pgfpathlineto{\pgfqpoint{1.313197in}{2.207374in}}%
\pgfpathlineto{\pgfqpoint{1.339892in}{2.279415in}}%
\pgfpathlineto{\pgfqpoint{1.373705in}{2.348470in}}%
\pgfpathlineto{\pgfqpoint{1.414243in}{2.413814in}}%
\pgfpathlineto{\pgfqpoint{1.461075in}{2.474798in}}%
\pgfpathlineto{\pgfqpoint{1.513787in}{2.530873in}}%
\pgfpathlineto{\pgfqpoint{1.571857in}{2.581429in}}%
\pgfpathlineto{\pgfqpoint{1.634716in}{2.625934in}}%
\pgfpathlineto{\pgfqpoint{1.701750in}{2.663932in}}%
\pgfpathlineto{\pgfqpoint{1.772302in}{2.695046in}}%
\pgfpathlineto{\pgfqpoint{1.845671in}{2.718975in}}%
\pgfpathlineto{\pgfqpoint{1.921109in}{2.735498in}}%
\pgfpathlineto{\pgfqpoint{1.997826in}{2.744470in}}%
\pgfpathlineto{\pgfqpoint{2.075017in}{2.745829in}}%
\pgfpathlineto{\pgfqpoint{2.151991in}{2.739582in}}%
\pgfpathlineto{\pgfqpoint{2.227950in}{2.725756in}}%
\pgfpathlineto{\pgfqpoint{2.302107in}{2.704452in}}%
\pgfpathlineto{\pgfqpoint{2.373715in}{2.675856in}}%
\pgfpathlineto{\pgfqpoint{2.442065in}{2.640237in}}%
\pgfpathlineto{\pgfqpoint{2.506486in}{2.597944in}}%
\pgfpathlineto{\pgfqpoint{2.566342in}{2.549410in}}%
\pgfpathlineto{\pgfqpoint{2.621041in}{2.495151in}}%
\pgfpathlineto{\pgfqpoint{2.670024in}{2.435765in}}%
\pgfpathlineto{\pgfqpoint{2.712796in}{2.371902in}}%
\pgfpathlineto{\pgfqpoint{2.748994in}{2.304121in}}%
\pgfpathlineto{\pgfqpoint{2.778213in}{2.233102in}}%
\pgfpathlineto{\pgfqpoint{2.800122in}{2.159559in}}%
\pgfpathlineto{\pgfqpoint{2.814478in}{2.084217in}}%
\pgfpathlineto{\pgfqpoint{2.821129in}{2.007815in}}%
\pgfpathlineto{\pgfqpoint{2.820012in}{1.931106in}}%
\pgfpathlineto{\pgfqpoint{2.811154in}{1.854855in}}%
\pgfpathlineto{\pgfqpoint{2.794674in}{1.779842in}}%
\pgfpathlineto{\pgfqpoint{2.770777in}{1.706859in}}%
\pgfpathlineto{\pgfqpoint{2.739726in}{1.636628in}}%
\pgfpathlineto{\pgfqpoint{2.701798in}{1.569791in}}%
\pgfpathlineto{\pgfqpoint{2.657355in}{1.507064in}}%
\pgfpathlineto{\pgfqpoint{2.606822in}{1.449098in}}%
\pgfpathlineto{\pgfqpoint{2.550686in}{1.396479in}}%
\pgfpathlineto{\pgfqpoint{2.489497in}{1.349729in}}%
\pgfpathlineto{\pgfqpoint{2.423864in}{1.309303in}}%
\pgfpathlineto{\pgfqpoint{2.354461in}{1.275591in}}%
\pgfpathlineto{\pgfqpoint{2.282022in}{1.248918in}}%
\pgfpathlineto{\pgfqpoint{2.207323in}{1.229537in}}%
\pgfpathlineto{\pgfqpoint{2.131038in}{1.217599in}}%
\pgfpathlineto{\pgfqpoint{2.053939in}{1.213262in}}%
\pgfpathlineto{\pgfqpoint{1.976818in}{1.216609in}}%
\pgfpathlineto{\pgfqpoint{1.900446in}{1.227637in}}%
\pgfpathlineto{\pgfqpoint{1.825577in}{1.246251in}}%
\pgfpathlineto{\pgfqpoint{1.752947in}{1.272271in}}%
\pgfpathlineto{\pgfqpoint{1.683274in}{1.305424in}}%
\pgfpathlineto{\pgfqpoint{1.617256in}{1.345351in}}%
\pgfpathlineto{\pgfqpoint{1.555574in}{1.391602in}}%
\pgfpathlineto{\pgfqpoint{1.498869in}{1.443660in}}%
\pgfpathlineto{\pgfqpoint{1.447631in}{1.501063in}}%
\pgfpathlineto{\pgfqpoint{1.402408in}{1.563246in}}%
\pgfpathlineto{\pgfqpoint{1.363689in}{1.629595in}}%
\pgfpathlineto{\pgfqpoint{1.331881in}{1.699457in}}%
\pgfpathlineto{\pgfqpoint{1.307307in}{1.772150in}}%
\pgfpathlineto{\pgfqpoint{1.290203in}{1.846955in}}%
\pgfpathlineto{\pgfqpoint{1.280727in}{1.923122in}}%
\pgfpathlineto{\pgfqpoint{1.278948in}{1.999865in}}%
\pgfpathlineto{\pgfqpoint{1.284855in}{2.076366in}}%
\pgfpathlineto{\pgfqpoint{1.298361in}{2.151899in}}%
\pgfpathlineto{\pgfqpoint{1.319353in}{2.225741in}}%
\pgfpathlineto{\pgfqpoint{1.347651in}{2.297111in}}%
\pgfpathlineto{\pgfqpoint{1.383003in}{2.365277in}}%
\pgfpathlineto{\pgfqpoint{1.425076in}{2.429556in}}%
\pgfpathlineto{\pgfqpoint{1.473463in}{2.489314in}}%
\pgfpathlineto{\pgfqpoint{1.527678in}{2.543965in}}%
\pgfpathlineto{\pgfqpoint{1.587160in}{2.592972in}}%
\pgfpathlineto{\pgfqpoint{1.651272in}{2.635849in}}%
\pgfpathlineto{\pgfqpoint{1.719299in}{2.672158in}}%
\pgfpathlineto{\pgfqpoint{1.790606in}{2.701589in}}%
\pgfpathlineto{\pgfqpoint{1.864512in}{2.723836in}}%
\pgfpathlineto{\pgfqpoint{1.940262in}{2.738630in}}%
\pgfpathlineto{\pgfqpoint{2.017101in}{2.745791in}}%
\pgfpathlineto{\pgfqpoint{2.094271in}{2.745233in}}%
\pgfpathlineto{\pgfqpoint{2.171014in}{2.736955in}}%
\pgfpathlineto{\pgfqpoint{2.246572in}{2.721052in}}%
\pgfpathlineto{\pgfqpoint{2.320184in}{2.697706in}}%
\pgfpathlineto{\pgfqpoint{2.391089in}{2.667190in}}%
\pgfpathlineto{\pgfqpoint{2.458537in}{2.629860in}}%
\pgfpathlineto{\pgfqpoint{2.521928in}{2.586059in}}%
\pgfpathlineto{\pgfqpoint{2.580609in}{2.536203in}}%
\pgfpathlineto{\pgfqpoint{2.633960in}{2.480778in}}%
\pgfpathlineto{\pgfqpoint{2.681431in}{2.420324in}}%
\pgfpathlineto{\pgfqpoint{2.722547in}{2.355433in}}%
\pgfpathlineto{\pgfqpoint{2.756905in}{2.286745in}}%
\pgfpathlineto{\pgfqpoint{2.784177in}{2.214955in}}%
\pgfpathlineto{\pgfqpoint{2.804107in}{2.140808in}}%
\pgfpathlineto{\pgfqpoint{2.816513in}{2.065098in}}%
\pgfpathlineto{\pgfqpoint{2.821311in}{1.988524in}}%
\pgfpathlineto{\pgfqpoint{2.818428in}{1.911850in}}%
\pgfpathlineto{\pgfqpoint{2.807852in}{1.835871in}}%
\pgfpathlineto{\pgfqpoint{2.789657in}{1.761354in}}%
\pgfpathlineto{\pgfqpoint{2.764003in}{1.689039in}}%
\pgfpathlineto{\pgfqpoint{2.731139in}{1.619638in}}%
\pgfpathlineto{\pgfqpoint{2.691399in}{1.553832in}}%
\pgfpathlineto{\pgfqpoint{2.645204in}{1.492277in}}%
\pgfpathlineto{\pgfqpoint{2.593063in}{1.435598in}}%
\pgfpathlineto{\pgfqpoint{2.535567in}{1.384389in}}%
\pgfpathlineto{\pgfqpoint{2.473246in}{1.339097in}}%
\pgfpathlineto{\pgfqpoint{2.406696in}{1.300188in}}%
\pgfpathlineto{\pgfqpoint{2.336586in}{1.268094in}}%
\pgfpathlineto{\pgfqpoint{2.263608in}{1.243162in}}%
\pgfpathlineto{\pgfqpoint{2.188476in}{1.225653in}}%
\pgfpathlineto{\pgfqpoint{2.111929in}{1.215739in}}%
\pgfpathlineto{\pgfqpoint{2.034731in}{1.213506in}}%
\pgfpathlineto{\pgfqpoint{1.957665in}{1.218953in}}%
\pgfpathlineto{\pgfqpoint{1.881543in}{1.231992in}}%
\pgfpathlineto{\pgfqpoint{1.807146in}{1.252459in}}%
\pgfpathlineto{\pgfqpoint{1.735136in}{1.280164in}}%
\pgfpathlineto{\pgfqpoint{1.666274in}{1.314858in}}%
\pgfpathlineto{\pgfqpoint{1.601279in}{1.356216in}}%
\pgfpathlineto{\pgfqpoint{1.540808in}{1.403848in}}%
\pgfpathlineto{\pgfqpoint{1.485465in}{1.457293in}}%
\pgfpathlineto{\pgfqpoint{1.435790in}{1.516021in}}%
\pgfpathlineto{\pgfqpoint{1.392267in}{1.579434in}}%
\pgfpathlineto{\pgfqpoint{1.355322in}{1.646863in}}%
\pgfpathlineto{\pgfqpoint{1.325322in}{1.717572in}}%
\pgfpathlineto{\pgfqpoint{1.302532in}{1.790860in}}%
\pgfpathlineto{\pgfqpoint{1.287170in}{1.866054in}}%
\pgfpathlineto{\pgfqpoint{1.279439in}{1.942376in}}%
\pgfpathlineto{\pgfqpoint{1.279451in}{2.019057in}}%
\pgfpathlineto{\pgfqpoint{1.287228in}{2.095338in}}%
\pgfpathlineto{\pgfqpoint{1.302701in}{2.170470in}}%
\pgfpathlineto{\pgfqpoint{1.325711in}{2.243713in}}%
\pgfpathlineto{\pgfqpoint{1.356005in}{2.314335in}}%
\pgfpathlineto{\pgfqpoint{1.393244in}{2.381615in}}%
\pgfpathlineto{\pgfqpoint{1.436995in}{2.444842in}}%
\pgfpathlineto{\pgfqpoint{1.486829in}{2.503428in}}%
\pgfpathlineto{\pgfqpoint{1.542288in}{2.556806in}}%
\pgfpathlineto{\pgfqpoint{1.602821in}{2.604399in}}%
\pgfpathlineto{\pgfqpoint{1.667837in}{2.645706in}}%
\pgfpathlineto{\pgfqpoint{1.736700in}{2.680307in}}%
\pgfpathlineto{\pgfqpoint{1.808732in}{2.707860in}}%
\pgfpathlineto{\pgfqpoint{1.883215in}{2.728103in}}%
\pgfpathlineto{\pgfqpoint{1.959386in}{2.740852in}}%
\pgfpathlineto{\pgfqpoint{2.036441in}{2.746003in}}%
\pgfpathlineto{\pgfqpoint{2.113595in}{2.743536in}}%
\pgfpathlineto{\pgfqpoint{2.190151in}{2.733478in}}%
\pgfpathlineto{\pgfqpoint{2.265302in}{2.715893in}}%
\pgfpathlineto{\pgfqpoint{2.338275in}{2.690924in}}%
\pgfpathlineto{\pgfqpoint{2.408336in}{2.658793in}}%
\pgfpathlineto{\pgfqpoint{2.474794in}{2.619808in}}%
\pgfpathlineto{\pgfqpoint{2.536995in}{2.574354in}}%
\pgfpathlineto{\pgfqpoint{2.594328in}{2.522901in}}%
\pgfpathlineto{\pgfqpoint{2.646221in}{2.465998in}}%
\pgfpathlineto{\pgfqpoint{2.692144in}{2.404277in}}%
\pgfpathlineto{\pgfqpoint{2.731651in}{2.338384in}}%
\pgfpathlineto{\pgfqpoint{2.764393in}{2.268899in}}%
\pgfpathlineto{\pgfqpoint{2.789992in}{2.196531in}}%
\pgfpathlineto{\pgfqpoint{2.808156in}{2.122003in}}%
\pgfpathlineto{\pgfqpoint{2.818684in}{2.046050in}}%
\pgfpathlineto{\pgfqpoint{2.821463in}{1.969417in}}%
\pgfpathlineto{\pgfqpoint{2.816472in}{1.892861in}}%
\pgfpathlineto{\pgfqpoint{2.803781in}{1.817148in}}%
\pgfpathlineto{\pgfqpoint{2.783547in}{1.743057in}}%
\pgfpathlineto{\pgfqpoint{2.756021in}{1.671376in}}%
\pgfpathlineto{\pgfqpoint{2.721481in}{1.602775in}}%
\pgfpathlineto{\pgfqpoint{2.680241in}{1.537919in}}%
\pgfpathlineto{\pgfqpoint{2.632697in}{1.477499in}}%
\pgfpathlineto{\pgfqpoint{2.579307in}{1.422138in}}%
\pgfpathlineto{\pgfqpoint{2.520588in}{1.372394in}}%
\pgfpathlineto{\pgfqpoint{2.457116in}{1.328758in}}%
\pgfpathlineto{\pgfqpoint{2.389530in}{1.291652in}}%
\pgfpathlineto{\pgfqpoint{2.318526in}{1.261435in}}%
\pgfpathlineto{\pgfqpoint{2.244860in}{1.238395in}}%
\pgfpathlineto{\pgfqpoint{2.169295in}{1.222737in}}%
\pgfpathlineto{\pgfqpoint{2.092511in}{1.214595in}}%
\pgfpathlineto{\pgfqpoint{2.015308in}{1.214096in}}%
\pgfpathlineto{\pgfqpoint{1.938470in}{1.221283in}}%
\pgfpathlineto{\pgfqpoint{1.862765in}{1.236109in}}%
\pgfpathlineto{\pgfqpoint{1.788937in}{1.258441in}}%
\pgfpathlineto{\pgfqpoint{1.717713in}{1.288055in}}%
\pgfpathlineto{\pgfqpoint{1.649798in}{1.324638in}}%
\pgfpathlineto{\pgfqpoint{1.585877in}{1.367792in}}%
\pgfpathlineto{\pgfqpoint{1.526615in}{1.417026in}}%
\pgfpathlineto{\pgfqpoint{1.472602in}{1.471818in}}%
\pgfpathlineto{\pgfqpoint{1.424317in}{1.531684in}}%
\pgfpathlineto{\pgfqpoint{1.382292in}{1.596026in}}%
\pgfpathlineto{\pgfqpoint{1.346981in}{1.664206in}}%
\pgfpathlineto{\pgfqpoint{1.318750in}{1.735558in}}%
\pgfpathlineto{\pgfqpoint{1.297883in}{1.809382in}}%
\pgfpathlineto{\pgfqpoint{1.284580in}{1.884947in}}%
\pgfpathlineto{\pgfqpoint{1.278956in}{1.961488in}}%
\pgfpathlineto{\pgfqpoint{1.281039in}{2.038212in}}%
\pgfpathlineto{\pgfqpoint{1.290777in}{2.114300in}}%
\pgfpathlineto{\pgfqpoint{1.308057in}{2.189070in}}%
\pgfpathlineto{\pgfqpoint{1.332734in}{2.261763in}}%
\pgfpathlineto{\pgfqpoint{1.364593in}{2.331618in}}%
\pgfpathlineto{\pgfqpoint{1.403340in}{2.397920in}}%
\pgfpathlineto{\pgfqpoint{1.448609in}{2.460010in}}%
\pgfpathlineto{\pgfqpoint{1.499956in}{2.517275in}}%
\pgfpathlineto{\pgfqpoint{1.556863in}{2.569155in}}%
\pgfpathlineto{\pgfqpoint{1.618735in}{2.615141in}}%
\pgfpathlineto{\pgfqpoint{1.684904in}{2.654774in}}%
\pgfpathlineto{\pgfqpoint{1.754644in}{2.687658in}}%
\pgfpathlineto{\pgfqpoint{1.827337in}{2.713520in}}%
\pgfpathlineto{\pgfqpoint{1.902256in}{2.732063in}}%
\pgfpathlineto{\pgfqpoint{1.978640in}{2.743062in}}%
\pgfpathlineto{\pgfqpoint{2.055729in}{2.746378in}}%
\pgfpathlineto{\pgfqpoint{2.132767in}{2.741965in}}%
\pgfpathlineto{\pgfqpoint{2.208995in}{2.729867in}}%
\pgfpathlineto{\pgfqpoint{2.283657in}{2.710218in}}%
\pgfpathlineto{\pgfqpoint{2.355998in}{2.683243in}}%
\pgfpathlineto{\pgfqpoint{2.425262in}{2.649257in}}%
\pgfpathlineto{\pgfqpoint{2.490736in}{2.608641in}}%
\pgfpathlineto{\pgfqpoint{2.551842in}{2.561753in}}%
\pgfpathlineto{\pgfqpoint{2.607928in}{2.509051in}}%
\pgfpathlineto{\pgfqpoint{2.658403in}{2.451049in}}%
\pgfpathlineto{\pgfqpoint{2.702753in}{2.388313in}}%
\pgfpathlineto{\pgfqpoint{2.740534in}{2.321456in}}%
\pgfpathlineto{\pgfqpoint{2.771380in}{2.251143in}}%
\pgfpathlineto{\pgfqpoint{2.794998in}{2.178089in}}%
\pgfpathlineto{\pgfqpoint{2.811170in}{2.103057in}}%
\pgfpathlineto{\pgfqpoint{2.819756in}{2.026841in}}%
\pgfpathlineto{\pgfqpoint{2.820703in}{1.950129in}}%
\pgfpathlineto{\pgfqpoint{2.813965in}{1.873708in}}%
\pgfpathlineto{\pgfqpoint{2.799570in}{1.798367in}}%
\pgfpathlineto{\pgfqpoint{2.777634in}{1.724865in}}%
\pgfpathlineto{\pgfqpoint{2.748357in}{1.653930in}}%
\pgfpathlineto{\pgfqpoint{2.712026in}{1.586257in}}%
\pgfpathlineto{\pgfqpoint{2.669014in}{1.522515in}}%
\pgfpathlineto{\pgfqpoint{2.619779in}{1.463337in}}%
\pgfpathlineto{\pgfqpoint{2.564866in}{1.409330in}}%
\pgfpathlineto{\pgfqpoint{2.504887in}{1.361049in}}%
\pgfpathlineto{\pgfqpoint{2.440372in}{1.318908in}}%
\pgfpathlineto{\pgfqpoint{2.371959in}{1.283361in}}%
\pgfpathlineto{\pgfqpoint{2.300332in}{1.254806in}}%
\pgfpathlineto{\pgfqpoint{2.226198in}{1.233550in}}%
\pgfpathlineto{\pgfqpoint{2.150282in}{1.219813in}}%
\pgfpathlineto{\pgfqpoint{2.073333in}{1.213730in}}%
\pgfpathlineto{\pgfqpoint{1.996120in}{1.215345in}}%
\pgfpathlineto{\pgfqpoint{1.919432in}{1.224616in}}%
\pgfpathlineto{\pgfqpoint{1.844081in}{1.241414in}}%
\pgfpathlineto{\pgfqpoint{1.770813in}{1.265543in}}%
\pgfpathlineto{\pgfqpoint{1.700297in}{1.296788in}}%
\pgfpathlineto{\pgfqpoint{1.633286in}{1.334860in}}%
\pgfpathlineto{\pgfqpoint{1.570474in}{1.379401in}}%
\pgfpathlineto{\pgfqpoint{1.512497in}{1.429985in}}%
\pgfpathlineto{\pgfqpoint{1.459929in}{1.486121in}}%
\pgfpathlineto{\pgfqpoint{1.413283in}{1.547249in}}%
\pgfpathlineto{\pgfqpoint{1.373013in}{1.612742in}}%
\pgfpathlineto{\pgfqpoint{1.339512in}{1.681907in}}%
\pgfpathlineto{\pgfqpoint{1.313110in}{1.753986in}}%
\pgfpathlineto{\pgfqpoint{1.294024in}{1.828313in}}%
\pgfpathlineto{\pgfqpoint{1.282465in}{1.904164in}}%
\pgfpathlineto{\pgfqpoint{1.278594in}{1.980757in}}%
\pgfpathlineto{\pgfqpoint{1.282481in}{2.057326in}}%
\pgfpathlineto{\pgfqpoint{1.294106in}{2.133114in}}%
\pgfpathlineto{\pgfqpoint{1.313359in}{2.207377in}}%
\pgfpathlineto{\pgfqpoint{1.340039in}{2.279382in}}%
\pgfpathlineto{\pgfqpoint{1.373854in}{2.348407in}}%
\pgfpathlineto{\pgfqpoint{1.414422in}{2.413742in}}%
\pgfpathlineto{\pgfqpoint{1.461277in}{2.474699in}}%
\pgfpathlineto{\pgfqpoint{1.513990in}{2.530739in}}%
\pgfpathlineto{\pgfqpoint{1.572057in}{2.581286in}}%
\pgfpathlineto{\pgfqpoint{1.634902in}{2.625795in}}%
\pgfpathlineto{\pgfqpoint{1.701912in}{2.663801in}}%
\pgfpathlineto{\pgfqpoint{1.772430in}{2.694919in}}%
\pgfpathlineto{\pgfqpoint{1.845761in}{2.718844in}}%
\pgfpathlineto{\pgfqpoint{1.921170in}{2.735352in}}%
\pgfpathlineto{\pgfqpoint{1.997881in}{2.744297in}}%
\pgfpathlineto{\pgfqpoint{2.075077in}{2.745616in}}%
\pgfpathlineto{\pgfqpoint{2.152008in}{2.739327in}}%
\pgfpathlineto{\pgfqpoint{2.227952in}{2.725482in}}%
\pgfpathlineto{\pgfqpoint{2.302106in}{2.704182in}}%
\pgfpathlineto{\pgfqpoint{2.373711in}{2.675610in}}%
\pgfpathlineto{\pgfqpoint{2.442050in}{2.640026in}}%
\pgfpathlineto{\pgfqpoint{2.506449in}{2.597775in}}%
\pgfpathlineto{\pgfqpoint{2.566275in}{2.549277in}}%
\pgfpathlineto{\pgfqpoint{2.620940in}{2.495037in}}%
\pgfpathlineto{\pgfqpoint{2.669898in}{2.435637in}}%
\pgfpathlineto{\pgfqpoint{2.712645in}{2.371739in}}%
\pgfpathlineto{\pgfqpoint{2.748793in}{2.303973in}}%
\pgfpathlineto{\pgfqpoint{2.777995in}{2.232958in}}%
\pgfpathlineto{\pgfqpoint{2.799910in}{2.159422in}}%
\pgfpathlineto{\pgfqpoint{2.814285in}{2.084097in}}%
\pgfpathlineto{\pgfqpoint{2.820958in}{2.007724in}}%
\pgfpathlineto{\pgfqpoint{2.819860in}{1.931053in}}%
\pgfpathlineto{\pgfqpoint{2.811010in}{1.854843in}}%
\pgfpathlineto{\pgfqpoint{2.794518in}{1.779859in}}%
\pgfpathlineto{\pgfqpoint{2.770585in}{1.706877in}}%
\pgfpathlineto{\pgfqpoint{2.739499in}{1.636672in}}%
\pgfpathlineto{\pgfqpoint{2.701554in}{1.569877in}}%
\pgfpathlineto{\pgfqpoint{2.657104in}{1.507166in}}%
\pgfpathlineto{\pgfqpoint{2.606579in}{1.449204in}}%
\pgfpathlineto{\pgfqpoint{2.550467in}{1.396587in}}%
\pgfpathlineto{\pgfqpoint{2.489313in}{1.349841in}}%
\pgfpathlineto{\pgfqpoint{2.423721in}{1.309426in}}%
\pgfpathlineto{\pgfqpoint{2.354354in}{1.275733in}}%
\pgfpathlineto{\pgfqpoint{2.281931in}{1.249084in}}%
\pgfpathlineto{\pgfqpoint{2.207229in}{1.229731in}}%
\pgfpathlineto{\pgfqpoint{2.130973in}{1.217832in}}%
\pgfpathlineto{\pgfqpoint{2.053887in}{1.213509in}}%
\pgfpathlineto{\pgfqpoint{1.976773in}{1.216853in}}%
\pgfpathlineto{\pgfqpoint{1.900412in}{1.227862in}}%
\pgfpathlineto{\pgfqpoint{1.825563in}{1.246450in}}%
\pgfpathlineto{\pgfqpoint{1.752961in}{1.272442in}}%
\pgfpathlineto{\pgfqpoint{1.683321in}{1.305573in}}%
\pgfpathlineto{\pgfqpoint{1.617334in}{1.345492in}}%
\pgfpathlineto{\pgfqpoint{1.555672in}{1.391759in}}%
\pgfpathlineto{\pgfqpoint{1.498981in}{1.443846in}}%
\pgfpathlineto{\pgfqpoint{1.447785in}{1.501243in}}%
\pgfpathlineto{\pgfqpoint{1.402578in}{1.563423in}}%
\pgfpathlineto{\pgfqpoint{1.363858in}{1.629760in}}%
\pgfpathlineto{\pgfqpoint{1.332042in}{1.699603in}}%
\pgfpathlineto{\pgfqpoint{1.307458in}{1.772265in}}%
\pgfpathlineto{\pgfqpoint{1.290351in}{1.847035in}}%
\pgfpathlineto{\pgfqpoint{1.280880in}{1.923168in}}%
\pgfpathlineto{\pgfqpoint{1.279119in}{1.999890in}}%
\pgfpathlineto{\pgfqpoint{1.285056in}{2.076398in}}%
\pgfpathlineto{\pgfqpoint{1.298608in}{2.151880in}}%
\pgfpathlineto{\pgfqpoint{1.319675in}{2.225629in}}%
\pgfpathlineto{\pgfqpoint{1.348054in}{2.296916in}}%
\pgfpathlineto{\pgfqpoint{1.348054in}{2.296916in}}%
\pgfusepath{stroke}%
\end{pgfscope}%
\begin{pgfscope}%
\pgfpathrectangle{\pgfqpoint{0.500000in}{0.440000in}}{\pgfqpoint{3.100000in}{3.080000in}}%
\pgfusepath{clip}%
\pgfsetrectcap%
\pgfsetroundjoin%
\pgfsetlinewidth{0.501875pt}%
\definecolor{currentstroke}{rgb}{0.890196,0.466667,0.760784}%
\pgfsetstrokecolor{currentstroke}%
\pgfsetdash{}{0pt}%
\pgfpathmoveto{\pgfqpoint{1.432479in}{1.366237in}}%
\pgfpathlineto{\pgfqpoint{1.373858in}{1.430625in}}%
\pgfpathlineto{\pgfqpoint{1.321957in}{1.500517in}}%
\pgfpathlineto{\pgfqpoint{1.277295in}{1.575203in}}%
\pgfpathlineto{\pgfqpoint{1.240374in}{1.653927in}}%
\pgfpathlineto{\pgfqpoint{1.211596in}{1.735909in}}%
\pgfpathlineto{\pgfqpoint{1.191262in}{1.820345in}}%
\pgfpathlineto{\pgfqpoint{1.179578in}{1.906405in}}%
\pgfpathlineto{\pgfqpoint{1.176645in}{1.993234in}}%
\pgfpathlineto{\pgfqpoint{1.182471in}{2.079954in}}%
\pgfpathlineto{\pgfqpoint{1.196959in}{2.165659in}}%
\pgfpathlineto{\pgfqpoint{1.219924in}{2.249458in}}%
\pgfpathlineto{\pgfqpoint{1.251147in}{2.330605in}}%
\pgfpathlineto{\pgfqpoint{1.290347in}{2.408250in}}%
\pgfpathlineto{\pgfqpoint{1.337157in}{2.481581in}}%
\pgfpathlineto{\pgfqpoint{1.391135in}{2.549851in}}%
\pgfpathlineto{\pgfqpoint{1.451761in}{2.612382in}}%
\pgfpathlineto{\pgfqpoint{1.518434in}{2.668560in}}%
\pgfpathlineto{\pgfqpoint{1.590477in}{2.717836in}}%
\pgfpathlineto{\pgfqpoint{1.667135in}{2.759732in}}%
\pgfpathlineto{\pgfqpoint{1.747574in}{2.793831in}}%
\pgfpathlineto{\pgfqpoint{1.830979in}{2.819827in}}%
\pgfpathlineto{\pgfqpoint{1.916595in}{2.837479in}}%
\pgfpathlineto{\pgfqpoint{2.003536in}{2.846556in}}%
\pgfpathlineto{\pgfqpoint{2.090924in}{2.846926in}}%
\pgfpathlineto{\pgfqpoint{2.177893in}{2.838560in}}%
\pgfpathlineto{\pgfqpoint{2.263588in}{2.821530in}}%
\pgfpathlineto{\pgfqpoint{2.347166in}{2.796010in}}%
\pgfpathlineto{\pgfqpoint{2.427794in}{2.762279in}}%
\pgfpathlineto{\pgfqpoint{2.504650in}{2.720714in}}%
\pgfpathlineto{\pgfqpoint{2.576925in}{2.671795in}}%
\pgfpathlineto{\pgfqpoint{2.643929in}{2.616019in}}%
\pgfpathlineto{\pgfqpoint{2.705034in}{2.553888in}}%
\pgfpathlineto{\pgfqpoint{2.759572in}{2.486020in}}%
\pgfpathlineto{\pgfqpoint{2.806972in}{2.413080in}}%
\pgfpathlineto{\pgfqpoint{2.846748in}{2.335781in}}%
\pgfpathlineto{\pgfqpoint{2.878507in}{2.254882in}}%
\pgfpathlineto{\pgfqpoint{2.901946in}{2.171189in}}%
\pgfpathlineto{\pgfqpoint{2.916850in}{2.085558in}}%
\pgfpathlineto{\pgfqpoint{2.923098in}{1.998889in}}%
\pgfpathlineto{\pgfqpoint{2.920660in}{1.912076in}}%
\pgfpathlineto{\pgfqpoint{2.909572in}{1.825900in}}%
\pgfpathlineto{\pgfqpoint{2.889904in}{1.741262in}}%
\pgfpathlineto{\pgfqpoint{2.861812in}{1.659038in}}%
\pgfpathlineto{\pgfqpoint{2.825548in}{1.580054in}}%
\pgfpathlineto{\pgfqpoint{2.781456in}{1.505093in}}%
\pgfpathlineto{\pgfqpoint{2.729972in}{1.434890in}}%
\pgfpathlineto{\pgfqpoint{2.671624in}{1.370136in}}%
\pgfpathlineto{\pgfqpoint{2.607033in}{1.311475in}}%
\pgfpathlineto{\pgfqpoint{2.536915in}{1.259505in}}%
\pgfpathlineto{\pgfqpoint{2.462015in}{1.214738in}}%
\pgfpathlineto{\pgfqpoint{2.382990in}{1.177561in}}%
\pgfpathlineto{\pgfqpoint{2.300644in}{1.148401in}}%
\pgfpathlineto{\pgfqpoint{2.215801in}{1.127592in}}%
\pgfpathlineto{\pgfqpoint{2.129298in}{1.115367in}}%
\pgfpathlineto{\pgfqpoint{2.041983in}{1.111855in}}%
\pgfpathlineto{\pgfqpoint{1.954719in}{1.117087in}}%
\pgfpathlineto{\pgfqpoint{1.868377in}{1.130990in}}%
\pgfpathlineto{\pgfqpoint{1.783843in}{1.153390in}}%
\pgfpathlineto{\pgfqpoint{1.702012in}{1.184012in}}%
\pgfpathlineto{\pgfqpoint{1.623671in}{1.222535in}}%
\pgfpathlineto{\pgfqpoint{1.549558in}{1.268614in}}%
\pgfpathlineto{\pgfqpoint{1.480465in}{1.321804in}}%
\pgfpathlineto{\pgfqpoint{1.417108in}{1.381595in}}%
\pgfpathlineto{\pgfqpoint{1.360126in}{1.447406in}}%
\pgfpathlineto{\pgfqpoint{1.310082in}{1.518593in}}%
\pgfpathlineto{\pgfqpoint{1.267462in}{1.594442in}}%
\pgfpathlineto{\pgfqpoint{1.232676in}{1.674173in}}%
\pgfpathlineto{\pgfqpoint{1.206057in}{1.756940in}}%
\pgfpathlineto{\pgfqpoint{1.187846in}{1.841877in}}%
\pgfpathlineto{\pgfqpoint{1.178191in}{1.928221in}}%
\pgfpathlineto{\pgfqpoint{1.177240in}{2.015077in}}%
\pgfpathlineto{\pgfqpoint{1.185047in}{2.101561in}}%
\pgfpathlineto{\pgfqpoint{1.201563in}{2.186807in}}%
\pgfpathlineto{\pgfqpoint{1.226641in}{2.269976in}}%
\pgfpathlineto{\pgfqpoint{1.260030in}{2.350250in}}%
\pgfpathlineto{\pgfqpoint{1.301382in}{2.426836in}}%
\pgfpathlineto{\pgfqpoint{1.350245in}{2.498962in}}%
\pgfpathlineto{\pgfqpoint{1.406067in}{2.565879in}}%
\pgfpathlineto{\pgfqpoint{1.468244in}{2.626913in}}%
\pgfpathlineto{\pgfqpoint{1.536230in}{2.681530in}}%
\pgfpathlineto{\pgfqpoint{1.609347in}{2.729129in}}%
\pgfpathlineto{\pgfqpoint{1.686871in}{2.769198in}}%
\pgfpathlineto{\pgfqpoint{1.768042in}{2.801317in}}%
\pgfpathlineto{\pgfqpoint{1.852065in}{2.825162in}}%
\pgfpathlineto{\pgfqpoint{1.938109in}{2.840504in}}%
\pgfpathlineto{\pgfqpoint{2.025305in}{2.847211in}}%
\pgfpathlineto{\pgfqpoint{2.112752in}{2.845243in}}%
\pgfpathlineto{\pgfqpoint{2.199514in}{2.834658in}}%
\pgfpathlineto{\pgfqpoint{2.284802in}{2.815582in}}%
\pgfpathlineto{\pgfqpoint{2.367767in}{2.788177in}}%
\pgfpathlineto{\pgfqpoint{2.447534in}{2.752681in}}%
\pgfpathlineto{\pgfqpoint{2.523287in}{2.709421in}}%
\pgfpathlineto{\pgfqpoint{2.594271in}{2.658804in}}%
\pgfpathlineto{\pgfqpoint{2.659785in}{2.601326in}}%
\pgfpathlineto{\pgfqpoint{2.719188in}{2.537566in}}%
\pgfpathlineto{\pgfqpoint{2.771898in}{2.468186in}}%
\pgfpathlineto{\pgfqpoint{2.817390in}{2.393936in}}%
\pgfpathlineto{\pgfqpoint{2.855206in}{2.315632in}}%
\pgfpathlineto{\pgfqpoint{2.885033in}{2.233978in}}%
\pgfpathlineto{\pgfqpoint{2.906536in}{2.149781in}}%
\pgfpathlineto{\pgfqpoint{2.919451in}{2.063899in}}%
\pgfpathlineto{\pgfqpoint{2.923617in}{1.977185in}}%
\pgfpathlineto{\pgfqpoint{2.918975in}{1.890494in}}%
\pgfpathlineto{\pgfqpoint{2.905570in}{1.804678in}}%
\pgfpathlineto{\pgfqpoint{2.883550in}{1.720586in}}%
\pgfpathlineto{\pgfqpoint{2.853166in}{1.639068in}}%
\pgfpathlineto{\pgfqpoint{2.814772in}{1.560973in}}%
\pgfpathlineto{\pgfqpoint{2.768805in}{1.487111in}}%
\pgfpathlineto{\pgfqpoint{2.715672in}{1.418136in}}%
\pgfpathlineto{\pgfqpoint{2.655890in}{1.354774in}}%
\pgfpathlineto{\pgfqpoint{2.590041in}{1.297696in}}%
\pgfpathlineto{\pgfqpoint{2.518769in}{1.247487in}}%
\pgfpathlineto{\pgfqpoint{2.442770in}{1.204648in}}%
\pgfpathlineto{\pgfqpoint{2.362798in}{1.169596in}}%
\pgfpathlineto{\pgfqpoint{2.279664in}{1.142664in}}%
\pgfpathlineto{\pgfqpoint{2.194235in}{1.124102in}}%
\pgfpathlineto{\pgfqpoint{2.107419in}{1.114071in}}%
\pgfpathlineto{\pgfqpoint{2.020005in}{1.112632in}}%
\pgfpathlineto{\pgfqpoint{1.932880in}{1.119836in}}%
\pgfpathlineto{\pgfqpoint{1.846948in}{1.135655in}}%
\pgfpathlineto{\pgfqpoint{1.763073in}{1.159965in}}%
\pgfpathlineto{\pgfqpoint{1.682089in}{1.192545in}}%
\pgfpathlineto{\pgfqpoint{1.604791in}{1.233078in}}%
\pgfpathlineto{\pgfqpoint{1.531940in}{1.281148in}}%
\pgfpathlineto{\pgfqpoint{1.464260in}{1.336246in}}%
\pgfpathlineto{\pgfqpoint{1.402442in}{1.397763in}}%
\pgfpathlineto{\pgfqpoint{1.347126in}{1.465011in}}%
\pgfpathlineto{\pgfqpoint{1.298787in}{1.537385in}}%
\pgfpathlineto{\pgfqpoint{1.257940in}{1.614179in}}%
\pgfpathlineto{\pgfqpoint{1.225039in}{1.694622in}}%
\pgfpathlineto{\pgfqpoint{1.200444in}{1.777920in}}%
\pgfpathlineto{\pgfqpoint{1.184409in}{1.863257in}}%
\pgfpathlineto{\pgfqpoint{1.177092in}{1.949793in}}%
\pgfpathlineto{\pgfqpoint{1.178549in}{2.036666in}}%
\pgfpathlineto{\pgfqpoint{1.188737in}{2.122988in}}%
\pgfpathlineto{\pgfqpoint{1.207513in}{2.207850in}}%
\pgfpathlineto{\pgfqpoint{1.234654in}{2.290395in}}%
\pgfpathlineto{\pgfqpoint{1.269916in}{2.369881in}}%
\pgfpathlineto{\pgfqpoint{1.312972in}{2.445462in}}%
\pgfpathlineto{\pgfqpoint{1.363418in}{2.516351in}}%
\pgfpathlineto{\pgfqpoint{1.420772in}{2.581831in}}%
\pgfpathlineto{\pgfqpoint{1.484479in}{2.641253in}}%
\pgfpathlineto{\pgfqpoint{1.553904in}{2.694034in}}%
\pgfpathlineto{\pgfqpoint{1.628338in}{2.739662in}}%
\pgfpathlineto{\pgfqpoint{1.706994in}{2.777692in}}%
\pgfpathlineto{\pgfqpoint{1.789012in}{2.807749in}}%
\pgfpathlineto{\pgfqpoint{1.873618in}{2.829583in}}%
\pgfpathlineto{\pgfqpoint{1.960002in}{2.842961in}}%
\pgfpathlineto{\pgfqpoint{2.047271in}{2.847697in}}%
\pgfpathlineto{\pgfqpoint{2.134550in}{2.843708in}}%
\pgfpathlineto{\pgfqpoint{2.220976in}{2.831010in}}%
\pgfpathlineto{\pgfqpoint{2.305699in}{2.809723in}}%
\pgfpathlineto{\pgfqpoint{2.387885in}{2.780068in}}%
\pgfpathlineto{\pgfqpoint{2.466710in}{2.742367in}}%
\pgfpathlineto{\pgfqpoint{2.541367in}{2.697045in}}%
\pgfpathlineto{\pgfqpoint{2.611068in}{2.644624in}}%
\pgfpathlineto{\pgfqpoint{2.675187in}{2.585596in}}%
\pgfpathlineto{\pgfqpoint{2.733078in}{2.520517in}}%
\pgfpathlineto{\pgfqpoint{2.784115in}{2.450034in}}%
\pgfpathlineto{\pgfqpoint{2.827763in}{2.374835in}}%
\pgfpathlineto{\pgfqpoint{2.863578in}{2.295657in}}%
\pgfpathlineto{\pgfqpoint{2.891208in}{2.213279in}}%
\pgfpathlineto{\pgfqpoint{2.910394in}{2.128526in}}%
\pgfpathlineto{\pgfqpoint{2.920966in}{2.042267in}}%
\pgfpathlineto{\pgfqpoint{2.922846in}{1.955418in}}%
\pgfpathlineto{\pgfqpoint{2.916053in}{1.868841in}}%
\pgfpathlineto{\pgfqpoint{2.900647in}{1.783333in}}%
\pgfpathlineto{\pgfqpoint{2.876741in}{1.699798in}}%
\pgfpathlineto{\pgfqpoint{2.844537in}{1.619095in}}%
\pgfpathlineto{\pgfqpoint{2.804330in}{1.542032in}}%
\pgfpathlineto{\pgfqpoint{2.756506in}{1.469371in}}%
\pgfpathlineto{\pgfqpoint{2.701540in}{1.401825in}}%
\pgfpathlineto{\pgfqpoint{2.640002in}{1.340059in}}%
\pgfpathlineto{\pgfqpoint{2.572549in}{1.284688in}}%
\pgfpathlineto{\pgfqpoint{2.499933in}{1.236278in}}%
\pgfpathlineto{\pgfqpoint{2.422885in}{1.195283in}}%
\pgfpathlineto{\pgfqpoint{2.342100in}{1.162080in}}%
\pgfpathlineto{\pgfqpoint{2.258405in}{1.137059in}}%
\pgfpathlineto{\pgfqpoint{2.172634in}{1.120509in}}%
\pgfpathlineto{\pgfqpoint{2.085633in}{1.112617in}}%
\pgfpathlineto{\pgfqpoint{1.998257in}{1.113468in}}%
\pgfpathlineto{\pgfqpoint{1.911368in}{1.123044in}}%
\pgfpathlineto{\pgfqpoint{1.825838in}{1.141227in}}%
\pgfpathlineto{\pgfqpoint{1.742548in}{1.167795in}}%
\pgfpathlineto{\pgfqpoint{1.662385in}{1.202429in}}%
\pgfpathlineto{\pgfqpoint{1.586082in}{1.244790in}}%
\pgfpathlineto{\pgfqpoint{1.514397in}{1.294490in}}%
\pgfpathlineto{\pgfqpoint{1.448091in}{1.351047in}}%
\pgfpathlineto{\pgfqpoint{1.387847in}{1.413913in}}%
\pgfpathlineto{\pgfqpoint{1.334270in}{1.482478in}}%
\pgfpathlineto{\pgfqpoint{1.287887in}{1.556066in}}%
\pgfpathlineto{\pgfqpoint{1.249147in}{1.633936in}}%
\pgfpathlineto{\pgfqpoint{1.218420in}{1.715283in}}%
\pgfpathlineto{\pgfqpoint{1.195999in}{1.799236in}}%
\pgfpathlineto{\pgfqpoint{1.182068in}{1.884965in}}%
\pgfpathlineto{\pgfqpoint{1.176763in}{1.971669in}}%
\pgfpathlineto{\pgfqpoint{1.180190in}{2.058444in}}%
\pgfpathlineto{\pgfqpoint{1.192355in}{2.144409in}}%
\pgfpathlineto{\pgfqpoint{1.213162in}{2.228711in}}%
\pgfpathlineto{\pgfqpoint{1.242416in}{2.310519in}}%
\pgfpathlineto{\pgfqpoint{1.279823in}{2.389029in}}%
\pgfpathlineto{\pgfqpoint{1.324986in}{2.463461in}}%
\pgfpathlineto{\pgfqpoint{1.377409in}{2.533062in}}%
\pgfpathlineto{\pgfqpoint{1.436496in}{2.597102in}}%
\pgfpathlineto{\pgfqpoint{1.501654in}{2.654977in}}%
\pgfpathlineto{\pgfqpoint{1.572291in}{2.706145in}}%
\pgfpathlineto{\pgfqpoint{1.647699in}{2.750036in}}%
\pgfpathlineto{\pgfqpoint{1.727134in}{2.786181in}}%
\pgfpathlineto{\pgfqpoint{1.809816in}{2.814203in}}%
\pgfpathlineto{\pgfqpoint{1.894936in}{2.833822in}}%
\pgfpathlineto{\pgfqpoint{1.981646in}{2.844856in}}%
\pgfpathlineto{\pgfqpoint{2.069069in}{2.847216in}}%
\pgfpathlineto{\pgfqpoint{2.156290in}{2.840911in}}%
\pgfpathlineto{\pgfqpoint{2.242392in}{2.826044in}}%
\pgfpathlineto{\pgfqpoint{2.326612in}{2.802765in}}%
\pgfpathlineto{\pgfqpoint{2.408076in}{2.771275in}}%
\pgfpathlineto{\pgfqpoint{2.485932in}{2.731856in}}%
\pgfpathlineto{\pgfqpoint{2.559389in}{2.684875in}}%
\pgfpathlineto{\pgfqpoint{2.627715in}{2.630780in}}%
\pgfpathlineto{\pgfqpoint{2.690240in}{2.570105in}}%
\pgfpathlineto{\pgfqpoint{2.746351in}{2.503465in}}%
\pgfpathlineto{\pgfqpoint{2.795499in}{2.431559in}}%
\pgfpathlineto{\pgfqpoint{2.837192in}{2.355170in}}%
\pgfpathlineto{\pgfqpoint{2.871026in}{2.275111in}}%
\pgfpathlineto{\pgfqpoint{2.896715in}{2.192086in}}%
\pgfpathlineto{\pgfqpoint{2.913948in}{2.106944in}}%
\pgfpathlineto{\pgfqpoint{2.922508in}{2.020545in}}%
\pgfpathlineto{\pgfqpoint{2.922283in}{1.933748in}}%
\pgfpathlineto{\pgfqpoint{2.913260in}{1.847404in}}%
\pgfpathlineto{\pgfqpoint{2.895533in}{1.762363in}}%
\pgfpathlineto{\pgfqpoint{2.869297in}{1.679472in}}%
\pgfpathlineto{\pgfqpoint{2.834851in}{1.599572in}}%
\pgfpathlineto{\pgfqpoint{2.792597in}{1.523503in}}%
\pgfpathlineto{\pgfqpoint{2.742987in}{1.452023in}}%
\pgfpathlineto{\pgfqpoint{2.686459in}{1.385778in}}%
\pgfpathlineto{\pgfqpoint{2.623571in}{1.325482in}}%
\pgfpathlineto{\pgfqpoint{2.554937in}{1.271769in}}%
\pgfpathlineto{\pgfqpoint{2.481227in}{1.225186in}}%
\pgfpathlineto{\pgfqpoint{2.403163in}{1.186195in}}%
\pgfpathlineto{\pgfqpoint{2.321523in}{1.155174in}}%
\pgfpathlineto{\pgfqpoint{2.237140in}{1.132413in}}%
\pgfpathlineto{\pgfqpoint{2.150900in}{1.118119in}}%
\pgfpathlineto{\pgfqpoint{2.063703in}{1.112403in}}%
\pgfpathlineto{\pgfqpoint{1.976332in}{1.115301in}}%
\pgfpathlineto{\pgfqpoint{1.889697in}{1.126825in}}%
\pgfpathlineto{\pgfqpoint{1.804690in}{1.146904in}}%
\pgfpathlineto{\pgfqpoint{1.722164in}{1.175366in}}%
\pgfpathlineto{\pgfqpoint{1.642937in}{1.211947in}}%
\pgfpathlineto{\pgfqpoint{1.567786in}{1.256286in}}%
\pgfpathlineto{\pgfqpoint{1.497451in}{1.307926in}}%
\pgfpathlineto{\pgfqpoint{1.432635in}{1.366315in}}%
\pgfpathlineto{\pgfqpoint{1.374002in}{1.430805in}}%
\pgfpathlineto{\pgfqpoint{1.322144in}{1.500694in}}%
\pgfpathlineto{\pgfqpoint{1.277506in}{1.575370in}}%
\pgfpathlineto{\pgfqpoint{1.240584in}{1.654083in}}%
\pgfpathlineto{\pgfqpoint{1.211792in}{1.736047in}}%
\pgfpathlineto{\pgfqpoint{1.191441in}{1.820455in}}%
\pgfpathlineto{\pgfqpoint{1.179742in}{1.906478in}}%
\pgfpathlineto{\pgfqpoint{1.176806in}{1.993269in}}%
\pgfpathlineto{\pgfqpoint{1.182643in}{2.079958in}}%
\pgfpathlineto{\pgfqpoint{1.197163in}{2.165657in}}%
\pgfpathlineto{\pgfqpoint{1.220175in}{2.249455in}}%
\pgfpathlineto{\pgfqpoint{1.251427in}{2.330546in}}%
\pgfpathlineto{\pgfqpoint{1.290641in}{2.408168in}}%
\pgfpathlineto{\pgfqpoint{1.337447in}{2.481494in}}%
\pgfpathlineto{\pgfqpoint{1.391402in}{2.549763in}}%
\pgfpathlineto{\pgfqpoint{1.451989in}{2.612289in}}%
\pgfpathlineto{\pgfqpoint{1.518615in}{2.668453in}}%
\pgfpathlineto{\pgfqpoint{1.590615in}{2.717706in}}%
\pgfpathlineto{\pgfqpoint{1.667247in}{2.759573in}}%
\pgfpathlineto{\pgfqpoint{1.747697in}{2.793645in}}%
\pgfpathlineto{\pgfqpoint{1.831094in}{2.819593in}}%
\pgfpathlineto{\pgfqpoint{1.916689in}{2.837209in}}%
\pgfpathlineto{\pgfqpoint{2.003621in}{2.846278in}}%
\pgfpathlineto{\pgfqpoint{2.091000in}{2.846661in}}%
\pgfpathlineto{\pgfqpoint{2.177951in}{2.838321in}}%
\pgfpathlineto{\pgfqpoint{2.263617in}{2.821323in}}%
\pgfpathlineto{\pgfqpoint{2.347157in}{2.795833in}}%
\pgfpathlineto{\pgfqpoint{2.427744in}{2.762117in}}%
\pgfpathlineto{\pgfqpoint{2.504568in}{2.720545in}}%
\pgfpathlineto{\pgfqpoint{2.576838in}{2.671587in}}%
\pgfpathlineto{\pgfqpoint{2.643799in}{2.615794in}}%
\pgfpathlineto{\pgfqpoint{2.704870in}{2.553663in}}%
\pgfpathlineto{\pgfqpoint{2.759402in}{2.485801in}}%
\pgfpathlineto{\pgfqpoint{2.806808in}{2.412879in}}%
\pgfpathlineto{\pgfqpoint{2.846594in}{2.335610in}}%
\pgfpathlineto{\pgfqpoint{2.878358in}{2.254751in}}%
\pgfpathlineto{\pgfqpoint{2.901791in}{2.171100in}}%
\pgfpathlineto{\pgfqpoint{2.916678in}{2.085500in}}%
\pgfpathlineto{\pgfqpoint{2.922893in}{1.998833in}}%
\pgfpathlineto{\pgfqpoint{2.920407in}{1.912027in}}%
\pgfpathlineto{\pgfqpoint{2.909278in}{1.825894in}}%
\pgfpathlineto{\pgfqpoint{2.889592in}{1.741272in}}%
\pgfpathlineto{\pgfqpoint{2.861507in}{1.659050in}}%
\pgfpathlineto{\pgfqpoint{2.825272in}{1.580071in}}%
\pgfpathlineto{\pgfqpoint{2.781222in}{1.505122in}}%
\pgfpathlineto{\pgfqpoint{2.729786in}{1.434944in}}%
\pgfpathlineto{\pgfqpoint{2.671478in}{1.370224in}}%
\pgfpathlineto{\pgfqpoint{2.606906in}{1.311599in}}%
\pgfpathlineto{\pgfqpoint{2.536766in}{1.259655in}}%
\pgfpathlineto{\pgfqpoint{2.461841in}{1.214925in}}%
\pgfpathlineto{\pgfqpoint{2.382832in}{1.177798in}}%
\pgfpathlineto{\pgfqpoint{2.300491in}{1.148652in}}%
\pgfpathlineto{\pgfqpoint{2.215657in}{1.127834in}}%
\pgfpathlineto{\pgfqpoint{2.129174in}{1.115586in}}%
\pgfpathlineto{\pgfqpoint{2.041894in}{1.112048in}}%
\pgfpathlineto{\pgfqpoint{1.954674in}{1.117259in}}%
\pgfpathlineto{\pgfqpoint{1.868377in}{1.131154in}}%
\pgfpathlineto{\pgfqpoint{1.783873in}{1.153569in}}%
\pgfpathlineto{\pgfqpoint{1.702039in}{1.184234in}}%
\pgfpathlineto{\pgfqpoint{1.623739in}{1.222789in}}%
\pgfpathlineto{\pgfqpoint{1.549668in}{1.268883in}}%
\pgfpathlineto{\pgfqpoint{1.480592in}{1.322078in}}%
\pgfpathlineto{\pgfqpoint{1.417240in}{1.381857in}}%
\pgfpathlineto{\pgfqpoint{1.360259in}{1.447641in}}%
\pgfpathlineto{\pgfqpoint{1.310221in}{1.518787in}}%
\pgfpathlineto{\pgfqpoint{1.267615in}{1.594591in}}%
\pgfpathlineto{\pgfqpoint{1.232849in}{1.674285in}}%
\pgfpathlineto{\pgfqpoint{1.206255in}{1.757038in}}%
\pgfpathlineto{\pgfqpoint{1.188081in}{1.841965in}}%
\pgfpathlineto{\pgfqpoint{1.178462in}{1.928281in}}%
\pgfpathlineto{\pgfqpoint{1.177522in}{2.015125in}}%
\pgfpathlineto{\pgfqpoint{1.185320in}{2.101598in}}%
\pgfpathlineto{\pgfqpoint{1.201814in}{2.186830in}}%
\pgfpathlineto{\pgfqpoint{1.226861in}{2.269976in}}%
\pgfpathlineto{\pgfqpoint{1.260220in}{2.350219in}}%
\pgfpathlineto{\pgfqpoint{1.301549in}{2.426769in}}%
\pgfpathlineto{\pgfqpoint{1.350406in}{2.498862in}}%
\pgfpathlineto{\pgfqpoint{1.406250in}{2.565762in}}%
\pgfpathlineto{\pgfqpoint{1.468447in}{2.626768in}}%
\pgfpathlineto{\pgfqpoint{1.536427in}{2.681345in}}%
\pgfpathlineto{\pgfqpoint{1.609538in}{2.728932in}}%
\pgfpathlineto{\pgfqpoint{1.687047in}{2.769004in}}%
\pgfpathlineto{\pgfqpoint{1.768193in}{2.801133in}}%
\pgfpathlineto{\pgfqpoint{1.852181in}{2.824987in}}%
\pgfpathlineto{\pgfqpoint{1.938184in}{2.840332in}}%
\pgfpathlineto{\pgfqpoint{2.025344in}{2.847031in}}%
\pgfpathlineto{\pgfqpoint{2.112771in}{2.845043in}}%
\pgfpathlineto{\pgfqpoint{2.199542in}{2.834423in}}%
\pgfpathlineto{\pgfqpoint{2.284779in}{2.815314in}}%
\pgfpathlineto{\pgfqpoint{2.367717in}{2.787890in}}%
\pgfpathlineto{\pgfqpoint{2.447473in}{2.752394in}}%
\pgfpathlineto{\pgfqpoint{2.523222in}{2.709151in}}%
\pgfpathlineto{\pgfqpoint{2.594197in}{2.658566in}}%
\pgfpathlineto{\pgfqpoint{2.659696in}{2.601128in}}%
\pgfpathlineto{\pgfqpoint{2.719075in}{2.537408in}}%
\pgfpathlineto{\pgfqpoint{2.771756in}{2.468059in}}%
\pgfpathlineto{\pgfqpoint{2.817220in}{2.393813in}}%
\pgfpathlineto{\pgfqpoint{2.855011in}{2.315488in}}%
\pgfpathlineto{\pgfqpoint{2.884791in}{2.233859in}}%
\pgfpathlineto{\pgfqpoint{2.906275in}{2.149675in}}%
\pgfpathlineto{\pgfqpoint{2.919192in}{2.063803in}}%
\pgfpathlineto{\pgfqpoint{2.923375in}{1.977106in}}%
\pgfpathlineto{\pgfqpoint{2.918756in}{1.890441in}}%
\pgfpathlineto{\pgfqpoint{2.905375in}{1.804659in}}%
\pgfpathlineto{\pgfqpoint{2.883371in}{1.720607in}}%
\pgfpathlineto{\pgfqpoint{2.852988in}{1.639124in}}%
\pgfpathlineto{\pgfqpoint{2.814573in}{1.561045in}}%
\pgfpathlineto{\pgfqpoint{2.768574in}{1.487197in}}%
\pgfpathlineto{\pgfqpoint{2.715440in}{1.418264in}}%
\pgfpathlineto{\pgfqpoint{2.655661in}{1.354918in}}%
\pgfpathlineto{\pgfqpoint{2.589826in}{1.297842in}}%
\pgfpathlineto{\pgfqpoint{2.518578in}{1.247629in}}%
\pgfpathlineto{\pgfqpoint{2.442614in}{1.204790in}}%
\pgfpathlineto{\pgfqpoint{2.362680in}{1.169745in}}%
\pgfpathlineto{\pgfqpoint{2.279577in}{1.142830in}}%
\pgfpathlineto{\pgfqpoint{2.194160in}{1.124294in}}%
\pgfpathlineto{\pgfqpoint{2.107333in}{1.114298in}}%
\pgfpathlineto{\pgfqpoint{2.019963in}{1.112907in}}%
\pgfpathlineto{\pgfqpoint{1.932857in}{1.120132in}}%
\pgfpathlineto{\pgfqpoint{1.846931in}{1.135950in}}%
\pgfpathlineto{\pgfqpoint{1.763063in}{1.160238in}}%
\pgfpathlineto{\pgfqpoint{1.682094in}{1.192783in}}%
\pgfpathlineto{\pgfqpoint{1.604822in}{1.233275in}}%
\pgfpathlineto{\pgfqpoint{1.532005in}{1.281309in}}%
\pgfpathlineto{\pgfqpoint{1.464363in}{1.336387in}}%
\pgfpathlineto{\pgfqpoint{1.402572in}{1.397915in}}%
\pgfpathlineto{\pgfqpoint{1.347273in}{1.465205in}}%
\pgfpathlineto{\pgfqpoint{1.298990in}{1.537567in}}%
\pgfpathlineto{\pgfqpoint{1.258165in}{1.614355in}}%
\pgfpathlineto{\pgfqpoint{1.225264in}{1.694789in}}%
\pgfpathlineto{\pgfqpoint{1.200653in}{1.778068in}}%
\pgfpathlineto{\pgfqpoint{1.184599in}{1.863374in}}%
\pgfpathlineto{\pgfqpoint{1.177266in}{1.949868in}}%
\pgfpathlineto{\pgfqpoint{1.178718in}{2.036697in}}%
\pgfpathlineto{\pgfqpoint{1.188919in}{2.122986in}}%
\pgfpathlineto{\pgfqpoint{1.207729in}{2.207844in}}%
\pgfpathlineto{\pgfqpoint{1.234910in}{2.290367in}}%
\pgfpathlineto{\pgfqpoint{1.270195in}{2.369807in}}%
\pgfpathlineto{\pgfqpoint{1.313261in}{2.445369in}}%
\pgfpathlineto{\pgfqpoint{1.363699in}{2.516253in}}%
\pgfpathlineto{\pgfqpoint{1.421028in}{2.581730in}}%
\pgfpathlineto{\pgfqpoint{1.484696in}{2.641145in}}%
\pgfpathlineto{\pgfqpoint{1.554075in}{2.693911in}}%
\pgfpathlineto{\pgfqpoint{1.628468in}{2.739516in}}%
\pgfpathlineto{\pgfqpoint{1.707103in}{2.777518in}}%
\pgfpathlineto{\pgfqpoint{1.789136in}{2.807550in}}%
\pgfpathlineto{\pgfqpoint{1.873715in}{2.829335in}}%
\pgfpathlineto{\pgfqpoint{1.960081in}{2.842689in}}%
\pgfpathlineto{\pgfqpoint{2.047341in}{2.847424in}}%
\pgfpathlineto{\pgfqpoint{2.134608in}{2.843451in}}%
\pgfpathlineto{\pgfqpoint{2.221013in}{2.830779in}}%
\pgfpathlineto{\pgfqpoint{2.305706in}{2.809522in}}%
\pgfpathlineto{\pgfqpoint{2.387854in}{2.779891in}}%
\pgfpathlineto{\pgfqpoint{2.466642in}{2.742202in}}%
\pgfpathlineto{\pgfqpoint{2.541273in}{2.696869in}}%
\pgfpathlineto{\pgfqpoint{2.610969in}{2.644409in}}%
\pgfpathlineto{\pgfqpoint{2.675037in}{2.585380in}}%
\pgfpathlineto{\pgfqpoint{2.732905in}{2.520304in}}%
\pgfpathlineto{\pgfqpoint{2.783938in}{2.449830in}}%
\pgfpathlineto{\pgfqpoint{2.827592in}{2.374652in}}%
\pgfpathlineto{\pgfqpoint{2.863415in}{2.295506in}}%
\pgfpathlineto{\pgfqpoint{2.891048in}{2.213167in}}%
\pgfpathlineto{\pgfqpoint{2.910227in}{2.128453in}}%
\pgfpathlineto{\pgfqpoint{2.920779in}{2.042220in}}%
\pgfpathlineto{\pgfqpoint{2.922629in}{1.955368in}}%
\pgfpathlineto{\pgfqpoint{2.915791in}{1.868818in}}%
\pgfpathlineto{\pgfqpoint{2.900354in}{1.783345in}}%
\pgfpathlineto{\pgfqpoint{2.876439in}{1.699822in}}%
\pgfpathlineto{\pgfqpoint{2.844247in}{1.619123in}}%
\pgfpathlineto{\pgfqpoint{2.804069in}{1.542067in}}%
\pgfpathlineto{\pgfqpoint{2.756286in}{1.469421in}}%
\pgfpathlineto{\pgfqpoint{2.701364in}{1.401901in}}%
\pgfpathlineto{\pgfqpoint{2.639860in}{1.340167in}}%
\pgfpathlineto{\pgfqpoint{2.572420in}{1.284828in}}%
\pgfpathlineto{\pgfqpoint{2.499777in}{1.236440in}}%
\pgfpathlineto{\pgfqpoint{2.422729in}{1.195491in}}%
\pgfpathlineto{\pgfqpoint{2.341957in}{1.162323in}}%
\pgfpathlineto{\pgfqpoint{2.258269in}{1.137310in}}%
\pgfpathlineto{\pgfqpoint{2.172511in}{1.120749in}}%
\pgfpathlineto{\pgfqpoint{2.085534in}{1.112835in}}%
\pgfpathlineto{\pgfqpoint{1.998192in}{1.113662in}}%
\pgfpathlineto{\pgfqpoint{1.911345in}{1.123221in}}%
\pgfpathlineto{\pgfqpoint{1.825856in}{1.141400in}}%
\pgfpathlineto{\pgfqpoint{1.742591in}{1.167985in}}%
\pgfpathlineto{\pgfqpoint{1.662423in}{1.202660in}}%
\pgfpathlineto{\pgfqpoint{1.586172in}{1.245039in}}%
\pgfpathlineto{\pgfqpoint{1.514520in}{1.294747in}}%
\pgfpathlineto{\pgfqpoint{1.448227in}{1.351302in}}%
\pgfpathlineto{\pgfqpoint{1.387987in}{1.414152in}}%
\pgfpathlineto{\pgfqpoint{1.334414in}{1.482688in}}%
\pgfpathlineto{\pgfqpoint{1.288039in}{1.556237in}}%
\pgfpathlineto{\pgfqpoint{1.249313in}{1.634067in}}%
\pgfpathlineto{\pgfqpoint{1.218607in}{1.715383in}}%
\pgfpathlineto{\pgfqpoint{1.196209in}{1.799331in}}%
\pgfpathlineto{\pgfqpoint{1.182316in}{1.885034in}}%
\pgfpathlineto{\pgfqpoint{1.177036in}{1.971716in}}%
\pgfpathlineto{\pgfqpoint{1.180467in}{2.058479in}}%
\pgfpathlineto{\pgfqpoint{1.192619in}{2.144433in}}%
\pgfpathlineto{\pgfqpoint{1.213403in}{2.228718in}}%
\pgfpathlineto{\pgfqpoint{1.242629in}{2.310501in}}%
\pgfpathlineto{\pgfqpoint{1.280008in}{2.388979in}}%
\pgfpathlineto{\pgfqpoint{1.325152in}{2.463377in}}%
\pgfpathlineto{\pgfqpoint{1.377575in}{2.532949in}}%
\pgfpathlineto{\pgfqpoint{1.436689in}{2.596978in}}%
\pgfpathlineto{\pgfqpoint{1.501847in}{2.654812in}}%
\pgfpathlineto{\pgfqpoint{1.572478in}{2.705951in}}%
\pgfpathlineto{\pgfqpoint{1.647876in}{2.749836in}}%
\pgfpathlineto{\pgfqpoint{1.727293in}{2.785985in}}%
\pgfpathlineto{\pgfqpoint{1.809948in}{2.814015in}}%
\pgfpathlineto{\pgfqpoint{1.895032in}{2.833642in}}%
\pgfpathlineto{\pgfqpoint{1.981704in}{2.844676in}}%
\pgfpathlineto{\pgfqpoint{2.069094in}{2.847026in}}%
\pgfpathlineto{\pgfqpoint{2.156301in}{2.840700in}}%
\pgfpathlineto{\pgfqpoint{2.242397in}{2.825799in}}%
\pgfpathlineto{\pgfqpoint{2.326571in}{2.802496in}}%
\pgfpathlineto{\pgfqpoint{2.408014in}{2.770996in}}%
\pgfpathlineto{\pgfqpoint{2.485860in}{2.731582in}}%
\pgfpathlineto{\pgfqpoint{2.559310in}{2.684621in}}%
\pgfpathlineto{\pgfqpoint{2.627626in}{2.630558in}}%
\pgfpathlineto{\pgfqpoint{2.690133in}{2.569922in}}%
\pgfpathlineto{\pgfqpoint{2.746221in}{2.503318in}}%
\pgfpathlineto{\pgfqpoint{2.795341in}{2.431437in}}%
\pgfpathlineto{\pgfqpoint{2.837010in}{2.355046in}}%
\pgfpathlineto{\pgfqpoint{2.870811in}{2.274985in}}%
\pgfpathlineto{\pgfqpoint{2.896466in}{2.191981in}}%
\pgfpathlineto{\pgfqpoint{2.913688in}{2.106851in}}%
\pgfpathlineto{\pgfqpoint{2.922255in}{2.020466in}}%
\pgfpathlineto{\pgfqpoint{2.922046in}{1.933687in}}%
\pgfpathlineto{\pgfqpoint{2.913045in}{1.847371in}}%
\pgfpathlineto{\pgfqpoint{2.895338in}{1.762366in}}%
\pgfpathlineto{\pgfqpoint{2.869114in}{1.679511in}}%
\pgfpathlineto{\pgfqpoint{2.834666in}{1.599642in}}%
\pgfpathlineto{\pgfqpoint{2.792387in}{1.523583in}}%
\pgfpathlineto{\pgfqpoint{2.742762in}{1.452133in}}%
\pgfpathlineto{\pgfqpoint{2.686236in}{1.385919in}}%
\pgfpathlineto{\pgfqpoint{2.623355in}{1.325634in}}%
\pgfpathlineto{\pgfqpoint{2.554738in}{1.271922in}}%
\pgfpathlineto{\pgfqpoint{2.481055in}{1.225338in}}%
\pgfpathlineto{\pgfqpoint{2.403025in}{1.186349in}}%
\pgfpathlineto{\pgfqpoint{2.321421in}{1.155337in}}%
\pgfpathlineto{\pgfqpoint{2.237065in}{1.132593in}}%
\pgfpathlineto{\pgfqpoint{2.150832in}{1.118324in}}%
\pgfpathlineto{\pgfqpoint{2.063643in}{1.112646in}}%
\pgfpathlineto{\pgfqpoint{1.976308in}{1.115578in}}%
\pgfpathlineto{\pgfqpoint{1.889688in}{1.127115in}}%
\pgfpathlineto{\pgfqpoint{1.804689in}{1.147186in}}%
\pgfpathlineto{\pgfqpoint{1.722173in}{1.175625in}}%
\pgfpathlineto{\pgfqpoint{1.642963in}{1.212173in}}%
\pgfpathlineto{\pgfqpoint{1.567839in}{1.256474in}}%
\pgfpathlineto{\pgfqpoint{1.497538in}{1.308084in}}%
\pgfpathlineto{\pgfqpoint{1.432756in}{1.366458in}}%
\pgfpathlineto{\pgfqpoint{1.374146in}{1.430964in}}%
\pgfpathlineto{\pgfqpoint{1.322316in}{1.500873in}}%
\pgfpathlineto{\pgfqpoint{1.277720in}{1.575537in}}%
\pgfpathlineto{\pgfqpoint{1.240813in}{1.654243in}}%
\pgfpathlineto{\pgfqpoint{1.212017in}{1.736195in}}%
\pgfpathlineto{\pgfqpoint{1.191651in}{1.820580in}}%
\pgfpathlineto{\pgfqpoint{1.179935in}{1.906571in}}%
\pgfpathlineto{\pgfqpoint{1.176987in}{1.993321in}}%
\pgfpathlineto{\pgfqpoint{1.182823in}{2.079971in}}%
\pgfpathlineto{\pgfqpoint{1.197357in}{2.165643in}}%
\pgfpathlineto{\pgfqpoint{1.220403in}{2.249442in}}%
\pgfpathlineto{\pgfqpoint{1.251682in}{2.330493in}}%
\pgfpathlineto{\pgfqpoint{1.290911in}{2.408079in}}%
\pgfpathlineto{\pgfqpoint{1.337720in}{2.481389in}}%
\pgfpathlineto{\pgfqpoint{1.391663in}{2.549653in}}%
\pgfpathlineto{\pgfqpoint{1.452223in}{2.612173in}}%
\pgfpathlineto{\pgfqpoint{1.518812in}{2.668327in}}%
\pgfpathlineto{\pgfqpoint{1.590769in}{2.717565in}}%
\pgfpathlineto{\pgfqpoint{1.667367in}{2.759409in}}%
\pgfpathlineto{\pgfqpoint{1.747802in}{2.793456in}}%
\pgfpathlineto{\pgfqpoint{1.831203in}{2.819376in}}%
\pgfpathlineto{\pgfqpoint{1.916769in}{2.836955in}}%
\pgfpathlineto{\pgfqpoint{2.003685in}{2.846009in}}%
\pgfpathlineto{\pgfqpoint{2.091053in}{2.846396in}}%
\pgfpathlineto{\pgfqpoint{2.177988in}{2.838073in}}%
\pgfpathlineto{\pgfqpoint{2.263631in}{2.821099in}}%
\pgfpathlineto{\pgfqpoint{2.347140in}{2.795635in}}%
\pgfpathlineto{\pgfqpoint{2.427691in}{2.761939in}}%
\pgfpathlineto{\pgfqpoint{2.504483in}{2.720374in}}%
\pgfpathlineto{\pgfqpoint{2.576732in}{2.671402in}}%
\pgfpathlineto{\pgfqpoint{2.643676in}{2.615585in}}%
\pgfpathlineto{\pgfqpoint{2.704706in}{2.553459in}}%
\pgfpathlineto{\pgfqpoint{2.759221in}{2.485603in}}%
\pgfpathlineto{\pgfqpoint{2.806625in}{2.412694in}}%
\pgfpathlineto{\pgfqpoint{2.846416in}{2.335448in}}%
\pgfpathlineto{\pgfqpoint{2.878185in}{2.254622in}}%
\pgfpathlineto{\pgfqpoint{2.901618in}{2.171008in}}%
\pgfpathlineto{\pgfqpoint{2.916496in}{2.085443in}}%
\pgfpathlineto{\pgfqpoint{2.922693in}{1.998797in}}%
\pgfpathlineto{\pgfqpoint{2.920177in}{1.911984in}}%
\pgfpathlineto{\pgfqpoint{2.909012in}{1.825895in}}%
\pgfpathlineto{\pgfqpoint{2.889306in}{1.741299in}}%
\pgfpathlineto{\pgfqpoint{2.861220in}{1.659089in}}%
\pgfpathlineto{\pgfqpoint{2.825000in}{1.580116in}}%
\pgfpathlineto{\pgfqpoint{2.780980in}{1.505178in}}%
\pgfpathlineto{\pgfqpoint{2.729580in}{1.435017in}}%
\pgfpathlineto{\pgfqpoint{2.671311in}{1.370322in}}%
\pgfpathlineto{\pgfqpoint{2.606768in}{1.311727in}}%
\pgfpathlineto{\pgfqpoint{2.536634in}{1.259812in}}%
\pgfpathlineto{\pgfqpoint{2.461683in}{1.215100in}}%
\pgfpathlineto{\pgfqpoint{2.382693in}{1.178020in}}%
\pgfpathlineto{\pgfqpoint{2.300364in}{1.148898in}}%
\pgfpathlineto{\pgfqpoint{2.215541in}{1.128082in}}%
\pgfpathlineto{\pgfqpoint{2.129075in}{1.115821in}}%
\pgfpathlineto{\pgfqpoint{2.041821in}{1.112265in}}%
\pgfpathlineto{\pgfqpoint{1.954635in}{1.117456in}}%
\pgfpathlineto{\pgfqpoint{1.868378in}{1.131338in}}%
\pgfpathlineto{\pgfqpoint{1.783910in}{1.153752in}}%
\pgfpathlineto{\pgfqpoint{1.702095in}{1.184434in}}%
\pgfpathlineto{\pgfqpoint{1.623800in}{1.223022in}}%
\pgfpathlineto{\pgfqpoint{1.549776in}{1.269121in}}%
\pgfpathlineto{\pgfqpoint{1.480721in}{1.322317in}}%
\pgfpathlineto{\pgfqpoint{1.417376in}{1.382086in}}%
\pgfpathlineto{\pgfqpoint{1.360399in}{1.447849in}}%
\pgfpathlineto{\pgfqpoint{1.310365in}{1.518965in}}%
\pgfpathlineto{\pgfqpoint{1.267768in}{1.594732in}}%
\pgfpathlineto{\pgfqpoint{1.233019in}{1.674393in}}%
\pgfpathlineto{\pgfqpoint{1.206448in}{1.757129in}}%
\pgfpathlineto{\pgfqpoint{1.188301in}{1.842061in}}%
\pgfpathlineto{\pgfqpoint{1.178723in}{1.928340in}}%
\pgfpathlineto{\pgfqpoint{1.177803in}{2.015165in}}%
\pgfpathlineto{\pgfqpoint{1.185600in}{2.101629in}}%
\pgfpathlineto{\pgfqpoint{1.202076in}{2.186849in}}%
\pgfpathlineto{\pgfqpoint{1.227096in}{2.269977in}}%
\pgfpathlineto{\pgfqpoint{1.260423in}{2.350193in}}%
\pgfpathlineto{\pgfqpoint{1.301724in}{2.426709in}}%
\pgfpathlineto{\pgfqpoint{1.350565in}{2.498767in}}%
\pgfpathlineto{\pgfqpoint{1.406417in}{2.565642in}}%
\pgfpathlineto{\pgfqpoint{1.468648in}{2.626637in}}%
\pgfpathlineto{\pgfqpoint{1.536618in}{2.681165in}}%
\pgfpathlineto{\pgfqpoint{1.609722in}{2.728731in}}%
\pgfpathlineto{\pgfqpoint{1.687220in}{2.768800in}}%
\pgfpathlineto{\pgfqpoint{1.768346in}{2.800936in}}%
\pgfpathlineto{\pgfqpoint{1.852303in}{2.824801in}}%
\pgfpathlineto{\pgfqpoint{1.938268in}{2.840154in}}%
\pgfpathlineto{\pgfqpoint{2.025389in}{2.846852in}}%
\pgfpathlineto{\pgfqpoint{2.112786in}{2.844851in}}%
\pgfpathlineto{\pgfqpoint{2.199553in}{2.834204in}}%
\pgfpathlineto{\pgfqpoint{2.284766in}{2.815061in}}%
\pgfpathlineto{\pgfqpoint{2.367664in}{2.787615in}}%
\pgfpathlineto{\pgfqpoint{2.447403in}{2.752112in}}%
\pgfpathlineto{\pgfqpoint{2.523145in}{2.708878in}}%
\pgfpathlineto{\pgfqpoint{2.594114in}{2.658317in}}%
\pgfpathlineto{\pgfqpoint{2.659601in}{2.600916in}}%
\pgfpathlineto{\pgfqpoint{2.718961in}{2.537236in}}%
\pgfpathlineto{\pgfqpoint{2.771617in}{2.467922in}}%
\pgfpathlineto{\pgfqpoint{2.817054in}{2.393696in}}%
\pgfpathlineto{\pgfqpoint{2.854822in}{2.315358in}}%
\pgfpathlineto{\pgfqpoint{2.884559in}{2.233744in}}%
\pgfpathlineto{\pgfqpoint{2.906017in}{2.149577in}}%
\pgfpathlineto{\pgfqpoint{2.918929in}{2.063716in}}%
\pgfpathlineto{\pgfqpoint{2.923123in}{1.977033in}}%
\pgfpathlineto{\pgfqpoint{2.918525in}{1.890389in}}%
\pgfpathlineto{\pgfqpoint{2.905167in}{1.804638in}}%
\pgfpathlineto{\pgfqpoint{2.883183in}{1.720622in}}%
\pgfpathlineto{\pgfqpoint{2.852810in}{1.639175in}}%
\pgfpathlineto{\pgfqpoint{2.814386in}{1.561122in}}%
\pgfpathlineto{\pgfqpoint{2.768354in}{1.487278in}}%
\pgfpathlineto{\pgfqpoint{2.715216in}{1.418390in}}%
\pgfpathlineto{\pgfqpoint{2.655439in}{1.355068in}}%
\pgfpathlineto{\pgfqpoint{2.589614in}{1.297998in}}%
\pgfpathlineto{\pgfqpoint{2.518386in}{1.247785in}}%
\pgfpathlineto{\pgfqpoint{2.442451in}{1.204943in}}%
\pgfpathlineto{\pgfqpoint{2.362554in}{1.169900in}}%
\pgfpathlineto{\pgfqpoint{2.279487in}{1.142996in}}%
\pgfpathlineto{\pgfqpoint{2.194093in}{1.124479in}}%
\pgfpathlineto{\pgfqpoint{2.107264in}{1.114511in}}%
\pgfpathlineto{\pgfqpoint{2.019920in}{1.113162in}}%
\pgfpathlineto{\pgfqpoint{1.932844in}{1.120418in}}%
\pgfpathlineto{\pgfqpoint{1.846929in}{1.136243in}}%
\pgfpathlineto{\pgfqpoint{1.763069in}{1.160519in}}%
\pgfpathlineto{\pgfqpoint{1.682110in}{1.193036in}}%
\pgfpathlineto{\pgfqpoint{1.604857in}{1.233491in}}%
\pgfpathlineto{\pgfqpoint{1.532069in}{1.281487in}}%
\pgfpathlineto{\pgfqpoint{1.464460in}{1.336536in}}%
\pgfpathlineto{\pgfqpoint{1.402702in}{1.398056in}}%
\pgfpathlineto{\pgfqpoint{1.347421in}{1.465372in}}%
\pgfpathlineto{\pgfqpoint{1.299181in}{1.537739in}}%
\pgfpathlineto{\pgfqpoint{1.258391in}{1.614517in}}%
\pgfpathlineto{\pgfqpoint{1.225498in}{1.694943in}}%
\pgfpathlineto{\pgfqpoint{1.200878in}{1.778208in}}%
\pgfpathlineto{\pgfqpoint{1.184807in}{1.863489in}}%
\pgfpathlineto{\pgfqpoint{1.177456in}{1.949949in}}%
\pgfpathlineto{\pgfqpoint{1.178897in}{2.036736in}}%
\pgfpathlineto{\pgfqpoint{1.189098in}{2.122986in}}%
\pgfpathlineto{\pgfqpoint{1.207927in}{2.207822in}}%
\pgfpathlineto{\pgfqpoint{1.235148in}{2.290351in}}%
\pgfpathlineto{\pgfqpoint{1.270454in}{2.369737in}}%
\pgfpathlineto{\pgfqpoint{1.313532in}{2.445269in}}%
\pgfpathlineto{\pgfqpoint{1.363969in}{2.516141in}}%
\pgfpathlineto{\pgfqpoint{1.421283in}{2.581613in}}%
\pgfpathlineto{\pgfqpoint{1.484920in}{2.641022in}}%
\pgfpathlineto{\pgfqpoint{1.554259in}{2.693778in}}%
\pgfpathlineto{\pgfqpoint{1.628610in}{2.739366in}}%
\pgfpathlineto{\pgfqpoint{1.707214in}{2.777346in}}%
\pgfpathlineto{\pgfqpoint{1.789241in}{2.807353in}}%
\pgfpathlineto{\pgfqpoint{1.873808in}{2.829103in}}%
\pgfpathlineto{\pgfqpoint{1.960151in}{2.842426in}}%
\pgfpathlineto{\pgfqpoint{2.047398in}{2.847153in}}%
\pgfpathlineto{\pgfqpoint{2.134653in}{2.843188in}}%
\pgfpathlineto{\pgfqpoint{2.221041in}{2.830537in}}%
\pgfpathlineto{\pgfqpoint{2.305709in}{2.809306in}}%
\pgfpathlineto{\pgfqpoint{2.387824in}{2.779701in}}%
\pgfpathlineto{\pgfqpoint{2.466576in}{2.742030in}}%
\pgfpathlineto{\pgfqpoint{2.541176in}{2.696699in}}%
\pgfpathlineto{\pgfqpoint{2.610859in}{2.644216in}}%
\pgfpathlineto{\pgfqpoint{2.674895in}{2.585175in}}%
\pgfpathlineto{\pgfqpoint{2.732729in}{2.520104in}}%
\pgfpathlineto{\pgfqpoint{2.783752in}{2.449638in}}%
\pgfpathlineto{\pgfqpoint{2.827407in}{2.374476in}}%
\pgfpathlineto{\pgfqpoint{2.863237in}{2.295356in}}%
\pgfpathlineto{\pgfqpoint{2.890874in}{2.213052in}}%
\pgfpathlineto{\pgfqpoint{2.910051in}{2.128375in}}%
\pgfpathlineto{\pgfqpoint{2.920593in}{2.042175in}}%
\pgfpathlineto{\pgfqpoint{2.922421in}{1.955338in}}%
\pgfpathlineto{\pgfqpoint{2.915550in}{1.868786in}}%
\pgfpathlineto{\pgfqpoint{2.900082in}{1.783361in}}%
\pgfpathlineto{\pgfqpoint{2.876152in}{1.699860in}}%
\pgfpathlineto{\pgfqpoint{2.843963in}{1.619170in}}%
\pgfpathlineto{\pgfqpoint{2.803805in}{1.542121in}}%
\pgfpathlineto{\pgfqpoint{2.756053in}{1.469487in}}%
\pgfpathlineto{\pgfqpoint{2.701170in}{1.401985in}}%
\pgfpathlineto{\pgfqpoint{2.639704in}{1.340277in}}%
\pgfpathlineto{\pgfqpoint{2.572288in}{1.284968in}}%
\pgfpathlineto{\pgfqpoint{2.499645in}{1.236606in}}%
\pgfpathlineto{\pgfqpoint{2.422579in}{1.195683in}}%
\pgfpathlineto{\pgfqpoint{2.341828in}{1.162558in}}%
\pgfpathlineto{\pgfqpoint{2.258151in}{1.137560in}}%
\pgfpathlineto{\pgfqpoint{2.172405in}{1.120996in}}%
\pgfpathlineto{\pgfqpoint{2.085446in}{1.113068in}}%
\pgfpathlineto{\pgfqpoint{1.998133in}{1.113874in}}%
\pgfpathlineto{\pgfqpoint{1.911322in}{1.123414in}}%
\pgfpathlineto{\pgfqpoint{1.825872in}{1.141581in}}%
\pgfpathlineto{\pgfqpoint{1.742640in}{1.168170in}}%
\pgfpathlineto{\pgfqpoint{1.662484in}{1.202869in}}%
\pgfpathlineto{\pgfqpoint{1.586254in}{1.245271in}}%
\pgfpathlineto{\pgfqpoint{1.514637in}{1.294982in}}%
\pgfpathlineto{\pgfqpoint{1.448358in}{1.351532in}}%
\pgfpathlineto{\pgfqpoint{1.388120in}{1.414368in}}%
\pgfpathlineto{\pgfqpoint{1.334547in}{1.482878in}}%
\pgfpathlineto{\pgfqpoint{1.288176in}{1.556393in}}%
\pgfpathlineto{\pgfqpoint{1.249462in}{1.634188in}}%
\pgfpathlineto{\pgfqpoint{1.218775in}{1.715477in}}%
\pgfpathlineto{\pgfqpoint{1.196404in}{1.799421in}}%
\pgfpathlineto{\pgfqpoint{1.182553in}{1.885122in}}%
\pgfpathlineto{\pgfqpoint{1.177315in}{1.971769in}}%
\pgfpathlineto{\pgfqpoint{1.180761in}{2.058519in}}%
\pgfpathlineto{\pgfqpoint{1.192906in}{2.144466in}}%
\pgfpathlineto{\pgfqpoint{1.213665in}{2.228741in}}%
\pgfpathlineto{\pgfqpoint{1.242855in}{2.310504in}}%
\pgfpathlineto{\pgfqpoint{1.280197in}{2.388952in}}%
\pgfpathlineto{\pgfqpoint{1.325312in}{2.463312in}}%
\pgfpathlineto{\pgfqpoint{1.377724in}{2.532849in}}%
\pgfpathlineto{\pgfqpoint{1.436860in}{2.596856in}}%
\pgfpathlineto{\pgfqpoint{1.502049in}{2.654664in}}%
\pgfpathlineto{\pgfqpoint{1.572672in}{2.705756in}}%
\pgfpathlineto{\pgfqpoint{1.648065in}{2.749625in}}%
\pgfpathlineto{\pgfqpoint{1.727471in}{2.785779in}}%
\pgfpathlineto{\pgfqpoint{1.810103in}{2.813823in}}%
\pgfpathlineto{\pgfqpoint{1.895151in}{2.833465in}}%
\pgfpathlineto{\pgfqpoint{1.981780in}{2.844509in}}%
\pgfpathlineto{\pgfqpoint{2.069129in}{2.846856in}}%
\pgfpathlineto{\pgfqpoint{2.156311in}{2.840510in}}%
\pgfpathlineto{\pgfqpoint{2.242417in}{2.825571in}}%
\pgfpathlineto{\pgfqpoint{2.326544in}{2.802230in}}%
\pgfpathlineto{\pgfqpoint{2.407951in}{2.770706in}}%
\pgfpathlineto{\pgfqpoint{2.485786in}{2.731288in}}%
\pgfpathlineto{\pgfqpoint{2.559233in}{2.684340in}}%
\pgfpathlineto{\pgfqpoint{2.627544in}{2.630309in}}%
\pgfpathlineto{\pgfqpoint{2.690040in}{2.569715in}}%
\pgfpathlineto{\pgfqpoint{2.746107in}{2.503157in}}%
\pgfpathlineto{\pgfqpoint{2.795199in}{2.431311in}}%
\pgfpathlineto{\pgfqpoint{2.836839in}{2.354931in}}%
\pgfpathlineto{\pgfqpoint{2.870613in}{2.274847in}}%
\pgfpathlineto{\pgfqpoint{2.896217in}{2.191872in}}%
\pgfpathlineto{\pgfqpoint{2.913418in}{2.106755in}}%
\pgfpathlineto{\pgfqpoint{2.921986in}{2.020378in}}%
\pgfpathlineto{\pgfqpoint{2.921795in}{1.933613in}}%
\pgfpathlineto{\pgfqpoint{2.912822in}{1.847321in}}%
\pgfpathlineto{\pgfqpoint{2.895143in}{1.762349in}}%
\pgfpathlineto{\pgfqpoint{2.868940in}{1.679535in}}%
\pgfpathlineto{\pgfqpoint{2.834497in}{1.599702in}}%
\pgfpathlineto{\pgfqpoint{2.792199in}{1.523664in}}%
\pgfpathlineto{\pgfqpoint{2.742534in}{1.452222in}}%
\pgfpathlineto{\pgfqpoint{2.686007in}{1.386058in}}%
\pgfpathlineto{\pgfqpoint{2.623128in}{1.325792in}}%
\pgfpathlineto{\pgfqpoint{2.554523in}{1.272080in}}%
\pgfpathlineto{\pgfqpoint{2.480863in}{1.225489in}}%
\pgfpathlineto{\pgfqpoint{2.402868in}{1.186496in}}%
\pgfpathlineto{\pgfqpoint{2.321304in}{1.155485in}}%
\pgfpathlineto{\pgfqpoint{2.236985in}{1.132755in}}%
\pgfpathlineto{\pgfqpoint{2.150770in}{1.118510in}}%
\pgfpathlineto{\pgfqpoint{2.063566in}{1.112868in}}%
\pgfpathlineto{\pgfqpoint{1.976276in}{1.115850in}}%
\pgfpathlineto{\pgfqpoint{1.889681in}{1.127416in}}%
\pgfpathlineto{\pgfqpoint{1.804688in}{1.147491in}}%
\pgfpathlineto{\pgfqpoint{1.722176in}{1.175911in}}%
\pgfpathlineto{\pgfqpoint{1.642976in}{1.212423in}}%
\pgfpathlineto{\pgfqpoint{1.567873in}{1.256681in}}%
\pgfpathlineto{\pgfqpoint{1.497603in}{1.308248in}}%
\pgfpathlineto{\pgfqpoint{1.432858in}{1.366594in}}%
\pgfpathlineto{\pgfqpoint{1.374280in}{1.431101in}}%
\pgfpathlineto{\pgfqpoint{1.322467in}{1.501055in}}%
\pgfpathlineto{\pgfqpoint{1.277930in}{1.575708in}}%
\pgfpathlineto{\pgfqpoint{1.241052in}{1.654407in}}%
\pgfpathlineto{\pgfqpoint{1.212257in}{1.736352in}}%
\pgfpathlineto{\pgfqpoint{1.191876in}{1.820722in}}%
\pgfpathlineto{\pgfqpoint{1.180137in}{1.906685in}}%
\pgfpathlineto{\pgfqpoint{1.177167in}{1.993396in}}%
\pgfpathlineto{\pgfqpoint{1.182991in}{2.080000in}}%
\pgfpathlineto{\pgfqpoint{1.197530in}{2.165633in}}%
\pgfpathlineto{\pgfqpoint{1.220604in}{2.249418in}}%
\pgfpathlineto{\pgfqpoint{1.251931in}{2.330468in}}%
\pgfpathlineto{\pgfqpoint{1.291180in}{2.407997in}}%
\pgfpathlineto{\pgfqpoint{1.338001in}{2.481280in}}%
\pgfpathlineto{\pgfqpoint{1.391940in}{2.549535in}}%
\pgfpathlineto{\pgfqpoint{1.452480in}{2.612054in}}%
\pgfpathlineto{\pgfqpoint{1.519033in}{2.668204in}}%
\pgfpathlineto{\pgfqpoint{1.590946in}{2.717431in}}%
\pgfpathlineto{\pgfqpoint{1.667499in}{2.759256in}}%
\pgfpathlineto{\pgfqpoint{1.747907in}{2.793279in}}%
\pgfpathlineto{\pgfqpoint{1.831315in}{2.819173in}}%
\pgfpathlineto{\pgfqpoint{1.916852in}{2.836706in}}%
\pgfpathlineto{\pgfqpoint{2.003750in}{2.845735in}}%
\pgfpathlineto{\pgfqpoint{2.091107in}{2.846119in}}%
\pgfpathlineto{\pgfqpoint{2.178032in}{2.837811in}}%
\pgfpathlineto{\pgfqpoint{2.263656in}{2.820863in}}%
\pgfpathlineto{\pgfqpoint{2.347137in}{2.795429in}}%
\pgfpathlineto{\pgfqpoint{2.427653in}{2.761761in}}%
\pgfpathlineto{\pgfqpoint{2.504407in}{2.720212in}}%
\pgfpathlineto{\pgfqpoint{2.576628in}{2.671234in}}%
\pgfpathlineto{\pgfqpoint{2.643566in}{2.615379in}}%
\pgfpathlineto{\pgfqpoint{2.704547in}{2.553254in}}%
\pgfpathlineto{\pgfqpoint{2.759035in}{2.485402in}}%
\pgfpathlineto{\pgfqpoint{2.806435in}{2.412501in}}%
\pgfpathlineto{\pgfqpoint{2.846232in}{2.335274in}}%
\pgfpathlineto{\pgfqpoint{2.878011in}{2.254478in}}%
\pgfpathlineto{\pgfqpoint{2.901450in}{2.170903in}}%
\pgfpathlineto{\pgfqpoint{2.916325in}{2.085377in}}%
\pgfpathlineto{\pgfqpoint{2.922507in}{1.998761in}}%
\pgfpathlineto{\pgfqpoint{2.919963in}{1.911953in}}%
\pgfpathlineto{\pgfqpoint{2.908758in}{1.825880in}}%
\pgfpathlineto{\pgfqpoint{2.889022in}{1.741326in}}%
\pgfpathlineto{\pgfqpoint{2.860924in}{1.659133in}}%
\pgfpathlineto{\pgfqpoint{2.824712in}{1.580166in}}%
\pgfpathlineto{\pgfqpoint{2.780717in}{1.505233in}}%
\pgfpathlineto{\pgfqpoint{2.729355in}{1.435084in}}%
\pgfpathlineto{\pgfqpoint{2.671129in}{1.370410in}}%
\pgfpathlineto{\pgfqpoint{2.606623in}{1.311844in}}%
\pgfpathlineto{\pgfqpoint{2.536510in}{1.259959in}}%
\pgfpathlineto{\pgfqpoint{2.461544in}{1.215272in}}%
\pgfpathlineto{\pgfqpoint{2.382551in}{1.178231in}}%
\pgfpathlineto{\pgfqpoint{2.300240in}{1.149146in}}%
\pgfpathlineto{\pgfqpoint{2.215425in}{1.128339in}}%
\pgfpathlineto{\pgfqpoint{2.128971in}{1.116070in}}%
\pgfpathlineto{\pgfqpoint{2.041738in}{1.112493in}}%
\pgfpathlineto{\pgfqpoint{1.954583in}{1.117660in}}%
\pgfpathlineto{\pgfqpoint{1.868365in}{1.131521in}}%
\pgfpathlineto{\pgfqpoint{1.783937in}{1.153926in}}%
\pgfpathlineto{\pgfqpoint{1.702152in}{1.184618in}}%
\pgfpathlineto{\pgfqpoint{1.623859in}{1.223242in}}%
\pgfpathlineto{\pgfqpoint{1.549874in}{1.269357in}}%
\pgfpathlineto{\pgfqpoint{1.480849in}{1.322554in}}%
\pgfpathlineto{\pgfqpoint{1.417512in}{1.382318in}}%
\pgfpathlineto{\pgfqpoint{1.360533in}{1.448062in}}%
\pgfpathlineto{\pgfqpoint{1.310497in}{1.519147in}}%
\pgfpathlineto{\pgfqpoint{1.267903in}{1.594878in}}%
\pgfpathlineto{\pgfqpoint{1.233168in}{1.674503in}}%
\pgfpathlineto{\pgfqpoint{1.206620in}{1.757215in}}%
\pgfpathlineto{\pgfqpoint{1.188504in}{1.842154in}}%
\pgfpathlineto{\pgfqpoint{1.178977in}{1.928414in}}%
\pgfpathlineto{\pgfqpoint{1.178095in}{2.015210in}}%
\pgfpathlineto{\pgfqpoint{1.185903in}{2.101665in}}%
\pgfpathlineto{\pgfqpoint{1.202366in}{2.186881in}}%
\pgfpathlineto{\pgfqpoint{1.227354in}{2.269998in}}%
\pgfpathlineto{\pgfqpoint{1.260641in}{2.350192in}}%
\pgfpathlineto{\pgfqpoint{1.301901in}{2.426675in}}%
\pgfpathlineto{\pgfqpoint{1.350713in}{2.498694in}}%
\pgfpathlineto{\pgfqpoint{1.406561in}{2.565533in}}%
\pgfpathlineto{\pgfqpoint{1.468829in}{2.626512in}}%
\pgfpathlineto{\pgfqpoint{1.536817in}{2.680998in}}%
\pgfpathlineto{\pgfqpoint{1.609916in}{2.728522in}}%
\pgfpathlineto{\pgfqpoint{1.687410in}{2.768582in}}%
\pgfpathlineto{\pgfqpoint{1.768523in}{2.800729in}}%
\pgfpathlineto{\pgfqpoint{1.852453in}{2.824613in}}%
\pgfpathlineto{\pgfqpoint{1.938379in}{2.839984in}}%
\pgfpathlineto{\pgfqpoint{2.025454in}{2.846692in}}%
\pgfpathlineto{\pgfqpoint{2.112810in}{2.844685in}}%
\pgfpathlineto{\pgfqpoint{2.199558in}{2.834011in}}%
\pgfpathlineto{\pgfqpoint{2.284783in}{2.814820in}}%
\pgfpathlineto{\pgfqpoint{2.367621in}{2.787339in}}%
\pgfpathlineto{\pgfqpoint{2.447331in}{2.751815in}}%
\pgfpathlineto{\pgfqpoint{2.523064in}{2.708578in}}%
\pgfpathlineto{\pgfqpoint{2.594032in}{2.658036in}}%
\pgfpathlineto{\pgfqpoint{2.659515in}{2.600670in}}%
\pgfpathlineto{\pgfqpoint{2.718864in}{2.537038in}}%
\pgfpathlineto{\pgfqpoint{2.771498in}{2.467772in}}%
\pgfpathlineto{\pgfqpoint{2.816905in}{2.393579in}}%
\pgfpathlineto{\pgfqpoint{2.854644in}{2.315241in}}%
\pgfpathlineto{\pgfqpoint{2.884342in}{2.233617in}}%
\pgfpathlineto{\pgfqpoint{2.905754in}{2.149475in}}%
\pgfpathlineto{\pgfqpoint{2.918651in}{2.063624in}}%
\pgfpathlineto{\pgfqpoint{2.922852in}{1.976949in}}%
\pgfpathlineto{\pgfqpoint{2.918278in}{1.890320in}}%
\pgfpathlineto{\pgfqpoint{2.904952in}{1.804595in}}%
\pgfpathlineto{\pgfqpoint{2.882998in}{1.720616in}}%
\pgfpathlineto{\pgfqpoint{2.852644in}{1.639211in}}%
\pgfpathlineto{\pgfqpoint{2.814220in}{1.561193in}}%
\pgfpathlineto{\pgfqpoint{2.768158in}{1.487363in}}%
\pgfpathlineto{\pgfqpoint{2.714988in}{1.418499in}}%
\pgfpathlineto{\pgfqpoint{2.655210in}{1.355221in}}%
\pgfpathlineto{\pgfqpoint{2.589387in}{1.298163in}}%
\pgfpathlineto{\pgfqpoint{2.518174in}{1.247945in}}%
\pgfpathlineto{\pgfqpoint{2.442266in}{1.205094in}}%
\pgfpathlineto{\pgfqpoint{2.362407in}{1.170045in}}%
\pgfpathlineto{\pgfqpoint{2.279383in}{1.143144in}}%
\pgfpathlineto{\pgfqpoint{2.194024in}{1.124642in}}%
\pgfpathlineto{\pgfqpoint{2.107206in}{1.114704in}}%
\pgfpathlineto{\pgfqpoint{2.019849in}{1.113398in}}%
\pgfpathlineto{\pgfqpoint{1.932825in}{1.120703in}}%
\pgfpathlineto{\pgfqpoint{1.846930in}{1.136552in}}%
\pgfpathlineto{\pgfqpoint{1.763074in}{1.160826in}}%
\pgfpathlineto{\pgfqpoint{1.682118in}{1.193320in}}%
\pgfpathlineto{\pgfqpoint{1.604875in}{1.233735in}}%
\pgfpathlineto{\pgfqpoint{1.532109in}{1.281684in}}%
\pgfpathlineto{\pgfqpoint{1.464534in}{1.336689in}}%
\pgfpathlineto{\pgfqpoint{1.402814in}{1.398184in}}%
\pgfpathlineto{\pgfqpoint{1.347563in}{1.465511in}}%
\pgfpathlineto{\pgfqpoint{1.299347in}{1.537922in}}%
\pgfpathlineto{\pgfqpoint{1.258617in}{1.614682in}}%
\pgfpathlineto{\pgfqpoint{1.225747in}{1.695103in}}%
\pgfpathlineto{\pgfqpoint{1.201123in}{1.778361in}}%
\pgfpathlineto{\pgfqpoint{1.185031in}{1.863625in}}%
\pgfpathlineto{\pgfqpoint{1.177653in}{1.950053in}}%
\pgfpathlineto{\pgfqpoint{1.179071in}{2.036798in}}%
\pgfpathlineto{\pgfqpoint{1.189262in}{2.123002in}}%
\pgfpathlineto{\pgfqpoint{1.208100in}{2.207802in}}%
\pgfpathlineto{\pgfqpoint{1.235358in}{2.290325in}}%
\pgfpathlineto{\pgfqpoint{1.270708in}{2.369692in}}%
\pgfpathlineto{\pgfqpoint{1.313804in}{2.445172in}}%
\pgfpathlineto{\pgfqpoint{1.364251in}{2.516022in}}%
\pgfpathlineto{\pgfqpoint{1.421557in}{2.581489in}}%
\pgfpathlineto{\pgfqpoint{1.485170in}{2.640896in}}%
\pgfpathlineto{\pgfqpoint{1.554471in}{2.693648in}}%
\pgfpathlineto{\pgfqpoint{1.628775in}{2.739224in}}%
\pgfpathlineto{\pgfqpoint{1.707336in}{2.777184in}}%
\pgfpathlineto{\pgfqpoint{1.789342in}{2.807167in}}%
\pgfpathlineto{\pgfqpoint{1.873916in}{2.828886in}}%
\pgfpathlineto{\pgfqpoint{1.960222in}{2.842165in}}%
\pgfpathlineto{\pgfqpoint{2.047454in}{2.846873in}}%
\pgfpathlineto{\pgfqpoint{2.134700in}{2.842911in}}%
\pgfpathlineto{\pgfqpoint{2.221076in}{2.830280in}}%
\pgfpathlineto{\pgfqpoint{2.305723in}{2.809078in}}%
\pgfpathlineto{\pgfqpoint{2.387808in}{2.779505in}}%
\pgfpathlineto{\pgfqpoint{2.466524in}{2.741860in}}%
\pgfpathlineto{\pgfqpoint{2.541088in}{2.696540in}}%
\pgfpathlineto{\pgfqpoint{2.610747in}{2.644043in}}%
\pgfpathlineto{\pgfqpoint{2.674771in}{2.584966in}}%
\pgfpathlineto{\pgfqpoint{2.732555in}{2.519903in}}%
\pgfpathlineto{\pgfqpoint{2.783558in}{2.449441in}}%
\pgfpathlineto{\pgfqpoint{2.827213in}{2.374290in}}%
\pgfpathlineto{\pgfqpoint{2.863052in}{2.295191in}}%
\pgfpathlineto{\pgfqpoint{2.890700in}{2.212920in}}%
\pgfpathlineto{\pgfqpoint{2.909883in}{2.128284in}}%
\pgfpathlineto{\pgfqpoint{2.920420in}{2.042123in}}%
\pgfpathlineto{\pgfqpoint{2.922229in}{1.955312in}}%
\pgfpathlineto{\pgfqpoint{2.915325in}{1.868755in}}%
\pgfpathlineto{\pgfqpoint{2.899817in}{1.783365in}}%
\pgfpathlineto{\pgfqpoint{2.875862in}{1.699899in}}%
\pgfpathlineto{\pgfqpoint{2.843667in}{1.619222in}}%
\pgfpathlineto{\pgfqpoint{2.803522in}{1.542177in}}%
\pgfpathlineto{\pgfqpoint{2.755799in}{1.469549in}}%
\pgfpathlineto{\pgfqpoint{2.700957in}{1.402061in}}%
\pgfpathlineto{\pgfqpoint{2.639534in}{1.340376in}}%
\pgfpathlineto{\pgfqpoint{2.572154in}{1.285096in}}%
\pgfpathlineto{\pgfqpoint{2.499523in}{1.236764in}}%
\pgfpathlineto{\pgfqpoint{2.422431in}{1.195862in}}%
\pgfpathlineto{\pgfqpoint{2.341698in}{1.162786in}}%
\pgfpathlineto{\pgfqpoint{2.258035in}{1.137817in}}%
\pgfpathlineto{\pgfqpoint{2.172297in}{1.121256in}}%
\pgfpathlineto{\pgfqpoint{2.085352in}{1.113313in}}%
\pgfpathlineto{\pgfqpoint{1.998062in}{1.114097in}}%
\pgfpathlineto{\pgfqpoint{1.911286in}{1.123612in}}%
\pgfpathlineto{\pgfqpoint{1.825875in}{1.141760in}}%
\pgfpathlineto{\pgfqpoint{1.742682in}{1.168343in}}%
\pgfpathlineto{\pgfqpoint{1.662549in}{1.203058in}}%
\pgfpathlineto{\pgfqpoint{1.586320in}{1.245501in}}%
\pgfpathlineto{\pgfqpoint{1.514753in}{1.295216in}}%
\pgfpathlineto{\pgfqpoint{1.448495in}{1.351766in}}%
\pgfpathlineto{\pgfqpoint{1.388261in}{1.414593in}}%
\pgfpathlineto{\pgfqpoint{1.334685in}{1.483081in}}%
\pgfpathlineto{\pgfqpoint{1.288312in}{1.556564in}}%
\pgfpathlineto{\pgfqpoint{1.249602in}{1.634320in}}%
\pgfpathlineto{\pgfqpoint{1.218931in}{1.715575in}}%
\pgfpathlineto{\pgfqpoint{1.196585in}{1.799501in}}%
\pgfpathlineto{\pgfqpoint{1.182767in}{1.885216in}}%
\pgfpathlineto{\pgfqpoint{1.177582in}{1.971826in}}%
\pgfpathlineto{\pgfqpoint{1.181060in}{2.058552in}}%
\pgfpathlineto{\pgfqpoint{1.193209in}{2.144493in}}%
\pgfpathlineto{\pgfqpoint{1.213950in}{2.228763in}}%
\pgfpathlineto{\pgfqpoint{1.243106in}{2.310514in}}%
\pgfpathlineto{\pgfqpoint{1.280405in}{2.388938in}}%
\pgfpathlineto{\pgfqpoint{1.325480in}{2.463265in}}%
\pgfpathlineto{\pgfqpoint{1.377867in}{2.532762in}}%
\pgfpathlineto{\pgfqpoint{1.437006in}{2.596737in}}%
\pgfpathlineto{\pgfqpoint{1.502242in}{2.654534in}}%
\pgfpathlineto{\pgfqpoint{1.572863in}{2.705570in}}%
\pgfpathlineto{\pgfqpoint{1.648252in}{2.749406in}}%
\pgfpathlineto{\pgfqpoint{1.727652in}{2.785556in}}%
\pgfpathlineto{\pgfqpoint{1.810268in}{2.813615in}}%
\pgfpathlineto{\pgfqpoint{1.895287in}{2.833277in}}%
\pgfpathlineto{\pgfqpoint{1.981874in}{2.844338in}}%
\pgfpathlineto{\pgfqpoint{2.069177in}{2.846694in}}%
\pgfpathlineto{\pgfqpoint{2.156322in}{2.840339in}}%
\pgfpathlineto{\pgfqpoint{2.242415in}{2.825369in}}%
\pgfpathlineto{\pgfqpoint{2.326543in}{2.801980in}}%
\pgfpathlineto{\pgfqpoint{2.407892in}{2.770428in}}%
\pgfpathlineto{\pgfqpoint{2.485702in}{2.730993in}}%
\pgfpathlineto{\pgfqpoint{2.559143in}{2.684048in}}%
\pgfpathlineto{\pgfqpoint{2.627453in}{2.630038in}}%
\pgfpathlineto{\pgfqpoint{2.689944in}{2.569482in}}%
\pgfpathlineto{\pgfqpoint{2.745997in}{2.502972in}}%
\pgfpathlineto{\pgfqpoint{2.795067in}{2.431172in}}%
\pgfpathlineto{\pgfqpoint{2.836677in}{2.354820in}}%
\pgfpathlineto{\pgfqpoint{2.870424in}{2.274728in}}%
\pgfpathlineto{\pgfqpoint{2.895982in}{2.191761in}}%
\pgfpathlineto{\pgfqpoint{2.913146in}{2.106665in}}%
\pgfpathlineto{\pgfqpoint{2.921707in}{2.020297in}}%
\pgfpathlineto{\pgfqpoint{2.921529in}{1.933541in}}%
\pgfpathlineto{\pgfqpoint{2.912581in}{1.847267in}}%
\pgfpathlineto{\pgfqpoint{2.894934in}{1.762323in}}%
\pgfpathlineto{\pgfqpoint{2.868760in}{1.679547in}}%
\pgfpathlineto{\pgfqpoint{2.834333in}{1.599754in}}%
\pgfpathlineto{\pgfqpoint{2.792029in}{1.523747in}}%
\pgfpathlineto{\pgfqpoint{2.742326in}{1.452311in}}%
\pgfpathlineto{\pgfqpoint{2.685784in}{1.386189in}}%
\pgfpathlineto{\pgfqpoint{2.622905in}{1.325956in}}%
\pgfpathlineto{\pgfqpoint{2.554305in}{1.272252in}}%
\pgfpathlineto{\pgfqpoint{2.480663in}{1.225655in}}%
\pgfpathlineto{\pgfqpoint{2.402698in}{1.186652in}}%
\pgfpathlineto{\pgfqpoint{2.321174in}{1.155636in}}%
\pgfpathlineto{\pgfqpoint{2.236897in}{1.132910in}}%
\pgfpathlineto{\pgfqpoint{2.150713in}{1.118683in}}%
\pgfpathlineto{\pgfqpoint{2.063513in}{1.113071in}}%
\pgfpathlineto{\pgfqpoint{1.976228in}{1.116100in}}%
\pgfpathlineto{\pgfqpoint{1.889727in}{1.127750in}}%
\pgfpathlineto{\pgfqpoint{1.804830in}{1.147932in}}%
\pgfpathlineto{\pgfqpoint{1.804830in}{1.147932in}}%
\pgfusepath{stroke}%
\end{pgfscope}%
\begin{pgfscope}%
\pgfsetrectcap%
\pgfsetmiterjoin%
\pgfsetlinewidth{0.803000pt}%
\definecolor{currentstroke}{rgb}{0.000000,0.000000,0.000000}%
\pgfsetstrokecolor{currentstroke}%
\pgfsetdash{}{0pt}%
\pgfpathmoveto{\pgfqpoint{0.500000in}{0.440000in}}%
\pgfpathlineto{\pgfqpoint{0.500000in}{3.520000in}}%
\pgfusepath{stroke}%
\end{pgfscope}%
\begin{pgfscope}%
\pgfsetrectcap%
\pgfsetmiterjoin%
\pgfsetlinewidth{0.803000pt}%
\definecolor{currentstroke}{rgb}{0.000000,0.000000,0.000000}%
\pgfsetstrokecolor{currentstroke}%
\pgfsetdash{}{0pt}%
\pgfpathmoveto{\pgfqpoint{0.500000in}{0.440000in}}%
\pgfpathlineto{\pgfqpoint{3.600000in}{0.440000in}}%
\pgfusepath{stroke}%
\end{pgfscope}%
\begin{pgfscope}%
\pgfsetroundcap%
\pgfsetroundjoin%
\pgfsetlinewidth{0.501875pt}%
\definecolor{currentstroke}{rgb}{0.000000,0.000000,0.000000}%
\pgfsetstrokecolor{currentstroke}%
\pgfsetdash{}{0pt}%
\pgfpathmoveto{\pgfqpoint{3.654236in}{0.440000in}}%
\pgfpathquadraticcurveto{\pgfqpoint{3.631000in}{0.440000in}}{\pgfqpoint{3.600000in}{0.440000in}}%
\pgfusepath{stroke}%
\end{pgfscope}%
\begin{pgfscope}%
\pgfsetroundcap%
\pgfsetroundjoin%
\pgfsetlinewidth{0.501875pt}%
\definecolor{currentstroke}{rgb}{0.000000,0.000000,0.000000}%
\pgfsetstrokecolor{currentstroke}%
\pgfsetdash{}{0pt}%
\pgfpathmoveto{\pgfqpoint{3.598680in}{0.467778in}}%
\pgfpathlineto{\pgfqpoint{3.654236in}{0.440000in}}%
\pgfpathlineto{\pgfqpoint{3.598680in}{0.412222in}}%
\pgfusepath{stroke}%
\end{pgfscope}%
\begin{pgfscope}%
\pgfsetroundcap%
\pgfsetroundjoin%
\pgfsetlinewidth{0.501875pt}%
\definecolor{currentstroke}{rgb}{0.000000,0.000000,0.000000}%
\pgfsetstrokecolor{currentstroke}%
\pgfsetdash{}{0pt}%
\pgfpathmoveto{\pgfqpoint{0.500000in}{3.573836in}}%
\pgfpathquadraticcurveto{\pgfqpoint{0.500000in}{3.550800in}}{\pgfqpoint{0.500000in}{3.520000in}}%
\pgfusepath{stroke}%
\end{pgfscope}%
\begin{pgfscope}%
\pgfsetroundcap%
\pgfsetroundjoin%
\pgfsetlinewidth{0.501875pt}%
\definecolor{currentstroke}{rgb}{0.000000,0.000000,0.000000}%
\pgfsetstrokecolor{currentstroke}%
\pgfsetdash{}{0pt}%
\pgfpathmoveto{\pgfqpoint{0.472222in}{3.518280in}}%
\pgfpathlineto{\pgfqpoint{0.500000in}{3.573836in}}%
\pgfpathlineto{\pgfqpoint{0.527778in}{3.518280in}}%
\pgfusepath{stroke}%
\end{pgfscope}%
\end{pgfpicture}%
\makeatother%
\endgroup%

    \caption{Lösungen des Differentialgleichungssystems %\ref{TODO}
    Diverse Anfangspunkte haben den selben periodischer Orbit als Omega-Limesmenge.}
\label{poinbendix:fig:fall_2}
\end{figure}

\subsection{Fall 3: $\omega(p)$ ist ein Geschlossener Orbit welcher Singularitäten verbindet} \label{poinbendix:subsection:fall3}
%TODO Beschreibung und Beispiel mit Formeln und Plot
