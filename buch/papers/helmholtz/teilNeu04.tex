%
% teil3.tex -- Beispiel-File für Teil 3
%
% (c) 2020 Prof Dr Andreas Müller, Hochschule Rapperswil
%
% !TEX root = ../../buch.tex
% !TEX encoding = UTF-8
%
\section{Akustische Grundlagen und Feldtheorie
\label{helmholtz:section:akustische_Grundlagen}}
\kopfrechts{Akustische Grundlagen und Feldtheorie}

Um die Anwendung der Helmholtz-Zerlegung in der Akustik zu verstehen, bedarf es grundlegender Kenntnisse der physikalischen Akustik.
Hier werden die zentralen physikalischen Grössen Schalldruck, Schallschnelle und deren Zusammenhänge erläutert.
In vielen akustischen Aufgaben wird mit der komplexen Schreibweise gearbeitet, weshalb die Formeln vorzugsweise im Frequenzbereich dargestellt werden.

\subsection{Grundbegriffe der physikalischen Akustik
\label{helmholtz:subsection:Grundbegriffe_Akustik}}

\subsubsection{Schalldruck $p$}
 
Der Schalldruck ist eine der fundamentalen Größen in der Akustik und beschreibt lokale Druckschwankungen im Medium:
 
\begin{itemize}
\item $P  (\boldsymbol{r})$ bezeichnet die Amplitude des Schalldrucks am Ort $\boldsymbol{r}$ und ist eine reelle Grösse.
\item $\omega$ ist die Kreisfrequenz.
\item $\phi  (\boldsymbol{r})$ beschreibt die Phase am Ort $\boldsymbol{r}$ und ist ebenfalls eine reelle Grösse.
\end{itemize}
 
Die zeitabhängige Darstellung des Schalldrucks lautet:
\begin{equation}
p(r,t) = P(\boldsymbol{r})  e^{j( \omega t - \phi(\boldsymbol{r}))} (\si{\pascal}).
\end{equation}
 
Im Frequenzbereich, auch Phasor-Darstellung genannt, bei einer festen Frequenz $\omega$ vereinfacht sich dies zu:
\begin{equation}
p(r) = P(r)  e^{j \phi (r)}.
\label{helmholtz:PhasorSchalldruck}
\end{equation}
 
\subsubsection{Schallschnelle $\boldsymbol{u}$}
 
Die Schallschnelle $\boldsymbol{u}$ auch Teilchengeschwindigkeit genannt, wird aus dem Druckgradienten $\nabla p$ abgeleitet.
Für die linearisierte Form im Frequenzbereich $\boldsymbol{u} = \frac{j}{\rho_0 \omega} \nabla p$ bei harmonischen Feldern $\boldsymbol{u}(r,t)$ ergibt sich:
\begin{equation}
\boldsymbol{u}(r,t)
=
\frac{1}{\omega \rho} \nabla p(r,t)
=
\frac{1}{\omega \rho} \bigl( P(r) \nabla \Phi(r) + j\nabla P(r) \bigr)
e^{j(\omega t -\Phi(r))}\; [\si{\metre / \second}].
\end{equation}
Dabei sind:
\begin{itemize}
\item $\nabla \phi (r)$ der Gradient der Phase zeigt in Richtung der steilsten Phasenänderung, senkrecht zur Wellenfront.
\item $\nabla P (r)$ der Gradient der Druckamplitude zeigt in Richtung der schnellsten Amplitudenänderung.
\item $\rho$ die Dichte des Mediums.
\end{itemize}
Als Phasor bei einer fixen Frequenz $\omega$ ergibt sich die Schallschnelle zu:
\begin{equation}
\boldsymbol{u}(r)
=
\frac{1}{\omega \rho}  \bigl( P(r)  \nabla \phi(r) + j - \nabla P(r) \bigr)
e^{j\phi (r)}.
\label{helmholtz:PhasorSchallschnelle}
\end{equation}
 
%\subsubsection{Wellengleichung und Helmholtz-Gleichung}
 
%In der linearen Akustik sind Schalldruck und Schallschnelle durch folgende Grundgleichungen verbunden:
 
%\begin{itemize}
%\item Die Euler-Gleichung %(Impulserhaltung):
%\begin{equation}
%\rho_0 \frac{\partial \boldsymbol{u}}{\partial t} = -\nabla p
%\end{equation}
 
%\item Die Kontinuitätsgleichung %(Massenerhaltung):
%\begin{equation}
%\frac{\partial \rho}{\partial t} + \rho_0 \nabla \cdot \boldsymbol{u} = 0
%\end{equation}
%\end{itemize}
 
%\noindent Durch Kombination dieser Gleichungen erhält man die Wellengleichung für den Schalldruck:
%\begin{equation}
%\nabla^2 p - \frac{1}{c^2}\frac{\partial^2 p}{\partial t^2} = 0
%\end{equation}
 
%\noindent Für zeitharmonische Anregung mit $e^{j\omega t}$ führt dies zur Helmholtz-Gleichung:
%\begin{equation}
%\nabla^2 p + k^2 p = 0
%\end{equation}
 
%\noindent wobei $k = \omega/c$ die Wellenzahl ist und $c$ die Schallgeschwindigkeit im Medium.

\subsection{Energiebetrachtungen in akustischen Feldern
\label{helmholtz:subsection:Energiebetrachtung}}

Die akustische Energie in einem Schallfeld setzt sich aus kinetischer
und potentieller Energie zusammen.
Die zeitlich gemittelte Energiedichte kann durch die Gleichung
\begin{equation}
\langle w \rangle
=
\frac{1}{4}\left(\frac{|p|^2}{\rho_0 c^2} + \rho_0 |\boldsymbol{u}|^2 \right)
\end{equation}
ausgedrückt werden,
wobei der erste Term die potentielle Energiedichte (durch die
Kompression des Mediums) und der zweite Term die kinetische
Energiedichte (durch die Bewegung der Teilchen) darstellt.

Der Transport dieser Energie wird durch die Schallintensität
beschrieben.
Für harmonische Schallfelder betrachten wir die zeitlich
gemittelte aktive Intensität:
\begin{equation}
\boldsymbol{I}
=
\frac{1}{T}\int_0^T \boldsymbol{I}_i(\boldsymbol{r},t)\,dt
=
\frac{1}{2}\Re (p(\boldsymbol{r}) \boldsymbol{u}^*(\boldsymbol{r})),
\end{equation}
wobei der Stern die komplexe Konjugation bezeichnet.
Diese aktive Intensität beschreibt den Netto-Energiefluss und ist für die Schallausbreitung von zentraler Bedeutung.

\subsubsection{Zeitabhängige und zeitgemittelte Größen}

In der Akustik unterscheiden wir zwischen momentanen, zeitabhängigen Größen und zeitgemittelten Größen:

\begin{itemize}
\item \textbf{Momentane Größen} wie die instantane Intensität $\boldsymbol{I}(r,t)$ beschreiben den augenblicklichen Energietransport an einem bestimmten Ort und zu einem bestimmten Zeitpunkt.

\item \textbf{Zeitgemittelte Größen} wie die aktive Intensität $\boldsymbol{I}(r)$ beschreiben den durchschnittlichen Energietransport über eine Periode der harmonischen Schwingung und sind besonders wichtig für die Analyse der effektiven Energieübertragung in akustischen Feldern.
\end{itemize}

Diese Unterscheidung ist von großer Bedeutung, da in vielen praktischen Anwendungen nicht die instantanen, sondern die zeitlich gemittelten Größen von Interesse sind.




