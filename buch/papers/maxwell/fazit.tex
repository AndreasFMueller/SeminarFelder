\section{Fazit}
Wir haben eine allgemein kovariante Darstellung der Maxwell-Gleichungen gefunden. Die Anzahl der Gleichungen hat sich von vier auf zwei reduziert
\[
\left.
\begin{aligned}
	\nabla \cdot \vec{E} &= \frac{\rho}{\varepsilon_0} \\
	\nabla \cdot \vec{B} &= 0 \\
	\nabla \times \vec{E} &= - \frac{\d \vec{B}}{\d t} \\
	\nabla \times \vec{B} &= \mu_0 \vec{J} + \mu_0 \varepsilon_0 \frac{\d \vec{E}}{\d t}
\end{aligned}
\right\}
	\qquad\Rightarrow\qquad
\left\{
	\begin{aligned}
		dF &= 0 \\
	{\ast}d{\ast}F &= \mu_0 J 
\end{aligned}
\right.
\]
und die neue Formulierung ist viel eleganter(!).

%us vier komplizierten Gleichungen werden 2 kleine
%• funktionieren in beliebigen Koordinatensystemen
%• Formulierung ist kovariant und für die spezielle und allgemeine Relativitätstheorie
%korrekt
%• E
% und B,
% A
% und ϕ sowie J ⃗ und ρ sind vereinigt