%
% 1-temperatur.tex
%
% (c) 2024 Prof Dr Andreas Müller
%

%
% Temperatur
%
\section{Temperatur
\label{buch:fallstudie:temperatur}}
\kopfrechts{Temperatur}
Die Temperatur ist eine Grösse, die ausdrückt, wie warm oder kalt ein
physikalisches ist.
Die wichtigste Eigenschaft der Temperatur ist zweifellos, dass sie
sich zwischen sich in Kontakt befindlichen Körpern ausgleicht.
Wird ein heisser Körper mit einem kalten Körper in Berührung gebracht,
beginnt sich die Temperatur auszugleichen.
Zwei Körper haben genau die gleiche Temperatur, wenn sie längere
Zeit in Kontakt waren und sich die Temperaturen angleichen konnten.

Diese Eigenschaft reicht bereits aus, verschiedene zunächst willkürliche
Temperaturskalen zu definieren.
Zum Beispiel beobachtet man, dass sich ein Gas bei gleichem Umgebungsdruck
ausdehnt, wenn die Temperatur erhöht wird.
Das Ausmass der Ausdehnung kann als Mass für die Temperatur verwendet
werden.
Ähnlich kann auch die Ausdehnung von Quecksilber in einer Glaskapillare
verwendet werden.
Teilt man die Ausdehnung zwischen der Temperatur gefrierenden Wassers und
siedenden Wassers in 100 gleiche Teile, erhält man die Celsius-Temperaturskala.
Allerdings zeigt sich bei genauer Messung, dass die Ausdehnung von
Quecksilber nicht genau linear ist, so dass sich eine kleine Abweichung
von einem auf der Ausdehnung eines Gases basierenden Thermometer
ergibt.

Da die Temperatur nach obiger Definition nur durch Temperaturausgleich
gemessen werden, wird die Messung durch die Wärmekapazität des Messgeräts
beeinflusst.
Hat das Messgerät stark abweichende Temepratur, ändert sich die Temperatur
des zu messenden Körpers durch Kontakt sehr stark.
Eine sinnvolle Temperaturangabe ist damit nicht mehr möglich.
Der technische Fortschritt hat in der Vergangenheit aber ermöglicht, sehr
kleine Temepratursensoren zu konstruieren.
Heutzutage enthalten sogar Mikrochips wie Computerprozessoren spezielle
Dioden, die dazu da sind, die Temperatur des Chips zu überwachen und damit
den Chip vor Überhitzung zu schützen.
Wir nehmen daher im folgenden idealisierend an, dass die Temperatur beliebig
kleiner Teilkörper genügend genau gemessen werden kann.

\subsection{Das Temperaturfeld}
Der Ausgleich der Temperatur braucht Zeit.
Sind verschiedene Teile eines Körpers, dann werden diese Unterschiede
mit der Zeit zwar kleiner werden, aber je kleiner die Unterschiede sind,
desto langsamer wird auch der Ausgleich sein.
Die Temperatur eines Körpers ist daher eine Idealisierung, die nur
nach sehr langer Wartezeit erreicht werden kann.
Genau genommen ist die Temperatur eines Körpers notwendigerweise eine
Funktion $T(t,x,y,z)$, die sowohl von der Zeit wie auch von den
Ortskoordinaten abhängt.
Wir nennen sie für die Zwecke dieses Kapitels das {\em Temperaturfeld}.
Das Temperaturfeld eines isolierten Körpers konvergiert für
$t\to\infty$ gegen die ideale Temperatur des Körpers.

\subsection{Lokalität und Ableitungen}
In der Thermodynamik wird ausserdem gelehrt, dass die Temperatur die
mittlere kinetische Energie der Atome oder Moleküle des Körpers wiedergibt.
Die statistische Mechanik untersucht, wie sich die kinetische Energie
zwischen verschiedenen Teilchen durch Stösse ausgleicht, bis sich eine
stationäre Verteilung einstellt.
Aus diesem Mechanismus kann man sofort schliessen, dass der
Temperaturausgleich ein {\em lokales} Phänomen ist.
Stellt man in einem Medium an zwei verschiedenen Punkten verschiedene
Temperaturen fest, kann sich die Temperatur nur ausgleichen, wenn
sich auch die Temperatur an allen Punkten zwischen diesen beiden
ausgleicht.
Die Gleichungen, die die Veränderung der Temperatur in einem Punkt
mit der Zeit beschreiben, dürfen also nur die Information verwenden,
die sich aus dem Temperaturfeld an diesem Punkt ableiten lässt.
Nehmen wir an, dass die Funktion sich in einem Punkt in eine konvergente
Taylor-Reihe entwickeln lässt, dann kann man
\[
T(t,x)
=
T(t,x_0) + DT(t,x_0)\, (x-x_0) + \frac{1}{2!} D^2T(t,x_0)(x-x_0,x-x_0) + \dots
\]
schreiben.
Die Ableitungen\footnote{Die Notation für die Ableitungen wird
in späteren Kapiteln definiert und erklärt, für die aktuelle Diskussion
kann sich der mit der Notation nicht vertraute Leser mit dem eindimensionalen
Fall $D^1f(x) = f'(x)$, $D^2f(x) = f''(x)$ oder $D^kf(x)=f^{(k)}(x)$ behelfen.}
\[
T(t,x_0), \quad
DT(t,x_0), \quad
D^2(t,x_0),\quad \dots\quad
D^k(t,x_0),\quad\dots
\]
sind Grössen, die nur von der Temperatur in einer beliebig kleinen
Umgebung des Punktes $x_0$ abhängen.
Es bleibt die offene Frage, ob sich auch noch weitere Grössen finden lassen.

\begin{aufgabe}
Es muss eine Theorie der lokalen Operatoren entwickelt werden, mit deren
Naturgesetze formuliert werden können.
Die Theorie muss den klassischen Begriff der Ableitung umfassen.
\end{aufgabe}

Die Geschwindigkeit, mit der sich die Temperatur in einem Punkt
an die Temperatur der näheren Umgebung anpasst, hängt von den
Ableitungen ab in diesem Punkt ab.
Die Zeitableitung von $T$ muss daher eine Funktion der Ableitungen
sein.
Die Zeitentwicklungsgleichung es muss daher eine Differentialgleichung
der Form
\begin{equation}
\frac{\partial T}{\partial t}
=
F\biggl(x,
T,
\frac{\partial T}{\partial x},
\frac{\partial^2 T}{\partial x^2},
\dots
\biggr)
\label{buch:fallstudie:eqn:allgwaermeleitung}
\end{equation}
sein.

\subsection{Symmetrien}
Es gibt keine experimentellen Anzeichen, dass die Wärmeausbreitung
eine Richtung bevorzugt.
Die Differentialgleichung
\eqref{buch:fallstudie:eqn:allgwaermeleitung}
muss daher bei einer Spiegelung $x\mapsto -x$ unverändert bleiben.
Bei einer solchen Spiegelung ändern die ungeraden Ableitungen
das Vorzeichen, die Funktion $F$ muss daher gerade in den Argumenten
für die ungeraden Ableitungen sein.
In der einfachsten Form einer Wärmeleitungsgleichung ist die rechte
Seite eine Funktion nur von $x$ und der zweiten Ableitung von $T$.

\subsection{Die Wärmeleitungsgleichung}
Aus Experimenten weiss man aber auch, dass die Geschwindigkeit der
Temperaturänderung proportional ist zur Temperaturdifferenz.
Die Temperaturänderungsrate im Punkt $x$, die von der Temperaturdifferenz
zum benachbarten Punkt $x+h$ verursacht wird, ist daher
\begin{align*}
\biggl(\frac{\partial T}{\partial t}\biggr)_{\text{rechts}}
&=
a(h) (T(x+h)-T(x))
\intertext{für eine geeignete, von $h$ abhängige Proportionalitätskonstante
$a(h)$.
Die vom linken Nachbarpunkt beigesteuerte Änderungsrate ist}
\biggl(\frac{\partial T}{\partial t}\biggr)_{\text{links}}
&=
a(h) (T(x-h)-T(x))
\intertext{Die gesamte Änderungsrate ist daher}
\frac{\partial T}{\partial t}
&=
a(h) \bigl(T(x+h) - 2T(x) + T(x-h)\bigr).
\end{align*}
Wenn die Temperaturdifferenzen zu den Nachbarpunkten gleich gross 
sind, verschwindet die Änderungsrate.
Sie kann daher nur von der zweiten Ableitung, $a(h)$ muss so
sein, dass
\[
\lim_{h\to 0}
a(h) \bigl( T(x+h)-2T(x)+T(x-h)\bigr)
=
a_0
\frac{\partial T^2}{\partial x^2}
\]
ist.
Damit ist die eindimensionale Wärmeleitungsgleichung
\begin{equation}
\frac{\partial T}{\partial t}
=
\kappa \frac{\partial^2 T}{\partial x^2}
\label{buch:fallstudie:waermeleitungsgleichung}
\end{equation}
gefunden.
Für beliebige Dimension ist die zweite Ableitung auf der rechten
Seite durch den Laplace-Operator
\[
\Delta T
=
\frac{\partial^2 T}{\partial x_1^2}
+\dots+
\frac{\partial^2 T}{\partial x_n^2}
\]
zu ersetzen.

\subsection{Wohldefinierte Lösungen}
Die Verwendung der Ableitungen führt jedoch auf ein anderes Problem.
Die statistische Theorie der Wärmelehre erklärt auch, dass die Temperatur
ein einem kleinen Teil eines Körpers zufälligen Fluktuationen unterworfen
ist, die umso grösser werden können, je kleiner der Teil ist.
Die Ableitung ist der Grenzwert der Differenzenquotienten
\[
\lim_{s\to 0}
\frac{T(x_0+s\Delta x) - T(x_0)}{s}.
\]
Die Differenz im Zähler wird aber für für kleiner werdendes $s$ immer
weniger sinnvoll.
Die Verwendung der Ableitung ist daher ebenfalls eine Idealisierung.
Die Bewegungsgleichungen für das Temperaturfeld müssen daher von einer
Art sein, dass ein Störung des Feldes um kleine Fluktuationen nicht
zu einer gänzlich anderen Lösung führen.

Ein chaotisches dynamisches System wie das Lorenz-System zeichnet sich
dadurch aus, dass unmessbar kleine Unterschiede in den Anfangsbedingungen,
die zum Beispiel durch statistische Fluktuationen verursacht sein können,
zu verschiedenen Lösungen führen können.
Das Verhalten des Feldes wird damit für alle praktischen Zwecke nicht
vorhersagbar.
Die Wärmeleitungsgleichung hat jedoch die interessante mathematische
Eigenschaft, dass Sie einem Maximumprinzip gehorcht.
Es besagt, dass die extreme Werte in einem Teil des Randes des Körpers
sich sofort ausgleichen und die Lösung der Wärmeleitungsgleichung
mindestens für kurze Zeiten nur Werte hat, die kleiner sind als die 
Extremwerte zu Beginn.

Das Problem, dass das Feld nicht exakt bestimmt werden kann, gefährdet
nicht nur beim Temperaturfeld die Vorhersagbarkeit.
Auch bei Strömungsfeldern, die von den nichtlinearen
Navier-Stokes-Gleichungen beschrieben werden, muss eine solche
Stabilitätseigenschaft mindestens für kurze Zeit weiterhin gelten,
wenn die Theorie überhaupt testbare Vorhersagen ermöglich soll.
Die Bewegungsgleichung einer erfolgreichen Feldtheorie müssen daher
von einer Art sein, die die Existenz und Eindeutigkeit von Lösungen
mindestens für kurze Zeit und kleine Raumgebiet garantieren.

\subsection{Wärmeleitfähigkeit}
Die Masseinheiten der beiden Seiten der Wärmeleitungsgleichung 
\eqref{buch:fallstudie:waermeleitungsgleichung}
sind
\[
\biggl[
\frac{\partial T}{\partial t}
\biggr]
=
\biggl[
\frac{\text{K}}{\text{s}}
\biggr]
=
[\kappa]
\biggl[
\frac{\partial^2 T}{\partial x^2}
\biggr]
=
[\kappa]
\biggl[
\frac{\text{K}}{\text{m}^2}
\biggr].
\]
Es folgt, dass die Masseinheit von $\kappa$
\[
[\kappa]
=
\biggl[
\frac{\text{s}}{\text{m}^2}
\biggr]
\]
keine absolute Grösse ist.
Sie hängt vom für die Zeit- und die Raumdimensionen gewählten
Koordinatensystem ab.
Es ist daher zu ermitteln, wie sich $\kappa$ zusammen mit den
Differentialoperatoren verändert, wenn das Koordinatensystem
geändert wird.

\begin{aufgabe}
Es ist eine Theorie der Differentialoperatoren zu entwickeln, die
unabhängig vom Koordinatensystem formuliert werden kann.
\end{aufgabe}

