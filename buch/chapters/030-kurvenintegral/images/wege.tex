%
% wege.tex -- template for standalon tikz images
%
% (c) 2021 Prof Dr Andreas Müller, OST Ostschweizer Fachhochschule
%
\documentclass[tikz]{standalone}
\usepackage{amsmath}
\usepackage{times}
\usepackage{txfonts}
\usepackage{pgfplots}
\usepackage{csvsimple}
\definecolor{darkred}{rgb}{0.8,0,0}
\usetikzlibrary{arrows,intersections,math}
\begin{document}
\def\skala{1}
\def\form{
\begin{scope}
	\clip (-2,-2) rectangle (2,2);
	\foreach \t in {0,30,...,330}{
		\draw[color=gray!50] (0,0) -- ({\t}:2.5);
	}
	\fill[color=white] (0,0) circle[radius=0.08];
\end{scope}
}
\begin{tikzpicture}[>=latex,thick,scale=\skala]

\begin{scope}[xshift=-4.3cm]
\form
\draw[line width=1.4pt,color=darkred] (0,-1.5) arc(-90:90:1.5);
\fill[color=darkred] (0,-1.5) circle[radius=0.08];
\fill[color=darkred] (0,1.5) circle[radius=0.08];
\node[color=darkred] at (0,-1.5) [left] {$A$};
\node[color=darkred] at (0,1.5) [left] {$B$};
\node[color=darkred] at (1.45,0.2) [above right] {$\gamma$};
\end{scope}

\begin{scope}
\form
\draw[color=darkred,line width=1.4pt] (0,-1.5) -- (0,-0.3) arc(-90:90:0.3) -- (0,1.5);
\fill[color=darkred] (0,-1.5) circle[radius=0.08];
\fill[color=darkred] (0,1.5) circle[radius=0.08];
\fill[color=darkred] (0,0.3) circle[radius=0.05];
\fill[color=darkred] (0,-0.3) circle[radius=0.05];
\node[color=darkred] at (0,-1.5) [left] {$A$};
\node[color=darkred] at (0,1.5) [left] {$B$};
\node[color=darkred] at (0,-0.3) [left] {$C$};
\node[color=darkred] at (0,0.3) [left] {$D$};
\node[color=darkred] at (0.3,0) [above right] {$\gamma_r$};
\end{scope}

\begin{scope}[xshift=4.3cm]
\form
\draw[line width=1.4pt,color=darkred!20]
	plot[domain=-90:90,samples=100] ({1.5*((90-\x)/90)*((90+\x)/90)},{1.5*\x/90});
\draw[line width=1.4pt,color=darkred]
	plot[domain=-90:90,samples=100] ({0.2*((90-\x)/90)*((90+\x)/90)},{1.5*\x/90});
\fill[color=darkred] (0,-1.5) circle[radius=0.08];
\fill[color=darkred] (0,1.5) circle[radius=0.08];
\fill[color=darkred] (1.5,0) circle[radius=0.08];
\node[color=darkred] at (0,-1.5) [left] {$A$};
\node[color=darkred] at (1.5,0) [above right] {$E$};
\node[color=darkred] at (0,1.5) [left] {$B$};
\node[color=darkred] at (0.2,0) [above right] {$\gamma_s$};
\end{scope}

\end{tikzpicture}
\end{document}

