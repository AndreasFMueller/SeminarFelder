%
% bipolar.tex -- Bipolar
%
% (c) 2021 Prof Dr Andreas Müller, OST Ostschweizer Fachhochschule
%
\documentclass[tikz]{standalone}
\usepackage{amsmath}
\usepackage{times}
\usepackage{txfonts}
\usepackage{pgfplots}
\usepackage{csvsimple}
\usetikzlibrary{arrows,intersections,math,calc}
\begin{document}
\def\skala{1}
\definecolor{darkred}{rgb}{0.8,0,0}
\definecolor{darkgreen}{rgb}{0,0.6,0}
\def\rechterwinkel#1#2#3{
	\fill[color=darkgreen!40] ({#1*\dx},{#2*\dy})
		-- ++(#3:0.12) arc(#3:{#3+90}:0.12) -- cycle;
}
\begin{tikzpicture}[>=latex,thick,scale=\skala]

\draw[->] (-4.2,0) -- (4.5,0) coordinate[label={$y^1$}];
\draw[->] (0,-4.1) -- (0,4.3) coordinate[label={right:$y^2$}];

\begin{scope}
	\clip (-4.1,-4.1) rectangle (4.1,4.1);
	\def\dx{1}
	\def\dy{1}

	%
% bipolarpaths.tex -- created by bipolar.m, do not modify
%
% (c) 2025 Prof Dr Andreas Müller 
%
% u = -1.413717
\def\upathA{
	({-0.8883*\dx},{0.0000*\dy})
	-- ({-0.8883*\dx},{0.0022*\dy})
	-- ({-0.8884*\dx},{0.0044*\dy})
	-- ({-0.8885*\dx},{0.0066*\dy})
	-- ({-0.8886*\dx},{0.0089*\dy})
	-- ({-0.8888*\dx},{0.0111*\dy})
	-- ({-0.8890*\dx},{0.0133*\dy})
	-- ({-0.8893*\dx},{0.0155*\dy})
	-- ({-0.8896*\dx},{0.0177*\dy})
	-- ({-0.8900*\dx},{0.0199*\dy})
	-- ({-0.8903*\dx},{0.0221*\dy})
	-- ({-0.8908*\dx},{0.0242*\dy})
	-- ({-0.8913*\dx},{0.0264*\dy})
	-- ({-0.8918*\dx},{0.0286*\dy})
	-- ({-0.8923*\dx},{0.0307*\dy})
	-- ({-0.8929*\dx},{0.0329*\dy})
	-- ({-0.8936*\dx},{0.0350*\dy})
	-- ({-0.8942*\dx},{0.0371*\dy})
	-- ({-0.8950*\dx},{0.0392*\dy})
	-- ({-0.8957*\dx},{0.0413*\dy})
	-- ({-0.8965*\dx},{0.0434*\dy})
	-- ({-0.8973*\dx},{0.0455*\dy})
	-- ({-0.8982*\dx},{0.0476*\dy})
	-- ({-0.8991*\dx},{0.0496*\dy})
	-- ({-0.9001*\dx},{0.0516*\dy})
	-- ({-0.9011*\dx},{0.0537*\dy})
	-- ({-0.9021*\dx},{0.0557*\dy})
	-- ({-0.9032*\dx},{0.0576*\dy})
	-- ({-0.9043*\dx},{0.0596*\dy})
	-- ({-0.9055*\dx},{0.0615*\dy})
	-- ({-0.9067*\dx},{0.0635*\dy})
	-- ({-0.9079*\dx},{0.0654*\dy})
	-- ({-0.9092*\dx},{0.0672*\dy})
	-- ({-0.9105*\dx},{0.0691*\dy})
	-- ({-0.9118*\dx},{0.0709*\dy})
	-- ({-0.9132*\dx},{0.0728*\dy})
	-- ({-0.9146*\dx},{0.0745*\dy})
	-- ({-0.9160*\dx},{0.0763*\dy})
	-- ({-0.9175*\dx},{0.0780*\dy})
	-- ({-0.9191*\dx},{0.0798*\dy})
	-- ({-0.9206*\dx},{0.0814*\dy})
	-- ({-0.9222*\dx},{0.0831*\dy})
	-- ({-0.9238*\dx},{0.0847*\dy})
	-- ({-0.9255*\dx},{0.0863*\dy})
	-- ({-0.9272*\dx},{0.0879*\dy})
	-- ({-0.9289*\dx},{0.0894*\dy})
	-- ({-0.9307*\dx},{0.0910*\dy})
	-- ({-0.9325*\dx},{0.0924*\dy})
	-- ({-0.9343*\dx},{0.0939*\dy})
	-- ({-0.9362*\dx},{0.0953*\dy})
	-- ({-0.9381*\dx},{0.0967*\dy})
	-- ({-0.9400*\dx},{0.0980*\dy})
	-- ({-0.9419*\dx},{0.0993*\dy})
	-- ({-0.9439*\dx},{0.1006*\dy})
	-- ({-0.9459*\dx},{0.1018*\dy})
	-- ({-0.9480*\dx},{0.1030*\dy})
	-- ({-0.9500*\dx},{0.1042*\dy})
	-- ({-0.9521*\dx},{0.1053*\dy})
	-- ({-0.9542*\dx},{0.1064*\dy})
	-- ({-0.9564*\dx},{0.1074*\dy})
	-- ({-0.9585*\dx},{0.1084*\dy})
	-- ({-0.9607*\dx},{0.1094*\dy})
	-- ({-0.9630*\dx},{0.1103*\dy})
	-- ({-0.9652*\dx},{0.1111*\dy})
	-- ({-0.9675*\dx},{0.1120*\dy})
	-- ({-0.9698*\dx},{0.1127*\dy})
	-- ({-0.9721*\dx},{0.1135*\dy})
	-- ({-0.9744*\dx},{0.1142*\dy})
	-- ({-0.9767*\dx},{0.1148*\dy})
	-- ({-0.9791*\dx},{0.1154*\dy})
	-- ({-0.9815*\dx},{0.1160*\dy})
	-- ({-0.9839*\dx},{0.1165*\dy})
	-- ({-0.9863*\dx},{0.1169*\dy})
	-- ({-0.9887*\dx},{0.1173*\dy})
	-- ({-0.9912*\dx},{0.1177*\dy})
	-- ({-0.9936*\dx},{0.1180*\dy})
	-- ({-0.9961*\dx},{0.1182*\dy})
	-- ({-0.9986*\dx},{0.1184*\dy})
	-- ({-1.0011*\dx},{0.1186*\dy})
	-- ({-1.0036*\dx},{0.1187*\dy})
	-- ({-1.0061*\dx},{0.1187*\dy})
	-- ({-1.0086*\dx},{0.1187*\dy})
	-- ({-1.0111*\dx},{0.1187*\dy})
	-- ({-1.0136*\dx},{0.1186*\dy})
	-- ({-1.0162*\dx},{0.1184*\dy})
	-- ({-1.0187*\dx},{0.1182*\dy})
	-- ({-1.0212*\dx},{0.1179*\dy})
	-- ({-1.0238*\dx},{0.1176*\dy})
	-- ({-1.0263*\dx},{0.1172*\dy})
	-- ({-1.0288*\dx},{0.1167*\dy})
	-- ({-1.0313*\dx},{0.1162*\dy})
	-- ({-1.0338*\dx},{0.1157*\dy})
	-- ({-1.0364*\dx},{0.1151*\dy})
	-- ({-1.0389*\dx},{0.1144*\dy})
	-- ({-1.0414*\dx},{0.1137*\dy})
	-- ({-1.0438*\dx},{0.1129*\dy})
	-- ({-1.0463*\dx},{0.1121*\dy})
	-- ({-1.0488*\dx},{0.1112*\dy})
	-- ({-1.0512*\dx},{0.1102*\dy})
	-- ({-1.0536*\dx},{0.1092*\dy})
	-- ({-1.0560*\dx},{0.1082*\dy})
	-- ({-1.0584*\dx},{0.1070*\dy})
	-- ({-1.0608*\dx},{0.1059*\dy})
	-- ({-1.0631*\dx},{0.1046*\dy})
	-- ({-1.0655*\dx},{0.1034*\dy})
	-- ({-1.0678*\dx},{0.1020*\dy})
	-- ({-1.0701*\dx},{0.1006*\dy})
	-- ({-1.0723*\dx},{0.0992*\dy})
	-- ({-1.0745*\dx},{0.0977*\dy})
	-- ({-1.0767*\dx},{0.0961*\dy})
	-- ({-1.0789*\dx},{0.0945*\dy})
	-- ({-1.0810*\dx},{0.0929*\dy})
	-- ({-1.0831*\dx},{0.0912*\dy})
	-- ({-1.0852*\dx},{0.0894*\dy})
	-- ({-1.0872*\dx},{0.0876*\dy})
	-- ({-1.0892*\dx},{0.0858*\dy})
	-- ({-1.0911*\dx},{0.0839*\dy})
	-- ({-1.0930*\dx},{0.0819*\dy})
	-- ({-1.0949*\dx},{0.0799*\dy})
	-- ({-1.0967*\dx},{0.0779*\dy})
	-- ({-1.0984*\dx},{0.0758*\dy})
	-- ({-1.1002*\dx},{0.0736*\dy})
	-- ({-1.1019*\dx},{0.0715*\dy})
	-- ({-1.1035*\dx},{0.0693*\dy})
	-- ({-1.1051*\dx},{0.0670*\dy})
	-- ({-1.1066*\dx},{0.0647*\dy})
	-- ({-1.1081*\dx},{0.0624*\dy})
	-- ({-1.1095*\dx},{0.0600*\dy})
	-- ({-1.1109*\dx},{0.0576*\dy})
	-- ({-1.1122*\dx},{0.0551*\dy})
	-- ({-1.1134*\dx},{0.0527*\dy})
	-- ({-1.1147*\dx},{0.0502*\dy})
	-- ({-1.1158*\dx},{0.0476*\dy})
	-- ({-1.1169*\dx},{0.0451*\dy})
	-- ({-1.1179*\dx},{0.0425*\dy})
	-- ({-1.1189*\dx},{0.0399*\dy})
	-- ({-1.1198*\dx},{0.0372*\dy})
	-- ({-1.1206*\dx},{0.0346*\dy})
	-- ({-1.1214*\dx},{0.0319*\dy})
	-- ({-1.1221*\dx},{0.0292*\dy})
	-- ({-1.1228*\dx},{0.0264*\dy})
	-- ({-1.1234*\dx},{0.0237*\dy})
	-- ({-1.1239*\dx},{0.0209*\dy})
	-- ({-1.1244*\dx},{0.0182*\dy})
	-- ({-1.1248*\dx},{0.0154*\dy})
	-- ({-1.1251*\dx},{0.0126*\dy})
	-- ({-1.1254*\dx},{0.0098*\dy})
	-- ({-1.1256*\dx},{0.0070*\dy})
	-- ({-1.1257*\dx},{0.0042*\dy})
	-- ({-1.1258*\dx},{0.0014*\dy})
	-- ({-1.1258*\dx},{-0.0014*\dy})
	-- ({-1.1257*\dx},{-0.0042*\dy})
	-- ({-1.1256*\dx},{-0.0070*\dy})
	-- ({-1.1254*\dx},{-0.0098*\dy})
	-- ({-1.1251*\dx},{-0.0126*\dy})
	-- ({-1.1248*\dx},{-0.0154*\dy})
	-- ({-1.1244*\dx},{-0.0182*\dy})
	-- ({-1.1239*\dx},{-0.0209*\dy})
	-- ({-1.1234*\dx},{-0.0237*\dy})
	-- ({-1.1228*\dx},{-0.0264*\dy})
	-- ({-1.1221*\dx},{-0.0292*\dy})
	-- ({-1.1214*\dx},{-0.0319*\dy})
	-- ({-1.1206*\dx},{-0.0346*\dy})
	-- ({-1.1198*\dx},{-0.0372*\dy})
	-- ({-1.1189*\dx},{-0.0399*\dy})
	-- ({-1.1179*\dx},{-0.0425*\dy})
	-- ({-1.1169*\dx},{-0.0451*\dy})
	-- ({-1.1158*\dx},{-0.0476*\dy})
	-- ({-1.1147*\dx},{-0.0502*\dy})
	-- ({-1.1134*\dx},{-0.0527*\dy})
	-- ({-1.1122*\dx},{-0.0551*\dy})
	-- ({-1.1109*\dx},{-0.0576*\dy})
	-- ({-1.1095*\dx},{-0.0600*\dy})
	-- ({-1.1081*\dx},{-0.0624*\dy})
	-- ({-1.1066*\dx},{-0.0647*\dy})
	-- ({-1.1051*\dx},{-0.0670*\dy})
	-- ({-1.1035*\dx},{-0.0693*\dy})
	-- ({-1.1019*\dx},{-0.0715*\dy})
	-- ({-1.1002*\dx},{-0.0736*\dy})
	-- ({-1.0984*\dx},{-0.0758*\dy})
	-- ({-1.0967*\dx},{-0.0779*\dy})
	-- ({-1.0949*\dx},{-0.0799*\dy})
	-- ({-1.0930*\dx},{-0.0819*\dy})
	-- ({-1.0911*\dx},{-0.0839*\dy})
	-- ({-1.0892*\dx},{-0.0858*\dy})
	-- ({-1.0872*\dx},{-0.0876*\dy})
	-- ({-1.0852*\dx},{-0.0894*\dy})
	-- ({-1.0831*\dx},{-0.0912*\dy})
	-- ({-1.0810*\dx},{-0.0929*\dy})
	-- ({-1.0789*\dx},{-0.0945*\dy})
	-- ({-1.0767*\dx},{-0.0961*\dy})
	-- ({-1.0745*\dx},{-0.0977*\dy})
	-- ({-1.0723*\dx},{-0.0992*\dy})
	-- ({-1.0701*\dx},{-0.1006*\dy})
	-- ({-1.0678*\dx},{-0.1020*\dy})
	-- ({-1.0655*\dx},{-0.1034*\dy})
	-- ({-1.0631*\dx},{-0.1046*\dy})
	-- ({-1.0608*\dx},{-0.1059*\dy})
	-- ({-1.0584*\dx},{-0.1070*\dy})
	-- ({-1.0560*\dx},{-0.1082*\dy})
	-- ({-1.0536*\dx},{-0.1092*\dy})
	-- ({-1.0512*\dx},{-0.1102*\dy})
	-- ({-1.0488*\dx},{-0.1112*\dy})
	-- ({-1.0463*\dx},{-0.1121*\dy})
	-- ({-1.0438*\dx},{-0.1129*\dy})
	-- ({-1.0414*\dx},{-0.1137*\dy})
	-- ({-1.0389*\dx},{-0.1144*\dy})
	-- ({-1.0364*\dx},{-0.1151*\dy})
	-- ({-1.0338*\dx},{-0.1157*\dy})
	-- ({-1.0313*\dx},{-0.1162*\dy})
	-- ({-1.0288*\dx},{-0.1167*\dy})
	-- ({-1.0263*\dx},{-0.1172*\dy})
	-- ({-1.0238*\dx},{-0.1176*\dy})
	-- ({-1.0212*\dx},{-0.1179*\dy})
	-- ({-1.0187*\dx},{-0.1182*\dy})
	-- ({-1.0162*\dx},{-0.1184*\dy})
	-- ({-1.0136*\dx},{-0.1186*\dy})
	-- ({-1.0111*\dx},{-0.1187*\dy})
	-- ({-1.0086*\dx},{-0.1187*\dy})
	-- ({-1.0061*\dx},{-0.1187*\dy})
	-- ({-1.0036*\dx},{-0.1187*\dy})
	-- ({-1.0011*\dx},{-0.1186*\dy})
	-- ({-0.9986*\dx},{-0.1184*\dy})
	-- ({-0.9961*\dx},{-0.1182*\dy})
	-- ({-0.9936*\dx},{-0.1180*\dy})
	-- ({-0.9912*\dx},{-0.1177*\dy})
	-- ({-0.9887*\dx},{-0.1173*\dy})
	-- ({-0.9863*\dx},{-0.1169*\dy})
	-- ({-0.9839*\dx},{-0.1165*\dy})
	-- ({-0.9815*\dx},{-0.1160*\dy})
	-- ({-0.9791*\dx},{-0.1154*\dy})
	-- ({-0.9767*\dx},{-0.1148*\dy})
	-- ({-0.9744*\dx},{-0.1142*\dy})
	-- ({-0.9721*\dx},{-0.1135*\dy})
	-- ({-0.9698*\dx},{-0.1127*\dy})
	-- ({-0.9675*\dx},{-0.1120*\dy})
	-- ({-0.9652*\dx},{-0.1111*\dy})
	-- ({-0.9630*\dx},{-0.1103*\dy})
	-- ({-0.9607*\dx},{-0.1094*\dy})
	-- ({-0.9585*\dx},{-0.1084*\dy})
	-- ({-0.9564*\dx},{-0.1074*\dy})
	-- ({-0.9542*\dx},{-0.1064*\dy})
	-- ({-0.9521*\dx},{-0.1053*\dy})
	-- ({-0.9500*\dx},{-0.1042*\dy})
	-- ({-0.9480*\dx},{-0.1030*\dy})
	-- ({-0.9459*\dx},{-0.1018*\dy})
	-- ({-0.9439*\dx},{-0.1006*\dy})
	-- ({-0.9419*\dx},{-0.0993*\dy})
	-- ({-0.9400*\dx},{-0.0980*\dy})
	-- ({-0.9381*\dx},{-0.0967*\dy})
	-- ({-0.9362*\dx},{-0.0953*\dy})
	-- ({-0.9343*\dx},{-0.0939*\dy})
	-- ({-0.9325*\dx},{-0.0924*\dy})
	-- ({-0.9307*\dx},{-0.0910*\dy})
	-- ({-0.9289*\dx},{-0.0894*\dy})
	-- ({-0.9272*\dx},{-0.0879*\dy})
	-- ({-0.9255*\dx},{-0.0863*\dy})
	-- ({-0.9238*\dx},{-0.0847*\dy})
	-- ({-0.9222*\dx},{-0.0831*\dy})
	-- ({-0.9206*\dx},{-0.0814*\dy})
	-- ({-0.9191*\dx},{-0.0798*\dy})
	-- ({-0.9175*\dx},{-0.0780*\dy})
	-- ({-0.9160*\dx},{-0.0763*\dy})
	-- ({-0.9146*\dx},{-0.0745*\dy})
	-- ({-0.9132*\dx},{-0.0728*\dy})
	-- ({-0.9118*\dx},{-0.0709*\dy})
	-- ({-0.9105*\dx},{-0.0691*\dy})
	-- ({-0.9092*\dx},{-0.0672*\dy})
	-- ({-0.9079*\dx},{-0.0654*\dy})
	-- ({-0.9067*\dx},{-0.0635*\dy})
	-- ({-0.9055*\dx},{-0.0615*\dy})
	-- ({-0.9043*\dx},{-0.0596*\dy})
	-- ({-0.9032*\dx},{-0.0576*\dy})
	-- ({-0.9021*\dx},{-0.0557*\dy})
	-- ({-0.9011*\dx},{-0.0537*\dy})
	-- ({-0.9001*\dx},{-0.0516*\dy})
	-- ({-0.8991*\dx},{-0.0496*\dy})
	-- ({-0.8982*\dx},{-0.0476*\dy})
	-- ({-0.8973*\dx},{-0.0455*\dy})
	-- ({-0.8965*\dx},{-0.0434*\dy})
	-- ({-0.8957*\dx},{-0.0413*\dy})
	-- ({-0.8950*\dx},{-0.0392*\dy})
	-- ({-0.8942*\dx},{-0.0371*\dy})
	-- ({-0.8936*\dx},{-0.0350*\dy})
	-- ({-0.8929*\dx},{-0.0329*\dy})
	-- ({-0.8923*\dx},{-0.0307*\dy})
	-- ({-0.8918*\dx},{-0.0286*\dy})
	-- ({-0.8913*\dx},{-0.0264*\dy})
	-- ({-0.8908*\dx},{-0.0242*\dy})
	-- ({-0.8903*\dx},{-0.0221*\dy})
	-- ({-0.8900*\dx},{-0.0199*\dy})
	-- ({-0.8896*\dx},{-0.0177*\dy})
	-- ({-0.8893*\dx},{-0.0155*\dy})
	-- ({-0.8890*\dx},{-0.0133*\dy})
	-- ({-0.8888*\dx},{-0.0111*\dy})
	-- ({-0.8886*\dx},{-0.0089*\dy})
	-- ({-0.8885*\dx},{-0.0066*\dy})
	-- ({-0.8884*\dx},{-0.0044*\dy})
	-- ({-0.8883*\dx},{-0.0022*\dy})
	-- ({-0.8883*\dx},{-0.0000*\dy})
}
% u = -1.247397
\def\upathB{
	({-0.8476*\dx},{0.0000*\dy})
	-- ({-0.8476*\dx},{0.0030*\dy})
	-- ({-0.8477*\dx},{0.0059*\dy})
	-- ({-0.8478*\dx},{0.0089*\dy})
	-- ({-0.8480*\dx},{0.0118*\dy})
	-- ({-0.8482*\dx},{0.0148*\dy})
	-- ({-0.8485*\dx},{0.0177*\dy})
	-- ({-0.8488*\dx},{0.0207*\dy})
	-- ({-0.8492*\dx},{0.0236*\dy})
	-- ({-0.8497*\dx},{0.0265*\dy})
	-- ({-0.8502*\dx},{0.0295*\dy})
	-- ({-0.8507*\dx},{0.0324*\dy})
	-- ({-0.8513*\dx},{0.0353*\dy})
	-- ({-0.8520*\dx},{0.0382*\dy})
	-- ({-0.8527*\dx},{0.0411*\dy})
	-- ({-0.8535*\dx},{0.0440*\dy})
	-- ({-0.8543*\dx},{0.0468*\dy})
	-- ({-0.8552*\dx},{0.0497*\dy})
	-- ({-0.8561*\dx},{0.0525*\dy})
	-- ({-0.8570*\dx},{0.0554*\dy})
	-- ({-0.8581*\dx},{0.0582*\dy})
	-- ({-0.8591*\dx},{0.0610*\dy})
	-- ({-0.8603*\dx},{0.0638*\dy})
	-- ({-0.8614*\dx},{0.0665*\dy})
	-- ({-0.8627*\dx},{0.0693*\dy})
	-- ({-0.8640*\dx},{0.0720*\dy})
	-- ({-0.8653*\dx},{0.0747*\dy})
	-- ({-0.8667*\dx},{0.0774*\dy})
	-- ({-0.8681*\dx},{0.0801*\dy})
	-- ({-0.8696*\dx},{0.0827*\dy})
	-- ({-0.8711*\dx},{0.0853*\dy})
	-- ({-0.8727*\dx},{0.0879*\dy})
	-- ({-0.8744*\dx},{0.0905*\dy})
	-- ({-0.8760*\dx},{0.0930*\dy})
	-- ({-0.8778*\dx},{0.0956*\dy})
	-- ({-0.8796*\dx},{0.0981*\dy})
	-- ({-0.8814*\dx},{0.1005*\dy})
	-- ({-0.8833*\dx},{0.1030*\dy})
	-- ({-0.8852*\dx},{0.1054*\dy})
	-- ({-0.8872*\dx},{0.1077*\dy})
	-- ({-0.8892*\dx},{0.1101*\dy})
	-- ({-0.8913*\dx},{0.1124*\dy})
	-- ({-0.8935*\dx},{0.1147*\dy})
	-- ({-0.8956*\dx},{0.1169*\dy})
	-- ({-0.8979*\dx},{0.1191*\dy})
	-- ({-0.9001*\dx},{0.1213*\dy})
	-- ({-0.9024*\dx},{0.1234*\dy})
	-- ({-0.9048*\dx},{0.1255*\dy})
	-- ({-0.9072*\dx},{0.1275*\dy})
	-- ({-0.9097*\dx},{0.1296*\dy})
	-- ({-0.9122*\dx},{0.1315*\dy})
	-- ({-0.9147*\dx},{0.1334*\dy})
	-- ({-0.9173*\dx},{0.1353*\dy})
	-- ({-0.9199*\dx},{0.1372*\dy})
	-- ({-0.9226*\dx},{0.1390*\dy})
	-- ({-0.9253*\dx},{0.1407*\dy})
	-- ({-0.9281*\dx},{0.1424*\dy})
	-- ({-0.9309*\dx},{0.1440*\dy})
	-- ({-0.9337*\dx},{0.1456*\dy})
	-- ({-0.9366*\dx},{0.1472*\dy})
	-- ({-0.9395*\dx},{0.1487*\dy})
	-- ({-0.9425*\dx},{0.1501*\dy})
	-- ({-0.9454*\dx},{0.1515*\dy})
	-- ({-0.9485*\dx},{0.1528*\dy})
	-- ({-0.9515*\dx},{0.1541*\dy})
	-- ({-0.9546*\dx},{0.1553*\dy})
	-- ({-0.9578*\dx},{0.1565*\dy})
	-- ({-0.9609*\dx},{0.1576*\dy})
	-- ({-0.9641*\dx},{0.1586*\dy})
	-- ({-0.9674*\dx},{0.1596*\dy})
	-- ({-0.9706*\dx},{0.1605*\dy})
	-- ({-0.9739*\dx},{0.1613*\dy})
	-- ({-0.9772*\dx},{0.1621*\dy})
	-- ({-0.9806*\dx},{0.1628*\dy})
	-- ({-0.9839*\dx},{0.1635*\dy})
	-- ({-0.9873*\dx},{0.1640*\dy})
	-- ({-0.9907*\dx},{0.1646*\dy})
	-- ({-0.9942*\dx},{0.1650*\dy})
	-- ({-0.9976*\dx},{0.1654*\dy})
	-- ({-1.0011*\dx},{0.1657*\dy})
	-- ({-1.0046*\dx},{0.1659*\dy})
	-- ({-1.0081*\dx},{0.1661*\dy})
	-- ({-1.0116*\dx},{0.1661*\dy})
	-- ({-1.0152*\dx},{0.1662*\dy})
	-- ({-1.0187*\dx},{0.1661*\dy})
	-- ({-1.0223*\dx},{0.1659*\dy})
	-- ({-1.0259*\dx},{0.1657*\dy})
	-- ({-1.0294*\dx},{0.1654*\dy})
	-- ({-1.0330*\dx},{0.1650*\dy})
	-- ({-1.0366*\dx},{0.1646*\dy})
	-- ({-1.0402*\dx},{0.1640*\dy})
	-- ({-1.0438*\dx},{0.1634*\dy})
	-- ({-1.0474*\dx},{0.1627*\dy})
	-- ({-1.0509*\dx},{0.1619*\dy})
	-- ({-1.0545*\dx},{0.1611*\dy})
	-- ({-1.0581*\dx},{0.1601*\dy})
	-- ({-1.0616*\dx},{0.1591*\dy})
	-- ({-1.0652*\dx},{0.1580*\dy})
	-- ({-1.0687*\dx},{0.1568*\dy})
	-- ({-1.0722*\dx},{0.1555*\dy})
	-- ({-1.0757*\dx},{0.1542*\dy})
	-- ({-1.0792*\dx},{0.1527*\dy})
	-- ({-1.0826*\dx},{0.1512*\dy})
	-- ({-1.0861*\dx},{0.1496*\dy})
	-- ({-1.0895*\dx},{0.1479*\dy})
	-- ({-1.0928*\dx},{0.1461*\dy})
	-- ({-1.0962*\dx},{0.1443*\dy})
	-- ({-1.0995*\dx},{0.1423*\dy})
	-- ({-1.1027*\dx},{0.1403*\dy})
	-- ({-1.1060*\dx},{0.1382*\dy})
	-- ({-1.1092*\dx},{0.1360*\dy})
	-- ({-1.1123*\dx},{0.1337*\dy})
	-- ({-1.1154*\dx},{0.1314*\dy})
	-- ({-1.1185*\dx},{0.1290*\dy})
	-- ({-1.1215*\dx},{0.1265*\dy})
	-- ({-1.1244*\dx},{0.1239*\dy})
	-- ({-1.1273*\dx},{0.1212*\dy})
	-- ({-1.1302*\dx},{0.1185*\dy})
	-- ({-1.1330*\dx},{0.1157*\dy})
	-- ({-1.1357*\dx},{0.1128*\dy})
	-- ({-1.1383*\dx},{0.1099*\dy})
	-- ({-1.1409*\dx},{0.1069*\dy})
	-- ({-1.1435*\dx},{0.1038*\dy})
	-- ({-1.1459*\dx},{0.1006*\dy})
	-- ({-1.1483*\dx},{0.0974*\dy})
	-- ({-1.1506*\dx},{0.0941*\dy})
	-- ({-1.1529*\dx},{0.0908*\dy})
	-- ({-1.1550*\dx},{0.0874*\dy})
	-- ({-1.1571*\dx},{0.0839*\dy})
	-- ({-1.1591*\dx},{0.0804*\dy})
	-- ({-1.1610*\dx},{0.0769*\dy})
	-- ({-1.1629*\dx},{0.0732*\dy})
	-- ({-1.1646*\dx},{0.0696*\dy})
	-- ({-1.1663*\dx},{0.0659*\dy})
	-- ({-1.1678*\dx},{0.0621*\dy})
	-- ({-1.1693*\dx},{0.0583*\dy})
	-- ({-1.1707*\dx},{0.0544*\dy})
	-- ({-1.1720*\dx},{0.0506*\dy})
	-- ({-1.1732*\dx},{0.0467*\dy})
	-- ({-1.1743*\dx},{0.0427*\dy})
	-- ({-1.1753*\dx},{0.0387*\dy})
	-- ({-1.1762*\dx},{0.0347*\dy})
	-- ({-1.1770*\dx},{0.0307*\dy})
	-- ({-1.1777*\dx},{0.0266*\dy})
	-- ({-1.1783*\dx},{0.0226*\dy})
	-- ({-1.1788*\dx},{0.0185*\dy})
	-- ({-1.1792*\dx},{0.0144*\dy})
	-- ({-1.1795*\dx},{0.0103*\dy})
	-- ({-1.1798*\dx},{0.0062*\dy})
	-- ({-1.1799*\dx},{0.0021*\dy})
	-- ({-1.1799*\dx},{-0.0021*\dy})
	-- ({-1.1798*\dx},{-0.0062*\dy})
	-- ({-1.1795*\dx},{-0.0103*\dy})
	-- ({-1.1792*\dx},{-0.0144*\dy})
	-- ({-1.1788*\dx},{-0.0185*\dy})
	-- ({-1.1783*\dx},{-0.0226*\dy})
	-- ({-1.1777*\dx},{-0.0266*\dy})
	-- ({-1.1770*\dx},{-0.0307*\dy})
	-- ({-1.1762*\dx},{-0.0347*\dy})
	-- ({-1.1753*\dx},{-0.0387*\dy})
	-- ({-1.1743*\dx},{-0.0427*\dy})
	-- ({-1.1732*\dx},{-0.0467*\dy})
	-- ({-1.1720*\dx},{-0.0506*\dy})
	-- ({-1.1707*\dx},{-0.0544*\dy})
	-- ({-1.1693*\dx},{-0.0583*\dy})
	-- ({-1.1678*\dx},{-0.0621*\dy})
	-- ({-1.1663*\dx},{-0.0659*\dy})
	-- ({-1.1646*\dx},{-0.0696*\dy})
	-- ({-1.1629*\dx},{-0.0732*\dy})
	-- ({-1.1610*\dx},{-0.0769*\dy})
	-- ({-1.1591*\dx},{-0.0804*\dy})
	-- ({-1.1571*\dx},{-0.0839*\dy})
	-- ({-1.1550*\dx},{-0.0874*\dy})
	-- ({-1.1529*\dx},{-0.0908*\dy})
	-- ({-1.1506*\dx},{-0.0941*\dy})
	-- ({-1.1483*\dx},{-0.0974*\dy})
	-- ({-1.1459*\dx},{-0.1006*\dy})
	-- ({-1.1435*\dx},{-0.1038*\dy})
	-- ({-1.1409*\dx},{-0.1069*\dy})
	-- ({-1.1383*\dx},{-0.1099*\dy})
	-- ({-1.1357*\dx},{-0.1128*\dy})
	-- ({-1.1330*\dx},{-0.1157*\dy})
	-- ({-1.1302*\dx},{-0.1185*\dy})
	-- ({-1.1273*\dx},{-0.1212*\dy})
	-- ({-1.1244*\dx},{-0.1239*\dy})
	-- ({-1.1215*\dx},{-0.1265*\dy})
	-- ({-1.1185*\dx},{-0.1290*\dy})
	-- ({-1.1154*\dx},{-0.1314*\dy})
	-- ({-1.1123*\dx},{-0.1337*\dy})
	-- ({-1.1092*\dx},{-0.1360*\dy})
	-- ({-1.1060*\dx},{-0.1382*\dy})
	-- ({-1.1027*\dx},{-0.1403*\dy})
	-- ({-1.0995*\dx},{-0.1423*\dy})
	-- ({-1.0962*\dx},{-0.1443*\dy})
	-- ({-1.0928*\dx},{-0.1461*\dy})
	-- ({-1.0895*\dx},{-0.1479*\dy})
	-- ({-1.0861*\dx},{-0.1496*\dy})
	-- ({-1.0826*\dx},{-0.1512*\dy})
	-- ({-1.0792*\dx},{-0.1527*\dy})
	-- ({-1.0757*\dx},{-0.1542*\dy})
	-- ({-1.0722*\dx},{-0.1555*\dy})
	-- ({-1.0687*\dx},{-0.1568*\dy})
	-- ({-1.0652*\dx},{-0.1580*\dy})
	-- ({-1.0616*\dx},{-0.1591*\dy})
	-- ({-1.0581*\dx},{-0.1601*\dy})
	-- ({-1.0545*\dx},{-0.1611*\dy})
	-- ({-1.0509*\dx},{-0.1619*\dy})
	-- ({-1.0474*\dx},{-0.1627*\dy})
	-- ({-1.0438*\dx},{-0.1634*\dy})
	-- ({-1.0402*\dx},{-0.1640*\dy})
	-- ({-1.0366*\dx},{-0.1646*\dy})
	-- ({-1.0330*\dx},{-0.1650*\dy})
	-- ({-1.0294*\dx},{-0.1654*\dy})
	-- ({-1.0259*\dx},{-0.1657*\dy})
	-- ({-1.0223*\dx},{-0.1659*\dy})
	-- ({-1.0187*\dx},{-0.1661*\dy})
	-- ({-1.0152*\dx},{-0.1662*\dy})
	-- ({-1.0116*\dx},{-0.1661*\dy})
	-- ({-1.0081*\dx},{-0.1661*\dy})
	-- ({-1.0046*\dx},{-0.1659*\dy})
	-- ({-1.0011*\dx},{-0.1657*\dy})
	-- ({-0.9976*\dx},{-0.1654*\dy})
	-- ({-0.9942*\dx},{-0.1650*\dy})
	-- ({-0.9907*\dx},{-0.1646*\dy})
	-- ({-0.9873*\dx},{-0.1640*\dy})
	-- ({-0.9839*\dx},{-0.1635*\dy})
	-- ({-0.9806*\dx},{-0.1628*\dy})
	-- ({-0.9772*\dx},{-0.1621*\dy})
	-- ({-0.9739*\dx},{-0.1613*\dy})
	-- ({-0.9706*\dx},{-0.1605*\dy})
	-- ({-0.9674*\dx},{-0.1596*\dy})
	-- ({-0.9641*\dx},{-0.1586*\dy})
	-- ({-0.9609*\dx},{-0.1576*\dy})
	-- ({-0.9578*\dx},{-0.1565*\dy})
	-- ({-0.9546*\dx},{-0.1553*\dy})
	-- ({-0.9515*\dx},{-0.1541*\dy})
	-- ({-0.9485*\dx},{-0.1528*\dy})
	-- ({-0.9454*\dx},{-0.1515*\dy})
	-- ({-0.9425*\dx},{-0.1501*\dy})
	-- ({-0.9395*\dx},{-0.1487*\dy})
	-- ({-0.9366*\dx},{-0.1472*\dy})
	-- ({-0.9337*\dx},{-0.1456*\dy})
	-- ({-0.9309*\dx},{-0.1440*\dy})
	-- ({-0.9281*\dx},{-0.1424*\dy})
	-- ({-0.9253*\dx},{-0.1407*\dy})
	-- ({-0.9226*\dx},{-0.1390*\dy})
	-- ({-0.9199*\dx},{-0.1372*\dy})
	-- ({-0.9173*\dx},{-0.1353*\dy})
	-- ({-0.9147*\dx},{-0.1334*\dy})
	-- ({-0.9122*\dx},{-0.1315*\dy})
	-- ({-0.9097*\dx},{-0.1296*\dy})
	-- ({-0.9072*\dx},{-0.1275*\dy})
	-- ({-0.9048*\dx},{-0.1255*\dy})
	-- ({-0.9024*\dx},{-0.1234*\dy})
	-- ({-0.9001*\dx},{-0.1213*\dy})
	-- ({-0.8979*\dx},{-0.1191*\dy})
	-- ({-0.8956*\dx},{-0.1169*\dy})
	-- ({-0.8935*\dx},{-0.1147*\dy})
	-- ({-0.8913*\dx},{-0.1124*\dy})
	-- ({-0.8892*\dx},{-0.1101*\dy})
	-- ({-0.8872*\dx},{-0.1077*\dy})
	-- ({-0.8852*\dx},{-0.1054*\dy})
	-- ({-0.8833*\dx},{-0.1030*\dy})
	-- ({-0.8814*\dx},{-0.1005*\dy})
	-- ({-0.8796*\dx},{-0.0981*\dy})
	-- ({-0.8778*\dx},{-0.0956*\dy})
	-- ({-0.8760*\dx},{-0.0930*\dy})
	-- ({-0.8744*\dx},{-0.0905*\dy})
	-- ({-0.8727*\dx},{-0.0879*\dy})
	-- ({-0.8711*\dx},{-0.0853*\dy})
	-- ({-0.8696*\dx},{-0.0827*\dy})
	-- ({-0.8681*\dx},{-0.0801*\dy})
	-- ({-0.8667*\dx},{-0.0774*\dy})
	-- ({-0.8653*\dx},{-0.0747*\dy})
	-- ({-0.8640*\dx},{-0.0720*\dy})
	-- ({-0.8627*\dx},{-0.0693*\dy})
	-- ({-0.8614*\dx},{-0.0665*\dy})
	-- ({-0.8603*\dx},{-0.0638*\dy})
	-- ({-0.8591*\dx},{-0.0610*\dy})
	-- ({-0.8581*\dx},{-0.0582*\dy})
	-- ({-0.8570*\dx},{-0.0554*\dy})
	-- ({-0.8561*\dx},{-0.0525*\dy})
	-- ({-0.8552*\dx},{-0.0497*\dy})
	-- ({-0.8543*\dx},{-0.0468*\dy})
	-- ({-0.8535*\dx},{-0.0440*\dy})
	-- ({-0.8527*\dx},{-0.0411*\dy})
	-- ({-0.8520*\dx},{-0.0382*\dy})
	-- ({-0.8513*\dx},{-0.0353*\dy})
	-- ({-0.8507*\dx},{-0.0324*\dy})
	-- ({-0.8502*\dx},{-0.0295*\dy})
	-- ({-0.8497*\dx},{-0.0265*\dy})
	-- ({-0.8492*\dx},{-0.0236*\dy})
	-- ({-0.8488*\dx},{-0.0207*\dy})
	-- ({-0.8485*\dx},{-0.0177*\dy})
	-- ({-0.8482*\dx},{-0.0148*\dy})
	-- ({-0.8480*\dx},{-0.0118*\dy})
	-- ({-0.8478*\dx},{-0.0089*\dy})
	-- ({-0.8477*\dx},{-0.0059*\dy})
	-- ({-0.8476*\dx},{-0.0030*\dy})
	-- ({-0.8476*\dx},{-0.0000*\dy})
}
% u = -1.081077
\def\upathC{
	({-0.7936*\dx},{0.0000*\dy})
	-- ({-0.7936*\dx},{0.0039*\dy})
	-- ({-0.7937*\dx},{0.0078*\dy})
	-- ({-0.7939*\dx},{0.0117*\dy})
	-- ({-0.7941*\dx},{0.0156*\dy})
	-- ({-0.7944*\dx},{0.0194*\dy})
	-- ({-0.7948*\dx},{0.0233*\dy})
	-- ({-0.7952*\dx},{0.0272*\dy})
	-- ({-0.7957*\dx},{0.0311*\dy})
	-- ({-0.7962*\dx},{0.0349*\dy})
	-- ({-0.7968*\dx},{0.0388*\dy})
	-- ({-0.7975*\dx},{0.0426*\dy})
	-- ({-0.7983*\dx},{0.0465*\dy})
	-- ({-0.7991*\dx},{0.0503*\dy})
	-- ({-0.8000*\dx},{0.0541*\dy})
	-- ({-0.8009*\dx},{0.0579*\dy})
	-- ({-0.8019*\dx},{0.0617*\dy})
	-- ({-0.8030*\dx},{0.0655*\dy})
	-- ({-0.8041*\dx},{0.0693*\dy})
	-- ({-0.8053*\dx},{0.0730*\dy})
	-- ({-0.8066*\dx},{0.0768*\dy})
	-- ({-0.8079*\dx},{0.0805*\dy})
	-- ({-0.8093*\dx},{0.0842*\dy})
	-- ({-0.8108*\dx},{0.0879*\dy})
	-- ({-0.8123*\dx},{0.0916*\dy})
	-- ({-0.8139*\dx},{0.0952*\dy})
	-- ({-0.8156*\dx},{0.0988*\dy})
	-- ({-0.8173*\dx},{0.1024*\dy})
	-- ({-0.8191*\dx},{0.1060*\dy})
	-- ({-0.8210*\dx},{0.1096*\dy})
	-- ({-0.8229*\dx},{0.1131*\dy})
	-- ({-0.8249*\dx},{0.1167*\dy})
	-- ({-0.8269*\dx},{0.1201*\dy})
	-- ({-0.8290*\dx},{0.1236*\dy})
	-- ({-0.8312*\dx},{0.1270*\dy})
	-- ({-0.8335*\dx},{0.1304*\dy})
	-- ({-0.8358*\dx},{0.1338*\dy})
	-- ({-0.8382*\dx},{0.1371*\dy})
	-- ({-0.8406*\dx},{0.1405*\dy})
	-- ({-0.8431*\dx},{0.1437*\dy})
	-- ({-0.8457*\dx},{0.1470*\dy})
	-- ({-0.8484*\dx},{0.1502*\dy})
	-- ({-0.8511*\dx},{0.1533*\dy})
	-- ({-0.8538*\dx},{0.1564*\dy})
	-- ({-0.8567*\dx},{0.1595*\dy})
	-- ({-0.8596*\dx},{0.1626*\dy})
	-- ({-0.8626*\dx},{0.1656*\dy})
	-- ({-0.8656*\dx},{0.1685*\dy})
	-- ({-0.8687*\dx},{0.1714*\dy})
	-- ({-0.8719*\dx},{0.1743*\dy})
	-- ({-0.8751*\dx},{0.1771*\dy})
	-- ({-0.8784*\dx},{0.1799*\dy})
	-- ({-0.8817*\dx},{0.1826*\dy})
	-- ({-0.8851*\dx},{0.1853*\dy})
	-- ({-0.8886*\dx},{0.1879*\dy})
	-- ({-0.8922*\dx},{0.1904*\dy})
	-- ({-0.8958*\dx},{0.1929*\dy})
	-- ({-0.8994*\dx},{0.1954*\dy})
	-- ({-0.9031*\dx},{0.1977*\dy})
	-- ({-0.9069*\dx},{0.2001*\dy})
	-- ({-0.9108*\dx},{0.2023*\dy})
	-- ({-0.9147*\dx},{0.2045*\dy})
	-- ({-0.9186*\dx},{0.2066*\dy})
	-- ({-0.9226*\dx},{0.2087*\dy})
	-- ({-0.9267*\dx},{0.2107*\dy})
	-- ({-0.9308*\dx},{0.2126*\dy})
	-- ({-0.9350*\dx},{0.2144*\dy})
	-- ({-0.9393*\dx},{0.2162*\dy})
	-- ({-0.9436*\dx},{0.2179*\dy})
	-- ({-0.9479*\dx},{0.2195*\dy})
	-- ({-0.9523*\dx},{0.2210*\dy})
	-- ({-0.9568*\dx},{0.2225*\dy})
	-- ({-0.9613*\dx},{0.2238*\dy})
	-- ({-0.9658*\dx},{0.2251*\dy})
	-- ({-0.9704*\dx},{0.2263*\dy})
	-- ({-0.9750*\dx},{0.2274*\dy})
	-- ({-0.9797*\dx},{0.2284*\dy})
	-- ({-0.9844*\dx},{0.2294*\dy})
	-- ({-0.9892*\dx},{0.2302*\dy})
	-- ({-0.9940*\dx},{0.2309*\dy})
	-- ({-0.9988*\dx},{0.2316*\dy})
	-- ({-1.0037*\dx},{0.2321*\dy})
	-- ({-1.0086*\dx},{0.2325*\dy})
	-- ({-1.0136*\dx},{0.2329*\dy})
	-- ({-1.0185*\dx},{0.2331*\dy})
	-- ({-1.0236*\dx},{0.2332*\dy})
	-- ({-1.0286*\dx},{0.2332*\dy})
	-- ({-1.0336*\dx},{0.2331*\dy})
	-- ({-1.0387*\dx},{0.2329*\dy})
	-- ({-1.0438*\dx},{0.2326*\dy})
	-- ({-1.0489*\dx},{0.2322*\dy})
	-- ({-1.0540*\dx},{0.2317*\dy})
	-- ({-1.0592*\dx},{0.2310*\dy})
	-- ({-1.0643*\dx},{0.2302*\dy})
	-- ({-1.0695*\dx},{0.2293*\dy})
	-- ({-1.0746*\dx},{0.2283*\dy})
	-- ({-1.0798*\dx},{0.2272*\dy})
	-- ({-1.0849*\dx},{0.2259*\dy})
	-- ({-1.0901*\dx},{0.2245*\dy})
	-- ({-1.0952*\dx},{0.2230*\dy})
	-- ({-1.1003*\dx},{0.2214*\dy})
	-- ({-1.1055*\dx},{0.2196*\dy})
	-- ({-1.1105*\dx},{0.2177*\dy})
	-- ({-1.1156*\dx},{0.2157*\dy})
	-- ({-1.1207*\dx},{0.2135*\dy})
	-- ({-1.1257*\dx},{0.2113*\dy})
	-- ({-1.1306*\dx},{0.2089*\dy})
	-- ({-1.1356*\dx},{0.2063*\dy})
	-- ({-1.1405*\dx},{0.2037*\dy})
	-- ({-1.1454*\dx},{0.2009*\dy})
	-- ({-1.1502*\dx},{0.1980*\dy})
	-- ({-1.1549*\dx},{0.1949*\dy})
	-- ({-1.1596*\dx},{0.1918*\dy})
	-- ({-1.1643*\dx},{0.1885*\dy})
	-- ({-1.1688*\dx},{0.1850*\dy})
	-- ({-1.1734*\dx},{0.1815*\dy})
	-- ({-1.1778*\dx},{0.1778*\dy})
	-- ({-1.1822*\dx},{0.1740*\dy})
	-- ({-1.1864*\dx},{0.1701*\dy})
	-- ({-1.1906*\dx},{0.1661*\dy})
	-- ({-1.1947*\dx},{0.1619*\dy})
	-- ({-1.1988*\dx},{0.1576*\dy})
	-- ({-1.2027*\dx},{0.1532*\dy})
	-- ({-1.2065*\dx},{0.1487*\dy})
	-- ({-1.2102*\dx},{0.1441*\dy})
	-- ({-1.2138*\dx},{0.1394*\dy})
	-- ({-1.2173*\dx},{0.1346*\dy})
	-- ({-1.2207*\dx},{0.1297*\dy})
	-- ({-1.2240*\dx},{0.1246*\dy})
	-- ({-1.2271*\dx},{0.1195*\dy})
	-- ({-1.2302*\dx},{0.1143*\dy})
	-- ({-1.2330*\dx},{0.1090*\dy})
	-- ({-1.2358*\dx},{0.1036*\dy})
	-- ({-1.2384*\dx},{0.0982*\dy})
	-- ({-1.2409*\dx},{0.0926*\dy})
	-- ({-1.2433*\dx},{0.0870*\dy})
	-- ({-1.2455*\dx},{0.0813*\dy})
	-- ({-1.2475*\dx},{0.0756*\dy})
	-- ({-1.2494*\dx},{0.0697*\dy})
	-- ({-1.2512*\dx},{0.0639*\dy})
	-- ({-1.2528*\dx},{0.0579*\dy})
	-- ({-1.2542*\dx},{0.0520*\dy})
	-- ({-1.2555*\dx},{0.0460*\dy})
	-- ({-1.2566*\dx},{0.0399*\dy})
	-- ({-1.2576*\dx},{0.0338*\dy})
	-- ({-1.2584*\dx},{0.0277*\dy})
	-- ({-1.2591*\dx},{0.0216*\dy})
	-- ({-1.2596*\dx},{0.0154*\dy})
	-- ({-1.2599*\dx},{0.0093*\dy})
	-- ({-1.2601*\dx},{0.0031*\dy})
	-- ({-1.2601*\dx},{-0.0031*\dy})
	-- ({-1.2599*\dx},{-0.0093*\dy})
	-- ({-1.2596*\dx},{-0.0154*\dy})
	-- ({-1.2591*\dx},{-0.0216*\dy})
	-- ({-1.2584*\dx},{-0.0277*\dy})
	-- ({-1.2576*\dx},{-0.0338*\dy})
	-- ({-1.2566*\dx},{-0.0399*\dy})
	-- ({-1.2555*\dx},{-0.0460*\dy})
	-- ({-1.2542*\dx},{-0.0520*\dy})
	-- ({-1.2528*\dx},{-0.0579*\dy})
	-- ({-1.2512*\dx},{-0.0639*\dy})
	-- ({-1.2494*\dx},{-0.0697*\dy})
	-- ({-1.2475*\dx},{-0.0756*\dy})
	-- ({-1.2455*\dx},{-0.0813*\dy})
	-- ({-1.2433*\dx},{-0.0870*\dy})
	-- ({-1.2409*\dx},{-0.0926*\dy})
	-- ({-1.2384*\dx},{-0.0982*\dy})
	-- ({-1.2358*\dx},{-0.1036*\dy})
	-- ({-1.2330*\dx},{-0.1090*\dy})
	-- ({-1.2302*\dx},{-0.1143*\dy})
	-- ({-1.2271*\dx},{-0.1195*\dy})
	-- ({-1.2240*\dx},{-0.1246*\dy})
	-- ({-1.2207*\dx},{-0.1297*\dy})
	-- ({-1.2173*\dx},{-0.1346*\dy})
	-- ({-1.2138*\dx},{-0.1394*\dy})
	-- ({-1.2102*\dx},{-0.1441*\dy})
	-- ({-1.2065*\dx},{-0.1487*\dy})
	-- ({-1.2027*\dx},{-0.1532*\dy})
	-- ({-1.1988*\dx},{-0.1576*\dy})
	-- ({-1.1947*\dx},{-0.1619*\dy})
	-- ({-1.1906*\dx},{-0.1661*\dy})
	-- ({-1.1864*\dx},{-0.1701*\dy})
	-- ({-1.1822*\dx},{-0.1740*\dy})
	-- ({-1.1778*\dx},{-0.1778*\dy})
	-- ({-1.1734*\dx},{-0.1815*\dy})
	-- ({-1.1688*\dx},{-0.1850*\dy})
	-- ({-1.1643*\dx},{-0.1885*\dy})
	-- ({-1.1596*\dx},{-0.1918*\dy})
	-- ({-1.1549*\dx},{-0.1949*\dy})
	-- ({-1.1502*\dx},{-0.1980*\dy})
	-- ({-1.1454*\dx},{-0.2009*\dy})
	-- ({-1.1405*\dx},{-0.2037*\dy})
	-- ({-1.1356*\dx},{-0.2063*\dy})
	-- ({-1.1306*\dx},{-0.2089*\dy})
	-- ({-1.1257*\dx},{-0.2113*\dy})
	-- ({-1.1207*\dx},{-0.2135*\dy})
	-- ({-1.1156*\dx},{-0.2157*\dy})
	-- ({-1.1105*\dx},{-0.2177*\dy})
	-- ({-1.1055*\dx},{-0.2196*\dy})
	-- ({-1.1003*\dx},{-0.2214*\dy})
	-- ({-1.0952*\dx},{-0.2230*\dy})
	-- ({-1.0901*\dx},{-0.2245*\dy})
	-- ({-1.0849*\dx},{-0.2259*\dy})
	-- ({-1.0798*\dx},{-0.2272*\dy})
	-- ({-1.0746*\dx},{-0.2283*\dy})
	-- ({-1.0695*\dx},{-0.2293*\dy})
	-- ({-1.0643*\dx},{-0.2302*\dy})
	-- ({-1.0592*\dx},{-0.2310*\dy})
	-- ({-1.0540*\dx},{-0.2317*\dy})
	-- ({-1.0489*\dx},{-0.2322*\dy})
	-- ({-1.0438*\dx},{-0.2326*\dy})
	-- ({-1.0387*\dx},{-0.2329*\dy})
	-- ({-1.0336*\dx},{-0.2331*\dy})
	-- ({-1.0286*\dx},{-0.2332*\dy})
	-- ({-1.0236*\dx},{-0.2332*\dy})
	-- ({-1.0185*\dx},{-0.2331*\dy})
	-- ({-1.0136*\dx},{-0.2329*\dy})
	-- ({-1.0086*\dx},{-0.2325*\dy})
	-- ({-1.0037*\dx},{-0.2321*\dy})
	-- ({-0.9988*\dx},{-0.2316*\dy})
	-- ({-0.9940*\dx},{-0.2309*\dy})
	-- ({-0.9892*\dx},{-0.2302*\dy})
	-- ({-0.9844*\dx},{-0.2294*\dy})
	-- ({-0.9797*\dx},{-0.2284*\dy})
	-- ({-0.9750*\dx},{-0.2274*\dy})
	-- ({-0.9704*\dx},{-0.2263*\dy})
	-- ({-0.9658*\dx},{-0.2251*\dy})
	-- ({-0.9613*\dx},{-0.2238*\dy})
	-- ({-0.9568*\dx},{-0.2225*\dy})
	-- ({-0.9523*\dx},{-0.2210*\dy})
	-- ({-0.9479*\dx},{-0.2195*\dy})
	-- ({-0.9436*\dx},{-0.2179*\dy})
	-- ({-0.9393*\dx},{-0.2162*\dy})
	-- ({-0.9350*\dx},{-0.2144*\dy})
	-- ({-0.9308*\dx},{-0.2126*\dy})
	-- ({-0.9267*\dx},{-0.2107*\dy})
	-- ({-0.9226*\dx},{-0.2087*\dy})
	-- ({-0.9186*\dx},{-0.2066*\dy})
	-- ({-0.9147*\dx},{-0.2045*\dy})
	-- ({-0.9108*\dx},{-0.2023*\dy})
	-- ({-0.9069*\dx},{-0.2001*\dy})
	-- ({-0.9031*\dx},{-0.1977*\dy})
	-- ({-0.8994*\dx},{-0.1954*\dy})
	-- ({-0.8958*\dx},{-0.1929*\dy})
	-- ({-0.8922*\dx},{-0.1904*\dy})
	-- ({-0.8886*\dx},{-0.1879*\dy})
	-- ({-0.8851*\dx},{-0.1853*\dy})
	-- ({-0.8817*\dx},{-0.1826*\dy})
	-- ({-0.8784*\dx},{-0.1799*\dy})
	-- ({-0.8751*\dx},{-0.1771*\dy})
	-- ({-0.8719*\dx},{-0.1743*\dy})
	-- ({-0.8687*\dx},{-0.1714*\dy})
	-- ({-0.8656*\dx},{-0.1685*\dy})
	-- ({-0.8626*\dx},{-0.1656*\dy})
	-- ({-0.8596*\dx},{-0.1626*\dy})
	-- ({-0.8567*\dx},{-0.1595*\dy})
	-- ({-0.8538*\dx},{-0.1564*\dy})
	-- ({-0.8511*\dx},{-0.1533*\dy})
	-- ({-0.8484*\dx},{-0.1502*\dy})
	-- ({-0.8457*\dx},{-0.1470*\dy})
	-- ({-0.8431*\dx},{-0.1437*\dy})
	-- ({-0.8406*\dx},{-0.1405*\dy})
	-- ({-0.8382*\dx},{-0.1371*\dy})
	-- ({-0.8358*\dx},{-0.1338*\dy})
	-- ({-0.8335*\dx},{-0.1304*\dy})
	-- ({-0.8312*\dx},{-0.1270*\dy})
	-- ({-0.8290*\dx},{-0.1236*\dy})
	-- ({-0.8269*\dx},{-0.1201*\dy})
	-- ({-0.8249*\dx},{-0.1167*\dy})
	-- ({-0.8229*\dx},{-0.1131*\dy})
	-- ({-0.8210*\dx},{-0.1096*\dy})
	-- ({-0.8191*\dx},{-0.1060*\dy})
	-- ({-0.8173*\dx},{-0.1024*\dy})
	-- ({-0.8156*\dx},{-0.0988*\dy})
	-- ({-0.8139*\dx},{-0.0952*\dy})
	-- ({-0.8123*\dx},{-0.0916*\dy})
	-- ({-0.8108*\dx},{-0.0879*\dy})
	-- ({-0.8093*\dx},{-0.0842*\dy})
	-- ({-0.8079*\dx},{-0.0805*\dy})
	-- ({-0.8066*\dx},{-0.0768*\dy})
	-- ({-0.8053*\dx},{-0.0730*\dy})
	-- ({-0.8041*\dx},{-0.0693*\dy})
	-- ({-0.8030*\dx},{-0.0655*\dy})
	-- ({-0.8019*\dx},{-0.0617*\dy})
	-- ({-0.8009*\dx},{-0.0579*\dy})
	-- ({-0.8000*\dx},{-0.0541*\dy})
	-- ({-0.7991*\dx},{-0.0503*\dy})
	-- ({-0.7983*\dx},{-0.0465*\dy})
	-- ({-0.7975*\dx},{-0.0426*\dy})
	-- ({-0.7968*\dx},{-0.0388*\dy})
	-- ({-0.7962*\dx},{-0.0349*\dy})
	-- ({-0.7957*\dx},{-0.0311*\dy})
	-- ({-0.7952*\dx},{-0.0272*\dy})
	-- ({-0.7948*\dx},{-0.0233*\dy})
	-- ({-0.7944*\dx},{-0.0194*\dy})
	-- ({-0.7941*\dx},{-0.0156*\dy})
	-- ({-0.7939*\dx},{-0.0117*\dy})
	-- ({-0.7937*\dx},{-0.0078*\dy})
	-- ({-0.7936*\dx},{-0.0039*\dy})
	-- ({-0.7936*\dx},{-0.0000*\dy})
}
% u = -0.914758
\def\upathD{
	({-0.7234*\dx},{0.0000*\dy})
	-- ({-0.7234*\dx},{0.0050*\dy})
	-- ({-0.7236*\dx},{0.0100*\dy})
	-- ({-0.7238*\dx},{0.0150*\dy})
	-- ({-0.7240*\dx},{0.0200*\dy})
	-- ({-0.7244*\dx},{0.0250*\dy})
	-- ({-0.7248*\dx},{0.0300*\dy})
	-- ({-0.7253*\dx},{0.0350*\dy})
	-- ({-0.7258*\dx},{0.0400*\dy})
	-- ({-0.7265*\dx},{0.0450*\dy})
	-- ({-0.7272*\dx},{0.0500*\dy})
	-- ({-0.7280*\dx},{0.0550*\dy})
	-- ({-0.7289*\dx},{0.0599*\dy})
	-- ({-0.7299*\dx},{0.0649*\dy})
	-- ({-0.7309*\dx},{0.0698*\dy})
	-- ({-0.7320*\dx},{0.0748*\dy})
	-- ({-0.7332*\dx},{0.0797*\dy})
	-- ({-0.7345*\dx},{0.0846*\dy})
	-- ({-0.7358*\dx},{0.0895*\dy})
	-- ({-0.7372*\dx},{0.0944*\dy})
	-- ({-0.7387*\dx},{0.0993*\dy})
	-- ({-0.7403*\dx},{0.1042*\dy})
	-- ({-0.7420*\dx},{0.1090*\dy})
	-- ({-0.7437*\dx},{0.1139*\dy})
	-- ({-0.7455*\dx},{0.1187*\dy})
	-- ({-0.7474*\dx},{0.1235*\dy})
	-- ({-0.7494*\dx},{0.1283*\dy})
	-- ({-0.7515*\dx},{0.1331*\dy})
	-- ({-0.7536*\dx},{0.1378*\dy})
	-- ({-0.7558*\dx},{0.1425*\dy})
	-- ({-0.7581*\dx},{0.1472*\dy})
	-- ({-0.7605*\dx},{0.1519*\dy})
	-- ({-0.7630*\dx},{0.1566*\dy})
	-- ({-0.7655*\dx},{0.1612*\dy})
	-- ({-0.7682*\dx},{0.1658*\dy})
	-- ({-0.7709*\dx},{0.1704*\dy})
	-- ({-0.7737*\dx},{0.1750*\dy})
	-- ({-0.7766*\dx},{0.1795*\dy})
	-- ({-0.7796*\dx},{0.1840*\dy})
	-- ({-0.7826*\dx},{0.1884*\dy})
	-- ({-0.7858*\dx},{0.1929*\dy})
	-- ({-0.7890*\dx},{0.1973*\dy})
	-- ({-0.7923*\dx},{0.2016*\dy})
	-- ({-0.7957*\dx},{0.2059*\dy})
	-- ({-0.7992*\dx},{0.2102*\dy})
	-- ({-0.8028*\dx},{0.2145*\dy})
	-- ({-0.8064*\dx},{0.2187*\dy})
	-- ({-0.8102*\dx},{0.2228*\dy})
	-- ({-0.8140*\dx},{0.2269*\dy})
	-- ({-0.8179*\dx},{0.2310*\dy})
	-- ({-0.8220*\dx},{0.2350*\dy})
	-- ({-0.8261*\dx},{0.2390*\dy})
	-- ({-0.8303*\dx},{0.2429*\dy})
	-- ({-0.8345*\dx},{0.2467*\dy})
	-- ({-0.8389*\dx},{0.2505*\dy})
	-- ({-0.8434*\dx},{0.2543*\dy})
	-- ({-0.8479*\dx},{0.2580*\dy})
	-- ({-0.8526*\dx},{0.2616*\dy})
	-- ({-0.8573*\dx},{0.2651*\dy})
	-- ({-0.8621*\dx},{0.2686*\dy})
	-- ({-0.8670*\dx},{0.2720*\dy})
	-- ({-0.8720*\dx},{0.2754*\dy})
	-- ({-0.8771*\dx},{0.2787*\dy})
	-- ({-0.8823*\dx},{0.2819*\dy})
	-- ({-0.8876*\dx},{0.2850*\dy})
	-- ({-0.8929*\dx},{0.2880*\dy})
	-- ({-0.8984*\dx},{0.2910*\dy})
	-- ({-0.9039*\dx},{0.2939*\dy})
	-- ({-0.9095*\dx},{0.2967*\dy})
	-- ({-0.9153*\dx},{0.2993*\dy})
	-- ({-0.9211*\dx},{0.3019*\dy})
	-- ({-0.9269*\dx},{0.3045*\dy})
	-- ({-0.9329*\dx},{0.3069*\dy})
	-- ({-0.9390*\dx},{0.3092*\dy})
	-- ({-0.9451*\dx},{0.3113*\dy})
	-- ({-0.9513*\dx},{0.3134*\dy})
	-- ({-0.9576*\dx},{0.3154*\dy})
	-- ({-0.9640*\dx},{0.3173*\dy})
	-- ({-0.9705*\dx},{0.3190*\dy})
	-- ({-0.9770*\dx},{0.3206*\dy})
	-- ({-0.9837*\dx},{0.3221*\dy})
	-- ({-0.9904*\dx},{0.3235*\dy})
	-- ({-0.9971*\dx},{0.3247*\dy})
	-- ({-1.0040*\dx},{0.3258*\dy})
	-- ({-1.0109*\dx},{0.3268*\dy})
	-- ({-1.0179*\dx},{0.3276*\dy})
	-- ({-1.0249*\dx},{0.3283*\dy})
	-- ({-1.0320*\dx},{0.3288*\dy})
	-- ({-1.0392*\dx},{0.3292*\dy})
	-- ({-1.0464*\dx},{0.3294*\dy})
	-- ({-1.0536*\dx},{0.3295*\dy})
	-- ({-1.0610*\dx},{0.3294*\dy})
	-- ({-1.0683*\dx},{0.3291*\dy})
	-- ({-1.0757*\dx},{0.3287*\dy})
	-- ({-1.0832*\dx},{0.3281*\dy})
	-- ({-1.0907*\dx},{0.3273*\dy})
	-- ({-1.0982*\dx},{0.3263*\dy})
	-- ({-1.1057*\dx},{0.3252*\dy})
	-- ({-1.1133*\dx},{0.3239*\dy})
	-- ({-1.1209*\dx},{0.3224*\dy})
	-- ({-1.1285*\dx},{0.3207*\dy})
	-- ({-1.1361*\dx},{0.3188*\dy})
	-- ({-1.1437*\dx},{0.3167*\dy})
	-- ({-1.1513*\dx},{0.3144*\dy})
	-- ({-1.1589*\dx},{0.3119*\dy})
	-- ({-1.1665*\dx},{0.3093*\dy})
	-- ({-1.1741*\dx},{0.3064*\dy})
	-- ({-1.1816*\dx},{0.3033*\dy})
	-- ({-1.1891*\dx},{0.3000*\dy})
	-- ({-1.1966*\dx},{0.2965*\dy})
	-- ({-1.2040*\dx},{0.2927*\dy})
	-- ({-1.2114*\dx},{0.2888*\dy})
	-- ({-1.2187*\dx},{0.2847*\dy})
	-- ({-1.2260*\dx},{0.2803*\dy})
	-- ({-1.2332*\dx},{0.2758*\dy})
	-- ({-1.2403*\dx},{0.2710*\dy})
	-- ({-1.2473*\dx},{0.2660*\dy})
	-- ({-1.2542*\dx},{0.2608*\dy})
	-- ({-1.2610*\dx},{0.2554*\dy})
	-- ({-1.2678*\dx},{0.2498*\dy})
	-- ({-1.2744*\dx},{0.2439*\dy})
	-- ({-1.2808*\dx},{0.2379*\dy})
	-- ({-1.2872*\dx},{0.2317*\dy})
	-- ({-1.2933*\dx},{0.2252*\dy})
	-- ({-1.2994*\dx},{0.2186*\dy})
	-- ({-1.3053*\dx},{0.2118*\dy})
	-- ({-1.3110*\dx},{0.2047*\dy})
	-- ({-1.3166*\dx},{0.1975*\dy})
	-- ({-1.3219*\dx},{0.1901*\dy})
	-- ({-1.3271*\dx},{0.1826*\dy})
	-- ({-1.3321*\dx},{0.1749*\dy})
	-- ({-1.3369*\dx},{0.1670*\dy})
	-- ({-1.3415*\dx},{0.1589*\dy})
	-- ({-1.3459*\dx},{0.1507*\dy})
	-- ({-1.3500*\dx},{0.1423*\dy})
	-- ({-1.3539*\dx},{0.1338*\dy})
	-- ({-1.3576*\dx},{0.1252*\dy})
	-- ({-1.3611*\dx},{0.1164*\dy})
	-- ({-1.3643*\dx},{0.1076*\dy})
	-- ({-1.3672*\dx},{0.0986*\dy})
	-- ({-1.3700*\dx},{0.0895*\dy})
	-- ({-1.3724*\dx},{0.0803*\dy})
	-- ({-1.3746*\dx},{0.0711*\dy})
	-- ({-1.3765*\dx},{0.0618*\dy})
	-- ({-1.3782*\dx},{0.0524*\dy})
	-- ({-1.3795*\dx},{0.0429*\dy})
	-- ({-1.3806*\dx},{0.0334*\dy})
	-- ({-1.3815*\dx},{0.0239*\dy})
	-- ({-1.3820*\dx},{0.0144*\dy})
	-- ({-1.3823*\dx},{0.0048*\dy})
	-- ({-1.3823*\dx},{-0.0048*\dy})
	-- ({-1.3820*\dx},{-0.0144*\dy})
	-- ({-1.3815*\dx},{-0.0239*\dy})
	-- ({-1.3806*\dx},{-0.0334*\dy})
	-- ({-1.3795*\dx},{-0.0429*\dy})
	-- ({-1.3782*\dx},{-0.0524*\dy})
	-- ({-1.3765*\dx},{-0.0618*\dy})
	-- ({-1.3746*\dx},{-0.0711*\dy})
	-- ({-1.3724*\dx},{-0.0803*\dy})
	-- ({-1.3700*\dx},{-0.0895*\dy})
	-- ({-1.3672*\dx},{-0.0986*\dy})
	-- ({-1.3643*\dx},{-0.1076*\dy})
	-- ({-1.3611*\dx},{-0.1164*\dy})
	-- ({-1.3576*\dx},{-0.1252*\dy})
	-- ({-1.3539*\dx},{-0.1338*\dy})
	-- ({-1.3500*\dx},{-0.1423*\dy})
	-- ({-1.3459*\dx},{-0.1507*\dy})
	-- ({-1.3415*\dx},{-0.1589*\dy})
	-- ({-1.3369*\dx},{-0.1670*\dy})
	-- ({-1.3321*\dx},{-0.1749*\dy})
	-- ({-1.3271*\dx},{-0.1826*\dy})
	-- ({-1.3219*\dx},{-0.1901*\dy})
	-- ({-1.3166*\dx},{-0.1975*\dy})
	-- ({-1.3110*\dx},{-0.2047*\dy})
	-- ({-1.3053*\dx},{-0.2118*\dy})
	-- ({-1.2994*\dx},{-0.2186*\dy})
	-- ({-1.2933*\dx},{-0.2252*\dy})
	-- ({-1.2872*\dx},{-0.2317*\dy})
	-- ({-1.2808*\dx},{-0.2379*\dy})
	-- ({-1.2744*\dx},{-0.2439*\dy})
	-- ({-1.2678*\dx},{-0.2498*\dy})
	-- ({-1.2610*\dx},{-0.2554*\dy})
	-- ({-1.2542*\dx},{-0.2608*\dy})
	-- ({-1.2473*\dx},{-0.2660*\dy})
	-- ({-1.2403*\dx},{-0.2710*\dy})
	-- ({-1.2332*\dx},{-0.2758*\dy})
	-- ({-1.2260*\dx},{-0.2803*\dy})
	-- ({-1.2187*\dx},{-0.2847*\dy})
	-- ({-1.2114*\dx},{-0.2888*\dy})
	-- ({-1.2040*\dx},{-0.2927*\dy})
	-- ({-1.1966*\dx},{-0.2965*\dy})
	-- ({-1.1891*\dx},{-0.3000*\dy})
	-- ({-1.1816*\dx},{-0.3033*\dy})
	-- ({-1.1741*\dx},{-0.3064*\dy})
	-- ({-1.1665*\dx},{-0.3093*\dy})
	-- ({-1.1589*\dx},{-0.3119*\dy})
	-- ({-1.1513*\dx},{-0.3144*\dy})
	-- ({-1.1437*\dx},{-0.3167*\dy})
	-- ({-1.1361*\dx},{-0.3188*\dy})
	-- ({-1.1285*\dx},{-0.3207*\dy})
	-- ({-1.1209*\dx},{-0.3224*\dy})
	-- ({-1.1133*\dx},{-0.3239*\dy})
	-- ({-1.1057*\dx},{-0.3252*\dy})
	-- ({-1.0982*\dx},{-0.3263*\dy})
	-- ({-1.0907*\dx},{-0.3273*\dy})
	-- ({-1.0832*\dx},{-0.3281*\dy})
	-- ({-1.0757*\dx},{-0.3287*\dy})
	-- ({-1.0683*\dx},{-0.3291*\dy})
	-- ({-1.0610*\dx},{-0.3294*\dy})
	-- ({-1.0536*\dx},{-0.3295*\dy})
	-- ({-1.0464*\dx},{-0.3294*\dy})
	-- ({-1.0392*\dx},{-0.3292*\dy})
	-- ({-1.0320*\dx},{-0.3288*\dy})
	-- ({-1.0249*\dx},{-0.3283*\dy})
	-- ({-1.0179*\dx},{-0.3276*\dy})
	-- ({-1.0109*\dx},{-0.3268*\dy})
	-- ({-1.0040*\dx},{-0.3258*\dy})
	-- ({-0.9971*\dx},{-0.3247*\dy})
	-- ({-0.9904*\dx},{-0.3235*\dy})
	-- ({-0.9837*\dx},{-0.3221*\dy})
	-- ({-0.9770*\dx},{-0.3206*\dy})
	-- ({-0.9705*\dx},{-0.3190*\dy})
	-- ({-0.9640*\dx},{-0.3173*\dy})
	-- ({-0.9576*\dx},{-0.3154*\dy})
	-- ({-0.9513*\dx},{-0.3134*\dy})
	-- ({-0.9451*\dx},{-0.3113*\dy})
	-- ({-0.9390*\dx},{-0.3092*\dy})
	-- ({-0.9329*\dx},{-0.3069*\dy})
	-- ({-0.9269*\dx},{-0.3045*\dy})
	-- ({-0.9211*\dx},{-0.3019*\dy})
	-- ({-0.9153*\dx},{-0.2993*\dy})
	-- ({-0.9095*\dx},{-0.2967*\dy})
	-- ({-0.9039*\dx},{-0.2939*\dy})
	-- ({-0.8984*\dx},{-0.2910*\dy})
	-- ({-0.8929*\dx},{-0.2880*\dy})
	-- ({-0.8876*\dx},{-0.2850*\dy})
	-- ({-0.8823*\dx},{-0.2819*\dy})
	-- ({-0.8771*\dx},{-0.2787*\dy})
	-- ({-0.8720*\dx},{-0.2754*\dy})
	-- ({-0.8670*\dx},{-0.2720*\dy})
	-- ({-0.8621*\dx},{-0.2686*\dy})
	-- ({-0.8573*\dx},{-0.2651*\dy})
	-- ({-0.8526*\dx},{-0.2616*\dy})
	-- ({-0.8479*\dx},{-0.2580*\dy})
	-- ({-0.8434*\dx},{-0.2543*\dy})
	-- ({-0.8389*\dx},{-0.2505*\dy})
	-- ({-0.8345*\dx},{-0.2467*\dy})
	-- ({-0.8303*\dx},{-0.2429*\dy})
	-- ({-0.8261*\dx},{-0.2390*\dy})
	-- ({-0.8220*\dx},{-0.2350*\dy})
	-- ({-0.8179*\dx},{-0.2310*\dy})
	-- ({-0.8140*\dx},{-0.2269*\dy})
	-- ({-0.8102*\dx},{-0.2228*\dy})
	-- ({-0.8064*\dx},{-0.2187*\dy})
	-- ({-0.8028*\dx},{-0.2145*\dy})
	-- ({-0.7992*\dx},{-0.2102*\dy})
	-- ({-0.7957*\dx},{-0.2059*\dy})
	-- ({-0.7923*\dx},{-0.2016*\dy})
	-- ({-0.7890*\dx},{-0.1973*\dy})
	-- ({-0.7858*\dx},{-0.1929*\dy})
	-- ({-0.7826*\dx},{-0.1884*\dy})
	-- ({-0.7796*\dx},{-0.1840*\dy})
	-- ({-0.7766*\dx},{-0.1795*\dy})
	-- ({-0.7737*\dx},{-0.1750*\dy})
	-- ({-0.7709*\dx},{-0.1704*\dy})
	-- ({-0.7682*\dx},{-0.1658*\dy})
	-- ({-0.7655*\dx},{-0.1612*\dy})
	-- ({-0.7630*\dx},{-0.1566*\dy})
	-- ({-0.7605*\dx},{-0.1519*\dy})
	-- ({-0.7581*\dx},{-0.1472*\dy})
	-- ({-0.7558*\dx},{-0.1425*\dy})
	-- ({-0.7536*\dx},{-0.1378*\dy})
	-- ({-0.7515*\dx},{-0.1331*\dy})
	-- ({-0.7494*\dx},{-0.1283*\dy})
	-- ({-0.7474*\dx},{-0.1235*\dy})
	-- ({-0.7455*\dx},{-0.1187*\dy})
	-- ({-0.7437*\dx},{-0.1139*\dy})
	-- ({-0.7420*\dx},{-0.1090*\dy})
	-- ({-0.7403*\dx},{-0.1042*\dy})
	-- ({-0.7387*\dx},{-0.0993*\dy})
	-- ({-0.7372*\dx},{-0.0944*\dy})
	-- ({-0.7358*\dx},{-0.0895*\dy})
	-- ({-0.7345*\dx},{-0.0846*\dy})
	-- ({-0.7332*\dx},{-0.0797*\dy})
	-- ({-0.7320*\dx},{-0.0748*\dy})
	-- ({-0.7309*\dx},{-0.0698*\dy})
	-- ({-0.7299*\dx},{-0.0649*\dy})
	-- ({-0.7289*\dx},{-0.0599*\dy})
	-- ({-0.7280*\dx},{-0.0550*\dy})
	-- ({-0.7272*\dx},{-0.0500*\dy})
	-- ({-0.7265*\dx},{-0.0450*\dy})
	-- ({-0.7258*\dx},{-0.0400*\dy})
	-- ({-0.7253*\dx},{-0.0350*\dy})
	-- ({-0.7248*\dx},{-0.0300*\dy})
	-- ({-0.7244*\dx},{-0.0250*\dy})
	-- ({-0.7240*\dx},{-0.0200*\dy})
	-- ({-0.7238*\dx},{-0.0150*\dy})
	-- ({-0.7236*\dx},{-0.0100*\dy})
	-- ({-0.7234*\dx},{-0.0050*\dy})
	-- ({-0.7234*\dx},{-0.0000*\dy})
}
% u = -0.748438
\def\upathE{
	({-0.6342*\dx},{0.0000*\dy})
	-- ({-0.6343*\dx},{0.0063*\dy})
	-- ({-0.6344*\dx},{0.0126*\dy})
	-- ({-0.6346*\dx},{0.0188*\dy})
	-- ({-0.6349*\dx},{0.0251*\dy})
	-- ({-0.6353*\dx},{0.0314*\dy})
	-- ({-0.6357*\dx},{0.0377*\dy})
	-- ({-0.6363*\dx},{0.0439*\dy})
	-- ({-0.6369*\dx},{0.0502*\dy})
	-- ({-0.6376*\dx},{0.0565*\dy})
	-- ({-0.6384*\dx},{0.0628*\dy})
	-- ({-0.6393*\dx},{0.0690*\dy})
	-- ({-0.6403*\dx},{0.0753*\dy})
	-- ({-0.6413*\dx},{0.0815*\dy})
	-- ({-0.6425*\dx},{0.0878*\dy})
	-- ({-0.6437*\dx},{0.0940*\dy})
	-- ({-0.6450*\dx},{0.1003*\dy})
	-- ({-0.6464*\dx},{0.1065*\dy})
	-- ({-0.6479*\dx},{0.1128*\dy})
	-- ({-0.6495*\dx},{0.1190*\dy})
	-- ({-0.6512*\dx},{0.1252*\dy})
	-- ({-0.6529*\dx},{0.1314*\dy})
	-- ({-0.6548*\dx},{0.1376*\dy})
	-- ({-0.6567*\dx},{0.1438*\dy})
	-- ({-0.6587*\dx},{0.1500*\dy})
	-- ({-0.6609*\dx},{0.1562*\dy})
	-- ({-0.6631*\dx},{0.1624*\dy})
	-- ({-0.6654*\dx},{0.1685*\dy})
	-- ({-0.6678*\dx},{0.1747*\dy})
	-- ({-0.6703*\dx},{0.1808*\dy})
	-- ({-0.6729*\dx},{0.1869*\dy})
	-- ({-0.6756*\dx},{0.1930*\dy})
	-- ({-0.6784*\dx},{0.1991*\dy})
	-- ({-0.6812*\dx},{0.2052*\dy})
	-- ({-0.6842*\dx},{0.2113*\dy})
	-- ({-0.6873*\dx},{0.2173*\dy})
	-- ({-0.6905*\dx},{0.2234*\dy})
	-- ({-0.6938*\dx},{0.2294*\dy})
	-- ({-0.6972*\dx},{0.2354*\dy})
	-- ({-0.7007*\dx},{0.2413*\dy})
	-- ({-0.7043*\dx},{0.2473*\dy})
	-- ({-0.7080*\dx},{0.2532*\dy})
	-- ({-0.7118*\dx},{0.2591*\dy})
	-- ({-0.7158*\dx},{0.2650*\dy})
	-- ({-0.7198*\dx},{0.2708*\dy})
	-- ({-0.7240*\dx},{0.2767*\dy})
	-- ({-0.7282*\dx},{0.2824*\dy})
	-- ({-0.7326*\dx},{0.2882*\dy})
	-- ({-0.7371*\dx},{0.2939*\dy})
	-- ({-0.7417*\dx},{0.2996*\dy})
	-- ({-0.7464*\dx},{0.3053*\dy})
	-- ({-0.7513*\dx},{0.3109*\dy})
	-- ({-0.7563*\dx},{0.3164*\dy})
	-- ({-0.7614*\dx},{0.3220*\dy})
	-- ({-0.7666*\dx},{0.3275*\dy})
	-- ({-0.7719*\dx},{0.3329*\dy})
	-- ({-0.7774*\dx},{0.3383*\dy})
	-- ({-0.7830*\dx},{0.3436*\dy})
	-- ({-0.7887*\dx},{0.3489*\dy})
	-- ({-0.7945*\dx},{0.3541*\dy})
	-- ({-0.8005*\dx},{0.3593*\dy})
	-- ({-0.8066*\dx},{0.3644*\dy})
	-- ({-0.8128*\dx},{0.3694*\dy})
	-- ({-0.8192*\dx},{0.3744*\dy})
	-- ({-0.8257*\dx},{0.3793*\dy})
	-- ({-0.8324*\dx},{0.3841*\dy})
	-- ({-0.8392*\dx},{0.3888*\dy})
	-- ({-0.8461*\dx},{0.3935*\dy})
	-- ({-0.8532*\dx},{0.3980*\dy})
	-- ({-0.8604*\dx},{0.4025*\dy})
	-- ({-0.8677*\dx},{0.4069*\dy})
	-- ({-0.8752*\dx},{0.4112*\dy})
	-- ({-0.8828*\dx},{0.4154*\dy})
	-- ({-0.8906*\dx},{0.4194*\dy})
	-- ({-0.8985*\dx},{0.4234*\dy})
	-- ({-0.9066*\dx},{0.4273*\dy})
	-- ({-0.9148*\dx},{0.4310*\dy})
	-- ({-0.9232*\dx},{0.4346*\dy})
	-- ({-0.9317*\dx},{0.4380*\dy})
	-- ({-0.9403*\dx},{0.4414*\dy})
	-- ({-0.9491*\dx},{0.4446*\dy})
	-- ({-0.9581*\dx},{0.4476*\dy})
	-- ({-0.9672*\dx},{0.4505*\dy})
	-- ({-0.9764*\dx},{0.4532*\dy})
	-- ({-0.9858*\dx},{0.4558*\dy})
	-- ({-0.9953*\dx},{0.4582*\dy})
	-- ({-1.0049*\dx},{0.4604*\dy})
	-- ({-1.0147*\dx},{0.4624*\dy})
	-- ({-1.0246*\dx},{0.4643*\dy})
	-- ({-1.0347*\dx},{0.4659*\dy})
	-- ({-1.0449*\dx},{0.4674*\dy})
	-- ({-1.0552*\dx},{0.4686*\dy})
	-- ({-1.0657*\dx},{0.4696*\dy})
	-- ({-1.0763*\dx},{0.4704*\dy})
	-- ({-1.0870*\dx},{0.4709*\dy})
	-- ({-1.0978*\dx},{0.4712*\dy})
	-- ({-1.1087*\dx},{0.4713*\dy})
	-- ({-1.1197*\dx},{0.4711*\dy})
	-- ({-1.1309*\dx},{0.4706*\dy})
	-- ({-1.1421*\dx},{0.4698*\dy})
	-- ({-1.1534*\dx},{0.4688*\dy})
	-- ({-1.1648*\dx},{0.4675*\dy})
	-- ({-1.1763*\dx},{0.4659*\dy})
	-- ({-1.1878*\dx},{0.4640*\dy})
	-- ({-1.1995*\dx},{0.4618*\dy})
	-- ({-1.2111*\dx},{0.4593*\dy})
	-- ({-1.2228*\dx},{0.4564*\dy})
	-- ({-1.2346*\dx},{0.4532*\dy})
	-- ({-1.2463*\dx},{0.4497*\dy})
	-- ({-1.2581*\dx},{0.4459*\dy})
	-- ({-1.2699*\dx},{0.4416*\dy})
	-- ({-1.2817*\dx},{0.4371*\dy})
	-- ({-1.2935*\dx},{0.4322*\dy})
	-- ({-1.3052*\dx},{0.4269*\dy})
	-- ({-1.3169*\dx},{0.4212*\dy})
	-- ({-1.3285*\dx},{0.4152*\dy})
	-- ({-1.3400*\dx},{0.4087*\dy})
	-- ({-1.3515*\dx},{0.4020*\dy})
	-- ({-1.3629*\dx},{0.3948*\dy})
	-- ({-1.3741*\dx},{0.3872*\dy})
	-- ({-1.3852*\dx},{0.3793*\dy})
	-- ({-1.3962*\dx},{0.3709*\dy})
	-- ({-1.4070*\dx},{0.3622*\dy})
	-- ({-1.4176*\dx},{0.3531*\dy})
	-- ({-1.4280*\dx},{0.3436*\dy})
	-- ({-1.4382*\dx},{0.3337*\dy})
	-- ({-1.4482*\dx},{0.3235*\dy})
	-- ({-1.4579*\dx},{0.3129*\dy})
	-- ({-1.4673*\dx},{0.3019*\dy})
	-- ({-1.4765*\dx},{0.2906*\dy})
	-- ({-1.4854*\dx},{0.2789*\dy})
	-- ({-1.4939*\dx},{0.2669*\dy})
	-- ({-1.5021*\dx},{0.2545*\dy})
	-- ({-1.5100*\dx},{0.2418*\dy})
	-- ({-1.5175*\dx},{0.2288*\dy})
	-- ({-1.5246*\dx},{0.2155*\dy})
	-- ({-1.5313*\dx},{0.2020*\dy})
	-- ({-1.5376*\dx},{0.1882*\dy})
	-- ({-1.5434*\dx},{0.1741*\dy})
	-- ({-1.5488*\dx},{0.1598*\dy})
	-- ({-1.5538*\dx},{0.1452*\dy})
	-- ({-1.5583*\dx},{0.1305*\dy})
	-- ({-1.5624*\dx},{0.1156*\dy})
	-- ({-1.5659*\dx},{0.1005*\dy})
	-- ({-1.5690*\dx},{0.0853*\dy})
	-- ({-1.5715*\dx},{0.0699*\dy})
	-- ({-1.5736*\dx},{0.0545*\dy})
	-- ({-1.5751*\dx},{0.0390*\dy})
	-- ({-1.5762*\dx},{0.0234*\dy})
	-- ({-1.5767*\dx},{0.0078*\dy})
	-- ({-1.5767*\dx},{-0.0078*\dy})
	-- ({-1.5762*\dx},{-0.0234*\dy})
	-- ({-1.5751*\dx},{-0.0390*\dy})
	-- ({-1.5736*\dx},{-0.0545*\dy})
	-- ({-1.5715*\dx},{-0.0699*\dy})
	-- ({-1.5690*\dx},{-0.0853*\dy})
	-- ({-1.5659*\dx},{-0.1005*\dy})
	-- ({-1.5624*\dx},{-0.1156*\dy})
	-- ({-1.5583*\dx},{-0.1305*\dy})
	-- ({-1.5538*\dx},{-0.1452*\dy})
	-- ({-1.5488*\dx},{-0.1598*\dy})
	-- ({-1.5434*\dx},{-0.1741*\dy})
	-- ({-1.5376*\dx},{-0.1882*\dy})
	-- ({-1.5313*\dx},{-0.2020*\dy})
	-- ({-1.5246*\dx},{-0.2155*\dy})
	-- ({-1.5175*\dx},{-0.2288*\dy})
	-- ({-1.5100*\dx},{-0.2418*\dy})
	-- ({-1.5021*\dx},{-0.2545*\dy})
	-- ({-1.4939*\dx},{-0.2669*\dy})
	-- ({-1.4854*\dx},{-0.2789*\dy})
	-- ({-1.4765*\dx},{-0.2906*\dy})
	-- ({-1.4673*\dx},{-0.3019*\dy})
	-- ({-1.4579*\dx},{-0.3129*\dy})
	-- ({-1.4482*\dx},{-0.3235*\dy})
	-- ({-1.4382*\dx},{-0.3337*\dy})
	-- ({-1.4280*\dx},{-0.3436*\dy})
	-- ({-1.4176*\dx},{-0.3531*\dy})
	-- ({-1.4070*\dx},{-0.3622*\dy})
	-- ({-1.3962*\dx},{-0.3709*\dy})
	-- ({-1.3852*\dx},{-0.3793*\dy})
	-- ({-1.3741*\dx},{-0.3872*\dy})
	-- ({-1.3629*\dx},{-0.3948*\dy})
	-- ({-1.3515*\dx},{-0.4020*\dy})
	-- ({-1.3400*\dx},{-0.4087*\dy})
	-- ({-1.3285*\dx},{-0.4152*\dy})
	-- ({-1.3169*\dx},{-0.4212*\dy})
	-- ({-1.3052*\dx},{-0.4269*\dy})
	-- ({-1.2935*\dx},{-0.4322*\dy})
	-- ({-1.2817*\dx},{-0.4371*\dy})
	-- ({-1.2699*\dx},{-0.4416*\dy})
	-- ({-1.2581*\dx},{-0.4459*\dy})
	-- ({-1.2463*\dx},{-0.4497*\dy})
	-- ({-1.2346*\dx},{-0.4532*\dy})
	-- ({-1.2228*\dx},{-0.4564*\dy})
	-- ({-1.2111*\dx},{-0.4593*\dy})
	-- ({-1.1995*\dx},{-0.4618*\dy})
	-- ({-1.1878*\dx},{-0.4640*\dy})
	-- ({-1.1763*\dx},{-0.4659*\dy})
	-- ({-1.1648*\dx},{-0.4675*\dy})
	-- ({-1.1534*\dx},{-0.4688*\dy})
	-- ({-1.1421*\dx},{-0.4698*\dy})
	-- ({-1.1309*\dx},{-0.4706*\dy})
	-- ({-1.1197*\dx},{-0.4711*\dy})
	-- ({-1.1087*\dx},{-0.4713*\dy})
	-- ({-1.0978*\dx},{-0.4712*\dy})
	-- ({-1.0870*\dx},{-0.4709*\dy})
	-- ({-1.0763*\dx},{-0.4704*\dy})
	-- ({-1.0657*\dx},{-0.4696*\dy})
	-- ({-1.0552*\dx},{-0.4686*\dy})
	-- ({-1.0449*\dx},{-0.4674*\dy})
	-- ({-1.0347*\dx},{-0.4659*\dy})
	-- ({-1.0246*\dx},{-0.4643*\dy})
	-- ({-1.0147*\dx},{-0.4624*\dy})
	-- ({-1.0049*\dx},{-0.4604*\dy})
	-- ({-0.9953*\dx},{-0.4582*\dy})
	-- ({-0.9858*\dx},{-0.4558*\dy})
	-- ({-0.9764*\dx},{-0.4532*\dy})
	-- ({-0.9672*\dx},{-0.4505*\dy})
	-- ({-0.9581*\dx},{-0.4476*\dy})
	-- ({-0.9491*\dx},{-0.4446*\dy})
	-- ({-0.9403*\dx},{-0.4414*\dy})
	-- ({-0.9317*\dx},{-0.4380*\dy})
	-- ({-0.9232*\dx},{-0.4346*\dy})
	-- ({-0.9148*\dx},{-0.4310*\dy})
	-- ({-0.9066*\dx},{-0.4273*\dy})
	-- ({-0.8985*\dx},{-0.4234*\dy})
	-- ({-0.8906*\dx},{-0.4194*\dy})
	-- ({-0.8828*\dx},{-0.4154*\dy})
	-- ({-0.8752*\dx},{-0.4112*\dy})
	-- ({-0.8677*\dx},{-0.4069*\dy})
	-- ({-0.8604*\dx},{-0.4025*\dy})
	-- ({-0.8532*\dx},{-0.3980*\dy})
	-- ({-0.8461*\dx},{-0.3935*\dy})
	-- ({-0.8392*\dx},{-0.3888*\dy})
	-- ({-0.8324*\dx},{-0.3841*\dy})
	-- ({-0.8257*\dx},{-0.3793*\dy})
	-- ({-0.8192*\dx},{-0.3744*\dy})
	-- ({-0.8128*\dx},{-0.3694*\dy})
	-- ({-0.8066*\dx},{-0.3644*\dy})
	-- ({-0.8005*\dx},{-0.3593*\dy})
	-- ({-0.7945*\dx},{-0.3541*\dy})
	-- ({-0.7887*\dx},{-0.3489*\dy})
	-- ({-0.7830*\dx},{-0.3436*\dy})
	-- ({-0.7774*\dx},{-0.3383*\dy})
	-- ({-0.7719*\dx},{-0.3329*\dy})
	-- ({-0.7666*\dx},{-0.3275*\dy})
	-- ({-0.7614*\dx},{-0.3220*\dy})
	-- ({-0.7563*\dx},{-0.3164*\dy})
	-- ({-0.7513*\dx},{-0.3109*\dy})
	-- ({-0.7464*\dx},{-0.3053*\dy})
	-- ({-0.7417*\dx},{-0.2996*\dy})
	-- ({-0.7371*\dx},{-0.2939*\dy})
	-- ({-0.7326*\dx},{-0.2882*\dy})
	-- ({-0.7282*\dx},{-0.2824*\dy})
	-- ({-0.7240*\dx},{-0.2767*\dy})
	-- ({-0.7198*\dx},{-0.2708*\dy})
	-- ({-0.7158*\dx},{-0.2650*\dy})
	-- ({-0.7118*\dx},{-0.2591*\dy})
	-- ({-0.7080*\dx},{-0.2532*\dy})
	-- ({-0.7043*\dx},{-0.2473*\dy})
	-- ({-0.7007*\dx},{-0.2413*\dy})
	-- ({-0.6972*\dx},{-0.2354*\dy})
	-- ({-0.6938*\dx},{-0.2294*\dy})
	-- ({-0.6905*\dx},{-0.2234*\dy})
	-- ({-0.6873*\dx},{-0.2173*\dy})
	-- ({-0.6842*\dx},{-0.2113*\dy})
	-- ({-0.6812*\dx},{-0.2052*\dy})
	-- ({-0.6784*\dx},{-0.1991*\dy})
	-- ({-0.6756*\dx},{-0.1930*\dy})
	-- ({-0.6729*\dx},{-0.1869*\dy})
	-- ({-0.6703*\dx},{-0.1808*\dy})
	-- ({-0.6678*\dx},{-0.1747*\dy})
	-- ({-0.6654*\dx},{-0.1685*\dy})
	-- ({-0.6631*\dx},{-0.1624*\dy})
	-- ({-0.6609*\dx},{-0.1562*\dy})
	-- ({-0.6587*\dx},{-0.1500*\dy})
	-- ({-0.6567*\dx},{-0.1438*\dy})
	-- ({-0.6548*\dx},{-0.1376*\dy})
	-- ({-0.6529*\dx},{-0.1314*\dy})
	-- ({-0.6512*\dx},{-0.1252*\dy})
	-- ({-0.6495*\dx},{-0.1190*\dy})
	-- ({-0.6479*\dx},{-0.1128*\dy})
	-- ({-0.6464*\dx},{-0.1065*\dy})
	-- ({-0.6450*\dx},{-0.1003*\dy})
	-- ({-0.6437*\dx},{-0.0940*\dy})
	-- ({-0.6425*\dx},{-0.0878*\dy})
	-- ({-0.6413*\dx},{-0.0815*\dy})
	-- ({-0.6403*\dx},{-0.0753*\dy})
	-- ({-0.6393*\dx},{-0.0690*\dy})
	-- ({-0.6384*\dx},{-0.0628*\dy})
	-- ({-0.6376*\dx},{-0.0565*\dy})
	-- ({-0.6369*\dx},{-0.0502*\dy})
	-- ({-0.6363*\dx},{-0.0439*\dy})
	-- ({-0.6357*\dx},{-0.0377*\dy})
	-- ({-0.6353*\dx},{-0.0314*\dy})
	-- ({-0.6349*\dx},{-0.0251*\dy})
	-- ({-0.6346*\dx},{-0.0188*\dy})
	-- ({-0.6344*\dx},{-0.0126*\dy})
	-- ({-0.6343*\dx},{-0.0063*\dy})
	-- ({-0.6342*\dx},{-0.0000*\dy})
}
% u = -0.582119
\def\upathF{
	({-0.5242*\dx},{0.0000*\dy})
	-- ({-0.5242*\dx},{0.0076*\dy})
	-- ({-0.5244*\dx},{0.0152*\dy})
	-- ({-0.5246*\dx},{0.0229*\dy})
	-- ({-0.5249*\dx},{0.0305*\dy})
	-- ({-0.5253*\dx},{0.0381*\dy})
	-- ({-0.5257*\dx},{0.0457*\dy})
	-- ({-0.5263*\dx},{0.0534*\dy})
	-- ({-0.5269*\dx},{0.0610*\dy})
	-- ({-0.5276*\dx},{0.0686*\dy})
	-- ({-0.5284*\dx},{0.0762*\dy})
	-- ({-0.5293*\dx},{0.0839*\dy})
	-- ({-0.5303*\dx},{0.0915*\dy})
	-- ({-0.5313*\dx},{0.0992*\dy})
	-- ({-0.5325*\dx},{0.1068*\dy})
	-- ({-0.5337*\dx},{0.1145*\dy})
	-- ({-0.5351*\dx},{0.1221*\dy})
	-- ({-0.5365*\dx},{0.1298*\dy})
	-- ({-0.5380*\dx},{0.1374*\dy})
	-- ({-0.5396*\dx},{0.1451*\dy})
	-- ({-0.5413*\dx},{0.1528*\dy})
	-- ({-0.5431*\dx},{0.1604*\dy})
	-- ({-0.5449*\dx},{0.1681*\dy})
	-- ({-0.5469*\dx},{0.1758*\dy})
	-- ({-0.5490*\dx},{0.1835*\dy})
	-- ({-0.5512*\dx},{0.1912*\dy})
	-- ({-0.5534*\dx},{0.1989*\dy})
	-- ({-0.5558*\dx},{0.2066*\dy})
	-- ({-0.5582*\dx},{0.2143*\dy})
	-- ({-0.5608*\dx},{0.2220*\dy})
	-- ({-0.5635*\dx},{0.2298*\dy})
	-- ({-0.5663*\dx},{0.2375*\dy})
	-- ({-0.5691*\dx},{0.2452*\dy})
	-- ({-0.5721*\dx},{0.2530*\dy})
	-- ({-0.5752*\dx},{0.2607*\dy})
	-- ({-0.5784*\dx},{0.2685*\dy})
	-- ({-0.5817*\dx},{0.2762*\dy})
	-- ({-0.5852*\dx},{0.2840*\dy})
	-- ({-0.5887*\dx},{0.2917*\dy})
	-- ({-0.5924*\dx},{0.2995*\dy})
	-- ({-0.5962*\dx},{0.3072*\dy})
	-- ({-0.6001*\dx},{0.3150*\dy})
	-- ({-0.6041*\dx},{0.3228*\dy})
	-- ({-0.6083*\dx},{0.3305*\dy})
	-- ({-0.6126*\dx},{0.3383*\dy})
	-- ({-0.6170*\dx},{0.3461*\dy})
	-- ({-0.6216*\dx},{0.3538*\dy})
	-- ({-0.6262*\dx},{0.3616*\dy})
	-- ({-0.6311*\dx},{0.3694*\dy})
	-- ({-0.6360*\dx},{0.3771*\dy})
	-- ({-0.6412*\dx},{0.3849*\dy})
	-- ({-0.6464*\dx},{0.3926*\dy})
	-- ({-0.6518*\dx},{0.4003*\dy})
	-- ({-0.6574*\dx},{0.4081*\dy})
	-- ({-0.6631*\dx},{0.4158*\dy})
	-- ({-0.6690*\dx},{0.4235*\dy})
	-- ({-0.6750*\dx},{0.4311*\dy})
	-- ({-0.6812*\dx},{0.4388*\dy})
	-- ({-0.6876*\dx},{0.4464*\dy})
	-- ({-0.6941*\dx},{0.4541*\dy})
	-- ({-0.7008*\dx},{0.4617*\dy})
	-- ({-0.7077*\dx},{0.4692*\dy})
	-- ({-0.7148*\dx},{0.4768*\dy})
	-- ({-0.7220*\dx},{0.4843*\dy})
	-- ({-0.7295*\dx},{0.4918*\dy})
	-- ({-0.7371*\dx},{0.4992*\dy})
	-- ({-0.7449*\dx},{0.5066*\dy})
	-- ({-0.7530*\dx},{0.5140*\dy})
	-- ({-0.7612*\dx},{0.5213*\dy})
	-- ({-0.7696*\dx},{0.5285*\dy})
	-- ({-0.7783*\dx},{0.5357*\dy})
	-- ({-0.7872*\dx},{0.5428*\dy})
	-- ({-0.7963*\dx},{0.5499*\dy})
	-- ({-0.8056*\dx},{0.5569*\dy})
	-- ({-0.8151*\dx},{0.5638*\dy})
	-- ({-0.8249*\dx},{0.5706*\dy})
	-- ({-0.8349*\dx},{0.5773*\dy})
	-- ({-0.8451*\dx},{0.5840*\dy})
	-- ({-0.8556*\dx},{0.5905*\dy})
	-- ({-0.8664*\dx},{0.5969*\dy})
	-- ({-0.8774*\dx},{0.6032*\dy})
	-- ({-0.8886*\dx},{0.6094*\dy})
	-- ({-0.9001*\dx},{0.6154*\dy})
	-- ({-0.9119*\dx},{0.6213*\dy})
	-- ({-0.9239*\dx},{0.6271*\dy})
	-- ({-0.9363*\dx},{0.6327*\dy})
	-- ({-0.9488*\dx},{0.6381*\dy})
	-- ({-0.9617*\dx},{0.6433*\dy})
	-- ({-0.9748*\dx},{0.6484*\dy})
	-- ({-0.9883*\dx},{0.6532*\dy})
	-- ({-1.0020*\dx},{0.6578*\dy})
	-- ({-1.0160*\dx},{0.6622*\dy})
	-- ({-1.0302*\dx},{0.6663*\dy})
	-- ({-1.0448*\dx},{0.6702*\dy})
	-- ({-1.0597*\dx},{0.6738*\dy})
	-- ({-1.0748*\dx},{0.6772*\dy})
	-- ({-1.0903*\dx},{0.6802*\dy})
	-- ({-1.1060*\dx},{0.6829*\dy})
	-- ({-1.1220*\dx},{0.6853*\dy})
	-- ({-1.1383*\dx},{0.6874*\dy})
	-- ({-1.1549*\dx},{0.6890*\dy})
	-- ({-1.1718*\dx},{0.6903*\dy})
	-- ({-1.1889*\dx},{0.6912*\dy})
	-- ({-1.2063*\dx},{0.6917*\dy})
	-- ({-1.2240*\dx},{0.6917*\dy})
	-- ({-1.2419*\dx},{0.6912*\dy})
	-- ({-1.2600*\dx},{0.6903*\dy})
	-- ({-1.2784*\dx},{0.6889*\dy})
	-- ({-1.2970*\dx},{0.6870*\dy})
	-- ({-1.3159*\dx},{0.6845*\dy})
	-- ({-1.3349*\dx},{0.6814*\dy})
	-- ({-1.3541*\dx},{0.6778*\dy})
	-- ({-1.3735*\dx},{0.6736*\dy})
	-- ({-1.3930*\dx},{0.6687*\dy})
	-- ({-1.4126*\dx},{0.6632*\dy})
	-- ({-1.4323*\dx},{0.6570*\dy})
	-- ({-1.4521*\dx},{0.6501*\dy})
	-- ({-1.4720*\dx},{0.6426*\dy})
	-- ({-1.4919*\dx},{0.6343*\dy})
	-- ({-1.5118*\dx},{0.6253*\dy})
	-- ({-1.5316*\dx},{0.6155*\dy})
	-- ({-1.5514*\dx},{0.6050*\dy})
	-- ({-1.5710*\dx},{0.5936*\dy})
	-- ({-1.5905*\dx},{0.5815*\dy})
	-- ({-1.6099*\dx},{0.5686*\dy})
	-- ({-1.6290*\dx},{0.5548*\dy})
	-- ({-1.6479*\dx},{0.5403*\dy})
	-- ({-1.6664*\dx},{0.5249*\dy})
	-- ({-1.6846*\dx},{0.5087*\dy})
	-- ({-1.7024*\dx},{0.4918*\dy})
	-- ({-1.7198*\dx},{0.4739*\dy})
	-- ({-1.7367*\dx},{0.4553*\dy})
	-- ({-1.7530*\dx},{0.4359*\dy})
	-- ({-1.7688*\dx},{0.4158*\dy})
	-- ({-1.7839*\dx},{0.3949*\dy})
	-- ({-1.7983*\dx},{0.3732*\dy})
	-- ({-1.8121*\dx},{0.3508*\dy})
	-- ({-1.8250*\dx},{0.3278*\dy})
	-- ({-1.8372*\dx},{0.3041*\dy})
	-- ({-1.8485*\dx},{0.2799*\dy})
	-- ({-1.8589*\dx},{0.2550*\dy})
	-- ({-1.8684*\dx},{0.2296*\dy})
	-- ({-1.8770*\dx},{0.2038*\dy})
	-- ({-1.8845*\dx},{0.1775*\dy})
	-- ({-1.8910*\dx},{0.1508*\dy})
	-- ({-1.8965*\dx},{0.1239*\dy})
	-- ({-1.9009*\dx},{0.0966*\dy})
	-- ({-1.9042*\dx},{0.0692*\dy})
	-- ({-1.9064*\dx},{0.0416*\dy})
	-- ({-1.9075*\dx},{0.0139*\dy})
	-- ({-1.9075*\dx},{-0.0139*\dy})
	-- ({-1.9064*\dx},{-0.0416*\dy})
	-- ({-1.9042*\dx},{-0.0692*\dy})
	-- ({-1.9009*\dx},{-0.0966*\dy})
	-- ({-1.8965*\dx},{-0.1239*\dy})
	-- ({-1.8910*\dx},{-0.1508*\dy})
	-- ({-1.8845*\dx},{-0.1775*\dy})
	-- ({-1.8770*\dx},{-0.2038*\dy})
	-- ({-1.8684*\dx},{-0.2296*\dy})
	-- ({-1.8589*\dx},{-0.2550*\dy})
	-- ({-1.8485*\dx},{-0.2799*\dy})
	-- ({-1.8372*\dx},{-0.3041*\dy})
	-- ({-1.8250*\dx},{-0.3278*\dy})
	-- ({-1.8121*\dx},{-0.3508*\dy})
	-- ({-1.7983*\dx},{-0.3732*\dy})
	-- ({-1.7839*\dx},{-0.3949*\dy})
	-- ({-1.7688*\dx},{-0.4158*\dy})
	-- ({-1.7530*\dx},{-0.4359*\dy})
	-- ({-1.7367*\dx},{-0.4553*\dy})
	-- ({-1.7198*\dx},{-0.4739*\dy})
	-- ({-1.7024*\dx},{-0.4918*\dy})
	-- ({-1.6846*\dx},{-0.5087*\dy})
	-- ({-1.6664*\dx},{-0.5249*\dy})
	-- ({-1.6479*\dx},{-0.5403*\dy})
	-- ({-1.6290*\dx},{-0.5548*\dy})
	-- ({-1.6099*\dx},{-0.5686*\dy})
	-- ({-1.5905*\dx},{-0.5815*\dy})
	-- ({-1.5710*\dx},{-0.5936*\dy})
	-- ({-1.5514*\dx},{-0.6050*\dy})
	-- ({-1.5316*\dx},{-0.6155*\dy})
	-- ({-1.5118*\dx},{-0.6253*\dy})
	-- ({-1.4919*\dx},{-0.6343*\dy})
	-- ({-1.4720*\dx},{-0.6426*\dy})
	-- ({-1.4521*\dx},{-0.6501*\dy})
	-- ({-1.4323*\dx},{-0.6570*\dy})
	-- ({-1.4126*\dx},{-0.6632*\dy})
	-- ({-1.3930*\dx},{-0.6687*\dy})
	-- ({-1.3735*\dx},{-0.6736*\dy})
	-- ({-1.3541*\dx},{-0.6778*\dy})
	-- ({-1.3349*\dx},{-0.6814*\dy})
	-- ({-1.3159*\dx},{-0.6845*\dy})
	-- ({-1.2970*\dx},{-0.6870*\dy})
	-- ({-1.2784*\dx},{-0.6889*\dy})
	-- ({-1.2600*\dx},{-0.6903*\dy})
	-- ({-1.2419*\dx},{-0.6912*\dy})
	-- ({-1.2240*\dx},{-0.6917*\dy})
	-- ({-1.2063*\dx},{-0.6917*\dy})
	-- ({-1.1889*\dx},{-0.6912*\dy})
	-- ({-1.1718*\dx},{-0.6903*\dy})
	-- ({-1.1549*\dx},{-0.6890*\dy})
	-- ({-1.1383*\dx},{-0.6874*\dy})
	-- ({-1.1220*\dx},{-0.6853*\dy})
	-- ({-1.1060*\dx},{-0.6829*\dy})
	-- ({-1.0903*\dx},{-0.6802*\dy})
	-- ({-1.0748*\dx},{-0.6772*\dy})
	-- ({-1.0597*\dx},{-0.6738*\dy})
	-- ({-1.0448*\dx},{-0.6702*\dy})
	-- ({-1.0302*\dx},{-0.6663*\dy})
	-- ({-1.0160*\dx},{-0.6622*\dy})
	-- ({-1.0020*\dx},{-0.6578*\dy})
	-- ({-0.9883*\dx},{-0.6532*\dy})
	-- ({-0.9748*\dx},{-0.6484*\dy})
	-- ({-0.9617*\dx},{-0.6433*\dy})
	-- ({-0.9488*\dx},{-0.6381*\dy})
	-- ({-0.9363*\dx},{-0.6327*\dy})
	-- ({-0.9239*\dx},{-0.6271*\dy})
	-- ({-0.9119*\dx},{-0.6213*\dy})
	-- ({-0.9001*\dx},{-0.6154*\dy})
	-- ({-0.8886*\dx},{-0.6094*\dy})
	-- ({-0.8774*\dx},{-0.6032*\dy})
	-- ({-0.8664*\dx},{-0.5969*\dy})
	-- ({-0.8556*\dx},{-0.5905*\dy})
	-- ({-0.8451*\dx},{-0.5840*\dy})
	-- ({-0.8349*\dx},{-0.5773*\dy})
	-- ({-0.8249*\dx},{-0.5706*\dy})
	-- ({-0.8151*\dx},{-0.5638*\dy})
	-- ({-0.8056*\dx},{-0.5569*\dy})
	-- ({-0.7963*\dx},{-0.5499*\dy})
	-- ({-0.7872*\dx},{-0.5428*\dy})
	-- ({-0.7783*\dx},{-0.5357*\dy})
	-- ({-0.7696*\dx},{-0.5285*\dy})
	-- ({-0.7612*\dx},{-0.5213*\dy})
	-- ({-0.7530*\dx},{-0.5140*\dy})
	-- ({-0.7449*\dx},{-0.5066*\dy})
	-- ({-0.7371*\dx},{-0.4992*\dy})
	-- ({-0.7295*\dx},{-0.4918*\dy})
	-- ({-0.7220*\dx},{-0.4843*\dy})
	-- ({-0.7148*\dx},{-0.4768*\dy})
	-- ({-0.7077*\dx},{-0.4692*\dy})
	-- ({-0.7008*\dx},{-0.4617*\dy})
	-- ({-0.6941*\dx},{-0.4541*\dy})
	-- ({-0.6876*\dx},{-0.4464*\dy})
	-- ({-0.6812*\dx},{-0.4388*\dy})
	-- ({-0.6750*\dx},{-0.4311*\dy})
	-- ({-0.6690*\dx},{-0.4235*\dy})
	-- ({-0.6631*\dx},{-0.4158*\dy})
	-- ({-0.6574*\dx},{-0.4081*\dy})
	-- ({-0.6518*\dx},{-0.4003*\dy})
	-- ({-0.6464*\dx},{-0.3926*\dy})
	-- ({-0.6412*\dx},{-0.3849*\dy})
	-- ({-0.6360*\dx},{-0.3771*\dy})
	-- ({-0.6311*\dx},{-0.3694*\dy})
	-- ({-0.6262*\dx},{-0.3616*\dy})
	-- ({-0.6216*\dx},{-0.3538*\dy})
	-- ({-0.6170*\dx},{-0.3461*\dy})
	-- ({-0.6126*\dx},{-0.3383*\dy})
	-- ({-0.6083*\dx},{-0.3305*\dy})
	-- ({-0.6041*\dx},{-0.3228*\dy})
	-- ({-0.6001*\dx},{-0.3150*\dy})
	-- ({-0.5962*\dx},{-0.3072*\dy})
	-- ({-0.5924*\dx},{-0.2995*\dy})
	-- ({-0.5887*\dx},{-0.2917*\dy})
	-- ({-0.5852*\dx},{-0.2840*\dy})
	-- ({-0.5817*\dx},{-0.2762*\dy})
	-- ({-0.5784*\dx},{-0.2685*\dy})
	-- ({-0.5752*\dx},{-0.2607*\dy})
	-- ({-0.5721*\dx},{-0.2530*\dy})
	-- ({-0.5691*\dx},{-0.2452*\dy})
	-- ({-0.5663*\dx},{-0.2375*\dy})
	-- ({-0.5635*\dx},{-0.2298*\dy})
	-- ({-0.5608*\dx},{-0.2220*\dy})
	-- ({-0.5582*\dx},{-0.2143*\dy})
	-- ({-0.5558*\dx},{-0.2066*\dy})
	-- ({-0.5534*\dx},{-0.1989*\dy})
	-- ({-0.5512*\dx},{-0.1912*\dy})
	-- ({-0.5490*\dx},{-0.1835*\dy})
	-- ({-0.5469*\dx},{-0.1758*\dy})
	-- ({-0.5449*\dx},{-0.1681*\dy})
	-- ({-0.5431*\dx},{-0.1604*\dy})
	-- ({-0.5413*\dx},{-0.1528*\dy})
	-- ({-0.5396*\dx},{-0.1451*\dy})
	-- ({-0.5380*\dx},{-0.1374*\dy})
	-- ({-0.5365*\dx},{-0.1298*\dy})
	-- ({-0.5351*\dx},{-0.1221*\dy})
	-- ({-0.5337*\dx},{-0.1145*\dy})
	-- ({-0.5325*\dx},{-0.1068*\dy})
	-- ({-0.5313*\dx},{-0.0992*\dy})
	-- ({-0.5303*\dx},{-0.0915*\dy})
	-- ({-0.5293*\dx},{-0.0839*\dy})
	-- ({-0.5284*\dx},{-0.0762*\dy})
	-- ({-0.5276*\dx},{-0.0686*\dy})
	-- ({-0.5269*\dx},{-0.0610*\dy})
	-- ({-0.5263*\dx},{-0.0534*\dy})
	-- ({-0.5257*\dx},{-0.0457*\dy})
	-- ({-0.5253*\dx},{-0.0381*\dy})
	-- ({-0.5249*\dx},{-0.0305*\dy})
	-- ({-0.5246*\dx},{-0.0229*\dy})
	-- ({-0.5244*\dx},{-0.0152*\dy})
	-- ({-0.5242*\dx},{-0.0076*\dy})
	-- ({-0.5242*\dx},{-0.0000*\dy})
}
% u = -0.415799
\def\upathG{
	({-0.3934*\dx},{0.0000*\dy})
	-- ({-0.3934*\dx},{0.0089*\dy})
	-- ({-0.3935*\dx},{0.0178*\dy})
	-- ({-0.3937*\dx},{0.0266*\dy})
	-- ({-0.3940*\dx},{0.0355*\dy})
	-- ({-0.3943*\dx},{0.0444*\dy})
	-- ({-0.3947*\dx},{0.0533*\dy})
	-- ({-0.3952*\dx},{0.0622*\dy})
	-- ({-0.3957*\dx},{0.0711*\dy})
	-- ({-0.3964*\dx},{0.0801*\dy})
	-- ({-0.3971*\dx},{0.0890*\dy})
	-- ({-0.3979*\dx},{0.0979*\dy})
	-- ({-0.3987*\dx},{0.1069*\dy})
	-- ({-0.3996*\dx},{0.1158*\dy})
	-- ({-0.4007*\dx},{0.1248*\dy})
	-- ({-0.4018*\dx},{0.1338*\dy})
	-- ({-0.4029*\dx},{0.1428*\dy})
	-- ({-0.4042*\dx},{0.1518*\dy})
	-- ({-0.4055*\dx},{0.1609*\dy})
	-- ({-0.4069*\dx},{0.1699*\dy})
	-- ({-0.4084*\dx},{0.1790*\dy})
	-- ({-0.4100*\dx},{0.1881*\dy})
	-- ({-0.4116*\dx},{0.1972*\dy})
	-- ({-0.4134*\dx},{0.2064*\dy})
	-- ({-0.4152*\dx},{0.2156*\dy})
	-- ({-0.4172*\dx},{0.2248*\dy})
	-- ({-0.4192*\dx},{0.2340*\dy})
	-- ({-0.4213*\dx},{0.2432*\dy})
	-- ({-0.4235*\dx},{0.2525*\dy})
	-- ({-0.4258*\dx},{0.2618*\dy})
	-- ({-0.4282*\dx},{0.2712*\dy})
	-- ({-0.4307*\dx},{0.2805*\dy})
	-- ({-0.4332*\dx},{0.2899*\dy})
	-- ({-0.4359*\dx},{0.2994*\dy})
	-- ({-0.4387*\dx},{0.3088*\dy})
	-- ({-0.4416*\dx},{0.3183*\dy})
	-- ({-0.4446*\dx},{0.3279*\dy})
	-- ({-0.4478*\dx},{0.3375*\dy})
	-- ({-0.4510*\dx},{0.3471*\dy})
	-- ({-0.4543*\dx},{0.3567*\dy})
	-- ({-0.4578*\dx},{0.3664*\dy})
	-- ({-0.4614*\dx},{0.3762*\dy})
	-- ({-0.4651*\dx},{0.3859*\dy})
	-- ({-0.4689*\dx},{0.3958*\dy})
	-- ({-0.4729*\dx},{0.4056*\dy})
	-- ({-0.4770*\dx},{0.4155*\dy})
	-- ({-0.4812*\dx},{0.4255*\dy})
	-- ({-0.4856*\dx},{0.4355*\dy})
	-- ({-0.4901*\dx},{0.4455*\dy})
	-- ({-0.4948*\dx},{0.4556*\dy})
	-- ({-0.4996*\dx},{0.4658*\dy})
	-- ({-0.5046*\dx},{0.4759*\dy})
	-- ({-0.5097*\dx},{0.4862*\dy})
	-- ({-0.5150*\dx},{0.4965*\dy})
	-- ({-0.5204*\dx},{0.5068*\dy})
	-- ({-0.5261*\dx},{0.5172*\dy})
	-- ({-0.5319*\dx},{0.5276*\dy})
	-- ({-0.5379*\dx},{0.5381*\dy})
	-- ({-0.5440*\dx},{0.5486*\dy})
	-- ({-0.5504*\dx},{0.5592*\dy})
	-- ({-0.5570*\dx},{0.5699*\dy})
	-- ({-0.5638*\dx},{0.5805*\dy})
	-- ({-0.5707*\dx},{0.5913*\dy})
	-- ({-0.5779*\dx},{0.6021*\dy})
	-- ({-0.5854*\dx},{0.6129*\dy})
	-- ({-0.5930*\dx},{0.6238*\dy})
	-- ({-0.6009*\dx},{0.6347*\dy})
	-- ({-0.6090*\dx},{0.6457*\dy})
	-- ({-0.6174*\dx},{0.6567*\dy})
	-- ({-0.6261*\dx},{0.6677*\dy})
	-- ({-0.6350*\dx},{0.6788*\dy})
	-- ({-0.6442*\dx},{0.6899*\dy})
	-- ({-0.6537*\dx},{0.7011*\dy})
	-- ({-0.6635*\dx},{0.7123*\dy})
	-- ({-0.6736*\dx},{0.7235*\dy})
	-- ({-0.6840*\dx},{0.7348*\dy})
	-- ({-0.6947*\dx},{0.7461*\dy})
	-- ({-0.7057*\dx},{0.7574*\dy})
	-- ({-0.7172*\dx},{0.7687*\dy})
	-- ({-0.7289*\dx},{0.7800*\dy})
	-- ({-0.7411*\dx},{0.7913*\dy})
	-- ({-0.7536*\dx},{0.8026*\dy})
	-- ({-0.7665*\dx},{0.8139*\dy})
	-- ({-0.7798*\dx},{0.8252*\dy})
	-- ({-0.7935*\dx},{0.8365*\dy})
	-- ({-0.8077*\dx},{0.8477*\dy})
	-- ({-0.8223*\dx},{0.8589*\dy})
	-- ({-0.8374*\dx},{0.8700*\dy})
	-- ({-0.8529*\dx},{0.8810*\dy})
	-- ({-0.8690*\dx},{0.8920*\dy})
	-- ({-0.8855*\dx},{0.9029*\dy})
	-- ({-0.9026*\dx},{0.9137*\dy})
	-- ({-0.9202*\dx},{0.9243*\dy})
	-- ({-0.9383*\dx},{0.9348*\dy})
	-- ({-0.9570*\dx},{0.9452*\dy})
	-- ({-0.9764*\dx},{0.9554*\dy})
	-- ({-0.9963*\dx},{0.9654*\dy})
	-- ({-1.0168*\dx},{0.9751*\dy})
	-- ({-1.0379*\dx},{0.9846*\dy})
	-- ({-1.0597*\dx},{0.9938*\dy})
	-- ({-1.0822*\dx},{1.0028*\dy})
	-- ({-1.1053*\dx},{1.0114*\dy})
	-- ({-1.1292*\dx},{1.0196*\dy})
	-- ({-1.1537*\dx},{1.0274*\dy})
	-- ({-1.1790*\dx},{1.0348*\dy})
	-- ({-1.2050*\dx},{1.0417*\dy})
	-- ({-1.2318*\dx},{1.0481*\dy})
	-- ({-1.2593*\dx},{1.0539*\dy})
	-- ({-1.2876*\dx},{1.0591*\dy})
	-- ({-1.3166*\dx},{1.0636*\dy})
	-- ({-1.3464*\dx},{1.0675*\dy})
	-- ({-1.3770*\dx},{1.0705*\dy})
	-- ({-1.4084*\dx},{1.0727*\dy})
	-- ({-1.4405*\dx},{1.0740*\dy})
	-- ({-1.4734*\dx},{1.0743*\dy})
	-- ({-1.5070*\dx},{1.0736*\dy})
	-- ({-1.5414*\dx},{1.0718*\dy})
	-- ({-1.5764*\dx},{1.0688*\dy})
	-- ({-1.6122*\dx},{1.0646*\dy})
	-- ({-1.6485*\dx},{1.0590*\dy})
	-- ({-1.6855*\dx},{1.0520*\dy})
	-- ({-1.7231*\dx},{1.0435*\dy})
	-- ({-1.7611*\dx},{1.0335*\dy})
	-- ({-1.7995*\dx},{1.0218*\dy})
	-- ({-1.8383*\dx},{1.0084*\dy})
	-- ({-1.8774*\dx},{0.9931*\dy})
	-- ({-1.9167*\dx},{0.9760*\dy})
	-- ({-1.9560*\dx},{0.9570*\dy})
	-- ({-1.9953*\dx},{0.9359*\dy})
	-- ({-2.0344*\dx},{0.9127*\dy})
	-- ({-2.0733*\dx},{0.8874*\dy})
	-- ({-2.1117*\dx},{0.8599*\dy})
	-- ({-2.1495*\dx},{0.8302*\dy})
	-- ({-2.1867*\dx},{0.7983*\dy})
	-- ({-2.2229*\dx},{0.7642*\dy})
	-- ({-2.2580*\dx},{0.7278*\dy})
	-- ({-2.2919*\dx},{0.6892*\dy})
	-- ({-2.3243*\dx},{0.6484*\dy})
	-- ({-2.3551*\dx},{0.6055*\dy})
	-- ({-2.3842*\dx},{0.5606*\dy})
	-- ({-2.4113*\dx},{0.5137*\dy})
	-- ({-2.4362*\dx},{0.4650*\dy})
	-- ({-2.4588*\dx},{0.4146*\dy})
	-- ({-2.4790*\dx},{0.3626*\dy})
	-- ({-2.4965*\dx},{0.3093*\dy})
	-- ({-2.5114*\dx},{0.2548*\dy})
	-- ({-2.5234*\dx},{0.1992*\dy})
	-- ({-2.5325*\dx},{0.1429*\dy})
	-- ({-2.5386*\dx},{0.0860*\dy})
	-- ({-2.5417*\dx},{0.0287*\dy})
	-- ({-2.5417*\dx},{-0.0287*\dy})
	-- ({-2.5386*\dx},{-0.0860*\dy})
	-- ({-2.5325*\dx},{-0.1429*\dy})
	-- ({-2.5234*\dx},{-0.1992*\dy})
	-- ({-2.5114*\dx},{-0.2548*\dy})
	-- ({-2.4965*\dx},{-0.3093*\dy})
	-- ({-2.4790*\dx},{-0.3626*\dy})
	-- ({-2.4588*\dx},{-0.4146*\dy})
	-- ({-2.4362*\dx},{-0.4650*\dy})
	-- ({-2.4113*\dx},{-0.5137*\dy})
	-- ({-2.3842*\dx},{-0.5606*\dy})
	-- ({-2.3551*\dx},{-0.6055*\dy})
	-- ({-2.3243*\dx},{-0.6484*\dy})
	-- ({-2.2919*\dx},{-0.6892*\dy})
	-- ({-2.2580*\dx},{-0.7278*\dy})
	-- ({-2.2229*\dx},{-0.7642*\dy})
	-- ({-2.1867*\dx},{-0.7983*\dy})
	-- ({-2.1495*\dx},{-0.8302*\dy})
	-- ({-2.1117*\dx},{-0.8599*\dy})
	-- ({-2.0733*\dx},{-0.8874*\dy})
	-- ({-2.0344*\dx},{-0.9127*\dy})
	-- ({-1.9953*\dx},{-0.9359*\dy})
	-- ({-1.9560*\dx},{-0.9570*\dy})
	-- ({-1.9167*\dx},{-0.9760*\dy})
	-- ({-1.8774*\dx},{-0.9931*\dy})
	-- ({-1.8383*\dx},{-1.0084*\dy})
	-- ({-1.7995*\dx},{-1.0218*\dy})
	-- ({-1.7611*\dx},{-1.0335*\dy})
	-- ({-1.7231*\dx},{-1.0435*\dy})
	-- ({-1.6855*\dx},{-1.0520*\dy})
	-- ({-1.6485*\dx},{-1.0590*\dy})
	-- ({-1.6122*\dx},{-1.0646*\dy})
	-- ({-1.5764*\dx},{-1.0688*\dy})
	-- ({-1.5414*\dx},{-1.0718*\dy})
	-- ({-1.5070*\dx},{-1.0736*\dy})
	-- ({-1.4734*\dx},{-1.0743*\dy})
	-- ({-1.4405*\dx},{-1.0740*\dy})
	-- ({-1.4084*\dx},{-1.0727*\dy})
	-- ({-1.3770*\dx},{-1.0705*\dy})
	-- ({-1.3464*\dx},{-1.0675*\dy})
	-- ({-1.3166*\dx},{-1.0636*\dy})
	-- ({-1.2876*\dx},{-1.0591*\dy})
	-- ({-1.2593*\dx},{-1.0539*\dy})
	-- ({-1.2318*\dx},{-1.0481*\dy})
	-- ({-1.2050*\dx},{-1.0417*\dy})
	-- ({-1.1790*\dx},{-1.0348*\dy})
	-- ({-1.1537*\dx},{-1.0274*\dy})
	-- ({-1.1292*\dx},{-1.0196*\dy})
	-- ({-1.1053*\dx},{-1.0114*\dy})
	-- ({-1.0822*\dx},{-1.0028*\dy})
	-- ({-1.0597*\dx},{-0.9938*\dy})
	-- ({-1.0379*\dx},{-0.9846*\dy})
	-- ({-1.0168*\dx},{-0.9751*\dy})
	-- ({-0.9963*\dx},{-0.9654*\dy})
	-- ({-0.9764*\dx},{-0.9554*\dy})
	-- ({-0.9570*\dx},{-0.9452*\dy})
	-- ({-0.9383*\dx},{-0.9348*\dy})
	-- ({-0.9202*\dx},{-0.9243*\dy})
	-- ({-0.9026*\dx},{-0.9137*\dy})
	-- ({-0.8855*\dx},{-0.9029*\dy})
	-- ({-0.8690*\dx},{-0.8920*\dy})
	-- ({-0.8529*\dx},{-0.8810*\dy})
	-- ({-0.8374*\dx},{-0.8700*\dy})
	-- ({-0.8223*\dx},{-0.8589*\dy})
	-- ({-0.8077*\dx},{-0.8477*\dy})
	-- ({-0.7935*\dx},{-0.8365*\dy})
	-- ({-0.7798*\dx},{-0.8252*\dy})
	-- ({-0.7665*\dx},{-0.8139*\dy})
	-- ({-0.7536*\dx},{-0.8026*\dy})
	-- ({-0.7411*\dx},{-0.7913*\dy})
	-- ({-0.7289*\dx},{-0.7800*\dy})
	-- ({-0.7172*\dx},{-0.7687*\dy})
	-- ({-0.7057*\dx},{-0.7574*\dy})
	-- ({-0.6947*\dx},{-0.7461*\dy})
	-- ({-0.6840*\dx},{-0.7348*\dy})
	-- ({-0.6736*\dx},{-0.7235*\dy})
	-- ({-0.6635*\dx},{-0.7123*\dy})
	-- ({-0.6537*\dx},{-0.7011*\dy})
	-- ({-0.6442*\dx},{-0.6899*\dy})
	-- ({-0.6350*\dx},{-0.6788*\dy})
	-- ({-0.6261*\dx},{-0.6677*\dy})
	-- ({-0.6174*\dx},{-0.6567*\dy})
	-- ({-0.6090*\dx},{-0.6457*\dy})
	-- ({-0.6009*\dx},{-0.6347*\dy})
	-- ({-0.5930*\dx},{-0.6238*\dy})
	-- ({-0.5854*\dx},{-0.6129*\dy})
	-- ({-0.5779*\dx},{-0.6021*\dy})
	-- ({-0.5707*\dx},{-0.5913*\dy})
	-- ({-0.5638*\dx},{-0.5805*\dy})
	-- ({-0.5570*\dx},{-0.5699*\dy})
	-- ({-0.5504*\dx},{-0.5592*\dy})
	-- ({-0.5440*\dx},{-0.5486*\dy})
	-- ({-0.5379*\dx},{-0.5381*\dy})
	-- ({-0.5319*\dx},{-0.5276*\dy})
	-- ({-0.5261*\dx},{-0.5172*\dy})
	-- ({-0.5204*\dx},{-0.5068*\dy})
	-- ({-0.5150*\dx},{-0.4965*\dy})
	-- ({-0.5097*\dx},{-0.4862*\dy})
	-- ({-0.5046*\dx},{-0.4759*\dy})
	-- ({-0.4996*\dx},{-0.4658*\dy})
	-- ({-0.4948*\dx},{-0.4556*\dy})
	-- ({-0.4901*\dx},{-0.4455*\dy})
	-- ({-0.4856*\dx},{-0.4355*\dy})
	-- ({-0.4812*\dx},{-0.4255*\dy})
	-- ({-0.4770*\dx},{-0.4155*\dy})
	-- ({-0.4729*\dx},{-0.4056*\dy})
	-- ({-0.4689*\dx},{-0.3958*\dy})
	-- ({-0.4651*\dx},{-0.3859*\dy})
	-- ({-0.4614*\dx},{-0.3762*\dy})
	-- ({-0.4578*\dx},{-0.3664*\dy})
	-- ({-0.4543*\dx},{-0.3567*\dy})
	-- ({-0.4510*\dx},{-0.3471*\dy})
	-- ({-0.4478*\dx},{-0.3375*\dy})
	-- ({-0.4446*\dx},{-0.3279*\dy})
	-- ({-0.4416*\dx},{-0.3183*\dy})
	-- ({-0.4387*\dx},{-0.3088*\dy})
	-- ({-0.4359*\dx},{-0.2994*\dy})
	-- ({-0.4332*\dx},{-0.2899*\dy})
	-- ({-0.4307*\dx},{-0.2805*\dy})
	-- ({-0.4282*\dx},{-0.2712*\dy})
	-- ({-0.4258*\dx},{-0.2618*\dy})
	-- ({-0.4235*\dx},{-0.2525*\dy})
	-- ({-0.4213*\dx},{-0.2432*\dy})
	-- ({-0.4192*\dx},{-0.2340*\dy})
	-- ({-0.4172*\dx},{-0.2248*\dy})
	-- ({-0.4152*\dx},{-0.2156*\dy})
	-- ({-0.4134*\dx},{-0.2064*\dy})
	-- ({-0.4116*\dx},{-0.1972*\dy})
	-- ({-0.4100*\dx},{-0.1881*\dy})
	-- ({-0.4084*\dx},{-0.1790*\dy})
	-- ({-0.4069*\dx},{-0.1699*\dy})
	-- ({-0.4055*\dx},{-0.1609*\dy})
	-- ({-0.4042*\dx},{-0.1518*\dy})
	-- ({-0.4029*\dx},{-0.1428*\dy})
	-- ({-0.4018*\dx},{-0.1338*\dy})
	-- ({-0.4007*\dx},{-0.1248*\dy})
	-- ({-0.3996*\dx},{-0.1158*\dy})
	-- ({-0.3987*\dx},{-0.1069*\dy})
	-- ({-0.3979*\dx},{-0.0979*\dy})
	-- ({-0.3971*\dx},{-0.0890*\dy})
	-- ({-0.3964*\dx},{-0.0801*\dy})
	-- ({-0.3957*\dx},{-0.0711*\dy})
	-- ({-0.3952*\dx},{-0.0622*\dy})
	-- ({-0.3947*\dx},{-0.0533*\dy})
	-- ({-0.3943*\dx},{-0.0444*\dy})
	-- ({-0.3940*\dx},{-0.0355*\dy})
	-- ({-0.3937*\dx},{-0.0266*\dy})
	-- ({-0.3935*\dx},{-0.0178*\dy})
	-- ({-0.3934*\dx},{-0.0089*\dy})
	-- ({-0.3934*\dx},{-0.0000*\dy})
}
% u = -0.249479
\def\upathH{
	({-0.2444*\dx},{0.0000*\dy})
	-- ({-0.2445*\dx},{0.0099*\dy})
	-- ({-0.2445*\dx},{0.0198*\dy})
	-- ({-0.2447*\dx},{0.0296*\dy})
	-- ({-0.2448*\dx},{0.0395*\dy})
	-- ({-0.2451*\dx},{0.0494*\dy})
	-- ({-0.2453*\dx},{0.0593*\dy})
	-- ({-0.2457*\dx},{0.0693*\dy})
	-- ({-0.2461*\dx},{0.0792*\dy})
	-- ({-0.2465*\dx},{0.0891*\dy})
	-- ({-0.2470*\dx},{0.0991*\dy})
	-- ({-0.2475*\dx},{0.1091*\dy})
	-- ({-0.2481*\dx},{0.1191*\dy})
	-- ({-0.2488*\dx},{0.1291*\dy})
	-- ({-0.2495*\dx},{0.1391*\dy})
	-- ({-0.2502*\dx},{0.1492*\dy})
	-- ({-0.2510*\dx},{0.1593*\dy})
	-- ({-0.2519*\dx},{0.1694*\dy})
	-- ({-0.2528*\dx},{0.1796*\dy})
	-- ({-0.2538*\dx},{0.1898*\dy})
	-- ({-0.2549*\dx},{0.2000*\dy})
	-- ({-0.2560*\dx},{0.2103*\dy})
	-- ({-0.2571*\dx},{0.2206*\dy})
	-- ({-0.2583*\dx},{0.2309*\dy})
	-- ({-0.2596*\dx},{0.2413*\dy})
	-- ({-0.2610*\dx},{0.2517*\dy})
	-- ({-0.2624*\dx},{0.2622*\dy})
	-- ({-0.2639*\dx},{0.2727*\dy})
	-- ({-0.2654*\dx},{0.2833*\dy})
	-- ({-0.2670*\dx},{0.2940*\dy})
	-- ({-0.2687*\dx},{0.3047*\dy})
	-- ({-0.2705*\dx},{0.3154*\dy})
	-- ({-0.2723*\dx},{0.3262*\dy})
	-- ({-0.2742*\dx},{0.3371*\dy})
	-- ({-0.2762*\dx},{0.3481*\dy})
	-- ({-0.2782*\dx},{0.3591*\dy})
	-- ({-0.2804*\dx},{0.3702*\dy})
	-- ({-0.2826*\dx},{0.3813*\dy})
	-- ({-0.2849*\dx},{0.3925*\dy})
	-- ({-0.2873*\dx},{0.4039*\dy})
	-- ({-0.2898*\dx},{0.4153*\dy})
	-- ({-0.2924*\dx},{0.4267*\dy})
	-- ({-0.2950*\dx},{0.4383*\dy})
	-- ({-0.2978*\dx},{0.4500*\dy})
	-- ({-0.3007*\dx},{0.4617*\dy})
	-- ({-0.3036*\dx},{0.4735*\dy})
	-- ({-0.3067*\dx},{0.4855*\dy})
	-- ({-0.3099*\dx},{0.4975*\dy})
	-- ({-0.3132*\dx},{0.5097*\dy})
	-- ({-0.3166*\dx},{0.5220*\dy})
	-- ({-0.3201*\dx},{0.5343*\dy})
	-- ({-0.3238*\dx},{0.5468*\dy})
	-- ({-0.3276*\dx},{0.5594*\dy})
	-- ({-0.3315*\dx},{0.5722*\dy})
	-- ({-0.3356*\dx},{0.5850*\dy})
	-- ({-0.3398*\dx},{0.5980*\dy})
	-- ({-0.3441*\dx},{0.6111*\dy})
	-- ({-0.3486*\dx},{0.6244*\dy})
	-- ({-0.3532*\dx},{0.6378*\dy})
	-- ({-0.3581*\dx},{0.6513*\dy})
	-- ({-0.3630*\dx},{0.6650*\dy})
	-- ({-0.3682*\dx},{0.6788*\dy})
	-- ({-0.3736*\dx},{0.6928*\dy})
	-- ({-0.3791*\dx},{0.7070*\dy})
	-- ({-0.3848*\dx},{0.7213*\dy})
	-- ({-0.3908*\dx},{0.7359*\dy})
	-- ({-0.3969*\dx},{0.7505*\dy})
	-- ({-0.4033*\dx},{0.7654*\dy})
	-- ({-0.4099*\dx},{0.7804*\dy})
	-- ({-0.4167*\dx},{0.7957*\dy})
	-- ({-0.4238*\dx},{0.8111*\dy})
	-- ({-0.4312*\dx},{0.8267*\dy})
	-- ({-0.4388*\dx},{0.8426*\dy})
	-- ({-0.4467*\dx},{0.8586*\dy})
	-- ({-0.4549*\dx},{0.8749*\dy})
	-- ({-0.4635*\dx},{0.8914*\dy})
	-- ({-0.4723*\dx},{0.9081*\dy})
	-- ({-0.4815*\dx},{0.9250*\dy})
	-- ({-0.4910*\dx},{0.9422*\dy})
	-- ({-0.5009*\dx},{0.9597*\dy})
	-- ({-0.5112*\dx},{0.9773*\dy})
	-- ({-0.5219*\dx},{0.9952*\dy})
	-- ({-0.5331*\dx},{1.0134*\dy})
	-- ({-0.5447*\dx},{1.0319*\dy})
	-- ({-0.5567*\dx},{1.0506*\dy})
	-- ({-0.5692*\dx},{1.0696*\dy})
	-- ({-0.5823*\dx},{1.0888*\dy})
	-- ({-0.5959*\dx},{1.1084*\dy})
	-- ({-0.6101*\dx},{1.1282*\dy})
	-- ({-0.6248*\dx},{1.1483*\dy})
	-- ({-0.6402*\dx},{1.1687*\dy})
	-- ({-0.6563*\dx},{1.1894*\dy})
	-- ({-0.6731*\dx},{1.2104*\dy})
	-- ({-0.6905*\dx},{1.2317*\dy})
	-- ({-0.7088*\dx},{1.2533*\dy})
	-- ({-0.7279*\dx},{1.2752*\dy})
	-- ({-0.7478*\dx},{1.2973*\dy})
	-- ({-0.7687*\dx},{1.3198*\dy})
	-- ({-0.7905*\dx},{1.3425*\dy})
	-- ({-0.8133*\dx},{1.3655*\dy})
	-- ({-0.8372*\dx},{1.3888*\dy})
	-- ({-0.8622*\dx},{1.4124*\dy})
	-- ({-0.8884*\dx},{1.4361*\dy})
	-- ({-0.9158*\dx},{1.4601*\dy})
	-- ({-0.9446*\dx},{1.4843*\dy})
	-- ({-0.9748*\dx},{1.5087*\dy})
	-- ({-1.0065*\dx},{1.5332*\dy})
	-- ({-1.0397*\dx},{1.5578*\dy})
	-- ({-1.0746*\dx},{1.5825*\dy})
	-- ({-1.1112*\dx},{1.6071*\dy})
	-- ({-1.1496*\dx},{1.6318*\dy})
	-- ({-1.1901*\dx},{1.6563*\dy})
	-- ({-1.2325*\dx},{1.6807*\dy})
	-- ({-1.2771*\dx},{1.7047*\dy})
	-- ({-1.3241*\dx},{1.7284*\dy})
	-- ({-1.3734*\dx},{1.7516*\dy})
	-- ({-1.4252*\dx},{1.7742*\dy})
	-- ({-1.4797*\dx},{1.7961*\dy})
	-- ({-1.5369*\dx},{1.8170*\dy})
	-- ({-1.5971*\dx},{1.8367*\dy})
	-- ({-1.6603*\dx},{1.8552*\dy})
	-- ({-1.7266*\dx},{1.8721*\dy})
	-- ({-1.7962*\dx},{1.8871*\dy})
	-- ({-1.8691*\dx},{1.9000*\dy})
	-- ({-1.9454*\dx},{1.9105*\dy})
	-- ({-2.0253*\dx},{1.9181*\dy})
	-- ({-2.1087*\dx},{1.9225*\dy})
	-- ({-2.1957*\dx},{1.9232*\dy})
	-- ({-2.2862*\dx},{1.9197*\dy})
	-- ({-2.3801*\dx},{1.9116*\dy})
	-- ({-2.4773*\dx},{1.8983*\dy})
	-- ({-2.5776*\dx},{1.8792*\dy})
	-- ({-2.6808*\dx},{1.8537*\dy})
	-- ({-2.7864*\dx},{1.8212*\dy})
	-- ({-2.8939*\dx},{1.7811*\dy})
	-- ({-3.0028*\dx},{1.7327*\dy})
	-- ({-3.1123*\dx},{1.6755*\dy})
	-- ({-3.2216*\dx},{1.6090*\dy})
	-- ({-3.3298*\dx},{1.5327*\dy})
	-- ({-3.4357*\dx},{1.4463*\dy})
	-- ({-3.5382*\dx},{1.3496*\dy})
	-- ({-3.6360*\dx},{1.2425*\dy})
	-- ({-3.7276*\dx},{1.1253*\dy})
	-- ({-3.8118*\dx},{0.9983*\dy})
	-- ({-3.8871*\dx},{0.8622*\dy})
	-- ({-3.9522*\dx},{0.7178*\dy})
	-- ({-4.0059*\dx},{0.5662*\dy})
	-- ({-4.0472*\dx},{0.4088*\dy})
	-- ({-4.0752*\dx},{0.2470*\dy})
	-- ({-4.0894*\dx},{0.0826*\dy})
	-- ({-4.0894*\dx},{-0.0826*\dy})
	-- ({-4.0752*\dx},{-0.2470*\dy})
	-- ({-4.0472*\dx},{-0.4088*\dy})
	-- ({-4.0059*\dx},{-0.5662*\dy})
	-- ({-3.9522*\dx},{-0.7178*\dy})
	-- ({-3.8871*\dx},{-0.8622*\dy})
	-- ({-3.8118*\dx},{-0.9983*\dy})
	-- ({-3.7276*\dx},{-1.1253*\dy})
	-- ({-3.6360*\dx},{-1.2425*\dy})
	-- ({-3.5382*\dx},{-1.3496*\dy})
	-- ({-3.4357*\dx},{-1.4463*\dy})
	-- ({-3.3298*\dx},{-1.5327*\dy})
	-- ({-3.2216*\dx},{-1.6090*\dy})
	-- ({-3.1123*\dx},{-1.6755*\dy})
	-- ({-3.0028*\dx},{-1.7327*\dy})
	-- ({-2.8939*\dx},{-1.7811*\dy})
	-- ({-2.7864*\dx},{-1.8212*\dy})
	-- ({-2.6808*\dx},{-1.8537*\dy})
	-- ({-2.5776*\dx},{-1.8792*\dy})
	-- ({-2.4773*\dx},{-1.8983*\dy})
	-- ({-2.3801*\dx},{-1.9116*\dy})
	-- ({-2.2862*\dx},{-1.9197*\dy})
	-- ({-2.1957*\dx},{-1.9232*\dy})
	-- ({-2.1087*\dx},{-1.9225*\dy})
	-- ({-2.0253*\dx},{-1.9181*\dy})
	-- ({-1.9454*\dx},{-1.9105*\dy})
	-- ({-1.8691*\dx},{-1.9000*\dy})
	-- ({-1.7962*\dx},{-1.8871*\dy})
	-- ({-1.7266*\dx},{-1.8721*\dy})
	-- ({-1.6603*\dx},{-1.8552*\dy})
	-- ({-1.5971*\dx},{-1.8367*\dy})
	-- ({-1.5369*\dx},{-1.8170*\dy})
	-- ({-1.4797*\dx},{-1.7961*\dy})
	-- ({-1.4252*\dx},{-1.7742*\dy})
	-- ({-1.3734*\dx},{-1.7516*\dy})
	-- ({-1.3241*\dx},{-1.7284*\dy})
	-- ({-1.2771*\dx},{-1.7047*\dy})
	-- ({-1.2325*\dx},{-1.6807*\dy})
	-- ({-1.1901*\dx},{-1.6563*\dy})
	-- ({-1.1496*\dx},{-1.6318*\dy})
	-- ({-1.1112*\dx},{-1.6071*\dy})
	-- ({-1.0746*\dx},{-1.5825*\dy})
	-- ({-1.0397*\dx},{-1.5578*\dy})
	-- ({-1.0065*\dx},{-1.5332*\dy})
	-- ({-0.9748*\dx},{-1.5087*\dy})
	-- ({-0.9446*\dx},{-1.4843*\dy})
	-- ({-0.9158*\dx},{-1.4601*\dy})
	-- ({-0.8884*\dx},{-1.4361*\dy})
	-- ({-0.8622*\dx},{-1.4124*\dy})
	-- ({-0.8372*\dx},{-1.3888*\dy})
	-- ({-0.8133*\dx},{-1.3655*\dy})
	-- ({-0.7905*\dx},{-1.3425*\dy})
	-- ({-0.7687*\dx},{-1.3198*\dy})
	-- ({-0.7478*\dx},{-1.2973*\dy})
	-- ({-0.7279*\dx},{-1.2752*\dy})
	-- ({-0.7088*\dx},{-1.2533*\dy})
	-- ({-0.6905*\dx},{-1.2317*\dy})
	-- ({-0.6731*\dx},{-1.2104*\dy})
	-- ({-0.6563*\dx},{-1.1894*\dy})
	-- ({-0.6402*\dx},{-1.1687*\dy})
	-- ({-0.6248*\dx},{-1.1483*\dy})
	-- ({-0.6101*\dx},{-1.1282*\dy})
	-- ({-0.5959*\dx},{-1.1084*\dy})
	-- ({-0.5823*\dx},{-1.0888*\dy})
	-- ({-0.5692*\dx},{-1.0696*\dy})
	-- ({-0.5567*\dx},{-1.0506*\dy})
	-- ({-0.5447*\dx},{-1.0319*\dy})
	-- ({-0.5331*\dx},{-1.0134*\dy})
	-- ({-0.5219*\dx},{-0.9952*\dy})
	-- ({-0.5112*\dx},{-0.9773*\dy})
	-- ({-0.5009*\dx},{-0.9597*\dy})
	-- ({-0.4910*\dx},{-0.9422*\dy})
	-- ({-0.4815*\dx},{-0.9250*\dy})
	-- ({-0.4723*\dx},{-0.9081*\dy})
	-- ({-0.4635*\dx},{-0.8914*\dy})
	-- ({-0.4549*\dx},{-0.8749*\dy})
	-- ({-0.4467*\dx},{-0.8586*\dy})
	-- ({-0.4388*\dx},{-0.8426*\dy})
	-- ({-0.4312*\dx},{-0.8267*\dy})
	-- ({-0.4238*\dx},{-0.8111*\dy})
	-- ({-0.4167*\dx},{-0.7957*\dy})
	-- ({-0.4099*\dx},{-0.7804*\dy})
	-- ({-0.4033*\dx},{-0.7654*\dy})
	-- ({-0.3969*\dx},{-0.7505*\dy})
	-- ({-0.3908*\dx},{-0.7359*\dy})
	-- ({-0.3848*\dx},{-0.7213*\dy})
	-- ({-0.3791*\dx},{-0.7070*\dy})
	-- ({-0.3736*\dx},{-0.6928*\dy})
	-- ({-0.3682*\dx},{-0.6788*\dy})
	-- ({-0.3630*\dx},{-0.6650*\dy})
	-- ({-0.3581*\dx},{-0.6513*\dy})
	-- ({-0.3532*\dx},{-0.6378*\dy})
	-- ({-0.3486*\dx},{-0.6244*\dy})
	-- ({-0.3441*\dx},{-0.6111*\dy})
	-- ({-0.3398*\dx},{-0.5980*\dy})
	-- ({-0.3356*\dx},{-0.5850*\dy})
	-- ({-0.3315*\dx},{-0.5722*\dy})
	-- ({-0.3276*\dx},{-0.5594*\dy})
	-- ({-0.3238*\dx},{-0.5468*\dy})
	-- ({-0.3201*\dx},{-0.5343*\dy})
	-- ({-0.3166*\dx},{-0.5220*\dy})
	-- ({-0.3132*\dx},{-0.5097*\dy})
	-- ({-0.3099*\dx},{-0.4975*\dy})
	-- ({-0.3067*\dx},{-0.4855*\dy})
	-- ({-0.3036*\dx},{-0.4735*\dy})
	-- ({-0.3007*\dx},{-0.4617*\dy})
	-- ({-0.2978*\dx},{-0.4500*\dy})
	-- ({-0.2950*\dx},{-0.4383*\dy})
	-- ({-0.2924*\dx},{-0.4267*\dy})
	-- ({-0.2898*\dx},{-0.4153*\dy})
	-- ({-0.2873*\dx},{-0.4039*\dy})
	-- ({-0.2849*\dx},{-0.3925*\dy})
	-- ({-0.2826*\dx},{-0.3813*\dy})
	-- ({-0.2804*\dx},{-0.3702*\dy})
	-- ({-0.2782*\dx},{-0.3591*\dy})
	-- ({-0.2762*\dx},{-0.3481*\dy})
	-- ({-0.2742*\dx},{-0.3371*\dy})
	-- ({-0.2723*\dx},{-0.3262*\dy})
	-- ({-0.2705*\dx},{-0.3154*\dy})
	-- ({-0.2687*\dx},{-0.3047*\dy})
	-- ({-0.2670*\dx},{-0.2940*\dy})
	-- ({-0.2654*\dx},{-0.2833*\dy})
	-- ({-0.2639*\dx},{-0.2727*\dy})
	-- ({-0.2624*\dx},{-0.2622*\dy})
	-- ({-0.2610*\dx},{-0.2517*\dy})
	-- ({-0.2596*\dx},{-0.2413*\dy})
	-- ({-0.2583*\dx},{-0.2309*\dy})
	-- ({-0.2571*\dx},{-0.2206*\dy})
	-- ({-0.2560*\dx},{-0.2103*\dy})
	-- ({-0.2549*\dx},{-0.2000*\dy})
	-- ({-0.2538*\dx},{-0.1898*\dy})
	-- ({-0.2528*\dx},{-0.1796*\dy})
	-- ({-0.2519*\dx},{-0.1694*\dy})
	-- ({-0.2510*\dx},{-0.1593*\dy})
	-- ({-0.2502*\dx},{-0.1492*\dy})
	-- ({-0.2495*\dx},{-0.1391*\dy})
	-- ({-0.2488*\dx},{-0.1291*\dy})
	-- ({-0.2481*\dx},{-0.1191*\dy})
	-- ({-0.2475*\dx},{-0.1091*\dy})
	-- ({-0.2470*\dx},{-0.0991*\dy})
	-- ({-0.2465*\dx},{-0.0891*\dy})
	-- ({-0.2461*\dx},{-0.0792*\dy})
	-- ({-0.2457*\dx},{-0.0693*\dy})
	-- ({-0.2453*\dx},{-0.0593*\dy})
	-- ({-0.2451*\dx},{-0.0494*\dy})
	-- ({-0.2448*\dx},{-0.0395*\dy})
	-- ({-0.2447*\dx},{-0.0296*\dy})
	-- ({-0.2445*\dx},{-0.0198*\dy})
	-- ({-0.2445*\dx},{-0.0099*\dy})
	-- ({-0.2444*\dx},{-0.0000*\dy})
}
% u = -0.083160
\def\upathI{
	({-0.0830*\dx},{0.0000*\dy})
	-- ({-0.0830*\dx},{0.0104*\dy})
	-- ({-0.0830*\dx},{0.0209*\dy})
	-- ({-0.0831*\dx},{0.0313*\dy})
	-- ({-0.0831*\dx},{0.0418*\dy})
	-- ({-0.0832*\dx},{0.0522*\dy})
	-- ({-0.0833*\dx},{0.0627*\dy})
	-- ({-0.0834*\dx},{0.0732*\dy})
	-- ({-0.0836*\dx},{0.0837*\dy})
	-- ({-0.0837*\dx},{0.0942*\dy})
	-- ({-0.0839*\dx},{0.1047*\dy})
	-- ({-0.0841*\dx},{0.1153*\dy})
	-- ({-0.0843*\dx},{0.1259*\dy})
	-- ({-0.0845*\dx},{0.1365*\dy})
	-- ({-0.0848*\dx},{0.1471*\dy})
	-- ({-0.0850*\dx},{0.1578*\dy})
	-- ({-0.0853*\dx},{0.1685*\dy})
	-- ({-0.0857*\dx},{0.1793*\dy})
	-- ({-0.0860*\dx},{0.1900*\dy})
	-- ({-0.0863*\dx},{0.2009*\dy})
	-- ({-0.0867*\dx},{0.2118*\dy})
	-- ({-0.0871*\dx},{0.2227*\dy})
	-- ({-0.0875*\dx},{0.2337*\dy})
	-- ({-0.0880*\dx},{0.2447*\dy})
	-- ({-0.0884*\dx},{0.2558*\dy})
	-- ({-0.0889*\dx},{0.2669*\dy})
	-- ({-0.0894*\dx},{0.2781*\dy})
	-- ({-0.0900*\dx},{0.2894*\dy})
	-- ({-0.0905*\dx},{0.3007*\dy})
	-- ({-0.0911*\dx},{0.3121*\dy})
	-- ({-0.0917*\dx},{0.3236*\dy})
	-- ({-0.0924*\dx},{0.3352*\dy})
	-- ({-0.0930*\dx},{0.3468*\dy})
	-- ({-0.0937*\dx},{0.3585*\dy})
	-- ({-0.0944*\dx},{0.3703*\dy})
	-- ({-0.0952*\dx},{0.3822*\dy})
	-- ({-0.0960*\dx},{0.3942*\dy})
	-- ({-0.0968*\dx},{0.4063*\dy})
	-- ({-0.0976*\dx},{0.4185*\dy})
	-- ({-0.0985*\dx},{0.4308*\dy})
	-- ({-0.0994*\dx},{0.4432*\dy})
	-- ({-0.1003*\dx},{0.4557*\dy})
	-- ({-0.1013*\dx},{0.4684*\dy})
	-- ({-0.1023*\dx},{0.4812*\dy})
	-- ({-0.1034*\dx},{0.4941*\dy})
	-- ({-0.1045*\dx},{0.5071*\dy})
	-- ({-0.1056*\dx},{0.5202*\dy})
	-- ({-0.1068*\dx},{0.5335*\dy})
	-- ({-0.1080*\dx},{0.5470*\dy})
	-- ({-0.1093*\dx},{0.5606*\dy})
	-- ({-0.1106*\dx},{0.5744*\dy})
	-- ({-0.1120*\dx},{0.5883*\dy})
	-- ({-0.1134*\dx},{0.6024*\dy})
	-- ({-0.1148*\dx},{0.6167*\dy})
	-- ({-0.1163*\dx},{0.6311*\dy})
	-- ({-0.1179*\dx},{0.6458*\dy})
	-- ({-0.1195*\dx},{0.6606*\dy})
	-- ({-0.1212*\dx},{0.6757*\dy})
	-- ({-0.1230*\dx},{0.6909*\dy})
	-- ({-0.1248*\dx},{0.7064*\dy})
	-- ({-0.1267*\dx},{0.7221*\dy})
	-- ({-0.1286*\dx},{0.7380*\dy})
	-- ({-0.1307*\dx},{0.7542*\dy})
	-- ({-0.1328*\dx},{0.7706*\dy})
	-- ({-0.1350*\dx},{0.7873*\dy})
	-- ({-0.1373*\dx},{0.8043*\dy})
	-- ({-0.1396*\dx},{0.8215*\dy})
	-- ({-0.1421*\dx},{0.8391*\dy})
	-- ({-0.1446*\dx},{0.8569*\dy})
	-- ({-0.1473*\dx},{0.8751*\dy})
	-- ({-0.1501*\dx},{0.8936*\dy})
	-- ({-0.1529*\dx},{0.9124*\dy})
	-- ({-0.1559*\dx},{0.9316*\dy})
	-- ({-0.1590*\dx},{0.9512*\dy})
	-- ({-0.1623*\dx},{0.9711*\dy})
	-- ({-0.1657*\dx},{0.9915*\dy})
	-- ({-0.1692*\dx},{1.0122*\dy})
	-- ({-0.1729*\dx},{1.0334*\dy})
	-- ({-0.1767*\dx},{1.0550*\dy})
	-- ({-0.1807*\dx},{1.0772*\dy})
	-- ({-0.1849*\dx},{1.0998*\dy})
	-- ({-0.1892*\dx},{1.1229*\dy})
	-- ({-0.1938*\dx},{1.1465*\dy})
	-- ({-0.1986*\dx},{1.1707*\dy})
	-- ({-0.2036*\dx},{1.1955*\dy})
	-- ({-0.2088*\dx},{1.2209*\dy})
	-- ({-0.2143*\dx},{1.2469*\dy})
	-- ({-0.2201*\dx},{1.2736*\dy})
	-- ({-0.2261*\dx},{1.3010*\dy})
	-- ({-0.2324*\dx},{1.3292*\dy})
	-- ({-0.2391*\dx},{1.3581*\dy})
	-- ({-0.2461*\dx},{1.3878*\dy})
	-- ({-0.2534*\dx},{1.4183*\dy})
	-- ({-0.2612*\dx},{1.4497*\dy})
	-- ({-0.2694*\dx},{1.4820*\dy})
	-- ({-0.2780*\dx},{1.5154*\dy})
	-- ({-0.2871*\dx},{1.5497*\dy})
	-- ({-0.2967*\dx},{1.5851*\dy})
	-- ({-0.3069*\dx},{1.6217*\dy})
	-- ({-0.3176*\dx},{1.6595*\dy})
	-- ({-0.3290*\dx},{1.6985*\dy})
	-- ({-0.3411*\dx},{1.7388*\dy})
	-- ({-0.3540*\dx},{1.7806*\dy})
	-- ({-0.3676*\dx},{1.8239*\dy})
	-- ({-0.3822*\dx},{1.8687*\dy})
	-- ({-0.3977*\dx},{1.9152*\dy})
	-- ({-0.4142*\dx},{1.9635*\dy})
	-- ({-0.4319*\dx},{2.0137*\dy})
	-- ({-0.4508*\dx},{2.0659*\dy})
	-- ({-0.4711*\dx},{2.1202*\dy})
	-- ({-0.4928*\dx},{2.1767*\dy})
	-- ({-0.5162*\dx},{2.2357*\dy})
	-- ({-0.5414*\dx},{2.2972*\dy})
	-- ({-0.5686*\dx},{2.3615*\dy})
	-- ({-0.5979*\dx},{2.4287*\dy})
	-- ({-0.6297*\dx},{2.4990*\dy})
	-- ({-0.6641*\dx},{2.5727*\dy})
	-- ({-0.7016*\dx},{2.6499*\dy})
	-- ({-0.7424*\dx},{2.7310*\dy})
	-- ({-0.7870*\dx},{2.8162*\dy})
	-- ({-0.8357*\dx},{2.9058*\dy})
	-- ({-0.8892*\dx},{3.0001*\dy})
	-- ({-0.9481*\dx},{3.0994*\dy})
	-- ({-1.0130*\dx},{3.2042*\dy})
	-- ({-1.0848*\dx},{3.3148*\dy})
	-- ({-1.1644*\dx},{3.4315*\dy})
	-- ({-1.2531*\dx},{3.5549*\dy})
	-- ({-1.3521*\dx},{3.6852*\dy})
	-- ({-1.4631*\dx},{3.8230*\dy})
	-- ({-1.5879*\dx},{3.9685*\dy})
	-- ({-1.7288*\dx},{4.1220*\dy})
	-- ({-1.8884*\dx},{4.2838*\dy})
	-- ({-2.0699*\dx},{4.4537*\dy})
	-- ({-2.2772*\dx},{4.6314*\dy})
	-- ({-2.5148*\dx},{4.8161*\dy})
	-- ({-2.7882*\dx},{5.0063*\dy})
	-- ({-3.1038*\dx},{5.1993*\dy})
	-- ({-3.4691*\dx},{5.3912*\dy})
	-- ({-3.8929*\dx},{5.5757*\dy})
	-- ({-4.3847*\dx},{5.7433*\dy})
	-- ({-4.9546*\dx},{5.8804*\dy})
	-- ({-5.6120*\dx},{5.9675*\dy})
	-- ({-6.3635*\dx},{5.9776*\dy})
	-- ({-7.2092*\dx},{5.8751*\dy})
	-- ({-8.1369*\dx},{5.6159*\dy})
	-- ({-9.1151*\dx},{5.1510*\dy})
	-- ({-10.0858*\dx},{4.4356*\dy})
	-- ({-10.9618*\dx},{3.4450*\dy})
	-- ({-11.6358*\dx},{2.1947*\dy})
	-- ({-12.0049*\dx},{0.7549*\dy})
	-- ({-12.0049*\dx},{-0.7549*\dy})
	-- ({-11.6358*\dx},{-2.1947*\dy})
	-- ({-10.9618*\dx},{-3.4450*\dy})
	-- ({-10.0858*\dx},{-4.4356*\dy})
	-- ({-9.1151*\dx},{-5.1510*\dy})
	-- ({-8.1369*\dx},{-5.6159*\dy})
	-- ({-7.2092*\dx},{-5.8751*\dy})
	-- ({-6.3635*\dx},{-5.9776*\dy})
	-- ({-5.6120*\dx},{-5.9675*\dy})
	-- ({-4.9546*\dx},{-5.8804*\dy})
	-- ({-4.3847*\dx},{-5.7433*\dy})
	-- ({-3.8929*\dx},{-5.5757*\dy})
	-- ({-3.4691*\dx},{-5.3912*\dy})
	-- ({-3.1038*\dx},{-5.1993*\dy})
	-- ({-2.7882*\dx},{-5.0063*\dy})
	-- ({-2.5148*\dx},{-4.8161*\dy})
	-- ({-2.2772*\dx},{-4.6314*\dy})
	-- ({-2.0699*\dx},{-4.4537*\dy})
	-- ({-1.8884*\dx},{-4.2838*\dy})
	-- ({-1.7288*\dx},{-4.1220*\dy})
	-- ({-1.5879*\dx},{-3.9685*\dy})
	-- ({-1.4631*\dx},{-3.8230*\dy})
	-- ({-1.3521*\dx},{-3.6852*\dy})
	-- ({-1.2531*\dx},{-3.5549*\dy})
	-- ({-1.1644*\dx},{-3.4315*\dy})
	-- ({-1.0848*\dx},{-3.3148*\dy})
	-- ({-1.0130*\dx},{-3.2042*\dy})
	-- ({-0.9481*\dx},{-3.0994*\dy})
	-- ({-0.8892*\dx},{-3.0001*\dy})
	-- ({-0.8357*\dx},{-2.9058*\dy})
	-- ({-0.7870*\dx},{-2.8162*\dy})
	-- ({-0.7424*\dx},{-2.7310*\dy})
	-- ({-0.7016*\dx},{-2.6499*\dy})
	-- ({-0.6641*\dx},{-2.5727*\dy})
	-- ({-0.6297*\dx},{-2.4990*\dy})
	-- ({-0.5979*\dx},{-2.4287*\dy})
	-- ({-0.5686*\dx},{-2.3615*\dy})
	-- ({-0.5414*\dx},{-2.2972*\dy})
	-- ({-0.5162*\dx},{-2.2357*\dy})
	-- ({-0.4928*\dx},{-2.1767*\dy})
	-- ({-0.4711*\dx},{-2.1202*\dy})
	-- ({-0.4508*\dx},{-2.0659*\dy})
	-- ({-0.4319*\dx},{-2.0137*\dy})
	-- ({-0.4142*\dx},{-1.9635*\dy})
	-- ({-0.3977*\dx},{-1.9152*\dy})
	-- ({-0.3822*\dx},{-1.8687*\dy})
	-- ({-0.3676*\dx},{-1.8239*\dy})
	-- ({-0.3540*\dx},{-1.7806*\dy})
	-- ({-0.3411*\dx},{-1.7388*\dy})
	-- ({-0.3290*\dx},{-1.6985*\dy})
	-- ({-0.3176*\dx},{-1.6595*\dy})
	-- ({-0.3069*\dx},{-1.6217*\dy})
	-- ({-0.2967*\dx},{-1.5851*\dy})
	-- ({-0.2871*\dx},{-1.5497*\dy})
	-- ({-0.2780*\dx},{-1.5154*\dy})
	-- ({-0.2694*\dx},{-1.4820*\dy})
	-- ({-0.2612*\dx},{-1.4497*\dy})
	-- ({-0.2534*\dx},{-1.4183*\dy})
	-- ({-0.2461*\dx},{-1.3878*\dy})
	-- ({-0.2391*\dx},{-1.3581*\dy})
	-- ({-0.2324*\dx},{-1.3292*\dy})
	-- ({-0.2261*\dx},{-1.3010*\dy})
	-- ({-0.2201*\dx},{-1.2736*\dy})
	-- ({-0.2143*\dx},{-1.2469*\dy})
	-- ({-0.2088*\dx},{-1.2209*\dy})
	-- ({-0.2036*\dx},{-1.1955*\dy})
	-- ({-0.1986*\dx},{-1.1707*\dy})
	-- ({-0.1938*\dx},{-1.1465*\dy})
	-- ({-0.1892*\dx},{-1.1229*\dy})
	-- ({-0.1849*\dx},{-1.0998*\dy})
	-- ({-0.1807*\dx},{-1.0772*\dy})
	-- ({-0.1767*\dx},{-1.0550*\dy})
	-- ({-0.1729*\dx},{-1.0334*\dy})
	-- ({-0.1692*\dx},{-1.0122*\dy})
	-- ({-0.1657*\dx},{-0.9915*\dy})
	-- ({-0.1623*\dx},{-0.9711*\dy})
	-- ({-0.1590*\dx},{-0.9512*\dy})
	-- ({-0.1559*\dx},{-0.9316*\dy})
	-- ({-0.1529*\dx},{-0.9124*\dy})
	-- ({-0.1501*\dx},{-0.8936*\dy})
	-- ({-0.1473*\dx},{-0.8751*\dy})
	-- ({-0.1446*\dx},{-0.8569*\dy})
	-- ({-0.1421*\dx},{-0.8391*\dy})
	-- ({-0.1396*\dx},{-0.8215*\dy})
	-- ({-0.1373*\dx},{-0.8043*\dy})
	-- ({-0.1350*\dx},{-0.7873*\dy})
	-- ({-0.1328*\dx},{-0.7706*\dy})
	-- ({-0.1307*\dx},{-0.7542*\dy})
	-- ({-0.1286*\dx},{-0.7380*\dy})
	-- ({-0.1267*\dx},{-0.7221*\dy})
	-- ({-0.1248*\dx},{-0.7064*\dy})
	-- ({-0.1230*\dx},{-0.6909*\dy})
	-- ({-0.1212*\dx},{-0.6757*\dy})
	-- ({-0.1195*\dx},{-0.6606*\dy})
	-- ({-0.1179*\dx},{-0.6458*\dy})
	-- ({-0.1163*\dx},{-0.6311*\dy})
	-- ({-0.1148*\dx},{-0.6167*\dy})
	-- ({-0.1134*\dx},{-0.6024*\dy})
	-- ({-0.1120*\dx},{-0.5883*\dy})
	-- ({-0.1106*\dx},{-0.5744*\dy})
	-- ({-0.1093*\dx},{-0.5606*\dy})
	-- ({-0.1080*\dx},{-0.5470*\dy})
	-- ({-0.1068*\dx},{-0.5335*\dy})
	-- ({-0.1056*\dx},{-0.5202*\dy})
	-- ({-0.1045*\dx},{-0.5071*\dy})
	-- ({-0.1034*\dx},{-0.4941*\dy})
	-- ({-0.1023*\dx},{-0.4812*\dy})
	-- ({-0.1013*\dx},{-0.4684*\dy})
	-- ({-0.1003*\dx},{-0.4557*\dy})
	-- ({-0.0994*\dx},{-0.4432*\dy})
	-- ({-0.0985*\dx},{-0.4308*\dy})
	-- ({-0.0976*\dx},{-0.4185*\dy})
	-- ({-0.0968*\dx},{-0.4063*\dy})
	-- ({-0.0960*\dx},{-0.3942*\dy})
	-- ({-0.0952*\dx},{-0.3822*\dy})
	-- ({-0.0944*\dx},{-0.3703*\dy})
	-- ({-0.0937*\dx},{-0.3585*\dy})
	-- ({-0.0930*\dx},{-0.3468*\dy})
	-- ({-0.0924*\dx},{-0.3352*\dy})
	-- ({-0.0917*\dx},{-0.3236*\dy})
	-- ({-0.0911*\dx},{-0.3121*\dy})
	-- ({-0.0905*\dx},{-0.3007*\dy})
	-- ({-0.0900*\dx},{-0.2894*\dy})
	-- ({-0.0894*\dx},{-0.2781*\dy})
	-- ({-0.0889*\dx},{-0.2669*\dy})
	-- ({-0.0884*\dx},{-0.2558*\dy})
	-- ({-0.0880*\dx},{-0.2447*\dy})
	-- ({-0.0875*\dx},{-0.2337*\dy})
	-- ({-0.0871*\dx},{-0.2227*\dy})
	-- ({-0.0867*\dx},{-0.2118*\dy})
	-- ({-0.0863*\dx},{-0.2009*\dy})
	-- ({-0.0860*\dx},{-0.1900*\dy})
	-- ({-0.0857*\dx},{-0.1793*\dy})
	-- ({-0.0853*\dx},{-0.1685*\dy})
	-- ({-0.0850*\dx},{-0.1578*\dy})
	-- ({-0.0848*\dx},{-0.1471*\dy})
	-- ({-0.0845*\dx},{-0.1365*\dy})
	-- ({-0.0843*\dx},{-0.1259*\dy})
	-- ({-0.0841*\dx},{-0.1153*\dy})
	-- ({-0.0839*\dx},{-0.1047*\dy})
	-- ({-0.0837*\dx},{-0.0942*\dy})
	-- ({-0.0836*\dx},{-0.0837*\dy})
	-- ({-0.0834*\dx},{-0.0732*\dy})
	-- ({-0.0833*\dx},{-0.0627*\dy})
	-- ({-0.0832*\dx},{-0.0522*\dy})
	-- ({-0.0831*\dx},{-0.0418*\dy})
	-- ({-0.0831*\dx},{-0.0313*\dy})
	-- ({-0.0830*\dx},{-0.0209*\dy})
	-- ({-0.0830*\dx},{-0.0104*\dy})
	-- ({-0.0830*\dx},{-0.0000*\dy})
}
% u = 0.083160
\def\upathJ{
	({0.0830*\dx},{0.0000*\dy})
	-- ({0.0830*\dx},{0.0104*\dy})
	-- ({0.0830*\dx},{0.0209*\dy})
	-- ({0.0831*\dx},{0.0313*\dy})
	-- ({0.0831*\dx},{0.0418*\dy})
	-- ({0.0832*\dx},{0.0522*\dy})
	-- ({0.0833*\dx},{0.0627*\dy})
	-- ({0.0834*\dx},{0.0732*\dy})
	-- ({0.0836*\dx},{0.0837*\dy})
	-- ({0.0837*\dx},{0.0942*\dy})
	-- ({0.0839*\dx},{0.1047*\dy})
	-- ({0.0841*\dx},{0.1153*\dy})
	-- ({0.0843*\dx},{0.1259*\dy})
	-- ({0.0845*\dx},{0.1365*\dy})
	-- ({0.0848*\dx},{0.1471*\dy})
	-- ({0.0850*\dx},{0.1578*\dy})
	-- ({0.0853*\dx},{0.1685*\dy})
	-- ({0.0857*\dx},{0.1793*\dy})
	-- ({0.0860*\dx},{0.1900*\dy})
	-- ({0.0863*\dx},{0.2009*\dy})
	-- ({0.0867*\dx},{0.2118*\dy})
	-- ({0.0871*\dx},{0.2227*\dy})
	-- ({0.0875*\dx},{0.2337*\dy})
	-- ({0.0880*\dx},{0.2447*\dy})
	-- ({0.0884*\dx},{0.2558*\dy})
	-- ({0.0889*\dx},{0.2669*\dy})
	-- ({0.0894*\dx},{0.2781*\dy})
	-- ({0.0900*\dx},{0.2894*\dy})
	-- ({0.0905*\dx},{0.3007*\dy})
	-- ({0.0911*\dx},{0.3121*\dy})
	-- ({0.0917*\dx},{0.3236*\dy})
	-- ({0.0924*\dx},{0.3352*\dy})
	-- ({0.0930*\dx},{0.3468*\dy})
	-- ({0.0937*\dx},{0.3585*\dy})
	-- ({0.0944*\dx},{0.3703*\dy})
	-- ({0.0952*\dx},{0.3822*\dy})
	-- ({0.0960*\dx},{0.3942*\dy})
	-- ({0.0968*\dx},{0.4063*\dy})
	-- ({0.0976*\dx},{0.4185*\dy})
	-- ({0.0985*\dx},{0.4308*\dy})
	-- ({0.0994*\dx},{0.4432*\dy})
	-- ({0.1003*\dx},{0.4557*\dy})
	-- ({0.1013*\dx},{0.4684*\dy})
	-- ({0.1023*\dx},{0.4812*\dy})
	-- ({0.1034*\dx},{0.4941*\dy})
	-- ({0.1045*\dx},{0.5071*\dy})
	-- ({0.1056*\dx},{0.5202*\dy})
	-- ({0.1068*\dx},{0.5335*\dy})
	-- ({0.1080*\dx},{0.5470*\dy})
	-- ({0.1093*\dx},{0.5606*\dy})
	-- ({0.1106*\dx},{0.5744*\dy})
	-- ({0.1120*\dx},{0.5883*\dy})
	-- ({0.1134*\dx},{0.6024*\dy})
	-- ({0.1148*\dx},{0.6167*\dy})
	-- ({0.1163*\dx},{0.6311*\dy})
	-- ({0.1179*\dx},{0.6458*\dy})
	-- ({0.1195*\dx},{0.6606*\dy})
	-- ({0.1212*\dx},{0.6757*\dy})
	-- ({0.1230*\dx},{0.6909*\dy})
	-- ({0.1248*\dx},{0.7064*\dy})
	-- ({0.1267*\dx},{0.7221*\dy})
	-- ({0.1286*\dx},{0.7380*\dy})
	-- ({0.1307*\dx},{0.7542*\dy})
	-- ({0.1328*\dx},{0.7706*\dy})
	-- ({0.1350*\dx},{0.7873*\dy})
	-- ({0.1373*\dx},{0.8043*\dy})
	-- ({0.1396*\dx},{0.8215*\dy})
	-- ({0.1421*\dx},{0.8391*\dy})
	-- ({0.1446*\dx},{0.8569*\dy})
	-- ({0.1473*\dx},{0.8751*\dy})
	-- ({0.1501*\dx},{0.8936*\dy})
	-- ({0.1529*\dx},{0.9124*\dy})
	-- ({0.1559*\dx},{0.9316*\dy})
	-- ({0.1590*\dx},{0.9512*\dy})
	-- ({0.1623*\dx},{0.9711*\dy})
	-- ({0.1657*\dx},{0.9915*\dy})
	-- ({0.1692*\dx},{1.0122*\dy})
	-- ({0.1729*\dx},{1.0334*\dy})
	-- ({0.1767*\dx},{1.0550*\dy})
	-- ({0.1807*\dx},{1.0772*\dy})
	-- ({0.1849*\dx},{1.0998*\dy})
	-- ({0.1892*\dx},{1.1229*\dy})
	-- ({0.1938*\dx},{1.1465*\dy})
	-- ({0.1986*\dx},{1.1707*\dy})
	-- ({0.2036*\dx},{1.1955*\dy})
	-- ({0.2088*\dx},{1.2209*\dy})
	-- ({0.2143*\dx},{1.2469*\dy})
	-- ({0.2201*\dx},{1.2736*\dy})
	-- ({0.2261*\dx},{1.3010*\dy})
	-- ({0.2324*\dx},{1.3292*\dy})
	-- ({0.2391*\dx},{1.3581*\dy})
	-- ({0.2461*\dx},{1.3878*\dy})
	-- ({0.2534*\dx},{1.4183*\dy})
	-- ({0.2612*\dx},{1.4497*\dy})
	-- ({0.2694*\dx},{1.4820*\dy})
	-- ({0.2780*\dx},{1.5154*\dy})
	-- ({0.2871*\dx},{1.5497*\dy})
	-- ({0.2967*\dx},{1.5851*\dy})
	-- ({0.3069*\dx},{1.6217*\dy})
	-- ({0.3176*\dx},{1.6595*\dy})
	-- ({0.3290*\dx},{1.6985*\dy})
	-- ({0.3411*\dx},{1.7388*\dy})
	-- ({0.3540*\dx},{1.7806*\dy})
	-- ({0.3676*\dx},{1.8239*\dy})
	-- ({0.3822*\dx},{1.8687*\dy})
	-- ({0.3977*\dx},{1.9152*\dy})
	-- ({0.4142*\dx},{1.9635*\dy})
	-- ({0.4319*\dx},{2.0137*\dy})
	-- ({0.4508*\dx},{2.0659*\dy})
	-- ({0.4711*\dx},{2.1202*\dy})
	-- ({0.4928*\dx},{2.1767*\dy})
	-- ({0.5162*\dx},{2.2357*\dy})
	-- ({0.5414*\dx},{2.2972*\dy})
	-- ({0.5686*\dx},{2.3615*\dy})
	-- ({0.5979*\dx},{2.4287*\dy})
	-- ({0.6297*\dx},{2.4990*\dy})
	-- ({0.6641*\dx},{2.5727*\dy})
	-- ({0.7016*\dx},{2.6499*\dy})
	-- ({0.7424*\dx},{2.7310*\dy})
	-- ({0.7870*\dx},{2.8162*\dy})
	-- ({0.8357*\dx},{2.9058*\dy})
	-- ({0.8892*\dx},{3.0001*\dy})
	-- ({0.9481*\dx},{3.0994*\dy})
	-- ({1.0130*\dx},{3.2042*\dy})
	-- ({1.0848*\dx},{3.3148*\dy})
	-- ({1.1644*\dx},{3.4315*\dy})
	-- ({1.2531*\dx},{3.5549*\dy})
	-- ({1.3521*\dx},{3.6852*\dy})
	-- ({1.4631*\dx},{3.8230*\dy})
	-- ({1.5879*\dx},{3.9685*\dy})
	-- ({1.7288*\dx},{4.1220*\dy})
	-- ({1.8884*\dx},{4.2838*\dy})
	-- ({2.0699*\dx},{4.4537*\dy})
	-- ({2.2772*\dx},{4.6314*\dy})
	-- ({2.5148*\dx},{4.8161*\dy})
	-- ({2.7882*\dx},{5.0063*\dy})
	-- ({3.1038*\dx},{5.1993*\dy})
	-- ({3.4691*\dx},{5.3912*\dy})
	-- ({3.8929*\dx},{5.5757*\dy})
	-- ({4.3847*\dx},{5.7433*\dy})
	-- ({4.9546*\dx},{5.8804*\dy})
	-- ({5.6120*\dx},{5.9675*\dy})
	-- ({6.3635*\dx},{5.9776*\dy})
	-- ({7.2092*\dx},{5.8751*\dy})
	-- ({8.1369*\dx},{5.6159*\dy})
	-- ({9.1151*\dx},{5.1510*\dy})
	-- ({10.0858*\dx},{4.4356*\dy})
	-- ({10.9618*\dx},{3.4450*\dy})
	-- ({11.6358*\dx},{2.1947*\dy})
	-- ({12.0049*\dx},{0.7549*\dy})
	-- ({12.0049*\dx},{-0.7549*\dy})
	-- ({11.6358*\dx},{-2.1947*\dy})
	-- ({10.9618*\dx},{-3.4450*\dy})
	-- ({10.0858*\dx},{-4.4356*\dy})
	-- ({9.1151*\dx},{-5.1510*\dy})
	-- ({8.1369*\dx},{-5.6159*\dy})
	-- ({7.2092*\dx},{-5.8751*\dy})
	-- ({6.3635*\dx},{-5.9776*\dy})
	-- ({5.6120*\dx},{-5.9675*\dy})
	-- ({4.9546*\dx},{-5.8804*\dy})
	-- ({4.3847*\dx},{-5.7433*\dy})
	-- ({3.8929*\dx},{-5.5757*\dy})
	-- ({3.4691*\dx},{-5.3912*\dy})
	-- ({3.1038*\dx},{-5.1993*\dy})
	-- ({2.7882*\dx},{-5.0063*\dy})
	-- ({2.5148*\dx},{-4.8161*\dy})
	-- ({2.2772*\dx},{-4.6314*\dy})
	-- ({2.0699*\dx},{-4.4537*\dy})
	-- ({1.8884*\dx},{-4.2838*\dy})
	-- ({1.7288*\dx},{-4.1220*\dy})
	-- ({1.5879*\dx},{-3.9685*\dy})
	-- ({1.4631*\dx},{-3.8230*\dy})
	-- ({1.3521*\dx},{-3.6852*\dy})
	-- ({1.2531*\dx},{-3.5549*\dy})
	-- ({1.1644*\dx},{-3.4315*\dy})
	-- ({1.0848*\dx},{-3.3148*\dy})
	-- ({1.0130*\dx},{-3.2042*\dy})
	-- ({0.9481*\dx},{-3.0994*\dy})
	-- ({0.8892*\dx},{-3.0001*\dy})
	-- ({0.8357*\dx},{-2.9058*\dy})
	-- ({0.7870*\dx},{-2.8162*\dy})
	-- ({0.7424*\dx},{-2.7310*\dy})
	-- ({0.7016*\dx},{-2.6499*\dy})
	-- ({0.6641*\dx},{-2.5727*\dy})
	-- ({0.6297*\dx},{-2.4990*\dy})
	-- ({0.5979*\dx},{-2.4287*\dy})
	-- ({0.5686*\dx},{-2.3615*\dy})
	-- ({0.5414*\dx},{-2.2972*\dy})
	-- ({0.5162*\dx},{-2.2357*\dy})
	-- ({0.4928*\dx},{-2.1767*\dy})
	-- ({0.4711*\dx},{-2.1202*\dy})
	-- ({0.4508*\dx},{-2.0659*\dy})
	-- ({0.4319*\dx},{-2.0137*\dy})
	-- ({0.4142*\dx},{-1.9635*\dy})
	-- ({0.3977*\dx},{-1.9152*\dy})
	-- ({0.3822*\dx},{-1.8687*\dy})
	-- ({0.3676*\dx},{-1.8239*\dy})
	-- ({0.3540*\dx},{-1.7806*\dy})
	-- ({0.3411*\dx},{-1.7388*\dy})
	-- ({0.3290*\dx},{-1.6985*\dy})
	-- ({0.3176*\dx},{-1.6595*\dy})
	-- ({0.3069*\dx},{-1.6217*\dy})
	-- ({0.2967*\dx},{-1.5851*\dy})
	-- ({0.2871*\dx},{-1.5497*\dy})
	-- ({0.2780*\dx},{-1.5154*\dy})
	-- ({0.2694*\dx},{-1.4820*\dy})
	-- ({0.2612*\dx},{-1.4497*\dy})
	-- ({0.2534*\dx},{-1.4183*\dy})
	-- ({0.2461*\dx},{-1.3878*\dy})
	-- ({0.2391*\dx},{-1.3581*\dy})
	-- ({0.2324*\dx},{-1.3292*\dy})
	-- ({0.2261*\dx},{-1.3010*\dy})
	-- ({0.2201*\dx},{-1.2736*\dy})
	-- ({0.2143*\dx},{-1.2469*\dy})
	-- ({0.2088*\dx},{-1.2209*\dy})
	-- ({0.2036*\dx},{-1.1955*\dy})
	-- ({0.1986*\dx},{-1.1707*\dy})
	-- ({0.1938*\dx},{-1.1465*\dy})
	-- ({0.1892*\dx},{-1.1229*\dy})
	-- ({0.1849*\dx},{-1.0998*\dy})
	-- ({0.1807*\dx},{-1.0772*\dy})
	-- ({0.1767*\dx},{-1.0550*\dy})
	-- ({0.1729*\dx},{-1.0334*\dy})
	-- ({0.1692*\dx},{-1.0122*\dy})
	-- ({0.1657*\dx},{-0.9915*\dy})
	-- ({0.1623*\dx},{-0.9711*\dy})
	-- ({0.1590*\dx},{-0.9512*\dy})
	-- ({0.1559*\dx},{-0.9316*\dy})
	-- ({0.1529*\dx},{-0.9124*\dy})
	-- ({0.1501*\dx},{-0.8936*\dy})
	-- ({0.1473*\dx},{-0.8751*\dy})
	-- ({0.1446*\dx},{-0.8569*\dy})
	-- ({0.1421*\dx},{-0.8391*\dy})
	-- ({0.1396*\dx},{-0.8215*\dy})
	-- ({0.1373*\dx},{-0.8043*\dy})
	-- ({0.1350*\dx},{-0.7873*\dy})
	-- ({0.1328*\dx},{-0.7706*\dy})
	-- ({0.1307*\dx},{-0.7542*\dy})
	-- ({0.1286*\dx},{-0.7380*\dy})
	-- ({0.1267*\dx},{-0.7221*\dy})
	-- ({0.1248*\dx},{-0.7064*\dy})
	-- ({0.1230*\dx},{-0.6909*\dy})
	-- ({0.1212*\dx},{-0.6757*\dy})
	-- ({0.1195*\dx},{-0.6606*\dy})
	-- ({0.1179*\dx},{-0.6458*\dy})
	-- ({0.1163*\dx},{-0.6311*\dy})
	-- ({0.1148*\dx},{-0.6167*\dy})
	-- ({0.1134*\dx},{-0.6024*\dy})
	-- ({0.1120*\dx},{-0.5883*\dy})
	-- ({0.1106*\dx},{-0.5744*\dy})
	-- ({0.1093*\dx},{-0.5606*\dy})
	-- ({0.1080*\dx},{-0.5470*\dy})
	-- ({0.1068*\dx},{-0.5335*\dy})
	-- ({0.1056*\dx},{-0.5202*\dy})
	-- ({0.1045*\dx},{-0.5071*\dy})
	-- ({0.1034*\dx},{-0.4941*\dy})
	-- ({0.1023*\dx},{-0.4812*\dy})
	-- ({0.1013*\dx},{-0.4684*\dy})
	-- ({0.1003*\dx},{-0.4557*\dy})
	-- ({0.0994*\dx},{-0.4432*\dy})
	-- ({0.0985*\dx},{-0.4308*\dy})
	-- ({0.0976*\dx},{-0.4185*\dy})
	-- ({0.0968*\dx},{-0.4063*\dy})
	-- ({0.0960*\dx},{-0.3942*\dy})
	-- ({0.0952*\dx},{-0.3822*\dy})
	-- ({0.0944*\dx},{-0.3703*\dy})
	-- ({0.0937*\dx},{-0.3585*\dy})
	-- ({0.0930*\dx},{-0.3468*\dy})
	-- ({0.0924*\dx},{-0.3352*\dy})
	-- ({0.0917*\dx},{-0.3236*\dy})
	-- ({0.0911*\dx},{-0.3121*\dy})
	-- ({0.0905*\dx},{-0.3007*\dy})
	-- ({0.0900*\dx},{-0.2894*\dy})
	-- ({0.0894*\dx},{-0.2781*\dy})
	-- ({0.0889*\dx},{-0.2669*\dy})
	-- ({0.0884*\dx},{-0.2558*\dy})
	-- ({0.0880*\dx},{-0.2447*\dy})
	-- ({0.0875*\dx},{-0.2337*\dy})
	-- ({0.0871*\dx},{-0.2227*\dy})
	-- ({0.0867*\dx},{-0.2118*\dy})
	-- ({0.0863*\dx},{-0.2009*\dy})
	-- ({0.0860*\dx},{-0.1900*\dy})
	-- ({0.0857*\dx},{-0.1793*\dy})
	-- ({0.0853*\dx},{-0.1685*\dy})
	-- ({0.0850*\dx},{-0.1578*\dy})
	-- ({0.0848*\dx},{-0.1471*\dy})
	-- ({0.0845*\dx},{-0.1365*\dy})
	-- ({0.0843*\dx},{-0.1259*\dy})
	-- ({0.0841*\dx},{-0.1153*\dy})
	-- ({0.0839*\dx},{-0.1047*\dy})
	-- ({0.0837*\dx},{-0.0942*\dy})
	-- ({0.0836*\dx},{-0.0837*\dy})
	-- ({0.0834*\dx},{-0.0732*\dy})
	-- ({0.0833*\dx},{-0.0627*\dy})
	-- ({0.0832*\dx},{-0.0522*\dy})
	-- ({0.0831*\dx},{-0.0418*\dy})
	-- ({0.0831*\dx},{-0.0313*\dy})
	-- ({0.0830*\dx},{-0.0209*\dy})
	-- ({0.0830*\dx},{-0.0104*\dy})
	-- ({0.0830*\dx},{-0.0000*\dy})
}
% u = 0.249479
\def\upathK{
	({0.2444*\dx},{0.0000*\dy})
	-- ({0.2445*\dx},{0.0099*\dy})
	-- ({0.2445*\dx},{0.0198*\dy})
	-- ({0.2447*\dx},{0.0296*\dy})
	-- ({0.2448*\dx},{0.0395*\dy})
	-- ({0.2451*\dx},{0.0494*\dy})
	-- ({0.2453*\dx},{0.0593*\dy})
	-- ({0.2457*\dx},{0.0693*\dy})
	-- ({0.2461*\dx},{0.0792*\dy})
	-- ({0.2465*\dx},{0.0891*\dy})
	-- ({0.2470*\dx},{0.0991*\dy})
	-- ({0.2475*\dx},{0.1091*\dy})
	-- ({0.2481*\dx},{0.1191*\dy})
	-- ({0.2488*\dx},{0.1291*\dy})
	-- ({0.2495*\dx},{0.1391*\dy})
	-- ({0.2502*\dx},{0.1492*\dy})
	-- ({0.2510*\dx},{0.1593*\dy})
	-- ({0.2519*\dx},{0.1694*\dy})
	-- ({0.2528*\dx},{0.1796*\dy})
	-- ({0.2538*\dx},{0.1898*\dy})
	-- ({0.2549*\dx},{0.2000*\dy})
	-- ({0.2560*\dx},{0.2103*\dy})
	-- ({0.2571*\dx},{0.2206*\dy})
	-- ({0.2583*\dx},{0.2309*\dy})
	-- ({0.2596*\dx},{0.2413*\dy})
	-- ({0.2610*\dx},{0.2517*\dy})
	-- ({0.2624*\dx},{0.2622*\dy})
	-- ({0.2639*\dx},{0.2727*\dy})
	-- ({0.2654*\dx},{0.2833*\dy})
	-- ({0.2670*\dx},{0.2940*\dy})
	-- ({0.2687*\dx},{0.3047*\dy})
	-- ({0.2705*\dx},{0.3154*\dy})
	-- ({0.2723*\dx},{0.3262*\dy})
	-- ({0.2742*\dx},{0.3371*\dy})
	-- ({0.2762*\dx},{0.3481*\dy})
	-- ({0.2782*\dx},{0.3591*\dy})
	-- ({0.2804*\dx},{0.3702*\dy})
	-- ({0.2826*\dx},{0.3813*\dy})
	-- ({0.2849*\dx},{0.3925*\dy})
	-- ({0.2873*\dx},{0.4039*\dy})
	-- ({0.2898*\dx},{0.4153*\dy})
	-- ({0.2924*\dx},{0.4267*\dy})
	-- ({0.2950*\dx},{0.4383*\dy})
	-- ({0.2978*\dx},{0.4500*\dy})
	-- ({0.3007*\dx},{0.4617*\dy})
	-- ({0.3036*\dx},{0.4735*\dy})
	-- ({0.3067*\dx},{0.4855*\dy})
	-- ({0.3099*\dx},{0.4975*\dy})
	-- ({0.3132*\dx},{0.5097*\dy})
	-- ({0.3166*\dx},{0.5220*\dy})
	-- ({0.3201*\dx},{0.5343*\dy})
	-- ({0.3238*\dx},{0.5468*\dy})
	-- ({0.3276*\dx},{0.5594*\dy})
	-- ({0.3315*\dx},{0.5722*\dy})
	-- ({0.3356*\dx},{0.5850*\dy})
	-- ({0.3398*\dx},{0.5980*\dy})
	-- ({0.3441*\dx},{0.6111*\dy})
	-- ({0.3486*\dx},{0.6244*\dy})
	-- ({0.3532*\dx},{0.6378*\dy})
	-- ({0.3581*\dx},{0.6513*\dy})
	-- ({0.3630*\dx},{0.6650*\dy})
	-- ({0.3682*\dx},{0.6788*\dy})
	-- ({0.3736*\dx},{0.6928*\dy})
	-- ({0.3791*\dx},{0.7070*\dy})
	-- ({0.3848*\dx},{0.7213*\dy})
	-- ({0.3908*\dx},{0.7359*\dy})
	-- ({0.3969*\dx},{0.7505*\dy})
	-- ({0.4033*\dx},{0.7654*\dy})
	-- ({0.4099*\dx},{0.7804*\dy})
	-- ({0.4167*\dx},{0.7957*\dy})
	-- ({0.4238*\dx},{0.8111*\dy})
	-- ({0.4312*\dx},{0.8267*\dy})
	-- ({0.4388*\dx},{0.8426*\dy})
	-- ({0.4467*\dx},{0.8586*\dy})
	-- ({0.4549*\dx},{0.8749*\dy})
	-- ({0.4635*\dx},{0.8914*\dy})
	-- ({0.4723*\dx},{0.9081*\dy})
	-- ({0.4815*\dx},{0.9250*\dy})
	-- ({0.4910*\dx},{0.9422*\dy})
	-- ({0.5009*\dx},{0.9597*\dy})
	-- ({0.5112*\dx},{0.9773*\dy})
	-- ({0.5219*\dx},{0.9952*\dy})
	-- ({0.5331*\dx},{1.0134*\dy})
	-- ({0.5447*\dx},{1.0319*\dy})
	-- ({0.5567*\dx},{1.0506*\dy})
	-- ({0.5692*\dx},{1.0696*\dy})
	-- ({0.5823*\dx},{1.0888*\dy})
	-- ({0.5959*\dx},{1.1084*\dy})
	-- ({0.6101*\dx},{1.1282*\dy})
	-- ({0.6248*\dx},{1.1483*\dy})
	-- ({0.6402*\dx},{1.1687*\dy})
	-- ({0.6563*\dx},{1.1894*\dy})
	-- ({0.6731*\dx},{1.2104*\dy})
	-- ({0.6905*\dx},{1.2317*\dy})
	-- ({0.7088*\dx},{1.2533*\dy})
	-- ({0.7279*\dx},{1.2752*\dy})
	-- ({0.7478*\dx},{1.2973*\dy})
	-- ({0.7687*\dx},{1.3198*\dy})
	-- ({0.7905*\dx},{1.3425*\dy})
	-- ({0.8133*\dx},{1.3655*\dy})
	-- ({0.8372*\dx},{1.3888*\dy})
	-- ({0.8622*\dx},{1.4124*\dy})
	-- ({0.8884*\dx},{1.4361*\dy})
	-- ({0.9158*\dx},{1.4601*\dy})
	-- ({0.9446*\dx},{1.4843*\dy})
	-- ({0.9748*\dx},{1.5087*\dy})
	-- ({1.0065*\dx},{1.5332*\dy})
	-- ({1.0397*\dx},{1.5578*\dy})
	-- ({1.0746*\dx},{1.5825*\dy})
	-- ({1.1112*\dx},{1.6071*\dy})
	-- ({1.1496*\dx},{1.6318*\dy})
	-- ({1.1901*\dx},{1.6563*\dy})
	-- ({1.2325*\dx},{1.6807*\dy})
	-- ({1.2771*\dx},{1.7047*\dy})
	-- ({1.3241*\dx},{1.7284*\dy})
	-- ({1.3734*\dx},{1.7516*\dy})
	-- ({1.4252*\dx},{1.7742*\dy})
	-- ({1.4797*\dx},{1.7961*\dy})
	-- ({1.5369*\dx},{1.8170*\dy})
	-- ({1.5971*\dx},{1.8367*\dy})
	-- ({1.6603*\dx},{1.8552*\dy})
	-- ({1.7266*\dx},{1.8721*\dy})
	-- ({1.7962*\dx},{1.8871*\dy})
	-- ({1.8691*\dx},{1.9000*\dy})
	-- ({1.9454*\dx},{1.9105*\dy})
	-- ({2.0253*\dx},{1.9181*\dy})
	-- ({2.1087*\dx},{1.9225*\dy})
	-- ({2.1957*\dx},{1.9232*\dy})
	-- ({2.2862*\dx},{1.9197*\dy})
	-- ({2.3801*\dx},{1.9116*\dy})
	-- ({2.4773*\dx},{1.8983*\dy})
	-- ({2.5776*\dx},{1.8792*\dy})
	-- ({2.6808*\dx},{1.8537*\dy})
	-- ({2.7864*\dx},{1.8212*\dy})
	-- ({2.8939*\dx},{1.7811*\dy})
	-- ({3.0028*\dx},{1.7327*\dy})
	-- ({3.1123*\dx},{1.6755*\dy})
	-- ({3.2216*\dx},{1.6090*\dy})
	-- ({3.3298*\dx},{1.5327*\dy})
	-- ({3.4357*\dx},{1.4463*\dy})
	-- ({3.5382*\dx},{1.3496*\dy})
	-- ({3.6360*\dx},{1.2425*\dy})
	-- ({3.7276*\dx},{1.1253*\dy})
	-- ({3.8118*\dx},{0.9983*\dy})
	-- ({3.8871*\dx},{0.8622*\dy})
	-- ({3.9522*\dx},{0.7178*\dy})
	-- ({4.0059*\dx},{0.5662*\dy})
	-- ({4.0472*\dx},{0.4088*\dy})
	-- ({4.0752*\dx},{0.2470*\dy})
	-- ({4.0894*\dx},{0.0826*\dy})
	-- ({4.0894*\dx},{-0.0826*\dy})
	-- ({4.0752*\dx},{-0.2470*\dy})
	-- ({4.0472*\dx},{-0.4088*\dy})
	-- ({4.0059*\dx},{-0.5662*\dy})
	-- ({3.9522*\dx},{-0.7178*\dy})
	-- ({3.8871*\dx},{-0.8622*\dy})
	-- ({3.8118*\dx},{-0.9983*\dy})
	-- ({3.7276*\dx},{-1.1253*\dy})
	-- ({3.6360*\dx},{-1.2425*\dy})
	-- ({3.5382*\dx},{-1.3496*\dy})
	-- ({3.4357*\dx},{-1.4463*\dy})
	-- ({3.3298*\dx},{-1.5327*\dy})
	-- ({3.2216*\dx},{-1.6090*\dy})
	-- ({3.1123*\dx},{-1.6755*\dy})
	-- ({3.0028*\dx},{-1.7327*\dy})
	-- ({2.8939*\dx},{-1.7811*\dy})
	-- ({2.7864*\dx},{-1.8212*\dy})
	-- ({2.6808*\dx},{-1.8537*\dy})
	-- ({2.5776*\dx},{-1.8792*\dy})
	-- ({2.4773*\dx},{-1.8983*\dy})
	-- ({2.3801*\dx},{-1.9116*\dy})
	-- ({2.2862*\dx},{-1.9197*\dy})
	-- ({2.1957*\dx},{-1.9232*\dy})
	-- ({2.1087*\dx},{-1.9225*\dy})
	-- ({2.0253*\dx},{-1.9181*\dy})
	-- ({1.9454*\dx},{-1.9105*\dy})
	-- ({1.8691*\dx},{-1.9000*\dy})
	-- ({1.7962*\dx},{-1.8871*\dy})
	-- ({1.7266*\dx},{-1.8721*\dy})
	-- ({1.6603*\dx},{-1.8552*\dy})
	-- ({1.5971*\dx},{-1.8367*\dy})
	-- ({1.5369*\dx},{-1.8170*\dy})
	-- ({1.4797*\dx},{-1.7961*\dy})
	-- ({1.4252*\dx},{-1.7742*\dy})
	-- ({1.3734*\dx},{-1.7516*\dy})
	-- ({1.3241*\dx},{-1.7284*\dy})
	-- ({1.2771*\dx},{-1.7047*\dy})
	-- ({1.2325*\dx},{-1.6807*\dy})
	-- ({1.1901*\dx},{-1.6563*\dy})
	-- ({1.1496*\dx},{-1.6318*\dy})
	-- ({1.1112*\dx},{-1.6071*\dy})
	-- ({1.0746*\dx},{-1.5825*\dy})
	-- ({1.0397*\dx},{-1.5578*\dy})
	-- ({1.0065*\dx},{-1.5332*\dy})
	-- ({0.9748*\dx},{-1.5087*\dy})
	-- ({0.9446*\dx},{-1.4843*\dy})
	-- ({0.9158*\dx},{-1.4601*\dy})
	-- ({0.8884*\dx},{-1.4361*\dy})
	-- ({0.8622*\dx},{-1.4124*\dy})
	-- ({0.8372*\dx},{-1.3888*\dy})
	-- ({0.8133*\dx},{-1.3655*\dy})
	-- ({0.7905*\dx},{-1.3425*\dy})
	-- ({0.7687*\dx},{-1.3198*\dy})
	-- ({0.7478*\dx},{-1.2973*\dy})
	-- ({0.7279*\dx},{-1.2752*\dy})
	-- ({0.7088*\dx},{-1.2533*\dy})
	-- ({0.6905*\dx},{-1.2317*\dy})
	-- ({0.6731*\dx},{-1.2104*\dy})
	-- ({0.6563*\dx},{-1.1894*\dy})
	-- ({0.6402*\dx},{-1.1687*\dy})
	-- ({0.6248*\dx},{-1.1483*\dy})
	-- ({0.6101*\dx},{-1.1282*\dy})
	-- ({0.5959*\dx},{-1.1084*\dy})
	-- ({0.5823*\dx},{-1.0888*\dy})
	-- ({0.5692*\dx},{-1.0696*\dy})
	-- ({0.5567*\dx},{-1.0506*\dy})
	-- ({0.5447*\dx},{-1.0319*\dy})
	-- ({0.5331*\dx},{-1.0134*\dy})
	-- ({0.5219*\dx},{-0.9952*\dy})
	-- ({0.5112*\dx},{-0.9773*\dy})
	-- ({0.5009*\dx},{-0.9597*\dy})
	-- ({0.4910*\dx},{-0.9422*\dy})
	-- ({0.4815*\dx},{-0.9250*\dy})
	-- ({0.4723*\dx},{-0.9081*\dy})
	-- ({0.4635*\dx},{-0.8914*\dy})
	-- ({0.4549*\dx},{-0.8749*\dy})
	-- ({0.4467*\dx},{-0.8586*\dy})
	-- ({0.4388*\dx},{-0.8426*\dy})
	-- ({0.4312*\dx},{-0.8267*\dy})
	-- ({0.4238*\dx},{-0.8111*\dy})
	-- ({0.4167*\dx},{-0.7957*\dy})
	-- ({0.4099*\dx},{-0.7804*\dy})
	-- ({0.4033*\dx},{-0.7654*\dy})
	-- ({0.3969*\dx},{-0.7505*\dy})
	-- ({0.3908*\dx},{-0.7359*\dy})
	-- ({0.3848*\dx},{-0.7213*\dy})
	-- ({0.3791*\dx},{-0.7070*\dy})
	-- ({0.3736*\dx},{-0.6928*\dy})
	-- ({0.3682*\dx},{-0.6788*\dy})
	-- ({0.3630*\dx},{-0.6650*\dy})
	-- ({0.3581*\dx},{-0.6513*\dy})
	-- ({0.3532*\dx},{-0.6378*\dy})
	-- ({0.3486*\dx},{-0.6244*\dy})
	-- ({0.3441*\dx},{-0.6111*\dy})
	-- ({0.3398*\dx},{-0.5980*\dy})
	-- ({0.3356*\dx},{-0.5850*\dy})
	-- ({0.3315*\dx},{-0.5722*\dy})
	-- ({0.3276*\dx},{-0.5594*\dy})
	-- ({0.3238*\dx},{-0.5468*\dy})
	-- ({0.3201*\dx},{-0.5343*\dy})
	-- ({0.3166*\dx},{-0.5220*\dy})
	-- ({0.3132*\dx},{-0.5097*\dy})
	-- ({0.3099*\dx},{-0.4975*\dy})
	-- ({0.3067*\dx},{-0.4855*\dy})
	-- ({0.3036*\dx},{-0.4735*\dy})
	-- ({0.3007*\dx},{-0.4617*\dy})
	-- ({0.2978*\dx},{-0.4500*\dy})
	-- ({0.2950*\dx},{-0.4383*\dy})
	-- ({0.2924*\dx},{-0.4267*\dy})
	-- ({0.2898*\dx},{-0.4153*\dy})
	-- ({0.2873*\dx},{-0.4039*\dy})
	-- ({0.2849*\dx},{-0.3925*\dy})
	-- ({0.2826*\dx},{-0.3813*\dy})
	-- ({0.2804*\dx},{-0.3702*\dy})
	-- ({0.2782*\dx},{-0.3591*\dy})
	-- ({0.2762*\dx},{-0.3481*\dy})
	-- ({0.2742*\dx},{-0.3371*\dy})
	-- ({0.2723*\dx},{-0.3262*\dy})
	-- ({0.2705*\dx},{-0.3154*\dy})
	-- ({0.2687*\dx},{-0.3047*\dy})
	-- ({0.2670*\dx},{-0.2940*\dy})
	-- ({0.2654*\dx},{-0.2833*\dy})
	-- ({0.2639*\dx},{-0.2727*\dy})
	-- ({0.2624*\dx},{-0.2622*\dy})
	-- ({0.2610*\dx},{-0.2517*\dy})
	-- ({0.2596*\dx},{-0.2413*\dy})
	-- ({0.2583*\dx},{-0.2309*\dy})
	-- ({0.2571*\dx},{-0.2206*\dy})
	-- ({0.2560*\dx},{-0.2103*\dy})
	-- ({0.2549*\dx},{-0.2000*\dy})
	-- ({0.2538*\dx},{-0.1898*\dy})
	-- ({0.2528*\dx},{-0.1796*\dy})
	-- ({0.2519*\dx},{-0.1694*\dy})
	-- ({0.2510*\dx},{-0.1593*\dy})
	-- ({0.2502*\dx},{-0.1492*\dy})
	-- ({0.2495*\dx},{-0.1391*\dy})
	-- ({0.2488*\dx},{-0.1291*\dy})
	-- ({0.2481*\dx},{-0.1191*\dy})
	-- ({0.2475*\dx},{-0.1091*\dy})
	-- ({0.2470*\dx},{-0.0991*\dy})
	-- ({0.2465*\dx},{-0.0891*\dy})
	-- ({0.2461*\dx},{-0.0792*\dy})
	-- ({0.2457*\dx},{-0.0693*\dy})
	-- ({0.2453*\dx},{-0.0593*\dy})
	-- ({0.2451*\dx},{-0.0494*\dy})
	-- ({0.2448*\dx},{-0.0395*\dy})
	-- ({0.2447*\dx},{-0.0296*\dy})
	-- ({0.2445*\dx},{-0.0198*\dy})
	-- ({0.2445*\dx},{-0.0099*\dy})
	-- ({0.2444*\dx},{-0.0000*\dy})
}
% u = 0.415799
\def\upathL{
	({0.3934*\dx},{0.0000*\dy})
	-- ({0.3934*\dx},{0.0089*\dy})
	-- ({0.3935*\dx},{0.0178*\dy})
	-- ({0.3937*\dx},{0.0266*\dy})
	-- ({0.3940*\dx},{0.0355*\dy})
	-- ({0.3943*\dx},{0.0444*\dy})
	-- ({0.3947*\dx},{0.0533*\dy})
	-- ({0.3952*\dx},{0.0622*\dy})
	-- ({0.3957*\dx},{0.0711*\dy})
	-- ({0.3964*\dx},{0.0801*\dy})
	-- ({0.3971*\dx},{0.0890*\dy})
	-- ({0.3979*\dx},{0.0979*\dy})
	-- ({0.3987*\dx},{0.1069*\dy})
	-- ({0.3996*\dx},{0.1158*\dy})
	-- ({0.4007*\dx},{0.1248*\dy})
	-- ({0.4018*\dx},{0.1338*\dy})
	-- ({0.4029*\dx},{0.1428*\dy})
	-- ({0.4042*\dx},{0.1518*\dy})
	-- ({0.4055*\dx},{0.1609*\dy})
	-- ({0.4069*\dx},{0.1699*\dy})
	-- ({0.4084*\dx},{0.1790*\dy})
	-- ({0.4100*\dx},{0.1881*\dy})
	-- ({0.4116*\dx},{0.1972*\dy})
	-- ({0.4134*\dx},{0.2064*\dy})
	-- ({0.4152*\dx},{0.2156*\dy})
	-- ({0.4172*\dx},{0.2248*\dy})
	-- ({0.4192*\dx},{0.2340*\dy})
	-- ({0.4213*\dx},{0.2432*\dy})
	-- ({0.4235*\dx},{0.2525*\dy})
	-- ({0.4258*\dx},{0.2618*\dy})
	-- ({0.4282*\dx},{0.2712*\dy})
	-- ({0.4307*\dx},{0.2805*\dy})
	-- ({0.4332*\dx},{0.2899*\dy})
	-- ({0.4359*\dx},{0.2994*\dy})
	-- ({0.4387*\dx},{0.3088*\dy})
	-- ({0.4416*\dx},{0.3183*\dy})
	-- ({0.4446*\dx},{0.3279*\dy})
	-- ({0.4478*\dx},{0.3375*\dy})
	-- ({0.4510*\dx},{0.3471*\dy})
	-- ({0.4543*\dx},{0.3567*\dy})
	-- ({0.4578*\dx},{0.3664*\dy})
	-- ({0.4614*\dx},{0.3762*\dy})
	-- ({0.4651*\dx},{0.3859*\dy})
	-- ({0.4689*\dx},{0.3958*\dy})
	-- ({0.4729*\dx},{0.4056*\dy})
	-- ({0.4770*\dx},{0.4155*\dy})
	-- ({0.4812*\dx},{0.4255*\dy})
	-- ({0.4856*\dx},{0.4355*\dy})
	-- ({0.4901*\dx},{0.4455*\dy})
	-- ({0.4948*\dx},{0.4556*\dy})
	-- ({0.4996*\dx},{0.4658*\dy})
	-- ({0.5046*\dx},{0.4759*\dy})
	-- ({0.5097*\dx},{0.4862*\dy})
	-- ({0.5150*\dx},{0.4965*\dy})
	-- ({0.5204*\dx},{0.5068*\dy})
	-- ({0.5261*\dx},{0.5172*\dy})
	-- ({0.5319*\dx},{0.5276*\dy})
	-- ({0.5379*\dx},{0.5381*\dy})
	-- ({0.5440*\dx},{0.5486*\dy})
	-- ({0.5504*\dx},{0.5592*\dy})
	-- ({0.5570*\dx},{0.5699*\dy})
	-- ({0.5638*\dx},{0.5805*\dy})
	-- ({0.5707*\dx},{0.5913*\dy})
	-- ({0.5779*\dx},{0.6021*\dy})
	-- ({0.5854*\dx},{0.6129*\dy})
	-- ({0.5930*\dx},{0.6238*\dy})
	-- ({0.6009*\dx},{0.6347*\dy})
	-- ({0.6090*\dx},{0.6457*\dy})
	-- ({0.6174*\dx},{0.6567*\dy})
	-- ({0.6261*\dx},{0.6677*\dy})
	-- ({0.6350*\dx},{0.6788*\dy})
	-- ({0.6442*\dx},{0.6899*\dy})
	-- ({0.6537*\dx},{0.7011*\dy})
	-- ({0.6635*\dx},{0.7123*\dy})
	-- ({0.6736*\dx},{0.7235*\dy})
	-- ({0.6840*\dx},{0.7348*\dy})
	-- ({0.6947*\dx},{0.7461*\dy})
	-- ({0.7057*\dx},{0.7574*\dy})
	-- ({0.7172*\dx},{0.7687*\dy})
	-- ({0.7289*\dx},{0.7800*\dy})
	-- ({0.7411*\dx},{0.7913*\dy})
	-- ({0.7536*\dx},{0.8026*\dy})
	-- ({0.7665*\dx},{0.8139*\dy})
	-- ({0.7798*\dx},{0.8252*\dy})
	-- ({0.7935*\dx},{0.8365*\dy})
	-- ({0.8077*\dx},{0.8477*\dy})
	-- ({0.8223*\dx},{0.8589*\dy})
	-- ({0.8374*\dx},{0.8700*\dy})
	-- ({0.8529*\dx},{0.8810*\dy})
	-- ({0.8690*\dx},{0.8920*\dy})
	-- ({0.8855*\dx},{0.9029*\dy})
	-- ({0.9026*\dx},{0.9137*\dy})
	-- ({0.9202*\dx},{0.9243*\dy})
	-- ({0.9383*\dx},{0.9348*\dy})
	-- ({0.9570*\dx},{0.9452*\dy})
	-- ({0.9764*\dx},{0.9554*\dy})
	-- ({0.9963*\dx},{0.9654*\dy})
	-- ({1.0168*\dx},{0.9751*\dy})
	-- ({1.0379*\dx},{0.9846*\dy})
	-- ({1.0597*\dx},{0.9938*\dy})
	-- ({1.0822*\dx},{1.0028*\dy})
	-- ({1.1053*\dx},{1.0114*\dy})
	-- ({1.1292*\dx},{1.0196*\dy})
	-- ({1.1537*\dx},{1.0274*\dy})
	-- ({1.1790*\dx},{1.0348*\dy})
	-- ({1.2050*\dx},{1.0417*\dy})
	-- ({1.2318*\dx},{1.0481*\dy})
	-- ({1.2593*\dx},{1.0539*\dy})
	-- ({1.2876*\dx},{1.0591*\dy})
	-- ({1.3166*\dx},{1.0636*\dy})
	-- ({1.3464*\dx},{1.0675*\dy})
	-- ({1.3770*\dx},{1.0705*\dy})
	-- ({1.4084*\dx},{1.0727*\dy})
	-- ({1.4405*\dx},{1.0740*\dy})
	-- ({1.4734*\dx},{1.0743*\dy})
	-- ({1.5070*\dx},{1.0736*\dy})
	-- ({1.5414*\dx},{1.0718*\dy})
	-- ({1.5764*\dx},{1.0688*\dy})
	-- ({1.6122*\dx},{1.0646*\dy})
	-- ({1.6485*\dx},{1.0590*\dy})
	-- ({1.6855*\dx},{1.0520*\dy})
	-- ({1.7231*\dx},{1.0435*\dy})
	-- ({1.7611*\dx},{1.0335*\dy})
	-- ({1.7995*\dx},{1.0218*\dy})
	-- ({1.8383*\dx},{1.0084*\dy})
	-- ({1.8774*\dx},{0.9931*\dy})
	-- ({1.9167*\dx},{0.9760*\dy})
	-- ({1.9560*\dx},{0.9570*\dy})
	-- ({1.9953*\dx},{0.9359*\dy})
	-- ({2.0344*\dx},{0.9127*\dy})
	-- ({2.0733*\dx},{0.8874*\dy})
	-- ({2.1117*\dx},{0.8599*\dy})
	-- ({2.1495*\dx},{0.8302*\dy})
	-- ({2.1867*\dx},{0.7983*\dy})
	-- ({2.2229*\dx},{0.7642*\dy})
	-- ({2.2580*\dx},{0.7278*\dy})
	-- ({2.2919*\dx},{0.6892*\dy})
	-- ({2.3243*\dx},{0.6484*\dy})
	-- ({2.3551*\dx},{0.6055*\dy})
	-- ({2.3842*\dx},{0.5606*\dy})
	-- ({2.4113*\dx},{0.5137*\dy})
	-- ({2.4362*\dx},{0.4650*\dy})
	-- ({2.4588*\dx},{0.4146*\dy})
	-- ({2.4790*\dx},{0.3626*\dy})
	-- ({2.4965*\dx},{0.3093*\dy})
	-- ({2.5114*\dx},{0.2548*\dy})
	-- ({2.5234*\dx},{0.1992*\dy})
	-- ({2.5325*\dx},{0.1429*\dy})
	-- ({2.5386*\dx},{0.0860*\dy})
	-- ({2.5417*\dx},{0.0287*\dy})
	-- ({2.5417*\dx},{-0.0287*\dy})
	-- ({2.5386*\dx},{-0.0860*\dy})
	-- ({2.5325*\dx},{-0.1429*\dy})
	-- ({2.5234*\dx},{-0.1992*\dy})
	-- ({2.5114*\dx},{-0.2548*\dy})
	-- ({2.4965*\dx},{-0.3093*\dy})
	-- ({2.4790*\dx},{-0.3626*\dy})
	-- ({2.4588*\dx},{-0.4146*\dy})
	-- ({2.4362*\dx},{-0.4650*\dy})
	-- ({2.4113*\dx},{-0.5137*\dy})
	-- ({2.3842*\dx},{-0.5606*\dy})
	-- ({2.3551*\dx},{-0.6055*\dy})
	-- ({2.3243*\dx},{-0.6484*\dy})
	-- ({2.2919*\dx},{-0.6892*\dy})
	-- ({2.2580*\dx},{-0.7278*\dy})
	-- ({2.2229*\dx},{-0.7642*\dy})
	-- ({2.1867*\dx},{-0.7983*\dy})
	-- ({2.1495*\dx},{-0.8302*\dy})
	-- ({2.1117*\dx},{-0.8599*\dy})
	-- ({2.0733*\dx},{-0.8874*\dy})
	-- ({2.0344*\dx},{-0.9127*\dy})
	-- ({1.9953*\dx},{-0.9359*\dy})
	-- ({1.9560*\dx},{-0.9570*\dy})
	-- ({1.9167*\dx},{-0.9760*\dy})
	-- ({1.8774*\dx},{-0.9931*\dy})
	-- ({1.8383*\dx},{-1.0084*\dy})
	-- ({1.7995*\dx},{-1.0218*\dy})
	-- ({1.7611*\dx},{-1.0335*\dy})
	-- ({1.7231*\dx},{-1.0435*\dy})
	-- ({1.6855*\dx},{-1.0520*\dy})
	-- ({1.6485*\dx},{-1.0590*\dy})
	-- ({1.6122*\dx},{-1.0646*\dy})
	-- ({1.5764*\dx},{-1.0688*\dy})
	-- ({1.5414*\dx},{-1.0718*\dy})
	-- ({1.5070*\dx},{-1.0736*\dy})
	-- ({1.4734*\dx},{-1.0743*\dy})
	-- ({1.4405*\dx},{-1.0740*\dy})
	-- ({1.4084*\dx},{-1.0727*\dy})
	-- ({1.3770*\dx},{-1.0705*\dy})
	-- ({1.3464*\dx},{-1.0675*\dy})
	-- ({1.3166*\dx},{-1.0636*\dy})
	-- ({1.2876*\dx},{-1.0591*\dy})
	-- ({1.2593*\dx},{-1.0539*\dy})
	-- ({1.2318*\dx},{-1.0481*\dy})
	-- ({1.2050*\dx},{-1.0417*\dy})
	-- ({1.1790*\dx},{-1.0348*\dy})
	-- ({1.1537*\dx},{-1.0274*\dy})
	-- ({1.1292*\dx},{-1.0196*\dy})
	-- ({1.1053*\dx},{-1.0114*\dy})
	-- ({1.0822*\dx},{-1.0028*\dy})
	-- ({1.0597*\dx},{-0.9938*\dy})
	-- ({1.0379*\dx},{-0.9846*\dy})
	-- ({1.0168*\dx},{-0.9751*\dy})
	-- ({0.9963*\dx},{-0.9654*\dy})
	-- ({0.9764*\dx},{-0.9554*\dy})
	-- ({0.9570*\dx},{-0.9452*\dy})
	-- ({0.9383*\dx},{-0.9348*\dy})
	-- ({0.9202*\dx},{-0.9243*\dy})
	-- ({0.9026*\dx},{-0.9137*\dy})
	-- ({0.8855*\dx},{-0.9029*\dy})
	-- ({0.8690*\dx},{-0.8920*\dy})
	-- ({0.8529*\dx},{-0.8810*\dy})
	-- ({0.8374*\dx},{-0.8700*\dy})
	-- ({0.8223*\dx},{-0.8589*\dy})
	-- ({0.8077*\dx},{-0.8477*\dy})
	-- ({0.7935*\dx},{-0.8365*\dy})
	-- ({0.7798*\dx},{-0.8252*\dy})
	-- ({0.7665*\dx},{-0.8139*\dy})
	-- ({0.7536*\dx},{-0.8026*\dy})
	-- ({0.7411*\dx},{-0.7913*\dy})
	-- ({0.7289*\dx},{-0.7800*\dy})
	-- ({0.7172*\dx},{-0.7687*\dy})
	-- ({0.7057*\dx},{-0.7574*\dy})
	-- ({0.6947*\dx},{-0.7461*\dy})
	-- ({0.6840*\dx},{-0.7348*\dy})
	-- ({0.6736*\dx},{-0.7235*\dy})
	-- ({0.6635*\dx},{-0.7123*\dy})
	-- ({0.6537*\dx},{-0.7011*\dy})
	-- ({0.6442*\dx},{-0.6899*\dy})
	-- ({0.6350*\dx},{-0.6788*\dy})
	-- ({0.6261*\dx},{-0.6677*\dy})
	-- ({0.6174*\dx},{-0.6567*\dy})
	-- ({0.6090*\dx},{-0.6457*\dy})
	-- ({0.6009*\dx},{-0.6347*\dy})
	-- ({0.5930*\dx},{-0.6238*\dy})
	-- ({0.5854*\dx},{-0.6129*\dy})
	-- ({0.5779*\dx},{-0.6021*\dy})
	-- ({0.5707*\dx},{-0.5913*\dy})
	-- ({0.5638*\dx},{-0.5805*\dy})
	-- ({0.5570*\dx},{-0.5699*\dy})
	-- ({0.5504*\dx},{-0.5592*\dy})
	-- ({0.5440*\dx},{-0.5486*\dy})
	-- ({0.5379*\dx},{-0.5381*\dy})
	-- ({0.5319*\dx},{-0.5276*\dy})
	-- ({0.5261*\dx},{-0.5172*\dy})
	-- ({0.5204*\dx},{-0.5068*\dy})
	-- ({0.5150*\dx},{-0.4965*\dy})
	-- ({0.5097*\dx},{-0.4862*\dy})
	-- ({0.5046*\dx},{-0.4759*\dy})
	-- ({0.4996*\dx},{-0.4658*\dy})
	-- ({0.4948*\dx},{-0.4556*\dy})
	-- ({0.4901*\dx},{-0.4455*\dy})
	-- ({0.4856*\dx},{-0.4355*\dy})
	-- ({0.4812*\dx},{-0.4255*\dy})
	-- ({0.4770*\dx},{-0.4155*\dy})
	-- ({0.4729*\dx},{-0.4056*\dy})
	-- ({0.4689*\dx},{-0.3958*\dy})
	-- ({0.4651*\dx},{-0.3859*\dy})
	-- ({0.4614*\dx},{-0.3762*\dy})
	-- ({0.4578*\dx},{-0.3664*\dy})
	-- ({0.4543*\dx},{-0.3567*\dy})
	-- ({0.4510*\dx},{-0.3471*\dy})
	-- ({0.4478*\dx},{-0.3375*\dy})
	-- ({0.4446*\dx},{-0.3279*\dy})
	-- ({0.4416*\dx},{-0.3183*\dy})
	-- ({0.4387*\dx},{-0.3088*\dy})
	-- ({0.4359*\dx},{-0.2994*\dy})
	-- ({0.4332*\dx},{-0.2899*\dy})
	-- ({0.4307*\dx},{-0.2805*\dy})
	-- ({0.4282*\dx},{-0.2712*\dy})
	-- ({0.4258*\dx},{-0.2618*\dy})
	-- ({0.4235*\dx},{-0.2525*\dy})
	-- ({0.4213*\dx},{-0.2432*\dy})
	-- ({0.4192*\dx},{-0.2340*\dy})
	-- ({0.4172*\dx},{-0.2248*\dy})
	-- ({0.4152*\dx},{-0.2156*\dy})
	-- ({0.4134*\dx},{-0.2064*\dy})
	-- ({0.4116*\dx},{-0.1972*\dy})
	-- ({0.4100*\dx},{-0.1881*\dy})
	-- ({0.4084*\dx},{-0.1790*\dy})
	-- ({0.4069*\dx},{-0.1699*\dy})
	-- ({0.4055*\dx},{-0.1609*\dy})
	-- ({0.4042*\dx},{-0.1518*\dy})
	-- ({0.4029*\dx},{-0.1428*\dy})
	-- ({0.4018*\dx},{-0.1338*\dy})
	-- ({0.4007*\dx},{-0.1248*\dy})
	-- ({0.3996*\dx},{-0.1158*\dy})
	-- ({0.3987*\dx},{-0.1069*\dy})
	-- ({0.3979*\dx},{-0.0979*\dy})
	-- ({0.3971*\dx},{-0.0890*\dy})
	-- ({0.3964*\dx},{-0.0801*\dy})
	-- ({0.3957*\dx},{-0.0711*\dy})
	-- ({0.3952*\dx},{-0.0622*\dy})
	-- ({0.3947*\dx},{-0.0533*\dy})
	-- ({0.3943*\dx},{-0.0444*\dy})
	-- ({0.3940*\dx},{-0.0355*\dy})
	-- ({0.3937*\dx},{-0.0266*\dy})
	-- ({0.3935*\dx},{-0.0178*\dy})
	-- ({0.3934*\dx},{-0.0089*\dy})
	-- ({0.3934*\dx},{-0.0000*\dy})
}
% u = 0.582119
\def\upathM{
	({0.5242*\dx},{0.0000*\dy})
	-- ({0.5242*\dx},{0.0076*\dy})
	-- ({0.5244*\dx},{0.0152*\dy})
	-- ({0.5246*\dx},{0.0229*\dy})
	-- ({0.5249*\dx},{0.0305*\dy})
	-- ({0.5253*\dx},{0.0381*\dy})
	-- ({0.5257*\dx},{0.0457*\dy})
	-- ({0.5263*\dx},{0.0534*\dy})
	-- ({0.5269*\dx},{0.0610*\dy})
	-- ({0.5276*\dx},{0.0686*\dy})
	-- ({0.5284*\dx},{0.0762*\dy})
	-- ({0.5293*\dx},{0.0839*\dy})
	-- ({0.5303*\dx},{0.0915*\dy})
	-- ({0.5313*\dx},{0.0992*\dy})
	-- ({0.5325*\dx},{0.1068*\dy})
	-- ({0.5337*\dx},{0.1145*\dy})
	-- ({0.5351*\dx},{0.1221*\dy})
	-- ({0.5365*\dx},{0.1298*\dy})
	-- ({0.5380*\dx},{0.1374*\dy})
	-- ({0.5396*\dx},{0.1451*\dy})
	-- ({0.5413*\dx},{0.1528*\dy})
	-- ({0.5431*\dx},{0.1604*\dy})
	-- ({0.5449*\dx},{0.1681*\dy})
	-- ({0.5469*\dx},{0.1758*\dy})
	-- ({0.5490*\dx},{0.1835*\dy})
	-- ({0.5512*\dx},{0.1912*\dy})
	-- ({0.5534*\dx},{0.1989*\dy})
	-- ({0.5558*\dx},{0.2066*\dy})
	-- ({0.5582*\dx},{0.2143*\dy})
	-- ({0.5608*\dx},{0.2220*\dy})
	-- ({0.5635*\dx},{0.2298*\dy})
	-- ({0.5663*\dx},{0.2375*\dy})
	-- ({0.5691*\dx},{0.2452*\dy})
	-- ({0.5721*\dx},{0.2530*\dy})
	-- ({0.5752*\dx},{0.2607*\dy})
	-- ({0.5784*\dx},{0.2685*\dy})
	-- ({0.5817*\dx},{0.2762*\dy})
	-- ({0.5852*\dx},{0.2840*\dy})
	-- ({0.5887*\dx},{0.2917*\dy})
	-- ({0.5924*\dx},{0.2995*\dy})
	-- ({0.5962*\dx},{0.3072*\dy})
	-- ({0.6001*\dx},{0.3150*\dy})
	-- ({0.6041*\dx},{0.3228*\dy})
	-- ({0.6083*\dx},{0.3305*\dy})
	-- ({0.6126*\dx},{0.3383*\dy})
	-- ({0.6170*\dx},{0.3461*\dy})
	-- ({0.6216*\dx},{0.3538*\dy})
	-- ({0.6262*\dx},{0.3616*\dy})
	-- ({0.6311*\dx},{0.3694*\dy})
	-- ({0.6360*\dx},{0.3771*\dy})
	-- ({0.6412*\dx},{0.3849*\dy})
	-- ({0.6464*\dx},{0.3926*\dy})
	-- ({0.6518*\dx},{0.4003*\dy})
	-- ({0.6574*\dx},{0.4081*\dy})
	-- ({0.6631*\dx},{0.4158*\dy})
	-- ({0.6690*\dx},{0.4235*\dy})
	-- ({0.6750*\dx},{0.4311*\dy})
	-- ({0.6812*\dx},{0.4388*\dy})
	-- ({0.6876*\dx},{0.4464*\dy})
	-- ({0.6941*\dx},{0.4541*\dy})
	-- ({0.7008*\dx},{0.4617*\dy})
	-- ({0.7077*\dx},{0.4692*\dy})
	-- ({0.7148*\dx},{0.4768*\dy})
	-- ({0.7220*\dx},{0.4843*\dy})
	-- ({0.7295*\dx},{0.4918*\dy})
	-- ({0.7371*\dx},{0.4992*\dy})
	-- ({0.7449*\dx},{0.5066*\dy})
	-- ({0.7530*\dx},{0.5140*\dy})
	-- ({0.7612*\dx},{0.5213*\dy})
	-- ({0.7696*\dx},{0.5285*\dy})
	-- ({0.7783*\dx},{0.5357*\dy})
	-- ({0.7872*\dx},{0.5428*\dy})
	-- ({0.7963*\dx},{0.5499*\dy})
	-- ({0.8056*\dx},{0.5569*\dy})
	-- ({0.8151*\dx},{0.5638*\dy})
	-- ({0.8249*\dx},{0.5706*\dy})
	-- ({0.8349*\dx},{0.5773*\dy})
	-- ({0.8451*\dx},{0.5840*\dy})
	-- ({0.8556*\dx},{0.5905*\dy})
	-- ({0.8664*\dx},{0.5969*\dy})
	-- ({0.8774*\dx},{0.6032*\dy})
	-- ({0.8886*\dx},{0.6094*\dy})
	-- ({0.9001*\dx},{0.6154*\dy})
	-- ({0.9119*\dx},{0.6213*\dy})
	-- ({0.9239*\dx},{0.6271*\dy})
	-- ({0.9363*\dx},{0.6327*\dy})
	-- ({0.9488*\dx},{0.6381*\dy})
	-- ({0.9617*\dx},{0.6433*\dy})
	-- ({0.9748*\dx},{0.6484*\dy})
	-- ({0.9883*\dx},{0.6532*\dy})
	-- ({1.0020*\dx},{0.6578*\dy})
	-- ({1.0160*\dx},{0.6622*\dy})
	-- ({1.0302*\dx},{0.6663*\dy})
	-- ({1.0448*\dx},{0.6702*\dy})
	-- ({1.0597*\dx},{0.6738*\dy})
	-- ({1.0748*\dx},{0.6772*\dy})
	-- ({1.0903*\dx},{0.6802*\dy})
	-- ({1.1060*\dx},{0.6829*\dy})
	-- ({1.1220*\dx},{0.6853*\dy})
	-- ({1.1383*\dx},{0.6874*\dy})
	-- ({1.1549*\dx},{0.6890*\dy})
	-- ({1.1718*\dx},{0.6903*\dy})
	-- ({1.1889*\dx},{0.6912*\dy})
	-- ({1.2063*\dx},{0.6917*\dy})
	-- ({1.2240*\dx},{0.6917*\dy})
	-- ({1.2419*\dx},{0.6912*\dy})
	-- ({1.2600*\dx},{0.6903*\dy})
	-- ({1.2784*\dx},{0.6889*\dy})
	-- ({1.2970*\dx},{0.6870*\dy})
	-- ({1.3159*\dx},{0.6845*\dy})
	-- ({1.3349*\dx},{0.6814*\dy})
	-- ({1.3541*\dx},{0.6778*\dy})
	-- ({1.3735*\dx},{0.6736*\dy})
	-- ({1.3930*\dx},{0.6687*\dy})
	-- ({1.4126*\dx},{0.6632*\dy})
	-- ({1.4323*\dx},{0.6570*\dy})
	-- ({1.4521*\dx},{0.6501*\dy})
	-- ({1.4720*\dx},{0.6426*\dy})
	-- ({1.4919*\dx},{0.6343*\dy})
	-- ({1.5118*\dx},{0.6253*\dy})
	-- ({1.5316*\dx},{0.6155*\dy})
	-- ({1.5514*\dx},{0.6050*\dy})
	-- ({1.5710*\dx},{0.5936*\dy})
	-- ({1.5905*\dx},{0.5815*\dy})
	-- ({1.6099*\dx},{0.5686*\dy})
	-- ({1.6290*\dx},{0.5548*\dy})
	-- ({1.6479*\dx},{0.5403*\dy})
	-- ({1.6664*\dx},{0.5249*\dy})
	-- ({1.6846*\dx},{0.5087*\dy})
	-- ({1.7024*\dx},{0.4918*\dy})
	-- ({1.7198*\dx},{0.4739*\dy})
	-- ({1.7367*\dx},{0.4553*\dy})
	-- ({1.7530*\dx},{0.4359*\dy})
	-- ({1.7688*\dx},{0.4158*\dy})
	-- ({1.7839*\dx},{0.3949*\dy})
	-- ({1.7983*\dx},{0.3732*\dy})
	-- ({1.8121*\dx},{0.3508*\dy})
	-- ({1.8250*\dx},{0.3278*\dy})
	-- ({1.8372*\dx},{0.3041*\dy})
	-- ({1.8485*\dx},{0.2799*\dy})
	-- ({1.8589*\dx},{0.2550*\dy})
	-- ({1.8684*\dx},{0.2296*\dy})
	-- ({1.8770*\dx},{0.2038*\dy})
	-- ({1.8845*\dx},{0.1775*\dy})
	-- ({1.8910*\dx},{0.1508*\dy})
	-- ({1.8965*\dx},{0.1239*\dy})
	-- ({1.9009*\dx},{0.0966*\dy})
	-- ({1.9042*\dx},{0.0692*\dy})
	-- ({1.9064*\dx},{0.0416*\dy})
	-- ({1.9075*\dx},{0.0139*\dy})
	-- ({1.9075*\dx},{-0.0139*\dy})
	-- ({1.9064*\dx},{-0.0416*\dy})
	-- ({1.9042*\dx},{-0.0692*\dy})
	-- ({1.9009*\dx},{-0.0966*\dy})
	-- ({1.8965*\dx},{-0.1239*\dy})
	-- ({1.8910*\dx},{-0.1508*\dy})
	-- ({1.8845*\dx},{-0.1775*\dy})
	-- ({1.8770*\dx},{-0.2038*\dy})
	-- ({1.8684*\dx},{-0.2296*\dy})
	-- ({1.8589*\dx},{-0.2550*\dy})
	-- ({1.8485*\dx},{-0.2799*\dy})
	-- ({1.8372*\dx},{-0.3041*\dy})
	-- ({1.8250*\dx},{-0.3278*\dy})
	-- ({1.8121*\dx},{-0.3508*\dy})
	-- ({1.7983*\dx},{-0.3732*\dy})
	-- ({1.7839*\dx},{-0.3949*\dy})
	-- ({1.7688*\dx},{-0.4158*\dy})
	-- ({1.7530*\dx},{-0.4359*\dy})
	-- ({1.7367*\dx},{-0.4553*\dy})
	-- ({1.7198*\dx},{-0.4739*\dy})
	-- ({1.7024*\dx},{-0.4918*\dy})
	-- ({1.6846*\dx},{-0.5087*\dy})
	-- ({1.6664*\dx},{-0.5249*\dy})
	-- ({1.6479*\dx},{-0.5403*\dy})
	-- ({1.6290*\dx},{-0.5548*\dy})
	-- ({1.6099*\dx},{-0.5686*\dy})
	-- ({1.5905*\dx},{-0.5815*\dy})
	-- ({1.5710*\dx},{-0.5936*\dy})
	-- ({1.5514*\dx},{-0.6050*\dy})
	-- ({1.5316*\dx},{-0.6155*\dy})
	-- ({1.5118*\dx},{-0.6253*\dy})
	-- ({1.4919*\dx},{-0.6343*\dy})
	-- ({1.4720*\dx},{-0.6426*\dy})
	-- ({1.4521*\dx},{-0.6501*\dy})
	-- ({1.4323*\dx},{-0.6570*\dy})
	-- ({1.4126*\dx},{-0.6632*\dy})
	-- ({1.3930*\dx},{-0.6687*\dy})
	-- ({1.3735*\dx},{-0.6736*\dy})
	-- ({1.3541*\dx},{-0.6778*\dy})
	-- ({1.3349*\dx},{-0.6814*\dy})
	-- ({1.3159*\dx},{-0.6845*\dy})
	-- ({1.2970*\dx},{-0.6870*\dy})
	-- ({1.2784*\dx},{-0.6889*\dy})
	-- ({1.2600*\dx},{-0.6903*\dy})
	-- ({1.2419*\dx},{-0.6912*\dy})
	-- ({1.2240*\dx},{-0.6917*\dy})
	-- ({1.2063*\dx},{-0.6917*\dy})
	-- ({1.1889*\dx},{-0.6912*\dy})
	-- ({1.1718*\dx},{-0.6903*\dy})
	-- ({1.1549*\dx},{-0.6890*\dy})
	-- ({1.1383*\dx},{-0.6874*\dy})
	-- ({1.1220*\dx},{-0.6853*\dy})
	-- ({1.1060*\dx},{-0.6829*\dy})
	-- ({1.0903*\dx},{-0.6802*\dy})
	-- ({1.0748*\dx},{-0.6772*\dy})
	-- ({1.0597*\dx},{-0.6738*\dy})
	-- ({1.0448*\dx},{-0.6702*\dy})
	-- ({1.0302*\dx},{-0.6663*\dy})
	-- ({1.0160*\dx},{-0.6622*\dy})
	-- ({1.0020*\dx},{-0.6578*\dy})
	-- ({0.9883*\dx},{-0.6532*\dy})
	-- ({0.9748*\dx},{-0.6484*\dy})
	-- ({0.9617*\dx},{-0.6433*\dy})
	-- ({0.9488*\dx},{-0.6381*\dy})
	-- ({0.9363*\dx},{-0.6327*\dy})
	-- ({0.9239*\dx},{-0.6271*\dy})
	-- ({0.9119*\dx},{-0.6213*\dy})
	-- ({0.9001*\dx},{-0.6154*\dy})
	-- ({0.8886*\dx},{-0.6094*\dy})
	-- ({0.8774*\dx},{-0.6032*\dy})
	-- ({0.8664*\dx},{-0.5969*\dy})
	-- ({0.8556*\dx},{-0.5905*\dy})
	-- ({0.8451*\dx},{-0.5840*\dy})
	-- ({0.8349*\dx},{-0.5773*\dy})
	-- ({0.8249*\dx},{-0.5706*\dy})
	-- ({0.8151*\dx},{-0.5638*\dy})
	-- ({0.8056*\dx},{-0.5569*\dy})
	-- ({0.7963*\dx},{-0.5499*\dy})
	-- ({0.7872*\dx},{-0.5428*\dy})
	-- ({0.7783*\dx},{-0.5357*\dy})
	-- ({0.7696*\dx},{-0.5285*\dy})
	-- ({0.7612*\dx},{-0.5213*\dy})
	-- ({0.7530*\dx},{-0.5140*\dy})
	-- ({0.7449*\dx},{-0.5066*\dy})
	-- ({0.7371*\dx},{-0.4992*\dy})
	-- ({0.7295*\dx},{-0.4918*\dy})
	-- ({0.7220*\dx},{-0.4843*\dy})
	-- ({0.7148*\dx},{-0.4768*\dy})
	-- ({0.7077*\dx},{-0.4692*\dy})
	-- ({0.7008*\dx},{-0.4617*\dy})
	-- ({0.6941*\dx},{-0.4541*\dy})
	-- ({0.6876*\dx},{-0.4464*\dy})
	-- ({0.6812*\dx},{-0.4388*\dy})
	-- ({0.6750*\dx},{-0.4311*\dy})
	-- ({0.6690*\dx},{-0.4235*\dy})
	-- ({0.6631*\dx},{-0.4158*\dy})
	-- ({0.6574*\dx},{-0.4081*\dy})
	-- ({0.6518*\dx},{-0.4003*\dy})
	-- ({0.6464*\dx},{-0.3926*\dy})
	-- ({0.6412*\dx},{-0.3849*\dy})
	-- ({0.6360*\dx},{-0.3771*\dy})
	-- ({0.6311*\dx},{-0.3694*\dy})
	-- ({0.6262*\dx},{-0.3616*\dy})
	-- ({0.6216*\dx},{-0.3538*\dy})
	-- ({0.6170*\dx},{-0.3461*\dy})
	-- ({0.6126*\dx},{-0.3383*\dy})
	-- ({0.6083*\dx},{-0.3305*\dy})
	-- ({0.6041*\dx},{-0.3228*\dy})
	-- ({0.6001*\dx},{-0.3150*\dy})
	-- ({0.5962*\dx},{-0.3072*\dy})
	-- ({0.5924*\dx},{-0.2995*\dy})
	-- ({0.5887*\dx},{-0.2917*\dy})
	-- ({0.5852*\dx},{-0.2840*\dy})
	-- ({0.5817*\dx},{-0.2762*\dy})
	-- ({0.5784*\dx},{-0.2685*\dy})
	-- ({0.5752*\dx},{-0.2607*\dy})
	-- ({0.5721*\dx},{-0.2530*\dy})
	-- ({0.5691*\dx},{-0.2452*\dy})
	-- ({0.5663*\dx},{-0.2375*\dy})
	-- ({0.5635*\dx},{-0.2298*\dy})
	-- ({0.5608*\dx},{-0.2220*\dy})
	-- ({0.5582*\dx},{-0.2143*\dy})
	-- ({0.5558*\dx},{-0.2066*\dy})
	-- ({0.5534*\dx},{-0.1989*\dy})
	-- ({0.5512*\dx},{-0.1912*\dy})
	-- ({0.5490*\dx},{-0.1835*\dy})
	-- ({0.5469*\dx},{-0.1758*\dy})
	-- ({0.5449*\dx},{-0.1681*\dy})
	-- ({0.5431*\dx},{-0.1604*\dy})
	-- ({0.5413*\dx},{-0.1528*\dy})
	-- ({0.5396*\dx},{-0.1451*\dy})
	-- ({0.5380*\dx},{-0.1374*\dy})
	-- ({0.5365*\dx},{-0.1298*\dy})
	-- ({0.5351*\dx},{-0.1221*\dy})
	-- ({0.5337*\dx},{-0.1145*\dy})
	-- ({0.5325*\dx},{-0.1068*\dy})
	-- ({0.5313*\dx},{-0.0992*\dy})
	-- ({0.5303*\dx},{-0.0915*\dy})
	-- ({0.5293*\dx},{-0.0839*\dy})
	-- ({0.5284*\dx},{-0.0762*\dy})
	-- ({0.5276*\dx},{-0.0686*\dy})
	-- ({0.5269*\dx},{-0.0610*\dy})
	-- ({0.5263*\dx},{-0.0534*\dy})
	-- ({0.5257*\dx},{-0.0457*\dy})
	-- ({0.5253*\dx},{-0.0381*\dy})
	-- ({0.5249*\dx},{-0.0305*\dy})
	-- ({0.5246*\dx},{-0.0229*\dy})
	-- ({0.5244*\dx},{-0.0152*\dy})
	-- ({0.5242*\dx},{-0.0076*\dy})
	-- ({0.5242*\dx},{-0.0000*\dy})
}
% u = 0.748438
\def\upathN{
	({0.6342*\dx},{0.0000*\dy})
	-- ({0.6343*\dx},{0.0063*\dy})
	-- ({0.6344*\dx},{0.0126*\dy})
	-- ({0.6346*\dx},{0.0188*\dy})
	-- ({0.6349*\dx},{0.0251*\dy})
	-- ({0.6353*\dx},{0.0314*\dy})
	-- ({0.6357*\dx},{0.0377*\dy})
	-- ({0.6363*\dx},{0.0439*\dy})
	-- ({0.6369*\dx},{0.0502*\dy})
	-- ({0.6376*\dx},{0.0565*\dy})
	-- ({0.6384*\dx},{0.0628*\dy})
	-- ({0.6393*\dx},{0.0690*\dy})
	-- ({0.6403*\dx},{0.0753*\dy})
	-- ({0.6413*\dx},{0.0815*\dy})
	-- ({0.6425*\dx},{0.0878*\dy})
	-- ({0.6437*\dx},{0.0940*\dy})
	-- ({0.6450*\dx},{0.1003*\dy})
	-- ({0.6464*\dx},{0.1065*\dy})
	-- ({0.6479*\dx},{0.1128*\dy})
	-- ({0.6495*\dx},{0.1190*\dy})
	-- ({0.6512*\dx},{0.1252*\dy})
	-- ({0.6529*\dx},{0.1314*\dy})
	-- ({0.6548*\dx},{0.1376*\dy})
	-- ({0.6567*\dx},{0.1438*\dy})
	-- ({0.6587*\dx},{0.1500*\dy})
	-- ({0.6609*\dx},{0.1562*\dy})
	-- ({0.6631*\dx},{0.1624*\dy})
	-- ({0.6654*\dx},{0.1685*\dy})
	-- ({0.6678*\dx},{0.1747*\dy})
	-- ({0.6703*\dx},{0.1808*\dy})
	-- ({0.6729*\dx},{0.1869*\dy})
	-- ({0.6756*\dx},{0.1930*\dy})
	-- ({0.6784*\dx},{0.1991*\dy})
	-- ({0.6812*\dx},{0.2052*\dy})
	-- ({0.6842*\dx},{0.2113*\dy})
	-- ({0.6873*\dx},{0.2173*\dy})
	-- ({0.6905*\dx},{0.2234*\dy})
	-- ({0.6938*\dx},{0.2294*\dy})
	-- ({0.6972*\dx},{0.2354*\dy})
	-- ({0.7007*\dx},{0.2413*\dy})
	-- ({0.7043*\dx},{0.2473*\dy})
	-- ({0.7080*\dx},{0.2532*\dy})
	-- ({0.7118*\dx},{0.2591*\dy})
	-- ({0.7158*\dx},{0.2650*\dy})
	-- ({0.7198*\dx},{0.2708*\dy})
	-- ({0.7240*\dx},{0.2767*\dy})
	-- ({0.7282*\dx},{0.2824*\dy})
	-- ({0.7326*\dx},{0.2882*\dy})
	-- ({0.7371*\dx},{0.2939*\dy})
	-- ({0.7417*\dx},{0.2996*\dy})
	-- ({0.7464*\dx},{0.3053*\dy})
	-- ({0.7513*\dx},{0.3109*\dy})
	-- ({0.7563*\dx},{0.3164*\dy})
	-- ({0.7614*\dx},{0.3220*\dy})
	-- ({0.7666*\dx},{0.3275*\dy})
	-- ({0.7719*\dx},{0.3329*\dy})
	-- ({0.7774*\dx},{0.3383*\dy})
	-- ({0.7830*\dx},{0.3436*\dy})
	-- ({0.7887*\dx},{0.3489*\dy})
	-- ({0.7945*\dx},{0.3541*\dy})
	-- ({0.8005*\dx},{0.3593*\dy})
	-- ({0.8066*\dx},{0.3644*\dy})
	-- ({0.8128*\dx},{0.3694*\dy})
	-- ({0.8192*\dx},{0.3744*\dy})
	-- ({0.8257*\dx},{0.3793*\dy})
	-- ({0.8324*\dx},{0.3841*\dy})
	-- ({0.8392*\dx},{0.3888*\dy})
	-- ({0.8461*\dx},{0.3935*\dy})
	-- ({0.8532*\dx},{0.3980*\dy})
	-- ({0.8604*\dx},{0.4025*\dy})
	-- ({0.8677*\dx},{0.4069*\dy})
	-- ({0.8752*\dx},{0.4112*\dy})
	-- ({0.8828*\dx},{0.4154*\dy})
	-- ({0.8906*\dx},{0.4194*\dy})
	-- ({0.8985*\dx},{0.4234*\dy})
	-- ({0.9066*\dx},{0.4273*\dy})
	-- ({0.9148*\dx},{0.4310*\dy})
	-- ({0.9232*\dx},{0.4346*\dy})
	-- ({0.9317*\dx},{0.4380*\dy})
	-- ({0.9403*\dx},{0.4414*\dy})
	-- ({0.9491*\dx},{0.4446*\dy})
	-- ({0.9581*\dx},{0.4476*\dy})
	-- ({0.9672*\dx},{0.4505*\dy})
	-- ({0.9764*\dx},{0.4532*\dy})
	-- ({0.9858*\dx},{0.4558*\dy})
	-- ({0.9953*\dx},{0.4582*\dy})
	-- ({1.0049*\dx},{0.4604*\dy})
	-- ({1.0147*\dx},{0.4624*\dy})
	-- ({1.0246*\dx},{0.4643*\dy})
	-- ({1.0347*\dx},{0.4659*\dy})
	-- ({1.0449*\dx},{0.4674*\dy})
	-- ({1.0552*\dx},{0.4686*\dy})
	-- ({1.0657*\dx},{0.4696*\dy})
	-- ({1.0763*\dx},{0.4704*\dy})
	-- ({1.0870*\dx},{0.4709*\dy})
	-- ({1.0978*\dx},{0.4712*\dy})
	-- ({1.1087*\dx},{0.4713*\dy})
	-- ({1.1197*\dx},{0.4711*\dy})
	-- ({1.1309*\dx},{0.4706*\dy})
	-- ({1.1421*\dx},{0.4698*\dy})
	-- ({1.1534*\dx},{0.4688*\dy})
	-- ({1.1648*\dx},{0.4675*\dy})
	-- ({1.1763*\dx},{0.4659*\dy})
	-- ({1.1878*\dx},{0.4640*\dy})
	-- ({1.1995*\dx},{0.4618*\dy})
	-- ({1.2111*\dx},{0.4593*\dy})
	-- ({1.2228*\dx},{0.4564*\dy})
	-- ({1.2346*\dx},{0.4532*\dy})
	-- ({1.2463*\dx},{0.4497*\dy})
	-- ({1.2581*\dx},{0.4459*\dy})
	-- ({1.2699*\dx},{0.4416*\dy})
	-- ({1.2817*\dx},{0.4371*\dy})
	-- ({1.2935*\dx},{0.4322*\dy})
	-- ({1.3052*\dx},{0.4269*\dy})
	-- ({1.3169*\dx},{0.4212*\dy})
	-- ({1.3285*\dx},{0.4152*\dy})
	-- ({1.3400*\dx},{0.4087*\dy})
	-- ({1.3515*\dx},{0.4020*\dy})
	-- ({1.3629*\dx},{0.3948*\dy})
	-- ({1.3741*\dx},{0.3872*\dy})
	-- ({1.3852*\dx},{0.3793*\dy})
	-- ({1.3962*\dx},{0.3709*\dy})
	-- ({1.4070*\dx},{0.3622*\dy})
	-- ({1.4176*\dx},{0.3531*\dy})
	-- ({1.4280*\dx},{0.3436*\dy})
	-- ({1.4382*\dx},{0.3337*\dy})
	-- ({1.4482*\dx},{0.3235*\dy})
	-- ({1.4579*\dx},{0.3129*\dy})
	-- ({1.4673*\dx},{0.3019*\dy})
	-- ({1.4765*\dx},{0.2906*\dy})
	-- ({1.4854*\dx},{0.2789*\dy})
	-- ({1.4939*\dx},{0.2669*\dy})
	-- ({1.5021*\dx},{0.2545*\dy})
	-- ({1.5100*\dx},{0.2418*\dy})
	-- ({1.5175*\dx},{0.2288*\dy})
	-- ({1.5246*\dx},{0.2155*\dy})
	-- ({1.5313*\dx},{0.2020*\dy})
	-- ({1.5376*\dx},{0.1882*\dy})
	-- ({1.5434*\dx},{0.1741*\dy})
	-- ({1.5488*\dx},{0.1598*\dy})
	-- ({1.5538*\dx},{0.1452*\dy})
	-- ({1.5583*\dx},{0.1305*\dy})
	-- ({1.5624*\dx},{0.1156*\dy})
	-- ({1.5659*\dx},{0.1005*\dy})
	-- ({1.5690*\dx},{0.0853*\dy})
	-- ({1.5715*\dx},{0.0699*\dy})
	-- ({1.5736*\dx},{0.0545*\dy})
	-- ({1.5751*\dx},{0.0390*\dy})
	-- ({1.5762*\dx},{0.0234*\dy})
	-- ({1.5767*\dx},{0.0078*\dy})
	-- ({1.5767*\dx},{-0.0078*\dy})
	-- ({1.5762*\dx},{-0.0234*\dy})
	-- ({1.5751*\dx},{-0.0390*\dy})
	-- ({1.5736*\dx},{-0.0545*\dy})
	-- ({1.5715*\dx},{-0.0699*\dy})
	-- ({1.5690*\dx},{-0.0853*\dy})
	-- ({1.5659*\dx},{-0.1005*\dy})
	-- ({1.5624*\dx},{-0.1156*\dy})
	-- ({1.5583*\dx},{-0.1305*\dy})
	-- ({1.5538*\dx},{-0.1452*\dy})
	-- ({1.5488*\dx},{-0.1598*\dy})
	-- ({1.5434*\dx},{-0.1741*\dy})
	-- ({1.5376*\dx},{-0.1882*\dy})
	-- ({1.5313*\dx},{-0.2020*\dy})
	-- ({1.5246*\dx},{-0.2155*\dy})
	-- ({1.5175*\dx},{-0.2288*\dy})
	-- ({1.5100*\dx},{-0.2418*\dy})
	-- ({1.5021*\dx},{-0.2545*\dy})
	-- ({1.4939*\dx},{-0.2669*\dy})
	-- ({1.4854*\dx},{-0.2789*\dy})
	-- ({1.4765*\dx},{-0.2906*\dy})
	-- ({1.4673*\dx},{-0.3019*\dy})
	-- ({1.4579*\dx},{-0.3129*\dy})
	-- ({1.4482*\dx},{-0.3235*\dy})
	-- ({1.4382*\dx},{-0.3337*\dy})
	-- ({1.4280*\dx},{-0.3436*\dy})
	-- ({1.4176*\dx},{-0.3531*\dy})
	-- ({1.4070*\dx},{-0.3622*\dy})
	-- ({1.3962*\dx},{-0.3709*\dy})
	-- ({1.3852*\dx},{-0.3793*\dy})
	-- ({1.3741*\dx},{-0.3872*\dy})
	-- ({1.3629*\dx},{-0.3948*\dy})
	-- ({1.3515*\dx},{-0.4020*\dy})
	-- ({1.3400*\dx},{-0.4087*\dy})
	-- ({1.3285*\dx},{-0.4152*\dy})
	-- ({1.3169*\dx},{-0.4212*\dy})
	-- ({1.3052*\dx},{-0.4269*\dy})
	-- ({1.2935*\dx},{-0.4322*\dy})
	-- ({1.2817*\dx},{-0.4371*\dy})
	-- ({1.2699*\dx},{-0.4416*\dy})
	-- ({1.2581*\dx},{-0.4459*\dy})
	-- ({1.2463*\dx},{-0.4497*\dy})
	-- ({1.2346*\dx},{-0.4532*\dy})
	-- ({1.2228*\dx},{-0.4564*\dy})
	-- ({1.2111*\dx},{-0.4593*\dy})
	-- ({1.1995*\dx},{-0.4618*\dy})
	-- ({1.1878*\dx},{-0.4640*\dy})
	-- ({1.1763*\dx},{-0.4659*\dy})
	-- ({1.1648*\dx},{-0.4675*\dy})
	-- ({1.1534*\dx},{-0.4688*\dy})
	-- ({1.1421*\dx},{-0.4698*\dy})
	-- ({1.1309*\dx},{-0.4706*\dy})
	-- ({1.1197*\dx},{-0.4711*\dy})
	-- ({1.1087*\dx},{-0.4713*\dy})
	-- ({1.0978*\dx},{-0.4712*\dy})
	-- ({1.0870*\dx},{-0.4709*\dy})
	-- ({1.0763*\dx},{-0.4704*\dy})
	-- ({1.0657*\dx},{-0.4696*\dy})
	-- ({1.0552*\dx},{-0.4686*\dy})
	-- ({1.0449*\dx},{-0.4674*\dy})
	-- ({1.0347*\dx},{-0.4659*\dy})
	-- ({1.0246*\dx},{-0.4643*\dy})
	-- ({1.0147*\dx},{-0.4624*\dy})
	-- ({1.0049*\dx},{-0.4604*\dy})
	-- ({0.9953*\dx},{-0.4582*\dy})
	-- ({0.9858*\dx},{-0.4558*\dy})
	-- ({0.9764*\dx},{-0.4532*\dy})
	-- ({0.9672*\dx},{-0.4505*\dy})
	-- ({0.9581*\dx},{-0.4476*\dy})
	-- ({0.9491*\dx},{-0.4446*\dy})
	-- ({0.9403*\dx},{-0.4414*\dy})
	-- ({0.9317*\dx},{-0.4380*\dy})
	-- ({0.9232*\dx},{-0.4346*\dy})
	-- ({0.9148*\dx},{-0.4310*\dy})
	-- ({0.9066*\dx},{-0.4273*\dy})
	-- ({0.8985*\dx},{-0.4234*\dy})
	-- ({0.8906*\dx},{-0.4194*\dy})
	-- ({0.8828*\dx},{-0.4154*\dy})
	-- ({0.8752*\dx},{-0.4112*\dy})
	-- ({0.8677*\dx},{-0.4069*\dy})
	-- ({0.8604*\dx},{-0.4025*\dy})
	-- ({0.8532*\dx},{-0.3980*\dy})
	-- ({0.8461*\dx},{-0.3935*\dy})
	-- ({0.8392*\dx},{-0.3888*\dy})
	-- ({0.8324*\dx},{-0.3841*\dy})
	-- ({0.8257*\dx},{-0.3793*\dy})
	-- ({0.8192*\dx},{-0.3744*\dy})
	-- ({0.8128*\dx},{-0.3694*\dy})
	-- ({0.8066*\dx},{-0.3644*\dy})
	-- ({0.8005*\dx},{-0.3593*\dy})
	-- ({0.7945*\dx},{-0.3541*\dy})
	-- ({0.7887*\dx},{-0.3489*\dy})
	-- ({0.7830*\dx},{-0.3436*\dy})
	-- ({0.7774*\dx},{-0.3383*\dy})
	-- ({0.7719*\dx},{-0.3329*\dy})
	-- ({0.7666*\dx},{-0.3275*\dy})
	-- ({0.7614*\dx},{-0.3220*\dy})
	-- ({0.7563*\dx},{-0.3164*\dy})
	-- ({0.7513*\dx},{-0.3109*\dy})
	-- ({0.7464*\dx},{-0.3053*\dy})
	-- ({0.7417*\dx},{-0.2996*\dy})
	-- ({0.7371*\dx},{-0.2939*\dy})
	-- ({0.7326*\dx},{-0.2882*\dy})
	-- ({0.7282*\dx},{-0.2824*\dy})
	-- ({0.7240*\dx},{-0.2767*\dy})
	-- ({0.7198*\dx},{-0.2708*\dy})
	-- ({0.7158*\dx},{-0.2650*\dy})
	-- ({0.7118*\dx},{-0.2591*\dy})
	-- ({0.7080*\dx},{-0.2532*\dy})
	-- ({0.7043*\dx},{-0.2473*\dy})
	-- ({0.7007*\dx},{-0.2413*\dy})
	-- ({0.6972*\dx},{-0.2354*\dy})
	-- ({0.6938*\dx},{-0.2294*\dy})
	-- ({0.6905*\dx},{-0.2234*\dy})
	-- ({0.6873*\dx},{-0.2173*\dy})
	-- ({0.6842*\dx},{-0.2113*\dy})
	-- ({0.6812*\dx},{-0.2052*\dy})
	-- ({0.6784*\dx},{-0.1991*\dy})
	-- ({0.6756*\dx},{-0.1930*\dy})
	-- ({0.6729*\dx},{-0.1869*\dy})
	-- ({0.6703*\dx},{-0.1808*\dy})
	-- ({0.6678*\dx},{-0.1747*\dy})
	-- ({0.6654*\dx},{-0.1685*\dy})
	-- ({0.6631*\dx},{-0.1624*\dy})
	-- ({0.6609*\dx},{-0.1562*\dy})
	-- ({0.6587*\dx},{-0.1500*\dy})
	-- ({0.6567*\dx},{-0.1438*\dy})
	-- ({0.6548*\dx},{-0.1376*\dy})
	-- ({0.6529*\dx},{-0.1314*\dy})
	-- ({0.6512*\dx},{-0.1252*\dy})
	-- ({0.6495*\dx},{-0.1190*\dy})
	-- ({0.6479*\dx},{-0.1128*\dy})
	-- ({0.6464*\dx},{-0.1065*\dy})
	-- ({0.6450*\dx},{-0.1003*\dy})
	-- ({0.6437*\dx},{-0.0940*\dy})
	-- ({0.6425*\dx},{-0.0878*\dy})
	-- ({0.6413*\dx},{-0.0815*\dy})
	-- ({0.6403*\dx},{-0.0753*\dy})
	-- ({0.6393*\dx},{-0.0690*\dy})
	-- ({0.6384*\dx},{-0.0628*\dy})
	-- ({0.6376*\dx},{-0.0565*\dy})
	-- ({0.6369*\dx},{-0.0502*\dy})
	-- ({0.6363*\dx},{-0.0439*\dy})
	-- ({0.6357*\dx},{-0.0377*\dy})
	-- ({0.6353*\dx},{-0.0314*\dy})
	-- ({0.6349*\dx},{-0.0251*\dy})
	-- ({0.6346*\dx},{-0.0188*\dy})
	-- ({0.6344*\dx},{-0.0126*\dy})
	-- ({0.6343*\dx},{-0.0063*\dy})
	-- ({0.6342*\dx},{-0.0000*\dy})
}
% u = 0.914758
\def\upathO{
	({0.7234*\dx},{0.0000*\dy})
	-- ({0.7234*\dx},{0.0050*\dy})
	-- ({0.7236*\dx},{0.0100*\dy})
	-- ({0.7238*\dx},{0.0150*\dy})
	-- ({0.7240*\dx},{0.0200*\dy})
	-- ({0.7244*\dx},{0.0250*\dy})
	-- ({0.7248*\dx},{0.0300*\dy})
	-- ({0.7253*\dx},{0.0350*\dy})
	-- ({0.7258*\dx},{0.0400*\dy})
	-- ({0.7265*\dx},{0.0450*\dy})
	-- ({0.7272*\dx},{0.0500*\dy})
	-- ({0.7280*\dx},{0.0550*\dy})
	-- ({0.7289*\dx},{0.0599*\dy})
	-- ({0.7299*\dx},{0.0649*\dy})
	-- ({0.7309*\dx},{0.0698*\dy})
	-- ({0.7320*\dx},{0.0748*\dy})
	-- ({0.7332*\dx},{0.0797*\dy})
	-- ({0.7345*\dx},{0.0846*\dy})
	-- ({0.7358*\dx},{0.0895*\dy})
	-- ({0.7372*\dx},{0.0944*\dy})
	-- ({0.7387*\dx},{0.0993*\dy})
	-- ({0.7403*\dx},{0.1042*\dy})
	-- ({0.7420*\dx},{0.1090*\dy})
	-- ({0.7437*\dx},{0.1139*\dy})
	-- ({0.7455*\dx},{0.1187*\dy})
	-- ({0.7474*\dx},{0.1235*\dy})
	-- ({0.7494*\dx},{0.1283*\dy})
	-- ({0.7515*\dx},{0.1331*\dy})
	-- ({0.7536*\dx},{0.1378*\dy})
	-- ({0.7558*\dx},{0.1425*\dy})
	-- ({0.7581*\dx},{0.1472*\dy})
	-- ({0.7605*\dx},{0.1519*\dy})
	-- ({0.7630*\dx},{0.1566*\dy})
	-- ({0.7655*\dx},{0.1612*\dy})
	-- ({0.7682*\dx},{0.1658*\dy})
	-- ({0.7709*\dx},{0.1704*\dy})
	-- ({0.7737*\dx},{0.1750*\dy})
	-- ({0.7766*\dx},{0.1795*\dy})
	-- ({0.7796*\dx},{0.1840*\dy})
	-- ({0.7826*\dx},{0.1884*\dy})
	-- ({0.7858*\dx},{0.1929*\dy})
	-- ({0.7890*\dx},{0.1973*\dy})
	-- ({0.7923*\dx},{0.2016*\dy})
	-- ({0.7957*\dx},{0.2059*\dy})
	-- ({0.7992*\dx},{0.2102*\dy})
	-- ({0.8028*\dx},{0.2145*\dy})
	-- ({0.8064*\dx},{0.2187*\dy})
	-- ({0.8102*\dx},{0.2228*\dy})
	-- ({0.8140*\dx},{0.2269*\dy})
	-- ({0.8179*\dx},{0.2310*\dy})
	-- ({0.8220*\dx},{0.2350*\dy})
	-- ({0.8261*\dx},{0.2390*\dy})
	-- ({0.8303*\dx},{0.2429*\dy})
	-- ({0.8345*\dx},{0.2467*\dy})
	-- ({0.8389*\dx},{0.2505*\dy})
	-- ({0.8434*\dx},{0.2543*\dy})
	-- ({0.8479*\dx},{0.2580*\dy})
	-- ({0.8526*\dx},{0.2616*\dy})
	-- ({0.8573*\dx},{0.2651*\dy})
	-- ({0.8621*\dx},{0.2686*\dy})
	-- ({0.8670*\dx},{0.2720*\dy})
	-- ({0.8720*\dx},{0.2754*\dy})
	-- ({0.8771*\dx},{0.2787*\dy})
	-- ({0.8823*\dx},{0.2819*\dy})
	-- ({0.8876*\dx},{0.2850*\dy})
	-- ({0.8929*\dx},{0.2880*\dy})
	-- ({0.8984*\dx},{0.2910*\dy})
	-- ({0.9039*\dx},{0.2939*\dy})
	-- ({0.9095*\dx},{0.2967*\dy})
	-- ({0.9153*\dx},{0.2993*\dy})
	-- ({0.9211*\dx},{0.3019*\dy})
	-- ({0.9269*\dx},{0.3045*\dy})
	-- ({0.9329*\dx},{0.3069*\dy})
	-- ({0.9390*\dx},{0.3092*\dy})
	-- ({0.9451*\dx},{0.3113*\dy})
	-- ({0.9513*\dx},{0.3134*\dy})
	-- ({0.9576*\dx},{0.3154*\dy})
	-- ({0.9640*\dx},{0.3173*\dy})
	-- ({0.9705*\dx},{0.3190*\dy})
	-- ({0.9770*\dx},{0.3206*\dy})
	-- ({0.9837*\dx},{0.3221*\dy})
	-- ({0.9904*\dx},{0.3235*\dy})
	-- ({0.9971*\dx},{0.3247*\dy})
	-- ({1.0040*\dx},{0.3258*\dy})
	-- ({1.0109*\dx},{0.3268*\dy})
	-- ({1.0179*\dx},{0.3276*\dy})
	-- ({1.0249*\dx},{0.3283*\dy})
	-- ({1.0320*\dx},{0.3288*\dy})
	-- ({1.0392*\dx},{0.3292*\dy})
	-- ({1.0464*\dx},{0.3294*\dy})
	-- ({1.0536*\dx},{0.3295*\dy})
	-- ({1.0610*\dx},{0.3294*\dy})
	-- ({1.0683*\dx},{0.3291*\dy})
	-- ({1.0757*\dx},{0.3287*\dy})
	-- ({1.0832*\dx},{0.3281*\dy})
	-- ({1.0907*\dx},{0.3273*\dy})
	-- ({1.0982*\dx},{0.3263*\dy})
	-- ({1.1057*\dx},{0.3252*\dy})
	-- ({1.1133*\dx},{0.3239*\dy})
	-- ({1.1209*\dx},{0.3224*\dy})
	-- ({1.1285*\dx},{0.3207*\dy})
	-- ({1.1361*\dx},{0.3188*\dy})
	-- ({1.1437*\dx},{0.3167*\dy})
	-- ({1.1513*\dx},{0.3144*\dy})
	-- ({1.1589*\dx},{0.3119*\dy})
	-- ({1.1665*\dx},{0.3093*\dy})
	-- ({1.1741*\dx},{0.3064*\dy})
	-- ({1.1816*\dx},{0.3033*\dy})
	-- ({1.1891*\dx},{0.3000*\dy})
	-- ({1.1966*\dx},{0.2965*\dy})
	-- ({1.2040*\dx},{0.2927*\dy})
	-- ({1.2114*\dx},{0.2888*\dy})
	-- ({1.2187*\dx},{0.2847*\dy})
	-- ({1.2260*\dx},{0.2803*\dy})
	-- ({1.2332*\dx},{0.2758*\dy})
	-- ({1.2403*\dx},{0.2710*\dy})
	-- ({1.2473*\dx},{0.2660*\dy})
	-- ({1.2542*\dx},{0.2608*\dy})
	-- ({1.2610*\dx},{0.2554*\dy})
	-- ({1.2678*\dx},{0.2498*\dy})
	-- ({1.2744*\dx},{0.2439*\dy})
	-- ({1.2808*\dx},{0.2379*\dy})
	-- ({1.2872*\dx},{0.2317*\dy})
	-- ({1.2933*\dx},{0.2252*\dy})
	-- ({1.2994*\dx},{0.2186*\dy})
	-- ({1.3053*\dx},{0.2118*\dy})
	-- ({1.3110*\dx},{0.2047*\dy})
	-- ({1.3166*\dx},{0.1975*\dy})
	-- ({1.3219*\dx},{0.1901*\dy})
	-- ({1.3271*\dx},{0.1826*\dy})
	-- ({1.3321*\dx},{0.1749*\dy})
	-- ({1.3369*\dx},{0.1670*\dy})
	-- ({1.3415*\dx},{0.1589*\dy})
	-- ({1.3459*\dx},{0.1507*\dy})
	-- ({1.3500*\dx},{0.1423*\dy})
	-- ({1.3539*\dx},{0.1338*\dy})
	-- ({1.3576*\dx},{0.1252*\dy})
	-- ({1.3611*\dx},{0.1164*\dy})
	-- ({1.3643*\dx},{0.1076*\dy})
	-- ({1.3672*\dx},{0.0986*\dy})
	-- ({1.3700*\dx},{0.0895*\dy})
	-- ({1.3724*\dx},{0.0803*\dy})
	-- ({1.3746*\dx},{0.0711*\dy})
	-- ({1.3765*\dx},{0.0618*\dy})
	-- ({1.3782*\dx},{0.0524*\dy})
	-- ({1.3795*\dx},{0.0429*\dy})
	-- ({1.3806*\dx},{0.0334*\dy})
	-- ({1.3815*\dx},{0.0239*\dy})
	-- ({1.3820*\dx},{0.0144*\dy})
	-- ({1.3823*\dx},{0.0048*\dy})
	-- ({1.3823*\dx},{-0.0048*\dy})
	-- ({1.3820*\dx},{-0.0144*\dy})
	-- ({1.3815*\dx},{-0.0239*\dy})
	-- ({1.3806*\dx},{-0.0334*\dy})
	-- ({1.3795*\dx},{-0.0429*\dy})
	-- ({1.3782*\dx},{-0.0524*\dy})
	-- ({1.3765*\dx},{-0.0618*\dy})
	-- ({1.3746*\dx},{-0.0711*\dy})
	-- ({1.3724*\dx},{-0.0803*\dy})
	-- ({1.3700*\dx},{-0.0895*\dy})
	-- ({1.3672*\dx},{-0.0986*\dy})
	-- ({1.3643*\dx},{-0.1076*\dy})
	-- ({1.3611*\dx},{-0.1164*\dy})
	-- ({1.3576*\dx},{-0.1252*\dy})
	-- ({1.3539*\dx},{-0.1338*\dy})
	-- ({1.3500*\dx},{-0.1423*\dy})
	-- ({1.3459*\dx},{-0.1507*\dy})
	-- ({1.3415*\dx},{-0.1589*\dy})
	-- ({1.3369*\dx},{-0.1670*\dy})
	-- ({1.3321*\dx},{-0.1749*\dy})
	-- ({1.3271*\dx},{-0.1826*\dy})
	-- ({1.3219*\dx},{-0.1901*\dy})
	-- ({1.3166*\dx},{-0.1975*\dy})
	-- ({1.3110*\dx},{-0.2047*\dy})
	-- ({1.3053*\dx},{-0.2118*\dy})
	-- ({1.2994*\dx},{-0.2186*\dy})
	-- ({1.2933*\dx},{-0.2252*\dy})
	-- ({1.2872*\dx},{-0.2317*\dy})
	-- ({1.2808*\dx},{-0.2379*\dy})
	-- ({1.2744*\dx},{-0.2439*\dy})
	-- ({1.2678*\dx},{-0.2498*\dy})
	-- ({1.2610*\dx},{-0.2554*\dy})
	-- ({1.2542*\dx},{-0.2608*\dy})
	-- ({1.2473*\dx},{-0.2660*\dy})
	-- ({1.2403*\dx},{-0.2710*\dy})
	-- ({1.2332*\dx},{-0.2758*\dy})
	-- ({1.2260*\dx},{-0.2803*\dy})
	-- ({1.2187*\dx},{-0.2847*\dy})
	-- ({1.2114*\dx},{-0.2888*\dy})
	-- ({1.2040*\dx},{-0.2927*\dy})
	-- ({1.1966*\dx},{-0.2965*\dy})
	-- ({1.1891*\dx},{-0.3000*\dy})
	-- ({1.1816*\dx},{-0.3033*\dy})
	-- ({1.1741*\dx},{-0.3064*\dy})
	-- ({1.1665*\dx},{-0.3093*\dy})
	-- ({1.1589*\dx},{-0.3119*\dy})
	-- ({1.1513*\dx},{-0.3144*\dy})
	-- ({1.1437*\dx},{-0.3167*\dy})
	-- ({1.1361*\dx},{-0.3188*\dy})
	-- ({1.1285*\dx},{-0.3207*\dy})
	-- ({1.1209*\dx},{-0.3224*\dy})
	-- ({1.1133*\dx},{-0.3239*\dy})
	-- ({1.1057*\dx},{-0.3252*\dy})
	-- ({1.0982*\dx},{-0.3263*\dy})
	-- ({1.0907*\dx},{-0.3273*\dy})
	-- ({1.0832*\dx},{-0.3281*\dy})
	-- ({1.0757*\dx},{-0.3287*\dy})
	-- ({1.0683*\dx},{-0.3291*\dy})
	-- ({1.0610*\dx},{-0.3294*\dy})
	-- ({1.0536*\dx},{-0.3295*\dy})
	-- ({1.0464*\dx},{-0.3294*\dy})
	-- ({1.0392*\dx},{-0.3292*\dy})
	-- ({1.0320*\dx},{-0.3288*\dy})
	-- ({1.0249*\dx},{-0.3283*\dy})
	-- ({1.0179*\dx},{-0.3276*\dy})
	-- ({1.0109*\dx},{-0.3268*\dy})
	-- ({1.0040*\dx},{-0.3258*\dy})
	-- ({0.9971*\dx},{-0.3247*\dy})
	-- ({0.9904*\dx},{-0.3235*\dy})
	-- ({0.9837*\dx},{-0.3221*\dy})
	-- ({0.9770*\dx},{-0.3206*\dy})
	-- ({0.9705*\dx},{-0.3190*\dy})
	-- ({0.9640*\dx},{-0.3173*\dy})
	-- ({0.9576*\dx},{-0.3154*\dy})
	-- ({0.9513*\dx},{-0.3134*\dy})
	-- ({0.9451*\dx},{-0.3113*\dy})
	-- ({0.9390*\dx},{-0.3092*\dy})
	-- ({0.9329*\dx},{-0.3069*\dy})
	-- ({0.9269*\dx},{-0.3045*\dy})
	-- ({0.9211*\dx},{-0.3019*\dy})
	-- ({0.9153*\dx},{-0.2993*\dy})
	-- ({0.9095*\dx},{-0.2967*\dy})
	-- ({0.9039*\dx},{-0.2939*\dy})
	-- ({0.8984*\dx},{-0.2910*\dy})
	-- ({0.8929*\dx},{-0.2880*\dy})
	-- ({0.8876*\dx},{-0.2850*\dy})
	-- ({0.8823*\dx},{-0.2819*\dy})
	-- ({0.8771*\dx},{-0.2787*\dy})
	-- ({0.8720*\dx},{-0.2754*\dy})
	-- ({0.8670*\dx},{-0.2720*\dy})
	-- ({0.8621*\dx},{-0.2686*\dy})
	-- ({0.8573*\dx},{-0.2651*\dy})
	-- ({0.8526*\dx},{-0.2616*\dy})
	-- ({0.8479*\dx},{-0.2580*\dy})
	-- ({0.8434*\dx},{-0.2543*\dy})
	-- ({0.8389*\dx},{-0.2505*\dy})
	-- ({0.8345*\dx},{-0.2467*\dy})
	-- ({0.8303*\dx},{-0.2429*\dy})
	-- ({0.8261*\dx},{-0.2390*\dy})
	-- ({0.8220*\dx},{-0.2350*\dy})
	-- ({0.8179*\dx},{-0.2310*\dy})
	-- ({0.8140*\dx},{-0.2269*\dy})
	-- ({0.8102*\dx},{-0.2228*\dy})
	-- ({0.8064*\dx},{-0.2187*\dy})
	-- ({0.8028*\dx},{-0.2145*\dy})
	-- ({0.7992*\dx},{-0.2102*\dy})
	-- ({0.7957*\dx},{-0.2059*\dy})
	-- ({0.7923*\dx},{-0.2016*\dy})
	-- ({0.7890*\dx},{-0.1973*\dy})
	-- ({0.7858*\dx},{-0.1929*\dy})
	-- ({0.7826*\dx},{-0.1884*\dy})
	-- ({0.7796*\dx},{-0.1840*\dy})
	-- ({0.7766*\dx},{-0.1795*\dy})
	-- ({0.7737*\dx},{-0.1750*\dy})
	-- ({0.7709*\dx},{-0.1704*\dy})
	-- ({0.7682*\dx},{-0.1658*\dy})
	-- ({0.7655*\dx},{-0.1612*\dy})
	-- ({0.7630*\dx},{-0.1566*\dy})
	-- ({0.7605*\dx},{-0.1519*\dy})
	-- ({0.7581*\dx},{-0.1472*\dy})
	-- ({0.7558*\dx},{-0.1425*\dy})
	-- ({0.7536*\dx},{-0.1378*\dy})
	-- ({0.7515*\dx},{-0.1331*\dy})
	-- ({0.7494*\dx},{-0.1283*\dy})
	-- ({0.7474*\dx},{-0.1235*\dy})
	-- ({0.7455*\dx},{-0.1187*\dy})
	-- ({0.7437*\dx},{-0.1139*\dy})
	-- ({0.7420*\dx},{-0.1090*\dy})
	-- ({0.7403*\dx},{-0.1042*\dy})
	-- ({0.7387*\dx},{-0.0993*\dy})
	-- ({0.7372*\dx},{-0.0944*\dy})
	-- ({0.7358*\dx},{-0.0895*\dy})
	-- ({0.7345*\dx},{-0.0846*\dy})
	-- ({0.7332*\dx},{-0.0797*\dy})
	-- ({0.7320*\dx},{-0.0748*\dy})
	-- ({0.7309*\dx},{-0.0698*\dy})
	-- ({0.7299*\dx},{-0.0649*\dy})
	-- ({0.7289*\dx},{-0.0599*\dy})
	-- ({0.7280*\dx},{-0.0550*\dy})
	-- ({0.7272*\dx},{-0.0500*\dy})
	-- ({0.7265*\dx},{-0.0450*\dy})
	-- ({0.7258*\dx},{-0.0400*\dy})
	-- ({0.7253*\dx},{-0.0350*\dy})
	-- ({0.7248*\dx},{-0.0300*\dy})
	-- ({0.7244*\dx},{-0.0250*\dy})
	-- ({0.7240*\dx},{-0.0200*\dy})
	-- ({0.7238*\dx},{-0.0150*\dy})
	-- ({0.7236*\dx},{-0.0100*\dy})
	-- ({0.7234*\dx},{-0.0050*\dy})
	-- ({0.7234*\dx},{-0.0000*\dy})
}
% u = 1.081077
\def\upathP{
	({0.7936*\dx},{0.0000*\dy})
	-- ({0.7936*\dx},{0.0039*\dy})
	-- ({0.7937*\dx},{0.0078*\dy})
	-- ({0.7939*\dx},{0.0117*\dy})
	-- ({0.7941*\dx},{0.0156*\dy})
	-- ({0.7944*\dx},{0.0194*\dy})
	-- ({0.7948*\dx},{0.0233*\dy})
	-- ({0.7952*\dx},{0.0272*\dy})
	-- ({0.7957*\dx},{0.0311*\dy})
	-- ({0.7962*\dx},{0.0349*\dy})
	-- ({0.7968*\dx},{0.0388*\dy})
	-- ({0.7975*\dx},{0.0426*\dy})
	-- ({0.7983*\dx},{0.0465*\dy})
	-- ({0.7991*\dx},{0.0503*\dy})
	-- ({0.8000*\dx},{0.0541*\dy})
	-- ({0.8009*\dx},{0.0579*\dy})
	-- ({0.8019*\dx},{0.0617*\dy})
	-- ({0.8030*\dx},{0.0655*\dy})
	-- ({0.8041*\dx},{0.0693*\dy})
	-- ({0.8053*\dx},{0.0730*\dy})
	-- ({0.8066*\dx},{0.0768*\dy})
	-- ({0.8079*\dx},{0.0805*\dy})
	-- ({0.8093*\dx},{0.0842*\dy})
	-- ({0.8108*\dx},{0.0879*\dy})
	-- ({0.8123*\dx},{0.0916*\dy})
	-- ({0.8139*\dx},{0.0952*\dy})
	-- ({0.8156*\dx},{0.0988*\dy})
	-- ({0.8173*\dx},{0.1024*\dy})
	-- ({0.8191*\dx},{0.1060*\dy})
	-- ({0.8210*\dx},{0.1096*\dy})
	-- ({0.8229*\dx},{0.1131*\dy})
	-- ({0.8249*\dx},{0.1167*\dy})
	-- ({0.8269*\dx},{0.1201*\dy})
	-- ({0.8290*\dx},{0.1236*\dy})
	-- ({0.8312*\dx},{0.1270*\dy})
	-- ({0.8335*\dx},{0.1304*\dy})
	-- ({0.8358*\dx},{0.1338*\dy})
	-- ({0.8382*\dx},{0.1371*\dy})
	-- ({0.8406*\dx},{0.1405*\dy})
	-- ({0.8431*\dx},{0.1437*\dy})
	-- ({0.8457*\dx},{0.1470*\dy})
	-- ({0.8484*\dx},{0.1502*\dy})
	-- ({0.8511*\dx},{0.1533*\dy})
	-- ({0.8538*\dx},{0.1564*\dy})
	-- ({0.8567*\dx},{0.1595*\dy})
	-- ({0.8596*\dx},{0.1626*\dy})
	-- ({0.8626*\dx},{0.1656*\dy})
	-- ({0.8656*\dx},{0.1685*\dy})
	-- ({0.8687*\dx},{0.1714*\dy})
	-- ({0.8719*\dx},{0.1743*\dy})
	-- ({0.8751*\dx},{0.1771*\dy})
	-- ({0.8784*\dx},{0.1799*\dy})
	-- ({0.8817*\dx},{0.1826*\dy})
	-- ({0.8851*\dx},{0.1853*\dy})
	-- ({0.8886*\dx},{0.1879*\dy})
	-- ({0.8922*\dx},{0.1904*\dy})
	-- ({0.8958*\dx},{0.1929*\dy})
	-- ({0.8994*\dx},{0.1954*\dy})
	-- ({0.9031*\dx},{0.1977*\dy})
	-- ({0.9069*\dx},{0.2001*\dy})
	-- ({0.9108*\dx},{0.2023*\dy})
	-- ({0.9147*\dx},{0.2045*\dy})
	-- ({0.9186*\dx},{0.2066*\dy})
	-- ({0.9226*\dx},{0.2087*\dy})
	-- ({0.9267*\dx},{0.2107*\dy})
	-- ({0.9308*\dx},{0.2126*\dy})
	-- ({0.9350*\dx},{0.2144*\dy})
	-- ({0.9393*\dx},{0.2162*\dy})
	-- ({0.9436*\dx},{0.2179*\dy})
	-- ({0.9479*\dx},{0.2195*\dy})
	-- ({0.9523*\dx},{0.2210*\dy})
	-- ({0.9568*\dx},{0.2225*\dy})
	-- ({0.9613*\dx},{0.2238*\dy})
	-- ({0.9658*\dx},{0.2251*\dy})
	-- ({0.9704*\dx},{0.2263*\dy})
	-- ({0.9750*\dx},{0.2274*\dy})
	-- ({0.9797*\dx},{0.2284*\dy})
	-- ({0.9844*\dx},{0.2294*\dy})
	-- ({0.9892*\dx},{0.2302*\dy})
	-- ({0.9940*\dx},{0.2309*\dy})
	-- ({0.9988*\dx},{0.2316*\dy})
	-- ({1.0037*\dx},{0.2321*\dy})
	-- ({1.0086*\dx},{0.2325*\dy})
	-- ({1.0136*\dx},{0.2329*\dy})
	-- ({1.0185*\dx},{0.2331*\dy})
	-- ({1.0236*\dx},{0.2332*\dy})
	-- ({1.0286*\dx},{0.2332*\dy})
	-- ({1.0336*\dx},{0.2331*\dy})
	-- ({1.0387*\dx},{0.2329*\dy})
	-- ({1.0438*\dx},{0.2326*\dy})
	-- ({1.0489*\dx},{0.2322*\dy})
	-- ({1.0540*\dx},{0.2317*\dy})
	-- ({1.0592*\dx},{0.2310*\dy})
	-- ({1.0643*\dx},{0.2302*\dy})
	-- ({1.0695*\dx},{0.2293*\dy})
	-- ({1.0746*\dx},{0.2283*\dy})
	-- ({1.0798*\dx},{0.2272*\dy})
	-- ({1.0849*\dx},{0.2259*\dy})
	-- ({1.0901*\dx},{0.2245*\dy})
	-- ({1.0952*\dx},{0.2230*\dy})
	-- ({1.1003*\dx},{0.2214*\dy})
	-- ({1.1055*\dx},{0.2196*\dy})
	-- ({1.1105*\dx},{0.2177*\dy})
	-- ({1.1156*\dx},{0.2157*\dy})
	-- ({1.1207*\dx},{0.2135*\dy})
	-- ({1.1257*\dx},{0.2113*\dy})
	-- ({1.1306*\dx},{0.2089*\dy})
	-- ({1.1356*\dx},{0.2063*\dy})
	-- ({1.1405*\dx},{0.2037*\dy})
	-- ({1.1454*\dx},{0.2009*\dy})
	-- ({1.1502*\dx},{0.1980*\dy})
	-- ({1.1549*\dx},{0.1949*\dy})
	-- ({1.1596*\dx},{0.1918*\dy})
	-- ({1.1643*\dx},{0.1885*\dy})
	-- ({1.1688*\dx},{0.1850*\dy})
	-- ({1.1734*\dx},{0.1815*\dy})
	-- ({1.1778*\dx},{0.1778*\dy})
	-- ({1.1822*\dx},{0.1740*\dy})
	-- ({1.1864*\dx},{0.1701*\dy})
	-- ({1.1906*\dx},{0.1661*\dy})
	-- ({1.1947*\dx},{0.1619*\dy})
	-- ({1.1988*\dx},{0.1576*\dy})
	-- ({1.2027*\dx},{0.1532*\dy})
	-- ({1.2065*\dx},{0.1487*\dy})
	-- ({1.2102*\dx},{0.1441*\dy})
	-- ({1.2138*\dx},{0.1394*\dy})
	-- ({1.2173*\dx},{0.1346*\dy})
	-- ({1.2207*\dx},{0.1297*\dy})
	-- ({1.2240*\dx},{0.1246*\dy})
	-- ({1.2271*\dx},{0.1195*\dy})
	-- ({1.2302*\dx},{0.1143*\dy})
	-- ({1.2330*\dx},{0.1090*\dy})
	-- ({1.2358*\dx},{0.1036*\dy})
	-- ({1.2384*\dx},{0.0982*\dy})
	-- ({1.2409*\dx},{0.0926*\dy})
	-- ({1.2433*\dx},{0.0870*\dy})
	-- ({1.2455*\dx},{0.0813*\dy})
	-- ({1.2475*\dx},{0.0756*\dy})
	-- ({1.2494*\dx},{0.0697*\dy})
	-- ({1.2512*\dx},{0.0639*\dy})
	-- ({1.2528*\dx},{0.0579*\dy})
	-- ({1.2542*\dx},{0.0520*\dy})
	-- ({1.2555*\dx},{0.0460*\dy})
	-- ({1.2566*\dx},{0.0399*\dy})
	-- ({1.2576*\dx},{0.0338*\dy})
	-- ({1.2584*\dx},{0.0277*\dy})
	-- ({1.2591*\dx},{0.0216*\dy})
	-- ({1.2596*\dx},{0.0154*\dy})
	-- ({1.2599*\dx},{0.0093*\dy})
	-- ({1.2601*\dx},{0.0031*\dy})
	-- ({1.2601*\dx},{-0.0031*\dy})
	-- ({1.2599*\dx},{-0.0093*\dy})
	-- ({1.2596*\dx},{-0.0154*\dy})
	-- ({1.2591*\dx},{-0.0216*\dy})
	-- ({1.2584*\dx},{-0.0277*\dy})
	-- ({1.2576*\dx},{-0.0338*\dy})
	-- ({1.2566*\dx},{-0.0399*\dy})
	-- ({1.2555*\dx},{-0.0460*\dy})
	-- ({1.2542*\dx},{-0.0520*\dy})
	-- ({1.2528*\dx},{-0.0579*\dy})
	-- ({1.2512*\dx},{-0.0639*\dy})
	-- ({1.2494*\dx},{-0.0697*\dy})
	-- ({1.2475*\dx},{-0.0756*\dy})
	-- ({1.2455*\dx},{-0.0813*\dy})
	-- ({1.2433*\dx},{-0.0870*\dy})
	-- ({1.2409*\dx},{-0.0926*\dy})
	-- ({1.2384*\dx},{-0.0982*\dy})
	-- ({1.2358*\dx},{-0.1036*\dy})
	-- ({1.2330*\dx},{-0.1090*\dy})
	-- ({1.2302*\dx},{-0.1143*\dy})
	-- ({1.2271*\dx},{-0.1195*\dy})
	-- ({1.2240*\dx},{-0.1246*\dy})
	-- ({1.2207*\dx},{-0.1297*\dy})
	-- ({1.2173*\dx},{-0.1346*\dy})
	-- ({1.2138*\dx},{-0.1394*\dy})
	-- ({1.2102*\dx},{-0.1441*\dy})
	-- ({1.2065*\dx},{-0.1487*\dy})
	-- ({1.2027*\dx},{-0.1532*\dy})
	-- ({1.1988*\dx},{-0.1576*\dy})
	-- ({1.1947*\dx},{-0.1619*\dy})
	-- ({1.1906*\dx},{-0.1661*\dy})
	-- ({1.1864*\dx},{-0.1701*\dy})
	-- ({1.1822*\dx},{-0.1740*\dy})
	-- ({1.1778*\dx},{-0.1778*\dy})
	-- ({1.1734*\dx},{-0.1815*\dy})
	-- ({1.1688*\dx},{-0.1850*\dy})
	-- ({1.1643*\dx},{-0.1885*\dy})
	-- ({1.1596*\dx},{-0.1918*\dy})
	-- ({1.1549*\dx},{-0.1949*\dy})
	-- ({1.1502*\dx},{-0.1980*\dy})
	-- ({1.1454*\dx},{-0.2009*\dy})
	-- ({1.1405*\dx},{-0.2037*\dy})
	-- ({1.1356*\dx},{-0.2063*\dy})
	-- ({1.1306*\dx},{-0.2089*\dy})
	-- ({1.1257*\dx},{-0.2113*\dy})
	-- ({1.1207*\dx},{-0.2135*\dy})
	-- ({1.1156*\dx},{-0.2157*\dy})
	-- ({1.1105*\dx},{-0.2177*\dy})
	-- ({1.1055*\dx},{-0.2196*\dy})
	-- ({1.1003*\dx},{-0.2214*\dy})
	-- ({1.0952*\dx},{-0.2230*\dy})
	-- ({1.0901*\dx},{-0.2245*\dy})
	-- ({1.0849*\dx},{-0.2259*\dy})
	-- ({1.0798*\dx},{-0.2272*\dy})
	-- ({1.0746*\dx},{-0.2283*\dy})
	-- ({1.0695*\dx},{-0.2293*\dy})
	-- ({1.0643*\dx},{-0.2302*\dy})
	-- ({1.0592*\dx},{-0.2310*\dy})
	-- ({1.0540*\dx},{-0.2317*\dy})
	-- ({1.0489*\dx},{-0.2322*\dy})
	-- ({1.0438*\dx},{-0.2326*\dy})
	-- ({1.0387*\dx},{-0.2329*\dy})
	-- ({1.0336*\dx},{-0.2331*\dy})
	-- ({1.0286*\dx},{-0.2332*\dy})
	-- ({1.0236*\dx},{-0.2332*\dy})
	-- ({1.0185*\dx},{-0.2331*\dy})
	-- ({1.0136*\dx},{-0.2329*\dy})
	-- ({1.0086*\dx},{-0.2325*\dy})
	-- ({1.0037*\dx},{-0.2321*\dy})
	-- ({0.9988*\dx},{-0.2316*\dy})
	-- ({0.9940*\dx},{-0.2309*\dy})
	-- ({0.9892*\dx},{-0.2302*\dy})
	-- ({0.9844*\dx},{-0.2294*\dy})
	-- ({0.9797*\dx},{-0.2284*\dy})
	-- ({0.9750*\dx},{-0.2274*\dy})
	-- ({0.9704*\dx},{-0.2263*\dy})
	-- ({0.9658*\dx},{-0.2251*\dy})
	-- ({0.9613*\dx},{-0.2238*\dy})
	-- ({0.9568*\dx},{-0.2225*\dy})
	-- ({0.9523*\dx},{-0.2210*\dy})
	-- ({0.9479*\dx},{-0.2195*\dy})
	-- ({0.9436*\dx},{-0.2179*\dy})
	-- ({0.9393*\dx},{-0.2162*\dy})
	-- ({0.9350*\dx},{-0.2144*\dy})
	-- ({0.9308*\dx},{-0.2126*\dy})
	-- ({0.9267*\dx},{-0.2107*\dy})
	-- ({0.9226*\dx},{-0.2087*\dy})
	-- ({0.9186*\dx},{-0.2066*\dy})
	-- ({0.9147*\dx},{-0.2045*\dy})
	-- ({0.9108*\dx},{-0.2023*\dy})
	-- ({0.9069*\dx},{-0.2001*\dy})
	-- ({0.9031*\dx},{-0.1977*\dy})
	-- ({0.8994*\dx},{-0.1954*\dy})
	-- ({0.8958*\dx},{-0.1929*\dy})
	-- ({0.8922*\dx},{-0.1904*\dy})
	-- ({0.8886*\dx},{-0.1879*\dy})
	-- ({0.8851*\dx},{-0.1853*\dy})
	-- ({0.8817*\dx},{-0.1826*\dy})
	-- ({0.8784*\dx},{-0.1799*\dy})
	-- ({0.8751*\dx},{-0.1771*\dy})
	-- ({0.8719*\dx},{-0.1743*\dy})
	-- ({0.8687*\dx},{-0.1714*\dy})
	-- ({0.8656*\dx},{-0.1685*\dy})
	-- ({0.8626*\dx},{-0.1656*\dy})
	-- ({0.8596*\dx},{-0.1626*\dy})
	-- ({0.8567*\dx},{-0.1595*\dy})
	-- ({0.8538*\dx},{-0.1564*\dy})
	-- ({0.8511*\dx},{-0.1533*\dy})
	-- ({0.8484*\dx},{-0.1502*\dy})
	-- ({0.8457*\dx},{-0.1470*\dy})
	-- ({0.8431*\dx},{-0.1437*\dy})
	-- ({0.8406*\dx},{-0.1405*\dy})
	-- ({0.8382*\dx},{-0.1371*\dy})
	-- ({0.8358*\dx},{-0.1338*\dy})
	-- ({0.8335*\dx},{-0.1304*\dy})
	-- ({0.8312*\dx},{-0.1270*\dy})
	-- ({0.8290*\dx},{-0.1236*\dy})
	-- ({0.8269*\dx},{-0.1201*\dy})
	-- ({0.8249*\dx},{-0.1167*\dy})
	-- ({0.8229*\dx},{-0.1131*\dy})
	-- ({0.8210*\dx},{-0.1096*\dy})
	-- ({0.8191*\dx},{-0.1060*\dy})
	-- ({0.8173*\dx},{-0.1024*\dy})
	-- ({0.8156*\dx},{-0.0988*\dy})
	-- ({0.8139*\dx},{-0.0952*\dy})
	-- ({0.8123*\dx},{-0.0916*\dy})
	-- ({0.8108*\dx},{-0.0879*\dy})
	-- ({0.8093*\dx},{-0.0842*\dy})
	-- ({0.8079*\dx},{-0.0805*\dy})
	-- ({0.8066*\dx},{-0.0768*\dy})
	-- ({0.8053*\dx},{-0.0730*\dy})
	-- ({0.8041*\dx},{-0.0693*\dy})
	-- ({0.8030*\dx},{-0.0655*\dy})
	-- ({0.8019*\dx},{-0.0617*\dy})
	-- ({0.8009*\dx},{-0.0579*\dy})
	-- ({0.8000*\dx},{-0.0541*\dy})
	-- ({0.7991*\dx},{-0.0503*\dy})
	-- ({0.7983*\dx},{-0.0465*\dy})
	-- ({0.7975*\dx},{-0.0426*\dy})
	-- ({0.7968*\dx},{-0.0388*\dy})
	-- ({0.7962*\dx},{-0.0349*\dy})
	-- ({0.7957*\dx},{-0.0311*\dy})
	-- ({0.7952*\dx},{-0.0272*\dy})
	-- ({0.7948*\dx},{-0.0233*\dy})
	-- ({0.7944*\dx},{-0.0194*\dy})
	-- ({0.7941*\dx},{-0.0156*\dy})
	-- ({0.7939*\dx},{-0.0117*\dy})
	-- ({0.7937*\dx},{-0.0078*\dy})
	-- ({0.7936*\dx},{-0.0039*\dy})
	-- ({0.7936*\dx},{-0.0000*\dy})
}
% u = 1.247397
\def\upathQ{
	({0.8476*\dx},{0.0000*\dy})
	-- ({0.8476*\dx},{0.0030*\dy})
	-- ({0.8477*\dx},{0.0059*\dy})
	-- ({0.8478*\dx},{0.0089*\dy})
	-- ({0.8480*\dx},{0.0118*\dy})
	-- ({0.8482*\dx},{0.0148*\dy})
	-- ({0.8485*\dx},{0.0177*\dy})
	-- ({0.8488*\dx},{0.0207*\dy})
	-- ({0.8492*\dx},{0.0236*\dy})
	-- ({0.8497*\dx},{0.0265*\dy})
	-- ({0.8502*\dx},{0.0295*\dy})
	-- ({0.8507*\dx},{0.0324*\dy})
	-- ({0.8513*\dx},{0.0353*\dy})
	-- ({0.8520*\dx},{0.0382*\dy})
	-- ({0.8527*\dx},{0.0411*\dy})
	-- ({0.8535*\dx},{0.0440*\dy})
	-- ({0.8543*\dx},{0.0468*\dy})
	-- ({0.8552*\dx},{0.0497*\dy})
	-- ({0.8561*\dx},{0.0525*\dy})
	-- ({0.8570*\dx},{0.0554*\dy})
	-- ({0.8581*\dx},{0.0582*\dy})
	-- ({0.8591*\dx},{0.0610*\dy})
	-- ({0.8603*\dx},{0.0638*\dy})
	-- ({0.8614*\dx},{0.0665*\dy})
	-- ({0.8627*\dx},{0.0693*\dy})
	-- ({0.8640*\dx},{0.0720*\dy})
	-- ({0.8653*\dx},{0.0747*\dy})
	-- ({0.8667*\dx},{0.0774*\dy})
	-- ({0.8681*\dx},{0.0801*\dy})
	-- ({0.8696*\dx},{0.0827*\dy})
	-- ({0.8711*\dx},{0.0853*\dy})
	-- ({0.8727*\dx},{0.0879*\dy})
	-- ({0.8744*\dx},{0.0905*\dy})
	-- ({0.8760*\dx},{0.0930*\dy})
	-- ({0.8778*\dx},{0.0956*\dy})
	-- ({0.8796*\dx},{0.0981*\dy})
	-- ({0.8814*\dx},{0.1005*\dy})
	-- ({0.8833*\dx},{0.1030*\dy})
	-- ({0.8852*\dx},{0.1054*\dy})
	-- ({0.8872*\dx},{0.1077*\dy})
	-- ({0.8892*\dx},{0.1101*\dy})
	-- ({0.8913*\dx},{0.1124*\dy})
	-- ({0.8935*\dx},{0.1147*\dy})
	-- ({0.8956*\dx},{0.1169*\dy})
	-- ({0.8979*\dx},{0.1191*\dy})
	-- ({0.9001*\dx},{0.1213*\dy})
	-- ({0.9024*\dx},{0.1234*\dy})
	-- ({0.9048*\dx},{0.1255*\dy})
	-- ({0.9072*\dx},{0.1275*\dy})
	-- ({0.9097*\dx},{0.1296*\dy})
	-- ({0.9122*\dx},{0.1315*\dy})
	-- ({0.9147*\dx},{0.1334*\dy})
	-- ({0.9173*\dx},{0.1353*\dy})
	-- ({0.9199*\dx},{0.1372*\dy})
	-- ({0.9226*\dx},{0.1390*\dy})
	-- ({0.9253*\dx},{0.1407*\dy})
	-- ({0.9281*\dx},{0.1424*\dy})
	-- ({0.9309*\dx},{0.1440*\dy})
	-- ({0.9337*\dx},{0.1456*\dy})
	-- ({0.9366*\dx},{0.1472*\dy})
	-- ({0.9395*\dx},{0.1487*\dy})
	-- ({0.9425*\dx},{0.1501*\dy})
	-- ({0.9454*\dx},{0.1515*\dy})
	-- ({0.9485*\dx},{0.1528*\dy})
	-- ({0.9515*\dx},{0.1541*\dy})
	-- ({0.9546*\dx},{0.1553*\dy})
	-- ({0.9578*\dx},{0.1565*\dy})
	-- ({0.9609*\dx},{0.1576*\dy})
	-- ({0.9641*\dx},{0.1586*\dy})
	-- ({0.9674*\dx},{0.1596*\dy})
	-- ({0.9706*\dx},{0.1605*\dy})
	-- ({0.9739*\dx},{0.1613*\dy})
	-- ({0.9772*\dx},{0.1621*\dy})
	-- ({0.9806*\dx},{0.1628*\dy})
	-- ({0.9839*\dx},{0.1635*\dy})
	-- ({0.9873*\dx},{0.1640*\dy})
	-- ({0.9907*\dx},{0.1646*\dy})
	-- ({0.9942*\dx},{0.1650*\dy})
	-- ({0.9976*\dx},{0.1654*\dy})
	-- ({1.0011*\dx},{0.1657*\dy})
	-- ({1.0046*\dx},{0.1659*\dy})
	-- ({1.0081*\dx},{0.1661*\dy})
	-- ({1.0116*\dx},{0.1661*\dy})
	-- ({1.0152*\dx},{0.1662*\dy})
	-- ({1.0187*\dx},{0.1661*\dy})
	-- ({1.0223*\dx},{0.1659*\dy})
	-- ({1.0259*\dx},{0.1657*\dy})
	-- ({1.0294*\dx},{0.1654*\dy})
	-- ({1.0330*\dx},{0.1650*\dy})
	-- ({1.0366*\dx},{0.1646*\dy})
	-- ({1.0402*\dx},{0.1640*\dy})
	-- ({1.0438*\dx},{0.1634*\dy})
	-- ({1.0474*\dx},{0.1627*\dy})
	-- ({1.0509*\dx},{0.1619*\dy})
	-- ({1.0545*\dx},{0.1611*\dy})
	-- ({1.0581*\dx},{0.1601*\dy})
	-- ({1.0616*\dx},{0.1591*\dy})
	-- ({1.0652*\dx},{0.1580*\dy})
	-- ({1.0687*\dx},{0.1568*\dy})
	-- ({1.0722*\dx},{0.1555*\dy})
	-- ({1.0757*\dx},{0.1542*\dy})
	-- ({1.0792*\dx},{0.1527*\dy})
	-- ({1.0826*\dx},{0.1512*\dy})
	-- ({1.0861*\dx},{0.1496*\dy})
	-- ({1.0895*\dx},{0.1479*\dy})
	-- ({1.0928*\dx},{0.1461*\dy})
	-- ({1.0962*\dx},{0.1443*\dy})
	-- ({1.0995*\dx},{0.1423*\dy})
	-- ({1.1027*\dx},{0.1403*\dy})
	-- ({1.1060*\dx},{0.1382*\dy})
	-- ({1.1092*\dx},{0.1360*\dy})
	-- ({1.1123*\dx},{0.1337*\dy})
	-- ({1.1154*\dx},{0.1314*\dy})
	-- ({1.1185*\dx},{0.1290*\dy})
	-- ({1.1215*\dx},{0.1265*\dy})
	-- ({1.1244*\dx},{0.1239*\dy})
	-- ({1.1273*\dx},{0.1212*\dy})
	-- ({1.1302*\dx},{0.1185*\dy})
	-- ({1.1330*\dx},{0.1157*\dy})
	-- ({1.1357*\dx},{0.1128*\dy})
	-- ({1.1383*\dx},{0.1099*\dy})
	-- ({1.1409*\dx},{0.1069*\dy})
	-- ({1.1435*\dx},{0.1038*\dy})
	-- ({1.1459*\dx},{0.1006*\dy})
	-- ({1.1483*\dx},{0.0974*\dy})
	-- ({1.1506*\dx},{0.0941*\dy})
	-- ({1.1529*\dx},{0.0908*\dy})
	-- ({1.1550*\dx},{0.0874*\dy})
	-- ({1.1571*\dx},{0.0839*\dy})
	-- ({1.1591*\dx},{0.0804*\dy})
	-- ({1.1610*\dx},{0.0769*\dy})
	-- ({1.1629*\dx},{0.0732*\dy})
	-- ({1.1646*\dx},{0.0696*\dy})
	-- ({1.1663*\dx},{0.0659*\dy})
	-- ({1.1678*\dx},{0.0621*\dy})
	-- ({1.1693*\dx},{0.0583*\dy})
	-- ({1.1707*\dx},{0.0544*\dy})
	-- ({1.1720*\dx},{0.0506*\dy})
	-- ({1.1732*\dx},{0.0467*\dy})
	-- ({1.1743*\dx},{0.0427*\dy})
	-- ({1.1753*\dx},{0.0387*\dy})
	-- ({1.1762*\dx},{0.0347*\dy})
	-- ({1.1770*\dx},{0.0307*\dy})
	-- ({1.1777*\dx},{0.0266*\dy})
	-- ({1.1783*\dx},{0.0226*\dy})
	-- ({1.1788*\dx},{0.0185*\dy})
	-- ({1.1792*\dx},{0.0144*\dy})
	-- ({1.1795*\dx},{0.0103*\dy})
	-- ({1.1798*\dx},{0.0062*\dy})
	-- ({1.1799*\dx},{0.0021*\dy})
	-- ({1.1799*\dx},{-0.0021*\dy})
	-- ({1.1798*\dx},{-0.0062*\dy})
	-- ({1.1795*\dx},{-0.0103*\dy})
	-- ({1.1792*\dx},{-0.0144*\dy})
	-- ({1.1788*\dx},{-0.0185*\dy})
	-- ({1.1783*\dx},{-0.0226*\dy})
	-- ({1.1777*\dx},{-0.0266*\dy})
	-- ({1.1770*\dx},{-0.0307*\dy})
	-- ({1.1762*\dx},{-0.0347*\dy})
	-- ({1.1753*\dx},{-0.0387*\dy})
	-- ({1.1743*\dx},{-0.0427*\dy})
	-- ({1.1732*\dx},{-0.0467*\dy})
	-- ({1.1720*\dx},{-0.0506*\dy})
	-- ({1.1707*\dx},{-0.0544*\dy})
	-- ({1.1693*\dx},{-0.0583*\dy})
	-- ({1.1678*\dx},{-0.0621*\dy})
	-- ({1.1663*\dx},{-0.0659*\dy})
	-- ({1.1646*\dx},{-0.0696*\dy})
	-- ({1.1629*\dx},{-0.0732*\dy})
	-- ({1.1610*\dx},{-0.0769*\dy})
	-- ({1.1591*\dx},{-0.0804*\dy})
	-- ({1.1571*\dx},{-0.0839*\dy})
	-- ({1.1550*\dx},{-0.0874*\dy})
	-- ({1.1529*\dx},{-0.0908*\dy})
	-- ({1.1506*\dx},{-0.0941*\dy})
	-- ({1.1483*\dx},{-0.0974*\dy})
	-- ({1.1459*\dx},{-0.1006*\dy})
	-- ({1.1435*\dx},{-0.1038*\dy})
	-- ({1.1409*\dx},{-0.1069*\dy})
	-- ({1.1383*\dx},{-0.1099*\dy})
	-- ({1.1357*\dx},{-0.1128*\dy})
	-- ({1.1330*\dx},{-0.1157*\dy})
	-- ({1.1302*\dx},{-0.1185*\dy})
	-- ({1.1273*\dx},{-0.1212*\dy})
	-- ({1.1244*\dx},{-0.1239*\dy})
	-- ({1.1215*\dx},{-0.1265*\dy})
	-- ({1.1185*\dx},{-0.1290*\dy})
	-- ({1.1154*\dx},{-0.1314*\dy})
	-- ({1.1123*\dx},{-0.1337*\dy})
	-- ({1.1092*\dx},{-0.1360*\dy})
	-- ({1.1060*\dx},{-0.1382*\dy})
	-- ({1.1027*\dx},{-0.1403*\dy})
	-- ({1.0995*\dx},{-0.1423*\dy})
	-- ({1.0962*\dx},{-0.1443*\dy})
	-- ({1.0928*\dx},{-0.1461*\dy})
	-- ({1.0895*\dx},{-0.1479*\dy})
	-- ({1.0861*\dx},{-0.1496*\dy})
	-- ({1.0826*\dx},{-0.1512*\dy})
	-- ({1.0792*\dx},{-0.1527*\dy})
	-- ({1.0757*\dx},{-0.1542*\dy})
	-- ({1.0722*\dx},{-0.1555*\dy})
	-- ({1.0687*\dx},{-0.1568*\dy})
	-- ({1.0652*\dx},{-0.1580*\dy})
	-- ({1.0616*\dx},{-0.1591*\dy})
	-- ({1.0581*\dx},{-0.1601*\dy})
	-- ({1.0545*\dx},{-0.1611*\dy})
	-- ({1.0509*\dx},{-0.1619*\dy})
	-- ({1.0474*\dx},{-0.1627*\dy})
	-- ({1.0438*\dx},{-0.1634*\dy})
	-- ({1.0402*\dx},{-0.1640*\dy})
	-- ({1.0366*\dx},{-0.1646*\dy})
	-- ({1.0330*\dx},{-0.1650*\dy})
	-- ({1.0294*\dx},{-0.1654*\dy})
	-- ({1.0259*\dx},{-0.1657*\dy})
	-- ({1.0223*\dx},{-0.1659*\dy})
	-- ({1.0187*\dx},{-0.1661*\dy})
	-- ({1.0152*\dx},{-0.1662*\dy})
	-- ({1.0116*\dx},{-0.1661*\dy})
	-- ({1.0081*\dx},{-0.1661*\dy})
	-- ({1.0046*\dx},{-0.1659*\dy})
	-- ({1.0011*\dx},{-0.1657*\dy})
	-- ({0.9976*\dx},{-0.1654*\dy})
	-- ({0.9942*\dx},{-0.1650*\dy})
	-- ({0.9907*\dx},{-0.1646*\dy})
	-- ({0.9873*\dx},{-0.1640*\dy})
	-- ({0.9839*\dx},{-0.1635*\dy})
	-- ({0.9806*\dx},{-0.1628*\dy})
	-- ({0.9772*\dx},{-0.1621*\dy})
	-- ({0.9739*\dx},{-0.1613*\dy})
	-- ({0.9706*\dx},{-0.1605*\dy})
	-- ({0.9674*\dx},{-0.1596*\dy})
	-- ({0.9641*\dx},{-0.1586*\dy})
	-- ({0.9609*\dx},{-0.1576*\dy})
	-- ({0.9578*\dx},{-0.1565*\dy})
	-- ({0.9546*\dx},{-0.1553*\dy})
	-- ({0.9515*\dx},{-0.1541*\dy})
	-- ({0.9485*\dx},{-0.1528*\dy})
	-- ({0.9454*\dx},{-0.1515*\dy})
	-- ({0.9425*\dx},{-0.1501*\dy})
	-- ({0.9395*\dx},{-0.1487*\dy})
	-- ({0.9366*\dx},{-0.1472*\dy})
	-- ({0.9337*\dx},{-0.1456*\dy})
	-- ({0.9309*\dx},{-0.1440*\dy})
	-- ({0.9281*\dx},{-0.1424*\dy})
	-- ({0.9253*\dx},{-0.1407*\dy})
	-- ({0.9226*\dx},{-0.1390*\dy})
	-- ({0.9199*\dx},{-0.1372*\dy})
	-- ({0.9173*\dx},{-0.1353*\dy})
	-- ({0.9147*\dx},{-0.1334*\dy})
	-- ({0.9122*\dx},{-0.1315*\dy})
	-- ({0.9097*\dx},{-0.1296*\dy})
	-- ({0.9072*\dx},{-0.1275*\dy})
	-- ({0.9048*\dx},{-0.1255*\dy})
	-- ({0.9024*\dx},{-0.1234*\dy})
	-- ({0.9001*\dx},{-0.1213*\dy})
	-- ({0.8979*\dx},{-0.1191*\dy})
	-- ({0.8956*\dx},{-0.1169*\dy})
	-- ({0.8935*\dx},{-0.1147*\dy})
	-- ({0.8913*\dx},{-0.1124*\dy})
	-- ({0.8892*\dx},{-0.1101*\dy})
	-- ({0.8872*\dx},{-0.1077*\dy})
	-- ({0.8852*\dx},{-0.1054*\dy})
	-- ({0.8833*\dx},{-0.1030*\dy})
	-- ({0.8814*\dx},{-0.1005*\dy})
	-- ({0.8796*\dx},{-0.0981*\dy})
	-- ({0.8778*\dx},{-0.0956*\dy})
	-- ({0.8760*\dx},{-0.0930*\dy})
	-- ({0.8744*\dx},{-0.0905*\dy})
	-- ({0.8727*\dx},{-0.0879*\dy})
	-- ({0.8711*\dx},{-0.0853*\dy})
	-- ({0.8696*\dx},{-0.0827*\dy})
	-- ({0.8681*\dx},{-0.0801*\dy})
	-- ({0.8667*\dx},{-0.0774*\dy})
	-- ({0.8653*\dx},{-0.0747*\dy})
	-- ({0.8640*\dx},{-0.0720*\dy})
	-- ({0.8627*\dx},{-0.0693*\dy})
	-- ({0.8614*\dx},{-0.0665*\dy})
	-- ({0.8603*\dx},{-0.0638*\dy})
	-- ({0.8591*\dx},{-0.0610*\dy})
	-- ({0.8581*\dx},{-0.0582*\dy})
	-- ({0.8570*\dx},{-0.0554*\dy})
	-- ({0.8561*\dx},{-0.0525*\dy})
	-- ({0.8552*\dx},{-0.0497*\dy})
	-- ({0.8543*\dx},{-0.0468*\dy})
	-- ({0.8535*\dx},{-0.0440*\dy})
	-- ({0.8527*\dx},{-0.0411*\dy})
	-- ({0.8520*\dx},{-0.0382*\dy})
	-- ({0.8513*\dx},{-0.0353*\dy})
	-- ({0.8507*\dx},{-0.0324*\dy})
	-- ({0.8502*\dx},{-0.0295*\dy})
	-- ({0.8497*\dx},{-0.0265*\dy})
	-- ({0.8492*\dx},{-0.0236*\dy})
	-- ({0.8488*\dx},{-0.0207*\dy})
	-- ({0.8485*\dx},{-0.0177*\dy})
	-- ({0.8482*\dx},{-0.0148*\dy})
	-- ({0.8480*\dx},{-0.0118*\dy})
	-- ({0.8478*\dx},{-0.0089*\dy})
	-- ({0.8477*\dx},{-0.0059*\dy})
	-- ({0.8476*\dx},{-0.0030*\dy})
	-- ({0.8476*\dx},{-0.0000*\dy})
}
% u = 1.413717
\def\upathR{
	({0.8883*\dx},{0.0000*\dy})
	-- ({0.8883*\dx},{0.0022*\dy})
	-- ({0.8884*\dx},{0.0044*\dy})
	-- ({0.8885*\dx},{0.0066*\dy})
	-- ({0.8886*\dx},{0.0089*\dy})
	-- ({0.8888*\dx},{0.0111*\dy})
	-- ({0.8890*\dx},{0.0133*\dy})
	-- ({0.8893*\dx},{0.0155*\dy})
	-- ({0.8896*\dx},{0.0177*\dy})
	-- ({0.8900*\dx},{0.0199*\dy})
	-- ({0.8903*\dx},{0.0221*\dy})
	-- ({0.8908*\dx},{0.0242*\dy})
	-- ({0.8913*\dx},{0.0264*\dy})
	-- ({0.8918*\dx},{0.0286*\dy})
	-- ({0.8923*\dx},{0.0307*\dy})
	-- ({0.8929*\dx},{0.0329*\dy})
	-- ({0.8936*\dx},{0.0350*\dy})
	-- ({0.8942*\dx},{0.0371*\dy})
	-- ({0.8950*\dx},{0.0392*\dy})
	-- ({0.8957*\dx},{0.0413*\dy})
	-- ({0.8965*\dx},{0.0434*\dy})
	-- ({0.8973*\dx},{0.0455*\dy})
	-- ({0.8982*\dx},{0.0476*\dy})
	-- ({0.8991*\dx},{0.0496*\dy})
	-- ({0.9001*\dx},{0.0516*\dy})
	-- ({0.9011*\dx},{0.0537*\dy})
	-- ({0.9021*\dx},{0.0557*\dy})
	-- ({0.9032*\dx},{0.0576*\dy})
	-- ({0.9043*\dx},{0.0596*\dy})
	-- ({0.9055*\dx},{0.0615*\dy})
	-- ({0.9067*\dx},{0.0635*\dy})
	-- ({0.9079*\dx},{0.0654*\dy})
	-- ({0.9092*\dx},{0.0672*\dy})
	-- ({0.9105*\dx},{0.0691*\dy})
	-- ({0.9118*\dx},{0.0709*\dy})
	-- ({0.9132*\dx},{0.0728*\dy})
	-- ({0.9146*\dx},{0.0745*\dy})
	-- ({0.9160*\dx},{0.0763*\dy})
	-- ({0.9175*\dx},{0.0780*\dy})
	-- ({0.9191*\dx},{0.0798*\dy})
	-- ({0.9206*\dx},{0.0814*\dy})
	-- ({0.9222*\dx},{0.0831*\dy})
	-- ({0.9238*\dx},{0.0847*\dy})
	-- ({0.9255*\dx},{0.0863*\dy})
	-- ({0.9272*\dx},{0.0879*\dy})
	-- ({0.9289*\dx},{0.0894*\dy})
	-- ({0.9307*\dx},{0.0910*\dy})
	-- ({0.9325*\dx},{0.0924*\dy})
	-- ({0.9343*\dx},{0.0939*\dy})
	-- ({0.9362*\dx},{0.0953*\dy})
	-- ({0.9381*\dx},{0.0967*\dy})
	-- ({0.9400*\dx},{0.0980*\dy})
	-- ({0.9419*\dx},{0.0993*\dy})
	-- ({0.9439*\dx},{0.1006*\dy})
	-- ({0.9459*\dx},{0.1018*\dy})
	-- ({0.9480*\dx},{0.1030*\dy})
	-- ({0.9500*\dx},{0.1042*\dy})
	-- ({0.9521*\dx},{0.1053*\dy})
	-- ({0.9542*\dx},{0.1064*\dy})
	-- ({0.9564*\dx},{0.1074*\dy})
	-- ({0.9585*\dx},{0.1084*\dy})
	-- ({0.9607*\dx},{0.1094*\dy})
	-- ({0.9630*\dx},{0.1103*\dy})
	-- ({0.9652*\dx},{0.1111*\dy})
	-- ({0.9675*\dx},{0.1120*\dy})
	-- ({0.9698*\dx},{0.1127*\dy})
	-- ({0.9721*\dx},{0.1135*\dy})
	-- ({0.9744*\dx},{0.1142*\dy})
	-- ({0.9767*\dx},{0.1148*\dy})
	-- ({0.9791*\dx},{0.1154*\dy})
	-- ({0.9815*\dx},{0.1160*\dy})
	-- ({0.9839*\dx},{0.1165*\dy})
	-- ({0.9863*\dx},{0.1169*\dy})
	-- ({0.9887*\dx},{0.1173*\dy})
	-- ({0.9912*\dx},{0.1177*\dy})
	-- ({0.9936*\dx},{0.1180*\dy})
	-- ({0.9961*\dx},{0.1182*\dy})
	-- ({0.9986*\dx},{0.1184*\dy})
	-- ({1.0011*\dx},{0.1186*\dy})
	-- ({1.0036*\dx},{0.1187*\dy})
	-- ({1.0061*\dx},{0.1187*\dy})
	-- ({1.0086*\dx},{0.1187*\dy})
	-- ({1.0111*\dx},{0.1187*\dy})
	-- ({1.0136*\dx},{0.1186*\dy})
	-- ({1.0162*\dx},{0.1184*\dy})
	-- ({1.0187*\dx},{0.1182*\dy})
	-- ({1.0212*\dx},{0.1179*\dy})
	-- ({1.0238*\dx},{0.1176*\dy})
	-- ({1.0263*\dx},{0.1172*\dy})
	-- ({1.0288*\dx},{0.1167*\dy})
	-- ({1.0313*\dx},{0.1162*\dy})
	-- ({1.0338*\dx},{0.1157*\dy})
	-- ({1.0364*\dx},{0.1151*\dy})
	-- ({1.0389*\dx},{0.1144*\dy})
	-- ({1.0414*\dx},{0.1137*\dy})
	-- ({1.0438*\dx},{0.1129*\dy})
	-- ({1.0463*\dx},{0.1121*\dy})
	-- ({1.0488*\dx},{0.1112*\dy})
	-- ({1.0512*\dx},{0.1102*\dy})
	-- ({1.0536*\dx},{0.1092*\dy})
	-- ({1.0560*\dx},{0.1082*\dy})
	-- ({1.0584*\dx},{0.1070*\dy})
	-- ({1.0608*\dx},{0.1059*\dy})
	-- ({1.0631*\dx},{0.1046*\dy})
	-- ({1.0655*\dx},{0.1034*\dy})
	-- ({1.0678*\dx},{0.1020*\dy})
	-- ({1.0701*\dx},{0.1006*\dy})
	-- ({1.0723*\dx},{0.0992*\dy})
	-- ({1.0745*\dx},{0.0977*\dy})
	-- ({1.0767*\dx},{0.0961*\dy})
	-- ({1.0789*\dx},{0.0945*\dy})
	-- ({1.0810*\dx},{0.0929*\dy})
	-- ({1.0831*\dx},{0.0912*\dy})
	-- ({1.0852*\dx},{0.0894*\dy})
	-- ({1.0872*\dx},{0.0876*\dy})
	-- ({1.0892*\dx},{0.0858*\dy})
	-- ({1.0911*\dx},{0.0839*\dy})
	-- ({1.0930*\dx},{0.0819*\dy})
	-- ({1.0949*\dx},{0.0799*\dy})
	-- ({1.0967*\dx},{0.0779*\dy})
	-- ({1.0984*\dx},{0.0758*\dy})
	-- ({1.1002*\dx},{0.0736*\dy})
	-- ({1.1019*\dx},{0.0715*\dy})
	-- ({1.1035*\dx},{0.0693*\dy})
	-- ({1.1051*\dx},{0.0670*\dy})
	-- ({1.1066*\dx},{0.0647*\dy})
	-- ({1.1081*\dx},{0.0624*\dy})
	-- ({1.1095*\dx},{0.0600*\dy})
	-- ({1.1109*\dx},{0.0576*\dy})
	-- ({1.1122*\dx},{0.0551*\dy})
	-- ({1.1134*\dx},{0.0527*\dy})
	-- ({1.1147*\dx},{0.0502*\dy})
	-- ({1.1158*\dx},{0.0476*\dy})
	-- ({1.1169*\dx},{0.0451*\dy})
	-- ({1.1179*\dx},{0.0425*\dy})
	-- ({1.1189*\dx},{0.0399*\dy})
	-- ({1.1198*\dx},{0.0372*\dy})
	-- ({1.1206*\dx},{0.0346*\dy})
	-- ({1.1214*\dx},{0.0319*\dy})
	-- ({1.1221*\dx},{0.0292*\dy})
	-- ({1.1228*\dx},{0.0264*\dy})
	-- ({1.1234*\dx},{0.0237*\dy})
	-- ({1.1239*\dx},{0.0209*\dy})
	-- ({1.1244*\dx},{0.0182*\dy})
	-- ({1.1248*\dx},{0.0154*\dy})
	-- ({1.1251*\dx},{0.0126*\dy})
	-- ({1.1254*\dx},{0.0098*\dy})
	-- ({1.1256*\dx},{0.0070*\dy})
	-- ({1.1257*\dx},{0.0042*\dy})
	-- ({1.1258*\dx},{0.0014*\dy})
	-- ({1.1258*\dx},{-0.0014*\dy})
	-- ({1.1257*\dx},{-0.0042*\dy})
	-- ({1.1256*\dx},{-0.0070*\dy})
	-- ({1.1254*\dx},{-0.0098*\dy})
	-- ({1.1251*\dx},{-0.0126*\dy})
	-- ({1.1248*\dx},{-0.0154*\dy})
	-- ({1.1244*\dx},{-0.0182*\dy})
	-- ({1.1239*\dx},{-0.0209*\dy})
	-- ({1.1234*\dx},{-0.0237*\dy})
	-- ({1.1228*\dx},{-0.0264*\dy})
	-- ({1.1221*\dx},{-0.0292*\dy})
	-- ({1.1214*\dx},{-0.0319*\dy})
	-- ({1.1206*\dx},{-0.0346*\dy})
	-- ({1.1198*\dx},{-0.0372*\dy})
	-- ({1.1189*\dx},{-0.0399*\dy})
	-- ({1.1179*\dx},{-0.0425*\dy})
	-- ({1.1169*\dx},{-0.0451*\dy})
	-- ({1.1158*\dx},{-0.0476*\dy})
	-- ({1.1147*\dx},{-0.0502*\dy})
	-- ({1.1134*\dx},{-0.0527*\dy})
	-- ({1.1122*\dx},{-0.0551*\dy})
	-- ({1.1109*\dx},{-0.0576*\dy})
	-- ({1.1095*\dx},{-0.0600*\dy})
	-- ({1.1081*\dx},{-0.0624*\dy})
	-- ({1.1066*\dx},{-0.0647*\dy})
	-- ({1.1051*\dx},{-0.0670*\dy})
	-- ({1.1035*\dx},{-0.0693*\dy})
	-- ({1.1019*\dx},{-0.0715*\dy})
	-- ({1.1002*\dx},{-0.0736*\dy})
	-- ({1.0984*\dx},{-0.0758*\dy})
	-- ({1.0967*\dx},{-0.0779*\dy})
	-- ({1.0949*\dx},{-0.0799*\dy})
	-- ({1.0930*\dx},{-0.0819*\dy})
	-- ({1.0911*\dx},{-0.0839*\dy})
	-- ({1.0892*\dx},{-0.0858*\dy})
	-- ({1.0872*\dx},{-0.0876*\dy})
	-- ({1.0852*\dx},{-0.0894*\dy})
	-- ({1.0831*\dx},{-0.0912*\dy})
	-- ({1.0810*\dx},{-0.0929*\dy})
	-- ({1.0789*\dx},{-0.0945*\dy})
	-- ({1.0767*\dx},{-0.0961*\dy})
	-- ({1.0745*\dx},{-0.0977*\dy})
	-- ({1.0723*\dx},{-0.0992*\dy})
	-- ({1.0701*\dx},{-0.1006*\dy})
	-- ({1.0678*\dx},{-0.1020*\dy})
	-- ({1.0655*\dx},{-0.1034*\dy})
	-- ({1.0631*\dx},{-0.1046*\dy})
	-- ({1.0608*\dx},{-0.1059*\dy})
	-- ({1.0584*\dx},{-0.1070*\dy})
	-- ({1.0560*\dx},{-0.1082*\dy})
	-- ({1.0536*\dx},{-0.1092*\dy})
	-- ({1.0512*\dx},{-0.1102*\dy})
	-- ({1.0488*\dx},{-0.1112*\dy})
	-- ({1.0463*\dx},{-0.1121*\dy})
	-- ({1.0438*\dx},{-0.1129*\dy})
	-- ({1.0414*\dx},{-0.1137*\dy})
	-- ({1.0389*\dx},{-0.1144*\dy})
	-- ({1.0364*\dx},{-0.1151*\dy})
	-- ({1.0338*\dx},{-0.1157*\dy})
	-- ({1.0313*\dx},{-0.1162*\dy})
	-- ({1.0288*\dx},{-0.1167*\dy})
	-- ({1.0263*\dx},{-0.1172*\dy})
	-- ({1.0238*\dx},{-0.1176*\dy})
	-- ({1.0212*\dx},{-0.1179*\dy})
	-- ({1.0187*\dx},{-0.1182*\dy})
	-- ({1.0162*\dx},{-0.1184*\dy})
	-- ({1.0136*\dx},{-0.1186*\dy})
	-- ({1.0111*\dx},{-0.1187*\dy})
	-- ({1.0086*\dx},{-0.1187*\dy})
	-- ({1.0061*\dx},{-0.1187*\dy})
	-- ({1.0036*\dx},{-0.1187*\dy})
	-- ({1.0011*\dx},{-0.1186*\dy})
	-- ({0.9986*\dx},{-0.1184*\dy})
	-- ({0.9961*\dx},{-0.1182*\dy})
	-- ({0.9936*\dx},{-0.1180*\dy})
	-- ({0.9912*\dx},{-0.1177*\dy})
	-- ({0.9887*\dx},{-0.1173*\dy})
	-- ({0.9863*\dx},{-0.1169*\dy})
	-- ({0.9839*\dx},{-0.1165*\dy})
	-- ({0.9815*\dx},{-0.1160*\dy})
	-- ({0.9791*\dx},{-0.1154*\dy})
	-- ({0.9767*\dx},{-0.1148*\dy})
	-- ({0.9744*\dx},{-0.1142*\dy})
	-- ({0.9721*\dx},{-0.1135*\dy})
	-- ({0.9698*\dx},{-0.1127*\dy})
	-- ({0.9675*\dx},{-0.1120*\dy})
	-- ({0.9652*\dx},{-0.1111*\dy})
	-- ({0.9630*\dx},{-0.1103*\dy})
	-- ({0.9607*\dx},{-0.1094*\dy})
	-- ({0.9585*\dx},{-0.1084*\dy})
	-- ({0.9564*\dx},{-0.1074*\dy})
	-- ({0.9542*\dx},{-0.1064*\dy})
	-- ({0.9521*\dx},{-0.1053*\dy})
	-- ({0.9500*\dx},{-0.1042*\dy})
	-- ({0.9480*\dx},{-0.1030*\dy})
	-- ({0.9459*\dx},{-0.1018*\dy})
	-- ({0.9439*\dx},{-0.1006*\dy})
	-- ({0.9419*\dx},{-0.0993*\dy})
	-- ({0.9400*\dx},{-0.0980*\dy})
	-- ({0.9381*\dx},{-0.0967*\dy})
	-- ({0.9362*\dx},{-0.0953*\dy})
	-- ({0.9343*\dx},{-0.0939*\dy})
	-- ({0.9325*\dx},{-0.0924*\dy})
	-- ({0.9307*\dx},{-0.0910*\dy})
	-- ({0.9289*\dx},{-0.0894*\dy})
	-- ({0.9272*\dx},{-0.0879*\dy})
	-- ({0.9255*\dx},{-0.0863*\dy})
	-- ({0.9238*\dx},{-0.0847*\dy})
	-- ({0.9222*\dx},{-0.0831*\dy})
	-- ({0.9206*\dx},{-0.0814*\dy})
	-- ({0.9191*\dx},{-0.0798*\dy})
	-- ({0.9175*\dx},{-0.0780*\dy})
	-- ({0.9160*\dx},{-0.0763*\dy})
	-- ({0.9146*\dx},{-0.0745*\dy})
	-- ({0.9132*\dx},{-0.0728*\dy})
	-- ({0.9118*\dx},{-0.0709*\dy})
	-- ({0.9105*\dx},{-0.0691*\dy})
	-- ({0.9092*\dx},{-0.0672*\dy})
	-- ({0.9079*\dx},{-0.0654*\dy})
	-- ({0.9067*\dx},{-0.0635*\dy})
	-- ({0.9055*\dx},{-0.0615*\dy})
	-- ({0.9043*\dx},{-0.0596*\dy})
	-- ({0.9032*\dx},{-0.0576*\dy})
	-- ({0.9021*\dx},{-0.0557*\dy})
	-- ({0.9011*\dx},{-0.0537*\dy})
	-- ({0.9001*\dx},{-0.0516*\dy})
	-- ({0.8991*\dx},{-0.0496*\dy})
	-- ({0.8982*\dx},{-0.0476*\dy})
	-- ({0.8973*\dx},{-0.0455*\dy})
	-- ({0.8965*\dx},{-0.0434*\dy})
	-- ({0.8957*\dx},{-0.0413*\dy})
	-- ({0.8950*\dx},{-0.0392*\dy})
	-- ({0.8942*\dx},{-0.0371*\dy})
	-- ({0.8936*\dx},{-0.0350*\dy})
	-- ({0.8929*\dx},{-0.0329*\dy})
	-- ({0.8923*\dx},{-0.0307*\dy})
	-- ({0.8918*\dx},{-0.0286*\dy})
	-- ({0.8913*\dx},{-0.0264*\dy})
	-- ({0.8908*\dx},{-0.0242*\dy})
	-- ({0.8903*\dx},{-0.0221*\dy})
	-- ({0.8900*\dx},{-0.0199*\dy})
	-- ({0.8896*\dx},{-0.0177*\dy})
	-- ({0.8893*\dx},{-0.0155*\dy})
	-- ({0.8890*\dx},{-0.0133*\dy})
	-- ({0.8888*\dx},{-0.0111*\dy})
	-- ({0.8886*\dx},{-0.0089*\dy})
	-- ({0.8885*\dx},{-0.0066*\dy})
	-- ({0.8884*\dx},{-0.0044*\dy})
	-- ({0.8883*\dx},{-0.0022*\dy})
	-- ({0.8883*\dx},{-0.0000*\dy})
}
% v = -1.413717
\def\vpathA{
	({-0.9207*\dx},{0.0246*\dy})
	-- ({-0.9191*\dx},{0.0251*\dy})
	-- ({-0.9175*\dx},{0.0256*\dy})
	-- ({-0.9158*\dx},{0.0261*\dy})
	-- ({-0.9141*\dx},{0.0266*\dy})
	-- ({-0.9123*\dx},{0.0271*\dy})
	-- ({-0.9105*\dx},{0.0277*\dy})
	-- ({-0.9087*\dx},{0.0282*\dy})
	-- ({-0.9068*\dx},{0.0287*\dy})
	-- ({-0.9049*\dx},{0.0293*\dy})
	-- ({-0.9030*\dx},{0.0298*\dy})
	-- ({-0.9010*\dx},{0.0304*\dy})
	-- ({-0.8990*\dx},{0.0310*\dy})
	-- ({-0.8970*\dx},{0.0316*\dy})
	-- ({-0.8949*\dx},{0.0322*\dy})
	-- ({-0.8928*\dx},{0.0328*\dy})
	-- ({-0.8906*\dx},{0.0334*\dy})
	-- ({-0.8884*\dx},{0.0340*\dy})
	-- ({-0.8862*\dx},{0.0347*\dy})
	-- ({-0.8839*\dx},{0.0353*\dy})
	-- ({-0.8815*\dx},{0.0360*\dy})
	-- ({-0.8792*\dx},{0.0367*\dy})
	-- ({-0.8767*\dx},{0.0374*\dy})
	-- ({-0.8743*\dx},{0.0380*\dy})
	-- ({-0.8718*\dx},{0.0388*\dy})
	-- ({-0.8692*\dx},{0.0395*\dy})
	-- ({-0.8666*\dx},{0.0402*\dy})
	-- ({-0.8639*\dx},{0.0409*\dy})
	-- ({-0.8612*\dx},{0.0417*\dy})
	-- ({-0.8585*\dx},{0.0424*\dy})
	-- ({-0.8557*\dx},{0.0432*\dy})
	-- ({-0.8528*\dx},{0.0440*\dy})
	-- ({-0.8499*\dx},{0.0448*\dy})
	-- ({-0.8469*\dx},{0.0456*\dy})
	-- ({-0.8439*\dx},{0.0464*\dy})
	-- ({-0.8408*\dx},{0.0472*\dy})
	-- ({-0.8377*\dx},{0.0481*\dy})
	-- ({-0.8345*\dx},{0.0489*\dy})
	-- ({-0.8313*\dx},{0.0498*\dy})
	-- ({-0.8280*\dx},{0.0507*\dy})
	-- ({-0.8246*\dx},{0.0516*\dy})
	-- ({-0.8212*\dx},{0.0525*\dy})
	-- ({-0.8177*\dx},{0.0534*\dy})
	-- ({-0.8142*\dx},{0.0543*\dy})
	-- ({-0.8106*\dx},{0.0552*\dy})
	-- ({-0.8069*\dx},{0.0562*\dy})
	-- ({-0.8032*\dx},{0.0571*\dy})
	-- ({-0.7994*\dx},{0.0581*\dy})
	-- ({-0.7955*\dx},{0.0591*\dy})
	-- ({-0.7916*\dx},{0.0601*\dy})
	-- ({-0.7876*\dx},{0.0611*\dy})
	-- ({-0.7835*\dx},{0.0621*\dy})
	-- ({-0.7794*\dx},{0.0631*\dy})
	-- ({-0.7752*\dx},{0.0642*\dy})
	-- ({-0.7709*\dx},{0.0652*\dy})
	-- ({-0.7666*\dx},{0.0663*\dy})
	-- ({-0.7621*\dx},{0.0674*\dy})
	-- ({-0.7577*\dx},{0.0684*\dy})
	-- ({-0.7531*\dx},{0.0695*\dy})
	-- ({-0.7485*\dx},{0.0706*\dy})
	-- ({-0.7438*\dx},{0.0718*\dy})
	-- ({-0.7390*\dx},{0.0729*\dy})
	-- ({-0.7341*\dx},{0.0740*\dy})
	-- ({-0.7292*\dx},{0.0752*\dy})
	-- ({-0.7241*\dx},{0.0763*\dy})
	-- ({-0.7190*\dx},{0.0775*\dy})
	-- ({-0.7139*\dx},{0.0787*\dy})
	-- ({-0.7086*\dx},{0.0798*\dy})
	-- ({-0.7033*\dx},{0.0810*\dy})
	-- ({-0.6978*\dx},{0.0822*\dy})
	-- ({-0.6923*\dx},{0.0835*\dy})
	-- ({-0.6868*\dx},{0.0847*\dy})
	-- ({-0.6811*\dx},{0.0859*\dy})
	-- ({-0.6753*\dx},{0.0871*\dy})
	-- ({-0.6695*\dx},{0.0884*\dy})
	-- ({-0.6636*\dx},{0.0896*\dy})
	-- ({-0.6576*\dx},{0.0909*\dy})
	-- ({-0.6515*\dx},{0.0921*\dy})
	-- ({-0.6453*\dx},{0.0934*\dy})
	-- ({-0.6390*\dx},{0.0947*\dy})
	-- ({-0.6327*\dx},{0.0959*\dy})
	-- ({-0.6262*\dx},{0.0972*\dy})
	-- ({-0.6197*\dx},{0.0985*\dy})
	-- ({-0.6131*\dx},{0.0998*\dy})
	-- ({-0.6064*\dx},{0.1011*\dy})
	-- ({-0.5996*\dx},{0.1024*\dy})
	-- ({-0.5927*\dx},{0.1036*\dy})
	-- ({-0.5857*\dx},{0.1049*\dy})
	-- ({-0.5787*\dx},{0.1062*\dy})
	-- ({-0.5715*\dx},{0.1075*\dy})
	-- ({-0.5643*\dx},{0.1088*\dy})
	-- ({-0.5569*\dx},{0.1101*\dy})
	-- ({-0.5495*\dx},{0.1114*\dy})
	-- ({-0.5420*\dx},{0.1127*\dy})
	-- ({-0.5344*\dx},{0.1140*\dy})
	-- ({-0.5267*\dx},{0.1152*\dy})
	-- ({-0.5190*\dx},{0.1165*\dy})
	-- ({-0.5111*\dx},{0.1178*\dy})
	-- ({-0.5031*\dx},{0.1190*\dy})
	-- ({-0.4951*\dx},{0.1203*\dy})
	-- ({-0.4870*\dx},{0.1215*\dy})
	-- ({-0.4788*\dx},{0.1228*\dy})
	-- ({-0.4705*\dx},{0.1240*\dy})
	-- ({-0.4621*\dx},{0.1252*\dy})
	-- ({-0.4536*\dx},{0.1264*\dy})
	-- ({-0.4450*\dx},{0.1276*\dy})
	-- ({-0.4364*\dx},{0.1288*\dy})
	-- ({-0.4277*\dx},{0.1300*\dy})
	-- ({-0.4189*\dx},{0.1312*\dy})
	-- ({-0.4100*\dx},{0.1323*\dy})
	-- ({-0.4010*\dx},{0.1334*\dy})
	-- ({-0.3920*\dx},{0.1346*\dy})
	-- ({-0.3829*\dx},{0.1357*\dy})
	-- ({-0.3737*\dx},{0.1367*\dy})
	-- ({-0.3644*\dx},{0.1378*\dy})
	-- ({-0.3550*\dx},{0.1388*\dy})
	-- ({-0.3456*\dx},{0.1399*\dy})
	-- ({-0.3361*\dx},{0.1409*\dy})
	-- ({-0.3266*\dx},{0.1419*\dy})
	-- ({-0.3169*\dx},{0.1428*\dy})
	-- ({-0.3072*\dx},{0.1438*\dy})
	-- ({-0.2975*\dx},{0.1447*\dy})
	-- ({-0.2876*\dx},{0.1456*\dy})
	-- ({-0.2778*\dx},{0.1464*\dy})
	-- ({-0.2678*\dx},{0.1473*\dy})
	-- ({-0.2578*\dx},{0.1481*\dy})
	-- ({-0.2477*\dx},{0.1489*\dy})
	-- ({-0.2376*\dx},{0.1497*\dy})
	-- ({-0.2274*\dx},{0.1504*\dy})
	-- ({-0.2172*\dx},{0.1511*\dy})
	-- ({-0.2069*\dx},{0.1518*\dy})
	-- ({-0.1966*\dx},{0.1524*\dy})
	-- ({-0.1862*\dx},{0.1530*\dy})
	-- ({-0.1758*\dx},{0.1536*\dy})
	-- ({-0.1654*\dx},{0.1542*\dy})
	-- ({-0.1549*\dx},{0.1547*\dy})
	-- ({-0.1444*\dx},{0.1552*\dy})
	-- ({-0.1338*\dx},{0.1556*\dy})
	-- ({-0.1232*\dx},{0.1560*\dy})
	-- ({-0.1126*\dx},{0.1564*\dy})
	-- ({-0.1020*\dx},{0.1568*\dy})
	-- ({-0.0913*\dx},{0.1571*\dy})
	-- ({-0.0806*\dx},{0.1574*\dy})
	-- ({-0.0699*\dx},{0.1576*\dy})
	-- ({-0.0592*\dx},{0.1578*\dy})
	-- ({-0.0484*\dx},{0.1580*\dy})
	-- ({-0.0377*\dx},{0.1582*\dy})
	-- ({-0.0269*\dx},{0.1583*\dy})
	-- ({-0.0162*\dx},{0.1583*\dy})
	-- ({-0.0054*\dx},{0.1584*\dy})
	-- ({0.0054*\dx},{0.1584*\dy})
	-- ({0.0162*\dx},{0.1583*\dy})
	-- ({0.0269*\dx},{0.1583*\dy})
	-- ({0.0377*\dx},{0.1582*\dy})
	-- ({0.0484*\dx},{0.1580*\dy})
	-- ({0.0592*\dx},{0.1578*\dy})
	-- ({0.0699*\dx},{0.1576*\dy})
	-- ({0.0806*\dx},{0.1574*\dy})
	-- ({0.0913*\dx},{0.1571*\dy})
	-- ({0.1020*\dx},{0.1568*\dy})
	-- ({0.1126*\dx},{0.1564*\dy})
	-- ({0.1232*\dx},{0.1560*\dy})
	-- ({0.1338*\dx},{0.1556*\dy})
	-- ({0.1444*\dx},{0.1552*\dy})
	-- ({0.1549*\dx},{0.1547*\dy})
	-- ({0.1654*\dx},{0.1542*\dy})
	-- ({0.1758*\dx},{0.1536*\dy})
	-- ({0.1862*\dx},{0.1530*\dy})
	-- ({0.1966*\dx},{0.1524*\dy})
	-- ({0.2069*\dx},{0.1518*\dy})
	-- ({0.2172*\dx},{0.1511*\dy})
	-- ({0.2274*\dx},{0.1504*\dy})
	-- ({0.2376*\dx},{0.1497*\dy})
	-- ({0.2477*\dx},{0.1489*\dy})
	-- ({0.2578*\dx},{0.1481*\dy})
	-- ({0.2678*\dx},{0.1473*\dy})
	-- ({0.2778*\dx},{0.1464*\dy})
	-- ({0.2876*\dx},{0.1456*\dy})
	-- ({0.2975*\dx},{0.1447*\dy})
	-- ({0.3072*\dx},{0.1438*\dy})
	-- ({0.3169*\dx},{0.1428*\dy})
	-- ({0.3266*\dx},{0.1419*\dy})
	-- ({0.3361*\dx},{0.1409*\dy})
	-- ({0.3456*\dx},{0.1399*\dy})
	-- ({0.3550*\dx},{0.1388*\dy})
	-- ({0.3644*\dx},{0.1378*\dy})
	-- ({0.3737*\dx},{0.1367*\dy})
	-- ({0.3829*\dx},{0.1357*\dy})
	-- ({0.3920*\dx},{0.1346*\dy})
	-- ({0.4010*\dx},{0.1334*\dy})
	-- ({0.4100*\dx},{0.1323*\dy})
	-- ({0.4189*\dx},{0.1312*\dy})
	-- ({0.4277*\dx},{0.1300*\dy})
	-- ({0.4364*\dx},{0.1288*\dy})
	-- ({0.4450*\dx},{0.1276*\dy})
	-- ({0.4536*\dx},{0.1264*\dy})
	-- ({0.4621*\dx},{0.1252*\dy})
	-- ({0.4705*\dx},{0.1240*\dy})
	-- ({0.4788*\dx},{0.1228*\dy})
	-- ({0.4870*\dx},{0.1215*\dy})
	-- ({0.4951*\dx},{0.1203*\dy})
	-- ({0.5031*\dx},{0.1190*\dy})
	-- ({0.5111*\dx},{0.1178*\dy})
	-- ({0.5190*\dx},{0.1165*\dy})
	-- ({0.5267*\dx},{0.1152*\dy})
	-- ({0.5344*\dx},{0.1140*\dy})
	-- ({0.5420*\dx},{0.1127*\dy})
	-- ({0.5495*\dx},{0.1114*\dy})
	-- ({0.5569*\dx},{0.1101*\dy})
	-- ({0.5643*\dx},{0.1088*\dy})
	-- ({0.5715*\dx},{0.1075*\dy})
	-- ({0.5787*\dx},{0.1062*\dy})
	-- ({0.5857*\dx},{0.1049*\dy})
	-- ({0.5927*\dx},{0.1036*\dy})
	-- ({0.5996*\dx},{0.1024*\dy})
	-- ({0.6064*\dx},{0.1011*\dy})
	-- ({0.6131*\dx},{0.0998*\dy})
	-- ({0.6197*\dx},{0.0985*\dy})
	-- ({0.6262*\dx},{0.0972*\dy})
	-- ({0.6327*\dx},{0.0959*\dy})
	-- ({0.6390*\dx},{0.0947*\dy})
	-- ({0.6453*\dx},{0.0934*\dy})
	-- ({0.6515*\dx},{0.0921*\dy})
	-- ({0.6576*\dx},{0.0909*\dy})
	-- ({0.6636*\dx},{0.0896*\dy})
	-- ({0.6695*\dx},{0.0884*\dy})
	-- ({0.6753*\dx},{0.0871*\dy})
	-- ({0.6811*\dx},{0.0859*\dy})
	-- ({0.6868*\dx},{0.0847*\dy})
	-- ({0.6923*\dx},{0.0835*\dy})
	-- ({0.6978*\dx},{0.0822*\dy})
	-- ({0.7033*\dx},{0.0810*\dy})
	-- ({0.7086*\dx},{0.0798*\dy})
	-- ({0.7139*\dx},{0.0787*\dy})
	-- ({0.7190*\dx},{0.0775*\dy})
	-- ({0.7241*\dx},{0.0763*\dy})
	-- ({0.7292*\dx},{0.0752*\dy})
	-- ({0.7341*\dx},{0.0740*\dy})
	-- ({0.7390*\dx},{0.0729*\dy})
	-- ({0.7438*\dx},{0.0718*\dy})
	-- ({0.7485*\dx},{0.0706*\dy})
	-- ({0.7531*\dx},{0.0695*\dy})
	-- ({0.7577*\dx},{0.0684*\dy})
	-- ({0.7621*\dx},{0.0674*\dy})
	-- ({0.7666*\dx},{0.0663*\dy})
	-- ({0.7709*\dx},{0.0652*\dy})
	-- ({0.7752*\dx},{0.0642*\dy})
	-- ({0.7794*\dx},{0.0631*\dy})
	-- ({0.7835*\dx},{0.0621*\dy})
	-- ({0.7876*\dx},{0.0611*\dy})
	-- ({0.7916*\dx},{0.0601*\dy})
	-- ({0.7955*\dx},{0.0591*\dy})
	-- ({0.7994*\dx},{0.0581*\dy})
	-- ({0.8032*\dx},{0.0571*\dy})
	-- ({0.8069*\dx},{0.0562*\dy})
	-- ({0.8106*\dx},{0.0552*\dy})
	-- ({0.8142*\dx},{0.0543*\dy})
	-- ({0.8177*\dx},{0.0534*\dy})
	-- ({0.8212*\dx},{0.0525*\dy})
	-- ({0.8246*\dx},{0.0516*\dy})
	-- ({0.8280*\dx},{0.0507*\dy})
	-- ({0.8313*\dx},{0.0498*\dy})
	-- ({0.8345*\dx},{0.0489*\dy})
	-- ({0.8377*\dx},{0.0481*\dy})
	-- ({0.8408*\dx},{0.0472*\dy})
	-- ({0.8439*\dx},{0.0464*\dy})
	-- ({0.8469*\dx},{0.0456*\dy})
	-- ({0.8499*\dx},{0.0448*\dy})
	-- ({0.8528*\dx},{0.0440*\dy})
	-- ({0.8557*\dx},{0.0432*\dy})
	-- ({0.8585*\dx},{0.0424*\dy})
	-- ({0.8612*\dx},{0.0417*\dy})
	-- ({0.8639*\dx},{0.0409*\dy})
	-- ({0.8666*\dx},{0.0402*\dy})
	-- ({0.8692*\dx},{0.0395*\dy})
	-- ({0.8718*\dx},{0.0388*\dy})
	-- ({0.8743*\dx},{0.0380*\dy})
	-- ({0.8767*\dx},{0.0374*\dy})
	-- ({0.8792*\dx},{0.0367*\dy})
	-- ({0.8815*\dx},{0.0360*\dy})
	-- ({0.8839*\dx},{0.0353*\dy})
	-- ({0.8862*\dx},{0.0347*\dy})
	-- ({0.8884*\dx},{0.0340*\dy})
	-- ({0.8906*\dx},{0.0334*\dy})
	-- ({0.8928*\dx},{0.0328*\dy})
	-- ({0.8949*\dx},{0.0322*\dy})
	-- ({0.8970*\dx},{0.0316*\dy})
	-- ({0.8990*\dx},{0.0310*\dy})
	-- ({0.9010*\dx},{0.0304*\dy})
	-- ({0.9030*\dx},{0.0298*\dy})
	-- ({0.9049*\dx},{0.0293*\dy})
	-- ({0.9068*\dx},{0.0287*\dy})
	-- ({0.9087*\dx},{0.0282*\dy})
	-- ({0.9105*\dx},{0.0277*\dy})
	-- ({0.9123*\dx},{0.0271*\dy})
	-- ({0.9141*\dx},{0.0266*\dy})
	-- ({0.9158*\dx},{0.0261*\dy})
	-- ({0.9175*\dx},{0.0256*\dy})
	-- ({0.9191*\dx},{0.0251*\dy})
	-- ({0.9207*\dx},{0.0246*\dy})
}
% v = -1.247397
\def\vpathB{
	({-0.9321*\dx},{0.0486*\dy})
	-- ({-0.9307*\dx},{0.0496*\dy})
	-- ({-0.9292*\dx},{0.0506*\dy})
	-- ({-0.9278*\dx},{0.0516*\dy})
	-- ({-0.9263*\dx},{0.0526*\dy})
	-- ({-0.9247*\dx},{0.0536*\dy})
	-- ({-0.9232*\dx},{0.0547*\dy})
	-- ({-0.9216*\dx},{0.0557*\dy})
	-- ({-0.9199*\dx},{0.0568*\dy})
	-- ({-0.9183*\dx},{0.0579*\dy})
	-- ({-0.9166*\dx},{0.0591*\dy})
	-- ({-0.9148*\dx},{0.0602*\dy})
	-- ({-0.9130*\dx},{0.0614*\dy})
	-- ({-0.9112*\dx},{0.0626*\dy})
	-- ({-0.9094*\dx},{0.0638*\dy})
	-- ({-0.9075*\dx},{0.0650*\dy})
	-- ({-0.9056*\dx},{0.0663*\dy})
	-- ({-0.9037*\dx},{0.0675*\dy})
	-- ({-0.9017*\dx},{0.0688*\dy})
	-- ({-0.8996*\dx},{0.0701*\dy})
	-- ({-0.8975*\dx},{0.0715*\dy})
	-- ({-0.8954*\dx},{0.0728*\dy})
	-- ({-0.8933*\dx},{0.0742*\dy})
	-- ({-0.8911*\dx},{0.0756*\dy})
	-- ({-0.8888*\dx},{0.0771*\dy})
	-- ({-0.8865*\dx},{0.0785*\dy})
	-- ({-0.8842*\dx},{0.0800*\dy})
	-- ({-0.8818*\dx},{0.0815*\dy})
	-- ({-0.8794*\dx},{0.0830*\dy})
	-- ({-0.8769*\dx},{0.0845*\dy})
	-- ({-0.8744*\dx},{0.0861*\dy})
	-- ({-0.8718*\dx},{0.0877*\dy})
	-- ({-0.8692*\dx},{0.0893*\dy})
	-- ({-0.8665*\dx},{0.0910*\dy})
	-- ({-0.8638*\dx},{0.0926*\dy})
	-- ({-0.8610*\dx},{0.0943*\dy})
	-- ({-0.8581*\dx},{0.0961*\dy})
	-- ({-0.8552*\dx},{0.0978*\dy})
	-- ({-0.8523*\dx},{0.0996*\dy})
	-- ({-0.8493*\dx},{0.1014*\dy})
	-- ({-0.8462*\dx},{0.1032*\dy})
	-- ({-0.8431*\dx},{0.1050*\dy})
	-- ({-0.8399*\dx},{0.1069*\dy})
	-- ({-0.8367*\dx},{0.1088*\dy})
	-- ({-0.8333*\dx},{0.1107*\dy})
	-- ({-0.8300*\dx},{0.1127*\dy})
	-- ({-0.8265*\dx},{0.1147*\dy})
	-- ({-0.8230*\dx},{0.1167*\dy})
	-- ({-0.8195*\dx},{0.1187*\dy})
	-- ({-0.8158*\dx},{0.1208*\dy})
	-- ({-0.8121*\dx},{0.1228*\dy})
	-- ({-0.8084*\dx},{0.1249*\dy})
	-- ({-0.8045*\dx},{0.1271*\dy})
	-- ({-0.8006*\dx},{0.1292*\dy})
	-- ({-0.7966*\dx},{0.1314*\dy})
	-- ({-0.7926*\dx},{0.1336*\dy})
	-- ({-0.7884*\dx},{0.1359*\dy})
	-- ({-0.7842*\dx},{0.1381*\dy})
	-- ({-0.7800*\dx},{0.1404*\dy})
	-- ({-0.7756*\dx},{0.1428*\dy})
	-- ({-0.7712*\dx},{0.1451*\dy})
	-- ({-0.7666*\dx},{0.1475*\dy})
	-- ({-0.7620*\dx},{0.1498*\dy})
	-- ({-0.7574*\dx},{0.1523*\dy})
	-- ({-0.7526*\dx},{0.1547*\dy})
	-- ({-0.7477*\dx},{0.1571*\dy})
	-- ({-0.7428*\dx},{0.1596*\dy})
	-- ({-0.7378*\dx},{0.1621*\dy})
	-- ({-0.7327*\dx},{0.1647*\dy})
	-- ({-0.7275*\dx},{0.1672*\dy})
	-- ({-0.7222*\dx},{0.1698*\dy})
	-- ({-0.7168*\dx},{0.1724*\dy})
	-- ({-0.7114*\dx},{0.1750*\dy})
	-- ({-0.7058*\dx},{0.1776*\dy})
	-- ({-0.7001*\dx},{0.1802*\dy})
	-- ({-0.6944*\dx},{0.1829*\dy})
	-- ({-0.6886*\dx},{0.1856*\dy})
	-- ({-0.6826*\dx},{0.1883*\dy})
	-- ({-0.6766*\dx},{0.1910*\dy})
	-- ({-0.6705*\dx},{0.1937*\dy})
	-- ({-0.6642*\dx},{0.1964*\dy})
	-- ({-0.6579*\dx},{0.1992*\dy})
	-- ({-0.6515*\dx},{0.2019*\dy})
	-- ({-0.6450*\dx},{0.2047*\dy})
	-- ({-0.6383*\dx},{0.2075*\dy})
	-- ({-0.6316*\dx},{0.2103*\dy})
	-- ({-0.6248*\dx},{0.2131*\dy})
	-- ({-0.6178*\dx},{0.2159*\dy})
	-- ({-0.6108*\dx},{0.2187*\dy})
	-- ({-0.6037*\dx},{0.2215*\dy})
	-- ({-0.5964*\dx},{0.2243*\dy})
	-- ({-0.5891*\dx},{0.2271*\dy})
	-- ({-0.5816*\dx},{0.2299*\dy})
	-- ({-0.5740*\dx},{0.2327*\dy})
	-- ({-0.5664*\dx},{0.2355*\dy})
	-- ({-0.5586*\dx},{0.2383*\dy})
	-- ({-0.5507*\dx},{0.2411*\dy})
	-- ({-0.5427*\dx},{0.2439*\dy})
	-- ({-0.5347*\dx},{0.2467*\dy})
	-- ({-0.5265*\dx},{0.2494*\dy})
	-- ({-0.5182*\dx},{0.2522*\dy})
	-- ({-0.5098*\dx},{0.2549*\dy})
	-- ({-0.5012*\dx},{0.2577*\dy})
	-- ({-0.4926*\dx},{0.2604*\dy})
	-- ({-0.4839*\dx},{0.2630*\dy})
	-- ({-0.4751*\dx},{0.2657*\dy})
	-- ({-0.4661*\dx},{0.2683*\dy})
	-- ({-0.4571*\dx},{0.2710*\dy})
	-- ({-0.4480*\dx},{0.2736*\dy})
	-- ({-0.4387*\dx},{0.2761*\dy})
	-- ({-0.4294*\dx},{0.2786*\dy})
	-- ({-0.4200*\dx},{0.2811*\dy})
	-- ({-0.4104*\dx},{0.2836*\dy})
	-- ({-0.4008*\dx},{0.2860*\dy})
	-- ({-0.3911*\dx},{0.2884*\dy})
	-- ({-0.3813*\dx},{0.2908*\dy})
	-- ({-0.3713*\dx},{0.2931*\dy})
	-- ({-0.3613*\dx},{0.2953*\dy})
	-- ({-0.3512*\dx},{0.2976*\dy})
	-- ({-0.3411*\dx},{0.2997*\dy})
	-- ({-0.3308*\dx},{0.3019*\dy})
	-- ({-0.3204*\dx},{0.3039*\dy})
	-- ({-0.3100*\dx},{0.3060*\dy})
	-- ({-0.2995*\dx},{0.3079*\dy})
	-- ({-0.2889*\dx},{0.3098*\dy})
	-- ({-0.2782*\dx},{0.3117*\dy})
	-- ({-0.2674*\dx},{0.3135*\dy})
	-- ({-0.2566*\dx},{0.3152*\dy})
	-- ({-0.2457*\dx},{0.3169*\dy})
	-- ({-0.2348*\dx},{0.3185*\dy})
	-- ({-0.2237*\dx},{0.3200*\dy})
	-- ({-0.2127*\dx},{0.3215*\dy})
	-- ({-0.2015*\dx},{0.3229*\dy})
	-- ({-0.1903*\dx},{0.3242*\dy})
	-- ({-0.1790*\dx},{0.3255*\dy})
	-- ({-0.1677*\dx},{0.3267*\dy})
	-- ({-0.1564*\dx},{0.3278*\dy})
	-- ({-0.1450*\dx},{0.3288*\dy})
	-- ({-0.1335*\dx},{0.3298*\dy})
	-- ({-0.1221*\dx},{0.3307*\dy})
	-- ({-0.1105*\dx},{0.3315*\dy})
	-- ({-0.0990*\dx},{0.3322*\dy})
	-- ({-0.0874*\dx},{0.3329*\dy})
	-- ({-0.0758*\dx},{0.3334*\dy})
	-- ({-0.0642*\dx},{0.3339*\dy})
	-- ({-0.0525*\dx},{0.3343*\dy})
	-- ({-0.0409*\dx},{0.3347*\dy})
	-- ({-0.0292*\dx},{0.3349*\dy})
	-- ({-0.0175*\dx},{0.3351*\dy})
	-- ({-0.0058*\dx},{0.3352*\dy})
	-- ({0.0058*\dx},{0.3352*\dy})
	-- ({0.0175*\dx},{0.3351*\dy})
	-- ({0.0292*\dx},{0.3349*\dy})
	-- ({0.0409*\dx},{0.3347*\dy})
	-- ({0.0525*\dx},{0.3343*\dy})
	-- ({0.0642*\dx},{0.3339*\dy})
	-- ({0.0758*\dx},{0.3334*\dy})
	-- ({0.0874*\dx},{0.3329*\dy})
	-- ({0.0990*\dx},{0.3322*\dy})
	-- ({0.1105*\dx},{0.3315*\dy})
	-- ({0.1221*\dx},{0.3307*\dy})
	-- ({0.1335*\dx},{0.3298*\dy})
	-- ({0.1450*\dx},{0.3288*\dy})
	-- ({0.1564*\dx},{0.3278*\dy})
	-- ({0.1677*\dx},{0.3267*\dy})
	-- ({0.1790*\dx},{0.3255*\dy})
	-- ({0.1903*\dx},{0.3242*\dy})
	-- ({0.2015*\dx},{0.3229*\dy})
	-- ({0.2127*\dx},{0.3215*\dy})
	-- ({0.2237*\dx},{0.3200*\dy})
	-- ({0.2348*\dx},{0.3185*\dy})
	-- ({0.2457*\dx},{0.3169*\dy})
	-- ({0.2566*\dx},{0.3152*\dy})
	-- ({0.2674*\dx},{0.3135*\dy})
	-- ({0.2782*\dx},{0.3117*\dy})
	-- ({0.2889*\dx},{0.3098*\dy})
	-- ({0.2995*\dx},{0.3079*\dy})
	-- ({0.3100*\dx},{0.3060*\dy})
	-- ({0.3204*\dx},{0.3039*\dy})
	-- ({0.3308*\dx},{0.3019*\dy})
	-- ({0.3411*\dx},{0.2997*\dy})
	-- ({0.3512*\dx},{0.2976*\dy})
	-- ({0.3613*\dx},{0.2953*\dy})
	-- ({0.3713*\dx},{0.2931*\dy})
	-- ({0.3813*\dx},{0.2908*\dy})
	-- ({0.3911*\dx},{0.2884*\dy})
	-- ({0.4008*\dx},{0.2860*\dy})
	-- ({0.4104*\dx},{0.2836*\dy})
	-- ({0.4200*\dx},{0.2811*\dy})
	-- ({0.4294*\dx},{0.2786*\dy})
	-- ({0.4387*\dx},{0.2761*\dy})
	-- ({0.4480*\dx},{0.2736*\dy})
	-- ({0.4571*\dx},{0.2710*\dy})
	-- ({0.4661*\dx},{0.2683*\dy})
	-- ({0.4751*\dx},{0.2657*\dy})
	-- ({0.4839*\dx},{0.2630*\dy})
	-- ({0.4926*\dx},{0.2604*\dy})
	-- ({0.5012*\dx},{0.2577*\dy})
	-- ({0.5098*\dx},{0.2549*\dy})
	-- ({0.5182*\dx},{0.2522*\dy})
	-- ({0.5265*\dx},{0.2494*\dy})
	-- ({0.5347*\dx},{0.2467*\dy})
	-- ({0.5427*\dx},{0.2439*\dy})
	-- ({0.5507*\dx},{0.2411*\dy})
	-- ({0.5586*\dx},{0.2383*\dy})
	-- ({0.5664*\dx},{0.2355*\dy})
	-- ({0.5740*\dx},{0.2327*\dy})
	-- ({0.5816*\dx},{0.2299*\dy})
	-- ({0.5891*\dx},{0.2271*\dy})
	-- ({0.5964*\dx},{0.2243*\dy})
	-- ({0.6037*\dx},{0.2215*\dy})
	-- ({0.6108*\dx},{0.2187*\dy})
	-- ({0.6178*\dx},{0.2159*\dy})
	-- ({0.6248*\dx},{0.2131*\dy})
	-- ({0.6316*\dx},{0.2103*\dy})
	-- ({0.6383*\dx},{0.2075*\dy})
	-- ({0.6450*\dx},{0.2047*\dy})
	-- ({0.6515*\dx},{0.2019*\dy})
	-- ({0.6579*\dx},{0.1992*\dy})
	-- ({0.6642*\dx},{0.1964*\dy})
	-- ({0.6705*\dx},{0.1937*\dy})
	-- ({0.6766*\dx},{0.1910*\dy})
	-- ({0.6826*\dx},{0.1883*\dy})
	-- ({0.6886*\dx},{0.1856*\dy})
	-- ({0.6944*\dx},{0.1829*\dy})
	-- ({0.7001*\dx},{0.1802*\dy})
	-- ({0.7058*\dx},{0.1776*\dy})
	-- ({0.7114*\dx},{0.1750*\dy})
	-- ({0.7168*\dx},{0.1724*\dy})
	-- ({0.7222*\dx},{0.1698*\dy})
	-- ({0.7275*\dx},{0.1672*\dy})
	-- ({0.7327*\dx},{0.1647*\dy})
	-- ({0.7378*\dx},{0.1621*\dy})
	-- ({0.7428*\dx},{0.1596*\dy})
	-- ({0.7477*\dx},{0.1571*\dy})
	-- ({0.7526*\dx},{0.1547*\dy})
	-- ({0.7574*\dx},{0.1523*\dy})
	-- ({0.7620*\dx},{0.1498*\dy})
	-- ({0.7666*\dx},{0.1475*\dy})
	-- ({0.7712*\dx},{0.1451*\dy})
	-- ({0.7756*\dx},{0.1428*\dy})
	-- ({0.7800*\dx},{0.1404*\dy})
	-- ({0.7842*\dx},{0.1381*\dy})
	-- ({0.7884*\dx},{0.1359*\dy})
	-- ({0.7926*\dx},{0.1336*\dy})
	-- ({0.7966*\dx},{0.1314*\dy})
	-- ({0.8006*\dx},{0.1292*\dy})
	-- ({0.8045*\dx},{0.1271*\dy})
	-- ({0.8084*\dx},{0.1249*\dy})
	-- ({0.8121*\dx},{0.1228*\dy})
	-- ({0.8158*\dx},{0.1208*\dy})
	-- ({0.8195*\dx},{0.1187*\dy})
	-- ({0.8230*\dx},{0.1167*\dy})
	-- ({0.8265*\dx},{0.1147*\dy})
	-- ({0.8300*\dx},{0.1127*\dy})
	-- ({0.8333*\dx},{0.1107*\dy})
	-- ({0.8367*\dx},{0.1088*\dy})
	-- ({0.8399*\dx},{0.1069*\dy})
	-- ({0.8431*\dx},{0.1050*\dy})
	-- ({0.8462*\dx},{0.1032*\dy})
	-- ({0.8493*\dx},{0.1014*\dy})
	-- ({0.8523*\dx},{0.0996*\dy})
	-- ({0.8552*\dx},{0.0978*\dy})
	-- ({0.8581*\dx},{0.0961*\dy})
	-- ({0.8610*\dx},{0.0943*\dy})
	-- ({0.8638*\dx},{0.0926*\dy})
	-- ({0.8665*\dx},{0.0910*\dy})
	-- ({0.8692*\dx},{0.0893*\dy})
	-- ({0.8718*\dx},{0.0877*\dy})
	-- ({0.8744*\dx},{0.0861*\dy})
	-- ({0.8769*\dx},{0.0845*\dy})
	-- ({0.8794*\dx},{0.0830*\dy})
	-- ({0.8818*\dx},{0.0815*\dy})
	-- ({0.8842*\dx},{0.0800*\dy})
	-- ({0.8865*\dx},{0.0785*\dy})
	-- ({0.8888*\dx},{0.0771*\dy})
	-- ({0.8911*\dx},{0.0756*\dy})
	-- ({0.8933*\dx},{0.0742*\dy})
	-- ({0.8954*\dx},{0.0728*\dy})
	-- ({0.8975*\dx},{0.0715*\dy})
	-- ({0.8996*\dx},{0.0701*\dy})
	-- ({0.9017*\dx},{0.0688*\dy})
	-- ({0.9037*\dx},{0.0675*\dy})
	-- ({0.9056*\dx},{0.0663*\dy})
	-- ({0.9075*\dx},{0.0650*\dy})
	-- ({0.9094*\dx},{0.0638*\dy})
	-- ({0.9112*\dx},{0.0626*\dy})
	-- ({0.9130*\dx},{0.0614*\dy})
	-- ({0.9148*\dx},{0.0602*\dy})
	-- ({0.9166*\dx},{0.0591*\dy})
	-- ({0.9183*\dx},{0.0579*\dy})
	-- ({0.9199*\dx},{0.0568*\dy})
	-- ({0.9216*\dx},{0.0557*\dy})
	-- ({0.9232*\dx},{0.0547*\dy})
	-- ({0.9247*\dx},{0.0536*\dy})
	-- ({0.9263*\dx},{0.0526*\dy})
	-- ({0.9278*\dx},{0.0516*\dy})
	-- ({0.9292*\dx},{0.0506*\dy})
	-- ({0.9307*\dx},{0.0496*\dy})
	-- ({0.9321*\dx},{0.0486*\dy})
}
% v = -1.081077
\def\vpathC{
	({-0.9506*\dx},{0.0683*\dy})
	-- ({-0.9495*\dx},{0.0697*\dy})
	-- ({-0.9484*\dx},{0.0711*\dy})
	-- ({-0.9473*\dx},{0.0725*\dy})
	-- ({-0.9461*\dx},{0.0740*\dy})
	-- ({-0.9449*\dx},{0.0755*\dy})
	-- ({-0.9437*\dx},{0.0770*\dy})
	-- ({-0.9425*\dx},{0.0785*\dy})
	-- ({-0.9413*\dx},{0.0801*\dy})
	-- ({-0.9400*\dx},{0.0817*\dy})
	-- ({-0.9387*\dx},{0.0833*\dy})
	-- ({-0.9373*\dx},{0.0850*\dy})
	-- ({-0.9360*\dx},{0.0867*\dy})
	-- ({-0.9346*\dx},{0.0884*\dy})
	-- ({-0.9332*\dx},{0.0902*\dy})
	-- ({-0.9317*\dx},{0.0919*\dy})
	-- ({-0.9302*\dx},{0.0938*\dy})
	-- ({-0.9287*\dx},{0.0956*\dy})
	-- ({-0.9271*\dx},{0.0975*\dy})
	-- ({-0.9255*\dx},{0.0994*\dy})
	-- ({-0.9239*\dx},{0.1014*\dy})
	-- ({-0.9222*\dx},{0.1033*\dy})
	-- ({-0.9205*\dx},{0.1054*\dy})
	-- ({-0.9188*\dx},{0.1074*\dy})
	-- ({-0.9170*\dx},{0.1095*\dy})
	-- ({-0.9152*\dx},{0.1116*\dy})
	-- ({-0.9133*\dx},{0.1138*\dy})
	-- ({-0.9114*\dx},{0.1160*\dy})
	-- ({-0.9095*\dx},{0.1183*\dy})
	-- ({-0.9075*\dx},{0.1205*\dy})
	-- ({-0.9055*\dx},{0.1228*\dy})
	-- ({-0.9034*\dx},{0.1252*\dy})
	-- ({-0.9013*\dx},{0.1276*\dy})
	-- ({-0.8991*\dx},{0.1300*\dy})
	-- ({-0.8969*\dx},{0.1325*\dy})
	-- ({-0.8947*\dx},{0.1350*\dy})
	-- ({-0.8923*\dx},{0.1376*\dy})
	-- ({-0.8900*\dx},{0.1402*\dy})
	-- ({-0.8876*\dx},{0.1428*\dy})
	-- ({-0.8851*\dx},{0.1455*\dy})
	-- ({-0.8826*\dx},{0.1483*\dy})
	-- ({-0.8800*\dx},{0.1510*\dy})
	-- ({-0.8773*\dx},{0.1538*\dy})
	-- ({-0.8746*\dx},{0.1567*\dy})
	-- ({-0.8719*\dx},{0.1596*\dy})
	-- ({-0.8691*\dx},{0.1625*\dy})
	-- ({-0.8662*\dx},{0.1655*\dy})
	-- ({-0.8632*\dx},{0.1686*\dy})
	-- ({-0.8602*\dx},{0.1717*\dy})
	-- ({-0.8572*\dx},{0.1748*\dy})
	-- ({-0.8540*\dx},{0.1779*\dy})
	-- ({-0.8508*\dx},{0.1812*\dy})
	-- ({-0.8475*\dx},{0.1844*\dy})
	-- ({-0.8442*\dx},{0.1877*\dy})
	-- ({-0.8407*\dx},{0.1911*\dy})
	-- ({-0.8372*\dx},{0.1945*\dy})
	-- ({-0.8337*\dx},{0.1979*\dy})
	-- ({-0.8300*\dx},{0.2014*\dy})
	-- ({-0.8263*\dx},{0.2050*\dy})
	-- ({-0.8225*\dx},{0.2085*\dy})
	-- ({-0.8186*\dx},{0.2122*\dy})
	-- ({-0.8146*\dx},{0.2158*\dy})
	-- ({-0.8105*\dx},{0.2196*\dy})
	-- ({-0.8064*\dx},{0.2233*\dy})
	-- ({-0.8021*\dx},{0.2271*\dy})
	-- ({-0.7978*\dx},{0.2310*\dy})
	-- ({-0.7933*\dx},{0.2349*\dy})
	-- ({-0.7888*\dx},{0.2388*\dy})
	-- ({-0.7842*\dx},{0.2428*\dy})
	-- ({-0.7795*\dx},{0.2468*\dy})
	-- ({-0.7747*\dx},{0.2509*\dy})
	-- ({-0.7698*\dx},{0.2550*\dy})
	-- ({-0.7648*\dx},{0.2591*\dy})
	-- ({-0.7596*\dx},{0.2633*\dy})
	-- ({-0.7544*\dx},{0.2675*\dy})
	-- ({-0.7491*\dx},{0.2718*\dy})
	-- ({-0.7436*\dx},{0.2761*\dy})
	-- ({-0.7381*\dx},{0.2804*\dy})
	-- ({-0.7324*\dx},{0.2848*\dy})
	-- ({-0.7266*\dx},{0.2892*\dy})
	-- ({-0.7207*\dx},{0.2936*\dy})
	-- ({-0.7147*\dx},{0.2981*\dy})
	-- ({-0.7086*\dx},{0.3026*\dy})
	-- ({-0.7023*\dx},{0.3071*\dy})
	-- ({-0.6960*\dx},{0.3116*\dy})
	-- ({-0.6895*\dx},{0.3162*\dy})
	-- ({-0.6828*\dx},{0.3208*\dy})
	-- ({-0.6761*\dx},{0.3254*\dy})
	-- ({-0.6692*\dx},{0.3300*\dy})
	-- ({-0.6622*\dx},{0.3347*\dy})
	-- ({-0.6551*\dx},{0.3393*\dy})
	-- ({-0.6478*\dx},{0.3440*\dy})
	-- ({-0.6404*\dx},{0.3487*\dy})
	-- ({-0.6328*\dx},{0.3534*\dy})
	-- ({-0.6251*\dx},{0.3581*\dy})
	-- ({-0.6173*\dx},{0.3628*\dy})
	-- ({-0.6094*\dx},{0.3675*\dy})
	-- ({-0.6013*\dx},{0.3722*\dy})
	-- ({-0.5930*\dx},{0.3769*\dy})
	-- ({-0.5847*\dx},{0.3816*\dy})
	-- ({-0.5762*\dx},{0.3863*\dy})
	-- ({-0.5675*\dx},{0.3910*\dy})
	-- ({-0.5587*\dx},{0.3956*\dy})
	-- ({-0.5498*\dx},{0.4003*\dy})
	-- ({-0.5407*\dx},{0.4049*\dy})
	-- ({-0.5314*\dx},{0.4095*\dy})
	-- ({-0.5221*\dx},{0.4140*\dy})
	-- ({-0.5125*\dx},{0.4185*\dy})
	-- ({-0.5029*\dx},{0.4230*\dy})
	-- ({-0.4931*\dx},{0.4275*\dy})
	-- ({-0.4831*\dx},{0.4319*\dy})
	-- ({-0.4731*\dx},{0.4362*\dy})
	-- ({-0.4628*\dx},{0.4406*\dy})
	-- ({-0.4525*\dx},{0.4448*\dy})
	-- ({-0.4420*\dx},{0.4490*\dy})
	-- ({-0.4313*\dx},{0.4532*\dy})
	-- ({-0.4205*\dx},{0.4572*\dy})
	-- ({-0.4096*\dx},{0.4612*\dy})
	-- ({-0.3986*\dx},{0.4652*\dy})
	-- ({-0.3874*\dx},{0.4690*\dy})
	-- ({-0.3761*\dx},{0.4728*\dy})
	-- ({-0.3647*\dx},{0.4765*\dy})
	-- ({-0.3531*\dx},{0.4801*\dy})
	-- ({-0.3414*\dx},{0.4836*\dy})
	-- ({-0.3296*\dx},{0.4870*\dy})
	-- ({-0.3177*\dx},{0.4904*\dy})
	-- ({-0.3057*\dx},{0.4936*\dy})
	-- ({-0.2936*\dx},{0.4967*\dy})
	-- ({-0.2813*\dx},{0.4997*\dy})
	-- ({-0.2690*\dx},{0.5026*\dy})
	-- ({-0.2565*\dx},{0.5054*\dy})
	-- ({-0.2440*\dx},{0.5081*\dy})
	-- ({-0.2313*\dx},{0.5106*\dy})
	-- ({-0.2186*\dx},{0.5130*\dy})
	-- ({-0.2058*\dx},{0.5153*\dy})
	-- ({-0.1929*\dx},{0.5175*\dy})
	-- ({-0.1799*\dx},{0.5195*\dy})
	-- ({-0.1669*\dx},{0.5214*\dy})
	-- ({-0.1538*\dx},{0.5232*\dy})
	-- ({-0.1406*\dx},{0.5248*\dy})
	-- ({-0.1274*\dx},{0.5263*\dy})
	-- ({-0.1141*\dx},{0.5276*\dy})
	-- ({-0.1008*\dx},{0.5288*\dy})
	-- ({-0.0874*\dx},{0.5298*\dy})
	-- ({-0.0741*\dx},{0.5307*\dy})
	-- ({-0.0606*\dx},{0.5315*\dy})
	-- ({-0.0472*\dx},{0.5321*\dy})
	-- ({-0.0337*\dx},{0.5326*\dy})
	-- ({-0.0202*\dx},{0.5329*\dy})
	-- ({-0.0067*\dx},{0.5330*\dy})
	-- ({0.0067*\dx},{0.5330*\dy})
	-- ({0.0202*\dx},{0.5329*\dy})
	-- ({0.0337*\dx},{0.5326*\dy})
	-- ({0.0472*\dx},{0.5321*\dy})
	-- ({0.0606*\dx},{0.5315*\dy})
	-- ({0.0741*\dx},{0.5307*\dy})
	-- ({0.0874*\dx},{0.5298*\dy})
	-- ({0.1008*\dx},{0.5288*\dy})
	-- ({0.1141*\dx},{0.5276*\dy})
	-- ({0.1274*\dx},{0.5263*\dy})
	-- ({0.1406*\dx},{0.5248*\dy})
	-- ({0.1538*\dx},{0.5232*\dy})
	-- ({0.1669*\dx},{0.5214*\dy})
	-- ({0.1799*\dx},{0.5195*\dy})
	-- ({0.1929*\dx},{0.5175*\dy})
	-- ({0.2058*\dx},{0.5153*\dy})
	-- ({0.2186*\dx},{0.5130*\dy})
	-- ({0.2313*\dx},{0.5106*\dy})
	-- ({0.2440*\dx},{0.5081*\dy})
	-- ({0.2565*\dx},{0.5054*\dy})
	-- ({0.2690*\dx},{0.5026*\dy})
	-- ({0.2813*\dx},{0.4997*\dy})
	-- ({0.2936*\dx},{0.4967*\dy})
	-- ({0.3057*\dx},{0.4936*\dy})
	-- ({0.3177*\dx},{0.4904*\dy})
	-- ({0.3296*\dx},{0.4870*\dy})
	-- ({0.3414*\dx},{0.4836*\dy})
	-- ({0.3531*\dx},{0.4801*\dy})
	-- ({0.3647*\dx},{0.4765*\dy})
	-- ({0.3761*\dx},{0.4728*\dy})
	-- ({0.3874*\dx},{0.4690*\dy})
	-- ({0.3986*\dx},{0.4652*\dy})
	-- ({0.4096*\dx},{0.4612*\dy})
	-- ({0.4205*\dx},{0.4572*\dy})
	-- ({0.4313*\dx},{0.4532*\dy})
	-- ({0.4420*\dx},{0.4490*\dy})
	-- ({0.4525*\dx},{0.4448*\dy})
	-- ({0.4628*\dx},{0.4406*\dy})
	-- ({0.4731*\dx},{0.4362*\dy})
	-- ({0.4831*\dx},{0.4319*\dy})
	-- ({0.4931*\dx},{0.4275*\dy})
	-- ({0.5029*\dx},{0.4230*\dy})
	-- ({0.5125*\dx},{0.4185*\dy})
	-- ({0.5221*\dx},{0.4140*\dy})
	-- ({0.5314*\dx},{0.4095*\dy})
	-- ({0.5407*\dx},{0.4049*\dy})
	-- ({0.5498*\dx},{0.4003*\dy})
	-- ({0.5587*\dx},{0.3956*\dy})
	-- ({0.5675*\dx},{0.3910*\dy})
	-- ({0.5762*\dx},{0.3863*\dy})
	-- ({0.5847*\dx},{0.3816*\dy})
	-- ({0.5930*\dx},{0.3769*\dy})
	-- ({0.6013*\dx},{0.3722*\dy})
	-- ({0.6094*\dx},{0.3675*\dy})
	-- ({0.6173*\dx},{0.3628*\dy})
	-- ({0.6251*\dx},{0.3581*\dy})
	-- ({0.6328*\dx},{0.3534*\dy})
	-- ({0.6404*\dx},{0.3487*\dy})
	-- ({0.6478*\dx},{0.3440*\dy})
	-- ({0.6551*\dx},{0.3393*\dy})
	-- ({0.6622*\dx},{0.3347*\dy})
	-- ({0.6692*\dx},{0.3300*\dy})
	-- ({0.6761*\dx},{0.3254*\dy})
	-- ({0.6828*\dx},{0.3208*\dy})
	-- ({0.6895*\dx},{0.3162*\dy})
	-- ({0.6960*\dx},{0.3116*\dy})
	-- ({0.7023*\dx},{0.3071*\dy})
	-- ({0.7086*\dx},{0.3026*\dy})
	-- ({0.7147*\dx},{0.2981*\dy})
	-- ({0.7207*\dx},{0.2936*\dy})
	-- ({0.7266*\dx},{0.2892*\dy})
	-- ({0.7324*\dx},{0.2848*\dy})
	-- ({0.7381*\dx},{0.2804*\dy})
	-- ({0.7436*\dx},{0.2761*\dy})
	-- ({0.7491*\dx},{0.2718*\dy})
	-- ({0.7544*\dx},{0.2675*\dy})
	-- ({0.7596*\dx},{0.2633*\dy})
	-- ({0.7648*\dx},{0.2591*\dy})
	-- ({0.7698*\dx},{0.2550*\dy})
	-- ({0.7747*\dx},{0.2509*\dy})
	-- ({0.7795*\dx},{0.2468*\dy})
	-- ({0.7842*\dx},{0.2428*\dy})
	-- ({0.7888*\dx},{0.2388*\dy})
	-- ({0.7933*\dx},{0.2349*\dy})
	-- ({0.7978*\dx},{0.2310*\dy})
	-- ({0.8021*\dx},{0.2271*\dy})
	-- ({0.8064*\dx},{0.2233*\dy})
	-- ({0.8105*\dx},{0.2196*\dy})
	-- ({0.8146*\dx},{0.2158*\dy})
	-- ({0.8186*\dx},{0.2122*\dy})
	-- ({0.8225*\dx},{0.2085*\dy})
	-- ({0.8263*\dx},{0.2050*\dy})
	-- ({0.8300*\dx},{0.2014*\dy})
	-- ({0.8337*\dx},{0.1979*\dy})
	-- ({0.8372*\dx},{0.1945*\dy})
	-- ({0.8407*\dx},{0.1911*\dy})
	-- ({0.8442*\dx},{0.1877*\dy})
	-- ({0.8475*\dx},{0.1844*\dy})
	-- ({0.8508*\dx},{0.1812*\dy})
	-- ({0.8540*\dx},{0.1779*\dy})
	-- ({0.8572*\dx},{0.1748*\dy})
	-- ({0.8602*\dx},{0.1717*\dy})
	-- ({0.8632*\dx},{0.1686*\dy})
	-- ({0.8662*\dx},{0.1655*\dy})
	-- ({0.8691*\dx},{0.1625*\dy})
	-- ({0.8719*\dx},{0.1596*\dy})
	-- ({0.8746*\dx},{0.1567*\dy})
	-- ({0.8773*\dx},{0.1538*\dy})
	-- ({0.8800*\dx},{0.1510*\dy})
	-- ({0.8826*\dx},{0.1483*\dy})
	-- ({0.8851*\dx},{0.1455*\dy})
	-- ({0.8876*\dx},{0.1428*\dy})
	-- ({0.8900*\dx},{0.1402*\dy})
	-- ({0.8923*\dx},{0.1376*\dy})
	-- ({0.8947*\dx},{0.1350*\dy})
	-- ({0.8969*\dx},{0.1325*\dy})
	-- ({0.8991*\dx},{0.1300*\dy})
	-- ({0.9013*\dx},{0.1276*\dy})
	-- ({0.9034*\dx},{0.1252*\dy})
	-- ({0.9055*\dx},{0.1228*\dy})
	-- ({0.9075*\dx},{0.1205*\dy})
	-- ({0.9095*\dx},{0.1183*\dy})
	-- ({0.9114*\dx},{0.1160*\dy})
	-- ({0.9133*\dx},{0.1138*\dy})
	-- ({0.9152*\dx},{0.1116*\dy})
	-- ({0.9170*\dx},{0.1095*\dy})
	-- ({0.9188*\dx},{0.1074*\dy})
	-- ({0.9205*\dx},{0.1054*\dy})
	-- ({0.9222*\dx},{0.1033*\dy})
	-- ({0.9239*\dx},{0.1014*\dy})
	-- ({0.9255*\dx},{0.0994*\dy})
	-- ({0.9271*\dx},{0.0975*\dy})
	-- ({0.9287*\dx},{0.0956*\dy})
	-- ({0.9302*\dx},{0.0938*\dy})
	-- ({0.9317*\dx},{0.0919*\dy})
	-- ({0.9332*\dx},{0.0902*\dy})
	-- ({0.9346*\dx},{0.0884*\dy})
	-- ({0.9360*\dx},{0.0867*\dy})
	-- ({0.9373*\dx},{0.0850*\dy})
	-- ({0.9387*\dx},{0.0833*\dy})
	-- ({0.9400*\dx},{0.0817*\dy})
	-- ({0.9413*\dx},{0.0801*\dy})
	-- ({0.9425*\dx},{0.0785*\dy})
	-- ({0.9437*\dx},{0.0770*\dy})
	-- ({0.9449*\dx},{0.0755*\dy})
	-- ({0.9461*\dx},{0.0740*\dy})
	-- ({0.9473*\dx},{0.0725*\dy})
	-- ({0.9484*\dx},{0.0711*\dy})
	-- ({0.9495*\dx},{0.0697*\dy})
	-- ({0.9506*\dx},{0.0683*\dy})
}
% v = -0.914758
\def\vpathD{
	({-0.9748*\dx},{0.0816*\dy})
	-- ({-0.9742*\dx},{0.0833*\dy})
	-- ({-0.9735*\dx},{0.0850*\dy})
	-- ({-0.9729*\dx},{0.0868*\dy})
	-- ({-0.9723*\dx},{0.0886*\dy})
	-- ({-0.9716*\dx},{0.0904*\dy})
	-- ({-0.9709*\dx},{0.0922*\dy})
	-- ({-0.9702*\dx},{0.0941*\dy})
	-- ({-0.9695*\dx},{0.0961*\dy})
	-- ({-0.9688*\dx},{0.0981*\dy})
	-- ({-0.9680*\dx},{0.1001*\dy})
	-- ({-0.9672*\dx},{0.1021*\dy})
	-- ({-0.9664*\dx},{0.1042*\dy})
	-- ({-0.9656*\dx},{0.1064*\dy})
	-- ({-0.9648*\dx},{0.1085*\dy})
	-- ({-0.9639*\dx},{0.1108*\dy})
	-- ({-0.9630*\dx},{0.1130*\dy})
	-- ({-0.9621*\dx},{0.1153*\dy})
	-- ({-0.9612*\dx},{0.1177*\dy})
	-- ({-0.9602*\dx},{0.1201*\dy})
	-- ({-0.9592*\dx},{0.1225*\dy})
	-- ({-0.9582*\dx},{0.1250*\dy})
	-- ({-0.9572*\dx},{0.1276*\dy})
	-- ({-0.9561*\dx},{0.1302*\dy})
	-- ({-0.9550*\dx},{0.1328*\dy})
	-- ({-0.9539*\dx},{0.1355*\dy})
	-- ({-0.9527*\dx},{0.1382*\dy})
	-- ({-0.9516*\dx},{0.1410*\dy})
	-- ({-0.9503*\dx},{0.1439*\dy})
	-- ({-0.9491*\dx},{0.1468*\dy})
	-- ({-0.9478*\dx},{0.1497*\dy})
	-- ({-0.9465*\dx},{0.1527*\dy})
	-- ({-0.9451*\dx},{0.1558*\dy})
	-- ({-0.9437*\dx},{0.1589*\dy})
	-- ({-0.9423*\dx},{0.1621*\dy})
	-- ({-0.9408*\dx},{0.1654*\dy})
	-- ({-0.9393*\dx},{0.1687*\dy})
	-- ({-0.9377*\dx},{0.1720*\dy})
	-- ({-0.9361*\dx},{0.1754*\dy})
	-- ({-0.9345*\dx},{0.1789*\dy})
	-- ({-0.9328*\dx},{0.1825*\dy})
	-- ({-0.9311*\dx},{0.1861*\dy})
	-- ({-0.9293*\dx},{0.1897*\dy})
	-- ({-0.9274*\dx},{0.1935*\dy})
	-- ({-0.9256*\dx},{0.1973*\dy})
	-- ({-0.9236*\dx},{0.2012*\dy})
	-- ({-0.9216*\dx},{0.2051*\dy})
	-- ({-0.9196*\dx},{0.2091*\dy})
	-- ({-0.9175*\dx},{0.2132*\dy})
	-- ({-0.9153*\dx},{0.2173*\dy})
	-- ({-0.9131*\dx},{0.2215*\dy})
	-- ({-0.9108*\dx},{0.2258*\dy})
	-- ({-0.9084*\dx},{0.2302*\dy})
	-- ({-0.9060*\dx},{0.2346*\dy})
	-- ({-0.9035*\dx},{0.2391*\dy})
	-- ({-0.9009*\dx},{0.2437*\dy})
	-- ({-0.8983*\dx},{0.2483*\dy})
	-- ({-0.8955*\dx},{0.2531*\dy})
	-- ({-0.8928*\dx},{0.2579*\dy})
	-- ({-0.8899*\dx},{0.2627*\dy})
	-- ({-0.8869*\dx},{0.2677*\dy})
	-- ({-0.8839*\dx},{0.2727*\dy})
	-- ({-0.8808*\dx},{0.2778*\dy})
	-- ({-0.8776*\dx},{0.2830*\dy})
	-- ({-0.8743*\dx},{0.2883*\dy})
	-- ({-0.8709*\dx},{0.2936*\dy})
	-- ({-0.8674*\dx},{0.2990*\dy})
	-- ({-0.8638*\dx},{0.3045*\dy})
	-- ({-0.8601*\dx},{0.3101*\dy})
	-- ({-0.8563*\dx},{0.3157*\dy})
	-- ({-0.8524*\dx},{0.3214*\dy})
	-- ({-0.8484*\dx},{0.3272*\dy})
	-- ({-0.8442*\dx},{0.3331*\dy})
	-- ({-0.8400*\dx},{0.3390*\dy})
	-- ({-0.8356*\dx},{0.3451*\dy})
	-- ({-0.8312*\dx},{0.3512*\dy})
	-- ({-0.8266*\dx},{0.3573*\dy})
	-- ({-0.8218*\dx},{0.3636*\dy})
	-- ({-0.8170*\dx},{0.3699*\dy})
	-- ({-0.8120*\dx},{0.3763*\dy})
	-- ({-0.8068*\dx},{0.3827*\dy})
	-- ({-0.8015*\dx},{0.3892*\dy})
	-- ({-0.7961*\dx},{0.3958*\dy})
	-- ({-0.7905*\dx},{0.4025*\dy})
	-- ({-0.7848*\dx},{0.4092*\dy})
	-- ({-0.7790*\dx},{0.4160*\dy})
	-- ({-0.7729*\dx},{0.4228*\dy})
	-- ({-0.7667*\dx},{0.4297*\dy})
	-- ({-0.7604*\dx},{0.4367*\dy})
	-- ({-0.7539*\dx},{0.4437*\dy})
	-- ({-0.7472*\dx},{0.4507*\dy})
	-- ({-0.7403*\dx},{0.4578*\dy})
	-- ({-0.7333*\dx},{0.4650*\dy})
	-- ({-0.7261*\dx},{0.4722*\dy})
	-- ({-0.7187*\dx},{0.4794*\dy})
	-- ({-0.7111*\dx},{0.4866*\dy})
	-- ({-0.7033*\dx},{0.4939*\dy})
	-- ({-0.6953*\dx},{0.5012*\dy})
	-- ({-0.6872*\dx},{0.5086*\dy})
	-- ({-0.6788*\dx},{0.5159*\dy})
	-- ({-0.6702*\dx},{0.5233*\dy})
	-- ({-0.6615*\dx},{0.5307*\dy})
	-- ({-0.6525*\dx},{0.5380*\dy})
	-- ({-0.6433*\dx},{0.5454*\dy})
	-- ({-0.6339*\dx},{0.5528*\dy})
	-- ({-0.6243*\dx},{0.5601*\dy})
	-- ({-0.6145*\dx},{0.5675*\dy})
	-- ({-0.6045*\dx},{0.5748*\dy})
	-- ({-0.5942*\dx},{0.5821*\dy})
	-- ({-0.5838*\dx},{0.5893*\dy})
	-- ({-0.5731*\dx},{0.5965*\dy})
	-- ({-0.5622*\dx},{0.6037*\dy})
	-- ({-0.5510*\dx},{0.6108*\dy})
	-- ({-0.5397*\dx},{0.6178*\dy})
	-- ({-0.5281*\dx},{0.6248*\dy})
	-- ({-0.5163*\dx},{0.6317*\dy})
	-- ({-0.5043*\dx},{0.6385*\dy})
	-- ({-0.4921*\dx},{0.6452*\dy})
	-- ({-0.4797*\dx},{0.6518*\dy})
	-- ({-0.4670*\dx},{0.6584*\dy})
	-- ({-0.4541*\dx},{0.6648*\dy})
	-- ({-0.4410*\dx},{0.6710*\dy})
	-- ({-0.4277*\dx},{0.6772*\dy})
	-- ({-0.4142*\dx},{0.6832*\dy})
	-- ({-0.4005*\dx},{0.6891*\dy})
	-- ({-0.3866*\dx},{0.6948*\dy})
	-- ({-0.3725*\dx},{0.7004*\dy})
	-- ({-0.3582*\dx},{0.7058*\dy})
	-- ({-0.3437*\dx},{0.7110*\dy})
	-- ({-0.3290*\dx},{0.7160*\dy})
	-- ({-0.3142*\dx},{0.7209*\dy})
	-- ({-0.2992*\dx},{0.7256*\dy})
	-- ({-0.2840*\dx},{0.7300*\dy})
	-- ({-0.2687*\dx},{0.7343*\dy})
	-- ({-0.2532*\dx},{0.7383*\dy})
	-- ({-0.2376*\dx},{0.7421*\dy})
	-- ({-0.2218*\dx},{0.7457*\dy})
	-- ({-0.2059*\dx},{0.7491*\dy})
	-- ({-0.1899*\dx},{0.7522*\dy})
	-- ({-0.1737*\dx},{0.7551*\dy})
	-- ({-0.1575*\dx},{0.7577*\dy})
	-- ({-0.1412*\dx},{0.7601*\dy})
	-- ({-0.1248*\dx},{0.7622*\dy})
	-- ({-0.1083*\dx},{0.7641*\dy})
	-- ({-0.0917*\dx},{0.7657*\dy})
	-- ({-0.0751*\dx},{0.7670*\dy})
	-- ({-0.0585*\dx},{0.7681*\dy})
	-- ({-0.0418*\dx},{0.7689*\dy})
	-- ({-0.0251*\dx},{0.7695*\dy})
	-- ({-0.0084*\dx},{0.7697*\dy})
	-- ({0.0084*\dx},{0.7697*\dy})
	-- ({0.0251*\dx},{0.7695*\dy})
	-- ({0.0418*\dx},{0.7689*\dy})
	-- ({0.0585*\dx},{0.7681*\dy})
	-- ({0.0751*\dx},{0.7670*\dy})
	-- ({0.0917*\dx},{0.7657*\dy})
	-- ({0.1083*\dx},{0.7641*\dy})
	-- ({0.1248*\dx},{0.7622*\dy})
	-- ({0.1412*\dx},{0.7601*\dy})
	-- ({0.1575*\dx},{0.7577*\dy})
	-- ({0.1737*\dx},{0.7551*\dy})
	-- ({0.1899*\dx},{0.7522*\dy})
	-- ({0.2059*\dx},{0.7491*\dy})
	-- ({0.2218*\dx},{0.7457*\dy})
	-- ({0.2376*\dx},{0.7421*\dy})
	-- ({0.2532*\dx},{0.7383*\dy})
	-- ({0.2687*\dx},{0.7343*\dy})
	-- ({0.2840*\dx},{0.7300*\dy})
	-- ({0.2992*\dx},{0.7256*\dy})
	-- ({0.3142*\dx},{0.7209*\dy})
	-- ({0.3290*\dx},{0.7160*\dy})
	-- ({0.3437*\dx},{0.7110*\dy})
	-- ({0.3582*\dx},{0.7058*\dy})
	-- ({0.3725*\dx},{0.7004*\dy})
	-- ({0.3866*\dx},{0.6948*\dy})
	-- ({0.4005*\dx},{0.6891*\dy})
	-- ({0.4142*\dx},{0.6832*\dy})
	-- ({0.4277*\dx},{0.6772*\dy})
	-- ({0.4410*\dx},{0.6710*\dy})
	-- ({0.4541*\dx},{0.6648*\dy})
	-- ({0.4670*\dx},{0.6584*\dy})
	-- ({0.4797*\dx},{0.6518*\dy})
	-- ({0.4921*\dx},{0.6452*\dy})
	-- ({0.5043*\dx},{0.6385*\dy})
	-- ({0.5163*\dx},{0.6317*\dy})
	-- ({0.5281*\dx},{0.6248*\dy})
	-- ({0.5397*\dx},{0.6178*\dy})
	-- ({0.5510*\dx},{0.6108*\dy})
	-- ({0.5622*\dx},{0.6037*\dy})
	-- ({0.5731*\dx},{0.5965*\dy})
	-- ({0.5838*\dx},{0.5893*\dy})
	-- ({0.5942*\dx},{0.5821*\dy})
	-- ({0.6045*\dx},{0.5748*\dy})
	-- ({0.6145*\dx},{0.5675*\dy})
	-- ({0.6243*\dx},{0.5601*\dy})
	-- ({0.6339*\dx},{0.5528*\dy})
	-- ({0.6433*\dx},{0.5454*\dy})
	-- ({0.6525*\dx},{0.5380*\dy})
	-- ({0.6615*\dx},{0.5307*\dy})
	-- ({0.6702*\dx},{0.5233*\dy})
	-- ({0.6788*\dx},{0.5159*\dy})
	-- ({0.6872*\dx},{0.5086*\dy})
	-- ({0.6953*\dx},{0.5012*\dy})
	-- ({0.7033*\dx},{0.4939*\dy})
	-- ({0.7111*\dx},{0.4866*\dy})
	-- ({0.7187*\dx},{0.4794*\dy})
	-- ({0.7261*\dx},{0.4722*\dy})
	-- ({0.7333*\dx},{0.4650*\dy})
	-- ({0.7403*\dx},{0.4578*\dy})
	-- ({0.7472*\dx},{0.4507*\dy})
	-- ({0.7539*\dx},{0.4437*\dy})
	-- ({0.7604*\dx},{0.4367*\dy})
	-- ({0.7667*\dx},{0.4297*\dy})
	-- ({0.7729*\dx},{0.4228*\dy})
	-- ({0.7790*\dx},{0.4160*\dy})
	-- ({0.7848*\dx},{0.4092*\dy})
	-- ({0.7905*\dx},{0.4025*\dy})
	-- ({0.7961*\dx},{0.3958*\dy})
	-- ({0.8015*\dx},{0.3892*\dy})
	-- ({0.8068*\dx},{0.3827*\dy})
	-- ({0.8120*\dx},{0.3763*\dy})
	-- ({0.8170*\dx},{0.3699*\dy})
	-- ({0.8218*\dx},{0.3636*\dy})
	-- ({0.8266*\dx},{0.3573*\dy})
	-- ({0.8312*\dx},{0.3512*\dy})
	-- ({0.8356*\dx},{0.3451*\dy})
	-- ({0.8400*\dx},{0.3390*\dy})
	-- ({0.8442*\dx},{0.3331*\dy})
	-- ({0.8484*\dx},{0.3272*\dy})
	-- ({0.8524*\dx},{0.3214*\dy})
	-- ({0.8563*\dx},{0.3157*\dy})
	-- ({0.8601*\dx},{0.3101*\dy})
	-- ({0.8638*\dx},{0.3045*\dy})
	-- ({0.8674*\dx},{0.2990*\dy})
	-- ({0.8709*\dx},{0.2936*\dy})
	-- ({0.8743*\dx},{0.2883*\dy})
	-- ({0.8776*\dx},{0.2830*\dy})
	-- ({0.8808*\dx},{0.2778*\dy})
	-- ({0.8839*\dx},{0.2727*\dy})
	-- ({0.8869*\dx},{0.2677*\dy})
	-- ({0.8899*\dx},{0.2627*\dy})
	-- ({0.8928*\dx},{0.2579*\dy})
	-- ({0.8955*\dx},{0.2531*\dy})
	-- ({0.8983*\dx},{0.2483*\dy})
	-- ({0.9009*\dx},{0.2437*\dy})
	-- ({0.9035*\dx},{0.2391*\dy})
	-- ({0.9060*\dx},{0.2346*\dy})
	-- ({0.9084*\dx},{0.2302*\dy})
	-- ({0.9108*\dx},{0.2258*\dy})
	-- ({0.9131*\dx},{0.2215*\dy})
	-- ({0.9153*\dx},{0.2173*\dy})
	-- ({0.9175*\dx},{0.2132*\dy})
	-- ({0.9196*\dx},{0.2091*\dy})
	-- ({0.9216*\dx},{0.2051*\dy})
	-- ({0.9236*\dx},{0.2012*\dy})
	-- ({0.9256*\dx},{0.1973*\dy})
	-- ({0.9274*\dx},{0.1935*\dy})
	-- ({0.9293*\dx},{0.1897*\dy})
	-- ({0.9311*\dx},{0.1861*\dy})
	-- ({0.9328*\dx},{0.1825*\dy})
	-- ({0.9345*\dx},{0.1789*\dy})
	-- ({0.9361*\dx},{0.1754*\dy})
	-- ({0.9377*\dx},{0.1720*\dy})
	-- ({0.9393*\dx},{0.1687*\dy})
	-- ({0.9408*\dx},{0.1654*\dy})
	-- ({0.9423*\dx},{0.1621*\dy})
	-- ({0.9437*\dx},{0.1589*\dy})
	-- ({0.9451*\dx},{0.1558*\dy})
	-- ({0.9465*\dx},{0.1527*\dy})
	-- ({0.9478*\dx},{0.1497*\dy})
	-- ({0.9491*\dx},{0.1468*\dy})
	-- ({0.9503*\dx},{0.1439*\dy})
	-- ({0.9516*\dx},{0.1410*\dy})
	-- ({0.9527*\dx},{0.1382*\dy})
	-- ({0.9539*\dx},{0.1355*\dy})
	-- ({0.9550*\dx},{0.1328*\dy})
	-- ({0.9561*\dx},{0.1302*\dy})
	-- ({0.9572*\dx},{0.1276*\dy})
	-- ({0.9582*\dx},{0.1250*\dy})
	-- ({0.9592*\dx},{0.1225*\dy})
	-- ({0.9602*\dx},{0.1201*\dy})
	-- ({0.9612*\dx},{0.1177*\dy})
	-- ({0.9621*\dx},{0.1153*\dy})
	-- ({0.9630*\dx},{0.1130*\dy})
	-- ({0.9639*\dx},{0.1108*\dy})
	-- ({0.9648*\dx},{0.1085*\dy})
	-- ({0.9656*\dx},{0.1064*\dy})
	-- ({0.9664*\dx},{0.1042*\dy})
	-- ({0.9672*\dx},{0.1021*\dy})
	-- ({0.9680*\dx},{0.1001*\dy})
	-- ({0.9688*\dx},{0.0981*\dy})
	-- ({0.9695*\dx},{0.0961*\dy})
	-- ({0.9702*\dx},{0.0941*\dy})
	-- ({0.9709*\dx},{0.0922*\dy})
	-- ({0.9716*\dx},{0.0904*\dy})
	-- ({0.9723*\dx},{0.0886*\dy})
	-- ({0.9729*\dx},{0.0868*\dy})
	-- ({0.9735*\dx},{0.0850*\dy})
	-- ({0.9742*\dx},{0.0833*\dy})
	-- ({0.9748*\dx},{0.0816*\dy})
}
% v = -0.748438
\def\vpathE{
	({-1.0027*\dx},{0.0866*\dy})
	-- ({-1.0026*\dx},{0.0884*\dy})
	-- ({-1.0026*\dx},{0.0903*\dy})
	-- ({-1.0026*\dx},{0.0922*\dy})
	-- ({-1.0025*\dx},{0.0942*\dy})
	-- ({-1.0025*\dx},{0.0962*\dy})
	-- ({-1.0024*\dx},{0.0982*\dy})
	-- ({-1.0024*\dx},{0.1003*\dy})
	-- ({-1.0023*\dx},{0.1025*\dy})
	-- ({-1.0023*\dx},{0.1047*\dy})
	-- ({-1.0022*\dx},{0.1069*\dy})
	-- ({-1.0021*\dx},{0.1092*\dy})
	-- ({-1.0020*\dx},{0.1115*\dy})
	-- ({-1.0019*\dx},{0.1139*\dy})
	-- ({-1.0018*\dx},{0.1163*\dy})
	-- ({-1.0017*\dx},{0.1188*\dy})
	-- ({-1.0016*\dx},{0.1213*\dy})
	-- ({-1.0015*\dx},{0.1239*\dy})
	-- ({-1.0014*\dx},{0.1265*\dy})
	-- ({-1.0012*\dx},{0.1292*\dy})
	-- ({-1.0011*\dx},{0.1319*\dy})
	-- ({-1.0009*\dx},{0.1347*\dy})
	-- ({-1.0007*\dx},{0.1376*\dy})
	-- ({-1.0005*\dx},{0.1405*\dy})
	-- ({-1.0003*\dx},{0.1435*\dy})
	-- ({-1.0001*\dx},{0.1466*\dy})
	-- ({-0.9999*\dx},{0.1497*\dy})
	-- ({-0.9996*\dx},{0.1528*\dy})
	-- ({-0.9994*\dx},{0.1561*\dy})
	-- ({-0.9991*\dx},{0.1594*\dy})
	-- ({-0.9988*\dx},{0.1628*\dy})
	-- ({-0.9985*\dx},{0.1662*\dy})
	-- ({-0.9982*\dx},{0.1698*\dy})
	-- ({-0.9978*\dx},{0.1734*\dy})
	-- ({-0.9974*\dx},{0.1770*\dy})
	-- ({-0.9970*\dx},{0.1808*\dy})
	-- ({-0.9966*\dx},{0.1846*\dy})
	-- ({-0.9962*\dx},{0.1885*\dy})
	-- ({-0.9957*\dx},{0.1925*\dy})
	-- ({-0.9952*\dx},{0.1966*\dy})
	-- ({-0.9947*\dx},{0.2007*\dy})
	-- ({-0.9942*\dx},{0.2050*\dy})
	-- ({-0.9936*\dx},{0.2093*\dy})
	-- ({-0.9930*\dx},{0.2137*\dy})
	-- ({-0.9923*\dx},{0.2182*\dy})
	-- ({-0.9916*\dx},{0.2228*\dy})
	-- ({-0.9909*\dx},{0.2275*\dy})
	-- ({-0.9902*\dx},{0.2323*\dy})
	-- ({-0.9894*\dx},{0.2372*\dy})
	-- ({-0.9886*\dx},{0.2421*\dy})
	-- ({-0.9877*\dx},{0.2472*\dy})
	-- ({-0.9868*\dx},{0.2524*\dy})
	-- ({-0.9858*\dx},{0.2577*\dy})
	-- ({-0.9848*\dx},{0.2631*\dy})
	-- ({-0.9837*\dx},{0.2686*\dy})
	-- ({-0.9826*\dx},{0.2742*\dy})
	-- ({-0.9814*\dx},{0.2799*\dy})
	-- ({-0.9801*\dx},{0.2857*\dy})
	-- ({-0.9788*\dx},{0.2917*\dy})
	-- ({-0.9775*\dx},{0.2977*\dy})
	-- ({-0.9760*\dx},{0.3039*\dy})
	-- ({-0.9745*\dx},{0.3102*\dy})
	-- ({-0.9730*\dx},{0.3166*\dy})
	-- ({-0.9713*\dx},{0.3231*\dy})
	-- ({-0.9696*\dx},{0.3298*\dy})
	-- ({-0.9678*\dx},{0.3366*\dy})
	-- ({-0.9659*\dx},{0.3435*\dy})
	-- ({-0.9639*\dx},{0.3505*\dy})
	-- ({-0.9618*\dx},{0.3577*\dy})
	-- ({-0.9596*\dx},{0.3650*\dy})
	-- ({-0.9573*\dx},{0.3724*\dy})
	-- ({-0.9549*\dx},{0.3800*\dy})
	-- ({-0.9524*\dx},{0.3877*\dy})
	-- ({-0.9498*\dx},{0.3955*\dy})
	-- ({-0.9471*\dx},{0.4034*\dy})
	-- ({-0.9442*\dx},{0.4115*\dy})
	-- ({-0.9413*\dx},{0.4198*\dy})
	-- ({-0.9381*\dx},{0.4281*\dy})
	-- ({-0.9349*\dx},{0.4367*\dy})
	-- ({-0.9315*\dx},{0.4453*\dy})
	-- ({-0.9279*\dx},{0.4541*\dy})
	-- ({-0.9242*\dx},{0.4630*\dy})
	-- ({-0.9204*\dx},{0.4721*\dy})
	-- ({-0.9163*\dx},{0.4813*\dy})
	-- ({-0.9121*\dx},{0.4906*\dy})
	-- ({-0.9077*\dx},{0.5001*\dy})
	-- ({-0.9032*\dx},{0.5097*\dy})
	-- ({-0.8984*\dx},{0.5194*\dy})
	-- ({-0.8934*\dx},{0.5293*\dy})
	-- ({-0.8883*\dx},{0.5393*\dy})
	-- ({-0.8829*\dx},{0.5494*\dy})
	-- ({-0.8773*\dx},{0.5597*\dy})
	-- ({-0.8715*\dx},{0.5701*\dy})
	-- ({-0.8654*\dx},{0.5806*\dy})
	-- ({-0.8591*\dx},{0.5912*\dy})
	-- ({-0.8526*\dx},{0.6019*\dy})
	-- ({-0.8458*\dx},{0.6127*\dy})
	-- ({-0.8387*\dx},{0.6237*\dy})
	-- ({-0.8313*\dx},{0.6347*\dy})
	-- ({-0.8237*\dx},{0.6459*\dy})
	-- ({-0.8158*\dx},{0.6571*\dy})
	-- ({-0.8076*\dx},{0.6684*\dy})
	-- ({-0.7991*\dx},{0.6798*\dy})
	-- ({-0.7903*\dx},{0.6912*\dy})
	-- ({-0.7812*\dx},{0.7027*\dy})
	-- ({-0.7717*\dx},{0.7143*\dy})
	-- ({-0.7620*\dx},{0.7259*\dy})
	-- ({-0.7519*\dx},{0.7375*\dy})
	-- ({-0.7414*\dx},{0.7492*\dy})
	-- ({-0.7306*\dx},{0.7609*\dy})
	-- ({-0.7194*\dx},{0.7725*\dy})
	-- ({-0.7079*\dx},{0.7842*\dy})
	-- ({-0.6960*\dx},{0.7959*\dy})
	-- ({-0.6838*\dx},{0.8075*\dy})
	-- ({-0.6711*\dx},{0.8191*\dy})
	-- ({-0.6581*\dx},{0.8306*\dy})
	-- ({-0.6447*\dx},{0.8420*\dy})
	-- ({-0.6309*\dx},{0.8534*\dy})
	-- ({-0.6168*\dx},{0.8647*\dy})
	-- ({-0.6022*\dx},{0.8758*\dy})
	-- ({-0.5873*\dx},{0.8868*\dy})
	-- ({-0.5719*\dx},{0.8977*\dy})
	-- ({-0.5562*\dx},{0.9084*\dy})
	-- ({-0.5401*\dx},{0.9189*\dy})
	-- ({-0.5236*\dx},{0.9293*\dy})
	-- ({-0.5067*\dx},{0.9394*\dy})
	-- ({-0.4894*\dx},{0.9493*\dy})
	-- ({-0.4717*\dx},{0.9589*\dy})
	-- ({-0.4537*\dx},{0.9683*\dy})
	-- ({-0.4354*\dx},{0.9774*\dy})
	-- ({-0.4166*\dx},{0.9861*\dy})
	-- ({-0.3976*\dx},{0.9946*\dy})
	-- ({-0.3782*\dx},{1.0028*\dy})
	-- ({-0.3584*\dx},{1.0105*\dy})
	-- ({-0.3384*\dx},{1.0180*\dy})
	-- ({-0.3180*\dx},{1.0250*\dy})
	-- ({-0.2974*\dx},{1.0317*\dy})
	-- ({-0.2765*\dx},{1.0379*\dy})
	-- ({-0.2554*\dx},{1.0437*\dy})
	-- ({-0.2340*\dx},{1.0491*\dy})
	-- ({-0.2124*\dx},{1.0540*\dy})
	-- ({-0.1906*\dx},{1.0585*\dy})
	-- ({-0.1686*\dx},{1.0625*\dy})
	-- ({-0.1465*\dx},{1.0660*\dy})
	-- ({-0.1242*\dx},{1.0691*\dy})
	-- ({-0.1018*\dx},{1.0716*\dy})
	-- ({-0.0793*\dx},{1.0737*\dy})
	-- ({-0.0567*\dx},{1.0752*\dy})
	-- ({-0.0340*\dx},{1.0762*\dy})
	-- ({-0.0113*\dx},{1.0767*\dy})
	-- ({0.0113*\dx},{1.0767*\dy})
	-- ({0.0340*\dx},{1.0762*\dy})
	-- ({0.0567*\dx},{1.0752*\dy})
	-- ({0.0793*\dx},{1.0737*\dy})
	-- ({0.1018*\dx},{1.0716*\dy})
	-- ({0.1242*\dx},{1.0691*\dy})
	-- ({0.1465*\dx},{1.0660*\dy})
	-- ({0.1686*\dx},{1.0625*\dy})
	-- ({0.1906*\dx},{1.0585*\dy})
	-- ({0.2124*\dx},{1.0540*\dy})
	-- ({0.2340*\dx},{1.0491*\dy})
	-- ({0.2554*\dx},{1.0437*\dy})
	-- ({0.2765*\dx},{1.0379*\dy})
	-- ({0.2974*\dx},{1.0317*\dy})
	-- ({0.3180*\dx},{1.0250*\dy})
	-- ({0.3384*\dx},{1.0180*\dy})
	-- ({0.3584*\dx},{1.0105*\dy})
	-- ({0.3782*\dx},{1.0028*\dy})
	-- ({0.3976*\dx},{0.9946*\dy})
	-- ({0.4166*\dx},{0.9861*\dy})
	-- ({0.4354*\dx},{0.9774*\dy})
	-- ({0.4537*\dx},{0.9683*\dy})
	-- ({0.4717*\dx},{0.9589*\dy})
	-- ({0.4894*\dx},{0.9493*\dy})
	-- ({0.5067*\dx},{0.9394*\dy})
	-- ({0.5236*\dx},{0.9293*\dy})
	-- ({0.5401*\dx},{0.9189*\dy})
	-- ({0.5562*\dx},{0.9084*\dy})
	-- ({0.5719*\dx},{0.8977*\dy})
	-- ({0.5873*\dx},{0.8868*\dy})
	-- ({0.6022*\dx},{0.8758*\dy})
	-- ({0.6168*\dx},{0.8647*\dy})
	-- ({0.6309*\dx},{0.8534*\dy})
	-- ({0.6447*\dx},{0.8420*\dy})
	-- ({0.6581*\dx},{0.8306*\dy})
	-- ({0.6711*\dx},{0.8191*\dy})
	-- ({0.6838*\dx},{0.8075*\dy})
	-- ({0.6960*\dx},{0.7959*\dy})
	-- ({0.7079*\dx},{0.7842*\dy})
	-- ({0.7194*\dx},{0.7725*\dy})
	-- ({0.7306*\dx},{0.7609*\dy})
	-- ({0.7414*\dx},{0.7492*\dy})
	-- ({0.7519*\dx},{0.7375*\dy})
	-- ({0.7620*\dx},{0.7259*\dy})
	-- ({0.7717*\dx},{0.7143*\dy})
	-- ({0.7812*\dx},{0.7027*\dy})
	-- ({0.7903*\dx},{0.6912*\dy})
	-- ({0.7991*\dx},{0.6798*\dy})
	-- ({0.8076*\dx},{0.6684*\dy})
	-- ({0.8158*\dx},{0.6571*\dy})
	-- ({0.8237*\dx},{0.6459*\dy})
	-- ({0.8313*\dx},{0.6347*\dy})
	-- ({0.8387*\dx},{0.6237*\dy})
	-- ({0.8458*\dx},{0.6127*\dy})
	-- ({0.8526*\dx},{0.6019*\dy})
	-- ({0.8591*\dx},{0.5912*\dy})
	-- ({0.8654*\dx},{0.5806*\dy})
	-- ({0.8715*\dx},{0.5701*\dy})
	-- ({0.8773*\dx},{0.5597*\dy})
	-- ({0.8829*\dx},{0.5494*\dy})
	-- ({0.8883*\dx},{0.5393*\dy})
	-- ({0.8934*\dx},{0.5293*\dy})
	-- ({0.8984*\dx},{0.5194*\dy})
	-- ({0.9032*\dx},{0.5097*\dy})
	-- ({0.9077*\dx},{0.5001*\dy})
	-- ({0.9121*\dx},{0.4906*\dy})
	-- ({0.9163*\dx},{0.4813*\dy})
	-- ({0.9204*\dx},{0.4721*\dy})
	-- ({0.9242*\dx},{0.4630*\dy})
	-- ({0.9279*\dx},{0.4541*\dy})
	-- ({0.9315*\dx},{0.4453*\dy})
	-- ({0.9349*\dx},{0.4367*\dy})
	-- ({0.9381*\dx},{0.4281*\dy})
	-- ({0.9413*\dx},{0.4198*\dy})
	-- ({0.9442*\dx},{0.4115*\dy})
	-- ({0.9471*\dx},{0.4034*\dy})
	-- ({0.9498*\dx},{0.3955*\dy})
	-- ({0.9524*\dx},{0.3877*\dy})
	-- ({0.9549*\dx},{0.3800*\dy})
	-- ({0.9573*\dx},{0.3724*\dy})
	-- ({0.9596*\dx},{0.3650*\dy})
	-- ({0.9618*\dx},{0.3577*\dy})
	-- ({0.9639*\dx},{0.3505*\dy})
	-- ({0.9659*\dx},{0.3435*\dy})
	-- ({0.9678*\dx},{0.3366*\dy})
	-- ({0.9696*\dx},{0.3298*\dy})
	-- ({0.9713*\dx},{0.3231*\dy})
	-- ({0.9730*\dx},{0.3166*\dy})
	-- ({0.9745*\dx},{0.3102*\dy})
	-- ({0.9760*\dx},{0.3039*\dy})
	-- ({0.9775*\dx},{0.2977*\dy})
	-- ({0.9788*\dx},{0.2917*\dy})
	-- ({0.9801*\dx},{0.2857*\dy})
	-- ({0.9814*\dx},{0.2799*\dy})
	-- ({0.9826*\dx},{0.2742*\dy})
	-- ({0.9837*\dx},{0.2686*\dy})
	-- ({0.9848*\dx},{0.2631*\dy})
	-- ({0.9858*\dx},{0.2577*\dy})
	-- ({0.9868*\dx},{0.2524*\dy})
	-- ({0.9877*\dx},{0.2472*\dy})
	-- ({0.9886*\dx},{0.2421*\dy})
	-- ({0.9894*\dx},{0.2372*\dy})
	-- ({0.9902*\dx},{0.2323*\dy})
	-- ({0.9909*\dx},{0.2275*\dy})
	-- ({0.9916*\dx},{0.2228*\dy})
	-- ({0.9923*\dx},{0.2182*\dy})
	-- ({0.9930*\dx},{0.2137*\dy})
	-- ({0.9936*\dx},{0.2093*\dy})
	-- ({0.9942*\dx},{0.2050*\dy})
	-- ({0.9947*\dx},{0.2007*\dy})
	-- ({0.9952*\dx},{0.1966*\dy})
	-- ({0.9957*\dx},{0.1925*\dy})
	-- ({0.9962*\dx},{0.1885*\dy})
	-- ({0.9966*\dx},{0.1846*\dy})
	-- ({0.9970*\dx},{0.1808*\dy})
	-- ({0.9974*\dx},{0.1770*\dy})
	-- ({0.9978*\dx},{0.1734*\dy})
	-- ({0.9982*\dx},{0.1698*\dy})
	-- ({0.9985*\dx},{0.1662*\dy})
	-- ({0.9988*\dx},{0.1628*\dy})
	-- ({0.9991*\dx},{0.1594*\dy})
	-- ({0.9994*\dx},{0.1561*\dy})
	-- ({0.9996*\dx},{0.1528*\dy})
	-- ({0.9999*\dx},{0.1497*\dy})
	-- ({1.0001*\dx},{0.1466*\dy})
	-- ({1.0003*\dx},{0.1435*\dy})
	-- ({1.0005*\dx},{0.1405*\dy})
	-- ({1.0007*\dx},{0.1376*\dy})
	-- ({1.0009*\dx},{0.1347*\dy})
	-- ({1.0011*\dx},{0.1319*\dy})
	-- ({1.0012*\dx},{0.1292*\dy})
	-- ({1.0014*\dx},{0.1265*\dy})
	-- ({1.0015*\dx},{0.1239*\dy})
	-- ({1.0016*\dx},{0.1213*\dy})
	-- ({1.0017*\dx},{0.1188*\dy})
	-- ({1.0018*\dx},{0.1163*\dy})
	-- ({1.0019*\dx},{0.1139*\dy})
	-- ({1.0020*\dx},{0.1115*\dy})
	-- ({1.0021*\dx},{0.1092*\dy})
	-- ({1.0022*\dx},{0.1069*\dy})
	-- ({1.0023*\dx},{0.1047*\dy})
	-- ({1.0023*\dx},{0.1025*\dy})
	-- ({1.0024*\dx},{0.1003*\dy})
	-- ({1.0024*\dx},{0.0982*\dy})
	-- ({1.0025*\dx},{0.0962*\dy})
	-- ({1.0025*\dx},{0.0942*\dy})
	-- ({1.0026*\dx},{0.0922*\dy})
	-- ({1.0026*\dx},{0.0903*\dy})
	-- ({1.0026*\dx},{0.0884*\dy})
	-- ({1.0027*\dx},{0.0866*\dy})
}
% v = -0.582119
\def\vpathF{
	({-1.0315*\dx},{0.0820*\dy})
	-- ({-1.0321*\dx},{0.0838*\dy})
	-- ({-1.0327*\dx},{0.0857*\dy})
	-- ({-1.0333*\dx},{0.0875*\dy})
	-- ({-1.0339*\dx},{0.0895*\dy})
	-- ({-1.0346*\dx},{0.0914*\dy})
	-- ({-1.0352*\dx},{0.0934*\dy})
	-- ({-1.0359*\dx},{0.0955*\dy})
	-- ({-1.0366*\dx},{0.0976*\dy})
	-- ({-1.0373*\dx},{0.0998*\dy})
	-- ({-1.0380*\dx},{0.1020*\dy})
	-- ({-1.0387*\dx},{0.1042*\dy})
	-- ({-1.0394*\dx},{0.1065*\dy})
	-- ({-1.0401*\dx},{0.1089*\dy})
	-- ({-1.0409*\dx},{0.1113*\dy})
	-- ({-1.0416*\dx},{0.1137*\dy})
	-- ({-1.0424*\dx},{0.1162*\dy})
	-- ({-1.0432*\dx},{0.1188*\dy})
	-- ({-1.0440*\dx},{0.1215*\dy})
	-- ({-1.0447*\dx},{0.1241*\dy})
	-- ({-1.0455*\dx},{0.1269*\dy})
	-- ({-1.0464*\dx},{0.1297*\dy})
	-- ({-1.0472*\dx},{0.1326*\dy})
	-- ({-1.0480*\dx},{0.1356*\dy})
	-- ({-1.0489*\dx},{0.1386*\dy})
	-- ({-1.0497*\dx},{0.1417*\dy})
	-- ({-1.0506*\dx},{0.1448*\dy})
	-- ({-1.0515*\dx},{0.1481*\dy})
	-- ({-1.0523*\dx},{0.1514*\dy})
	-- ({-1.0532*\dx},{0.1548*\dy})
	-- ({-1.0541*\dx},{0.1582*\dy})
	-- ({-1.0551*\dx},{0.1618*\dy})
	-- ({-1.0560*\dx},{0.1654*\dy})
	-- ({-1.0569*\dx},{0.1691*\dy})
	-- ({-1.0578*\dx},{0.1729*\dy})
	-- ({-1.0588*\dx},{0.1768*\dy})
	-- ({-1.0597*\dx},{0.1808*\dy})
	-- ({-1.0607*\dx},{0.1849*\dy})
	-- ({-1.0616*\dx},{0.1890*\dy})
	-- ({-1.0626*\dx},{0.1933*\dy})
	-- ({-1.0635*\dx},{0.1977*\dy})
	-- ({-1.0645*\dx},{0.2021*\dy})
	-- ({-1.0655*\dx},{0.2067*\dy})
	-- ({-1.0665*\dx},{0.2114*\dy})
	-- ({-1.0674*\dx},{0.2162*\dy})
	-- ({-1.0684*\dx},{0.2211*\dy})
	-- ({-1.0694*\dx},{0.2261*\dy})
	-- ({-1.0704*\dx},{0.2312*\dy})
	-- ({-1.0713*\dx},{0.2365*\dy})
	-- ({-1.0723*\dx},{0.2419*\dy})
	-- ({-1.0732*\dx},{0.2474*\dy})
	-- ({-1.0742*\dx},{0.2531*\dy})
	-- ({-1.0751*\dx},{0.2588*\dy})
	-- ({-1.0760*\dx},{0.2648*\dy})
	-- ({-1.0770*\dx},{0.2708*\dy})
	-- ({-1.0779*\dx},{0.2770*\dy})
	-- ({-1.0788*\dx},{0.2834*\dy})
	-- ({-1.0796*\dx},{0.2899*\dy})
	-- ({-1.0805*\dx},{0.2965*\dy})
	-- ({-1.0813*\dx},{0.3033*\dy})
	-- ({-1.0821*\dx},{0.3103*\dy})
	-- ({-1.0829*\dx},{0.3174*\dy})
	-- ({-1.0836*\dx},{0.3247*\dy})
	-- ({-1.0843*\dx},{0.3322*\dy})
	-- ({-1.0850*\dx},{0.3399*\dy})
	-- ({-1.0856*\dx},{0.3477*\dy})
	-- ({-1.0862*\dx},{0.3558*\dy})
	-- ({-1.0867*\dx},{0.3640*\dy})
	-- ({-1.0872*\dx},{0.3724*\dy})
	-- ({-1.0876*\dx},{0.3810*\dy})
	-- ({-1.0880*\dx},{0.3898*\dy})
	-- ({-1.0883*\dx},{0.3988*\dy})
	-- ({-1.0885*\dx},{0.4080*\dy})
	-- ({-1.0887*\dx},{0.4175*\dy})
	-- ({-1.0887*\dx},{0.4271*\dy})
	-- ({-1.0887*\dx},{0.4370*\dy})
	-- ({-1.0886*\dx},{0.4471*\dy})
	-- ({-1.0884*\dx},{0.4575*\dy})
	-- ({-1.0881*\dx},{0.4681*\dy})
	-- ({-1.0877*\dx},{0.4789*\dy})
	-- ({-1.0871*\dx},{0.4900*\dy})
	-- ({-1.0864*\dx},{0.5013*\dy})
	-- ({-1.0856*\dx},{0.5129*\dy})
	-- ({-1.0847*\dx},{0.5247*\dy})
	-- ({-1.0836*\dx},{0.5368*\dy})
	-- ({-1.0823*\dx},{0.5491*\dy})
	-- ({-1.0808*\dx},{0.5618*\dy})
	-- ({-1.0792*\dx},{0.5746*\dy})
	-- ({-1.0773*\dx},{0.5878*\dy})
	-- ({-1.0753*\dx},{0.6013*\dy})
	-- ({-1.0730*\dx},{0.6150*\dy})
	-- ({-1.0705*\dx},{0.6290*\dy})
	-- ({-1.0678*\dx},{0.6433*\dy})
	-- ({-1.0648*\dx},{0.6579*\dy})
	-- ({-1.0615*\dx},{0.6727*\dy})
	-- ({-1.0579*\dx},{0.6879*\dy})
	-- ({-1.0540*\dx},{0.7033*\dy})
	-- ({-1.0498*\dx},{0.7190*\dy})
	-- ({-1.0453*\dx},{0.7350*\dy})
	-- ({-1.0404*\dx},{0.7513*\dy})
	-- ({-1.0352*\dx},{0.7679*\dy})
	-- ({-1.0295*\dx},{0.7847*\dy})
	-- ({-1.0235*\dx},{0.8018*\dy})
	-- ({-1.0170*\dx},{0.8192*\dy})
	-- ({-1.0101*\dx},{0.8369*\dy})
	-- ({-1.0027*\dx},{0.8547*\dy})
	-- ({-0.9948*\dx},{0.8729*\dy})
	-- ({-0.9865*\dx},{0.8912*\dy})
	-- ({-0.9776*\dx},{0.9098*\dy})
	-- ({-0.9681*\dx},{0.9286*\dy})
	-- ({-0.9581*\dx},{0.9476*\dy})
	-- ({-0.9475*\dx},{0.9668*\dy})
	-- ({-0.9363*\dx},{0.9861*\dy})
	-- ({-0.9245*\dx},{1.0056*\dy})
	-- ({-0.9120*\dx},{1.0252*\dy})
	-- ({-0.8989*\dx},{1.0448*\dy})
	-- ({-0.8851*\dx},{1.0646*\dy})
	-- ({-0.8705*\dx},{1.0844*\dy})
	-- ({-0.8552*\dx},{1.1043*\dy})
	-- ({-0.8392*\dx},{1.1241*\dy})
	-- ({-0.8225*\dx},{1.1439*\dy})
	-- ({-0.8049*\dx},{1.1637*\dy})
	-- ({-0.7866*\dx},{1.1833*\dy})
	-- ({-0.7675*\dx},{1.2028*\dy})
	-- ({-0.7476*\dx},{1.2221*\dy})
	-- ({-0.7268*\dx},{1.2411*\dy})
	-- ({-0.7053*\dx},{1.2600*\dy})
	-- ({-0.6829*\dx},{1.2785*\dy})
	-- ({-0.6597*\dx},{1.2966*\dy})
	-- ({-0.6357*\dx},{1.3144*\dy})
	-- ({-0.6109*\dx},{1.3317*\dy})
	-- ({-0.5853*\dx},{1.3486*\dy})
	-- ({-0.5589*\dx},{1.3649*\dy})
	-- ({-0.5317*\dx},{1.3806*\dy})
	-- ({-0.5038*\dx},{1.3957*\dy})
	-- ({-0.4751*\dx},{1.4102*\dy})
	-- ({-0.4457*\dx},{1.4239*\dy})
	-- ({-0.4157*\dx},{1.4368*\dy})
	-- ({-0.3850*\dx},{1.4490*\dy})
	-- ({-0.3536*\dx},{1.4603*\dy})
	-- ({-0.3218*\dx},{1.4707*\dy})
	-- ({-0.2894*\dx},{1.4801*\dy})
	-- ({-0.2565*\dx},{1.4886*\dy})
	-- ({-0.2232*\dx},{1.4962*\dy})
	-- ({-0.1895*\dx},{1.5027*\dy})
	-- ({-0.1555*\dx},{1.5081*\dy})
	-- ({-0.1212*\dx},{1.5125*\dy})
	-- ({-0.0867*\dx},{1.5158*\dy})
	-- ({-0.0521*\dx},{1.5180*\dy})
	-- ({-0.0174*\dx},{1.5192*\dy})
	-- ({0.0174*\dx},{1.5192*\dy})
	-- ({0.0521*\dx},{1.5180*\dy})
	-- ({0.0867*\dx},{1.5158*\dy})
	-- ({0.1212*\dx},{1.5125*\dy})
	-- ({0.1555*\dx},{1.5081*\dy})
	-- ({0.1895*\dx},{1.5027*\dy})
	-- ({0.2232*\dx},{1.4962*\dy})
	-- ({0.2565*\dx},{1.4886*\dy})
	-- ({0.2894*\dx},{1.4801*\dy})
	-- ({0.3218*\dx},{1.4707*\dy})
	-- ({0.3536*\dx},{1.4603*\dy})
	-- ({0.3850*\dx},{1.4490*\dy})
	-- ({0.4157*\dx},{1.4368*\dy})
	-- ({0.4457*\dx},{1.4239*\dy})
	-- ({0.4751*\dx},{1.4102*\dy})
	-- ({0.5038*\dx},{1.3957*\dy})
	-- ({0.5317*\dx},{1.3806*\dy})
	-- ({0.5589*\dx},{1.3649*\dy})
	-- ({0.5853*\dx},{1.3486*\dy})
	-- ({0.6109*\dx},{1.3317*\dy})
	-- ({0.6357*\dx},{1.3144*\dy})
	-- ({0.6597*\dx},{1.2966*\dy})
	-- ({0.6829*\dx},{1.2785*\dy})
	-- ({0.7053*\dx},{1.2600*\dy})
	-- ({0.7268*\dx},{1.2411*\dy})
	-- ({0.7476*\dx},{1.2221*\dy})
	-- ({0.7675*\dx},{1.2028*\dy})
	-- ({0.7866*\dx},{1.1833*\dy})
	-- ({0.8049*\dx},{1.1637*\dy})
	-- ({0.8225*\dx},{1.1439*\dy})
	-- ({0.8392*\dx},{1.1241*\dy})
	-- ({0.8552*\dx},{1.1043*\dy})
	-- ({0.8705*\dx},{1.0844*\dy})
	-- ({0.8851*\dx},{1.0646*\dy})
	-- ({0.8989*\dx},{1.0448*\dy})
	-- ({0.9120*\dx},{1.0252*\dy})
	-- ({0.9245*\dx},{1.0056*\dy})
	-- ({0.9363*\dx},{0.9861*\dy})
	-- ({0.9475*\dx},{0.9668*\dy})
	-- ({0.9581*\dx},{0.9476*\dy})
	-- ({0.9681*\dx},{0.9286*\dy})
	-- ({0.9776*\dx},{0.9098*\dy})
	-- ({0.9865*\dx},{0.8912*\dy})
	-- ({0.9948*\dx},{0.8729*\dy})
	-- ({1.0027*\dx},{0.8547*\dy})
	-- ({1.0101*\dx},{0.8369*\dy})
	-- ({1.0170*\dx},{0.8192*\dy})
	-- ({1.0235*\dx},{0.8018*\dy})
	-- ({1.0295*\dx},{0.7847*\dy})
	-- ({1.0352*\dx},{0.7679*\dy})
	-- ({1.0404*\dx},{0.7513*\dy})
	-- ({1.0453*\dx},{0.7350*\dy})
	-- ({1.0498*\dx},{0.7190*\dy})
	-- ({1.0540*\dx},{0.7033*\dy})
	-- ({1.0579*\dx},{0.6879*\dy})
	-- ({1.0615*\dx},{0.6727*\dy})
	-- ({1.0648*\dx},{0.6579*\dy})
	-- ({1.0678*\dx},{0.6433*\dy})
	-- ({1.0705*\dx},{0.6290*\dy})
	-- ({1.0730*\dx},{0.6150*\dy})
	-- ({1.0753*\dx},{0.6013*\dy})
	-- ({1.0773*\dx},{0.5878*\dy})
	-- ({1.0792*\dx},{0.5746*\dy})
	-- ({1.0808*\dx},{0.5618*\dy})
	-- ({1.0823*\dx},{0.5491*\dy})
	-- ({1.0836*\dx},{0.5368*\dy})
	-- ({1.0847*\dx},{0.5247*\dy})
	-- ({1.0856*\dx},{0.5129*\dy})
	-- ({1.0864*\dx},{0.5013*\dy})
	-- ({1.0871*\dx},{0.4900*\dy})
	-- ({1.0877*\dx},{0.4789*\dy})
	-- ({1.0881*\dx},{0.4681*\dy})
	-- ({1.0884*\dx},{0.4575*\dy})
	-- ({1.0886*\dx},{0.4471*\dy})
	-- ({1.0887*\dx},{0.4370*\dy})
	-- ({1.0887*\dx},{0.4271*\dy})
	-- ({1.0887*\dx},{0.4175*\dy})
	-- ({1.0885*\dx},{0.4080*\dy})
	-- ({1.0883*\dx},{0.3988*\dy})
	-- ({1.0880*\dx},{0.3898*\dy})
	-- ({1.0876*\dx},{0.3810*\dy})
	-- ({1.0872*\dx},{0.3724*\dy})
	-- ({1.0867*\dx},{0.3640*\dy})
	-- ({1.0862*\dx},{0.3558*\dy})
	-- ({1.0856*\dx},{0.3477*\dy})
	-- ({1.0850*\dx},{0.3399*\dy})
	-- ({1.0843*\dx},{0.3322*\dy})
	-- ({1.0836*\dx},{0.3247*\dy})
	-- ({1.0829*\dx},{0.3174*\dy})
	-- ({1.0821*\dx},{0.3103*\dy})
	-- ({1.0813*\dx},{0.3033*\dy})
	-- ({1.0805*\dx},{0.2965*\dy})
	-- ({1.0796*\dx},{0.2899*\dy})
	-- ({1.0788*\dx},{0.2834*\dy})
	-- ({1.0779*\dx},{0.2770*\dy})
	-- ({1.0770*\dx},{0.2708*\dy})
	-- ({1.0760*\dx},{0.2648*\dy})
	-- ({1.0751*\dx},{0.2588*\dy})
	-- ({1.0742*\dx},{0.2531*\dy})
	-- ({1.0732*\dx},{0.2474*\dy})
	-- ({1.0723*\dx},{0.2419*\dy})
	-- ({1.0713*\dx},{0.2365*\dy})
	-- ({1.0704*\dx},{0.2312*\dy})
	-- ({1.0694*\dx},{0.2261*\dy})
	-- ({1.0684*\dx},{0.2211*\dy})
	-- ({1.0674*\dx},{0.2162*\dy})
	-- ({1.0665*\dx},{0.2114*\dy})
	-- ({1.0655*\dx},{0.2067*\dy})
	-- ({1.0645*\dx},{0.2021*\dy})
	-- ({1.0635*\dx},{0.1977*\dy})
	-- ({1.0626*\dx},{0.1933*\dy})
	-- ({1.0616*\dx},{0.1890*\dy})
	-- ({1.0607*\dx},{0.1849*\dy})
	-- ({1.0597*\dx},{0.1808*\dy})
	-- ({1.0588*\dx},{0.1768*\dy})
	-- ({1.0578*\dx},{0.1729*\dy})
	-- ({1.0569*\dx},{0.1691*\dy})
	-- ({1.0560*\dx},{0.1654*\dy})
	-- ({1.0551*\dx},{0.1618*\dy})
	-- ({1.0541*\dx},{0.1582*\dy})
	-- ({1.0532*\dx},{0.1548*\dy})
	-- ({1.0523*\dx},{0.1514*\dy})
	-- ({1.0515*\dx},{0.1481*\dy})
	-- ({1.0506*\dx},{0.1448*\dy})
	-- ({1.0497*\dx},{0.1417*\dy})
	-- ({1.0489*\dx},{0.1386*\dy})
	-- ({1.0480*\dx},{0.1356*\dy})
	-- ({1.0472*\dx},{0.1326*\dy})
	-- ({1.0464*\dx},{0.1297*\dy})
	-- ({1.0455*\dx},{0.1269*\dy})
	-- ({1.0447*\dx},{0.1241*\dy})
	-- ({1.0440*\dx},{0.1215*\dy})
	-- ({1.0432*\dx},{0.1188*\dy})
	-- ({1.0424*\dx},{0.1162*\dy})
	-- ({1.0416*\dx},{0.1137*\dy})
	-- ({1.0409*\dx},{0.1113*\dy})
	-- ({1.0401*\dx},{0.1089*\dy})
	-- ({1.0394*\dx},{0.1065*\dy})
	-- ({1.0387*\dx},{0.1042*\dy})
	-- ({1.0380*\dx},{0.1020*\dy})
	-- ({1.0373*\dx},{0.0998*\dy})
	-- ({1.0366*\dx},{0.0976*\dy})
	-- ({1.0359*\dx},{0.0955*\dy})
	-- ({1.0352*\dx},{0.0934*\dy})
	-- ({1.0346*\dx},{0.0914*\dy})
	-- ({1.0339*\dx},{0.0895*\dy})
	-- ({1.0333*\dx},{0.0875*\dy})
	-- ({1.0327*\dx},{0.0857*\dy})
	-- ({1.0321*\dx},{0.0838*\dy})
	-- ({1.0315*\dx},{0.0820*\dy})
}
% v = -0.415799
\def\vpathG{
	({-1.0577*\dx},{0.0677*\dy})
	-- ({-1.0590*\dx},{0.0692*\dy})
	-- ({-1.0602*\dx},{0.0708*\dy})
	-- ({-1.0615*\dx},{0.0724*\dy})
	-- ({-1.0627*\dx},{0.0740*\dy})
	-- ({-1.0641*\dx},{0.0757*\dy})
	-- ({-1.0654*\dx},{0.0774*\dy})
	-- ({-1.0668*\dx},{0.0791*\dy})
	-- ({-1.0682*\dx},{0.0809*\dy})
	-- ({-1.0696*\dx},{0.0828*\dy})
	-- ({-1.0711*\dx},{0.0846*\dy})
	-- ({-1.0725*\dx},{0.0866*\dy})
	-- ({-1.0741*\dx},{0.0886*\dy})
	-- ({-1.0756*\dx},{0.0906*\dy})
	-- ({-1.0772*\dx},{0.0926*\dy})
	-- ({-1.0788*\dx},{0.0948*\dy})
	-- ({-1.0804*\dx},{0.0969*\dy})
	-- ({-1.0821*\dx},{0.0992*\dy})
	-- ({-1.0838*\dx},{0.1015*\dy})
	-- ({-1.0856*\dx},{0.1038*\dy})
	-- ({-1.0874*\dx},{0.1062*\dy})
	-- ({-1.0892*\dx},{0.1086*\dy})
	-- ({-1.0910*\dx},{0.1112*\dy})
	-- ({-1.0929*\dx},{0.1137*\dy})
	-- ({-1.0948*\dx},{0.1164*\dy})
	-- ({-1.0968*\dx},{0.1191*\dy})
	-- ({-1.0988*\dx},{0.1219*\dy})
	-- ({-1.1008*\dx},{0.1247*\dy})
	-- ({-1.1029*\dx},{0.1277*\dy})
	-- ({-1.1050*\dx},{0.1306*\dy})
	-- ({-1.1072*\dx},{0.1337*\dy})
	-- ({-1.1094*\dx},{0.1369*\dy})
	-- ({-1.1117*\dx},{0.1401*\dy})
	-- ({-1.1140*\dx},{0.1434*\dy})
	-- ({-1.1163*\dx},{0.1468*\dy})
	-- ({-1.1187*\dx},{0.1503*\dy})
	-- ({-1.1211*\dx},{0.1539*\dy})
	-- ({-1.1236*\dx},{0.1576*\dy})
	-- ({-1.1261*\dx},{0.1613*\dy})
	-- ({-1.1287*\dx},{0.1652*\dy})
	-- ({-1.1313*\dx},{0.1692*\dy})
	-- ({-1.1339*\dx},{0.1732*\dy})
	-- ({-1.1367*\dx},{0.1774*\dy})
	-- ({-1.1394*\dx},{0.1817*\dy})
	-- ({-1.1422*\dx},{0.1861*\dy})
	-- ({-1.1451*\dx},{0.1907*\dy})
	-- ({-1.1480*\dx},{0.1953*\dy})
	-- ({-1.1510*\dx},{0.2001*\dy})
	-- ({-1.1540*\dx},{0.2050*\dy})
	-- ({-1.1571*\dx},{0.2100*\dy})
	-- ({-1.1602*\dx},{0.2152*\dy})
	-- ({-1.1634*\dx},{0.2205*\dy})
	-- ({-1.1666*\dx},{0.2260*\dy})
	-- ({-1.1699*\dx},{0.2316*\dy})
	-- ({-1.1732*\dx},{0.2374*\dy})
	-- ({-1.1766*\dx},{0.2433*\dy})
	-- ({-1.1800*\dx},{0.2494*\dy})
	-- ({-1.1835*\dx},{0.2557*\dy})
	-- ({-1.1871*\dx},{0.2621*\dy})
	-- ({-1.1907*\dx},{0.2687*\dy})
	-- ({-1.1944*\dx},{0.2756*\dy})
	-- ({-1.1981*\dx},{0.2826*\dy})
	-- ({-1.2018*\dx},{0.2898*\dy})
	-- ({-1.2056*\dx},{0.2972*\dy})
	-- ({-1.2095*\dx},{0.3049*\dy})
	-- ({-1.2134*\dx},{0.3127*\dy})
	-- ({-1.2174*\dx},{0.3208*\dy})
	-- ({-1.2214*\dx},{0.3291*\dy})
	-- ({-1.2254*\dx},{0.3377*\dy})
	-- ({-1.2295*\dx},{0.3465*\dy})
	-- ({-1.2337*\dx},{0.3556*\dy})
	-- ({-1.2378*\dx},{0.3650*\dy})
	-- ({-1.2420*\dx},{0.3746*\dy})
	-- ({-1.2463*\dx},{0.3845*\dy})
	-- ({-1.2506*\dx},{0.3948*\dy})
	-- ({-1.2549*\dx},{0.4053*\dy})
	-- ({-1.2592*\dx},{0.4161*\dy})
	-- ({-1.2635*\dx},{0.4273*\dy})
	-- ({-1.2679*\dx},{0.4388*\dy})
	-- ({-1.2722*\dx},{0.4507*\dy})
	-- ({-1.2766*\dx},{0.4629*\dy})
	-- ({-1.2810*\dx},{0.4755*\dy})
	-- ({-1.2853*\dx},{0.4885*\dy})
	-- ({-1.2897*\dx},{0.5019*\dy})
	-- ({-1.2940*\dx},{0.5157*\dy})
	-- ({-1.2982*\dx},{0.5300*\dy})
	-- ({-1.3025*\dx},{0.5447*\dy})
	-- ({-1.3066*\dx},{0.5598*\dy})
	-- ({-1.3107*\dx},{0.5754*\dy})
	-- ({-1.3148*\dx},{0.5915*\dy})
	-- ({-1.3187*\dx},{0.6081*\dy})
	-- ({-1.3225*\dx},{0.6252*\dy})
	-- ({-1.3262*\dx},{0.6429*\dy})
	-- ({-1.3298*\dx},{0.6611*\dy})
	-- ({-1.3332*\dx},{0.6798*\dy})
	-- ({-1.3364*\dx},{0.6992*\dy})
	-- ({-1.3394*\dx},{0.7191*\dy})
	-- ({-1.3422*\dx},{0.7396*\dy})
	-- ({-1.3447*\dx},{0.7608*\dy})
	-- ({-1.3470*\dx},{0.7826*\dy})
	-- ({-1.3490*\dx},{0.8051*\dy})
	-- ({-1.3506*\dx},{0.8283*\dy})
	-- ({-1.3519*\dx},{0.8521*\dy})
	-- ({-1.3527*\dx},{0.8767*\dy})
	-- ({-1.3531*\dx},{0.9020*\dy})
	-- ({-1.3531*\dx},{0.9280*\dy})
	-- ({-1.3525*\dx},{0.9548*\dy})
	-- ({-1.3513*\dx},{0.9823*\dy})
	-- ({-1.3495*\dx},{1.0106*\dy})
	-- ({-1.3471*\dx},{1.0396*\dy})
	-- ({-1.3439*\dx},{1.0694*\dy})
	-- ({-1.3400*\dx},{1.1000*\dy})
	-- ({-1.3352*\dx},{1.1314*\dy})
	-- ({-1.3295*\dx},{1.1635*\dy})
	-- ({-1.3229*\dx},{1.1964*\dy})
	-- ({-1.3152*\dx},{1.2300*\dy})
	-- ({-1.3064*\dx},{1.2644*\dy})
	-- ({-1.2964*\dx},{1.2994*\dy})
	-- ({-1.2852*\dx},{1.3351*\dy})
	-- ({-1.2726*\dx},{1.3715*\dy})
	-- ({-1.2586*\dx},{1.4085*\dy})
	-- ({-1.2432*\dx},{1.4460*\dy})
	-- ({-1.2261*\dx},{1.4840*\dy})
	-- ({-1.2074*\dx},{1.5225*\dy})
	-- ({-1.1870*\dx},{1.5613*\dy})
	-- ({-1.1648*\dx},{1.6003*\dy})
	-- ({-1.1407*\dx},{1.6396*\dy})
	-- ({-1.1146*\dx},{1.6789*\dy})
	-- ({-1.0865*\dx},{1.7182*\dy})
	-- ({-1.0564*\dx},{1.7573*\dy})
	-- ({-1.0240*\dx},{1.7961*\dy})
	-- ({-0.9896*\dx},{1.8346*\dy})
	-- ({-0.9529*\dx},{1.8724*\dy})
	-- ({-0.9140*\dx},{1.9095*\dy})
	-- ({-0.8728*\dx},{1.9457*\dy})
	-- ({-0.8294*\dx},{1.9808*\dy})
	-- ({-0.7838*\dx},{2.0147*\dy})
	-- ({-0.7360*\dx},{2.0471*\dy})
	-- ({-0.6861*\dx},{2.0779*\dy})
	-- ({-0.6342*\dx},{2.1070*\dy})
	-- ({-0.5803*\dx},{2.1340*\dy})
	-- ({-0.5246*\dx},{2.1590*\dy})
	-- ({-0.4672*\dx},{2.1816*\dy})
	-- ({-0.4082*\dx},{2.2017*\dy})
	-- ({-0.3479*\dx},{2.2193*\dy})
	-- ({-0.2863*\dx},{2.2341*\dy})
	-- ({-0.2237*\dx},{2.2462*\dy})
	-- ({-0.1604*\dx},{2.2552*\dy})
	-- ({-0.0965*\dx},{2.2613*\dy})
	-- ({-0.0322*\dx},{2.2644*\dy})
	-- ({0.0322*\dx},{2.2644*\dy})
	-- ({0.0965*\dx},{2.2613*\dy})
	-- ({0.1604*\dx},{2.2552*\dy})
	-- ({0.2237*\dx},{2.2462*\dy})
	-- ({0.2863*\dx},{2.2341*\dy})
	-- ({0.3479*\dx},{2.2193*\dy})
	-- ({0.4082*\dx},{2.2017*\dy})
	-- ({0.4672*\dx},{2.1816*\dy})
	-- ({0.5246*\dx},{2.1590*\dy})
	-- ({0.5803*\dx},{2.1340*\dy})
	-- ({0.6342*\dx},{2.1070*\dy})
	-- ({0.6861*\dx},{2.0779*\dy})
	-- ({0.7360*\dx},{2.0471*\dy})
	-- ({0.7838*\dx},{2.0147*\dy})
	-- ({0.8294*\dx},{1.9808*\dy})
	-- ({0.8728*\dx},{1.9457*\dy})
	-- ({0.9140*\dx},{1.9095*\dy})
	-- ({0.9529*\dx},{1.8724*\dy})
	-- ({0.9896*\dx},{1.8346*\dy})
	-- ({1.0240*\dx},{1.7961*\dy})
	-- ({1.0564*\dx},{1.7573*\dy})
	-- ({1.0865*\dx},{1.7182*\dy})
	-- ({1.1146*\dx},{1.6789*\dy})
	-- ({1.1407*\dx},{1.6396*\dy})
	-- ({1.1648*\dx},{1.6003*\dy})
	-- ({1.1870*\dx},{1.5613*\dy})
	-- ({1.2074*\dx},{1.5225*\dy})
	-- ({1.2261*\dx},{1.4840*\dy})
	-- ({1.2432*\dx},{1.4460*\dy})
	-- ({1.2586*\dx},{1.4085*\dy})
	-- ({1.2726*\dx},{1.3715*\dy})
	-- ({1.2852*\dx},{1.3351*\dy})
	-- ({1.2964*\dx},{1.2994*\dy})
	-- ({1.3064*\dx},{1.2644*\dy})
	-- ({1.3152*\dx},{1.2300*\dy})
	-- ({1.3229*\dx},{1.1964*\dy})
	-- ({1.3295*\dx},{1.1635*\dy})
	-- ({1.3352*\dx},{1.1314*\dy})
	-- ({1.3400*\dx},{1.1000*\dy})
	-- ({1.3439*\dx},{1.0694*\dy})
	-- ({1.3471*\dx},{1.0396*\dy})
	-- ({1.3495*\dx},{1.0106*\dy})
	-- ({1.3513*\dx},{0.9823*\dy})
	-- ({1.3525*\dx},{0.9548*\dy})
	-- ({1.3531*\dx},{0.9280*\dy})
	-- ({1.3531*\dx},{0.9020*\dy})
	-- ({1.3527*\dx},{0.8767*\dy})
	-- ({1.3519*\dx},{0.8521*\dy})
	-- ({1.3506*\dx},{0.8283*\dy})
	-- ({1.3490*\dx},{0.8051*\dy})
	-- ({1.3470*\dx},{0.7826*\dy})
	-- ({1.3447*\dx},{0.7608*\dy})
	-- ({1.3422*\dx},{0.7396*\dy})
	-- ({1.3394*\dx},{0.7191*\dy})
	-- ({1.3364*\dx},{0.6992*\dy})
	-- ({1.3332*\dx},{0.6798*\dy})
	-- ({1.3298*\dx},{0.6611*\dy})
	-- ({1.3262*\dx},{0.6429*\dy})
	-- ({1.3225*\dx},{0.6252*\dy})
	-- ({1.3187*\dx},{0.6081*\dy})
	-- ({1.3148*\dx},{0.5915*\dy})
	-- ({1.3107*\dx},{0.5754*\dy})
	-- ({1.3066*\dx},{0.5598*\dy})
	-- ({1.3025*\dx},{0.5447*\dy})
	-- ({1.2982*\dx},{0.5300*\dy})
	-- ({1.2940*\dx},{0.5157*\dy})
	-- ({1.2897*\dx},{0.5019*\dy})
	-- ({1.2853*\dx},{0.4885*\dy})
	-- ({1.2810*\dx},{0.4755*\dy})
	-- ({1.2766*\dx},{0.4629*\dy})
	-- ({1.2722*\dx},{0.4507*\dy})
	-- ({1.2679*\dx},{0.4388*\dy})
	-- ({1.2635*\dx},{0.4273*\dy})
	-- ({1.2592*\dx},{0.4161*\dy})
	-- ({1.2549*\dx},{0.4053*\dy})
	-- ({1.2506*\dx},{0.3948*\dy})
	-- ({1.2463*\dx},{0.3845*\dy})
	-- ({1.2420*\dx},{0.3746*\dy})
	-- ({1.2378*\dx},{0.3650*\dy})
	-- ({1.2337*\dx},{0.3556*\dy})
	-- ({1.2295*\dx},{0.3465*\dy})
	-- ({1.2254*\dx},{0.3377*\dy})
	-- ({1.2214*\dx},{0.3291*\dy})
	-- ({1.2174*\dx},{0.3208*\dy})
	-- ({1.2134*\dx},{0.3127*\dy})
	-- ({1.2095*\dx},{0.3049*\dy})
	-- ({1.2056*\dx},{0.2972*\dy})
	-- ({1.2018*\dx},{0.2898*\dy})
	-- ({1.1981*\dx},{0.2826*\dy})
	-- ({1.1944*\dx},{0.2756*\dy})
	-- ({1.1907*\dx},{0.2687*\dy})
	-- ({1.1871*\dx},{0.2621*\dy})
	-- ({1.1835*\dx},{0.2557*\dy})
	-- ({1.1800*\dx},{0.2494*\dy})
	-- ({1.1766*\dx},{0.2433*\dy})
	-- ({1.1732*\dx},{0.2374*\dy})
	-- ({1.1699*\dx},{0.2316*\dy})
	-- ({1.1666*\dx},{0.2260*\dy})
	-- ({1.1634*\dx},{0.2205*\dy})
	-- ({1.1602*\dx},{0.2152*\dy})
	-- ({1.1571*\dx},{0.2100*\dy})
	-- ({1.1540*\dx},{0.2050*\dy})
	-- ({1.1510*\dx},{0.2001*\dy})
	-- ({1.1480*\dx},{0.1953*\dy})
	-- ({1.1451*\dx},{0.1907*\dy})
	-- ({1.1422*\dx},{0.1861*\dy})
	-- ({1.1394*\dx},{0.1817*\dy})
	-- ({1.1367*\dx},{0.1774*\dy})
	-- ({1.1339*\dx},{0.1732*\dy})
	-- ({1.1313*\dx},{0.1692*\dy})
	-- ({1.1287*\dx},{0.1652*\dy})
	-- ({1.1261*\dx},{0.1613*\dy})
	-- ({1.1236*\dx},{0.1576*\dy})
	-- ({1.1211*\dx},{0.1539*\dy})
	-- ({1.1187*\dx},{0.1503*\dy})
	-- ({1.1163*\dx},{0.1468*\dy})
	-- ({1.1140*\dx},{0.1434*\dy})
	-- ({1.1117*\dx},{0.1401*\dy})
	-- ({1.1094*\dx},{0.1369*\dy})
	-- ({1.1072*\dx},{0.1337*\dy})
	-- ({1.1050*\dx},{0.1306*\dy})
	-- ({1.1029*\dx},{0.1277*\dy})
	-- ({1.1008*\dx},{0.1247*\dy})
	-- ({1.0988*\dx},{0.1219*\dy})
	-- ({1.0968*\dx},{0.1191*\dy})
	-- ({1.0948*\dx},{0.1164*\dy})
	-- ({1.0929*\dx},{0.1137*\dy})
	-- ({1.0910*\dx},{0.1112*\dy})
	-- ({1.0892*\dx},{0.1086*\dy})
	-- ({1.0874*\dx},{0.1062*\dy})
	-- ({1.0856*\dx},{0.1038*\dy})
	-- ({1.0838*\dx},{0.1015*\dy})
	-- ({1.0821*\dx},{0.0992*\dy})
	-- ({1.0804*\dx},{0.0969*\dy})
	-- ({1.0788*\dx},{0.0948*\dy})
	-- ({1.0772*\dx},{0.0926*\dy})
	-- ({1.0756*\dx},{0.0906*\dy})
	-- ({1.0741*\dx},{0.0886*\dy})
	-- ({1.0725*\dx},{0.0866*\dy})
	-- ({1.0711*\dx},{0.0846*\dy})
	-- ({1.0696*\dx},{0.0828*\dy})
	-- ({1.0682*\dx},{0.0809*\dy})
	-- ({1.0668*\dx},{0.0791*\dy})
	-- ({1.0654*\dx},{0.0774*\dy})
	-- ({1.0641*\dx},{0.0757*\dy})
	-- ({1.0627*\dx},{0.0740*\dy})
	-- ({1.0615*\dx},{0.0724*\dy})
	-- ({1.0602*\dx},{0.0708*\dy})
	-- ({1.0590*\dx},{0.0692*\dy})
	-- ({1.0577*\dx},{0.0677*\dy})
}
% v = -0.249479
\def\vpathH{
	({-1.0779*\dx},{0.0447*\dy})
	-- ({-1.0796*\dx},{0.0457*\dy})
	-- ({-1.0814*\dx},{0.0467*\dy})
	-- ({-1.0831*\dx},{0.0478*\dy})
	-- ({-1.0850*\dx},{0.0489*\dy})
	-- ({-1.0868*\dx},{0.0500*\dy})
	-- ({-1.0887*\dx},{0.0512*\dy})
	-- ({-1.0906*\dx},{0.0524*\dy})
	-- ({-1.0926*\dx},{0.0536*\dy})
	-- ({-1.0947*\dx},{0.0548*\dy})
	-- ({-1.0967*\dx},{0.0561*\dy})
	-- ({-1.0989*\dx},{0.0574*\dy})
	-- ({-1.1010*\dx},{0.0588*\dy})
	-- ({-1.1032*\dx},{0.0602*\dy})
	-- ({-1.1055*\dx},{0.0616*\dy})
	-- ({-1.1078*\dx},{0.0630*\dy})
	-- ({-1.1102*\dx},{0.0645*\dy})
	-- ({-1.1126*\dx},{0.0660*\dy})
	-- ({-1.1151*\dx},{0.0676*\dy})
	-- ({-1.1177*\dx},{0.0692*\dy})
	-- ({-1.1203*\dx},{0.0708*\dy})
	-- ({-1.1229*\dx},{0.0725*\dy})
	-- ({-1.1256*\dx},{0.0743*\dy})
	-- ({-1.1284*\dx},{0.0760*\dy})
	-- ({-1.1312*\dx},{0.0779*\dy})
	-- ({-1.1342*\dx},{0.0797*\dy})
	-- ({-1.1371*\dx},{0.0817*\dy})
	-- ({-1.1402*\dx},{0.0836*\dy})
	-- ({-1.1433*\dx},{0.0857*\dy})
	-- ({-1.1465*\dx},{0.0878*\dy})
	-- ({-1.1497*\dx},{0.0899*\dy})
	-- ({-1.1531*\dx},{0.0921*\dy})
	-- ({-1.1565*\dx},{0.0944*\dy})
	-- ({-1.1600*\dx},{0.0967*\dy})
	-- ({-1.1635*\dx},{0.0991*\dy})
	-- ({-1.1672*\dx},{0.1015*\dy})
	-- ({-1.1709*\dx},{0.1041*\dy})
	-- ({-1.1748*\dx},{0.1067*\dy})
	-- ({-1.1787*\dx},{0.1093*\dy})
	-- ({-1.1827*\dx},{0.1121*\dy})
	-- ({-1.1868*\dx},{0.1149*\dy})
	-- ({-1.1910*\dx},{0.1178*\dy})
	-- ({-1.1953*\dx},{0.1208*\dy})
	-- ({-1.1997*\dx},{0.1239*\dy})
	-- ({-1.2042*\dx},{0.1271*\dy})
	-- ({-1.2088*\dx},{0.1303*\dy})
	-- ({-1.2136*\dx},{0.1337*\dy})
	-- ({-1.2184*\dx},{0.1371*\dy})
	-- ({-1.2233*\dx},{0.1407*\dy})
	-- ({-1.2284*\dx},{0.1444*\dy})
	-- ({-1.2336*\dx},{0.1482*\dy})
	-- ({-1.2389*\dx},{0.1521*\dy})
	-- ({-1.2444*\dx},{0.1561*\dy})
	-- ({-1.2499*\dx},{0.1602*\dy})
	-- ({-1.2556*\dx},{0.1645*\dy})
	-- ({-1.2615*\dx},{0.1689*\dy})
	-- ({-1.2675*\dx},{0.1735*\dy})
	-- ({-1.2736*\dx},{0.1781*\dy})
	-- ({-1.2799*\dx},{0.1830*\dy})
	-- ({-1.2863*\dx},{0.1880*\dy})
	-- ({-1.2929*\dx},{0.1932*\dy})
	-- ({-1.2996*\dx},{0.1985*\dy})
	-- ({-1.3065*\dx},{0.2040*\dy})
	-- ({-1.3136*\dx},{0.2097*\dy})
	-- ({-1.3209*\dx},{0.2156*\dy})
	-- ({-1.3283*\dx},{0.2217*\dy})
	-- ({-1.3359*\dx},{0.2280*\dy})
	-- ({-1.3437*\dx},{0.2345*\dy})
	-- ({-1.3517*\dx},{0.2412*\dy})
	-- ({-1.3599*\dx},{0.2482*\dy})
	-- ({-1.3682*\dx},{0.2554*\dy})
	-- ({-1.3768*\dx},{0.2629*\dy})
	-- ({-1.3856*\dx},{0.2706*\dy})
	-- ({-1.3946*\dx},{0.2786*\dy})
	-- ({-1.4038*\dx},{0.2869*\dy})
	-- ({-1.4133*\dx},{0.2956*\dy})
	-- ({-1.4230*\dx},{0.3045*\dy})
	-- ({-1.4329*\dx},{0.3138*\dy})
	-- ({-1.4430*\dx},{0.3234*\dy})
	-- ({-1.4534*\dx},{0.3334*\dy})
	-- ({-1.4641*\dx},{0.3438*\dy})
	-- ({-1.4750*\dx},{0.3545*\dy})
	-- ({-1.4861*\dx},{0.3657*\dy})
	-- ({-1.4975*\dx},{0.3774*\dy})
	-- ({-1.5092*\dx},{0.3895*\dy})
	-- ({-1.5212*\dx},{0.4021*\dy})
	-- ({-1.5335*\dx},{0.4152*\dy})
	-- ({-1.5460*\dx},{0.4289*\dy})
	-- ({-1.5588*\dx},{0.4431*\dy})
	-- ({-1.5719*\dx},{0.4579*\dy})
	-- ({-1.5853*\dx},{0.4734*\dy})
	-- ({-1.5990*\dx},{0.4895*\dy})
	-- ({-1.6130*\dx},{0.5063*\dy})
	-- ({-1.6273*\dx},{0.5238*\dy})
	-- ({-1.6419*\dx},{0.5421*\dy})
	-- ({-1.6567*\dx},{0.5612*\dy})
	-- ({-1.6719*\dx},{0.5812*\dy})
	-- ({-1.6874*\dx},{0.6021*\dy})
	-- ({-1.7031*\dx},{0.6239*\dy})
	-- ({-1.7191*\dx},{0.6468*\dy})
	-- ({-1.7354*\dx},{0.6707*\dy})
	-- ({-1.7519*\dx},{0.6957*\dy})
	-- ({-1.7687*\dx},{0.7219*\dy})
	-- ({-1.7857*\dx},{0.7494*\dy})
	-- ({-1.8029*\dx},{0.7782*\dy})
	-- ({-1.8202*\dx},{0.8084*\dy})
	-- ({-1.8377*\dx},{0.8400*\dy})
	-- ({-1.8554*\dx},{0.8733*\dy})
	-- ({-1.8730*\dx},{0.9082*\dy})
	-- ({-1.8907*\dx},{0.9448*\dy})
	-- ({-1.9084*\dx},{0.9833*\dy})
	-- ({-1.9259*\dx},{1.0237*\dy})
	-- ({-1.9432*\dx},{1.0662*\dy})
	-- ({-1.9603*\dx},{1.1108*\dy})
	-- ({-1.9770*\dx},{1.1577*\dy})
	-- ({-1.9932*\dx},{1.2070*\dy})
	-- ({-2.0088*\dx},{1.2589*\dy})
	-- ({-2.0237*\dx},{1.3134*\dy})
	-- ({-2.0376*\dx},{1.3706*\dy})
	-- ({-2.0503*\dx},{1.4308*\dy})
	-- ({-2.0618*\dx},{1.4940*\dy})
	-- ({-2.0717*\dx},{1.5603*\dy})
	-- ({-2.0797*\dx},{1.6298*\dy})
	-- ({-2.0856*\dx},{1.7028*\dy})
	-- ({-2.0891*\dx},{1.7791*\dy})
	-- ({-2.0897*\dx},{1.8590*\dy})
	-- ({-2.0871*\dx},{1.9424*\dy})
	-- ({-2.0808*\dx},{2.0294*\dy})
	-- ({-2.0703*\dx},{2.1199*\dy})
	-- ({-2.0552*\dx},{2.2138*\dy})
	-- ({-2.0349*\dx},{2.3110*\dy})
	-- ({-2.0088*\dx},{2.4113*\dy})
	-- ({-1.9763*\dx},{2.5145*\dy})
	-- ({-1.9368*\dx},{2.6200*\dy})
	-- ({-1.8896*\dx},{2.7276*\dy})
	-- ({-1.8343*\dx},{2.8364*\dy})
	-- ({-1.7701*\dx},{2.9460*\dy})
	-- ({-1.6965*\dx},{3.0553*\dy})
	-- ({-1.6132*\dx},{3.1635*\dy})
	-- ({-1.5198*\dx},{3.2694*\dy})
	-- ({-1.4161*\dx},{3.3719*\dy})
	-- ({-1.3020*\dx},{3.4696*\dy})
	-- ({-1.1778*\dx},{3.5613*\dy})
	-- ({-1.0438*\dx},{3.6455*\dy})
	-- ({-0.9007*\dx},{3.7208*\dy})
	-- ({-0.7493*\dx},{3.7859*\dy})
	-- ({-0.5907*\dx},{3.8396*\dy})
	-- ({-0.4263*\dx},{3.8809*\dy})
	-- ({-0.2575*\dx},{3.9089*\dy})
	-- ({-0.0861*\dx},{3.9231*\dy})
	-- ({0.0861*\dx},{3.9231*\dy})
	-- ({0.2575*\dx},{3.9089*\dy})
	-- ({0.4263*\dx},{3.8809*\dy})
	-- ({0.5907*\dx},{3.8396*\dy})
	-- ({0.7493*\dx},{3.7859*\dy})
	-- ({0.9007*\dx},{3.7208*\dy})
	-- ({1.0438*\dx},{3.6455*\dy})
	-- ({1.1778*\dx},{3.5613*\dy})
	-- ({1.3020*\dx},{3.4696*\dy})
	-- ({1.4161*\dx},{3.3719*\dy})
	-- ({1.5198*\dx},{3.2694*\dy})
	-- ({1.6132*\dx},{3.1635*\dy})
	-- ({1.6965*\dx},{3.0553*\dy})
	-- ({1.7701*\dx},{2.9460*\dy})
	-- ({1.8343*\dx},{2.8364*\dy})
	-- ({1.8896*\dx},{2.7276*\dy})
	-- ({1.9368*\dx},{2.6200*\dy})
	-- ({1.9763*\dx},{2.5145*\dy})
	-- ({2.0088*\dx},{2.4113*\dy})
	-- ({2.0349*\dx},{2.3110*\dy})
	-- ({2.0552*\dx},{2.2138*\dy})
	-- ({2.0703*\dx},{2.1199*\dy})
	-- ({2.0808*\dx},{2.0294*\dy})
	-- ({2.0871*\dx},{1.9424*\dy})
	-- ({2.0897*\dx},{1.8590*\dy})
	-- ({2.0891*\dx},{1.7791*\dy})
	-- ({2.0856*\dx},{1.7028*\dy})
	-- ({2.0797*\dx},{1.6298*\dy})
	-- ({2.0717*\dx},{1.5603*\dy})
	-- ({2.0618*\dx},{1.4940*\dy})
	-- ({2.0503*\dx},{1.4308*\dy})
	-- ({2.0376*\dx},{1.3706*\dy})
	-- ({2.0237*\dx},{1.3134*\dy})
	-- ({2.0088*\dx},{1.2589*\dy})
	-- ({1.9932*\dx},{1.2070*\dy})
	-- ({1.9770*\dx},{1.1577*\dy})
	-- ({1.9603*\dx},{1.1108*\dy})
	-- ({1.9432*\dx},{1.0662*\dy})
	-- ({1.9259*\dx},{1.0237*\dy})
	-- ({1.9084*\dx},{0.9833*\dy})
	-- ({1.8907*\dx},{0.9448*\dy})
	-- ({1.8730*\dx},{0.9082*\dy})
	-- ({1.8554*\dx},{0.8733*\dy})
	-- ({1.8377*\dx},{0.8400*\dy})
	-- ({1.8202*\dx},{0.8084*\dy})
	-- ({1.8029*\dx},{0.7782*\dy})
	-- ({1.7857*\dx},{0.7494*\dy})
	-- ({1.7687*\dx},{0.7219*\dy})
	-- ({1.7519*\dx},{0.6957*\dy})
	-- ({1.7354*\dx},{0.6707*\dy})
	-- ({1.7191*\dx},{0.6468*\dy})
	-- ({1.7031*\dx},{0.6239*\dy})
	-- ({1.6874*\dx},{0.6021*\dy})
	-- ({1.6719*\dx},{0.5812*\dy})
	-- ({1.6567*\dx},{0.5612*\dy})
	-- ({1.6419*\dx},{0.5421*\dy})
	-- ({1.6273*\dx},{0.5238*\dy})
	-- ({1.6130*\dx},{0.5063*\dy})
	-- ({1.5990*\dx},{0.4895*\dy})
	-- ({1.5853*\dx},{0.4734*\dy})
	-- ({1.5719*\dx},{0.4579*\dy})
	-- ({1.5588*\dx},{0.4431*\dy})
	-- ({1.5460*\dx},{0.4289*\dy})
	-- ({1.5335*\dx},{0.4152*\dy})
	-- ({1.5212*\dx},{0.4021*\dy})
	-- ({1.5092*\dx},{0.3895*\dy})
	-- ({1.4975*\dx},{0.3774*\dy})
	-- ({1.4861*\dx},{0.3657*\dy})
	-- ({1.4750*\dx},{0.3545*\dy})
	-- ({1.4641*\dx},{0.3438*\dy})
	-- ({1.4534*\dx},{0.3334*\dy})
	-- ({1.4430*\dx},{0.3234*\dy})
	-- ({1.4329*\dx},{0.3138*\dy})
	-- ({1.4230*\dx},{0.3045*\dy})
	-- ({1.4133*\dx},{0.2956*\dy})
	-- ({1.4038*\dx},{0.2869*\dy})
	-- ({1.3946*\dx},{0.2786*\dy})
	-- ({1.3856*\dx},{0.2706*\dy})
	-- ({1.3768*\dx},{0.2629*\dy})
	-- ({1.3682*\dx},{0.2554*\dy})
	-- ({1.3599*\dx},{0.2482*\dy})
	-- ({1.3517*\dx},{0.2412*\dy})
	-- ({1.3437*\dx},{0.2345*\dy})
	-- ({1.3359*\dx},{0.2280*\dy})
	-- ({1.3283*\dx},{0.2217*\dy})
	-- ({1.3209*\dx},{0.2156*\dy})
	-- ({1.3136*\dx},{0.2097*\dy})
	-- ({1.3065*\dx},{0.2040*\dy})
	-- ({1.2996*\dx},{0.1985*\dy})
	-- ({1.2929*\dx},{0.1932*\dy})
	-- ({1.2863*\dx},{0.1880*\dy})
	-- ({1.2799*\dx},{0.1830*\dy})
	-- ({1.2736*\dx},{0.1781*\dy})
	-- ({1.2675*\dx},{0.1735*\dy})
	-- ({1.2615*\dx},{0.1689*\dy})
	-- ({1.2556*\dx},{0.1645*\dy})
	-- ({1.2499*\dx},{0.1602*\dy})
	-- ({1.2444*\dx},{0.1561*\dy})
	-- ({1.2389*\dx},{0.1521*\dy})
	-- ({1.2336*\dx},{0.1482*\dy})
	-- ({1.2284*\dx},{0.1444*\dy})
	-- ({1.2233*\dx},{0.1407*\dy})
	-- ({1.2184*\dx},{0.1371*\dy})
	-- ({1.2136*\dx},{0.1337*\dy})
	-- ({1.2088*\dx},{0.1303*\dy})
	-- ({1.2042*\dx},{0.1271*\dy})
	-- ({1.1997*\dx},{0.1239*\dy})
	-- ({1.1953*\dx},{0.1208*\dy})
	-- ({1.1910*\dx},{0.1178*\dy})
	-- ({1.1868*\dx},{0.1149*\dy})
	-- ({1.1827*\dx},{0.1121*\dy})
	-- ({1.1787*\dx},{0.1093*\dy})
	-- ({1.1748*\dx},{0.1067*\dy})
	-- ({1.1709*\dx},{0.1041*\dy})
	-- ({1.1672*\dx},{0.1015*\dy})
	-- ({1.1635*\dx},{0.0991*\dy})
	-- ({1.1600*\dx},{0.0967*\dy})
	-- ({1.1565*\dx},{0.0944*\dy})
	-- ({1.1531*\dx},{0.0921*\dy})
	-- ({1.1497*\dx},{0.0899*\dy})
	-- ({1.1465*\dx},{0.0878*\dy})
	-- ({1.1433*\dx},{0.0857*\dy})
	-- ({1.1402*\dx},{0.0836*\dy})
	-- ({1.1371*\dx},{0.0817*\dy})
	-- ({1.1342*\dx},{0.0797*\dy})
	-- ({1.1312*\dx},{0.0779*\dy})
	-- ({1.1284*\dx},{0.0760*\dy})
	-- ({1.1256*\dx},{0.0743*\dy})
	-- ({1.1229*\dx},{0.0725*\dy})
	-- ({1.1203*\dx},{0.0708*\dy})
	-- ({1.1177*\dx},{0.0692*\dy})
	-- ({1.1151*\dx},{0.0676*\dy})
	-- ({1.1126*\dx},{0.0660*\dy})
	-- ({1.1102*\dx},{0.0645*\dy})
	-- ({1.1078*\dx},{0.0630*\dy})
	-- ({1.1055*\dx},{0.0616*\dy})
	-- ({1.1032*\dx},{0.0602*\dy})
	-- ({1.1010*\dx},{0.0588*\dy})
	-- ({1.0989*\dx},{0.0574*\dy})
	-- ({1.0967*\dx},{0.0561*\dy})
	-- ({1.0947*\dx},{0.0548*\dy})
	-- ({1.0926*\dx},{0.0536*\dy})
	-- ({1.0906*\dx},{0.0524*\dy})
	-- ({1.0887*\dx},{0.0512*\dy})
	-- ({1.0868*\dx},{0.0500*\dy})
	-- ({1.0850*\dx},{0.0489*\dy})
	-- ({1.0831*\dx},{0.0478*\dy})
	-- ({1.0814*\dx},{0.0467*\dy})
	-- ({1.0796*\dx},{0.0457*\dy})
	-- ({1.0779*\dx},{0.0447*\dy})
}
% v = -0.083160
\def\vpathI{
	({-1.0889*\dx},{0.0156*\dy})
	-- ({-1.0909*\dx},{0.0160*\dy})
	-- ({-1.0929*\dx},{0.0163*\dy})
	-- ({-1.0950*\dx},{0.0167*\dy})
	-- ({-1.0971*\dx},{0.0171*\dy})
	-- ({-1.0992*\dx},{0.0175*\dy})
	-- ({-1.1014*\dx},{0.0179*\dy})
	-- ({-1.1037*\dx},{0.0183*\dy})
	-- ({-1.1060*\dx},{0.0188*\dy})
	-- ({-1.1084*\dx},{0.0192*\dy})
	-- ({-1.1108*\dx},{0.0197*\dy})
	-- ({-1.1133*\dx},{0.0201*\dy})
	-- ({-1.1158*\dx},{0.0206*\dy})
	-- ({-1.1184*\dx},{0.0211*\dy})
	-- ({-1.1211*\dx},{0.0216*\dy})
	-- ({-1.1238*\dx},{0.0221*\dy})
	-- ({-1.1266*\dx},{0.0226*\dy})
	-- ({-1.1295*\dx},{0.0232*\dy})
	-- ({-1.1324*\dx},{0.0237*\dy})
	-- ({-1.1354*\dx},{0.0243*\dy})
	-- ({-1.1385*\dx},{0.0249*\dy})
	-- ({-1.1416*\dx},{0.0255*\dy})
	-- ({-1.1448*\dx},{0.0261*\dy})
	-- ({-1.1481*\dx},{0.0268*\dy})
	-- ({-1.1515*\dx},{0.0274*\dy})
	-- ({-1.1550*\dx},{0.0281*\dy})
	-- ({-1.1585*\dx},{0.0288*\dy})
	-- ({-1.1621*\dx},{0.0295*\dy})
	-- ({-1.1659*\dx},{0.0302*\dy})
	-- ({-1.1697*\dx},{0.0310*\dy})
	-- ({-1.1736*\dx},{0.0318*\dy})
	-- ({-1.1776*\dx},{0.0325*\dy})
	-- ({-1.1817*\dx},{0.0334*\dy})
	-- ({-1.1859*\dx},{0.0342*\dy})
	-- ({-1.1902*\dx},{0.0351*\dy})
	-- ({-1.1946*\dx},{0.0360*\dy})
	-- ({-1.1991*\dx},{0.0369*\dy})
	-- ({-1.2038*\dx},{0.0378*\dy})
	-- ({-1.2086*\dx},{0.0388*\dy})
	-- ({-1.2134*\dx},{0.0398*\dy})
	-- ({-1.2184*\dx},{0.0408*\dy})
	-- ({-1.2236*\dx},{0.0419*\dy})
	-- ({-1.2289*\dx},{0.0430*\dy})
	-- ({-1.2343*\dx},{0.0441*\dy})
	-- ({-1.2398*\dx},{0.0453*\dy})
	-- ({-1.2455*\dx},{0.0465*\dy})
	-- ({-1.2514*\dx},{0.0477*\dy})
	-- ({-1.2574*\dx},{0.0490*\dy})
	-- ({-1.2635*\dx},{0.0503*\dy})
	-- ({-1.2698*\dx},{0.0516*\dy})
	-- ({-1.2763*\dx},{0.0530*\dy})
	-- ({-1.2830*\dx},{0.0545*\dy})
	-- ({-1.2899*\dx},{0.0560*\dy})
	-- ({-1.2969*\dx},{0.0575*\dy})
	-- ({-1.3041*\dx},{0.0591*\dy})
	-- ({-1.3115*\dx},{0.0608*\dy})
	-- ({-1.3192*\dx},{0.0625*\dy})
	-- ({-1.3270*\dx},{0.0642*\dy})
	-- ({-1.3351*\dx},{0.0660*\dy})
	-- ({-1.3434*\dx},{0.0679*\dy})
	-- ({-1.3519*\dx},{0.0699*\dy})
	-- ({-1.3607*\dx},{0.0719*\dy})
	-- ({-1.3697*\dx},{0.0740*\dy})
	-- ({-1.3790*\dx},{0.0762*\dy})
	-- ({-1.3885*\dx},{0.0784*\dy})
	-- ({-1.3983*\dx},{0.0807*\dy})
	-- ({-1.4084*\dx},{0.0831*\dy})
	-- ({-1.4189*\dx},{0.0857*\dy})
	-- ({-1.4296*\dx},{0.0883*\dy})
	-- ({-1.4406*\dx},{0.0910*\dy})
	-- ({-1.4520*\dx},{0.0938*\dy})
	-- ({-1.4637*\dx},{0.0967*\dy})
	-- ({-1.4758*\dx},{0.0997*\dy})
	-- ({-1.4883*\dx},{0.1029*\dy})
	-- ({-1.5011*\dx},{0.1062*\dy})
	-- ({-1.5144*\dx},{0.1096*\dy})
	-- ({-1.5281*\dx},{0.1131*\dy})
	-- ({-1.5422*\dx},{0.1168*\dy})
	-- ({-1.5568*\dx},{0.1207*\dy})
	-- ({-1.5718*\dx},{0.1247*\dy})
	-- ({-1.5873*\dx},{0.1290*\dy})
	-- ({-1.6034*\dx},{0.1333*\dy})
	-- ({-1.6200*\dx},{0.1379*\dy})
	-- ({-1.6371*\dx},{0.1427*\dy})
	-- ({-1.6549*\dx},{0.1478*\dy})
	-- ({-1.6732*\dx},{0.1530*\dy})
	-- ({-1.6922*\dx},{0.1585*\dy})
	-- ({-1.7119*\dx},{0.1643*\dy})
	-- ({-1.7322*\dx},{0.1704*\dy})
	-- ({-1.7533*\dx},{0.1767*\dy})
	-- ({-1.7752*\dx},{0.1834*\dy})
	-- ({-1.7978*\dx},{0.1904*\dy})
	-- ({-1.8213*\dx},{0.1978*\dy})
	-- ({-1.8457*\dx},{0.2056*\dy})
	-- ({-1.8710*\dx},{0.2137*\dy})
	-- ({-1.8973*\dx},{0.2224*\dy})
	-- ({-1.9246*\dx},{0.2315*\dy})
	-- ({-1.9530*\dx},{0.2411*\dy})
	-- ({-1.9826*\dx},{0.2513*\dy})
	-- ({-2.0133*\dx},{0.2621*\dy})
	-- ({-2.0453*\dx},{0.2735*\dy})
	-- ({-2.0787*\dx},{0.2856*\dy})
	-- ({-2.1134*\dx},{0.2984*\dy})
	-- ({-2.1497*\dx},{0.3121*\dy})
	-- ({-2.1875*\dx},{0.3267*\dy})
	-- ({-2.2270*\dx},{0.3422*\dy})
	-- ({-2.2683*\dx},{0.3587*\dy})
	-- ({-2.3114*\dx},{0.3764*\dy})
	-- ({-2.3566*\dx},{0.3953*\dy})
	-- ({-2.4039*\dx},{0.4156*\dy})
	-- ({-2.4534*\dx},{0.4374*\dy})
	-- ({-2.5054*\dx},{0.4607*\dy})
	-- ({-2.5599*\dx},{0.4859*\dy})
	-- ({-2.6172*\dx},{0.5131*\dy})
	-- ({-2.6774*\dx},{0.5424*\dy})
	-- ({-2.7407*\dx},{0.5742*\dy})
	-- ({-2.8074*\dx},{0.6087*\dy})
	-- ({-2.8776*\dx},{0.6461*\dy})
	-- ({-2.9517*\dx},{0.6869*\dy})
	-- ({-3.0298*\dx},{0.7315*\dy})
	-- ({-3.1124*\dx},{0.7803*\dy})
	-- ({-3.1997*\dx},{0.8338*\dy})
	-- ({-3.2921*\dx},{0.8926*\dy})
	-- ({-3.3898*\dx},{0.9575*\dy})
	-- ({-3.4934*\dx},{1.0293*\dy})
	-- ({-3.6031*\dx},{1.1090*\dy})
	-- ({-3.7195*\dx},{1.1977*\dy})
	-- ({-3.8428*\dx},{1.2967*\dy})
	-- ({-3.9736*\dx},{1.4077*\dy})
	-- ({-4.1121*\dx},{1.5325*\dy})
	-- ({-4.2586*\dx},{1.6733*\dy})
	-- ({-4.4134*\dx},{1.8329*\dy})
	-- ({-4.5763*\dx},{2.0145*\dy})
	-- ({-4.7470*\dx},{2.2218*\dy})
	-- ({-4.9247*\dx},{2.4594*\dy})
	-- ({-5.1078*\dx},{2.7327*\dy})
	-- ({-5.2939*\dx},{3.0483*\dy})
	-- ({-5.4788*\dx},{3.4137*\dy})
	-- ({-5.6563*\dx},{3.8374*\dy})
	-- ({-5.8169*\dx},{4.3293*\dy})
	-- ({-5.9470*\dx},{4.8992*\dy})
	-- ({-6.0270*\dx},{5.5566*\dy})
	-- ({-6.0301*\dx},{6.3081*\dy})
	-- ({-5.9206*\dx},{7.1537*\dy})
	-- ({-5.6544*\dx},{8.0814*\dy})
	-- ({-5.1825*\dx},{9.0597*\dy})
	-- ({-4.4601*\dx},{10.0304*\dy})
	-- ({-3.4625*\dx},{10.9064*\dy})
	-- ({-2.2052*\dx},{11.5804*\dy})
	-- ({-0.7584*\dx},{11.9495*\dy})
	-- ({0.7584*\dx},{11.9495*\dy})
	-- ({2.2052*\dx},{11.5804*\dy})
	-- ({3.4625*\dx},{10.9064*\dy})
	-- ({4.4601*\dx},{10.0304*\dy})
	-- ({5.1825*\dx},{9.0597*\dy})
	-- ({5.6544*\dx},{8.0814*\dy})
	-- ({5.9206*\dx},{7.1537*\dy})
	-- ({6.0301*\dx},{6.3081*\dy})
	-- ({6.0270*\dx},{5.5566*\dy})
	-- ({5.9470*\dx},{4.8992*\dy})
	-- ({5.8169*\dx},{4.3293*\dy})
	-- ({5.6563*\dx},{3.8374*\dy})
	-- ({5.4788*\dx},{3.4137*\dy})
	-- ({5.2939*\dx},{3.0483*\dy})
	-- ({5.1078*\dx},{2.7327*\dy})
	-- ({4.9247*\dx},{2.4594*\dy})
	-- ({4.7470*\dx},{2.2218*\dy})
	-- ({4.5763*\dx},{2.0145*\dy})
	-- ({4.4134*\dx},{1.8329*\dy})
	-- ({4.2586*\dx},{1.6733*\dy})
	-- ({4.1121*\dx},{1.5325*\dy})
	-- ({3.9736*\dx},{1.4077*\dy})
	-- ({3.8428*\dx},{1.2967*\dy})
	-- ({3.7195*\dx},{1.1977*\dy})
	-- ({3.6031*\dx},{1.1090*\dy})
	-- ({3.4934*\dx},{1.0293*\dy})
	-- ({3.3898*\dx},{0.9575*\dy})
	-- ({3.2921*\dx},{0.8926*\dy})
	-- ({3.1997*\dx},{0.8338*\dy})
	-- ({3.1124*\dx},{0.7803*\dy})
	-- ({3.0298*\dx},{0.7315*\dy})
	-- ({2.9517*\dx},{0.6869*\dy})
	-- ({2.8776*\dx},{0.6461*\dy})
	-- ({2.8074*\dx},{0.6087*\dy})
	-- ({2.7407*\dx},{0.5742*\dy})
	-- ({2.6774*\dx},{0.5424*\dy})
	-- ({2.6172*\dx},{0.5131*\dy})
	-- ({2.5599*\dx},{0.4859*\dy})
	-- ({2.5054*\dx},{0.4607*\dy})
	-- ({2.4534*\dx},{0.4374*\dy})
	-- ({2.4039*\dx},{0.4156*\dy})
	-- ({2.3566*\dx},{0.3953*\dy})
	-- ({2.3114*\dx},{0.3764*\dy})
	-- ({2.2683*\dx},{0.3587*\dy})
	-- ({2.2270*\dx},{0.3422*\dy})
	-- ({2.1875*\dx},{0.3267*\dy})
	-- ({2.1497*\dx},{0.3121*\dy})
	-- ({2.1134*\dx},{0.2984*\dy})
	-- ({2.0787*\dx},{0.2856*\dy})
	-- ({2.0453*\dx},{0.2735*\dy})
	-- ({2.0133*\dx},{0.2621*\dy})
	-- ({1.9826*\dx},{0.2513*\dy})
	-- ({1.9530*\dx},{0.2411*\dy})
	-- ({1.9246*\dx},{0.2315*\dy})
	-- ({1.8973*\dx},{0.2224*\dy})
	-- ({1.8710*\dx},{0.2137*\dy})
	-- ({1.8457*\dx},{0.2056*\dy})
	-- ({1.8213*\dx},{0.1978*\dy})
	-- ({1.7978*\dx},{0.1904*\dy})
	-- ({1.7752*\dx},{0.1834*\dy})
	-- ({1.7533*\dx},{0.1767*\dy})
	-- ({1.7322*\dx},{0.1704*\dy})
	-- ({1.7119*\dx},{0.1643*\dy})
	-- ({1.6922*\dx},{0.1585*\dy})
	-- ({1.6732*\dx},{0.1530*\dy})
	-- ({1.6549*\dx},{0.1478*\dy})
	-- ({1.6371*\dx},{0.1427*\dy})
	-- ({1.6200*\dx},{0.1379*\dy})
	-- ({1.6034*\dx},{0.1333*\dy})
	-- ({1.5873*\dx},{0.1290*\dy})
	-- ({1.5718*\dx},{0.1247*\dy})
	-- ({1.5568*\dx},{0.1207*\dy})
	-- ({1.5422*\dx},{0.1168*\dy})
	-- ({1.5281*\dx},{0.1131*\dy})
	-- ({1.5144*\dx},{0.1096*\dy})
	-- ({1.5011*\dx},{0.1062*\dy})
	-- ({1.4883*\dx},{0.1029*\dy})
	-- ({1.4758*\dx},{0.0997*\dy})
	-- ({1.4637*\dx},{0.0967*\dy})
	-- ({1.4520*\dx},{0.0938*\dy})
	-- ({1.4406*\dx},{0.0910*\dy})
	-- ({1.4296*\dx},{0.0883*\dy})
	-- ({1.4189*\dx},{0.0857*\dy})
	-- ({1.4084*\dx},{0.0831*\dy})
	-- ({1.3983*\dx},{0.0807*\dy})
	-- ({1.3885*\dx},{0.0784*\dy})
	-- ({1.3790*\dx},{0.0762*\dy})
	-- ({1.3697*\dx},{0.0740*\dy})
	-- ({1.3607*\dx},{0.0719*\dy})
	-- ({1.3519*\dx},{0.0699*\dy})
	-- ({1.3434*\dx},{0.0679*\dy})
	-- ({1.3351*\dx},{0.0660*\dy})
	-- ({1.3270*\dx},{0.0642*\dy})
	-- ({1.3192*\dx},{0.0625*\dy})
	-- ({1.3115*\dx},{0.0608*\dy})
	-- ({1.3041*\dx},{0.0591*\dy})
	-- ({1.2969*\dx},{0.0575*\dy})
	-- ({1.2899*\dx},{0.0560*\dy})
	-- ({1.2830*\dx},{0.0545*\dy})
	-- ({1.2763*\dx},{0.0530*\dy})
	-- ({1.2698*\dx},{0.0516*\dy})
	-- ({1.2635*\dx},{0.0503*\dy})
	-- ({1.2574*\dx},{0.0490*\dy})
	-- ({1.2514*\dx},{0.0477*\dy})
	-- ({1.2455*\dx},{0.0465*\dy})
	-- ({1.2398*\dx},{0.0453*\dy})
	-- ({1.2343*\dx},{0.0441*\dy})
	-- ({1.2289*\dx},{0.0430*\dy})
	-- ({1.2236*\dx},{0.0419*\dy})
	-- ({1.2184*\dx},{0.0408*\dy})
	-- ({1.2134*\dx},{0.0398*\dy})
	-- ({1.2086*\dx},{0.0388*\dy})
	-- ({1.2038*\dx},{0.0378*\dy})
	-- ({1.1991*\dx},{0.0369*\dy})
	-- ({1.1946*\dx},{0.0360*\dy})
	-- ({1.1902*\dx},{0.0351*\dy})
	-- ({1.1859*\dx},{0.0342*\dy})
	-- ({1.1817*\dx},{0.0334*\dy})
	-- ({1.1776*\dx},{0.0325*\dy})
	-- ({1.1736*\dx},{0.0318*\dy})
	-- ({1.1697*\dx},{0.0310*\dy})
	-- ({1.1659*\dx},{0.0302*\dy})
	-- ({1.1621*\dx},{0.0295*\dy})
	-- ({1.1585*\dx},{0.0288*\dy})
	-- ({1.1550*\dx},{0.0281*\dy})
	-- ({1.1515*\dx},{0.0274*\dy})
	-- ({1.1481*\dx},{0.0268*\dy})
	-- ({1.1448*\dx},{0.0261*\dy})
	-- ({1.1416*\dx},{0.0255*\dy})
	-- ({1.1385*\dx},{0.0249*\dy})
	-- ({1.1354*\dx},{0.0243*\dy})
	-- ({1.1324*\dx},{0.0237*\dy})
	-- ({1.1295*\dx},{0.0232*\dy})
	-- ({1.1266*\dx},{0.0226*\dy})
	-- ({1.1238*\dx},{0.0221*\dy})
	-- ({1.1211*\dx},{0.0216*\dy})
	-- ({1.1184*\dx},{0.0211*\dy})
	-- ({1.1158*\dx},{0.0206*\dy})
	-- ({1.1133*\dx},{0.0201*\dy})
	-- ({1.1108*\dx},{0.0197*\dy})
	-- ({1.1084*\dx},{0.0192*\dy})
	-- ({1.1060*\dx},{0.0188*\dy})
	-- ({1.1037*\dx},{0.0183*\dy})
	-- ({1.1014*\dx},{0.0179*\dy})
	-- ({1.0992*\dx},{0.0175*\dy})
	-- ({1.0971*\dx},{0.0171*\dy})
	-- ({1.0950*\dx},{0.0167*\dy})
	-- ({1.0929*\dx},{0.0163*\dy})
	-- ({1.0909*\dx},{0.0160*\dy})
	-- ({1.0889*\dx},{0.0156*\dy})
}
% v = 0.083160
\def\vpathJ{
	({-1.0889*\dx},{-0.0156*\dy})
	-- ({-1.0909*\dx},{-0.0160*\dy})
	-- ({-1.0929*\dx},{-0.0163*\dy})
	-- ({-1.0950*\dx},{-0.0167*\dy})
	-- ({-1.0971*\dx},{-0.0171*\dy})
	-- ({-1.0992*\dx},{-0.0175*\dy})
	-- ({-1.1014*\dx},{-0.0179*\dy})
	-- ({-1.1037*\dx},{-0.0183*\dy})
	-- ({-1.1060*\dx},{-0.0188*\dy})
	-- ({-1.1084*\dx},{-0.0192*\dy})
	-- ({-1.1108*\dx},{-0.0197*\dy})
	-- ({-1.1133*\dx},{-0.0201*\dy})
	-- ({-1.1158*\dx},{-0.0206*\dy})
	-- ({-1.1184*\dx},{-0.0211*\dy})
	-- ({-1.1211*\dx},{-0.0216*\dy})
	-- ({-1.1238*\dx},{-0.0221*\dy})
	-- ({-1.1266*\dx},{-0.0226*\dy})
	-- ({-1.1295*\dx},{-0.0232*\dy})
	-- ({-1.1324*\dx},{-0.0237*\dy})
	-- ({-1.1354*\dx},{-0.0243*\dy})
	-- ({-1.1385*\dx},{-0.0249*\dy})
	-- ({-1.1416*\dx},{-0.0255*\dy})
	-- ({-1.1448*\dx},{-0.0261*\dy})
	-- ({-1.1481*\dx},{-0.0268*\dy})
	-- ({-1.1515*\dx},{-0.0274*\dy})
	-- ({-1.1550*\dx},{-0.0281*\dy})
	-- ({-1.1585*\dx},{-0.0288*\dy})
	-- ({-1.1621*\dx},{-0.0295*\dy})
	-- ({-1.1659*\dx},{-0.0302*\dy})
	-- ({-1.1697*\dx},{-0.0310*\dy})
	-- ({-1.1736*\dx},{-0.0318*\dy})
	-- ({-1.1776*\dx},{-0.0325*\dy})
	-- ({-1.1817*\dx},{-0.0334*\dy})
	-- ({-1.1859*\dx},{-0.0342*\dy})
	-- ({-1.1902*\dx},{-0.0351*\dy})
	-- ({-1.1946*\dx},{-0.0360*\dy})
	-- ({-1.1991*\dx},{-0.0369*\dy})
	-- ({-1.2038*\dx},{-0.0378*\dy})
	-- ({-1.2086*\dx},{-0.0388*\dy})
	-- ({-1.2134*\dx},{-0.0398*\dy})
	-- ({-1.2184*\dx},{-0.0408*\dy})
	-- ({-1.2236*\dx},{-0.0419*\dy})
	-- ({-1.2289*\dx},{-0.0430*\dy})
	-- ({-1.2343*\dx},{-0.0441*\dy})
	-- ({-1.2398*\dx},{-0.0453*\dy})
	-- ({-1.2455*\dx},{-0.0465*\dy})
	-- ({-1.2514*\dx},{-0.0477*\dy})
	-- ({-1.2574*\dx},{-0.0490*\dy})
	-- ({-1.2635*\dx},{-0.0503*\dy})
	-- ({-1.2698*\dx},{-0.0516*\dy})
	-- ({-1.2763*\dx},{-0.0530*\dy})
	-- ({-1.2830*\dx},{-0.0545*\dy})
	-- ({-1.2899*\dx},{-0.0560*\dy})
	-- ({-1.2969*\dx},{-0.0575*\dy})
	-- ({-1.3041*\dx},{-0.0591*\dy})
	-- ({-1.3115*\dx},{-0.0608*\dy})
	-- ({-1.3192*\dx},{-0.0625*\dy})
	-- ({-1.3270*\dx},{-0.0642*\dy})
	-- ({-1.3351*\dx},{-0.0660*\dy})
	-- ({-1.3434*\dx},{-0.0679*\dy})
	-- ({-1.3519*\dx},{-0.0699*\dy})
	-- ({-1.3607*\dx},{-0.0719*\dy})
	-- ({-1.3697*\dx},{-0.0740*\dy})
	-- ({-1.3790*\dx},{-0.0762*\dy})
	-- ({-1.3885*\dx},{-0.0784*\dy})
	-- ({-1.3983*\dx},{-0.0807*\dy})
	-- ({-1.4084*\dx},{-0.0831*\dy})
	-- ({-1.4189*\dx},{-0.0857*\dy})
	-- ({-1.4296*\dx},{-0.0883*\dy})
	-- ({-1.4406*\dx},{-0.0910*\dy})
	-- ({-1.4520*\dx},{-0.0938*\dy})
	-- ({-1.4637*\dx},{-0.0967*\dy})
	-- ({-1.4758*\dx},{-0.0997*\dy})
	-- ({-1.4883*\dx},{-0.1029*\dy})
	-- ({-1.5011*\dx},{-0.1062*\dy})
	-- ({-1.5144*\dx},{-0.1096*\dy})
	-- ({-1.5281*\dx},{-0.1131*\dy})
	-- ({-1.5422*\dx},{-0.1168*\dy})
	-- ({-1.5568*\dx},{-0.1207*\dy})
	-- ({-1.5718*\dx},{-0.1247*\dy})
	-- ({-1.5873*\dx},{-0.1290*\dy})
	-- ({-1.6034*\dx},{-0.1333*\dy})
	-- ({-1.6200*\dx},{-0.1379*\dy})
	-- ({-1.6371*\dx},{-0.1427*\dy})
	-- ({-1.6549*\dx},{-0.1478*\dy})
	-- ({-1.6732*\dx},{-0.1530*\dy})
	-- ({-1.6922*\dx},{-0.1585*\dy})
	-- ({-1.7119*\dx},{-0.1643*\dy})
	-- ({-1.7322*\dx},{-0.1704*\dy})
	-- ({-1.7533*\dx},{-0.1767*\dy})
	-- ({-1.7752*\dx},{-0.1834*\dy})
	-- ({-1.7978*\dx},{-0.1904*\dy})
	-- ({-1.8213*\dx},{-0.1978*\dy})
	-- ({-1.8457*\dx},{-0.2056*\dy})
	-- ({-1.8710*\dx},{-0.2137*\dy})
	-- ({-1.8973*\dx},{-0.2224*\dy})
	-- ({-1.9246*\dx},{-0.2315*\dy})
	-- ({-1.9530*\dx},{-0.2411*\dy})
	-- ({-1.9826*\dx},{-0.2513*\dy})
	-- ({-2.0133*\dx},{-0.2621*\dy})
	-- ({-2.0453*\dx},{-0.2735*\dy})
	-- ({-2.0787*\dx},{-0.2856*\dy})
	-- ({-2.1134*\dx},{-0.2984*\dy})
	-- ({-2.1497*\dx},{-0.3121*\dy})
	-- ({-2.1875*\dx},{-0.3267*\dy})
	-- ({-2.2270*\dx},{-0.3422*\dy})
	-- ({-2.2683*\dx},{-0.3587*\dy})
	-- ({-2.3114*\dx},{-0.3764*\dy})
	-- ({-2.3566*\dx},{-0.3953*\dy})
	-- ({-2.4039*\dx},{-0.4156*\dy})
	-- ({-2.4534*\dx},{-0.4374*\dy})
	-- ({-2.5054*\dx},{-0.4607*\dy})
	-- ({-2.5599*\dx},{-0.4859*\dy})
	-- ({-2.6172*\dx},{-0.5131*\dy})
	-- ({-2.6774*\dx},{-0.5424*\dy})
	-- ({-2.7407*\dx},{-0.5742*\dy})
	-- ({-2.8074*\dx},{-0.6087*\dy})
	-- ({-2.8776*\dx},{-0.6461*\dy})
	-- ({-2.9517*\dx},{-0.6869*\dy})
	-- ({-3.0298*\dx},{-0.7315*\dy})
	-- ({-3.1124*\dx},{-0.7803*\dy})
	-- ({-3.1997*\dx},{-0.8338*\dy})
	-- ({-3.2921*\dx},{-0.8926*\dy})
	-- ({-3.3898*\dx},{-0.9575*\dy})
	-- ({-3.4934*\dx},{-1.0293*\dy})
	-- ({-3.6031*\dx},{-1.1090*\dy})
	-- ({-3.7195*\dx},{-1.1977*\dy})
	-- ({-3.8428*\dx},{-1.2967*\dy})
	-- ({-3.9736*\dx},{-1.4077*\dy})
	-- ({-4.1121*\dx},{-1.5325*\dy})
	-- ({-4.2586*\dx},{-1.6733*\dy})
	-- ({-4.4134*\dx},{-1.8329*\dy})
	-- ({-4.5763*\dx},{-2.0145*\dy})
	-- ({-4.7470*\dx},{-2.2218*\dy})
	-- ({-4.9247*\dx},{-2.4594*\dy})
	-- ({-5.1078*\dx},{-2.7327*\dy})
	-- ({-5.2939*\dx},{-3.0483*\dy})
	-- ({-5.4788*\dx},{-3.4137*\dy})
	-- ({-5.6563*\dx},{-3.8374*\dy})
	-- ({-5.8169*\dx},{-4.3293*\dy})
	-- ({-5.9470*\dx},{-4.8992*\dy})
	-- ({-6.0270*\dx},{-5.5566*\dy})
	-- ({-6.0301*\dx},{-6.3081*\dy})
	-- ({-5.9206*\dx},{-7.1537*\dy})
	-- ({-5.6544*\dx},{-8.0814*\dy})
	-- ({-5.1825*\dx},{-9.0597*\dy})
	-- ({-4.4601*\dx},{-10.0304*\dy})
	-- ({-3.4625*\dx},{-10.9064*\dy})
	-- ({-2.2052*\dx},{-11.5804*\dy})
	-- ({-0.7584*\dx},{-11.9495*\dy})
	-- ({0.7584*\dx},{-11.9495*\dy})
	-- ({2.2052*\dx},{-11.5804*\dy})
	-- ({3.4625*\dx},{-10.9064*\dy})
	-- ({4.4601*\dx},{-10.0304*\dy})
	-- ({5.1825*\dx},{-9.0597*\dy})
	-- ({5.6544*\dx},{-8.0814*\dy})
	-- ({5.9206*\dx},{-7.1537*\dy})
	-- ({6.0301*\dx},{-6.3081*\dy})
	-- ({6.0270*\dx},{-5.5566*\dy})
	-- ({5.9470*\dx},{-4.8992*\dy})
	-- ({5.8169*\dx},{-4.3293*\dy})
	-- ({5.6563*\dx},{-3.8374*\dy})
	-- ({5.4788*\dx},{-3.4137*\dy})
	-- ({5.2939*\dx},{-3.0483*\dy})
	-- ({5.1078*\dx},{-2.7327*\dy})
	-- ({4.9247*\dx},{-2.4594*\dy})
	-- ({4.7470*\dx},{-2.2218*\dy})
	-- ({4.5763*\dx},{-2.0145*\dy})
	-- ({4.4134*\dx},{-1.8329*\dy})
	-- ({4.2586*\dx},{-1.6733*\dy})
	-- ({4.1121*\dx},{-1.5325*\dy})
	-- ({3.9736*\dx},{-1.4077*\dy})
	-- ({3.8428*\dx},{-1.2967*\dy})
	-- ({3.7195*\dx},{-1.1977*\dy})
	-- ({3.6031*\dx},{-1.1090*\dy})
	-- ({3.4934*\dx},{-1.0293*\dy})
	-- ({3.3898*\dx},{-0.9575*\dy})
	-- ({3.2921*\dx},{-0.8926*\dy})
	-- ({3.1997*\dx},{-0.8338*\dy})
	-- ({3.1124*\dx},{-0.7803*\dy})
	-- ({3.0298*\dx},{-0.7315*\dy})
	-- ({2.9517*\dx},{-0.6869*\dy})
	-- ({2.8776*\dx},{-0.6461*\dy})
	-- ({2.8074*\dx},{-0.6087*\dy})
	-- ({2.7407*\dx},{-0.5742*\dy})
	-- ({2.6774*\dx},{-0.5424*\dy})
	-- ({2.6172*\dx},{-0.5131*\dy})
	-- ({2.5599*\dx},{-0.4859*\dy})
	-- ({2.5054*\dx},{-0.4607*\dy})
	-- ({2.4534*\dx},{-0.4374*\dy})
	-- ({2.4039*\dx},{-0.4156*\dy})
	-- ({2.3566*\dx},{-0.3953*\dy})
	-- ({2.3114*\dx},{-0.3764*\dy})
	-- ({2.2683*\dx},{-0.3587*\dy})
	-- ({2.2270*\dx},{-0.3422*\dy})
	-- ({2.1875*\dx},{-0.3267*\dy})
	-- ({2.1497*\dx},{-0.3121*\dy})
	-- ({2.1134*\dx},{-0.2984*\dy})
	-- ({2.0787*\dx},{-0.2856*\dy})
	-- ({2.0453*\dx},{-0.2735*\dy})
	-- ({2.0133*\dx},{-0.2621*\dy})
	-- ({1.9826*\dx},{-0.2513*\dy})
	-- ({1.9530*\dx},{-0.2411*\dy})
	-- ({1.9246*\dx},{-0.2315*\dy})
	-- ({1.8973*\dx},{-0.2224*\dy})
	-- ({1.8710*\dx},{-0.2137*\dy})
	-- ({1.8457*\dx},{-0.2056*\dy})
	-- ({1.8213*\dx},{-0.1978*\dy})
	-- ({1.7978*\dx},{-0.1904*\dy})
	-- ({1.7752*\dx},{-0.1834*\dy})
	-- ({1.7533*\dx},{-0.1767*\dy})
	-- ({1.7322*\dx},{-0.1704*\dy})
	-- ({1.7119*\dx},{-0.1643*\dy})
	-- ({1.6922*\dx},{-0.1585*\dy})
	-- ({1.6732*\dx},{-0.1530*\dy})
	-- ({1.6549*\dx},{-0.1478*\dy})
	-- ({1.6371*\dx},{-0.1427*\dy})
	-- ({1.6200*\dx},{-0.1379*\dy})
	-- ({1.6034*\dx},{-0.1333*\dy})
	-- ({1.5873*\dx},{-0.1290*\dy})
	-- ({1.5718*\dx},{-0.1247*\dy})
	-- ({1.5568*\dx},{-0.1207*\dy})
	-- ({1.5422*\dx},{-0.1168*\dy})
	-- ({1.5281*\dx},{-0.1131*\dy})
	-- ({1.5144*\dx},{-0.1096*\dy})
	-- ({1.5011*\dx},{-0.1062*\dy})
	-- ({1.4883*\dx},{-0.1029*\dy})
	-- ({1.4758*\dx},{-0.0997*\dy})
	-- ({1.4637*\dx},{-0.0967*\dy})
	-- ({1.4520*\dx},{-0.0938*\dy})
	-- ({1.4406*\dx},{-0.0910*\dy})
	-- ({1.4296*\dx},{-0.0883*\dy})
	-- ({1.4189*\dx},{-0.0857*\dy})
	-- ({1.4084*\dx},{-0.0831*\dy})
	-- ({1.3983*\dx},{-0.0807*\dy})
	-- ({1.3885*\dx},{-0.0784*\dy})
	-- ({1.3790*\dx},{-0.0762*\dy})
	-- ({1.3697*\dx},{-0.0740*\dy})
	-- ({1.3607*\dx},{-0.0719*\dy})
	-- ({1.3519*\dx},{-0.0699*\dy})
	-- ({1.3434*\dx},{-0.0679*\dy})
	-- ({1.3351*\dx},{-0.0660*\dy})
	-- ({1.3270*\dx},{-0.0642*\dy})
	-- ({1.3192*\dx},{-0.0625*\dy})
	-- ({1.3115*\dx},{-0.0608*\dy})
	-- ({1.3041*\dx},{-0.0591*\dy})
	-- ({1.2969*\dx},{-0.0575*\dy})
	-- ({1.2899*\dx},{-0.0560*\dy})
	-- ({1.2830*\dx},{-0.0545*\dy})
	-- ({1.2763*\dx},{-0.0530*\dy})
	-- ({1.2698*\dx},{-0.0516*\dy})
	-- ({1.2635*\dx},{-0.0503*\dy})
	-- ({1.2574*\dx},{-0.0490*\dy})
	-- ({1.2514*\dx},{-0.0477*\dy})
	-- ({1.2455*\dx},{-0.0465*\dy})
	-- ({1.2398*\dx},{-0.0453*\dy})
	-- ({1.2343*\dx},{-0.0441*\dy})
	-- ({1.2289*\dx},{-0.0430*\dy})
	-- ({1.2236*\dx},{-0.0419*\dy})
	-- ({1.2184*\dx},{-0.0408*\dy})
	-- ({1.2134*\dx},{-0.0398*\dy})
	-- ({1.2086*\dx},{-0.0388*\dy})
	-- ({1.2038*\dx},{-0.0378*\dy})
	-- ({1.1991*\dx},{-0.0369*\dy})
	-- ({1.1946*\dx},{-0.0360*\dy})
	-- ({1.1902*\dx},{-0.0351*\dy})
	-- ({1.1859*\dx},{-0.0342*\dy})
	-- ({1.1817*\dx},{-0.0334*\dy})
	-- ({1.1776*\dx},{-0.0325*\dy})
	-- ({1.1736*\dx},{-0.0318*\dy})
	-- ({1.1697*\dx},{-0.0310*\dy})
	-- ({1.1659*\dx},{-0.0302*\dy})
	-- ({1.1621*\dx},{-0.0295*\dy})
	-- ({1.1585*\dx},{-0.0288*\dy})
	-- ({1.1550*\dx},{-0.0281*\dy})
	-- ({1.1515*\dx},{-0.0274*\dy})
	-- ({1.1481*\dx},{-0.0268*\dy})
	-- ({1.1448*\dx},{-0.0261*\dy})
	-- ({1.1416*\dx},{-0.0255*\dy})
	-- ({1.1385*\dx},{-0.0249*\dy})
	-- ({1.1354*\dx},{-0.0243*\dy})
	-- ({1.1324*\dx},{-0.0237*\dy})
	-- ({1.1295*\dx},{-0.0232*\dy})
	-- ({1.1266*\dx},{-0.0226*\dy})
	-- ({1.1238*\dx},{-0.0221*\dy})
	-- ({1.1211*\dx},{-0.0216*\dy})
	-- ({1.1184*\dx},{-0.0211*\dy})
	-- ({1.1158*\dx},{-0.0206*\dy})
	-- ({1.1133*\dx},{-0.0201*\dy})
	-- ({1.1108*\dx},{-0.0197*\dy})
	-- ({1.1084*\dx},{-0.0192*\dy})
	-- ({1.1060*\dx},{-0.0188*\dy})
	-- ({1.1037*\dx},{-0.0183*\dy})
	-- ({1.1014*\dx},{-0.0179*\dy})
	-- ({1.0992*\dx},{-0.0175*\dy})
	-- ({1.0971*\dx},{-0.0171*\dy})
	-- ({1.0950*\dx},{-0.0167*\dy})
	-- ({1.0929*\dx},{-0.0163*\dy})
	-- ({1.0909*\dx},{-0.0160*\dy})
	-- ({1.0889*\dx},{-0.0156*\dy})
}
% v = 0.249479
\def\vpathK{
	({-1.0779*\dx},{-0.0447*\dy})
	-- ({-1.0796*\dx},{-0.0457*\dy})
	-- ({-1.0814*\dx},{-0.0467*\dy})
	-- ({-1.0831*\dx},{-0.0478*\dy})
	-- ({-1.0850*\dx},{-0.0489*\dy})
	-- ({-1.0868*\dx},{-0.0500*\dy})
	-- ({-1.0887*\dx},{-0.0512*\dy})
	-- ({-1.0906*\dx},{-0.0524*\dy})
	-- ({-1.0926*\dx},{-0.0536*\dy})
	-- ({-1.0947*\dx},{-0.0548*\dy})
	-- ({-1.0967*\dx},{-0.0561*\dy})
	-- ({-1.0989*\dx},{-0.0574*\dy})
	-- ({-1.1010*\dx},{-0.0588*\dy})
	-- ({-1.1032*\dx},{-0.0602*\dy})
	-- ({-1.1055*\dx},{-0.0616*\dy})
	-- ({-1.1078*\dx},{-0.0630*\dy})
	-- ({-1.1102*\dx},{-0.0645*\dy})
	-- ({-1.1126*\dx},{-0.0660*\dy})
	-- ({-1.1151*\dx},{-0.0676*\dy})
	-- ({-1.1177*\dx},{-0.0692*\dy})
	-- ({-1.1203*\dx},{-0.0708*\dy})
	-- ({-1.1229*\dx},{-0.0725*\dy})
	-- ({-1.1256*\dx},{-0.0743*\dy})
	-- ({-1.1284*\dx},{-0.0760*\dy})
	-- ({-1.1312*\dx},{-0.0779*\dy})
	-- ({-1.1342*\dx},{-0.0797*\dy})
	-- ({-1.1371*\dx},{-0.0817*\dy})
	-- ({-1.1402*\dx},{-0.0836*\dy})
	-- ({-1.1433*\dx},{-0.0857*\dy})
	-- ({-1.1465*\dx},{-0.0878*\dy})
	-- ({-1.1497*\dx},{-0.0899*\dy})
	-- ({-1.1531*\dx},{-0.0921*\dy})
	-- ({-1.1565*\dx},{-0.0944*\dy})
	-- ({-1.1600*\dx},{-0.0967*\dy})
	-- ({-1.1635*\dx},{-0.0991*\dy})
	-- ({-1.1672*\dx},{-0.1015*\dy})
	-- ({-1.1709*\dx},{-0.1041*\dy})
	-- ({-1.1748*\dx},{-0.1067*\dy})
	-- ({-1.1787*\dx},{-0.1093*\dy})
	-- ({-1.1827*\dx},{-0.1121*\dy})
	-- ({-1.1868*\dx},{-0.1149*\dy})
	-- ({-1.1910*\dx},{-0.1178*\dy})
	-- ({-1.1953*\dx},{-0.1208*\dy})
	-- ({-1.1997*\dx},{-0.1239*\dy})
	-- ({-1.2042*\dx},{-0.1271*\dy})
	-- ({-1.2088*\dx},{-0.1303*\dy})
	-- ({-1.2136*\dx},{-0.1337*\dy})
	-- ({-1.2184*\dx},{-0.1371*\dy})
	-- ({-1.2233*\dx},{-0.1407*\dy})
	-- ({-1.2284*\dx},{-0.1444*\dy})
	-- ({-1.2336*\dx},{-0.1482*\dy})
	-- ({-1.2389*\dx},{-0.1521*\dy})
	-- ({-1.2444*\dx},{-0.1561*\dy})
	-- ({-1.2499*\dx},{-0.1602*\dy})
	-- ({-1.2556*\dx},{-0.1645*\dy})
	-- ({-1.2615*\dx},{-0.1689*\dy})
	-- ({-1.2675*\dx},{-0.1735*\dy})
	-- ({-1.2736*\dx},{-0.1781*\dy})
	-- ({-1.2799*\dx},{-0.1830*\dy})
	-- ({-1.2863*\dx},{-0.1880*\dy})
	-- ({-1.2929*\dx},{-0.1932*\dy})
	-- ({-1.2996*\dx},{-0.1985*\dy})
	-- ({-1.3065*\dx},{-0.2040*\dy})
	-- ({-1.3136*\dx},{-0.2097*\dy})
	-- ({-1.3209*\dx},{-0.2156*\dy})
	-- ({-1.3283*\dx},{-0.2217*\dy})
	-- ({-1.3359*\dx},{-0.2280*\dy})
	-- ({-1.3437*\dx},{-0.2345*\dy})
	-- ({-1.3517*\dx},{-0.2412*\dy})
	-- ({-1.3599*\dx},{-0.2482*\dy})
	-- ({-1.3682*\dx},{-0.2554*\dy})
	-- ({-1.3768*\dx},{-0.2629*\dy})
	-- ({-1.3856*\dx},{-0.2706*\dy})
	-- ({-1.3946*\dx},{-0.2786*\dy})
	-- ({-1.4038*\dx},{-0.2869*\dy})
	-- ({-1.4133*\dx},{-0.2956*\dy})
	-- ({-1.4230*\dx},{-0.3045*\dy})
	-- ({-1.4329*\dx},{-0.3138*\dy})
	-- ({-1.4430*\dx},{-0.3234*\dy})
	-- ({-1.4534*\dx},{-0.3334*\dy})
	-- ({-1.4641*\dx},{-0.3438*\dy})
	-- ({-1.4750*\dx},{-0.3545*\dy})
	-- ({-1.4861*\dx},{-0.3657*\dy})
	-- ({-1.4975*\dx},{-0.3774*\dy})
	-- ({-1.5092*\dx},{-0.3895*\dy})
	-- ({-1.5212*\dx},{-0.4021*\dy})
	-- ({-1.5335*\dx},{-0.4152*\dy})
	-- ({-1.5460*\dx},{-0.4289*\dy})
	-- ({-1.5588*\dx},{-0.4431*\dy})
	-- ({-1.5719*\dx},{-0.4579*\dy})
	-- ({-1.5853*\dx},{-0.4734*\dy})
	-- ({-1.5990*\dx},{-0.4895*\dy})
	-- ({-1.6130*\dx},{-0.5063*\dy})
	-- ({-1.6273*\dx},{-0.5238*\dy})
	-- ({-1.6419*\dx},{-0.5421*\dy})
	-- ({-1.6567*\dx},{-0.5612*\dy})
	-- ({-1.6719*\dx},{-0.5812*\dy})
	-- ({-1.6874*\dx},{-0.6021*\dy})
	-- ({-1.7031*\dx},{-0.6239*\dy})
	-- ({-1.7191*\dx},{-0.6468*\dy})
	-- ({-1.7354*\dx},{-0.6707*\dy})
	-- ({-1.7519*\dx},{-0.6957*\dy})
	-- ({-1.7687*\dx},{-0.7219*\dy})
	-- ({-1.7857*\dx},{-0.7494*\dy})
	-- ({-1.8029*\dx},{-0.7782*\dy})
	-- ({-1.8202*\dx},{-0.8084*\dy})
	-- ({-1.8377*\dx},{-0.8400*\dy})
	-- ({-1.8554*\dx},{-0.8733*\dy})
	-- ({-1.8730*\dx},{-0.9082*\dy})
	-- ({-1.8907*\dx},{-0.9448*\dy})
	-- ({-1.9084*\dx},{-0.9833*\dy})
	-- ({-1.9259*\dx},{-1.0237*\dy})
	-- ({-1.9432*\dx},{-1.0662*\dy})
	-- ({-1.9603*\dx},{-1.1108*\dy})
	-- ({-1.9770*\dx},{-1.1577*\dy})
	-- ({-1.9932*\dx},{-1.2070*\dy})
	-- ({-2.0088*\dx},{-1.2589*\dy})
	-- ({-2.0237*\dx},{-1.3134*\dy})
	-- ({-2.0376*\dx},{-1.3706*\dy})
	-- ({-2.0503*\dx},{-1.4308*\dy})
	-- ({-2.0618*\dx},{-1.4940*\dy})
	-- ({-2.0717*\dx},{-1.5603*\dy})
	-- ({-2.0797*\dx},{-1.6298*\dy})
	-- ({-2.0856*\dx},{-1.7028*\dy})
	-- ({-2.0891*\dx},{-1.7791*\dy})
	-- ({-2.0897*\dx},{-1.8590*\dy})
	-- ({-2.0871*\dx},{-1.9424*\dy})
	-- ({-2.0808*\dx},{-2.0294*\dy})
	-- ({-2.0703*\dx},{-2.1199*\dy})
	-- ({-2.0552*\dx},{-2.2138*\dy})
	-- ({-2.0349*\dx},{-2.3110*\dy})
	-- ({-2.0088*\dx},{-2.4113*\dy})
	-- ({-1.9763*\dx},{-2.5145*\dy})
	-- ({-1.9368*\dx},{-2.6200*\dy})
	-- ({-1.8896*\dx},{-2.7276*\dy})
	-- ({-1.8343*\dx},{-2.8364*\dy})
	-- ({-1.7701*\dx},{-2.9460*\dy})
	-- ({-1.6965*\dx},{-3.0553*\dy})
	-- ({-1.6132*\dx},{-3.1635*\dy})
	-- ({-1.5198*\dx},{-3.2694*\dy})
	-- ({-1.4161*\dx},{-3.3719*\dy})
	-- ({-1.3020*\dx},{-3.4696*\dy})
	-- ({-1.1778*\dx},{-3.5613*\dy})
	-- ({-1.0438*\dx},{-3.6455*\dy})
	-- ({-0.9007*\dx},{-3.7208*\dy})
	-- ({-0.7493*\dx},{-3.7859*\dy})
	-- ({-0.5907*\dx},{-3.8396*\dy})
	-- ({-0.4263*\dx},{-3.8809*\dy})
	-- ({-0.2575*\dx},{-3.9089*\dy})
	-- ({-0.0861*\dx},{-3.9231*\dy})
	-- ({0.0861*\dx},{-3.9231*\dy})
	-- ({0.2575*\dx},{-3.9089*\dy})
	-- ({0.4263*\dx},{-3.8809*\dy})
	-- ({0.5907*\dx},{-3.8396*\dy})
	-- ({0.7493*\dx},{-3.7859*\dy})
	-- ({0.9007*\dx},{-3.7208*\dy})
	-- ({1.0438*\dx},{-3.6455*\dy})
	-- ({1.1778*\dx},{-3.5613*\dy})
	-- ({1.3020*\dx},{-3.4696*\dy})
	-- ({1.4161*\dx},{-3.3719*\dy})
	-- ({1.5198*\dx},{-3.2694*\dy})
	-- ({1.6132*\dx},{-3.1635*\dy})
	-- ({1.6965*\dx},{-3.0553*\dy})
	-- ({1.7701*\dx},{-2.9460*\dy})
	-- ({1.8343*\dx},{-2.8364*\dy})
	-- ({1.8896*\dx},{-2.7276*\dy})
	-- ({1.9368*\dx},{-2.6200*\dy})
	-- ({1.9763*\dx},{-2.5145*\dy})
	-- ({2.0088*\dx},{-2.4113*\dy})
	-- ({2.0349*\dx},{-2.3110*\dy})
	-- ({2.0552*\dx},{-2.2138*\dy})
	-- ({2.0703*\dx},{-2.1199*\dy})
	-- ({2.0808*\dx},{-2.0294*\dy})
	-- ({2.0871*\dx},{-1.9424*\dy})
	-- ({2.0897*\dx},{-1.8590*\dy})
	-- ({2.0891*\dx},{-1.7791*\dy})
	-- ({2.0856*\dx},{-1.7028*\dy})
	-- ({2.0797*\dx},{-1.6298*\dy})
	-- ({2.0717*\dx},{-1.5603*\dy})
	-- ({2.0618*\dx},{-1.4940*\dy})
	-- ({2.0503*\dx},{-1.4308*\dy})
	-- ({2.0376*\dx},{-1.3706*\dy})
	-- ({2.0237*\dx},{-1.3134*\dy})
	-- ({2.0088*\dx},{-1.2589*\dy})
	-- ({1.9932*\dx},{-1.2070*\dy})
	-- ({1.9770*\dx},{-1.1577*\dy})
	-- ({1.9603*\dx},{-1.1108*\dy})
	-- ({1.9432*\dx},{-1.0662*\dy})
	-- ({1.9259*\dx},{-1.0237*\dy})
	-- ({1.9084*\dx},{-0.9833*\dy})
	-- ({1.8907*\dx},{-0.9448*\dy})
	-- ({1.8730*\dx},{-0.9082*\dy})
	-- ({1.8554*\dx},{-0.8733*\dy})
	-- ({1.8377*\dx},{-0.8400*\dy})
	-- ({1.8202*\dx},{-0.8084*\dy})
	-- ({1.8029*\dx},{-0.7782*\dy})
	-- ({1.7857*\dx},{-0.7494*\dy})
	-- ({1.7687*\dx},{-0.7219*\dy})
	-- ({1.7519*\dx},{-0.6957*\dy})
	-- ({1.7354*\dx},{-0.6707*\dy})
	-- ({1.7191*\dx},{-0.6468*\dy})
	-- ({1.7031*\dx},{-0.6239*\dy})
	-- ({1.6874*\dx},{-0.6021*\dy})
	-- ({1.6719*\dx},{-0.5812*\dy})
	-- ({1.6567*\dx},{-0.5612*\dy})
	-- ({1.6419*\dx},{-0.5421*\dy})
	-- ({1.6273*\dx},{-0.5238*\dy})
	-- ({1.6130*\dx},{-0.5063*\dy})
	-- ({1.5990*\dx},{-0.4895*\dy})
	-- ({1.5853*\dx},{-0.4734*\dy})
	-- ({1.5719*\dx},{-0.4579*\dy})
	-- ({1.5588*\dx},{-0.4431*\dy})
	-- ({1.5460*\dx},{-0.4289*\dy})
	-- ({1.5335*\dx},{-0.4152*\dy})
	-- ({1.5212*\dx},{-0.4021*\dy})
	-- ({1.5092*\dx},{-0.3895*\dy})
	-- ({1.4975*\dx},{-0.3774*\dy})
	-- ({1.4861*\dx},{-0.3657*\dy})
	-- ({1.4750*\dx},{-0.3545*\dy})
	-- ({1.4641*\dx},{-0.3438*\dy})
	-- ({1.4534*\dx},{-0.3334*\dy})
	-- ({1.4430*\dx},{-0.3234*\dy})
	-- ({1.4329*\dx},{-0.3138*\dy})
	-- ({1.4230*\dx},{-0.3045*\dy})
	-- ({1.4133*\dx},{-0.2956*\dy})
	-- ({1.4038*\dx},{-0.2869*\dy})
	-- ({1.3946*\dx},{-0.2786*\dy})
	-- ({1.3856*\dx},{-0.2706*\dy})
	-- ({1.3768*\dx},{-0.2629*\dy})
	-- ({1.3682*\dx},{-0.2554*\dy})
	-- ({1.3599*\dx},{-0.2482*\dy})
	-- ({1.3517*\dx},{-0.2412*\dy})
	-- ({1.3437*\dx},{-0.2345*\dy})
	-- ({1.3359*\dx},{-0.2280*\dy})
	-- ({1.3283*\dx},{-0.2217*\dy})
	-- ({1.3209*\dx},{-0.2156*\dy})
	-- ({1.3136*\dx},{-0.2097*\dy})
	-- ({1.3065*\dx},{-0.2040*\dy})
	-- ({1.2996*\dx},{-0.1985*\dy})
	-- ({1.2929*\dx},{-0.1932*\dy})
	-- ({1.2863*\dx},{-0.1880*\dy})
	-- ({1.2799*\dx},{-0.1830*\dy})
	-- ({1.2736*\dx},{-0.1781*\dy})
	-- ({1.2675*\dx},{-0.1735*\dy})
	-- ({1.2615*\dx},{-0.1689*\dy})
	-- ({1.2556*\dx},{-0.1645*\dy})
	-- ({1.2499*\dx},{-0.1602*\dy})
	-- ({1.2444*\dx},{-0.1561*\dy})
	-- ({1.2389*\dx},{-0.1521*\dy})
	-- ({1.2336*\dx},{-0.1482*\dy})
	-- ({1.2284*\dx},{-0.1444*\dy})
	-- ({1.2233*\dx},{-0.1407*\dy})
	-- ({1.2184*\dx},{-0.1371*\dy})
	-- ({1.2136*\dx},{-0.1337*\dy})
	-- ({1.2088*\dx},{-0.1303*\dy})
	-- ({1.2042*\dx},{-0.1271*\dy})
	-- ({1.1997*\dx},{-0.1239*\dy})
	-- ({1.1953*\dx},{-0.1208*\dy})
	-- ({1.1910*\dx},{-0.1178*\dy})
	-- ({1.1868*\dx},{-0.1149*\dy})
	-- ({1.1827*\dx},{-0.1121*\dy})
	-- ({1.1787*\dx},{-0.1093*\dy})
	-- ({1.1748*\dx},{-0.1067*\dy})
	-- ({1.1709*\dx},{-0.1041*\dy})
	-- ({1.1672*\dx},{-0.1015*\dy})
	-- ({1.1635*\dx},{-0.0991*\dy})
	-- ({1.1600*\dx},{-0.0967*\dy})
	-- ({1.1565*\dx},{-0.0944*\dy})
	-- ({1.1531*\dx},{-0.0921*\dy})
	-- ({1.1497*\dx},{-0.0899*\dy})
	-- ({1.1465*\dx},{-0.0878*\dy})
	-- ({1.1433*\dx},{-0.0857*\dy})
	-- ({1.1402*\dx},{-0.0836*\dy})
	-- ({1.1371*\dx},{-0.0817*\dy})
	-- ({1.1342*\dx},{-0.0797*\dy})
	-- ({1.1312*\dx},{-0.0779*\dy})
	-- ({1.1284*\dx},{-0.0760*\dy})
	-- ({1.1256*\dx},{-0.0743*\dy})
	-- ({1.1229*\dx},{-0.0725*\dy})
	-- ({1.1203*\dx},{-0.0708*\dy})
	-- ({1.1177*\dx},{-0.0692*\dy})
	-- ({1.1151*\dx},{-0.0676*\dy})
	-- ({1.1126*\dx},{-0.0660*\dy})
	-- ({1.1102*\dx},{-0.0645*\dy})
	-- ({1.1078*\dx},{-0.0630*\dy})
	-- ({1.1055*\dx},{-0.0616*\dy})
	-- ({1.1032*\dx},{-0.0602*\dy})
	-- ({1.1010*\dx},{-0.0588*\dy})
	-- ({1.0989*\dx},{-0.0574*\dy})
	-- ({1.0967*\dx},{-0.0561*\dy})
	-- ({1.0947*\dx},{-0.0548*\dy})
	-- ({1.0926*\dx},{-0.0536*\dy})
	-- ({1.0906*\dx},{-0.0524*\dy})
	-- ({1.0887*\dx},{-0.0512*\dy})
	-- ({1.0868*\dx},{-0.0500*\dy})
	-- ({1.0850*\dx},{-0.0489*\dy})
	-- ({1.0831*\dx},{-0.0478*\dy})
	-- ({1.0814*\dx},{-0.0467*\dy})
	-- ({1.0796*\dx},{-0.0457*\dy})
	-- ({1.0779*\dx},{-0.0447*\dy})
}
% v = 0.415799
\def\vpathL{
	({-1.0577*\dx},{-0.0677*\dy})
	-- ({-1.0590*\dx},{-0.0692*\dy})
	-- ({-1.0602*\dx},{-0.0708*\dy})
	-- ({-1.0615*\dx},{-0.0724*\dy})
	-- ({-1.0627*\dx},{-0.0740*\dy})
	-- ({-1.0641*\dx},{-0.0757*\dy})
	-- ({-1.0654*\dx},{-0.0774*\dy})
	-- ({-1.0668*\dx},{-0.0791*\dy})
	-- ({-1.0682*\dx},{-0.0809*\dy})
	-- ({-1.0696*\dx},{-0.0828*\dy})
	-- ({-1.0711*\dx},{-0.0846*\dy})
	-- ({-1.0725*\dx},{-0.0866*\dy})
	-- ({-1.0741*\dx},{-0.0886*\dy})
	-- ({-1.0756*\dx},{-0.0906*\dy})
	-- ({-1.0772*\dx},{-0.0926*\dy})
	-- ({-1.0788*\dx},{-0.0948*\dy})
	-- ({-1.0804*\dx},{-0.0969*\dy})
	-- ({-1.0821*\dx},{-0.0992*\dy})
	-- ({-1.0838*\dx},{-0.1015*\dy})
	-- ({-1.0856*\dx},{-0.1038*\dy})
	-- ({-1.0874*\dx},{-0.1062*\dy})
	-- ({-1.0892*\dx},{-0.1086*\dy})
	-- ({-1.0910*\dx},{-0.1112*\dy})
	-- ({-1.0929*\dx},{-0.1137*\dy})
	-- ({-1.0948*\dx},{-0.1164*\dy})
	-- ({-1.0968*\dx},{-0.1191*\dy})
	-- ({-1.0988*\dx},{-0.1219*\dy})
	-- ({-1.1008*\dx},{-0.1247*\dy})
	-- ({-1.1029*\dx},{-0.1277*\dy})
	-- ({-1.1050*\dx},{-0.1306*\dy})
	-- ({-1.1072*\dx},{-0.1337*\dy})
	-- ({-1.1094*\dx},{-0.1369*\dy})
	-- ({-1.1117*\dx},{-0.1401*\dy})
	-- ({-1.1140*\dx},{-0.1434*\dy})
	-- ({-1.1163*\dx},{-0.1468*\dy})
	-- ({-1.1187*\dx},{-0.1503*\dy})
	-- ({-1.1211*\dx},{-0.1539*\dy})
	-- ({-1.1236*\dx},{-0.1576*\dy})
	-- ({-1.1261*\dx},{-0.1613*\dy})
	-- ({-1.1287*\dx},{-0.1652*\dy})
	-- ({-1.1313*\dx},{-0.1692*\dy})
	-- ({-1.1339*\dx},{-0.1732*\dy})
	-- ({-1.1367*\dx},{-0.1774*\dy})
	-- ({-1.1394*\dx},{-0.1817*\dy})
	-- ({-1.1422*\dx},{-0.1861*\dy})
	-- ({-1.1451*\dx},{-0.1907*\dy})
	-- ({-1.1480*\dx},{-0.1953*\dy})
	-- ({-1.1510*\dx},{-0.2001*\dy})
	-- ({-1.1540*\dx},{-0.2050*\dy})
	-- ({-1.1571*\dx},{-0.2100*\dy})
	-- ({-1.1602*\dx},{-0.2152*\dy})
	-- ({-1.1634*\dx},{-0.2205*\dy})
	-- ({-1.1666*\dx},{-0.2260*\dy})
	-- ({-1.1699*\dx},{-0.2316*\dy})
	-- ({-1.1732*\dx},{-0.2374*\dy})
	-- ({-1.1766*\dx},{-0.2433*\dy})
	-- ({-1.1800*\dx},{-0.2494*\dy})
	-- ({-1.1835*\dx},{-0.2557*\dy})
	-- ({-1.1871*\dx},{-0.2621*\dy})
	-- ({-1.1907*\dx},{-0.2687*\dy})
	-- ({-1.1944*\dx},{-0.2756*\dy})
	-- ({-1.1981*\dx},{-0.2826*\dy})
	-- ({-1.2018*\dx},{-0.2898*\dy})
	-- ({-1.2056*\dx},{-0.2972*\dy})
	-- ({-1.2095*\dx},{-0.3049*\dy})
	-- ({-1.2134*\dx},{-0.3127*\dy})
	-- ({-1.2174*\dx},{-0.3208*\dy})
	-- ({-1.2214*\dx},{-0.3291*\dy})
	-- ({-1.2254*\dx},{-0.3377*\dy})
	-- ({-1.2295*\dx},{-0.3465*\dy})
	-- ({-1.2337*\dx},{-0.3556*\dy})
	-- ({-1.2378*\dx},{-0.3650*\dy})
	-- ({-1.2420*\dx},{-0.3746*\dy})
	-- ({-1.2463*\dx},{-0.3845*\dy})
	-- ({-1.2506*\dx},{-0.3948*\dy})
	-- ({-1.2549*\dx},{-0.4053*\dy})
	-- ({-1.2592*\dx},{-0.4161*\dy})
	-- ({-1.2635*\dx},{-0.4273*\dy})
	-- ({-1.2679*\dx},{-0.4388*\dy})
	-- ({-1.2722*\dx},{-0.4507*\dy})
	-- ({-1.2766*\dx},{-0.4629*\dy})
	-- ({-1.2810*\dx},{-0.4755*\dy})
	-- ({-1.2853*\dx},{-0.4885*\dy})
	-- ({-1.2897*\dx},{-0.5019*\dy})
	-- ({-1.2940*\dx},{-0.5157*\dy})
	-- ({-1.2982*\dx},{-0.5300*\dy})
	-- ({-1.3025*\dx},{-0.5447*\dy})
	-- ({-1.3066*\dx},{-0.5598*\dy})
	-- ({-1.3107*\dx},{-0.5754*\dy})
	-- ({-1.3148*\dx},{-0.5915*\dy})
	-- ({-1.3187*\dx},{-0.6081*\dy})
	-- ({-1.3225*\dx},{-0.6252*\dy})
	-- ({-1.3262*\dx},{-0.6429*\dy})
	-- ({-1.3298*\dx},{-0.6611*\dy})
	-- ({-1.3332*\dx},{-0.6798*\dy})
	-- ({-1.3364*\dx},{-0.6992*\dy})
	-- ({-1.3394*\dx},{-0.7191*\dy})
	-- ({-1.3422*\dx},{-0.7396*\dy})
	-- ({-1.3447*\dx},{-0.7608*\dy})
	-- ({-1.3470*\dx},{-0.7826*\dy})
	-- ({-1.3490*\dx},{-0.8051*\dy})
	-- ({-1.3506*\dx},{-0.8283*\dy})
	-- ({-1.3519*\dx},{-0.8521*\dy})
	-- ({-1.3527*\dx},{-0.8767*\dy})
	-- ({-1.3531*\dx},{-0.9020*\dy})
	-- ({-1.3531*\dx},{-0.9280*\dy})
	-- ({-1.3525*\dx},{-0.9548*\dy})
	-- ({-1.3513*\dx},{-0.9823*\dy})
	-- ({-1.3495*\dx},{-1.0106*\dy})
	-- ({-1.3471*\dx},{-1.0396*\dy})
	-- ({-1.3439*\dx},{-1.0694*\dy})
	-- ({-1.3400*\dx},{-1.1000*\dy})
	-- ({-1.3352*\dx},{-1.1314*\dy})
	-- ({-1.3295*\dx},{-1.1635*\dy})
	-- ({-1.3229*\dx},{-1.1964*\dy})
	-- ({-1.3152*\dx},{-1.2300*\dy})
	-- ({-1.3064*\dx},{-1.2644*\dy})
	-- ({-1.2964*\dx},{-1.2994*\dy})
	-- ({-1.2852*\dx},{-1.3351*\dy})
	-- ({-1.2726*\dx},{-1.3715*\dy})
	-- ({-1.2586*\dx},{-1.4085*\dy})
	-- ({-1.2432*\dx},{-1.4460*\dy})
	-- ({-1.2261*\dx},{-1.4840*\dy})
	-- ({-1.2074*\dx},{-1.5225*\dy})
	-- ({-1.1870*\dx},{-1.5613*\dy})
	-- ({-1.1648*\dx},{-1.6003*\dy})
	-- ({-1.1407*\dx},{-1.6396*\dy})
	-- ({-1.1146*\dx},{-1.6789*\dy})
	-- ({-1.0865*\dx},{-1.7182*\dy})
	-- ({-1.0564*\dx},{-1.7573*\dy})
	-- ({-1.0240*\dx},{-1.7961*\dy})
	-- ({-0.9896*\dx},{-1.8346*\dy})
	-- ({-0.9529*\dx},{-1.8724*\dy})
	-- ({-0.9140*\dx},{-1.9095*\dy})
	-- ({-0.8728*\dx},{-1.9457*\dy})
	-- ({-0.8294*\dx},{-1.9808*\dy})
	-- ({-0.7838*\dx},{-2.0147*\dy})
	-- ({-0.7360*\dx},{-2.0471*\dy})
	-- ({-0.6861*\dx},{-2.0779*\dy})
	-- ({-0.6342*\dx},{-2.1070*\dy})
	-- ({-0.5803*\dx},{-2.1340*\dy})
	-- ({-0.5246*\dx},{-2.1590*\dy})
	-- ({-0.4672*\dx},{-2.1816*\dy})
	-- ({-0.4082*\dx},{-2.2017*\dy})
	-- ({-0.3479*\dx},{-2.2193*\dy})
	-- ({-0.2863*\dx},{-2.2341*\dy})
	-- ({-0.2237*\dx},{-2.2462*\dy})
	-- ({-0.1604*\dx},{-2.2552*\dy})
	-- ({-0.0965*\dx},{-2.2613*\dy})
	-- ({-0.0322*\dx},{-2.2644*\dy})
	-- ({0.0322*\dx},{-2.2644*\dy})
	-- ({0.0965*\dx},{-2.2613*\dy})
	-- ({0.1604*\dx},{-2.2552*\dy})
	-- ({0.2237*\dx},{-2.2462*\dy})
	-- ({0.2863*\dx},{-2.2341*\dy})
	-- ({0.3479*\dx},{-2.2193*\dy})
	-- ({0.4082*\dx},{-2.2017*\dy})
	-- ({0.4672*\dx},{-2.1816*\dy})
	-- ({0.5246*\dx},{-2.1590*\dy})
	-- ({0.5803*\dx},{-2.1340*\dy})
	-- ({0.6342*\dx},{-2.1070*\dy})
	-- ({0.6861*\dx},{-2.0779*\dy})
	-- ({0.7360*\dx},{-2.0471*\dy})
	-- ({0.7838*\dx},{-2.0147*\dy})
	-- ({0.8294*\dx},{-1.9808*\dy})
	-- ({0.8728*\dx},{-1.9457*\dy})
	-- ({0.9140*\dx},{-1.9095*\dy})
	-- ({0.9529*\dx},{-1.8724*\dy})
	-- ({0.9896*\dx},{-1.8346*\dy})
	-- ({1.0240*\dx},{-1.7961*\dy})
	-- ({1.0564*\dx},{-1.7573*\dy})
	-- ({1.0865*\dx},{-1.7182*\dy})
	-- ({1.1146*\dx},{-1.6789*\dy})
	-- ({1.1407*\dx},{-1.6396*\dy})
	-- ({1.1648*\dx},{-1.6003*\dy})
	-- ({1.1870*\dx},{-1.5613*\dy})
	-- ({1.2074*\dx},{-1.5225*\dy})
	-- ({1.2261*\dx},{-1.4840*\dy})
	-- ({1.2432*\dx},{-1.4460*\dy})
	-- ({1.2586*\dx},{-1.4085*\dy})
	-- ({1.2726*\dx},{-1.3715*\dy})
	-- ({1.2852*\dx},{-1.3351*\dy})
	-- ({1.2964*\dx},{-1.2994*\dy})
	-- ({1.3064*\dx},{-1.2644*\dy})
	-- ({1.3152*\dx},{-1.2300*\dy})
	-- ({1.3229*\dx},{-1.1964*\dy})
	-- ({1.3295*\dx},{-1.1635*\dy})
	-- ({1.3352*\dx},{-1.1314*\dy})
	-- ({1.3400*\dx},{-1.1000*\dy})
	-- ({1.3439*\dx},{-1.0694*\dy})
	-- ({1.3471*\dx},{-1.0396*\dy})
	-- ({1.3495*\dx},{-1.0106*\dy})
	-- ({1.3513*\dx},{-0.9823*\dy})
	-- ({1.3525*\dx},{-0.9548*\dy})
	-- ({1.3531*\dx},{-0.9280*\dy})
	-- ({1.3531*\dx},{-0.9020*\dy})
	-- ({1.3527*\dx},{-0.8767*\dy})
	-- ({1.3519*\dx},{-0.8521*\dy})
	-- ({1.3506*\dx},{-0.8283*\dy})
	-- ({1.3490*\dx},{-0.8051*\dy})
	-- ({1.3470*\dx},{-0.7826*\dy})
	-- ({1.3447*\dx},{-0.7608*\dy})
	-- ({1.3422*\dx},{-0.7396*\dy})
	-- ({1.3394*\dx},{-0.7191*\dy})
	-- ({1.3364*\dx},{-0.6992*\dy})
	-- ({1.3332*\dx},{-0.6798*\dy})
	-- ({1.3298*\dx},{-0.6611*\dy})
	-- ({1.3262*\dx},{-0.6429*\dy})
	-- ({1.3225*\dx},{-0.6252*\dy})
	-- ({1.3187*\dx},{-0.6081*\dy})
	-- ({1.3148*\dx},{-0.5915*\dy})
	-- ({1.3107*\dx},{-0.5754*\dy})
	-- ({1.3066*\dx},{-0.5598*\dy})
	-- ({1.3025*\dx},{-0.5447*\dy})
	-- ({1.2982*\dx},{-0.5300*\dy})
	-- ({1.2940*\dx},{-0.5157*\dy})
	-- ({1.2897*\dx},{-0.5019*\dy})
	-- ({1.2853*\dx},{-0.4885*\dy})
	-- ({1.2810*\dx},{-0.4755*\dy})
	-- ({1.2766*\dx},{-0.4629*\dy})
	-- ({1.2722*\dx},{-0.4507*\dy})
	-- ({1.2679*\dx},{-0.4388*\dy})
	-- ({1.2635*\dx},{-0.4273*\dy})
	-- ({1.2592*\dx},{-0.4161*\dy})
	-- ({1.2549*\dx},{-0.4053*\dy})
	-- ({1.2506*\dx},{-0.3948*\dy})
	-- ({1.2463*\dx},{-0.3845*\dy})
	-- ({1.2420*\dx},{-0.3746*\dy})
	-- ({1.2378*\dx},{-0.3650*\dy})
	-- ({1.2337*\dx},{-0.3556*\dy})
	-- ({1.2295*\dx},{-0.3465*\dy})
	-- ({1.2254*\dx},{-0.3377*\dy})
	-- ({1.2214*\dx},{-0.3291*\dy})
	-- ({1.2174*\dx},{-0.3208*\dy})
	-- ({1.2134*\dx},{-0.3127*\dy})
	-- ({1.2095*\dx},{-0.3049*\dy})
	-- ({1.2056*\dx},{-0.2972*\dy})
	-- ({1.2018*\dx},{-0.2898*\dy})
	-- ({1.1981*\dx},{-0.2826*\dy})
	-- ({1.1944*\dx},{-0.2756*\dy})
	-- ({1.1907*\dx},{-0.2687*\dy})
	-- ({1.1871*\dx},{-0.2621*\dy})
	-- ({1.1835*\dx},{-0.2557*\dy})
	-- ({1.1800*\dx},{-0.2494*\dy})
	-- ({1.1766*\dx},{-0.2433*\dy})
	-- ({1.1732*\dx},{-0.2374*\dy})
	-- ({1.1699*\dx},{-0.2316*\dy})
	-- ({1.1666*\dx},{-0.2260*\dy})
	-- ({1.1634*\dx},{-0.2205*\dy})
	-- ({1.1602*\dx},{-0.2152*\dy})
	-- ({1.1571*\dx},{-0.2100*\dy})
	-- ({1.1540*\dx},{-0.2050*\dy})
	-- ({1.1510*\dx},{-0.2001*\dy})
	-- ({1.1480*\dx},{-0.1953*\dy})
	-- ({1.1451*\dx},{-0.1907*\dy})
	-- ({1.1422*\dx},{-0.1861*\dy})
	-- ({1.1394*\dx},{-0.1817*\dy})
	-- ({1.1367*\dx},{-0.1774*\dy})
	-- ({1.1339*\dx},{-0.1732*\dy})
	-- ({1.1313*\dx},{-0.1692*\dy})
	-- ({1.1287*\dx},{-0.1652*\dy})
	-- ({1.1261*\dx},{-0.1613*\dy})
	-- ({1.1236*\dx},{-0.1576*\dy})
	-- ({1.1211*\dx},{-0.1539*\dy})
	-- ({1.1187*\dx},{-0.1503*\dy})
	-- ({1.1163*\dx},{-0.1468*\dy})
	-- ({1.1140*\dx},{-0.1434*\dy})
	-- ({1.1117*\dx},{-0.1401*\dy})
	-- ({1.1094*\dx},{-0.1369*\dy})
	-- ({1.1072*\dx},{-0.1337*\dy})
	-- ({1.1050*\dx},{-0.1306*\dy})
	-- ({1.1029*\dx},{-0.1277*\dy})
	-- ({1.1008*\dx},{-0.1247*\dy})
	-- ({1.0988*\dx},{-0.1219*\dy})
	-- ({1.0968*\dx},{-0.1191*\dy})
	-- ({1.0948*\dx},{-0.1164*\dy})
	-- ({1.0929*\dx},{-0.1137*\dy})
	-- ({1.0910*\dx},{-0.1112*\dy})
	-- ({1.0892*\dx},{-0.1086*\dy})
	-- ({1.0874*\dx},{-0.1062*\dy})
	-- ({1.0856*\dx},{-0.1038*\dy})
	-- ({1.0838*\dx},{-0.1015*\dy})
	-- ({1.0821*\dx},{-0.0992*\dy})
	-- ({1.0804*\dx},{-0.0969*\dy})
	-- ({1.0788*\dx},{-0.0948*\dy})
	-- ({1.0772*\dx},{-0.0926*\dy})
	-- ({1.0756*\dx},{-0.0906*\dy})
	-- ({1.0741*\dx},{-0.0886*\dy})
	-- ({1.0725*\dx},{-0.0866*\dy})
	-- ({1.0711*\dx},{-0.0846*\dy})
	-- ({1.0696*\dx},{-0.0828*\dy})
	-- ({1.0682*\dx},{-0.0809*\dy})
	-- ({1.0668*\dx},{-0.0791*\dy})
	-- ({1.0654*\dx},{-0.0774*\dy})
	-- ({1.0641*\dx},{-0.0757*\dy})
	-- ({1.0627*\dx},{-0.0740*\dy})
	-- ({1.0615*\dx},{-0.0724*\dy})
	-- ({1.0602*\dx},{-0.0708*\dy})
	-- ({1.0590*\dx},{-0.0692*\dy})
	-- ({1.0577*\dx},{-0.0677*\dy})
}
% v = 0.582119
\def\vpathM{
	({-1.0315*\dx},{-0.0820*\dy})
	-- ({-1.0321*\dx},{-0.0838*\dy})
	-- ({-1.0327*\dx},{-0.0857*\dy})
	-- ({-1.0333*\dx},{-0.0875*\dy})
	-- ({-1.0339*\dx},{-0.0895*\dy})
	-- ({-1.0346*\dx},{-0.0914*\dy})
	-- ({-1.0352*\dx},{-0.0934*\dy})
	-- ({-1.0359*\dx},{-0.0955*\dy})
	-- ({-1.0366*\dx},{-0.0976*\dy})
	-- ({-1.0373*\dx},{-0.0998*\dy})
	-- ({-1.0380*\dx},{-0.1020*\dy})
	-- ({-1.0387*\dx},{-0.1042*\dy})
	-- ({-1.0394*\dx},{-0.1065*\dy})
	-- ({-1.0401*\dx},{-0.1089*\dy})
	-- ({-1.0409*\dx},{-0.1113*\dy})
	-- ({-1.0416*\dx},{-0.1137*\dy})
	-- ({-1.0424*\dx},{-0.1162*\dy})
	-- ({-1.0432*\dx},{-0.1188*\dy})
	-- ({-1.0440*\dx},{-0.1215*\dy})
	-- ({-1.0447*\dx},{-0.1241*\dy})
	-- ({-1.0455*\dx},{-0.1269*\dy})
	-- ({-1.0464*\dx},{-0.1297*\dy})
	-- ({-1.0472*\dx},{-0.1326*\dy})
	-- ({-1.0480*\dx},{-0.1356*\dy})
	-- ({-1.0489*\dx},{-0.1386*\dy})
	-- ({-1.0497*\dx},{-0.1417*\dy})
	-- ({-1.0506*\dx},{-0.1448*\dy})
	-- ({-1.0515*\dx},{-0.1481*\dy})
	-- ({-1.0523*\dx},{-0.1514*\dy})
	-- ({-1.0532*\dx},{-0.1548*\dy})
	-- ({-1.0541*\dx},{-0.1582*\dy})
	-- ({-1.0551*\dx},{-0.1618*\dy})
	-- ({-1.0560*\dx},{-0.1654*\dy})
	-- ({-1.0569*\dx},{-0.1691*\dy})
	-- ({-1.0578*\dx},{-0.1729*\dy})
	-- ({-1.0588*\dx},{-0.1768*\dy})
	-- ({-1.0597*\dx},{-0.1808*\dy})
	-- ({-1.0607*\dx},{-0.1849*\dy})
	-- ({-1.0616*\dx},{-0.1890*\dy})
	-- ({-1.0626*\dx},{-0.1933*\dy})
	-- ({-1.0635*\dx},{-0.1977*\dy})
	-- ({-1.0645*\dx},{-0.2021*\dy})
	-- ({-1.0655*\dx},{-0.2067*\dy})
	-- ({-1.0665*\dx},{-0.2114*\dy})
	-- ({-1.0674*\dx},{-0.2162*\dy})
	-- ({-1.0684*\dx},{-0.2211*\dy})
	-- ({-1.0694*\dx},{-0.2261*\dy})
	-- ({-1.0704*\dx},{-0.2312*\dy})
	-- ({-1.0713*\dx},{-0.2365*\dy})
	-- ({-1.0723*\dx},{-0.2419*\dy})
	-- ({-1.0732*\dx},{-0.2474*\dy})
	-- ({-1.0742*\dx},{-0.2531*\dy})
	-- ({-1.0751*\dx},{-0.2588*\dy})
	-- ({-1.0760*\dx},{-0.2648*\dy})
	-- ({-1.0770*\dx},{-0.2708*\dy})
	-- ({-1.0779*\dx},{-0.2770*\dy})
	-- ({-1.0788*\dx},{-0.2834*\dy})
	-- ({-1.0796*\dx},{-0.2899*\dy})
	-- ({-1.0805*\dx},{-0.2965*\dy})
	-- ({-1.0813*\dx},{-0.3033*\dy})
	-- ({-1.0821*\dx},{-0.3103*\dy})
	-- ({-1.0829*\dx},{-0.3174*\dy})
	-- ({-1.0836*\dx},{-0.3247*\dy})
	-- ({-1.0843*\dx},{-0.3322*\dy})
	-- ({-1.0850*\dx},{-0.3399*\dy})
	-- ({-1.0856*\dx},{-0.3477*\dy})
	-- ({-1.0862*\dx},{-0.3558*\dy})
	-- ({-1.0867*\dx},{-0.3640*\dy})
	-- ({-1.0872*\dx},{-0.3724*\dy})
	-- ({-1.0876*\dx},{-0.3810*\dy})
	-- ({-1.0880*\dx},{-0.3898*\dy})
	-- ({-1.0883*\dx},{-0.3988*\dy})
	-- ({-1.0885*\dx},{-0.4080*\dy})
	-- ({-1.0887*\dx},{-0.4175*\dy})
	-- ({-1.0887*\dx},{-0.4271*\dy})
	-- ({-1.0887*\dx},{-0.4370*\dy})
	-- ({-1.0886*\dx},{-0.4471*\dy})
	-- ({-1.0884*\dx},{-0.4575*\dy})
	-- ({-1.0881*\dx},{-0.4681*\dy})
	-- ({-1.0877*\dx},{-0.4789*\dy})
	-- ({-1.0871*\dx},{-0.4900*\dy})
	-- ({-1.0864*\dx},{-0.5013*\dy})
	-- ({-1.0856*\dx},{-0.5129*\dy})
	-- ({-1.0847*\dx},{-0.5247*\dy})
	-- ({-1.0836*\dx},{-0.5368*\dy})
	-- ({-1.0823*\dx},{-0.5491*\dy})
	-- ({-1.0808*\dx},{-0.5618*\dy})
	-- ({-1.0792*\dx},{-0.5746*\dy})
	-- ({-1.0773*\dx},{-0.5878*\dy})
	-- ({-1.0753*\dx},{-0.6013*\dy})
	-- ({-1.0730*\dx},{-0.6150*\dy})
	-- ({-1.0705*\dx},{-0.6290*\dy})
	-- ({-1.0678*\dx},{-0.6433*\dy})
	-- ({-1.0648*\dx},{-0.6579*\dy})
	-- ({-1.0615*\dx},{-0.6727*\dy})
	-- ({-1.0579*\dx},{-0.6879*\dy})
	-- ({-1.0540*\dx},{-0.7033*\dy})
	-- ({-1.0498*\dx},{-0.7190*\dy})
	-- ({-1.0453*\dx},{-0.7350*\dy})
	-- ({-1.0404*\dx},{-0.7513*\dy})
	-- ({-1.0352*\dx},{-0.7679*\dy})
	-- ({-1.0295*\dx},{-0.7847*\dy})
	-- ({-1.0235*\dx},{-0.8018*\dy})
	-- ({-1.0170*\dx},{-0.8192*\dy})
	-- ({-1.0101*\dx},{-0.8369*\dy})
	-- ({-1.0027*\dx},{-0.8547*\dy})
	-- ({-0.9948*\dx},{-0.8729*\dy})
	-- ({-0.9865*\dx},{-0.8912*\dy})
	-- ({-0.9776*\dx},{-0.9098*\dy})
	-- ({-0.9681*\dx},{-0.9286*\dy})
	-- ({-0.9581*\dx},{-0.9476*\dy})
	-- ({-0.9475*\dx},{-0.9668*\dy})
	-- ({-0.9363*\dx},{-0.9861*\dy})
	-- ({-0.9245*\dx},{-1.0056*\dy})
	-- ({-0.9120*\dx},{-1.0252*\dy})
	-- ({-0.8989*\dx},{-1.0448*\dy})
	-- ({-0.8851*\dx},{-1.0646*\dy})
	-- ({-0.8705*\dx},{-1.0844*\dy})
	-- ({-0.8552*\dx},{-1.1043*\dy})
	-- ({-0.8392*\dx},{-1.1241*\dy})
	-- ({-0.8225*\dx},{-1.1439*\dy})
	-- ({-0.8049*\dx},{-1.1637*\dy})
	-- ({-0.7866*\dx},{-1.1833*\dy})
	-- ({-0.7675*\dx},{-1.2028*\dy})
	-- ({-0.7476*\dx},{-1.2221*\dy})
	-- ({-0.7268*\dx},{-1.2411*\dy})
	-- ({-0.7053*\dx},{-1.2600*\dy})
	-- ({-0.6829*\dx},{-1.2785*\dy})
	-- ({-0.6597*\dx},{-1.2966*\dy})
	-- ({-0.6357*\dx},{-1.3144*\dy})
	-- ({-0.6109*\dx},{-1.3317*\dy})
	-- ({-0.5853*\dx},{-1.3486*\dy})
	-- ({-0.5589*\dx},{-1.3649*\dy})
	-- ({-0.5317*\dx},{-1.3806*\dy})
	-- ({-0.5038*\dx},{-1.3957*\dy})
	-- ({-0.4751*\dx},{-1.4102*\dy})
	-- ({-0.4457*\dx},{-1.4239*\dy})
	-- ({-0.4157*\dx},{-1.4368*\dy})
	-- ({-0.3850*\dx},{-1.4490*\dy})
	-- ({-0.3536*\dx},{-1.4603*\dy})
	-- ({-0.3218*\dx},{-1.4707*\dy})
	-- ({-0.2894*\dx},{-1.4801*\dy})
	-- ({-0.2565*\dx},{-1.4886*\dy})
	-- ({-0.2232*\dx},{-1.4962*\dy})
	-- ({-0.1895*\dx},{-1.5027*\dy})
	-- ({-0.1555*\dx},{-1.5081*\dy})
	-- ({-0.1212*\dx},{-1.5125*\dy})
	-- ({-0.0867*\dx},{-1.5158*\dy})
	-- ({-0.0521*\dx},{-1.5180*\dy})
	-- ({-0.0174*\dx},{-1.5192*\dy})
	-- ({0.0174*\dx},{-1.5192*\dy})
	-- ({0.0521*\dx},{-1.5180*\dy})
	-- ({0.0867*\dx},{-1.5158*\dy})
	-- ({0.1212*\dx},{-1.5125*\dy})
	-- ({0.1555*\dx},{-1.5081*\dy})
	-- ({0.1895*\dx},{-1.5027*\dy})
	-- ({0.2232*\dx},{-1.4962*\dy})
	-- ({0.2565*\dx},{-1.4886*\dy})
	-- ({0.2894*\dx},{-1.4801*\dy})
	-- ({0.3218*\dx},{-1.4707*\dy})
	-- ({0.3536*\dx},{-1.4603*\dy})
	-- ({0.3850*\dx},{-1.4490*\dy})
	-- ({0.4157*\dx},{-1.4368*\dy})
	-- ({0.4457*\dx},{-1.4239*\dy})
	-- ({0.4751*\dx},{-1.4102*\dy})
	-- ({0.5038*\dx},{-1.3957*\dy})
	-- ({0.5317*\dx},{-1.3806*\dy})
	-- ({0.5589*\dx},{-1.3649*\dy})
	-- ({0.5853*\dx},{-1.3486*\dy})
	-- ({0.6109*\dx},{-1.3317*\dy})
	-- ({0.6357*\dx},{-1.3144*\dy})
	-- ({0.6597*\dx},{-1.2966*\dy})
	-- ({0.6829*\dx},{-1.2785*\dy})
	-- ({0.7053*\dx},{-1.2600*\dy})
	-- ({0.7268*\dx},{-1.2411*\dy})
	-- ({0.7476*\dx},{-1.2221*\dy})
	-- ({0.7675*\dx},{-1.2028*\dy})
	-- ({0.7866*\dx},{-1.1833*\dy})
	-- ({0.8049*\dx},{-1.1637*\dy})
	-- ({0.8225*\dx},{-1.1439*\dy})
	-- ({0.8392*\dx},{-1.1241*\dy})
	-- ({0.8552*\dx},{-1.1043*\dy})
	-- ({0.8705*\dx},{-1.0844*\dy})
	-- ({0.8851*\dx},{-1.0646*\dy})
	-- ({0.8989*\dx},{-1.0448*\dy})
	-- ({0.9120*\dx},{-1.0252*\dy})
	-- ({0.9245*\dx},{-1.0056*\dy})
	-- ({0.9363*\dx},{-0.9861*\dy})
	-- ({0.9475*\dx},{-0.9668*\dy})
	-- ({0.9581*\dx},{-0.9476*\dy})
	-- ({0.9681*\dx},{-0.9286*\dy})
	-- ({0.9776*\dx},{-0.9098*\dy})
	-- ({0.9865*\dx},{-0.8912*\dy})
	-- ({0.9948*\dx},{-0.8729*\dy})
	-- ({1.0027*\dx},{-0.8547*\dy})
	-- ({1.0101*\dx},{-0.8369*\dy})
	-- ({1.0170*\dx},{-0.8192*\dy})
	-- ({1.0235*\dx},{-0.8018*\dy})
	-- ({1.0295*\dx},{-0.7847*\dy})
	-- ({1.0352*\dx},{-0.7679*\dy})
	-- ({1.0404*\dx},{-0.7513*\dy})
	-- ({1.0453*\dx},{-0.7350*\dy})
	-- ({1.0498*\dx},{-0.7190*\dy})
	-- ({1.0540*\dx},{-0.7033*\dy})
	-- ({1.0579*\dx},{-0.6879*\dy})
	-- ({1.0615*\dx},{-0.6727*\dy})
	-- ({1.0648*\dx},{-0.6579*\dy})
	-- ({1.0678*\dx},{-0.6433*\dy})
	-- ({1.0705*\dx},{-0.6290*\dy})
	-- ({1.0730*\dx},{-0.6150*\dy})
	-- ({1.0753*\dx},{-0.6013*\dy})
	-- ({1.0773*\dx},{-0.5878*\dy})
	-- ({1.0792*\dx},{-0.5746*\dy})
	-- ({1.0808*\dx},{-0.5618*\dy})
	-- ({1.0823*\dx},{-0.5491*\dy})
	-- ({1.0836*\dx},{-0.5368*\dy})
	-- ({1.0847*\dx},{-0.5247*\dy})
	-- ({1.0856*\dx},{-0.5129*\dy})
	-- ({1.0864*\dx},{-0.5013*\dy})
	-- ({1.0871*\dx},{-0.4900*\dy})
	-- ({1.0877*\dx},{-0.4789*\dy})
	-- ({1.0881*\dx},{-0.4681*\dy})
	-- ({1.0884*\dx},{-0.4575*\dy})
	-- ({1.0886*\dx},{-0.4471*\dy})
	-- ({1.0887*\dx},{-0.4370*\dy})
	-- ({1.0887*\dx},{-0.4271*\dy})
	-- ({1.0887*\dx},{-0.4175*\dy})
	-- ({1.0885*\dx},{-0.4080*\dy})
	-- ({1.0883*\dx},{-0.3988*\dy})
	-- ({1.0880*\dx},{-0.3898*\dy})
	-- ({1.0876*\dx},{-0.3810*\dy})
	-- ({1.0872*\dx},{-0.3724*\dy})
	-- ({1.0867*\dx},{-0.3640*\dy})
	-- ({1.0862*\dx},{-0.3558*\dy})
	-- ({1.0856*\dx},{-0.3477*\dy})
	-- ({1.0850*\dx},{-0.3399*\dy})
	-- ({1.0843*\dx},{-0.3322*\dy})
	-- ({1.0836*\dx},{-0.3247*\dy})
	-- ({1.0829*\dx},{-0.3174*\dy})
	-- ({1.0821*\dx},{-0.3103*\dy})
	-- ({1.0813*\dx},{-0.3033*\dy})
	-- ({1.0805*\dx},{-0.2965*\dy})
	-- ({1.0796*\dx},{-0.2899*\dy})
	-- ({1.0788*\dx},{-0.2834*\dy})
	-- ({1.0779*\dx},{-0.2770*\dy})
	-- ({1.0770*\dx},{-0.2708*\dy})
	-- ({1.0760*\dx},{-0.2648*\dy})
	-- ({1.0751*\dx},{-0.2588*\dy})
	-- ({1.0742*\dx},{-0.2531*\dy})
	-- ({1.0732*\dx},{-0.2474*\dy})
	-- ({1.0723*\dx},{-0.2419*\dy})
	-- ({1.0713*\dx},{-0.2365*\dy})
	-- ({1.0704*\dx},{-0.2312*\dy})
	-- ({1.0694*\dx},{-0.2261*\dy})
	-- ({1.0684*\dx},{-0.2211*\dy})
	-- ({1.0674*\dx},{-0.2162*\dy})
	-- ({1.0665*\dx},{-0.2114*\dy})
	-- ({1.0655*\dx},{-0.2067*\dy})
	-- ({1.0645*\dx},{-0.2021*\dy})
	-- ({1.0635*\dx},{-0.1977*\dy})
	-- ({1.0626*\dx},{-0.1933*\dy})
	-- ({1.0616*\dx},{-0.1890*\dy})
	-- ({1.0607*\dx},{-0.1849*\dy})
	-- ({1.0597*\dx},{-0.1808*\dy})
	-- ({1.0588*\dx},{-0.1768*\dy})
	-- ({1.0578*\dx},{-0.1729*\dy})
	-- ({1.0569*\dx},{-0.1691*\dy})
	-- ({1.0560*\dx},{-0.1654*\dy})
	-- ({1.0551*\dx},{-0.1618*\dy})
	-- ({1.0541*\dx},{-0.1582*\dy})
	-- ({1.0532*\dx},{-0.1548*\dy})
	-- ({1.0523*\dx},{-0.1514*\dy})
	-- ({1.0515*\dx},{-0.1481*\dy})
	-- ({1.0506*\dx},{-0.1448*\dy})
	-- ({1.0497*\dx},{-0.1417*\dy})
	-- ({1.0489*\dx},{-0.1386*\dy})
	-- ({1.0480*\dx},{-0.1356*\dy})
	-- ({1.0472*\dx},{-0.1326*\dy})
	-- ({1.0464*\dx},{-0.1297*\dy})
	-- ({1.0455*\dx},{-0.1269*\dy})
	-- ({1.0447*\dx},{-0.1241*\dy})
	-- ({1.0440*\dx},{-0.1215*\dy})
	-- ({1.0432*\dx},{-0.1188*\dy})
	-- ({1.0424*\dx},{-0.1162*\dy})
	-- ({1.0416*\dx},{-0.1137*\dy})
	-- ({1.0409*\dx},{-0.1113*\dy})
	-- ({1.0401*\dx},{-0.1089*\dy})
	-- ({1.0394*\dx},{-0.1065*\dy})
	-- ({1.0387*\dx},{-0.1042*\dy})
	-- ({1.0380*\dx},{-0.1020*\dy})
	-- ({1.0373*\dx},{-0.0998*\dy})
	-- ({1.0366*\dx},{-0.0976*\dy})
	-- ({1.0359*\dx},{-0.0955*\dy})
	-- ({1.0352*\dx},{-0.0934*\dy})
	-- ({1.0346*\dx},{-0.0914*\dy})
	-- ({1.0339*\dx},{-0.0895*\dy})
	-- ({1.0333*\dx},{-0.0875*\dy})
	-- ({1.0327*\dx},{-0.0857*\dy})
	-- ({1.0321*\dx},{-0.0838*\dy})
	-- ({1.0315*\dx},{-0.0820*\dy})
}
% v = 0.748438
\def\vpathN{
	({-1.0027*\dx},{-0.0866*\dy})
	-- ({-1.0026*\dx},{-0.0884*\dy})
	-- ({-1.0026*\dx},{-0.0903*\dy})
	-- ({-1.0026*\dx},{-0.0922*\dy})
	-- ({-1.0025*\dx},{-0.0942*\dy})
	-- ({-1.0025*\dx},{-0.0962*\dy})
	-- ({-1.0024*\dx},{-0.0982*\dy})
	-- ({-1.0024*\dx},{-0.1003*\dy})
	-- ({-1.0023*\dx},{-0.1025*\dy})
	-- ({-1.0023*\dx},{-0.1047*\dy})
	-- ({-1.0022*\dx},{-0.1069*\dy})
	-- ({-1.0021*\dx},{-0.1092*\dy})
	-- ({-1.0020*\dx},{-0.1115*\dy})
	-- ({-1.0019*\dx},{-0.1139*\dy})
	-- ({-1.0018*\dx},{-0.1163*\dy})
	-- ({-1.0017*\dx},{-0.1188*\dy})
	-- ({-1.0016*\dx},{-0.1213*\dy})
	-- ({-1.0015*\dx},{-0.1239*\dy})
	-- ({-1.0014*\dx},{-0.1265*\dy})
	-- ({-1.0012*\dx},{-0.1292*\dy})
	-- ({-1.0011*\dx},{-0.1319*\dy})
	-- ({-1.0009*\dx},{-0.1347*\dy})
	-- ({-1.0007*\dx},{-0.1376*\dy})
	-- ({-1.0005*\dx},{-0.1405*\dy})
	-- ({-1.0003*\dx},{-0.1435*\dy})
	-- ({-1.0001*\dx},{-0.1466*\dy})
	-- ({-0.9999*\dx},{-0.1497*\dy})
	-- ({-0.9996*\dx},{-0.1528*\dy})
	-- ({-0.9994*\dx},{-0.1561*\dy})
	-- ({-0.9991*\dx},{-0.1594*\dy})
	-- ({-0.9988*\dx},{-0.1628*\dy})
	-- ({-0.9985*\dx},{-0.1662*\dy})
	-- ({-0.9982*\dx},{-0.1698*\dy})
	-- ({-0.9978*\dx},{-0.1734*\dy})
	-- ({-0.9974*\dx},{-0.1770*\dy})
	-- ({-0.9970*\dx},{-0.1808*\dy})
	-- ({-0.9966*\dx},{-0.1846*\dy})
	-- ({-0.9962*\dx},{-0.1885*\dy})
	-- ({-0.9957*\dx},{-0.1925*\dy})
	-- ({-0.9952*\dx},{-0.1966*\dy})
	-- ({-0.9947*\dx},{-0.2007*\dy})
	-- ({-0.9942*\dx},{-0.2050*\dy})
	-- ({-0.9936*\dx},{-0.2093*\dy})
	-- ({-0.9930*\dx},{-0.2137*\dy})
	-- ({-0.9923*\dx},{-0.2182*\dy})
	-- ({-0.9916*\dx},{-0.2228*\dy})
	-- ({-0.9909*\dx},{-0.2275*\dy})
	-- ({-0.9902*\dx},{-0.2323*\dy})
	-- ({-0.9894*\dx},{-0.2372*\dy})
	-- ({-0.9886*\dx},{-0.2421*\dy})
	-- ({-0.9877*\dx},{-0.2472*\dy})
	-- ({-0.9868*\dx},{-0.2524*\dy})
	-- ({-0.9858*\dx},{-0.2577*\dy})
	-- ({-0.9848*\dx},{-0.2631*\dy})
	-- ({-0.9837*\dx},{-0.2686*\dy})
	-- ({-0.9826*\dx},{-0.2742*\dy})
	-- ({-0.9814*\dx},{-0.2799*\dy})
	-- ({-0.9801*\dx},{-0.2857*\dy})
	-- ({-0.9788*\dx},{-0.2917*\dy})
	-- ({-0.9775*\dx},{-0.2977*\dy})
	-- ({-0.9760*\dx},{-0.3039*\dy})
	-- ({-0.9745*\dx},{-0.3102*\dy})
	-- ({-0.9730*\dx},{-0.3166*\dy})
	-- ({-0.9713*\dx},{-0.3231*\dy})
	-- ({-0.9696*\dx},{-0.3298*\dy})
	-- ({-0.9678*\dx},{-0.3366*\dy})
	-- ({-0.9659*\dx},{-0.3435*\dy})
	-- ({-0.9639*\dx},{-0.3505*\dy})
	-- ({-0.9618*\dx},{-0.3577*\dy})
	-- ({-0.9596*\dx},{-0.3650*\dy})
	-- ({-0.9573*\dx},{-0.3724*\dy})
	-- ({-0.9549*\dx},{-0.3800*\dy})
	-- ({-0.9524*\dx},{-0.3877*\dy})
	-- ({-0.9498*\dx},{-0.3955*\dy})
	-- ({-0.9471*\dx},{-0.4034*\dy})
	-- ({-0.9442*\dx},{-0.4115*\dy})
	-- ({-0.9413*\dx},{-0.4198*\dy})
	-- ({-0.9381*\dx},{-0.4281*\dy})
	-- ({-0.9349*\dx},{-0.4367*\dy})
	-- ({-0.9315*\dx},{-0.4453*\dy})
	-- ({-0.9279*\dx},{-0.4541*\dy})
	-- ({-0.9242*\dx},{-0.4630*\dy})
	-- ({-0.9204*\dx},{-0.4721*\dy})
	-- ({-0.9163*\dx},{-0.4813*\dy})
	-- ({-0.9121*\dx},{-0.4906*\dy})
	-- ({-0.9077*\dx},{-0.5001*\dy})
	-- ({-0.9032*\dx},{-0.5097*\dy})
	-- ({-0.8984*\dx},{-0.5194*\dy})
	-- ({-0.8934*\dx},{-0.5293*\dy})
	-- ({-0.8883*\dx},{-0.5393*\dy})
	-- ({-0.8829*\dx},{-0.5494*\dy})
	-- ({-0.8773*\dx},{-0.5597*\dy})
	-- ({-0.8715*\dx},{-0.5701*\dy})
	-- ({-0.8654*\dx},{-0.5806*\dy})
	-- ({-0.8591*\dx},{-0.5912*\dy})
	-- ({-0.8526*\dx},{-0.6019*\dy})
	-- ({-0.8458*\dx},{-0.6127*\dy})
	-- ({-0.8387*\dx},{-0.6237*\dy})
	-- ({-0.8313*\dx},{-0.6347*\dy})
	-- ({-0.8237*\dx},{-0.6459*\dy})
	-- ({-0.8158*\dx},{-0.6571*\dy})
	-- ({-0.8076*\dx},{-0.6684*\dy})
	-- ({-0.7991*\dx},{-0.6798*\dy})
	-- ({-0.7903*\dx},{-0.6912*\dy})
	-- ({-0.7812*\dx},{-0.7027*\dy})
	-- ({-0.7717*\dx},{-0.7143*\dy})
	-- ({-0.7620*\dx},{-0.7259*\dy})
	-- ({-0.7519*\dx},{-0.7375*\dy})
	-- ({-0.7414*\dx},{-0.7492*\dy})
	-- ({-0.7306*\dx},{-0.7609*\dy})
	-- ({-0.7194*\dx},{-0.7725*\dy})
	-- ({-0.7079*\dx},{-0.7842*\dy})
	-- ({-0.6960*\dx},{-0.7959*\dy})
	-- ({-0.6838*\dx},{-0.8075*\dy})
	-- ({-0.6711*\dx},{-0.8191*\dy})
	-- ({-0.6581*\dx},{-0.8306*\dy})
	-- ({-0.6447*\dx},{-0.8420*\dy})
	-- ({-0.6309*\dx},{-0.8534*\dy})
	-- ({-0.6168*\dx},{-0.8647*\dy})
	-- ({-0.6022*\dx},{-0.8758*\dy})
	-- ({-0.5873*\dx},{-0.8868*\dy})
	-- ({-0.5719*\dx},{-0.8977*\dy})
	-- ({-0.5562*\dx},{-0.9084*\dy})
	-- ({-0.5401*\dx},{-0.9189*\dy})
	-- ({-0.5236*\dx},{-0.9293*\dy})
	-- ({-0.5067*\dx},{-0.9394*\dy})
	-- ({-0.4894*\dx},{-0.9493*\dy})
	-- ({-0.4717*\dx},{-0.9589*\dy})
	-- ({-0.4537*\dx},{-0.9683*\dy})
	-- ({-0.4354*\dx},{-0.9774*\dy})
	-- ({-0.4166*\dx},{-0.9861*\dy})
	-- ({-0.3976*\dx},{-0.9946*\dy})
	-- ({-0.3782*\dx},{-1.0028*\dy})
	-- ({-0.3584*\dx},{-1.0105*\dy})
	-- ({-0.3384*\dx},{-1.0180*\dy})
	-- ({-0.3180*\dx},{-1.0250*\dy})
	-- ({-0.2974*\dx},{-1.0317*\dy})
	-- ({-0.2765*\dx},{-1.0379*\dy})
	-- ({-0.2554*\dx},{-1.0437*\dy})
	-- ({-0.2340*\dx},{-1.0491*\dy})
	-- ({-0.2124*\dx},{-1.0540*\dy})
	-- ({-0.1906*\dx},{-1.0585*\dy})
	-- ({-0.1686*\dx},{-1.0625*\dy})
	-- ({-0.1465*\dx},{-1.0660*\dy})
	-- ({-0.1242*\dx},{-1.0691*\dy})
	-- ({-0.1018*\dx},{-1.0716*\dy})
	-- ({-0.0793*\dx},{-1.0737*\dy})
	-- ({-0.0567*\dx},{-1.0752*\dy})
	-- ({-0.0340*\dx},{-1.0762*\dy})
	-- ({-0.0113*\dx},{-1.0767*\dy})
	-- ({0.0113*\dx},{-1.0767*\dy})
	-- ({0.0340*\dx},{-1.0762*\dy})
	-- ({0.0567*\dx},{-1.0752*\dy})
	-- ({0.0793*\dx},{-1.0737*\dy})
	-- ({0.1018*\dx},{-1.0716*\dy})
	-- ({0.1242*\dx},{-1.0691*\dy})
	-- ({0.1465*\dx},{-1.0660*\dy})
	-- ({0.1686*\dx},{-1.0625*\dy})
	-- ({0.1906*\dx},{-1.0585*\dy})
	-- ({0.2124*\dx},{-1.0540*\dy})
	-- ({0.2340*\dx},{-1.0491*\dy})
	-- ({0.2554*\dx},{-1.0437*\dy})
	-- ({0.2765*\dx},{-1.0379*\dy})
	-- ({0.2974*\dx},{-1.0317*\dy})
	-- ({0.3180*\dx},{-1.0250*\dy})
	-- ({0.3384*\dx},{-1.0180*\dy})
	-- ({0.3584*\dx},{-1.0105*\dy})
	-- ({0.3782*\dx},{-1.0028*\dy})
	-- ({0.3976*\dx},{-0.9946*\dy})
	-- ({0.4166*\dx},{-0.9861*\dy})
	-- ({0.4354*\dx},{-0.9774*\dy})
	-- ({0.4537*\dx},{-0.9683*\dy})
	-- ({0.4717*\dx},{-0.9589*\dy})
	-- ({0.4894*\dx},{-0.9493*\dy})
	-- ({0.5067*\dx},{-0.9394*\dy})
	-- ({0.5236*\dx},{-0.9293*\dy})
	-- ({0.5401*\dx},{-0.9189*\dy})
	-- ({0.5562*\dx},{-0.9084*\dy})
	-- ({0.5719*\dx},{-0.8977*\dy})
	-- ({0.5873*\dx},{-0.8868*\dy})
	-- ({0.6022*\dx},{-0.8758*\dy})
	-- ({0.6168*\dx},{-0.8647*\dy})
	-- ({0.6309*\dx},{-0.8534*\dy})
	-- ({0.6447*\dx},{-0.8420*\dy})
	-- ({0.6581*\dx},{-0.8306*\dy})
	-- ({0.6711*\dx},{-0.8191*\dy})
	-- ({0.6838*\dx},{-0.8075*\dy})
	-- ({0.6960*\dx},{-0.7959*\dy})
	-- ({0.7079*\dx},{-0.7842*\dy})
	-- ({0.7194*\dx},{-0.7725*\dy})
	-- ({0.7306*\dx},{-0.7609*\dy})
	-- ({0.7414*\dx},{-0.7492*\dy})
	-- ({0.7519*\dx},{-0.7375*\dy})
	-- ({0.7620*\dx},{-0.7259*\dy})
	-- ({0.7717*\dx},{-0.7143*\dy})
	-- ({0.7812*\dx},{-0.7027*\dy})
	-- ({0.7903*\dx},{-0.6912*\dy})
	-- ({0.7991*\dx},{-0.6798*\dy})
	-- ({0.8076*\dx},{-0.6684*\dy})
	-- ({0.8158*\dx},{-0.6571*\dy})
	-- ({0.8237*\dx},{-0.6459*\dy})
	-- ({0.8313*\dx},{-0.6347*\dy})
	-- ({0.8387*\dx},{-0.6237*\dy})
	-- ({0.8458*\dx},{-0.6127*\dy})
	-- ({0.8526*\dx},{-0.6019*\dy})
	-- ({0.8591*\dx},{-0.5912*\dy})
	-- ({0.8654*\dx},{-0.5806*\dy})
	-- ({0.8715*\dx},{-0.5701*\dy})
	-- ({0.8773*\dx},{-0.5597*\dy})
	-- ({0.8829*\dx},{-0.5494*\dy})
	-- ({0.8883*\dx},{-0.5393*\dy})
	-- ({0.8934*\dx},{-0.5293*\dy})
	-- ({0.8984*\dx},{-0.5194*\dy})
	-- ({0.9032*\dx},{-0.5097*\dy})
	-- ({0.9077*\dx},{-0.5001*\dy})
	-- ({0.9121*\dx},{-0.4906*\dy})
	-- ({0.9163*\dx},{-0.4813*\dy})
	-- ({0.9204*\dx},{-0.4721*\dy})
	-- ({0.9242*\dx},{-0.4630*\dy})
	-- ({0.9279*\dx},{-0.4541*\dy})
	-- ({0.9315*\dx},{-0.4453*\dy})
	-- ({0.9349*\dx},{-0.4367*\dy})
	-- ({0.9381*\dx},{-0.4281*\dy})
	-- ({0.9413*\dx},{-0.4198*\dy})
	-- ({0.9442*\dx},{-0.4115*\dy})
	-- ({0.9471*\dx},{-0.4034*\dy})
	-- ({0.9498*\dx},{-0.3955*\dy})
	-- ({0.9524*\dx},{-0.3877*\dy})
	-- ({0.9549*\dx},{-0.3800*\dy})
	-- ({0.9573*\dx},{-0.3724*\dy})
	-- ({0.9596*\dx},{-0.3650*\dy})
	-- ({0.9618*\dx},{-0.3577*\dy})
	-- ({0.9639*\dx},{-0.3505*\dy})
	-- ({0.9659*\dx},{-0.3435*\dy})
	-- ({0.9678*\dx},{-0.3366*\dy})
	-- ({0.9696*\dx},{-0.3298*\dy})
	-- ({0.9713*\dx},{-0.3231*\dy})
	-- ({0.9730*\dx},{-0.3166*\dy})
	-- ({0.9745*\dx},{-0.3102*\dy})
	-- ({0.9760*\dx},{-0.3039*\dy})
	-- ({0.9775*\dx},{-0.2977*\dy})
	-- ({0.9788*\dx},{-0.2917*\dy})
	-- ({0.9801*\dx},{-0.2857*\dy})
	-- ({0.9814*\dx},{-0.2799*\dy})
	-- ({0.9826*\dx},{-0.2742*\dy})
	-- ({0.9837*\dx},{-0.2686*\dy})
	-- ({0.9848*\dx},{-0.2631*\dy})
	-- ({0.9858*\dx},{-0.2577*\dy})
	-- ({0.9868*\dx},{-0.2524*\dy})
	-- ({0.9877*\dx},{-0.2472*\dy})
	-- ({0.9886*\dx},{-0.2421*\dy})
	-- ({0.9894*\dx},{-0.2372*\dy})
	-- ({0.9902*\dx},{-0.2323*\dy})
	-- ({0.9909*\dx},{-0.2275*\dy})
	-- ({0.9916*\dx},{-0.2228*\dy})
	-- ({0.9923*\dx},{-0.2182*\dy})
	-- ({0.9930*\dx},{-0.2137*\dy})
	-- ({0.9936*\dx},{-0.2093*\dy})
	-- ({0.9942*\dx},{-0.2050*\dy})
	-- ({0.9947*\dx},{-0.2007*\dy})
	-- ({0.9952*\dx},{-0.1966*\dy})
	-- ({0.9957*\dx},{-0.1925*\dy})
	-- ({0.9962*\dx},{-0.1885*\dy})
	-- ({0.9966*\dx},{-0.1846*\dy})
	-- ({0.9970*\dx},{-0.1808*\dy})
	-- ({0.9974*\dx},{-0.1770*\dy})
	-- ({0.9978*\dx},{-0.1734*\dy})
	-- ({0.9982*\dx},{-0.1698*\dy})
	-- ({0.9985*\dx},{-0.1662*\dy})
	-- ({0.9988*\dx},{-0.1628*\dy})
	-- ({0.9991*\dx},{-0.1594*\dy})
	-- ({0.9994*\dx},{-0.1561*\dy})
	-- ({0.9996*\dx},{-0.1528*\dy})
	-- ({0.9999*\dx},{-0.1497*\dy})
	-- ({1.0001*\dx},{-0.1466*\dy})
	-- ({1.0003*\dx},{-0.1435*\dy})
	-- ({1.0005*\dx},{-0.1405*\dy})
	-- ({1.0007*\dx},{-0.1376*\dy})
	-- ({1.0009*\dx},{-0.1347*\dy})
	-- ({1.0011*\dx},{-0.1319*\dy})
	-- ({1.0012*\dx},{-0.1292*\dy})
	-- ({1.0014*\dx},{-0.1265*\dy})
	-- ({1.0015*\dx},{-0.1239*\dy})
	-- ({1.0016*\dx},{-0.1213*\dy})
	-- ({1.0017*\dx},{-0.1188*\dy})
	-- ({1.0018*\dx},{-0.1163*\dy})
	-- ({1.0019*\dx},{-0.1139*\dy})
	-- ({1.0020*\dx},{-0.1115*\dy})
	-- ({1.0021*\dx},{-0.1092*\dy})
	-- ({1.0022*\dx},{-0.1069*\dy})
	-- ({1.0023*\dx},{-0.1047*\dy})
	-- ({1.0023*\dx},{-0.1025*\dy})
	-- ({1.0024*\dx},{-0.1003*\dy})
	-- ({1.0024*\dx},{-0.0982*\dy})
	-- ({1.0025*\dx},{-0.0962*\dy})
	-- ({1.0025*\dx},{-0.0942*\dy})
	-- ({1.0026*\dx},{-0.0922*\dy})
	-- ({1.0026*\dx},{-0.0903*\dy})
	-- ({1.0026*\dx},{-0.0884*\dy})
	-- ({1.0027*\dx},{-0.0866*\dy})
}
% v = 0.914758
\def\vpathO{
	({-0.9748*\dx},{-0.0816*\dy})
	-- ({-0.9742*\dx},{-0.0833*\dy})
	-- ({-0.9735*\dx},{-0.0850*\dy})
	-- ({-0.9729*\dx},{-0.0868*\dy})
	-- ({-0.9723*\dx},{-0.0886*\dy})
	-- ({-0.9716*\dx},{-0.0904*\dy})
	-- ({-0.9709*\dx},{-0.0922*\dy})
	-- ({-0.9702*\dx},{-0.0941*\dy})
	-- ({-0.9695*\dx},{-0.0961*\dy})
	-- ({-0.9688*\dx},{-0.0981*\dy})
	-- ({-0.9680*\dx},{-0.1001*\dy})
	-- ({-0.9672*\dx},{-0.1021*\dy})
	-- ({-0.9664*\dx},{-0.1042*\dy})
	-- ({-0.9656*\dx},{-0.1064*\dy})
	-- ({-0.9648*\dx},{-0.1085*\dy})
	-- ({-0.9639*\dx},{-0.1108*\dy})
	-- ({-0.9630*\dx},{-0.1130*\dy})
	-- ({-0.9621*\dx},{-0.1153*\dy})
	-- ({-0.9612*\dx},{-0.1177*\dy})
	-- ({-0.9602*\dx},{-0.1201*\dy})
	-- ({-0.9592*\dx},{-0.1225*\dy})
	-- ({-0.9582*\dx},{-0.1250*\dy})
	-- ({-0.9572*\dx},{-0.1276*\dy})
	-- ({-0.9561*\dx},{-0.1302*\dy})
	-- ({-0.9550*\dx},{-0.1328*\dy})
	-- ({-0.9539*\dx},{-0.1355*\dy})
	-- ({-0.9527*\dx},{-0.1382*\dy})
	-- ({-0.9516*\dx},{-0.1410*\dy})
	-- ({-0.9503*\dx},{-0.1439*\dy})
	-- ({-0.9491*\dx},{-0.1468*\dy})
	-- ({-0.9478*\dx},{-0.1497*\dy})
	-- ({-0.9465*\dx},{-0.1527*\dy})
	-- ({-0.9451*\dx},{-0.1558*\dy})
	-- ({-0.9437*\dx},{-0.1589*\dy})
	-- ({-0.9423*\dx},{-0.1621*\dy})
	-- ({-0.9408*\dx},{-0.1654*\dy})
	-- ({-0.9393*\dx},{-0.1687*\dy})
	-- ({-0.9377*\dx},{-0.1720*\dy})
	-- ({-0.9361*\dx},{-0.1754*\dy})
	-- ({-0.9345*\dx},{-0.1789*\dy})
	-- ({-0.9328*\dx},{-0.1825*\dy})
	-- ({-0.9311*\dx},{-0.1861*\dy})
	-- ({-0.9293*\dx},{-0.1897*\dy})
	-- ({-0.9274*\dx},{-0.1935*\dy})
	-- ({-0.9256*\dx},{-0.1973*\dy})
	-- ({-0.9236*\dx},{-0.2012*\dy})
	-- ({-0.9216*\dx},{-0.2051*\dy})
	-- ({-0.9196*\dx},{-0.2091*\dy})
	-- ({-0.9175*\dx},{-0.2132*\dy})
	-- ({-0.9153*\dx},{-0.2173*\dy})
	-- ({-0.9131*\dx},{-0.2215*\dy})
	-- ({-0.9108*\dx},{-0.2258*\dy})
	-- ({-0.9084*\dx},{-0.2302*\dy})
	-- ({-0.9060*\dx},{-0.2346*\dy})
	-- ({-0.9035*\dx},{-0.2391*\dy})
	-- ({-0.9009*\dx},{-0.2437*\dy})
	-- ({-0.8983*\dx},{-0.2483*\dy})
	-- ({-0.8955*\dx},{-0.2531*\dy})
	-- ({-0.8928*\dx},{-0.2579*\dy})
	-- ({-0.8899*\dx},{-0.2627*\dy})
	-- ({-0.8869*\dx},{-0.2677*\dy})
	-- ({-0.8839*\dx},{-0.2727*\dy})
	-- ({-0.8808*\dx},{-0.2778*\dy})
	-- ({-0.8776*\dx},{-0.2830*\dy})
	-- ({-0.8743*\dx},{-0.2883*\dy})
	-- ({-0.8709*\dx},{-0.2936*\dy})
	-- ({-0.8674*\dx},{-0.2990*\dy})
	-- ({-0.8638*\dx},{-0.3045*\dy})
	-- ({-0.8601*\dx},{-0.3101*\dy})
	-- ({-0.8563*\dx},{-0.3157*\dy})
	-- ({-0.8524*\dx},{-0.3214*\dy})
	-- ({-0.8484*\dx},{-0.3272*\dy})
	-- ({-0.8442*\dx},{-0.3331*\dy})
	-- ({-0.8400*\dx},{-0.3390*\dy})
	-- ({-0.8356*\dx},{-0.3451*\dy})
	-- ({-0.8312*\dx},{-0.3512*\dy})
	-- ({-0.8266*\dx},{-0.3573*\dy})
	-- ({-0.8218*\dx},{-0.3636*\dy})
	-- ({-0.8170*\dx},{-0.3699*\dy})
	-- ({-0.8120*\dx},{-0.3763*\dy})
	-- ({-0.8068*\dx},{-0.3827*\dy})
	-- ({-0.8015*\dx},{-0.3892*\dy})
	-- ({-0.7961*\dx},{-0.3958*\dy})
	-- ({-0.7905*\dx},{-0.4025*\dy})
	-- ({-0.7848*\dx},{-0.4092*\dy})
	-- ({-0.7790*\dx},{-0.4160*\dy})
	-- ({-0.7729*\dx},{-0.4228*\dy})
	-- ({-0.7667*\dx},{-0.4297*\dy})
	-- ({-0.7604*\dx},{-0.4367*\dy})
	-- ({-0.7539*\dx},{-0.4437*\dy})
	-- ({-0.7472*\dx},{-0.4507*\dy})
	-- ({-0.7403*\dx},{-0.4578*\dy})
	-- ({-0.7333*\dx},{-0.4650*\dy})
	-- ({-0.7261*\dx},{-0.4722*\dy})
	-- ({-0.7187*\dx},{-0.4794*\dy})
	-- ({-0.7111*\dx},{-0.4866*\dy})
	-- ({-0.7033*\dx},{-0.4939*\dy})
	-- ({-0.6953*\dx},{-0.5012*\dy})
	-- ({-0.6872*\dx},{-0.5086*\dy})
	-- ({-0.6788*\dx},{-0.5159*\dy})
	-- ({-0.6702*\dx},{-0.5233*\dy})
	-- ({-0.6615*\dx},{-0.5307*\dy})
	-- ({-0.6525*\dx},{-0.5380*\dy})
	-- ({-0.6433*\dx},{-0.5454*\dy})
	-- ({-0.6339*\dx},{-0.5528*\dy})
	-- ({-0.6243*\dx},{-0.5601*\dy})
	-- ({-0.6145*\dx},{-0.5675*\dy})
	-- ({-0.6045*\dx},{-0.5748*\dy})
	-- ({-0.5942*\dx},{-0.5821*\dy})
	-- ({-0.5838*\dx},{-0.5893*\dy})
	-- ({-0.5731*\dx},{-0.5965*\dy})
	-- ({-0.5622*\dx},{-0.6037*\dy})
	-- ({-0.5510*\dx},{-0.6108*\dy})
	-- ({-0.5397*\dx},{-0.6178*\dy})
	-- ({-0.5281*\dx},{-0.6248*\dy})
	-- ({-0.5163*\dx},{-0.6317*\dy})
	-- ({-0.5043*\dx},{-0.6385*\dy})
	-- ({-0.4921*\dx},{-0.6452*\dy})
	-- ({-0.4797*\dx},{-0.6518*\dy})
	-- ({-0.4670*\dx},{-0.6584*\dy})
	-- ({-0.4541*\dx},{-0.6648*\dy})
	-- ({-0.4410*\dx},{-0.6710*\dy})
	-- ({-0.4277*\dx},{-0.6772*\dy})
	-- ({-0.4142*\dx},{-0.6832*\dy})
	-- ({-0.4005*\dx},{-0.6891*\dy})
	-- ({-0.3866*\dx},{-0.6948*\dy})
	-- ({-0.3725*\dx},{-0.7004*\dy})
	-- ({-0.3582*\dx},{-0.7058*\dy})
	-- ({-0.3437*\dx},{-0.7110*\dy})
	-- ({-0.3290*\dx},{-0.7160*\dy})
	-- ({-0.3142*\dx},{-0.7209*\dy})
	-- ({-0.2992*\dx},{-0.7256*\dy})
	-- ({-0.2840*\dx},{-0.7300*\dy})
	-- ({-0.2687*\dx},{-0.7343*\dy})
	-- ({-0.2532*\dx},{-0.7383*\dy})
	-- ({-0.2376*\dx},{-0.7421*\dy})
	-- ({-0.2218*\dx},{-0.7457*\dy})
	-- ({-0.2059*\dx},{-0.7491*\dy})
	-- ({-0.1899*\dx},{-0.7522*\dy})
	-- ({-0.1737*\dx},{-0.7551*\dy})
	-- ({-0.1575*\dx},{-0.7577*\dy})
	-- ({-0.1412*\dx},{-0.7601*\dy})
	-- ({-0.1248*\dx},{-0.7622*\dy})
	-- ({-0.1083*\dx},{-0.7641*\dy})
	-- ({-0.0917*\dx},{-0.7657*\dy})
	-- ({-0.0751*\dx},{-0.7670*\dy})
	-- ({-0.0585*\dx},{-0.7681*\dy})
	-- ({-0.0418*\dx},{-0.7689*\dy})
	-- ({-0.0251*\dx},{-0.7695*\dy})
	-- ({-0.0084*\dx},{-0.7697*\dy})
	-- ({0.0084*\dx},{-0.7697*\dy})
	-- ({0.0251*\dx},{-0.7695*\dy})
	-- ({0.0418*\dx},{-0.7689*\dy})
	-- ({0.0585*\dx},{-0.7681*\dy})
	-- ({0.0751*\dx},{-0.7670*\dy})
	-- ({0.0917*\dx},{-0.7657*\dy})
	-- ({0.1083*\dx},{-0.7641*\dy})
	-- ({0.1248*\dx},{-0.7622*\dy})
	-- ({0.1412*\dx},{-0.7601*\dy})
	-- ({0.1575*\dx},{-0.7577*\dy})
	-- ({0.1737*\dx},{-0.7551*\dy})
	-- ({0.1899*\dx},{-0.7522*\dy})
	-- ({0.2059*\dx},{-0.7491*\dy})
	-- ({0.2218*\dx},{-0.7457*\dy})
	-- ({0.2376*\dx},{-0.7421*\dy})
	-- ({0.2532*\dx},{-0.7383*\dy})
	-- ({0.2687*\dx},{-0.7343*\dy})
	-- ({0.2840*\dx},{-0.7300*\dy})
	-- ({0.2992*\dx},{-0.7256*\dy})
	-- ({0.3142*\dx},{-0.7209*\dy})
	-- ({0.3290*\dx},{-0.7160*\dy})
	-- ({0.3437*\dx},{-0.7110*\dy})
	-- ({0.3582*\dx},{-0.7058*\dy})
	-- ({0.3725*\dx},{-0.7004*\dy})
	-- ({0.3866*\dx},{-0.6948*\dy})
	-- ({0.4005*\dx},{-0.6891*\dy})
	-- ({0.4142*\dx},{-0.6832*\dy})
	-- ({0.4277*\dx},{-0.6772*\dy})
	-- ({0.4410*\dx},{-0.6710*\dy})
	-- ({0.4541*\dx},{-0.6648*\dy})
	-- ({0.4670*\dx},{-0.6584*\dy})
	-- ({0.4797*\dx},{-0.6518*\dy})
	-- ({0.4921*\dx},{-0.6452*\dy})
	-- ({0.5043*\dx},{-0.6385*\dy})
	-- ({0.5163*\dx},{-0.6317*\dy})
	-- ({0.5281*\dx},{-0.6248*\dy})
	-- ({0.5397*\dx},{-0.6178*\dy})
	-- ({0.5510*\dx},{-0.6108*\dy})
	-- ({0.5622*\dx},{-0.6037*\dy})
	-- ({0.5731*\dx},{-0.5965*\dy})
	-- ({0.5838*\dx},{-0.5893*\dy})
	-- ({0.5942*\dx},{-0.5821*\dy})
	-- ({0.6045*\dx},{-0.5748*\dy})
	-- ({0.6145*\dx},{-0.5675*\dy})
	-- ({0.6243*\dx},{-0.5601*\dy})
	-- ({0.6339*\dx},{-0.5528*\dy})
	-- ({0.6433*\dx},{-0.5454*\dy})
	-- ({0.6525*\dx},{-0.5380*\dy})
	-- ({0.6615*\dx},{-0.5307*\dy})
	-- ({0.6702*\dx},{-0.5233*\dy})
	-- ({0.6788*\dx},{-0.5159*\dy})
	-- ({0.6872*\dx},{-0.5086*\dy})
	-- ({0.6953*\dx},{-0.5012*\dy})
	-- ({0.7033*\dx},{-0.4939*\dy})
	-- ({0.7111*\dx},{-0.4866*\dy})
	-- ({0.7187*\dx},{-0.4794*\dy})
	-- ({0.7261*\dx},{-0.4722*\dy})
	-- ({0.7333*\dx},{-0.4650*\dy})
	-- ({0.7403*\dx},{-0.4578*\dy})
	-- ({0.7472*\dx},{-0.4507*\dy})
	-- ({0.7539*\dx},{-0.4437*\dy})
	-- ({0.7604*\dx},{-0.4367*\dy})
	-- ({0.7667*\dx},{-0.4297*\dy})
	-- ({0.7729*\dx},{-0.4228*\dy})
	-- ({0.7790*\dx},{-0.4160*\dy})
	-- ({0.7848*\dx},{-0.4092*\dy})
	-- ({0.7905*\dx},{-0.4025*\dy})
	-- ({0.7961*\dx},{-0.3958*\dy})
	-- ({0.8015*\dx},{-0.3892*\dy})
	-- ({0.8068*\dx},{-0.3827*\dy})
	-- ({0.8120*\dx},{-0.3763*\dy})
	-- ({0.8170*\dx},{-0.3699*\dy})
	-- ({0.8218*\dx},{-0.3636*\dy})
	-- ({0.8266*\dx},{-0.3573*\dy})
	-- ({0.8312*\dx},{-0.3512*\dy})
	-- ({0.8356*\dx},{-0.3451*\dy})
	-- ({0.8400*\dx},{-0.3390*\dy})
	-- ({0.8442*\dx},{-0.3331*\dy})
	-- ({0.8484*\dx},{-0.3272*\dy})
	-- ({0.8524*\dx},{-0.3214*\dy})
	-- ({0.8563*\dx},{-0.3157*\dy})
	-- ({0.8601*\dx},{-0.3101*\dy})
	-- ({0.8638*\dx},{-0.3045*\dy})
	-- ({0.8674*\dx},{-0.2990*\dy})
	-- ({0.8709*\dx},{-0.2936*\dy})
	-- ({0.8743*\dx},{-0.2883*\dy})
	-- ({0.8776*\dx},{-0.2830*\dy})
	-- ({0.8808*\dx},{-0.2778*\dy})
	-- ({0.8839*\dx},{-0.2727*\dy})
	-- ({0.8869*\dx},{-0.2677*\dy})
	-- ({0.8899*\dx},{-0.2627*\dy})
	-- ({0.8928*\dx},{-0.2579*\dy})
	-- ({0.8955*\dx},{-0.2531*\dy})
	-- ({0.8983*\dx},{-0.2483*\dy})
	-- ({0.9009*\dx},{-0.2437*\dy})
	-- ({0.9035*\dx},{-0.2391*\dy})
	-- ({0.9060*\dx},{-0.2346*\dy})
	-- ({0.9084*\dx},{-0.2302*\dy})
	-- ({0.9108*\dx},{-0.2258*\dy})
	-- ({0.9131*\dx},{-0.2215*\dy})
	-- ({0.9153*\dx},{-0.2173*\dy})
	-- ({0.9175*\dx},{-0.2132*\dy})
	-- ({0.9196*\dx},{-0.2091*\dy})
	-- ({0.9216*\dx},{-0.2051*\dy})
	-- ({0.9236*\dx},{-0.2012*\dy})
	-- ({0.9256*\dx},{-0.1973*\dy})
	-- ({0.9274*\dx},{-0.1935*\dy})
	-- ({0.9293*\dx},{-0.1897*\dy})
	-- ({0.9311*\dx},{-0.1861*\dy})
	-- ({0.9328*\dx},{-0.1825*\dy})
	-- ({0.9345*\dx},{-0.1789*\dy})
	-- ({0.9361*\dx},{-0.1754*\dy})
	-- ({0.9377*\dx},{-0.1720*\dy})
	-- ({0.9393*\dx},{-0.1687*\dy})
	-- ({0.9408*\dx},{-0.1654*\dy})
	-- ({0.9423*\dx},{-0.1621*\dy})
	-- ({0.9437*\dx},{-0.1589*\dy})
	-- ({0.9451*\dx},{-0.1558*\dy})
	-- ({0.9465*\dx},{-0.1527*\dy})
	-- ({0.9478*\dx},{-0.1497*\dy})
	-- ({0.9491*\dx},{-0.1468*\dy})
	-- ({0.9503*\dx},{-0.1439*\dy})
	-- ({0.9516*\dx},{-0.1410*\dy})
	-- ({0.9527*\dx},{-0.1382*\dy})
	-- ({0.9539*\dx},{-0.1355*\dy})
	-- ({0.9550*\dx},{-0.1328*\dy})
	-- ({0.9561*\dx},{-0.1302*\dy})
	-- ({0.9572*\dx},{-0.1276*\dy})
	-- ({0.9582*\dx},{-0.1250*\dy})
	-- ({0.9592*\dx},{-0.1225*\dy})
	-- ({0.9602*\dx},{-0.1201*\dy})
	-- ({0.9612*\dx},{-0.1177*\dy})
	-- ({0.9621*\dx},{-0.1153*\dy})
	-- ({0.9630*\dx},{-0.1130*\dy})
	-- ({0.9639*\dx},{-0.1108*\dy})
	-- ({0.9648*\dx},{-0.1085*\dy})
	-- ({0.9656*\dx},{-0.1064*\dy})
	-- ({0.9664*\dx},{-0.1042*\dy})
	-- ({0.9672*\dx},{-0.1021*\dy})
	-- ({0.9680*\dx},{-0.1001*\dy})
	-- ({0.9688*\dx},{-0.0981*\dy})
	-- ({0.9695*\dx},{-0.0961*\dy})
	-- ({0.9702*\dx},{-0.0941*\dy})
	-- ({0.9709*\dx},{-0.0922*\dy})
	-- ({0.9716*\dx},{-0.0904*\dy})
	-- ({0.9723*\dx},{-0.0886*\dy})
	-- ({0.9729*\dx},{-0.0868*\dy})
	-- ({0.9735*\dx},{-0.0850*\dy})
	-- ({0.9742*\dx},{-0.0833*\dy})
	-- ({0.9748*\dx},{-0.0816*\dy})
}
% v = 1.081077
\def\vpathP{
	({-0.9506*\dx},{-0.0683*\dy})
	-- ({-0.9495*\dx},{-0.0697*\dy})
	-- ({-0.9484*\dx},{-0.0711*\dy})
	-- ({-0.9473*\dx},{-0.0725*\dy})
	-- ({-0.9461*\dx},{-0.0740*\dy})
	-- ({-0.9449*\dx},{-0.0755*\dy})
	-- ({-0.9437*\dx},{-0.0770*\dy})
	-- ({-0.9425*\dx},{-0.0785*\dy})
	-- ({-0.9413*\dx},{-0.0801*\dy})
	-- ({-0.9400*\dx},{-0.0817*\dy})
	-- ({-0.9387*\dx},{-0.0833*\dy})
	-- ({-0.9373*\dx},{-0.0850*\dy})
	-- ({-0.9360*\dx},{-0.0867*\dy})
	-- ({-0.9346*\dx},{-0.0884*\dy})
	-- ({-0.9332*\dx},{-0.0902*\dy})
	-- ({-0.9317*\dx},{-0.0919*\dy})
	-- ({-0.9302*\dx},{-0.0938*\dy})
	-- ({-0.9287*\dx},{-0.0956*\dy})
	-- ({-0.9271*\dx},{-0.0975*\dy})
	-- ({-0.9255*\dx},{-0.0994*\dy})
	-- ({-0.9239*\dx},{-0.1014*\dy})
	-- ({-0.9222*\dx},{-0.1033*\dy})
	-- ({-0.9205*\dx},{-0.1054*\dy})
	-- ({-0.9188*\dx},{-0.1074*\dy})
	-- ({-0.9170*\dx},{-0.1095*\dy})
	-- ({-0.9152*\dx},{-0.1116*\dy})
	-- ({-0.9133*\dx},{-0.1138*\dy})
	-- ({-0.9114*\dx},{-0.1160*\dy})
	-- ({-0.9095*\dx},{-0.1183*\dy})
	-- ({-0.9075*\dx},{-0.1205*\dy})
	-- ({-0.9055*\dx},{-0.1228*\dy})
	-- ({-0.9034*\dx},{-0.1252*\dy})
	-- ({-0.9013*\dx},{-0.1276*\dy})
	-- ({-0.8991*\dx},{-0.1300*\dy})
	-- ({-0.8969*\dx},{-0.1325*\dy})
	-- ({-0.8947*\dx},{-0.1350*\dy})
	-- ({-0.8923*\dx},{-0.1376*\dy})
	-- ({-0.8900*\dx},{-0.1402*\dy})
	-- ({-0.8876*\dx},{-0.1428*\dy})
	-- ({-0.8851*\dx},{-0.1455*\dy})
	-- ({-0.8826*\dx},{-0.1483*\dy})
	-- ({-0.8800*\dx},{-0.1510*\dy})
	-- ({-0.8773*\dx},{-0.1538*\dy})
	-- ({-0.8746*\dx},{-0.1567*\dy})
	-- ({-0.8719*\dx},{-0.1596*\dy})
	-- ({-0.8691*\dx},{-0.1625*\dy})
	-- ({-0.8662*\dx},{-0.1655*\dy})
	-- ({-0.8632*\dx},{-0.1686*\dy})
	-- ({-0.8602*\dx},{-0.1717*\dy})
	-- ({-0.8572*\dx},{-0.1748*\dy})
	-- ({-0.8540*\dx},{-0.1779*\dy})
	-- ({-0.8508*\dx},{-0.1812*\dy})
	-- ({-0.8475*\dx},{-0.1844*\dy})
	-- ({-0.8442*\dx},{-0.1877*\dy})
	-- ({-0.8407*\dx},{-0.1911*\dy})
	-- ({-0.8372*\dx},{-0.1945*\dy})
	-- ({-0.8337*\dx},{-0.1979*\dy})
	-- ({-0.8300*\dx},{-0.2014*\dy})
	-- ({-0.8263*\dx},{-0.2050*\dy})
	-- ({-0.8225*\dx},{-0.2085*\dy})
	-- ({-0.8186*\dx},{-0.2122*\dy})
	-- ({-0.8146*\dx},{-0.2158*\dy})
	-- ({-0.8105*\dx},{-0.2196*\dy})
	-- ({-0.8064*\dx},{-0.2233*\dy})
	-- ({-0.8021*\dx},{-0.2271*\dy})
	-- ({-0.7978*\dx},{-0.2310*\dy})
	-- ({-0.7933*\dx},{-0.2349*\dy})
	-- ({-0.7888*\dx},{-0.2388*\dy})
	-- ({-0.7842*\dx},{-0.2428*\dy})
	-- ({-0.7795*\dx},{-0.2468*\dy})
	-- ({-0.7747*\dx},{-0.2509*\dy})
	-- ({-0.7698*\dx},{-0.2550*\dy})
	-- ({-0.7648*\dx},{-0.2591*\dy})
	-- ({-0.7596*\dx},{-0.2633*\dy})
	-- ({-0.7544*\dx},{-0.2675*\dy})
	-- ({-0.7491*\dx},{-0.2718*\dy})
	-- ({-0.7436*\dx},{-0.2761*\dy})
	-- ({-0.7381*\dx},{-0.2804*\dy})
	-- ({-0.7324*\dx},{-0.2848*\dy})
	-- ({-0.7266*\dx},{-0.2892*\dy})
	-- ({-0.7207*\dx},{-0.2936*\dy})
	-- ({-0.7147*\dx},{-0.2981*\dy})
	-- ({-0.7086*\dx},{-0.3026*\dy})
	-- ({-0.7023*\dx},{-0.3071*\dy})
	-- ({-0.6960*\dx},{-0.3116*\dy})
	-- ({-0.6895*\dx},{-0.3162*\dy})
	-- ({-0.6828*\dx},{-0.3208*\dy})
	-- ({-0.6761*\dx},{-0.3254*\dy})
	-- ({-0.6692*\dx},{-0.3300*\dy})
	-- ({-0.6622*\dx},{-0.3347*\dy})
	-- ({-0.6551*\dx},{-0.3393*\dy})
	-- ({-0.6478*\dx},{-0.3440*\dy})
	-- ({-0.6404*\dx},{-0.3487*\dy})
	-- ({-0.6328*\dx},{-0.3534*\dy})
	-- ({-0.6251*\dx},{-0.3581*\dy})
	-- ({-0.6173*\dx},{-0.3628*\dy})
	-- ({-0.6094*\dx},{-0.3675*\dy})
	-- ({-0.6013*\dx},{-0.3722*\dy})
	-- ({-0.5930*\dx},{-0.3769*\dy})
	-- ({-0.5847*\dx},{-0.3816*\dy})
	-- ({-0.5762*\dx},{-0.3863*\dy})
	-- ({-0.5675*\dx},{-0.3910*\dy})
	-- ({-0.5587*\dx},{-0.3956*\dy})
	-- ({-0.5498*\dx},{-0.4003*\dy})
	-- ({-0.5407*\dx},{-0.4049*\dy})
	-- ({-0.5314*\dx},{-0.4095*\dy})
	-- ({-0.5221*\dx},{-0.4140*\dy})
	-- ({-0.5125*\dx},{-0.4185*\dy})
	-- ({-0.5029*\dx},{-0.4230*\dy})
	-- ({-0.4931*\dx},{-0.4275*\dy})
	-- ({-0.4831*\dx},{-0.4319*\dy})
	-- ({-0.4731*\dx},{-0.4362*\dy})
	-- ({-0.4628*\dx},{-0.4406*\dy})
	-- ({-0.4525*\dx},{-0.4448*\dy})
	-- ({-0.4420*\dx},{-0.4490*\dy})
	-- ({-0.4313*\dx},{-0.4532*\dy})
	-- ({-0.4205*\dx},{-0.4572*\dy})
	-- ({-0.4096*\dx},{-0.4612*\dy})
	-- ({-0.3986*\dx},{-0.4652*\dy})
	-- ({-0.3874*\dx},{-0.4690*\dy})
	-- ({-0.3761*\dx},{-0.4728*\dy})
	-- ({-0.3647*\dx},{-0.4765*\dy})
	-- ({-0.3531*\dx},{-0.4801*\dy})
	-- ({-0.3414*\dx},{-0.4836*\dy})
	-- ({-0.3296*\dx},{-0.4870*\dy})
	-- ({-0.3177*\dx},{-0.4904*\dy})
	-- ({-0.3057*\dx},{-0.4936*\dy})
	-- ({-0.2936*\dx},{-0.4967*\dy})
	-- ({-0.2813*\dx},{-0.4997*\dy})
	-- ({-0.2690*\dx},{-0.5026*\dy})
	-- ({-0.2565*\dx},{-0.5054*\dy})
	-- ({-0.2440*\dx},{-0.5081*\dy})
	-- ({-0.2313*\dx},{-0.5106*\dy})
	-- ({-0.2186*\dx},{-0.5130*\dy})
	-- ({-0.2058*\dx},{-0.5153*\dy})
	-- ({-0.1929*\dx},{-0.5175*\dy})
	-- ({-0.1799*\dx},{-0.5195*\dy})
	-- ({-0.1669*\dx},{-0.5214*\dy})
	-- ({-0.1538*\dx},{-0.5232*\dy})
	-- ({-0.1406*\dx},{-0.5248*\dy})
	-- ({-0.1274*\dx},{-0.5263*\dy})
	-- ({-0.1141*\dx},{-0.5276*\dy})
	-- ({-0.1008*\dx},{-0.5288*\dy})
	-- ({-0.0874*\dx},{-0.5298*\dy})
	-- ({-0.0741*\dx},{-0.5307*\dy})
	-- ({-0.0606*\dx},{-0.5315*\dy})
	-- ({-0.0472*\dx},{-0.5321*\dy})
	-- ({-0.0337*\dx},{-0.5326*\dy})
	-- ({-0.0202*\dx},{-0.5329*\dy})
	-- ({-0.0067*\dx},{-0.5330*\dy})
	-- ({0.0067*\dx},{-0.5330*\dy})
	-- ({0.0202*\dx},{-0.5329*\dy})
	-- ({0.0337*\dx},{-0.5326*\dy})
	-- ({0.0472*\dx},{-0.5321*\dy})
	-- ({0.0606*\dx},{-0.5315*\dy})
	-- ({0.0741*\dx},{-0.5307*\dy})
	-- ({0.0874*\dx},{-0.5298*\dy})
	-- ({0.1008*\dx},{-0.5288*\dy})
	-- ({0.1141*\dx},{-0.5276*\dy})
	-- ({0.1274*\dx},{-0.5263*\dy})
	-- ({0.1406*\dx},{-0.5248*\dy})
	-- ({0.1538*\dx},{-0.5232*\dy})
	-- ({0.1669*\dx},{-0.5214*\dy})
	-- ({0.1799*\dx},{-0.5195*\dy})
	-- ({0.1929*\dx},{-0.5175*\dy})
	-- ({0.2058*\dx},{-0.5153*\dy})
	-- ({0.2186*\dx},{-0.5130*\dy})
	-- ({0.2313*\dx},{-0.5106*\dy})
	-- ({0.2440*\dx},{-0.5081*\dy})
	-- ({0.2565*\dx},{-0.5054*\dy})
	-- ({0.2690*\dx},{-0.5026*\dy})
	-- ({0.2813*\dx},{-0.4997*\dy})
	-- ({0.2936*\dx},{-0.4967*\dy})
	-- ({0.3057*\dx},{-0.4936*\dy})
	-- ({0.3177*\dx},{-0.4904*\dy})
	-- ({0.3296*\dx},{-0.4870*\dy})
	-- ({0.3414*\dx},{-0.4836*\dy})
	-- ({0.3531*\dx},{-0.4801*\dy})
	-- ({0.3647*\dx},{-0.4765*\dy})
	-- ({0.3761*\dx},{-0.4728*\dy})
	-- ({0.3874*\dx},{-0.4690*\dy})
	-- ({0.3986*\dx},{-0.4652*\dy})
	-- ({0.4096*\dx},{-0.4612*\dy})
	-- ({0.4205*\dx},{-0.4572*\dy})
	-- ({0.4313*\dx},{-0.4532*\dy})
	-- ({0.4420*\dx},{-0.4490*\dy})
	-- ({0.4525*\dx},{-0.4448*\dy})
	-- ({0.4628*\dx},{-0.4406*\dy})
	-- ({0.4731*\dx},{-0.4362*\dy})
	-- ({0.4831*\dx},{-0.4319*\dy})
	-- ({0.4931*\dx},{-0.4275*\dy})
	-- ({0.5029*\dx},{-0.4230*\dy})
	-- ({0.5125*\dx},{-0.4185*\dy})
	-- ({0.5221*\dx},{-0.4140*\dy})
	-- ({0.5314*\dx},{-0.4095*\dy})
	-- ({0.5407*\dx},{-0.4049*\dy})
	-- ({0.5498*\dx},{-0.4003*\dy})
	-- ({0.5587*\dx},{-0.3956*\dy})
	-- ({0.5675*\dx},{-0.3910*\dy})
	-- ({0.5762*\dx},{-0.3863*\dy})
	-- ({0.5847*\dx},{-0.3816*\dy})
	-- ({0.5930*\dx},{-0.3769*\dy})
	-- ({0.6013*\dx},{-0.3722*\dy})
	-- ({0.6094*\dx},{-0.3675*\dy})
	-- ({0.6173*\dx},{-0.3628*\dy})
	-- ({0.6251*\dx},{-0.3581*\dy})
	-- ({0.6328*\dx},{-0.3534*\dy})
	-- ({0.6404*\dx},{-0.3487*\dy})
	-- ({0.6478*\dx},{-0.3440*\dy})
	-- ({0.6551*\dx},{-0.3393*\dy})
	-- ({0.6622*\dx},{-0.3347*\dy})
	-- ({0.6692*\dx},{-0.3300*\dy})
	-- ({0.6761*\dx},{-0.3254*\dy})
	-- ({0.6828*\dx},{-0.3208*\dy})
	-- ({0.6895*\dx},{-0.3162*\dy})
	-- ({0.6960*\dx},{-0.3116*\dy})
	-- ({0.7023*\dx},{-0.3071*\dy})
	-- ({0.7086*\dx},{-0.3026*\dy})
	-- ({0.7147*\dx},{-0.2981*\dy})
	-- ({0.7207*\dx},{-0.2936*\dy})
	-- ({0.7266*\dx},{-0.2892*\dy})
	-- ({0.7324*\dx},{-0.2848*\dy})
	-- ({0.7381*\dx},{-0.2804*\dy})
	-- ({0.7436*\dx},{-0.2761*\dy})
	-- ({0.7491*\dx},{-0.2718*\dy})
	-- ({0.7544*\dx},{-0.2675*\dy})
	-- ({0.7596*\dx},{-0.2633*\dy})
	-- ({0.7648*\dx},{-0.2591*\dy})
	-- ({0.7698*\dx},{-0.2550*\dy})
	-- ({0.7747*\dx},{-0.2509*\dy})
	-- ({0.7795*\dx},{-0.2468*\dy})
	-- ({0.7842*\dx},{-0.2428*\dy})
	-- ({0.7888*\dx},{-0.2388*\dy})
	-- ({0.7933*\dx},{-0.2349*\dy})
	-- ({0.7978*\dx},{-0.2310*\dy})
	-- ({0.8021*\dx},{-0.2271*\dy})
	-- ({0.8064*\dx},{-0.2233*\dy})
	-- ({0.8105*\dx},{-0.2196*\dy})
	-- ({0.8146*\dx},{-0.2158*\dy})
	-- ({0.8186*\dx},{-0.2122*\dy})
	-- ({0.8225*\dx},{-0.2085*\dy})
	-- ({0.8263*\dx},{-0.2050*\dy})
	-- ({0.8300*\dx},{-0.2014*\dy})
	-- ({0.8337*\dx},{-0.1979*\dy})
	-- ({0.8372*\dx},{-0.1945*\dy})
	-- ({0.8407*\dx},{-0.1911*\dy})
	-- ({0.8442*\dx},{-0.1877*\dy})
	-- ({0.8475*\dx},{-0.1844*\dy})
	-- ({0.8508*\dx},{-0.1812*\dy})
	-- ({0.8540*\dx},{-0.1779*\dy})
	-- ({0.8572*\dx},{-0.1748*\dy})
	-- ({0.8602*\dx},{-0.1717*\dy})
	-- ({0.8632*\dx},{-0.1686*\dy})
	-- ({0.8662*\dx},{-0.1655*\dy})
	-- ({0.8691*\dx},{-0.1625*\dy})
	-- ({0.8719*\dx},{-0.1596*\dy})
	-- ({0.8746*\dx},{-0.1567*\dy})
	-- ({0.8773*\dx},{-0.1538*\dy})
	-- ({0.8800*\dx},{-0.1510*\dy})
	-- ({0.8826*\dx},{-0.1483*\dy})
	-- ({0.8851*\dx},{-0.1455*\dy})
	-- ({0.8876*\dx},{-0.1428*\dy})
	-- ({0.8900*\dx},{-0.1402*\dy})
	-- ({0.8923*\dx},{-0.1376*\dy})
	-- ({0.8947*\dx},{-0.1350*\dy})
	-- ({0.8969*\dx},{-0.1325*\dy})
	-- ({0.8991*\dx},{-0.1300*\dy})
	-- ({0.9013*\dx},{-0.1276*\dy})
	-- ({0.9034*\dx},{-0.1252*\dy})
	-- ({0.9055*\dx},{-0.1228*\dy})
	-- ({0.9075*\dx},{-0.1205*\dy})
	-- ({0.9095*\dx},{-0.1183*\dy})
	-- ({0.9114*\dx},{-0.1160*\dy})
	-- ({0.9133*\dx},{-0.1138*\dy})
	-- ({0.9152*\dx},{-0.1116*\dy})
	-- ({0.9170*\dx},{-0.1095*\dy})
	-- ({0.9188*\dx},{-0.1074*\dy})
	-- ({0.9205*\dx},{-0.1054*\dy})
	-- ({0.9222*\dx},{-0.1033*\dy})
	-- ({0.9239*\dx},{-0.1014*\dy})
	-- ({0.9255*\dx},{-0.0994*\dy})
	-- ({0.9271*\dx},{-0.0975*\dy})
	-- ({0.9287*\dx},{-0.0956*\dy})
	-- ({0.9302*\dx},{-0.0938*\dy})
	-- ({0.9317*\dx},{-0.0919*\dy})
	-- ({0.9332*\dx},{-0.0902*\dy})
	-- ({0.9346*\dx},{-0.0884*\dy})
	-- ({0.9360*\dx},{-0.0867*\dy})
	-- ({0.9373*\dx},{-0.0850*\dy})
	-- ({0.9387*\dx},{-0.0833*\dy})
	-- ({0.9400*\dx},{-0.0817*\dy})
	-- ({0.9413*\dx},{-0.0801*\dy})
	-- ({0.9425*\dx},{-0.0785*\dy})
	-- ({0.9437*\dx},{-0.0770*\dy})
	-- ({0.9449*\dx},{-0.0755*\dy})
	-- ({0.9461*\dx},{-0.0740*\dy})
	-- ({0.9473*\dx},{-0.0725*\dy})
	-- ({0.9484*\dx},{-0.0711*\dy})
	-- ({0.9495*\dx},{-0.0697*\dy})
	-- ({0.9506*\dx},{-0.0683*\dy})
}
% v = 1.247397
\def\vpathQ{
	({-0.9321*\dx},{-0.0486*\dy})
	-- ({-0.9307*\dx},{-0.0496*\dy})
	-- ({-0.9292*\dx},{-0.0506*\dy})
	-- ({-0.9278*\dx},{-0.0516*\dy})
	-- ({-0.9263*\dx},{-0.0526*\dy})
	-- ({-0.9247*\dx},{-0.0536*\dy})
	-- ({-0.9232*\dx},{-0.0547*\dy})
	-- ({-0.9216*\dx},{-0.0557*\dy})
	-- ({-0.9199*\dx},{-0.0568*\dy})
	-- ({-0.9183*\dx},{-0.0579*\dy})
	-- ({-0.9166*\dx},{-0.0591*\dy})
	-- ({-0.9148*\dx},{-0.0602*\dy})
	-- ({-0.9130*\dx},{-0.0614*\dy})
	-- ({-0.9112*\dx},{-0.0626*\dy})
	-- ({-0.9094*\dx},{-0.0638*\dy})
	-- ({-0.9075*\dx},{-0.0650*\dy})
	-- ({-0.9056*\dx},{-0.0663*\dy})
	-- ({-0.9037*\dx},{-0.0675*\dy})
	-- ({-0.9017*\dx},{-0.0688*\dy})
	-- ({-0.8996*\dx},{-0.0701*\dy})
	-- ({-0.8975*\dx},{-0.0715*\dy})
	-- ({-0.8954*\dx},{-0.0728*\dy})
	-- ({-0.8933*\dx},{-0.0742*\dy})
	-- ({-0.8911*\dx},{-0.0756*\dy})
	-- ({-0.8888*\dx},{-0.0771*\dy})
	-- ({-0.8865*\dx},{-0.0785*\dy})
	-- ({-0.8842*\dx},{-0.0800*\dy})
	-- ({-0.8818*\dx},{-0.0815*\dy})
	-- ({-0.8794*\dx},{-0.0830*\dy})
	-- ({-0.8769*\dx},{-0.0845*\dy})
	-- ({-0.8744*\dx},{-0.0861*\dy})
	-- ({-0.8718*\dx},{-0.0877*\dy})
	-- ({-0.8692*\dx},{-0.0893*\dy})
	-- ({-0.8665*\dx},{-0.0910*\dy})
	-- ({-0.8638*\dx},{-0.0926*\dy})
	-- ({-0.8610*\dx},{-0.0943*\dy})
	-- ({-0.8581*\dx},{-0.0961*\dy})
	-- ({-0.8552*\dx},{-0.0978*\dy})
	-- ({-0.8523*\dx},{-0.0996*\dy})
	-- ({-0.8493*\dx},{-0.1014*\dy})
	-- ({-0.8462*\dx},{-0.1032*\dy})
	-- ({-0.8431*\dx},{-0.1050*\dy})
	-- ({-0.8399*\dx},{-0.1069*\dy})
	-- ({-0.8367*\dx},{-0.1088*\dy})
	-- ({-0.8333*\dx},{-0.1107*\dy})
	-- ({-0.8300*\dx},{-0.1127*\dy})
	-- ({-0.8265*\dx},{-0.1147*\dy})
	-- ({-0.8230*\dx},{-0.1167*\dy})
	-- ({-0.8195*\dx},{-0.1187*\dy})
	-- ({-0.8158*\dx},{-0.1208*\dy})
	-- ({-0.8121*\dx},{-0.1228*\dy})
	-- ({-0.8084*\dx},{-0.1249*\dy})
	-- ({-0.8045*\dx},{-0.1271*\dy})
	-- ({-0.8006*\dx},{-0.1292*\dy})
	-- ({-0.7966*\dx},{-0.1314*\dy})
	-- ({-0.7926*\dx},{-0.1336*\dy})
	-- ({-0.7884*\dx},{-0.1359*\dy})
	-- ({-0.7842*\dx},{-0.1381*\dy})
	-- ({-0.7800*\dx},{-0.1404*\dy})
	-- ({-0.7756*\dx},{-0.1428*\dy})
	-- ({-0.7712*\dx},{-0.1451*\dy})
	-- ({-0.7666*\dx},{-0.1475*\dy})
	-- ({-0.7620*\dx},{-0.1498*\dy})
	-- ({-0.7574*\dx},{-0.1523*\dy})
	-- ({-0.7526*\dx},{-0.1547*\dy})
	-- ({-0.7477*\dx},{-0.1571*\dy})
	-- ({-0.7428*\dx},{-0.1596*\dy})
	-- ({-0.7378*\dx},{-0.1621*\dy})
	-- ({-0.7327*\dx},{-0.1647*\dy})
	-- ({-0.7275*\dx},{-0.1672*\dy})
	-- ({-0.7222*\dx},{-0.1698*\dy})
	-- ({-0.7168*\dx},{-0.1724*\dy})
	-- ({-0.7114*\dx},{-0.1750*\dy})
	-- ({-0.7058*\dx},{-0.1776*\dy})
	-- ({-0.7001*\dx},{-0.1802*\dy})
	-- ({-0.6944*\dx},{-0.1829*\dy})
	-- ({-0.6886*\dx},{-0.1856*\dy})
	-- ({-0.6826*\dx},{-0.1883*\dy})
	-- ({-0.6766*\dx},{-0.1910*\dy})
	-- ({-0.6705*\dx},{-0.1937*\dy})
	-- ({-0.6642*\dx},{-0.1964*\dy})
	-- ({-0.6579*\dx},{-0.1992*\dy})
	-- ({-0.6515*\dx},{-0.2019*\dy})
	-- ({-0.6450*\dx},{-0.2047*\dy})
	-- ({-0.6383*\dx},{-0.2075*\dy})
	-- ({-0.6316*\dx},{-0.2103*\dy})
	-- ({-0.6248*\dx},{-0.2131*\dy})
	-- ({-0.6178*\dx},{-0.2159*\dy})
	-- ({-0.6108*\dx},{-0.2187*\dy})
	-- ({-0.6037*\dx},{-0.2215*\dy})
	-- ({-0.5964*\dx},{-0.2243*\dy})
	-- ({-0.5891*\dx},{-0.2271*\dy})
	-- ({-0.5816*\dx},{-0.2299*\dy})
	-- ({-0.5740*\dx},{-0.2327*\dy})
	-- ({-0.5664*\dx},{-0.2355*\dy})
	-- ({-0.5586*\dx},{-0.2383*\dy})
	-- ({-0.5507*\dx},{-0.2411*\dy})
	-- ({-0.5427*\dx},{-0.2439*\dy})
	-- ({-0.5347*\dx},{-0.2467*\dy})
	-- ({-0.5265*\dx},{-0.2494*\dy})
	-- ({-0.5182*\dx},{-0.2522*\dy})
	-- ({-0.5098*\dx},{-0.2549*\dy})
	-- ({-0.5012*\dx},{-0.2577*\dy})
	-- ({-0.4926*\dx},{-0.2604*\dy})
	-- ({-0.4839*\dx},{-0.2630*\dy})
	-- ({-0.4751*\dx},{-0.2657*\dy})
	-- ({-0.4661*\dx},{-0.2683*\dy})
	-- ({-0.4571*\dx},{-0.2710*\dy})
	-- ({-0.4480*\dx},{-0.2736*\dy})
	-- ({-0.4387*\dx},{-0.2761*\dy})
	-- ({-0.4294*\dx},{-0.2786*\dy})
	-- ({-0.4200*\dx},{-0.2811*\dy})
	-- ({-0.4104*\dx},{-0.2836*\dy})
	-- ({-0.4008*\dx},{-0.2860*\dy})
	-- ({-0.3911*\dx},{-0.2884*\dy})
	-- ({-0.3813*\dx},{-0.2908*\dy})
	-- ({-0.3713*\dx},{-0.2931*\dy})
	-- ({-0.3613*\dx},{-0.2953*\dy})
	-- ({-0.3512*\dx},{-0.2976*\dy})
	-- ({-0.3411*\dx},{-0.2997*\dy})
	-- ({-0.3308*\dx},{-0.3019*\dy})
	-- ({-0.3204*\dx},{-0.3039*\dy})
	-- ({-0.3100*\dx},{-0.3060*\dy})
	-- ({-0.2995*\dx},{-0.3079*\dy})
	-- ({-0.2889*\dx},{-0.3098*\dy})
	-- ({-0.2782*\dx},{-0.3117*\dy})
	-- ({-0.2674*\dx},{-0.3135*\dy})
	-- ({-0.2566*\dx},{-0.3152*\dy})
	-- ({-0.2457*\dx},{-0.3169*\dy})
	-- ({-0.2348*\dx},{-0.3185*\dy})
	-- ({-0.2237*\dx},{-0.3200*\dy})
	-- ({-0.2127*\dx},{-0.3215*\dy})
	-- ({-0.2015*\dx},{-0.3229*\dy})
	-- ({-0.1903*\dx},{-0.3242*\dy})
	-- ({-0.1790*\dx},{-0.3255*\dy})
	-- ({-0.1677*\dx},{-0.3267*\dy})
	-- ({-0.1564*\dx},{-0.3278*\dy})
	-- ({-0.1450*\dx},{-0.3288*\dy})
	-- ({-0.1335*\dx},{-0.3298*\dy})
	-- ({-0.1221*\dx},{-0.3307*\dy})
	-- ({-0.1105*\dx},{-0.3315*\dy})
	-- ({-0.0990*\dx},{-0.3322*\dy})
	-- ({-0.0874*\dx},{-0.3329*\dy})
	-- ({-0.0758*\dx},{-0.3334*\dy})
	-- ({-0.0642*\dx},{-0.3339*\dy})
	-- ({-0.0525*\dx},{-0.3343*\dy})
	-- ({-0.0409*\dx},{-0.3347*\dy})
	-- ({-0.0292*\dx},{-0.3349*\dy})
	-- ({-0.0175*\dx},{-0.3351*\dy})
	-- ({-0.0058*\dx},{-0.3352*\dy})
	-- ({0.0058*\dx},{-0.3352*\dy})
	-- ({0.0175*\dx},{-0.3351*\dy})
	-- ({0.0292*\dx},{-0.3349*\dy})
	-- ({0.0409*\dx},{-0.3347*\dy})
	-- ({0.0525*\dx},{-0.3343*\dy})
	-- ({0.0642*\dx},{-0.3339*\dy})
	-- ({0.0758*\dx},{-0.3334*\dy})
	-- ({0.0874*\dx},{-0.3329*\dy})
	-- ({0.0990*\dx},{-0.3322*\dy})
	-- ({0.1105*\dx},{-0.3315*\dy})
	-- ({0.1221*\dx},{-0.3307*\dy})
	-- ({0.1335*\dx},{-0.3298*\dy})
	-- ({0.1450*\dx},{-0.3288*\dy})
	-- ({0.1564*\dx},{-0.3278*\dy})
	-- ({0.1677*\dx},{-0.3267*\dy})
	-- ({0.1790*\dx},{-0.3255*\dy})
	-- ({0.1903*\dx},{-0.3242*\dy})
	-- ({0.2015*\dx},{-0.3229*\dy})
	-- ({0.2127*\dx},{-0.3215*\dy})
	-- ({0.2237*\dx},{-0.3200*\dy})
	-- ({0.2348*\dx},{-0.3185*\dy})
	-- ({0.2457*\dx},{-0.3169*\dy})
	-- ({0.2566*\dx},{-0.3152*\dy})
	-- ({0.2674*\dx},{-0.3135*\dy})
	-- ({0.2782*\dx},{-0.3117*\dy})
	-- ({0.2889*\dx},{-0.3098*\dy})
	-- ({0.2995*\dx},{-0.3079*\dy})
	-- ({0.3100*\dx},{-0.3060*\dy})
	-- ({0.3204*\dx},{-0.3039*\dy})
	-- ({0.3308*\dx},{-0.3019*\dy})
	-- ({0.3411*\dx},{-0.2997*\dy})
	-- ({0.3512*\dx},{-0.2976*\dy})
	-- ({0.3613*\dx},{-0.2953*\dy})
	-- ({0.3713*\dx},{-0.2931*\dy})
	-- ({0.3813*\dx},{-0.2908*\dy})
	-- ({0.3911*\dx},{-0.2884*\dy})
	-- ({0.4008*\dx},{-0.2860*\dy})
	-- ({0.4104*\dx},{-0.2836*\dy})
	-- ({0.4200*\dx},{-0.2811*\dy})
	-- ({0.4294*\dx},{-0.2786*\dy})
	-- ({0.4387*\dx},{-0.2761*\dy})
	-- ({0.4480*\dx},{-0.2736*\dy})
	-- ({0.4571*\dx},{-0.2710*\dy})
	-- ({0.4661*\dx},{-0.2683*\dy})
	-- ({0.4751*\dx},{-0.2657*\dy})
	-- ({0.4839*\dx},{-0.2630*\dy})
	-- ({0.4926*\dx},{-0.2604*\dy})
	-- ({0.5012*\dx},{-0.2577*\dy})
	-- ({0.5098*\dx},{-0.2549*\dy})
	-- ({0.5182*\dx},{-0.2522*\dy})
	-- ({0.5265*\dx},{-0.2494*\dy})
	-- ({0.5347*\dx},{-0.2467*\dy})
	-- ({0.5427*\dx},{-0.2439*\dy})
	-- ({0.5507*\dx},{-0.2411*\dy})
	-- ({0.5586*\dx},{-0.2383*\dy})
	-- ({0.5664*\dx},{-0.2355*\dy})
	-- ({0.5740*\dx},{-0.2327*\dy})
	-- ({0.5816*\dx},{-0.2299*\dy})
	-- ({0.5891*\dx},{-0.2271*\dy})
	-- ({0.5964*\dx},{-0.2243*\dy})
	-- ({0.6037*\dx},{-0.2215*\dy})
	-- ({0.6108*\dx},{-0.2187*\dy})
	-- ({0.6178*\dx},{-0.2159*\dy})
	-- ({0.6248*\dx},{-0.2131*\dy})
	-- ({0.6316*\dx},{-0.2103*\dy})
	-- ({0.6383*\dx},{-0.2075*\dy})
	-- ({0.6450*\dx},{-0.2047*\dy})
	-- ({0.6515*\dx},{-0.2019*\dy})
	-- ({0.6579*\dx},{-0.1992*\dy})
	-- ({0.6642*\dx},{-0.1964*\dy})
	-- ({0.6705*\dx},{-0.1937*\dy})
	-- ({0.6766*\dx},{-0.1910*\dy})
	-- ({0.6826*\dx},{-0.1883*\dy})
	-- ({0.6886*\dx},{-0.1856*\dy})
	-- ({0.6944*\dx},{-0.1829*\dy})
	-- ({0.7001*\dx},{-0.1802*\dy})
	-- ({0.7058*\dx},{-0.1776*\dy})
	-- ({0.7114*\dx},{-0.1750*\dy})
	-- ({0.7168*\dx},{-0.1724*\dy})
	-- ({0.7222*\dx},{-0.1698*\dy})
	-- ({0.7275*\dx},{-0.1672*\dy})
	-- ({0.7327*\dx},{-0.1647*\dy})
	-- ({0.7378*\dx},{-0.1621*\dy})
	-- ({0.7428*\dx},{-0.1596*\dy})
	-- ({0.7477*\dx},{-0.1571*\dy})
	-- ({0.7526*\dx},{-0.1547*\dy})
	-- ({0.7574*\dx},{-0.1523*\dy})
	-- ({0.7620*\dx},{-0.1498*\dy})
	-- ({0.7666*\dx},{-0.1475*\dy})
	-- ({0.7712*\dx},{-0.1451*\dy})
	-- ({0.7756*\dx},{-0.1428*\dy})
	-- ({0.7800*\dx},{-0.1404*\dy})
	-- ({0.7842*\dx},{-0.1381*\dy})
	-- ({0.7884*\dx},{-0.1359*\dy})
	-- ({0.7926*\dx},{-0.1336*\dy})
	-- ({0.7966*\dx},{-0.1314*\dy})
	-- ({0.8006*\dx},{-0.1292*\dy})
	-- ({0.8045*\dx},{-0.1271*\dy})
	-- ({0.8084*\dx},{-0.1249*\dy})
	-- ({0.8121*\dx},{-0.1228*\dy})
	-- ({0.8158*\dx},{-0.1208*\dy})
	-- ({0.8195*\dx},{-0.1187*\dy})
	-- ({0.8230*\dx},{-0.1167*\dy})
	-- ({0.8265*\dx},{-0.1147*\dy})
	-- ({0.8300*\dx},{-0.1127*\dy})
	-- ({0.8333*\dx},{-0.1107*\dy})
	-- ({0.8367*\dx},{-0.1088*\dy})
	-- ({0.8399*\dx},{-0.1069*\dy})
	-- ({0.8431*\dx},{-0.1050*\dy})
	-- ({0.8462*\dx},{-0.1032*\dy})
	-- ({0.8493*\dx},{-0.1014*\dy})
	-- ({0.8523*\dx},{-0.0996*\dy})
	-- ({0.8552*\dx},{-0.0978*\dy})
	-- ({0.8581*\dx},{-0.0961*\dy})
	-- ({0.8610*\dx},{-0.0943*\dy})
	-- ({0.8638*\dx},{-0.0926*\dy})
	-- ({0.8665*\dx},{-0.0910*\dy})
	-- ({0.8692*\dx},{-0.0893*\dy})
	-- ({0.8718*\dx},{-0.0877*\dy})
	-- ({0.8744*\dx},{-0.0861*\dy})
	-- ({0.8769*\dx},{-0.0845*\dy})
	-- ({0.8794*\dx},{-0.0830*\dy})
	-- ({0.8818*\dx},{-0.0815*\dy})
	-- ({0.8842*\dx},{-0.0800*\dy})
	-- ({0.8865*\dx},{-0.0785*\dy})
	-- ({0.8888*\dx},{-0.0771*\dy})
	-- ({0.8911*\dx},{-0.0756*\dy})
	-- ({0.8933*\dx},{-0.0742*\dy})
	-- ({0.8954*\dx},{-0.0728*\dy})
	-- ({0.8975*\dx},{-0.0715*\dy})
	-- ({0.8996*\dx},{-0.0701*\dy})
	-- ({0.9017*\dx},{-0.0688*\dy})
	-- ({0.9037*\dx},{-0.0675*\dy})
	-- ({0.9056*\dx},{-0.0663*\dy})
	-- ({0.9075*\dx},{-0.0650*\dy})
	-- ({0.9094*\dx},{-0.0638*\dy})
	-- ({0.9112*\dx},{-0.0626*\dy})
	-- ({0.9130*\dx},{-0.0614*\dy})
	-- ({0.9148*\dx},{-0.0602*\dy})
	-- ({0.9166*\dx},{-0.0591*\dy})
	-- ({0.9183*\dx},{-0.0579*\dy})
	-- ({0.9199*\dx},{-0.0568*\dy})
	-- ({0.9216*\dx},{-0.0557*\dy})
	-- ({0.9232*\dx},{-0.0547*\dy})
	-- ({0.9247*\dx},{-0.0536*\dy})
	-- ({0.9263*\dx},{-0.0526*\dy})
	-- ({0.9278*\dx},{-0.0516*\dy})
	-- ({0.9292*\dx},{-0.0506*\dy})
	-- ({0.9307*\dx},{-0.0496*\dy})
	-- ({0.9321*\dx},{-0.0486*\dy})
}
% v = 1.413717
\def\vpathR{
	({-0.9207*\dx},{-0.0246*\dy})
	-- ({-0.9191*\dx},{-0.0251*\dy})
	-- ({-0.9175*\dx},{-0.0256*\dy})
	-- ({-0.9158*\dx},{-0.0261*\dy})
	-- ({-0.9141*\dx},{-0.0266*\dy})
	-- ({-0.9123*\dx},{-0.0271*\dy})
	-- ({-0.9105*\dx},{-0.0277*\dy})
	-- ({-0.9087*\dx},{-0.0282*\dy})
	-- ({-0.9068*\dx},{-0.0287*\dy})
	-- ({-0.9049*\dx},{-0.0293*\dy})
	-- ({-0.9030*\dx},{-0.0298*\dy})
	-- ({-0.9010*\dx},{-0.0304*\dy})
	-- ({-0.8990*\dx},{-0.0310*\dy})
	-- ({-0.8970*\dx},{-0.0316*\dy})
	-- ({-0.8949*\dx},{-0.0322*\dy})
	-- ({-0.8928*\dx},{-0.0328*\dy})
	-- ({-0.8906*\dx},{-0.0334*\dy})
	-- ({-0.8884*\dx},{-0.0340*\dy})
	-- ({-0.8862*\dx},{-0.0347*\dy})
	-- ({-0.8839*\dx},{-0.0353*\dy})
	-- ({-0.8815*\dx},{-0.0360*\dy})
	-- ({-0.8792*\dx},{-0.0367*\dy})
	-- ({-0.8767*\dx},{-0.0374*\dy})
	-- ({-0.8743*\dx},{-0.0380*\dy})
	-- ({-0.8718*\dx},{-0.0388*\dy})
	-- ({-0.8692*\dx},{-0.0395*\dy})
	-- ({-0.8666*\dx},{-0.0402*\dy})
	-- ({-0.8639*\dx},{-0.0409*\dy})
	-- ({-0.8612*\dx},{-0.0417*\dy})
	-- ({-0.8585*\dx},{-0.0424*\dy})
	-- ({-0.8557*\dx},{-0.0432*\dy})
	-- ({-0.8528*\dx},{-0.0440*\dy})
	-- ({-0.8499*\dx},{-0.0448*\dy})
	-- ({-0.8469*\dx},{-0.0456*\dy})
	-- ({-0.8439*\dx},{-0.0464*\dy})
	-- ({-0.8408*\dx},{-0.0472*\dy})
	-- ({-0.8377*\dx},{-0.0481*\dy})
	-- ({-0.8345*\dx},{-0.0489*\dy})
	-- ({-0.8313*\dx},{-0.0498*\dy})
	-- ({-0.8280*\dx},{-0.0507*\dy})
	-- ({-0.8246*\dx},{-0.0516*\dy})
	-- ({-0.8212*\dx},{-0.0525*\dy})
	-- ({-0.8177*\dx},{-0.0534*\dy})
	-- ({-0.8142*\dx},{-0.0543*\dy})
	-- ({-0.8106*\dx},{-0.0552*\dy})
	-- ({-0.8069*\dx},{-0.0562*\dy})
	-- ({-0.8032*\dx},{-0.0571*\dy})
	-- ({-0.7994*\dx},{-0.0581*\dy})
	-- ({-0.7955*\dx},{-0.0591*\dy})
	-- ({-0.7916*\dx},{-0.0601*\dy})
	-- ({-0.7876*\dx},{-0.0611*\dy})
	-- ({-0.7835*\dx},{-0.0621*\dy})
	-- ({-0.7794*\dx},{-0.0631*\dy})
	-- ({-0.7752*\dx},{-0.0642*\dy})
	-- ({-0.7709*\dx},{-0.0652*\dy})
	-- ({-0.7666*\dx},{-0.0663*\dy})
	-- ({-0.7621*\dx},{-0.0674*\dy})
	-- ({-0.7577*\dx},{-0.0684*\dy})
	-- ({-0.7531*\dx},{-0.0695*\dy})
	-- ({-0.7485*\dx},{-0.0706*\dy})
	-- ({-0.7438*\dx},{-0.0718*\dy})
	-- ({-0.7390*\dx},{-0.0729*\dy})
	-- ({-0.7341*\dx},{-0.0740*\dy})
	-- ({-0.7292*\dx},{-0.0752*\dy})
	-- ({-0.7241*\dx},{-0.0763*\dy})
	-- ({-0.7190*\dx},{-0.0775*\dy})
	-- ({-0.7139*\dx},{-0.0787*\dy})
	-- ({-0.7086*\dx},{-0.0798*\dy})
	-- ({-0.7033*\dx},{-0.0810*\dy})
	-- ({-0.6978*\dx},{-0.0822*\dy})
	-- ({-0.6923*\dx},{-0.0835*\dy})
	-- ({-0.6868*\dx},{-0.0847*\dy})
	-- ({-0.6811*\dx},{-0.0859*\dy})
	-- ({-0.6753*\dx},{-0.0871*\dy})
	-- ({-0.6695*\dx},{-0.0884*\dy})
	-- ({-0.6636*\dx},{-0.0896*\dy})
	-- ({-0.6576*\dx},{-0.0909*\dy})
	-- ({-0.6515*\dx},{-0.0921*\dy})
	-- ({-0.6453*\dx},{-0.0934*\dy})
	-- ({-0.6390*\dx},{-0.0947*\dy})
	-- ({-0.6327*\dx},{-0.0959*\dy})
	-- ({-0.6262*\dx},{-0.0972*\dy})
	-- ({-0.6197*\dx},{-0.0985*\dy})
	-- ({-0.6131*\dx},{-0.0998*\dy})
	-- ({-0.6064*\dx},{-0.1011*\dy})
	-- ({-0.5996*\dx},{-0.1024*\dy})
	-- ({-0.5927*\dx},{-0.1036*\dy})
	-- ({-0.5857*\dx},{-0.1049*\dy})
	-- ({-0.5787*\dx},{-0.1062*\dy})
	-- ({-0.5715*\dx},{-0.1075*\dy})
	-- ({-0.5643*\dx},{-0.1088*\dy})
	-- ({-0.5569*\dx},{-0.1101*\dy})
	-- ({-0.5495*\dx},{-0.1114*\dy})
	-- ({-0.5420*\dx},{-0.1127*\dy})
	-- ({-0.5344*\dx},{-0.1140*\dy})
	-- ({-0.5267*\dx},{-0.1152*\dy})
	-- ({-0.5190*\dx},{-0.1165*\dy})
	-- ({-0.5111*\dx},{-0.1178*\dy})
	-- ({-0.5031*\dx},{-0.1190*\dy})
	-- ({-0.4951*\dx},{-0.1203*\dy})
	-- ({-0.4870*\dx},{-0.1215*\dy})
	-- ({-0.4788*\dx},{-0.1228*\dy})
	-- ({-0.4705*\dx},{-0.1240*\dy})
	-- ({-0.4621*\dx},{-0.1252*\dy})
	-- ({-0.4536*\dx},{-0.1264*\dy})
	-- ({-0.4450*\dx},{-0.1276*\dy})
	-- ({-0.4364*\dx},{-0.1288*\dy})
	-- ({-0.4277*\dx},{-0.1300*\dy})
	-- ({-0.4189*\dx},{-0.1312*\dy})
	-- ({-0.4100*\dx},{-0.1323*\dy})
	-- ({-0.4010*\dx},{-0.1334*\dy})
	-- ({-0.3920*\dx},{-0.1346*\dy})
	-- ({-0.3829*\dx},{-0.1357*\dy})
	-- ({-0.3737*\dx},{-0.1367*\dy})
	-- ({-0.3644*\dx},{-0.1378*\dy})
	-- ({-0.3550*\dx},{-0.1388*\dy})
	-- ({-0.3456*\dx},{-0.1399*\dy})
	-- ({-0.3361*\dx},{-0.1409*\dy})
	-- ({-0.3266*\dx},{-0.1419*\dy})
	-- ({-0.3169*\dx},{-0.1428*\dy})
	-- ({-0.3072*\dx},{-0.1438*\dy})
	-- ({-0.2975*\dx},{-0.1447*\dy})
	-- ({-0.2876*\dx},{-0.1456*\dy})
	-- ({-0.2778*\dx},{-0.1464*\dy})
	-- ({-0.2678*\dx},{-0.1473*\dy})
	-- ({-0.2578*\dx},{-0.1481*\dy})
	-- ({-0.2477*\dx},{-0.1489*\dy})
	-- ({-0.2376*\dx},{-0.1497*\dy})
	-- ({-0.2274*\dx},{-0.1504*\dy})
	-- ({-0.2172*\dx},{-0.1511*\dy})
	-- ({-0.2069*\dx},{-0.1518*\dy})
	-- ({-0.1966*\dx},{-0.1524*\dy})
	-- ({-0.1862*\dx},{-0.1530*\dy})
	-- ({-0.1758*\dx},{-0.1536*\dy})
	-- ({-0.1654*\dx},{-0.1542*\dy})
	-- ({-0.1549*\dx},{-0.1547*\dy})
	-- ({-0.1444*\dx},{-0.1552*\dy})
	-- ({-0.1338*\dx},{-0.1556*\dy})
	-- ({-0.1232*\dx},{-0.1560*\dy})
	-- ({-0.1126*\dx},{-0.1564*\dy})
	-- ({-0.1020*\dx},{-0.1568*\dy})
	-- ({-0.0913*\dx},{-0.1571*\dy})
	-- ({-0.0806*\dx},{-0.1574*\dy})
	-- ({-0.0699*\dx},{-0.1576*\dy})
	-- ({-0.0592*\dx},{-0.1578*\dy})
	-- ({-0.0484*\dx},{-0.1580*\dy})
	-- ({-0.0377*\dx},{-0.1582*\dy})
	-- ({-0.0269*\dx},{-0.1583*\dy})
	-- ({-0.0162*\dx},{-0.1583*\dy})
	-- ({-0.0054*\dx},{-0.1584*\dy})
	-- ({0.0054*\dx},{-0.1584*\dy})
	-- ({0.0162*\dx},{-0.1583*\dy})
	-- ({0.0269*\dx},{-0.1583*\dy})
	-- ({0.0377*\dx},{-0.1582*\dy})
	-- ({0.0484*\dx},{-0.1580*\dy})
	-- ({0.0592*\dx},{-0.1578*\dy})
	-- ({0.0699*\dx},{-0.1576*\dy})
	-- ({0.0806*\dx},{-0.1574*\dy})
	-- ({0.0913*\dx},{-0.1571*\dy})
	-- ({0.1020*\dx},{-0.1568*\dy})
	-- ({0.1126*\dx},{-0.1564*\dy})
	-- ({0.1232*\dx},{-0.1560*\dy})
	-- ({0.1338*\dx},{-0.1556*\dy})
	-- ({0.1444*\dx},{-0.1552*\dy})
	-- ({0.1549*\dx},{-0.1547*\dy})
	-- ({0.1654*\dx},{-0.1542*\dy})
	-- ({0.1758*\dx},{-0.1536*\dy})
	-- ({0.1862*\dx},{-0.1530*\dy})
	-- ({0.1966*\dx},{-0.1524*\dy})
	-- ({0.2069*\dx},{-0.1518*\dy})
	-- ({0.2172*\dx},{-0.1511*\dy})
	-- ({0.2274*\dx},{-0.1504*\dy})
	-- ({0.2376*\dx},{-0.1497*\dy})
	-- ({0.2477*\dx},{-0.1489*\dy})
	-- ({0.2578*\dx},{-0.1481*\dy})
	-- ({0.2678*\dx},{-0.1473*\dy})
	-- ({0.2778*\dx},{-0.1464*\dy})
	-- ({0.2876*\dx},{-0.1456*\dy})
	-- ({0.2975*\dx},{-0.1447*\dy})
	-- ({0.3072*\dx},{-0.1438*\dy})
	-- ({0.3169*\dx},{-0.1428*\dy})
	-- ({0.3266*\dx},{-0.1419*\dy})
	-- ({0.3361*\dx},{-0.1409*\dy})
	-- ({0.3456*\dx},{-0.1399*\dy})
	-- ({0.3550*\dx},{-0.1388*\dy})
	-- ({0.3644*\dx},{-0.1378*\dy})
	-- ({0.3737*\dx},{-0.1367*\dy})
	-- ({0.3829*\dx},{-0.1357*\dy})
	-- ({0.3920*\dx},{-0.1346*\dy})
	-- ({0.4010*\dx},{-0.1334*\dy})
	-- ({0.4100*\dx},{-0.1323*\dy})
	-- ({0.4189*\dx},{-0.1312*\dy})
	-- ({0.4277*\dx},{-0.1300*\dy})
	-- ({0.4364*\dx},{-0.1288*\dy})
	-- ({0.4450*\dx},{-0.1276*\dy})
	-- ({0.4536*\dx},{-0.1264*\dy})
	-- ({0.4621*\dx},{-0.1252*\dy})
	-- ({0.4705*\dx},{-0.1240*\dy})
	-- ({0.4788*\dx},{-0.1228*\dy})
	-- ({0.4870*\dx},{-0.1215*\dy})
	-- ({0.4951*\dx},{-0.1203*\dy})
	-- ({0.5031*\dx},{-0.1190*\dy})
	-- ({0.5111*\dx},{-0.1178*\dy})
	-- ({0.5190*\dx},{-0.1165*\dy})
	-- ({0.5267*\dx},{-0.1152*\dy})
	-- ({0.5344*\dx},{-0.1140*\dy})
	-- ({0.5420*\dx},{-0.1127*\dy})
	-- ({0.5495*\dx},{-0.1114*\dy})
	-- ({0.5569*\dx},{-0.1101*\dy})
	-- ({0.5643*\dx},{-0.1088*\dy})
	-- ({0.5715*\dx},{-0.1075*\dy})
	-- ({0.5787*\dx},{-0.1062*\dy})
	-- ({0.5857*\dx},{-0.1049*\dy})
	-- ({0.5927*\dx},{-0.1036*\dy})
	-- ({0.5996*\dx},{-0.1024*\dy})
	-- ({0.6064*\dx},{-0.1011*\dy})
	-- ({0.6131*\dx},{-0.0998*\dy})
	-- ({0.6197*\dx},{-0.0985*\dy})
	-- ({0.6262*\dx},{-0.0972*\dy})
	-- ({0.6327*\dx},{-0.0959*\dy})
	-- ({0.6390*\dx},{-0.0947*\dy})
	-- ({0.6453*\dx},{-0.0934*\dy})
	-- ({0.6515*\dx},{-0.0921*\dy})
	-- ({0.6576*\dx},{-0.0909*\dy})
	-- ({0.6636*\dx},{-0.0896*\dy})
	-- ({0.6695*\dx},{-0.0884*\dy})
	-- ({0.6753*\dx},{-0.0871*\dy})
	-- ({0.6811*\dx},{-0.0859*\dy})
	-- ({0.6868*\dx},{-0.0847*\dy})
	-- ({0.6923*\dx},{-0.0835*\dy})
	-- ({0.6978*\dx},{-0.0822*\dy})
	-- ({0.7033*\dx},{-0.0810*\dy})
	-- ({0.7086*\dx},{-0.0798*\dy})
	-- ({0.7139*\dx},{-0.0787*\dy})
	-- ({0.7190*\dx},{-0.0775*\dy})
	-- ({0.7241*\dx},{-0.0763*\dy})
	-- ({0.7292*\dx},{-0.0752*\dy})
	-- ({0.7341*\dx},{-0.0740*\dy})
	-- ({0.7390*\dx},{-0.0729*\dy})
	-- ({0.7438*\dx},{-0.0718*\dy})
	-- ({0.7485*\dx},{-0.0706*\dy})
	-- ({0.7531*\dx},{-0.0695*\dy})
	-- ({0.7577*\dx},{-0.0684*\dy})
	-- ({0.7621*\dx},{-0.0674*\dy})
	-- ({0.7666*\dx},{-0.0663*\dy})
	-- ({0.7709*\dx},{-0.0652*\dy})
	-- ({0.7752*\dx},{-0.0642*\dy})
	-- ({0.7794*\dx},{-0.0631*\dy})
	-- ({0.7835*\dx},{-0.0621*\dy})
	-- ({0.7876*\dx},{-0.0611*\dy})
	-- ({0.7916*\dx},{-0.0601*\dy})
	-- ({0.7955*\dx},{-0.0591*\dy})
	-- ({0.7994*\dx},{-0.0581*\dy})
	-- ({0.8032*\dx},{-0.0571*\dy})
	-- ({0.8069*\dx},{-0.0562*\dy})
	-- ({0.8106*\dx},{-0.0552*\dy})
	-- ({0.8142*\dx},{-0.0543*\dy})
	-- ({0.8177*\dx},{-0.0534*\dy})
	-- ({0.8212*\dx},{-0.0525*\dy})
	-- ({0.8246*\dx},{-0.0516*\dy})
	-- ({0.8280*\dx},{-0.0507*\dy})
	-- ({0.8313*\dx},{-0.0498*\dy})
	-- ({0.8345*\dx},{-0.0489*\dy})
	-- ({0.8377*\dx},{-0.0481*\dy})
	-- ({0.8408*\dx},{-0.0472*\dy})
	-- ({0.8439*\dx},{-0.0464*\dy})
	-- ({0.8469*\dx},{-0.0456*\dy})
	-- ({0.8499*\dx},{-0.0448*\dy})
	-- ({0.8528*\dx},{-0.0440*\dy})
	-- ({0.8557*\dx},{-0.0432*\dy})
	-- ({0.8585*\dx},{-0.0424*\dy})
	-- ({0.8612*\dx},{-0.0417*\dy})
	-- ({0.8639*\dx},{-0.0409*\dy})
	-- ({0.8666*\dx},{-0.0402*\dy})
	-- ({0.8692*\dx},{-0.0395*\dy})
	-- ({0.8718*\dx},{-0.0388*\dy})
	-- ({0.8743*\dx},{-0.0380*\dy})
	-- ({0.8767*\dx},{-0.0374*\dy})
	-- ({0.8792*\dx},{-0.0367*\dy})
	-- ({0.8815*\dx},{-0.0360*\dy})
	-- ({0.8839*\dx},{-0.0353*\dy})
	-- ({0.8862*\dx},{-0.0347*\dy})
	-- ({0.8884*\dx},{-0.0340*\dy})
	-- ({0.8906*\dx},{-0.0334*\dy})
	-- ({0.8928*\dx},{-0.0328*\dy})
	-- ({0.8949*\dx},{-0.0322*\dy})
	-- ({0.8970*\dx},{-0.0316*\dy})
	-- ({0.8990*\dx},{-0.0310*\dy})
	-- ({0.9010*\dx},{-0.0304*\dy})
	-- ({0.9030*\dx},{-0.0298*\dy})
	-- ({0.9049*\dx},{-0.0293*\dy})
	-- ({0.9068*\dx},{-0.0287*\dy})
	-- ({0.9087*\dx},{-0.0282*\dy})
	-- ({0.9105*\dx},{-0.0277*\dy})
	-- ({0.9123*\dx},{-0.0271*\dy})
	-- ({0.9141*\dx},{-0.0266*\dy})
	-- ({0.9158*\dx},{-0.0261*\dy})
	-- ({0.9175*\dx},{-0.0256*\dy})
	-- ({0.9191*\dx},{-0.0251*\dy})
	-- ({0.9207*\dx},{-0.0246*\dy})
}
\def\rechtewinkel{
% u = -1.413717, v = -1.413717
\rechterwinkel{-0.8929}{0.0328}{16.0167}
% u = -1.247397, v = -1.413717
\rechterwinkel{-0.8534}{0.0438}{15.2913}
% u = -1.081077, v = -1.413717
\rechterwinkel{-0.8009}{0.0577}{14.3283}
% u = -0.914758, v = -1.413717
\rechterwinkel{-0.7319}{0.0745}{13.0725}
% u = -0.748438, v = -1.413717
\rechterwinkel{-0.6436}{0.0937}{11.4722}
% u = -0.582119, v = -1.413717
\rechterwinkel{-0.5337}{0.1141}{9.4923}
% u = -0.415799, v = -1.413717
\rechterwinkel{-0.4017}{0.1334}{7.1305}
% u = -0.249479, v = -1.413717
\rechterwinkel{-0.2502}{0.1487}{4.4341}
% u = -0.083160, v = -1.413717
\rechterwinkel{-0.0850}{0.1573}{1.5058}
% u = 0.083160, v = -1.413717
\rechterwinkel{0.0850}{0.1573}{-1.5058}
% u = 0.249479, v = -1.413717
\rechterwinkel{0.2502}{0.1487}{-4.4341}
% u = 0.415799, v = -1.413717
\rechterwinkel{0.4017}{0.1334}{-7.1305}
% u = 0.582119, v = -1.413717
\rechterwinkel{0.5337}{0.1141}{-9.4923}
% u = 0.748438, v = -1.413717
\rechterwinkel{0.6436}{0.0937}{-11.4722}
% u = 0.914758, v = -1.413717
\rechterwinkel{0.7319}{0.0745}{-13.0725}
% u = 1.081077, v = -1.413717
\rechterwinkel{0.8009}{0.0577}{-14.3283}
% u = 1.247397, v = -1.413717
\rechterwinkel{0.8534}{0.0438}{-15.2913}
% u = 1.413717, v = -1.413717
\rechterwinkel{0.8929}{0.0328}{-16.0167}
% u = -1.413717, v = -1.247397
\rechterwinkel{-0.9076}{0.0649}{33.1588}
% u = -1.247397, v = -1.247397
\rechterwinkel{-0.8724}{0.0874}{31.7166}
% u = -1.081077, v = -1.247397
\rechterwinkel{-0.8244}{0.1159}{29.7901}
% u = -0.914758, v = -1.247397
\rechterwinkel{-0.7600}{0.1509}{27.2581}
% u = -0.748438, v = -1.247397
\rechterwinkel{-0.6750}{0.1917}{24.0012}
% u = -0.582119, v = -1.247397
\rechterwinkel{-0.5656}{0.2358}{19.9298}
% u = -0.415799, v = -1.247397
\rechterwinkel{-0.4301}{0.2785}{15.0222}
% u = -0.249479, v = -1.247397
\rechterwinkel{-0.2701}{0.3130}{9.3669}
% u = -0.083160, v = -1.247397
\rechterwinkel{-0.0922}{0.3326}{3.1858}
% u = 0.083160, v = -1.247397
\rechterwinkel{0.0922}{0.3326}{-3.1858}
% u = 0.249479, v = -1.247397
\rechterwinkel{0.2701}{0.3130}{-9.3669}
% u = 0.415799, v = -1.247397
\rechterwinkel{0.4301}{0.2785}{-15.0222}
% u = 0.582119, v = -1.247397
\rechterwinkel{0.5656}{0.2358}{-19.9298}
% u = 0.748438, v = -1.247397
\rechterwinkel{0.6750}{0.1917}{-24.0012}
% u = 0.914758, v = -1.247397
\rechterwinkel{0.7600}{0.1509}{-27.2581}
% u = 1.081077, v = -1.247397
\rechterwinkel{0.8244}{0.1159}{-29.7901}
% u = 1.247397, v = -1.247397
\rechterwinkel{0.8724}{0.0874}{-31.7166}
% u = 1.413717, v = -1.247397
\rechterwinkel{0.9076}{0.0649}{-33.1588}
% u = -1.413717, v = -1.081077
\rechterwinkel{-0.9318}{0.0919}{50.6730}
% u = -1.247397, v = -1.081077
\rechterwinkel{-0.9039}{0.1247}{48.6239}
% u = -1.081077, v = -1.081077
\rechterwinkel{-0.8644}{0.1674}{45.8576}
% u = -0.914758, v = -1.081077
\rechterwinkel{-0.8087}{0.2212}{42.1729}
% u = -0.748438, v = -1.081077
\rechterwinkel{-0.7309}{0.2859}{37.3561}
% u = -0.582119, v = -1.081077
\rechterwinkel{-0.6244}{0.3586}{31.2223}
% u = -0.415799, v = -1.081077
\rechterwinkel{-0.4839}{0.4316}{23.6850}
% u = -0.249479, v = -1.081077
\rechterwinkel{-0.3086}{0.4928}{14.8462}
% u = -0.083160, v = -1.081077
\rechterwinkel{-0.1063}{0.5283}{5.0645}
% u = 0.083160, v = -1.081077
\rechterwinkel{0.1063}{0.5283}{-5.0645}
% u = 0.249479, v = -1.081077
\rechterwinkel{0.3086}{0.4928}{-14.8462}
% u = 0.415799, v = -1.081077
\rechterwinkel{0.4839}{0.4316}{-23.6850}
% u = 0.582119, v = -1.081077
\rechterwinkel{0.6244}{0.3586}{-31.2223}
% u = 0.748438, v = -1.081077
\rechterwinkel{0.7309}{0.2859}{-37.3561}
% u = 0.914758, v = -1.081077
\rechterwinkel{0.8087}{0.2212}{-42.1729}
% u = 1.081077, v = -1.081077
\rechterwinkel{0.8644}{0.1674}{-45.8576}
% u = 1.247397, v = -1.081077
\rechterwinkel{0.9039}{0.1247}{-48.6239}
% u = 1.413717, v = -1.081077
\rechterwinkel{0.9318}{0.0919}{-50.6730}
% u = -1.413717, v = -0.914758
\rechterwinkel{-0.9639}{0.1107}{68.7269}
% u = -1.247397, v = -0.914758
\rechterwinkel{-0.9468}{0.1521}{66.2429}
% u = -1.081077, v = -0.914758
\rechterwinkel{-0.9204}{0.2075}{62.8410}
% u = -0.914758, v = -0.914758
\rechterwinkel{-0.8794}{0.2801}{58.2236}
% u = -0.748438, v = -0.914758
\rechterwinkel{-0.8156}{0.3716}{52.0437}
% u = -0.582119, v = -0.914758
\rechterwinkel{-0.7179}{0.4801}{43.9501}
% u = -0.415799, v = -0.914758
\rechterwinkel{-0.5739}{0.5960}{33.6946}
% u = -0.249479, v = -0.914758
\rechterwinkel{-0.3760}{0.6990}{21.3119}
% u = -0.083160, v = -0.914758
\rechterwinkel{-0.1316}{0.7614}{7.3087}
% u = 0.083160, v = -0.914758
\rechterwinkel{0.1316}{0.7614}{-7.3087}
% u = 0.249479, v = -0.914758
\rechterwinkel{0.3760}{0.6990}{-21.3119}
% u = 0.415799, v = -0.914758
\rechterwinkel{0.5739}{0.5960}{-33.6946}
% u = 0.582119, v = -0.914758
\rechterwinkel{0.7179}{0.4801}{-43.9501}
% u = 0.748438, v = -0.914758
\rechterwinkel{0.8156}{0.3716}{-52.0437}
% u = 0.914758, v = -0.914758
\rechterwinkel{0.8794}{0.2801}{-58.2236}
% u = 1.081077, v = -0.914758
\rechterwinkel{0.9204}{0.2075}{-62.8410}
% u = 1.247397, v = -0.914758
\rechterwinkel{0.9468}{0.1521}{-66.2429}
% u = 1.413717, v = -0.914758
\rechterwinkel{0.9639}{0.1107}{-68.7269}
% u = -1.413717, v = -0.748438
\rechterwinkel{-1.0017}{0.1186}{87.4523}
% u = -1.247397, v = -0.748438
\rechterwinkel{-0.9986}{0.1655}{84.7695}
% u = -1.081077, v = -0.748438
\rechterwinkel{-0.9905}{0.2304}{81.0305}
% u = -0.914758, v = -0.748438
\rechterwinkel{-0.9722}{0.3195}{75.8344}
% u = -0.748438, v = -0.748438
\rechterwinkel{-0.9340}{0.4390}{68.6599}
% u = -0.582119, v = -0.748438
\rechterwinkel{-0.8585}{0.5922}{58.8859}
% u = -0.415799, v = -0.748438
\rechterwinkel{-0.7203}{0.7717}{45.9145}
% u = -0.249479, v = -0.748438
\rechterwinkel{-0.4936}{0.9469}{29.4916}
% u = -0.083160, v = -0.748438
\rechterwinkel{-0.1778}{1.0609}{10.2105}
% u = 0.083160, v = -0.748438
\rechterwinkel{0.1778}{1.0609}{-10.2105}
% u = 0.249479, v = -0.748438
\rechterwinkel{0.4936}{0.9469}{-29.4916}
% u = 0.415799, v = -0.748438
\rechterwinkel{0.7203}{0.7717}{-45.9145}
% u = 0.582119, v = -0.748438
\rechterwinkel{0.8585}{0.5922}{-58.8859}
% u = 0.748438, v = -0.748438
\rechterwinkel{0.9340}{0.4390}{-68.6599}
% u = 0.914758, v = -0.748438
\rechterwinkel{0.9722}{0.3195}{-75.8344}
% u = 1.081077, v = -0.748438
\rechterwinkel{0.9905}{0.2304}{-81.0305}
% u = 1.247397, v = -0.748438
\rechterwinkel{0.9986}{0.1655}{-84.7695}
% u = 1.413717, v = -0.748438
\rechterwinkel{1.0017}{0.1186}{-87.4523}
% u = -1.413717, v = -0.582119
\rechterwinkel{-1.0416}{0.1136}{106.9244}
% u = -1.247397, v = -0.582119
\rechterwinkel{-1.0548}{0.1610}{104.3349}
% u = -1.081077, v = -0.582119
\rechterwinkel{-1.0700}{0.2292}{100.6562}
% u = -0.914758, v = -0.582119
\rechterwinkel{-1.0839}{0.3280}{95.4043}
% u = -0.748438, v = -0.582119
\rechterwinkel{-1.0880}{0.4709}{87.8737}
% u = -0.582119, v = -0.582119
\rechterwinkel{-1.0611}{0.6742}{77.0690}
% u = -0.415799, v = -0.582119
\rechterwinkel{-0.9589}{0.9462}{61.7308}
% u = -0.249479, v = -0.582119
\rechterwinkel{-0.7106}{1.2554}{40.7459}
% u = -0.083160, v = -0.582119
\rechterwinkel{-0.2702}{1.4852}{14.3689}
% u = 0.083160, v = -0.582119
\rechterwinkel{0.2702}{1.4852}{-14.3689}
% u = 0.249479, v = -0.582119
\rechterwinkel{0.7106}{1.2554}{-40.7459}
% u = 0.415799, v = -0.582119
\rechterwinkel{0.9589}{0.9462}{-61.7308}
% u = 0.582119, v = -0.582119
\rechterwinkel{1.0611}{0.6742}{-77.0690}
% u = 0.748438, v = -0.582119
\rechterwinkel{1.0880}{0.4709}{-87.8737}
% u = 0.914758, v = -0.582119
\rechterwinkel{1.0839}{0.3280}{-95.4043}
% u = 1.081077, v = -0.582119
\rechterwinkel{1.0700}{0.2292}{-100.6562}
% u = 1.247397, v = -0.582119
\rechterwinkel{1.0548}{0.1610}{-104.3349}
% u = 1.413717, v = -0.582119
\rechterwinkel{1.0416}{0.1136}{-106.9244}
% u = -1.413717, v = -0.415799
\rechterwinkel{-1.0787}{0.0947}{127.1382}
% u = -1.247397, v = -0.415799
\rechterwinkel{-1.1089}{0.1362}{124.9643}
% u = -1.081077, v = -0.415799
\rechterwinkel{-1.1498}{0.1982}{121.8185}
% u = -0.914758, v = -0.415799
\rechterwinkel{-1.2035}{0.2930}{117.2031}
% u = -0.748438, v = -0.415799
\rechterwinkel{-1.2690}{0.4420}{110.3085}
% u = -0.582119, v = -0.415799
\rechterwinkel{-1.3335}{0.6817}{99.7843}
% u = -0.415799, v = -0.415799
\rechterwinkel{-1.3442}{1.0672}{83.3978}
% u = -0.249479, v = -0.415799
\rechterwinkel{-1.1467}{1.6300}{57.9360}
% u = -0.083160, v = -0.415799
\rechterwinkel{-0.4912}{2.1725}{21.2842}
% u = 0.083160, v = -0.415799
\rechterwinkel{0.4912}{2.1725}{-21.2843}
% u = 0.249479, v = -0.415799
\rechterwinkel{1.1467}{1.6300}{-57.9360}
% u = 0.415799, v = -0.415799
\rechterwinkel{1.3442}{1.0672}{-83.3978}
% u = 0.582119, v = -0.415799
\rechterwinkel{1.3335}{0.6817}{-99.7843}
% u = 0.748438, v = -0.415799
\rechterwinkel{1.2690}{0.4420}{-110.3085}
% u = 0.914758, v = -0.415799
\rechterwinkel{1.2035}{0.2930}{-117.2031}
% u = 1.081077, v = -0.415799
\rechterwinkel{1.1498}{0.1982}{-121.8185}
% u = 1.247397, v = -0.415799
\rechterwinkel{1.1089}{0.1362}{-124.9643}
% u = 1.413717, v = -0.415799
\rechterwinkel{1.0787}{0.0947}{-127.1383}
% u = -1.413717, v = -0.249479
\rechterwinkel{-1.1077}{0.0629}{147.9908}
% u = -1.247397, v = -0.249479
\rechterwinkel{-1.1523}{0.0916}{146.5368}
% u = -1.081077, v = -0.249479
\rechterwinkel{-1.2165}{0.1358}{144.4012}
% u = -0.914758, v = -0.249479
\rechterwinkel{-1.3096}{0.2065}{141.1950}
% u = -0.748438, v = -0.249479
\rechterwinkel{-1.4458}{0.3260}{136.2258}
% u = -0.582119, v = -0.249479
\rechterwinkel{-1.6433}{0.5439}{128.1562}
% u = -0.415799, v = -0.249479
\rechterwinkel{-1.9071}{0.9804}{114.1394}
% u = -0.249479, v = -0.249479
\rechterwinkel{-2.0880}{1.9217}{87.6227}
% u = -0.083160, v = -0.249479
\rechterwinkel{-1.2306}{3.5241}{36.0745}
% u = 0.083160, v = -0.249479
\rechterwinkel{1.2306}{3.5241}{-36.0745}
% u = 0.249479, v = -0.249479
\rechterwinkel{2.0880}{1.9217}{-87.6227}
% u = 0.415799, v = -0.249479
\rechterwinkel{1.9071}{0.9804}{-114.1394}
% u = 0.582119, v = -0.249479
\rechterwinkel{1.6433}{0.5439}{-128.1562}
% u = 0.748438, v = -0.249479
\rechterwinkel{1.4458}{0.3260}{-136.2258}
% u = 0.914758, v = -0.249479
\rechterwinkel{1.3096}{0.2065}{-141.1950}
% u = 1.081077, v = -0.249479
\rechterwinkel{1.2165}{0.1358}{-144.4012}
% u = 1.247397, v = -0.249479
\rechterwinkel{1.1523}{0.0916}{-146.5368}
% u = 1.413717, v = -0.249479
\rechterwinkel{1.1077}{0.0629}{-147.9908}
% u = -1.413717, v = -0.083160
\rechterwinkel{-1.1237}{0.0221}{169.2787}
% u = -1.247397, v = -0.083160
\rechterwinkel{-1.1767}{0.0324}{168.7667}
% u = -1.081077, v = -0.083160
\rechterwinkel{-1.2550}{0.0485}{168.0084}
% u = -0.914758, v = -0.083160
\rechterwinkel{-1.3737}{0.0749}{166.8546}
% u = -0.748438, v = -0.083160
\rechterwinkel{-1.5607}{0.1218}{165.0256}
% u = -0.582119, v = -0.083160
\rechterwinkel{-1.8735}{0.2146}{161.9304}
% u = -0.415799, v = -0.083160
\rechterwinkel{-2.4497}{0.4357}{156.0738}
% u = -0.249479, v = -0.083160
\rechterwinkel{-3.6904}{1.1751}{142.3405}
% u = -0.083160, v = -0.083160
\rechterwinkel{-6.0403}{5.9848}{89.7358}
% u = 0.083160, v = -0.083160
\rechterwinkel{6.0403}{5.9848}{-89.7359}
% u = 0.249479, v = -0.083160
\rechterwinkel{3.6904}{1.1751}{-142.3406}
% u = 0.415799, v = -0.083160
\rechterwinkel{2.4497}{0.4357}{-156.0738}
% u = 0.582119, v = -0.083160
\rechterwinkel{1.8735}{0.2146}{-161.9304}
% u = 0.748438, v = -0.083160
\rechterwinkel{1.5607}{0.1218}{-165.0256}
% u = 0.914758, v = -0.083160
\rechterwinkel{1.3737}{0.0749}{-166.8546}
% u = 1.081077, v = -0.083160
\rechterwinkel{1.2550}{0.0485}{-168.0084}
% u = 1.247397, v = -0.083160
\rechterwinkel{1.1767}{0.0324}{-168.7667}
% u = 1.413717, v = -0.083160
\rechterwinkel{1.1237}{0.0221}{-169.2787}
% u = -1.413717, v = 0.083160
\rechterwinkel{-1.1237}{-0.0221}{-169.2787}
% u = -1.247397, v = 0.083160
\rechterwinkel{-1.1767}{-0.0324}{-168.7667}
% u = -1.081077, v = 0.083160
\rechterwinkel{-1.2550}{-0.0485}{-168.0084}
% u = -0.914758, v = 0.083160
\rechterwinkel{-1.3737}{-0.0749}{-166.8546}
% u = -0.748438, v = 0.083160
\rechterwinkel{-1.5607}{-0.1218}{-165.0256}
% u = -0.582119, v = 0.083160
\rechterwinkel{-1.8735}{-0.2146}{-161.9304}
% u = -0.415799, v = 0.083160
\rechterwinkel{-2.4497}{-0.4357}{-156.0738}
% u = -0.249479, v = 0.083160
\rechterwinkel{-3.6904}{-1.1751}{-142.3405}
% u = -0.083160, v = 0.083160
\rechterwinkel{-6.0403}{-5.9848}{-89.7358}
% u = 0.083160, v = 0.083160
\rechterwinkel{6.0403}{-5.9848}{89.7359}
% u = 0.249479, v = 0.083160
\rechterwinkel{3.6904}{-1.1751}{142.3406}
% u = 0.415799, v = 0.083160
\rechterwinkel{2.4497}{-0.4357}{156.0738}
% u = 0.582119, v = 0.083160
\rechterwinkel{1.8735}{-0.2146}{161.9304}
% u = 0.748438, v = 0.083160
\rechterwinkel{1.5607}{-0.1218}{165.0256}
% u = 0.914758, v = 0.083160
\rechterwinkel{1.3737}{-0.0749}{166.8546}
% u = 1.081077, v = 0.083160
\rechterwinkel{1.2550}{-0.0485}{168.0084}
% u = 1.247397, v = 0.083160
\rechterwinkel{1.1767}{-0.0324}{168.7667}
% u = 1.413717, v = 0.083160
\rechterwinkel{1.1237}{-0.0221}{169.2787}
% u = -1.413717, v = 0.249479
\rechterwinkel{-1.1077}{-0.0629}{-147.9908}
% u = -1.247397, v = 0.249479
\rechterwinkel{-1.1523}{-0.0916}{-146.5368}
% u = -1.081077, v = 0.249479
\rechterwinkel{-1.2165}{-0.1358}{-144.4012}
% u = -0.914758, v = 0.249479
\rechterwinkel{-1.3096}{-0.2065}{-141.1950}
% u = -0.748438, v = 0.249479
\rechterwinkel{-1.4458}{-0.3260}{-136.2258}
% u = -0.582119, v = 0.249479
\rechterwinkel{-1.6433}{-0.5439}{-128.1562}
% u = -0.415799, v = 0.249479
\rechterwinkel{-1.9071}{-0.9804}{-114.1394}
% u = -0.249479, v = 0.249479
\rechterwinkel{-2.0880}{-1.9217}{-87.6227}
% u = -0.083160, v = 0.249479
\rechterwinkel{-1.2306}{-3.5241}{-36.0745}
% u = 0.083160, v = 0.249479
\rechterwinkel{1.2306}{-3.5241}{36.0745}
% u = 0.249479, v = 0.249479
\rechterwinkel{2.0880}{-1.9217}{87.6227}
% u = 0.415799, v = 0.249479
\rechterwinkel{1.9071}{-0.9804}{114.1394}
% u = 0.582119, v = 0.249479
\rechterwinkel{1.6433}{-0.5439}{128.1562}
% u = 0.748438, v = 0.249479
\rechterwinkel{1.4458}{-0.3260}{136.2258}
% u = 0.914758, v = 0.249479
\rechterwinkel{1.3096}{-0.2065}{141.1950}
% u = 1.081077, v = 0.249479
\rechterwinkel{1.2165}{-0.1358}{144.4012}
% u = 1.247397, v = 0.249479
\rechterwinkel{1.1523}{-0.0916}{146.5368}
% u = 1.413717, v = 0.249479
\rechterwinkel{1.1077}{-0.0629}{147.9908}
% u = -1.413717, v = 0.415799
\rechterwinkel{-1.0787}{-0.0947}{-127.1382}
% u = -1.247397, v = 0.415799
\rechterwinkel{-1.1089}{-0.1362}{-124.9643}
% u = -1.081077, v = 0.415799
\rechterwinkel{-1.1498}{-0.1982}{-121.8185}
% u = -0.914758, v = 0.415799
\rechterwinkel{-1.2035}{-0.2930}{-117.2031}
% u = -0.748438, v = 0.415799
\rechterwinkel{-1.2690}{-0.4420}{-110.3085}
% u = -0.582119, v = 0.415799
\rechterwinkel{-1.3335}{-0.6817}{-99.7843}
% u = -0.415799, v = 0.415799
\rechterwinkel{-1.3442}{-1.0672}{-83.3978}
% u = -0.249479, v = 0.415799
\rechterwinkel{-1.1467}{-1.6300}{-57.9360}
% u = -0.083160, v = 0.415799
\rechterwinkel{-0.4912}{-2.1725}{-21.2842}
% u = 0.083160, v = 0.415799
\rechterwinkel{0.4912}{-2.1725}{21.2843}
% u = 0.249479, v = 0.415799
\rechterwinkel{1.1467}{-1.6300}{57.9360}
% u = 0.415799, v = 0.415799
\rechterwinkel{1.3442}{-1.0672}{83.3978}
% u = 0.582119, v = 0.415799
\rechterwinkel{1.3335}{-0.6817}{99.7843}
% u = 0.748438, v = 0.415799
\rechterwinkel{1.2690}{-0.4420}{110.3085}
% u = 0.914758, v = 0.415799
\rechterwinkel{1.2035}{-0.2930}{117.2031}
% u = 1.081077, v = 0.415799
\rechterwinkel{1.1498}{-0.1982}{121.8185}
% u = 1.247397, v = 0.415799
\rechterwinkel{1.1089}{-0.1362}{124.9643}
% u = 1.413717, v = 0.415799
\rechterwinkel{1.0787}{-0.0947}{127.1383}
% u = -1.413717, v = 0.582119
\rechterwinkel{-1.0416}{-0.1136}{-106.9244}
% u = -1.247397, v = 0.582119
\rechterwinkel{-1.0548}{-0.1610}{-104.3349}
% u = -1.081077, v = 0.582119
\rechterwinkel{-1.0700}{-0.2292}{-100.6562}
% u = -0.914758, v = 0.582119
\rechterwinkel{-1.0839}{-0.3280}{-95.4043}
% u = -0.748438, v = 0.582119
\rechterwinkel{-1.0880}{-0.4709}{-87.8737}
% u = -0.582119, v = 0.582119
\rechterwinkel{-1.0611}{-0.6742}{-77.0690}
% u = -0.415799, v = 0.582119
\rechterwinkel{-0.9589}{-0.9462}{-61.7308}
% u = -0.249479, v = 0.582119
\rechterwinkel{-0.7106}{-1.2554}{-40.7459}
% u = -0.083160, v = 0.582119
\rechterwinkel{-0.2702}{-1.4852}{-14.3689}
% u = 0.083160, v = 0.582119
\rechterwinkel{0.2702}{-1.4852}{14.3689}
% u = 0.249479, v = 0.582119
\rechterwinkel{0.7106}{-1.2554}{40.7459}
% u = 0.415799, v = 0.582119
\rechterwinkel{0.9589}{-0.9462}{61.7308}
% u = 0.582119, v = 0.582119
\rechterwinkel{1.0611}{-0.6742}{77.0690}
% u = 0.748438, v = 0.582119
\rechterwinkel{1.0880}{-0.4709}{87.8737}
% u = 0.914758, v = 0.582119
\rechterwinkel{1.0839}{-0.3280}{95.4043}
% u = 1.081077, v = 0.582119
\rechterwinkel{1.0700}{-0.2292}{100.6562}
% u = 1.247397, v = 0.582119
\rechterwinkel{1.0548}{-0.1610}{104.3349}
% u = 1.413717, v = 0.582119
\rechterwinkel{1.0416}{-0.1136}{106.9244}
% u = -1.413717, v = 0.748438
\rechterwinkel{-1.0017}{-0.1186}{-87.4523}
% u = -1.247397, v = 0.748438
\rechterwinkel{-0.9986}{-0.1655}{-84.7695}
% u = -1.081077, v = 0.748438
\rechterwinkel{-0.9905}{-0.2304}{-81.0305}
% u = -0.914758, v = 0.748438
\rechterwinkel{-0.9722}{-0.3195}{-75.8344}
% u = -0.748438, v = 0.748438
\rechterwinkel{-0.9340}{-0.4390}{-68.6599}
% u = -0.582119, v = 0.748438
\rechterwinkel{-0.8585}{-0.5922}{-58.8859}
% u = -0.415799, v = 0.748438
\rechterwinkel{-0.7203}{-0.7717}{-45.9145}
% u = -0.249479, v = 0.748438
\rechterwinkel{-0.4936}{-0.9469}{-29.4916}
% u = -0.083160, v = 0.748438
\rechterwinkel{-0.1778}{-1.0609}{-10.2105}
% u = 0.083160, v = 0.748438
\rechterwinkel{0.1778}{-1.0609}{10.2105}
% u = 0.249479, v = 0.748438
\rechterwinkel{0.4936}{-0.9469}{29.4916}
% u = 0.415799, v = 0.748438
\rechterwinkel{0.7203}{-0.7717}{45.9145}
% u = 0.582119, v = 0.748438
\rechterwinkel{0.8585}{-0.5922}{58.8859}
% u = 0.748438, v = 0.748438
\rechterwinkel{0.9340}{-0.4390}{68.6599}
% u = 0.914758, v = 0.748438
\rechterwinkel{0.9722}{-0.3195}{75.8344}
% u = 1.081077, v = 0.748438
\rechterwinkel{0.9905}{-0.2304}{81.0305}
% u = 1.247397, v = 0.748438
\rechterwinkel{0.9986}{-0.1655}{84.7695}
% u = 1.413717, v = 0.748438
\rechterwinkel{1.0017}{-0.1186}{87.4523}
% u = -1.413717, v = 0.914758
\rechterwinkel{-0.9639}{-0.1107}{-68.7269}
% u = -1.247397, v = 0.914758
\rechterwinkel{-0.9468}{-0.1521}{-66.2429}
% u = -1.081077, v = 0.914758
\rechterwinkel{-0.9204}{-0.2075}{-62.8410}
% u = -0.914758, v = 0.914758
\rechterwinkel{-0.8794}{-0.2801}{-58.2236}
% u = -0.748438, v = 0.914758
\rechterwinkel{-0.8156}{-0.3716}{-52.0437}
% u = -0.582119, v = 0.914758
\rechterwinkel{-0.7179}{-0.4801}{-43.9501}
% u = -0.415799, v = 0.914758
\rechterwinkel{-0.5739}{-0.5960}{-33.6946}
% u = -0.249479, v = 0.914758
\rechterwinkel{-0.3760}{-0.6990}{-21.3119}
% u = -0.083160, v = 0.914758
\rechterwinkel{-0.1316}{-0.7614}{-7.3087}
% u = 0.083160, v = 0.914758
\rechterwinkel{0.1316}{-0.7614}{7.3087}
% u = 0.249479, v = 0.914758
\rechterwinkel{0.3760}{-0.6990}{21.3119}
% u = 0.415799, v = 0.914758
\rechterwinkel{0.5739}{-0.5960}{33.6946}
% u = 0.582119, v = 0.914758
\rechterwinkel{0.7179}{-0.4801}{43.9501}
% u = 0.748438, v = 0.914758
\rechterwinkel{0.8156}{-0.3716}{52.0437}
% u = 0.914758, v = 0.914758
\rechterwinkel{0.8794}{-0.2801}{58.2236}
% u = 1.081077, v = 0.914758
\rechterwinkel{0.9204}{-0.2075}{62.8410}
% u = 1.247397, v = 0.914758
\rechterwinkel{0.9468}{-0.1521}{66.2429}
% u = 1.413717, v = 0.914758
\rechterwinkel{0.9639}{-0.1107}{68.7269}
% u = -1.413717, v = 1.081077
\rechterwinkel{-0.9318}{-0.0919}{-50.6730}
% u = -1.247397, v = 1.081077
\rechterwinkel{-0.9039}{-0.1247}{-48.6239}
% u = -1.081077, v = 1.081077
\rechterwinkel{-0.8644}{-0.1674}{-45.8576}
% u = -0.914758, v = 1.081077
\rechterwinkel{-0.8087}{-0.2212}{-42.1729}
% u = -0.748438, v = 1.081077
\rechterwinkel{-0.7309}{-0.2859}{-37.3561}
% u = -0.582119, v = 1.081077
\rechterwinkel{-0.6244}{-0.3586}{-31.2223}
% u = -0.415799, v = 1.081077
\rechterwinkel{-0.4839}{-0.4316}{-23.6850}
% u = -0.249479, v = 1.081077
\rechterwinkel{-0.3086}{-0.4928}{-14.8462}
% u = -0.083160, v = 1.081077
\rechterwinkel{-0.1063}{-0.5283}{-5.0645}
% u = 0.083160, v = 1.081077
\rechterwinkel{0.1063}{-0.5283}{5.0645}
% u = 0.249479, v = 1.081077
\rechterwinkel{0.3086}{-0.4928}{14.8462}
% u = 0.415799, v = 1.081077
\rechterwinkel{0.4839}{-0.4316}{23.6850}
% u = 0.582119, v = 1.081077
\rechterwinkel{0.6244}{-0.3586}{31.2223}
% u = 0.748438, v = 1.081077
\rechterwinkel{0.7309}{-0.2859}{37.3561}
% u = 0.914758, v = 1.081077
\rechterwinkel{0.8087}{-0.2212}{42.1729}
% u = 1.081077, v = 1.081077
\rechterwinkel{0.8644}{-0.1674}{45.8576}
% u = 1.247397, v = 1.081077
\rechterwinkel{0.9039}{-0.1247}{48.6239}
% u = 1.413717, v = 1.081077
\rechterwinkel{0.9318}{-0.0919}{50.6730}
% u = -1.413717, v = 1.247397
\rechterwinkel{-0.9076}{-0.0649}{-33.1588}
% u = -1.247397, v = 1.247397
\rechterwinkel{-0.8724}{-0.0874}{-31.7166}
% u = -1.081077, v = 1.247397
\rechterwinkel{-0.8244}{-0.1159}{-29.7901}
% u = -0.914758, v = 1.247397
\rechterwinkel{-0.7600}{-0.1509}{-27.2581}
% u = -0.748438, v = 1.247397
\rechterwinkel{-0.6750}{-0.1917}{-24.0012}
% u = -0.582119, v = 1.247397
\rechterwinkel{-0.5656}{-0.2358}{-19.9298}
% u = -0.415799, v = 1.247397
\rechterwinkel{-0.4301}{-0.2785}{-15.0222}
% u = -0.249479, v = 1.247397
\rechterwinkel{-0.2701}{-0.3130}{-9.3669}
% u = -0.083160, v = 1.247397
\rechterwinkel{-0.0922}{-0.3326}{-3.1858}
% u = 0.083160, v = 1.247397
\rechterwinkel{0.0922}{-0.3326}{3.1858}
% u = 0.249479, v = 1.247397
\rechterwinkel{0.2701}{-0.3130}{9.3669}
% u = 0.415799, v = 1.247397
\rechterwinkel{0.4301}{-0.2785}{15.0222}
% u = 0.582119, v = 1.247397
\rechterwinkel{0.5656}{-0.2358}{19.9298}
% u = 0.748438, v = 1.247397
\rechterwinkel{0.6750}{-0.1917}{24.0012}
% u = 0.914758, v = 1.247397
\rechterwinkel{0.7600}{-0.1509}{27.2581}
% u = 1.081077, v = 1.247397
\rechterwinkel{0.8244}{-0.1159}{29.7901}
% u = 1.247397, v = 1.247397
\rechterwinkel{0.8724}{-0.0874}{31.7166}
% u = 1.413717, v = 1.247397
\rechterwinkel{0.9076}{-0.0649}{33.1588}
% u = -1.413717, v = 1.413717
\rechterwinkel{-0.8929}{-0.0328}{-16.0167}
% u = -1.247397, v = 1.413717
\rechterwinkel{-0.8534}{-0.0438}{-15.2913}
% u = -1.081077, v = 1.413717
\rechterwinkel{-0.8009}{-0.0577}{-14.3283}
% u = -0.914758, v = 1.413717
\rechterwinkel{-0.7319}{-0.0745}{-13.0725}
% u = -0.748438, v = 1.413717
\rechterwinkel{-0.6436}{-0.0937}{-11.4722}
% u = -0.582119, v = 1.413717
\rechterwinkel{-0.5337}{-0.1141}{-9.4923}
% u = -0.415799, v = 1.413717
\rechterwinkel{-0.4017}{-0.1334}{-7.1305}
% u = -0.249479, v = 1.413717
\rechterwinkel{-0.2502}{-0.1487}{-4.4341}
% u = -0.083160, v = 1.413717
\rechterwinkel{-0.0850}{-0.1573}{-1.5058}
% u = 0.083160, v = 1.413717
\rechterwinkel{0.0850}{-0.1573}{1.5058}
% u = 0.249479, v = 1.413717
\rechterwinkel{0.2502}{-0.1487}{4.4341}
% u = 0.415799, v = 1.413717
\rechterwinkel{0.4017}{-0.1334}{7.1305}
% u = 0.582119, v = 1.413717
\rechterwinkel{0.5337}{-0.1141}{9.4923}
% u = 0.748438, v = 1.413717
\rechterwinkel{0.6436}{-0.0937}{11.4722}
% u = 0.914758, v = 1.413717
\rechterwinkel{0.7319}{-0.0745}{13.0725}
% u = 1.081077, v = 1.413717
\rechterwinkel{0.8009}{-0.0577}{14.3283}
% u = 1.247397, v = 1.413717
\rechterwinkel{0.8534}{-0.0438}{15.2913}
% u = 1.413717, v = 1.413717
\rechterwinkel{0.8929}{-0.0328}{16.0167}
}


	\rechtewinkel

	\draw[color=darkred] \upathA;
	\draw[color=darkred] \upathB;
	\draw[color=darkred] \upathC;
	\draw[color=darkred] \upathD;
	\draw[color=darkred] \upathE;
	\draw[color=darkred] \upathF;
	\draw[color=darkred] \upathG;
	\draw[color=darkred] \upathH;
	\draw[color=darkred] \upathI;
	\draw[color=darkred] \upathJ;
	\draw[color=darkred] \upathK;
	\draw[color=darkred] \upathL;
	\draw[color=darkred] \upathM;
	\draw[color=darkred] \upathN;
	\draw[color=darkred] \upathO;
	\draw[color=darkred] \upathP;
	\draw[color=darkred] \upathQ;
	\draw[color=darkred] \upathR;

	\draw[color=blue] \vpathA;
	\draw[color=blue] \vpathB;
	\draw[color=blue] \vpathC;
	\draw[color=blue] \vpathD;
	\draw[color=blue] \vpathE;
	\draw[color=blue] \vpathF;
	\draw[color=blue] \vpathG;
	\draw[color=blue] \vpathH;
	\draw[color=blue] \vpathI;
	\draw[color=blue] \vpathJ;
	\draw[color=blue] \vpathK;
	\draw[color=blue] \vpathL;
	\draw[color=blue] \vpathM;
	\draw[color=blue] \vpathN;
	\draw[color=blue] \vpathO;
	\draw[color=blue] \vpathP;
	\draw[color=blue] \vpathQ;
	\draw[color=blue] \vpathR;

\end{scope}

\end{tikzpicture}
\end{document}

