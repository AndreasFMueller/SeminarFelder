
Rossby-Wellen, auch planetare Wellen genannt, sind grossräumige Wellenerscheinungen, die in erster Linie durch die Breitenabhängigkeit des Coriolis-Parameters verursacht werden. Diese Breitenabhängigkeit wird durch den sogenannten \emph{Beta-Term}
\begin{equation}
	\beta = \frac{\partial f}{\partial y}
	\label{eq:beta_term}
\end{equation}
beschrieben. Bewegt sich ein Luftpaket meridional, erfährt es eine Änderung seiner planetaren Vorticity \(f\). Unter der Erhaltung der potenziellen Vorticity erzwingt dies eine kompensierende Änderung der relativen Vorticity~\(\zeta\), was zu einer wellenförmigen Rückstellbewegung führt. Diese Dynamik macht Rossby-Wellen zu einer direkten Konsequenz der PV-Erhaltung auf einer Kugel oder Beta-Ebene.

\subsubsection{Lineare Theorie barotroper Rossby-Wellen}

Abschnitt inspiriert von \cite{rossby:mueller2018}.

In Äquatornähe dominiert eine mittlere Ost–West-Strömung mit Geschwindigkeit \(U\). Betrachten wir kleine Störungen \((u,v)\) dieser Strömung:
\begin{equation}
	u' = U + u, \quad v' = v, \quad \text{mit } |u|, |v| \ll U.
	\label{eq:perturbation}
\end{equation}
Für eine quellenfreie Strömung existiert eine Stromfunktion~\(\psi\):
\begin{equation}
	u = -\frac{\partial \psi}{\partial y}, \quad v = \frac{\partial \psi}{\partial x}.
	\label{eq:stream_function}
\end{equation}
Die relative Vorticity ergibt sich zu
\begin{equation}
	\zeta = \frac{\partial v}{\partial x} - \frac{\partial u}{\partial y} = \Delta \psi,
	\label{eq:relative_vorticity}
\end{equation}
und die absolute Vorticity ist \(\zeta + f\) mit \(f = f(y)\). Unter der Annahme, dass die absolute Vorticity in der reibungsfreien Strömung erhalten bleibt,
\begin{equation}
	\frac{d}{dt} (\zeta + f) = 0,
	\label{eq:absolute_vorticity_conservation}
\end{equation}
und unter Verwendung der Näherungen
\begin{equation}
	u \ll U, \quad \frac{\partial \zeta}{\partial y} \ll \frac{\partial f}{\partial y} = \beta, \quad v = \frac{\partial \psi}{\partial x},
	\label{eq:linear_approximations}
\end{equation}
erhält man die linearisierte Vorticity-Gleichung
\begin{equation}
	\frac{\partial \zeta}{\partial t} + U \frac{\partial \zeta}{\partial x} + \beta \frac{\partial \psi}{\partial x} = 0.
	\label{eq:linear_vorticity_equation}
\end{equation}
Mit \(\zeta = \Delta \psi\) folgt
\begin{equation}
	\frac{\partial \Delta \psi}{\partial t} + U \frac{\partial \Delta \psi}{\partial x} + \beta \frac{\partial \psi}{\partial x} = 0.
	\label{eq:linear_vorticity_equation_psi}
\end{equation}

\subsubsection{Wellenlösung und Dispersionsrelation}

Wir setzen eine ebene Welle der Form
\begin{equation}
	\psi(x,y,t) = \cos(kx + ly - \omega t)
	\label{eq:wave_solution}
\end{equation}
ein und erhalten die Dispersionsrelation
\begin{equation}
	\omega = U k - \frac{\beta k}{k^2 + l^2}.
	\label{eq:dispersion_relation}
\end{equation}
Die Phasengeschwindigkeit lautet
\begin{equation}
	c = \frac{\omega}{k} = U - \frac{\beta}{k^2 + l^2}.
	\label{eq:phase_speed}
\end{equation}
Daraus folgt, dass sich Rossby-Wellen relativ zur mittleren Strömung stets nach Westen ausbreiten. In der Atmosphäre bedeutet dies: Selbst bei einer nach Osten gerichteten Grundströmung (z. B. im Jetstream) bewegt sich die Wellenform gegen die Strömung. Rossby-Wellen sind besonders stabil und können über viele Tage bestehen, wodurch sie grossräumig die Lage von Hoch- und Tiefdrucksystemen und die Pfade von Wetterfronten beeinflussen.


\subsection{Rossby-Wellen und Wetterextreme}

Ein eindrucksvolles Beispiel für den Einfluss quasi-stationärer Rossby-Wellen auf Extremwetterereignisse ist der Sommer 2010.
In dieser Periode dominierten atmosphärische Zirkulationsmuster mit einer Wellenzahl \(k = 7\), die zu gleichzeitigen, aber geographisch weit entfernten Extremen führten.

Über Westrussland etablierte sich ein blockierendes Hochdruckgebiet, das über Wochen bestehen blieb und eine extreme Hitzewelle mit Temperaturen über 40\,$^\circ$C auslöste.
Die anhaltende Trockenheit begünstigte grossflächige Waldbrände und dichten Smog, besonders im Raum Moskau, mit tausenden Todesopfern.

Zur gleichen Zeit erlebte Pakistan ungewöhnlich starken Monsunregen.
Ein persistentes Tiefdruckgebiet führte zu einer Jahrhundertflut, von der über 20 Millionen Menschen betroffen waren.

Beide Ereignisse lassen sich durch dasselbe planetare Wellenmuster erklären: Die quasi-stationäre Rossby-Welle blockierte die Westwinddrift, sodass ein stabiles Hoch über Russland und ein stationäres Tief über Pakistan bestehen blieben.
Dieses Beispiel verdeutlicht, wie grossskalige Dynamik auf der planetaren Skala direkte, langanhaltende Auswirkungen auf regionale Extremwetterlagen haben kann \cite{rossby:petoukhov2013}.
