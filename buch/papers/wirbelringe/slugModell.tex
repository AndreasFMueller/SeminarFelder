%
% slugModell.tex -- Erleutert das Slug Modell
%
% !TEX root = ../../buch.tex
% !TEX encoding = UTF-8
%
\section{Slug Modell}
In diesem Kapitel werden wir uns das Slug Modell etwas genauer unter die Lupe nehmen, um die Entstehung eines Wirbelrings etwas genauer zu verstehen.
Die genaue mathematische Beschreibung des Entstehungsprozesses ist sehr komplex und würde den Rahmen dieses Kapitels sprengen.
Dafür kann mittels Slug Modell dieser Prozess relativ genau angenähert werden.

\subsection{Grundlegende Idee}
Im Slug Modell wird der Impulsaustritt eines Fluidvolumens (eines Fluid-Slug) aus einer Öffnung beschrieben.
Man kann sich dies als Zylinder aus Flüssigkeit vorstellen, welcher innert kürzester Zeit aus einer Öffnung geschoben wird.
Dabei definieren wir folgende Grössen:
\begin{itemize}
    \item Zirkulation $\Gamma$
    \item Wirbelstärke $\omega$
    \item Slug Geschwindigkeit $u_p(t)$
    \item Slug Durchmesser $D$
    \item Slug Länge $L$
\end{itemize}
Tritt nun das Slug aus dem Zylinder aus, so trifft die Aussenseite des Slugs auf das stehende Fluid (bspw. Luft oder Wasser) ausserhalb.
Dies bewirkt aufgrund der Geschwindigkeitsdifferenz zwischen dem Inneren des Slugs und dem stehenden Fluid aussen eine Scherung.
Die Scherung hat zur Folge, dass sich das Slug beginnt \glqq Aufzurollen\grqq und somit einen Wirbel formt. 
Die in Kapitel \ref{paper:Wirbelringe:Stokes} genannten Formel für die Zirkulation kann approximativ auf
\[
\Gamma \approx \frac{1}{2}UL
\]
vereinfacht werden.

