%
% teil1.tex -- Beispiel-File für das Paper
%
% (c) 2020 Prof Dr Andreas Müller, Hochschule Rapperswil
%
% !TEX root = ../../buch.tex
% !TEX encoding = UTF-8
%

\section{Von Fourier zu Feldern
\label{fourier:section:vonFourierZuFeldern}}
In diesem Kapitel wird die Verbindung von Fourierreihen/ Fouriertransformationen und Feldern aufgezeigt.
%Zunächst erfolgt ein Exkurs in die Quantentheorie, um schliesslich zum Laser zu gelangen. % Todo überarbeiten

\subsection{Quantentheorie
\label{fourier:subsection:Quantentheorie}}
...
Energie wird in Form von Quanten gespeichert.
Vielfache der Grundfrequenz sind möglich.

Elektromagnetisches Feld $\rightarrow$ Fourier $\rightarrow$ Feldgleichung wie Gleichung vom Federpendel

\subsubsection{Der harmonische Oszillator
\label{fourier:subsubsection:derHarmonischeOszillator}}
Warum genau schauen wir uns den harmonischen Oszillator an?
In der Quantenelektrodynamik werden die elektromagnetischen Felder als quantisierte harmonische Oszillatoren modelliert, deren Energie ebenfalls in Vielfachen von $\hbar\cdot\omega$ vorliegt.
Die Zustände des Feldes (Photonenzahlenzustände) werden durch die Schwingungszustände des harmonischen Oszillators beschrieben.

Diese Quantisierung erklärt, warum Licht in Lasern nicht kontinuierlich, sondern in diskreten Paketen (Photonen) emittiert wird. % todo: diesen Satz an anderer passenden Stelle einfügen

Die Wellengleichung lautet bekanntlich
\begin{equation}
    \frac{\partial^2 u}{\partial t^2} = c^2 \left( \frac{\partial^2 u}{\partial x^2} + \frac{\partial^2 u}{\partial y^2} \right).
\end{equation}
Wenn nun der y-Anteil als konstant betrachtet wird, sind alle partiellen Ableitungen nach y gleich Null.
Dies führt zu der vereinfachten Gleichung
\begin{equation}
    \frac{\partial^2 u}{\partial t^2} = c^2 \frac{\partial^2 u}{\partial x^2}.
\end{equation}
Daraus lässt sich die Differentialgleichung
\begin{equation}
    \ddot{a}(t) = -k^2 a(t)
\end{equation}
aufstellen.
$a(t)$ ist hierbei ein Fourier-Koeffizient des elektromagnetischen Wellenfeldes.
Die Lösung dieser Differentialgleichung lautet
\begin{equation}
    u(t,x) = a_k(t) \cos(kx)
\end{equation}

% Besser sin verwenden --> besser e^ikx verwenden; Überlegen, ob wir komplex arbeiten möchten
% In Präsentation mit cos und im Paper mit e^ikx


% Wichtiger Schritt: Der harmonische Oszillator in der Quantenmechanik --> Unterlagen anschauen. 
% Evtl. Termin mit ihm, um Unklarheiten zu klären.

Diese Gleichung ist analog zur Gleichung eines Federpendels.
Es handelt sich hier um ein "Quanten-Federpendel".
% \begin{tikzpicture}

%     % Zeichne die horizontale Linie (Boden)
%     \draw[thick] (-2,0) -- (2,0);
    
%     % Zeichne die Wand
%     \draw[thick] (0,0) -- (0,2);
    
%     % Zeichne die Feder
%     \draw[thick] (0,2) -- (0,4);
%     \draw[thick] (0,4) -- (0.5,4.5);
%     \draw[thick] (0.5,4.5) -- (0,5);
    
%     % Zeichne die Masse
%     \filldraw[fill=gray] (0,5) circle (0.2);
    
%     % Zeichne die Schnur
%     \draw[thick] (0,5) -- (0,6);
    
%     \end{tikzpicture} % unschönes bild, sollte ein Federpendel werden; todo
% todo: weiterschreiben


\section{Teil 1
\label{fourier:section:teil1}}
\kopfrechts{Problemstellung}
Sed ut perspiciatis unde omnis iste natus error sit voluptatem
accusantium doloremque laudantium, totam rem aperiam, eaque ipsa
quae ab illo inventore veritatis et quasi architecto beatae vitae
dicta sunt explicabo.
Nemo enim ipsam voluptatem quia voluptas sit aspernatur aut odit
aut fugit, sed quia consequuntur magni dolores eos qui ratione
voluptatem sequi nesciunt
\begin{equation}
\int_a^b x^2\, dx
=
\left[ \frac13 x^3 \right]_a^b
=
\frac{b^3-a^3}3.
\label{fourier:equation1}
\end{equation}
Neque porro quisquam est, qui dolorem ipsum quia dolor sit amet,
consectetur, adipisci velit, sed quia non numquam eius modi tempora
incidunt ut labore et dolore magnam aliquam quaerat voluptatem.

Ut enim ad minima veniam, quis nostrum exercitationem ullam corporis
suscipit laboriosam, nisi ut aliquid ex ea commodi consequatur?
Quis autem vel eum iure reprehenderit qui in ea voluptate velit
esse quam nihil molestiae consequatur, vel illum qui dolorem eum
fugiat quo voluptas nulla pariatur?

\subsection{De finibus bonorum et malorum
\label{fourier:subsection:finibus}}
At vero eos et accusamus et iusto odio dignissimos ducimus qui
blanditiis praesentium voluptatum deleniti atque corrupti quos
dolores et quas molestias excepturi sint occaecati cupiditate non
provident, similique sunt in culpa qui officia deserunt mollitia
animi, id est laborum et dolorum fuga \eqref{fourier:equation1}.

Et harum quidem rerum facilis est et expedita distinctio
\ref{fourier:section:teil2}.
Nam libero tempore, cum soluta nobis est eligendi optio cumque nihil
impedit quo minus id quod maxime placeat facere possimus, omnis
voluptas assumenda est, omnis dolor repellendus
\ref{fourier:section:teil3}.
Temporibus autem quibusdam et aut officiis debitis aut rerum
necessitatibus saepe eveniet ut et voluptates repudiandae sint et
molestiae non recusandae.
Itaque earum rerum hic tenetur a sapiente delectus, ut aut reiciendis
voluptatibus maiores alias consequatur aut perferendis doloribus
asperiores repellat.


