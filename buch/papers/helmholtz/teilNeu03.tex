%
% teil3.tex -- Beispiel-File für Teil 3
%
% (c) 2020 Prof Dr Andreas Müller, Hochschule Rapperswil
%
% !TEX root = ../../buch.tex
% !TEX encoding = UTF-8
%
\section{Die Helmholtz-Zerlegung
\label{helmholtz:section:Helmholtz_Zerlegung}}
\kopfrechts{Die Helmholtz-Zerlegung}

Die Grundidee der Helmholtz-Zerlegung besteht darin, dass ein
differenzierbares  Vektorfeld $\boldsymbol{F}$, welches im Unendlichen
schnell genug abfällt, sich in zwei in noch zu spezifizierendem
Sinne orthogonale Komponenten zerlegen lässt: einen eindeutig
wirbelfreien Teil und einen quellenfreien Teil.
Diese Zerlegung spiegelt die physikalischen Eigenschaften des
Vektorfeldes wider und ist insbesondere in der Akustik, Strömungsmechanik
\index{Stromungsmechanik@Strömungsmechanik}%
und Elektrodynamik von zentraler Bedeutung.
\index{Elektrodynamik}%

\subsection{Formale Definition der Helmholtz-Zerlegung
\label{helmholtz:subsection:def_Helmholtz_Zerlegung}}

Mit der Idee der Helmholtz-Zerlegung wird das Vektorfeld geschrieben als
\begin{equation*}
\underbrace{
\boldsymbol{F}
}_{
\textstyle\text{zu zerlegendes Vektorfeld}
}
=
\underbrace{
-\nabla \Phi
}_{
\textstyle\text{irrotationaler Teil}
}
+
\underbrace{
\nabla \times \boldsymbol{\Psi}
}_{
\textstyle\text{solenoidaler Teil}
}
%\label{helmholtz:equationAllgemein}
\end{equation*}
und die einzelnen Komponenten lässt sich wie folgt beschreiben:

\begin{itemize}
\item
Der erste Term $ -\nabla \Phi $ ist wirbelfrei, da seine Rotation verschwindet.
\begin{itemize}
\item
Dieser Anteil hat nach 
\eqref{helmholtz:eqn:rotgradphi}
keine Rotation: $\nabla \times (-\nabla \Phi) = 0$.
(Nachweis später in Abschitt~\ref{helmholtz:subsection:math-Nachweis}..)
\item
Die Feldlinien verlaufen strahlenförmig von Quellen zu Senken.
\end{itemize}

\item
Der zweite Term $\nabla \times \boldsymbol{\Psi}$ hat gemäss der
Definition immer Divergenz gleich null und wird auch solenoidal
genannt.
\begin{itemize}
\item
Dieser Anteil hat nach
\eqref{helmholtz:eqn:divrotF}
keine Divergenz: $\nabla \cdot (\nabla \times
\boldsymbol{\Psi}) = 0$.
(Nachweis später in Abschitt~\ref{helmholtz:subsection:math-Nachweis}.)
\item
Die Feldlinien bilden geschlossene Schleifen ähnlich einem Magnetfeld.
\end{itemize}
\end{itemize}

\begin{table}
\centering
\begin{tabular}{l|l|l|l}
\hline
Komponente & Potential & Eigenschaft & Bezeichnung \\
\hline 
Irrotationaler Teil & $\Phi$ (Skalarpotential) & $\nabla \times (-\nabla \Phi) = 0$ & wirbelfrei\\
Solenoidaler Teil & $\boldsymbol{\Psi}$ (Vektorpotential) & $\nabla \cdot (\nabla \times \boldsymbol{\Psi}) = 0$ & quellenfrei\\
\hline
\end{tabular}
\caption{Übersicht der Helmholtz-Zerlegung}
\label{tab:helmholtz_overview}
\end{table}

In der Tabelle \ref{tab:helmholtz_overview} ist die komplementäre
Eigeschaft, die Bezeichnung und die Potentialbeschreibung der
jeweiligen Komponenten zusammengefasst.

Um die Zerlegung anwenden zu können, müssen die Potentiale $\Phi$
und $\boldsymbol{\Psi}$ berechnet werden.
Dies kann auf verschiedene Weisen erfolgen, abhängig von den Randbedingungen.

\subsection{Mathematischer Nachweis der Eigenschaften
\label{helmholtz:subsection:math-Nachweis}}

\subsubsection{Nachweis der Wirbelfreiheit des irrotationalen Anteils
(Identität \eqref{helmholtz:eqn:rotgradphi})}
Sei $\Phi$ ein zweimal stetig differenzierbares Skalarfeld.

\begin{enumerate}
    \item Zuerst wird der Gradient $\nabla\Phi$ berechnet:
    Der Gradient ergibt
    \[
    \nabla \Phi =
	\renewcommand{\arraystretch}{2.0}
    \begin{pmatrix}
        \displaystyle \frac{\partial \Phi}{\partial x} \\
        \displaystyle \frac{\partial \Phi}{\partial y} \\
        \displaystyle \frac{\partial \Phi}{\partial z}
    \end{pmatrix}.
    \]

    \item Nun wird die Rotation dieses Gradientenfeldes berechnet:
    Die Definitionsformel
    der Rotation $\nabla \times \boldsymbol{F}$ auf den Gradientenvektor
    anwenden ergibt
    \[
    \nabla \times (\nabla \Phi) =
	\renewcommand{\arraystretch}{2.0}
    \begin{pmatrix}
        \displaystyle\frac{\partial}{\partial y}\left(\frac{\partial \Phi}{\partial z}\right) - \frac{\partial}{\partial z}\left(\frac{\partial \Phi}{\partial y}\right) \\
        \displaystyle\frac{\partial}{\partial z}\left(\frac{\partial \Phi}{\partial x}\right) - \frac{\partial}{\partial x}\left(\frac{\partial \Phi}{\partial z}\right) \\
        \displaystyle\frac{\partial}{\partial x}\left(\frac{\partial \Phi}{\partial y}\right) - \frac{\partial}{\partial y}\left(\frac{\partial \Phi}{\partial x}\right)
    \end{pmatrix}.
    \]

    \item %Den \emph{Satz von Schwarz} anwenden:
    
    Zur besseren Übersicht werden die gemischten Ableitungen ausgeschrieben:
    \[
    \nabla \times (\nabla \Phi) =
	\renewcommand{\arraystretch}{2.0}
    \begin{pmatrix}
        \displaystyle\frac{\partial^2 \Phi}{\partial y\, \partial z} - \frac{\partial^2 \Phi}{\partial z\, \partial y} \\
        \displaystyle\frac{\partial^2 \Phi}{\partial z\, \partial x} - \frac{\partial^2 \Phi}{\partial x\, \partial z} \\
        \displaystyle\frac{\partial^2 \Phi}{\partial x\, \partial y} - \frac{\partial^2 \Phi}{\partial y\, \partial x}
    \end{pmatrix}.
    \]
    Der \emph{Satz von Schwarz} besagt, dass bei stetigen differenzierbaren
    Funktionen die Reihenfolge der partiellen Ableitungen vertauscht werden
    kann.
    Damit heben sich die Terme in jeder Komponente gegenseitig auf
    und es bleibt
    \[
    \nabla \times (\nabla \Phi)  =
    \begin{pmatrix}
        0 \\
        0 \\
        0
    \end{pmatrix} = \boldsymbol{0}.
    \]
\end{enumerate}
Damit ist die Identität \eqref{helmholtz:eqn:rotgradphi} beweisen.

\subsubsection{Nachweis der Quellenfreiheit des solenoidalen Anteils
(Identität \eqref{helmholtz:eqn:divrotF})}

\begin{enumerate}
    \item Berechnung der Rotation $\nabla \times \boldsymbol{\Psi}$:
	Zuerst wird der Rotations-Operator auf $\boldsymbol{\Psi}$ angewendet:
    \[
    \nabla \times \boldsymbol{\Psi} =
	\renewcommand{\arraystretch}{2.0}
    \begin{pmatrix}
        \displaystyle\frac{\partial \Psi_z}{\partial y} - \frac{\partial \Psi_y}{\partial z} \\
        \displaystyle\frac{\partial \Psi_x}{\partial z} - \frac{\partial \Psi_z}{\partial x} \\
        \displaystyle\frac{\partial \Psi_y}{\partial x} - \frac{\partial \Psi_x}{\partial y}
    \end{pmatrix}.
    \]

    \item Berechnung der Divergenz des Ergebnisses: Nun wird der
    Divergenz-Operator auf das resultierende Vektorfeld angewendet.
    Dies geschieht durch die Summe der partiellen Ableitungen der
    Komponenten nach der jeweiligen Koordinate:
    \begin{align*}
    % Schritt 1: Definition der Divergenz anwenden
    \nabla \cdot (\nabla \times \boldsymbol{\Psi}) &= \frac{\partial}{\partial x}\left( \frac{\partial \Psi_z}{\partial y} - \frac{\partial \Psi_y}{\partial z} \right) + \frac{\partial}{\partial y}\left( \frac{\partial \Psi_x}{\partial z} - \frac{\partial \Psi_z}{\partial x} \right) + \frac{\partial}{\partial z}\left( \frac{\partial \Psi_y}{\partial x} - \frac{\partial \Psi_x}{\partial y} \right) \\
    % Schritt 2: Die Ableitungen ausführen
    &= \frac{\partial^2 \Psi_z}{\partial x\, \partial y} - \frac{\partial^2 \Psi_y}{\partial x\, \partial z} + \frac{\partial^2 \Psi_x}{\partial y\, \partial z} - \frac{\partial^2 \Psi_z}{\partial y\, \partial x} + \frac{\partial^2 \Psi_y}{\partial z\, \partial x} - \frac{\partial^2 \Psi_x}{\partial z\, \partial y} \\
    % Schritt 3: Terme umsortieren, um Paare zu bilden
    &= \left( \frac{\partial^2 \Psi_x}{\partial y\, \partial z} - \frac{\partial^2 \Psi_x}{\partial z\, \partial y} \right) + \left( \frac{\partial^2 \Psi_y}{\partial z\, \partial x} - \frac{\partial^2 \Psi_y}{\partial x\, \partial z} \right) + \left( \frac{\partial^2 \Psi_z}{\partial x\, \partial y} - \frac{\partial^2 \Psi_z}{\partial y\, \partial x} \right) = 0.
    \end{align*}
    
    \item Nach dem \emph{Satz von Schwarz} ist die Reihenfolge der
    partiellen Ableitungen bei zweimal stetig differenzierbaren
    Funktionen irrelevant.
    Daher gilt:
    \[
    \frac{\partial^2 f}{\partial x\, \partial y} = \frac{\partial^2 f}{\partial y\, \partial x}.
    \]
    Angewendet auf die obige Gleichung bedeutet dies, dass jeder
    Term in den Klammern Null ergibt:
    \begin{align*}
    \nabla \cdot (\nabla \times \boldsymbol{\Psi})
&=
\underbrace{\left( \frac{\partial^2 \Psi_x}{\partial y\, \partial z} - \frac{\partial^2 \Psi_x}{\partial y\, \partial z} \right)}_{\displaystyle=0}
+
\underbrace{\left( \frac{\partial^2 \Psi_y}{\partial z\, \partial x} - \frac{\partial^2 \Psi_y}{\partial z\, \partial x} \right)}_{\displaystyle=0}
+
\underbrace{\left( \frac{\partial^2 \Psi_z}{\partial x\, \partial y} - \frac{\partial^2 \Psi_z}{\partial x\, \partial y} \right)}_{\displaystyle=0} \\
%    &= 0 + 0 + 0 \\
    &= 0.
    \end{align*}
\end{enumerate}
Damit ist die Identität $\nabla \cdot (\nabla \times \boldsymbol{\Psi})
= 0$ bewiesen.

\subsection{Berechnung der Potentiale
\label{helmholtz:subsection:Berechnung der Potentiale}}

\subsubsection{Greenscher Ansatz}
Ein möglicher Ansatz zur Lösung der partiellen Differentialgleichung
ist die Verwendung der Dirac-$\delta$-Funktion und der
greenschen Funktionen,
\index{greensche Funktion}%
wie dies in \cite{baird_helmholtz} beschrieben ist.

Die Vorgehensweise zur Lösung der Differentialgleichung
\begin{equation}
\Delta F = b
\label{helmholtz:green:eqn:gl}
\end{equation}
ist analog zur Lösung eines linearen Gleichungsysstems
$Ax=b$.
Dazu muss eine Matrix $A^{-1}$ gefunden werden, welche $A^{-1}A=I$
erfüllt.
Die Rolle der Multiplikation $A^{-1}b$ wird übernommen von einem
Integral
\[
F(\boldsymbol{x})
=
\int_V
G(\boldsymbol{x},\boldsymbol{x}')
b(\boldsymbol{x}')
\,dV'
\]
über das Argument der Funktion auf der rechten Seite von
\eqref{helmholtz:green:eqn:gl}.
Die Rolle der Matrix wird übernommen von einer Funktion
$G(\boldsymbol{x},\boldsymbol{x}')$, der sogenannten
\emph{greenschen Funktion}.
Die Zusammensetzung mit dem Laplace-Operator muss den ``Einheitsoperator''
geben, der die Funktion reproduziert.
Die Dirac-$\delta$-Funktion hat diese Eigenschaft, denn es gilt
\[
f(\boldsymbol{x})
=
\int_V
\delta(\boldsymbol{x}-\boldsymbol{x}')
f(\boldsymbol{x}')
\,dV'\!.
\]
Man kann zeigen, dass 
\[
\delta(\boldsymbol{x}-\boldsymbol{x}')
=
\frac{1}{4\pi} \nabla^2 \frac{1}{|\boldsymbol{x}-\boldsymbol{x}'|}
\]
die gewünschte Eigenschaft hat,
dass also das Integral mit der Funktion
\[
G(\boldsymbol{x},\boldsymbol{x}')
=
\frac{1}{4\pi}\,
\frac{1}{|\boldsymbol{x}-\boldsymbol{x}'|}
\]
über $\boldsymbol{x}'$ die gesuchte Inverse des Laplace-Operators ist.
%Dies entspricht der Integralform einer Matrixmultiplikation, wie
%in Tabelle \ref{tab:helmholtz_matrix_analogie} veranschaulicht.
%Dabei wird die Funktion $\boldsymbol{F}$, die in diesem Fall einen
%Vektor darstellt, mit der Delta-Funktion multipliziert wie folgt:
%
%\begin{equation}
%\Delta F = -4 \pi \rho. 
%\label{helmholtz:DGL_idee}
%\end{equation}
%
%\begin{equation}
%\underbrace{\frac{1}{4 \pi} \Delta F}_{\text{invertiert}} = \rho 
%\label{helmholtz:DGL_idee_umformung}
%\end{equation}
%
%\begin{table}
%\centering
%\begin{tabular}{l|l}
%  & Idee in Matrixnotation \\
%\hline
%$\Delta F = -4 \pi \rho$  & $Au = f$ \\
%$\underbrace{\frac{1}{4 \pi} \Delta F}_{invertiert} = \rho$ & $A^{-1}Au = A^{-1}f$  \\
% & $u = A^{-1}f$  \\
% \hline
%\end{tabular}
%\caption{Analogie zur Matrixinversion}
%\label{tab:helmholtz_matrix_analogie}
%\end{table}
Es gilt also
\[
F(\boldsymbol{r})
=
\int_V
F(\boldsymbol{r}')
\delta(\boldsymbol{r}-\boldsymbol{r}')
\,dV'
=
-\frac{1}{4\pi}
\int_V
\nabla^2 \frac{1}{|\boldsymbol{r}-\boldsymbol{r}'|}
F(\boldsymbol{r}')
\,dV'
\]
für jede Funktion $F$.
Setzt man anstelle von $F$ ein Vektorfeld $\boldsymbol{F}$ ein, gilt
sie auch für die Komponenten des Vektorfeldes.

%Grundlegend für diesen Ansatz ist die Beziehung
%\begin{equation}
%\delta (\boldsymbol{x} - \boldsymbol{x}')
%=
%\frac{1}{4 \pi} \nabla^2 \frac{1}{|\boldsymbol{x} - \boldsymbol{x}'|}.
%\label{helmholtz:dirac}
%\end{equation}
%Diese Beziehung lässt sich analog zur Multiplikation mit der
%Einheitsmatrix verstehen, was es ermöglicht, das Vektorfeld
%$\boldsymbol{F}$ durch das Volumenintegral
%\begin{equation}
%\boldsymbol{F}(\boldsymbol{r})
%=
%-\frac{1}{4\pi} \int_V \boldsymbol{F}(\boldsymbol{r}') \nabla^2
%\left( \frac{1}{|\boldsymbol{r} - \boldsymbol{r}'|} \right)
%\,dV'
%\end{equation}
%auszudrücken.

Unter Verwendung der Vektoridentität
\begin{equation*}
\nabla^2 \boldsymbol{A}
=
\underbrace{
\nabla ( \nabla \cdot \boldsymbol{A} )
}_{
\textstyle\text{Gradient\strut}
}
-
\underbrace{
\nabla \times (\nabla \times \boldsymbol{A} )
}_{
\textstyle\text{Rotation\strut}
},
\end{equation*}
(siehe auch \eqref{helmholtz:eqn:fundamental})
kann das Vektorfeld $\boldsymbol{F}$ als
\begin{equation}
\boldsymbol{F}(\boldsymbol{r})
=
- \frac{1}{4 \pi} \nabla \biggl(
\underbrace{
\nabla \cdot \int_V
\frac{\boldsymbol{F}(\boldsymbol{r}')
}{
|\boldsymbol{r}
-
\boldsymbol{r}'|} \,dV'
}_{\displaystyle = \Phi(\boldsymbol{r})}
\biggr)
+
\frac{1}{4 \pi} \nabla \times \biggl(
\underbrace{
\nabla \times
\int_V
\frac{\boldsymbol{F}(\boldsymbol{r}')
}{
|\boldsymbol{r} - \boldsymbol{r}'|
}
\,dV'
}_{\displaystyle = \boldsymbol{\Psi}(\boldsymbol{r})}
\biggr)
\label{helmholtz:eqn:potentiale}
\end{equation}
geschrieben werden.
Diese Gleichung repräsentiert die Helmholtz-Zerlegung des Vektorfeldes
$\boldsymbol{F}$.
Die Klammerausdrücke von \eqref{helmholtz:eqn:potentiale}
sind die gesuchten Potentiale.

In der Form \eqref{helmholtz:eqn:potentiale} ist die Helmholtz-Zerlegung
noch nicht wirklich nützlich.
Dazu muss es in ein Integral über eine geeignete Ableitung des
Vektorfeldes $\boldsymbol{F}(\boldsymbol{r})$ umgewandelt werden.
Dies ist mithilfe geeigneter Randbedingungen und partieller Integration
möglich.

\subsubsection{Berechnung der Potentiale in unbeschränkten Gebieten}

Im Falle eines unbeschränkten Gebietes gibt es keine Randterme und
die Potentiale $\Phi$ und $\boldsymbol{\Psi}$ lassen sich mit Hilfe
der greenschen Funktion wie folgt berechnen:
\begin{itemize}
\item Skalares Potential:
\begin{equation*}
\Phi(\boldsymbol{r})
=
\frac{1}{4 \pi}
\int_{V}
\frac{\nabla' \cdot \boldsymbol{F}(\boldsymbol{r}')}{|\boldsymbol{r} - \boldsymbol{r}'|}
\,dV'\!.
\end{equation*}

\item Vektorpotential:
\begin{equation*}
\boldsymbol{\Psi}(\boldsymbol{r})
=
\frac{1}{4 \pi}
\int_{V}
\frac{\nabla' \times \boldsymbol{F}(\boldsymbol{r}')}{|\boldsymbol{r} - \boldsymbol{r}'|}
\,dV'\!.
\end{equation*}
\end{itemize}
Hier bezeichnet $\nabla'$ den Nabla-Operator bezüglich der
Integrationsvariablen $\boldsymbol{r}'$.

\subsubsection{Berechnung der Potentiale in beschränkten Gebieten}
Für endliche Volumina $V$ mit Oberfläche $S$ bleiben bei der partiellen
Integraltion zusätzliche Oberflächenintegrale stehen:
\begin{itemize}
\item Skalares Potential:
\begin{equation*}
\Phi (\boldsymbol{r})
=
\frac{1}{4\pi}
\int_V
\frac{\nabla' \cdot \boldsymbol{F}(\boldsymbol{r}')}{|\boldsymbol{r} - \boldsymbol{r}'|}
\,dV'
+
\frac{1}{4\pi}
\oint_S \frac{\boldsymbol{F}(\boldsymbol{r}') \cdot \boldsymbol{n}}{|\boldsymbol{r} - \boldsymbol{r}'|}
\,dS'\!.
\end{equation*}

\item Vektorpotential:
\begin{equation*}
\boldsymbol{\Psi}(\boldsymbol{r})
=
\frac{1}{4\pi}
\int_V
\frac{\nabla' \times \boldsymbol{F}(\boldsymbol{r}')}{|\boldsymbol{r} - \boldsymbol{r}'|}
\,dV'
+
\frac{1}{4\pi}
\oint_S
\frac{\boldsymbol{n} \times \boldsymbol{F}(\boldsymbol{r}')}{|\boldsymbol{r} - \boldsymbol{r}'|}
\,dS'\!.
\end{equation*}
\end{itemize}
Hier bezeichnet $\boldsymbol{n}$ den nach aussen gerichteten
Normalenvektor auf der Oberfläche $S$\!.

\subsection{Bedingungen und Eindeutigkeit
\label{helmholtz:subsection:Bedingungen_Eindeutigkeit}}

\subsubsection{Voraussetzungen für die Anwendbarkeit}
Damit die Zerlegung korrekt durchgeführt werden kann, muss das
Vektorfeld $\boldsymbol{F}$ folgende Bedingungen erfüllen:

\begin{itemize}
\item Glattheit: $\boldsymbol{F}$ muss in dem betrachteten Gebiet
stetig differenzierbar sein.
\item Abfallverhalten: $\boldsymbol{F}$ muss im Unendlichen schneller
als $\frac{1}{r}$ abfallen, d.~h.
\begin{equation*}
\lim_{r \to \infty} r\cdot |\boldsymbol{F}(\boldsymbol{r})| = 0,
\end{equation*}
wobei $r = |\boldsymbol{r}|$ der Abstand vom Ursprung ist.
\end{itemize}
Diese Bedingungen stellen sicher, dass die Integrale in der greenschen
Formulierung konvergieren und die Zerlegung eindeutig ist.

\subsubsection{Eindeutigkeit der Zerlegung
\label{helmholtz:subsection:Bedingungen_onlyEindeutigkeit}}

Die Eindeutigkeit der Helmholtz-Zerlegung hängt von den Randbedingungen ab:
\index{Randbedingung}%

\begin{itemize}
\item Für unbeschränkte Gebiete: Die Zerlegung ist eindeutig, wenn
das Vektorfeld im Unendlichen gegen null geht.
Diese Bedingung stellt sicher, dass bestimmte Oberflächenintegrale
bei der Herleitung der Zerlegung verschwinden \cite{wiki:helmholtz}.

\item
Für beschränkte Gebiete: Die Eindeutigkeit erfordert zusätzliche
Randbedingungen.
Obwohl das Vektorfeld auch als Summe eines irrotationalen und eines
solenoidalen Anteils geschrieben werden kann, ist für eine eindeutige
Lösung die Festlegung von Randbedingungen notwendig \cite{wiki:helmholtz}.
Typischerweise sind dies:
  \begin{itemize}
    \item Festlegung der Normalkomponente des irrotationalen Teils
    auf dem Rand.
    \item Festlegung der Tangentialkomponente des solenoidalen Teils
    auf dem Rand.
  \end{itemize}
\end{itemize}

\subsubsection{Orthogonalität der Komponenten}

Eine wichtige Eigenschaft der Helmholtz-Zerlegung ist die Orthogonalität
der beiden Komponenten.
\index{Orthogonalitat@Orthogonalität}%
Für ein Vektorfeld
$\boldsymbol{F} = \boldsymbol{F}_{\text{irr}} + \boldsymbol{F}_{\text{sol}}$
gilt unter geeigneten Randbedingungen:
\begin{equation*}
\int_V \boldsymbol{F}_{\text{irr}} \cdot \boldsymbol{F}_{\text{sol}} \, dV = 0.
\end{equation*}
%
% XXX Inkorrekter Satz
%
Das Integral ist null, weil es mithilfe des gaussschen Integralsatzes
in ein Oberflächenintegral umgewandelt werden kann, das unter den
für die Helmholtz-Zerlegung Randbedingungen, wie oben
%\ref{helmholtz:subsection:Bedingungen_onlyEindeutigkeit}
beschrieben, verschwindet.
 
In physikalischen Anwendungen, wie beispielsweise in der Akustik,
bedeutet diese mathematische Orthogonalität, dass die beiden
Feldkomponenten unterschiedliche physikalische Phänomene beschreiben,
die unabhängig voneinander analysiert werden können.
Mehr dazu wird im Abschnitt~\ref{helmholtz:Energie_Interpretation}
erläutert.


