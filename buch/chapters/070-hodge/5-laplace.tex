%
% 5-laplace.tex -- Laplace-Operator
%
% (c) 2025 Prof Dr Andreas Müller
%
\section{Der Laplace-Operator}
\kopfrechts{Der Laplace-Operator}
Das Kodifferential $\delta$ reduziert die Ordnung einer Differentialform,
während $d$ die Ordnung erhöht.
Die Zusammensetzung dieser beiden Operatoren ist ein Differentialoperator
zweiter Ordnung, die den Grad einer Differentialform nicht ändert.

%
% Das Zusammenspiel von $d$ und $\delta$
%
\subsection{Das Zusammenspiel von $d$ und $\delta$}
Um diese Operatoren besser zu verstehen, berechnen wir $\delta d$ und
$d\delta$ in einigen Beispielen.

\begin{beispiel}
Die Operatoren $\delta d$ und $d\delta$ auf $\Omega^*(\mathbb{R}^2)$.
Eine 0-Form ist eine Funktion $f$, $\delta f={\ast}d{\ast}f =0$, weil
$*f$ eine $n$-Form ist, aber $d$ macht alle $n$-Formen zu 0.
Daher ist auch $d\delta=0$ auf 0-Formen.
Umgekehrt ist $\delta d=0$ auf $n$-Formen.

Auf der Funktion $f$ ist 
\begin{align*}
\delta d f
&=
{\ast}d{\ast}
\biggl(
\frac{\partial f}{\partial x^1}dx^1
+
\frac{\partial f}{\partial x^2}dx^2
\biggr)
\\
&=
{\ast}d
\biggl(
\frac{\partial f}{\partial x^1}dx^2
-
\frac{\partial f}{\partial x^2}dx^1
\biggr)
\\
&=
{\ast}\biggl(
\frac{\partial^2 f}{\partial (x^1)^2}dx^1\wedge dx^2
-
\frac{\partial^2 f}{\partial (x^2)^2}dx^2\wedge dx^1
\biggr)
=
\ast \Delta f dx^1\wedge dx^2
=
\Delta f.
\intertext{Umgekehrt ist auf einer $n$-Form $\omega = f\,dx^1\wedge dx^2$}
d\delta \omega
&=
d{\ast}d{\ast}(f\,dx^1\wedge dx^2)
\\
&= d{\ast}df
\\
&=
d{\ast}
\biggl(
\frac{\partial f}{\partial x^1}dx^1
+
\frac{\partial f}{\partial x^2}dx^2
\biggr)
\\
&=
d
\biggl(
\frac{\partial f}{\partial x^1}dx^2
-
\frac{\partial f}{\partial x^2}dx^1
\biggr)
\\
&=
\frac{\partial^2 f}{\partial (x^1)^2} dx^1\wedge dx^2
-
\frac{\partial^2 f}{\partial (x^2)^2} dx^2\wedge dx^1
\\
&=
\Delta f\, dx^1\wedge dx^2
.
\end{align*}
Der Operator $d\delta$ ist also im Wesentlichen die Wirkung
des Laplace-Operators auf dem Koeffizienten $f$.

Schliesslich können wir die Operatoren $d\delta$ und $\delta d$ auf
der 1-Form $\omega = f_1\,dx^1+f_2\,dx^2$ berechnen:
\begin{align*}
d\delta\omega
&=
d{\ast}d{\ast}\omega
\\
&=
d{\ast}d\bigl(f_1\,dx_2 - f_2\,dx_1)
\\
&=
d{\ast}\biggl(
\frac{\partial f_1}{\partial x^1}\,dx^1\wedge dx^2
-
\frac{\partial f_2}{\partial x^2}\,dx^2\wedge dx^1
\biggr)
\\
&=
d\biggl(
\frac{\partial f_1}{\partial x^1}
+
\frac{\partial f_2}{\partial x^2}
\biggr)
\\
&=
\frac{\partial^2 f_1}{\partial (x^1)^2}\,dx^1
+
\frac{\partial^2 f_1}{\partial x^2\,\partial x^1}\,dx^2
+
\frac{\partial^2 f_2}{\partial x^1\,\partial x^2}\, dx^1
+
\frac{\partial^2 f_2}{\partial (x^2)^2}\, dx^2
\\
&=
\biggl(
\frac{\partial^2 f_1}{\partial (x^1)^2}
+
\frac{\partial^2 f_2}{\partial x^1\,\partial x^2}
\biggr)
\, dx_1
+
\biggl(
\frac{\partial^2 f_1}{\partial x^2\,\partial x^1}
+
\frac{\partial^2 f_2}{\partial (x^2)^2}
\biggr)
\, dx_2
\\
\delta d\omega
&=
{\ast}d{\ast}
\biggl(
\frac{\partial f_1}{\partial x^2}dx^2\wedge dx^1
+
\frac{\partial f_2}{\partial x^2}dx^2\wedge dx^2
\biggr)
\\
&=
{\ast}d{\ast}
\biggl(
\frac{\partial f_1}{\partial x^2}
-
\frac{\partial f_2}{\partial x^1}
\biggr) dx^1\wedge dx^2
\\
&=
{\ast}d
\biggl(
\frac{\partial f_1}{\partial x^2}
-
\frac{\partial f_2}{\partial x^1}
\biggr)
\\
&=
{\ast}
\biggl(
\frac{\partial^2 f_1}{\partial x^1\,\partial x^2}
-
\frac{\partial^2 f_2}{\partial (x^1)^2}
\biggr)
dx^1
+
{\ast}
\biggl(
\frac{\partial^2 f_1}{\partial (x^2)^2}
-
\frac{\partial^2 f_2}{\partial x^1\,\partial x^2}
\biggr) 
dx^2
\\
&=
\biggl(
\frac{\partial^2 f_1}{\partial x^1\,\partial x^2}
-
\frac{\partial^2 f_2}{\partial (x^1)^2}
\biggr)
dx^2
+
\biggl(
\frac{\partial^2 f_1}{\partial x^1\,\partial x^2}
-
\frac{\partial^2 f_2}{\partial (x^2)^2}
\biggr) 
dx^1.
\intertext{Daraus kann man ableiten, dass}
(d\delta+\delta d)\omega
&=
\Delta f_1\,dx^1
+
\Delta F_2\,dx^2
\end{align*}
In allen Fällen ist also $d\delta + \delta d$ im wesentlichen
der Laplace-Operator.
\end{beispiel}

Das Beispiel zeigt, dass der Laplace-Operator koordinatenunabhängig
definiert werden kann.

%
% Der Hodge-Laplace-Operator
%
\subsection{Der Hodge-Laplace-Operator}
Man kann nachrechnen, dass für eine beliebige Funktion auf
$\mathbb{R}^n$ mit der euklidischen Metrik der Operator $\delta d$
der Laplace-Operator ist:
\begin{align*}
\delta d f
&=
{\ast}d{\ast}\sum_{k=1}^n \frac{\partial f}{\partial x^k}\,dx^k
\\
&=
{\ast}d\sum_{k=1} \frac{\partial f}{\partial x^k}
(-1)^k
dx^1\wedge\dots\wedge \widehat{dx^k}\wedge\dots\wedge dx^n
\\
&=
{\ast}
\sum_{k=1}^n
(-1)^k
\frac{\partial^2 f}{\partial (x^k)^2}
dx^k\wedge dx^1\wedge\dots\wedge\widehat{dx^k}\wedge\dots\wedge dx^n
\\
&=
{\ast}
\biggl(
\sum_{k=1}^n
\frac{\partial^2 f}{\partial (x^k)^2}
\biggr)
dx^1\wedge\dots\wedge dx^n
\\
&=
\sum_{k=1}^n \frac{\partial^2 f}{\partial (x^k)^2}
=
\Delta f,
\end{align*}
wobei der Hut $\widehat{dx^k}$ bedeutet, dass diese 1-Form im
Produkt weggelassen werden muss.

Die Verkettung $\delta d$ allein ist aber kein gut definierter
Operator auf $p$-Formen, da er auf $n$-Formen verschwindet.
Für $n$-Formen ist dagegen $d\delta$ ein nicht trivialer Operator.
Der sogenannte Hodge-Laplace-Operator wird daher wie folgt definiert.

\begin{definition}
\label{buch:hodge:laplace:def:hodgelaplace}
Sei $M$ eine Mannigfaltigkeit mit einer Metrik, dann ist
der {\em Hodge-Laplace-Operator} der Operator
\index{Hodge-Laplace-Operator}%
\[
\Delta = d\delta + \delta d.
\]
Der Hodge-Laplace-Operator ist auf ganz $\Omega^*$ definiert.
\end{definition}

\begin{beispiel}
Wir berechnen den Hodge-Laplace-Operator auf $\mathbb{R}^2$ in
Polarkoordinaten.
\index{Hodge-Laplace-Operator!in Polarkoordinaten}%
Dazu verwenden wir die früher berechnete Darstellung des Hodge-Operators
in Polarkoordinaten.
\begin{align*}
\Delta f
&=
d\delta f
+
\delta d f
=
\delta d f
\\
&=
(-1)^{1\cdot 1+1}
{\ast}d{\ast}d f
=
{\ast}d{\ast}d f
\\
&=
{\ast}d{\ast}
\biggl(
\frac{\partial f}{\partial r}\,dr
+
\frac{\partial f}{\partial \varphi}\,d\varphi
\biggr)
\\
&=
{\ast}d\biggl(
\frac{\partial f}{\partial r}\,r\,d\varphi
-
\frac{\partial f}{\partial \varphi}\,\frac{1}{r}\,dr
\biggr)
\\
&=
{\ast}
\biggl(
\biggl(
\frac{\partial }{\partial r} r \frac{\partial f}{\partial r}
+
\frac{1}{r}
\frac{\partial^2 f}{\partial\varphi^2}
\biggr)
\,dr\wedge d\varphi
\biggr)
\\
&=
\biggl(
\frac{1}{r}
\frac{\partial}{\partial r} 
r
\frac{\partial}{\partial r}
+
\frac{1}{r^2}
\frac{\partial^2}{\partial \varphi^2}
\biggr)
f.
\end{align*}
Dies ist tatsächlich die wohlbekannte Darstellung des
Laplace-Operators in Polarkoordinaten
\end{beispiel}

\begin{beispiel}
Der Hodge-Laplace-Operator auf $\mathbb{R}^3$ in
Kugelkoordinaten.
\index{Hodge-Laplace-Operator!in Kugelkoordinaten}%
Die gleiche Art von Rechnung wie für den Hodge-Laplace-Operator in
Polarkoordinaten, aber unter Verwendung der Resultate von
Beispiel~\ref{buch:hodge:hodge:beispiel:r3}, ergibt sich
\begin{align*}
\Delta f
&=
(d\delta + \delta d) f
=
\delta d f
\\
&=
(-1)^{1\cdot 2+2}
{\ast}d{\ast}df
\\
&=
{\ast}d{\ast}\biggl(
\frac{\partial f}{\partial r}\,dr
+
\frac{\partial f}{\partial\vartheta}\,d\vartheta
+
\frac{\partial f}{\partial\varphi}\,d\varphi
\biggr)
\\
&=
{\ast}d\biggl(
\frac{\partial f}{\partial r} r^2\sin\vartheta \, d\vartheta\wedge d\varphi
-
\frac{\partial f}{\partial\vartheta}\,\sin\vartheta\,dr\wedge d\varphi
+
\frac{\partial f}{\partial\varphi}\,\frac{1}{\sin\vartheta}\,dr\wedge d\vartheta
\biggr)
\\
&=
{\ast}
\biggl(
\frac{\partial}{\partial r}r^2\frac{\partial f}{\partial r}
\sin\vartheta
+
\frac{\partial}{\partial \vartheta}
\sin\vartheta
\frac{\partial f}{\partial \vartheta}
+
\frac{1}{\sin\vartheta}
\frac{\partial^2 f}{\partial\varphi^2}
\biggr)
\,dr\wedge d\vartheta\wedge d\varphi
\\
&=
\frac{1}{r^2}
\frac{\partial}{\partial r}r^2\frac{\partial f}{\partial r}
+
\frac{1}{r^2\sin\vartheta}
\frac{\partial}{\partial \vartheta}
\sin\vartheta
\frac{\partial f}{\partial \vartheta}
+
\frac{1}{r^2\sin^2\vartheta}
\frac{\partial^2 f}{\partial \varphi^2}.
\end{align*}
Dies ist die wohlbekannte Formel für den Laplace-Operator in 
Kugelkoordinaten.
\end{beispiel}

\begin{satz}
Für den Hodge-Laplace-Operator gilt $\Delta = (d+\delta)^2$.
\end{satz}

\begin{proof}
Durch direkte Rechnung finden wir
\[
(d+\delta)^2
=
dd+d\delta+\delta d+\delta\delta
=
d\delta +\delta d.
\qedhere
\]
\end{proof}

An dieser Stelle wird klar, warum das Vorzeichen $(-1)^p$ in der
Definition des Kodifferentials zweckmässig war.

%
% Rotation
%
\subsection{Rotation einer Rotation
\label{buch:hodge:laplace:subsection:rotrot}}
In \eqref{buch:vektoranalysis:eqn:rotrot} wurde bereits eine Formel
für die Rotation einer Rotation gefunden, in der der Laplace-Operator
eine prominente Rolle spielte.
Dass dies nicht nur ein Zufall ist, zeigt die folgende allgemeine
Überlegung.
Der Hodge-Laplace-Opertor wird ist gemäss
Definition~\ref{buch:hodge:laplace:def:hodgelaplace}
als 
\[
\Delta 
=
d\delta + \delta d
\]
definiert.
In drei Dimensionen ist das Kodifferential auf $p$-Formen durch
die Definition~\ref{buch:hodge:kodifferential:def:delta} gegeben,
die
\[
\Delta
=
(-1)^{p(3-p)+p}
{*}d{*}
=
(-1)^{4p-p^2}
{*}d{*}
=
(-1)^p
{*}d{*}
\]
ergibt.
Eingesetzt in die Definition des Laplace-Operators finden wir
\[
\Delta
=
(-1)^p
d{*}d{*}
+
(-1)^{p+1}
{*}d{*}d.
\]
Für $p=1$, also für 1-Formen, kann man nach dem letzten Term auflösen
und erhält
\begin{equation}
{*}d{*}d
=
d{*}d{*}
+
\Delta.
\end{equation}
Die linke Seite entspricht beim Übergang von Vektorfeldern zu 1-Formen
mit der Abbildung $V$ der Formel \eqref{buch:vektoranalysis:eqn:rotrot}
für die iterierte Rotation.
Man kann daher die Definition des Hodge-Laplace-Opertors auch als
eine Erweiterung der Identität~\eqref{buch:vektoranalysis:eqn:rotrot}
der Vektoranalysis auf beliebige $p$-Formen ansehen.




