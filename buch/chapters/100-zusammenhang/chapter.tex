%
% chapter.tex -- Zusammenhang und kovariante Ableitung
%
% (c) 2025 Prof Dr Andreas Müller
%
\chapter{Zusammenhang und kovariante Ableitung
\label{chapter:zusammenhang}}
\kopflinks{Zusammenhang und kovariante Ableitung}

\noindent
Die Diskussion der Lie-Ableitung in
Abschnitt~\ref{buch:koordinaten:section:differentialoperatoren}
hat gezeigt, dass die zweite Ableitung eines Kurve nicht auf
allgemein kovariante Art definiert werden kann.
Der Grund dafür ist, dass die Tangentialvektoren in den Punkten
$\gamma(t)$ und $\gamma(t+\Delta t)$ in den disjunkten Vektorräumen
$T_{\gamma(t)}M$ und $T_{\gamma(t+\Delta t)}M$ liegen und sich nicht
einfach so miteinander verrechnen lassen
(Abbildung~\ref{buch:zusammenhang:fig:differenz}).
\input{chapters/100-zusammenhang/fig/fig-differenz.tex}%
Dazu wird eine Methode benötigt, die Tangentialvektoren vom einen
Punkt in einen nahegelegenen Punkt zu transportieren.
Die einfachste Methode, dies zu erreichen, ist eine Metrik $g$ auf der
Mannigfaltigkeit zu verwenden, und Vektoren ``parallel'' zu transportieren,
wie dies in Abschnitt~\ref{buch:zusammenhang:section:paralleltransport}
unternommen wird.
{\em Geodäten} sind kürzeste Verbindungen auf einer Mannigfaltigkeit, sie
lassen sich aber auch als Kurven charakterisieren, die ihren Tangentialvektor
parallel transportieren.
\index{Geodate@Geodäte}%
Dies definiert eine Differentialgleichung zweiter Ordnung für die
Geodäten (Abschnitt~\ref{buch:zusammenhang:section:geodaeten}).
Die kovariante Ableitung von
Abschnitt~\ref{buch:zusammenhang:section:kovarianteableitung}
verallgemeinert den Differentialoperator, der in der Differentialgleichung
der Geodäten vorkommt, für beliebige Vektoren.
Die kovariante Ableitung ist festgelegt, wenn man auf der Mannigfaltigkeit
eine sogenannten {\em Zusammenhang} definiert hat
(Abschnitt~\ref{buch:zusammenhang:section:zusammenhang}). 
\index{Zusammenhang}%
In Abschnitt~\ref{buch:zusammenhang:section:paralleltransport} wurden
Vektoren mit Hilfe des Paralleltransports verglichen.
Ein Zusammenhang verallgemeinert diese Idee und ermöglicht damit auch
Vektoren zu vergleichen, die nicht mit der Metrik verrechnet werden
können.

\input{chapters/100-zusammenhang/1-parallel.tex}

%
% 2-geodaeten.tex
%
% (c) 2025 Prof Dr Andreas Müller
%

%
% Geodäten
%
\section{Geodäten
\label{buch:zusammenhang:section:geodaeten}}
\kopfrechts{Geodäten}
Die kovariante Ableitung war motiviert als ein Kriterium, mit
dem man erkennen kann, ob ein Vektor parallel transportiert
worden ist.
Eine Geodäte kann daher als Kurve definiert werden, entlang der
der Tangentialvektor parallel transportiert wird.
Diese Bedingung äussert sich als Differentialgleichung, die in
Abschnitt~\ref{buch:zusammenhang:geodaeten:subsection:differentiagleichung}
diskutiert wird.
Meist werden Geodäten aber Kurven minimaler Länge zwischen zwei Punkten
definiert.
Sie minimieren das Längenfunktional, welches aus der Metrik gewonnen
werden kann.
Nach den Regeln der Variationsrechnung lässt sich aus dem Längenfunktional
mit der Euler-Lagrange-Differentialgleichung eine Differentialgleichung
für die kürzesten Kurven finden.
In Abschnitt~\ref{buch:zusammenhang:geodaeten:subsection:kuerzeste}
wird gezeigt, dass sie mit der Differentialgleichung übereinstimmt.

%
% Differentialgleichung
%
\subsection{Differentialgleichung für Geodäten
\label{buch:zusammenhang:geodaeten:subsection:differentialgleichung}}
Eine Geodäte ist eine Kurve, entlang der der Tangentialvektor an die
Kurve parallel transportiert wird.
Dies bedeutet, dass in jedem Punkt der Kurve $\gamma(t)$ die kovariante
Ableitung des Tangentialvektors $\dot{\gamma}$ in Richtung
$\dot{\gamma}$ verschwindet.
Es gilt also
\[
\nabla_{\dot{\gamma}(t)} \dot{\gamma}(t)
=
0
\]
Die Komponenten des Tangentialvektors in einer Karte sind $\dot{x}^i(t)$.
Mit den Christoffel-Symbolen ist die kovariante Ableitung
\begin{align*}
\dot{x}^k
\biggl(
\frac{\partial\dot{x}^i}{\partial x^k}
+
\Gamma^i_{kl}\dot{x}^l
\biggr)
&=
0
\\
\frac{\partial\dot{x}^i}{\partial x^k}
\dot{x}^k
+
\Gamma^i_{kl} \dot{x}^l \dot{x}^k
&=
0.
\end{align*}
Der erste Term ist wegen
\[
\frac{dA^i}{dt}
=
\frac{\partial A^i}{\partial x^k}\frac{dx^k}{dt}
=
\frac{\partial A^i}{\partial x^k}
\dot{x}^k
\]
die Ableitung von $\dot{x}^i$ nach der Zeit.

\begin{satz}
Eine Geodäte $\gamma(t)$ erfüllt in jeder Karte die Differentialgleichung
\begin{equation}
\ddot{x}^i
+
\Gamma^i_{kl} \dot{x}^l \dot{x}^k
=
0.
\label{buch:zusammenhang:geodaeten:eqn:dgl}
\end{equation}
\end{satz}

%
% Beispiele
%
\subsection{Beispiele
\label{buch:zusammenhang:geodaeten:subsection:beispiele}}
Falls die Christoffel-Symbole alle verschwinden, bleibt von der
Differentialgleichung~\eqref{buch:zusammenhang:geodaeten:eqn:dgl}
nur noch die Gleichungen
\[
\ddot{x}^i = 0
\]
übrig, welche als Lösungen alle Geraden im $n$-dimensionalen
Raum haben.

\subsubsection{Polarkoordinaten}
Für Polarkoordinaten wurden die Christoffel-Symbole bereits
in Beispiel~\ref{buch:zusammenhang:paralleltransport:bsp:polar}
berechnet.
Die Differentialgleichungen
\begin{align*}
\ddot{x}^i=-\Gamma^i_{kl}\dot{x}^k\dot{x}^l
\end{align*}
können mit den Resultaten von 
Beispiel~\ref{buch:zusammenhang:paralleltransport:bsp:polar}
ausgeschrieben werden als die beiden Differentialgleichungen
\begin{equation}
\left.
\begin{aligned}
\ddot{x}^1
&=
\Gamma^1_{22}(\dot{x}^2)^2
\\
\ddot{x}^2
&=
\Gamma^2_{21}\dot{x}^1\dot{x}^2
+
\Gamma^2_{12}\dot{x}^2\dot{x}^1
\end{aligned}
\quad
\right\}
\qquad
\Rightarrow
\qquad
\left\{
\quad
\begin{aligned}
\ddot{r}
&=
r \dot{\varphi}^2
\\
\ddot{\varphi}
&=
-\frac{2}{r}\dot{\varphi}\dot{r}
\end{aligned}
\right.
\label{buch:zusammenhang:paralleltransport:bsp:polardgl}
\end{equation}
Daraus lässt sich bereits ablesen, dass die radialen Geraden, die
durch $\varphi=\text{const}$ charakterisiert sind, Lösungen
der Differentialgleichung
\eqref{buch:zusammenhang:paralleltransport:bsp:polardgl}
sind.
In diesem Fall ist $\ddot{\varphi}=\dot{\varphi}=0$ und
daher $\ddot{r}=0$, die als Lösung $r=vt+r_0$, wobei $v$ die Geschwindigkeit
und $r_0$ der Radius zur Zeit $t=0$ ist.

Schreibt man $\omega=\dot{\varphi}$ für die Winkelgeschwindigkeit, dann
wird die zweite Gleichung von
\eqref{buch:zusammenhang:paralleltransport:bsp:polardgl}
zu
\[
\frac{\dot{\omega}}{\omega}
=
-2\frac{\dot{r}}{r}
\qquad\Rightarrow\qquad
\frac{d}{dt}\log \omega
=
\frac{d}{dt}\bigl(-2\log r\bigr)
=
\frac{d}{dt}\log\frac{1}{r^2}.
\]
Durch Integration findet man, dass
\begin{equation}
\log \omega = \log\frac{1}{r^2} + C
\qquad\Rightarrow\qquad
\omega = \frac{D}{r^2}
\qquad\Rightarrow\qquad
\omega r^2 = D
\label{buch:zusammenhang:paralleltransport:bsp:drehimpuls}
\end{equation}
mit $D=e^C$ ist.
Multipliziert man die letzte Gleichung mit $m$, erhält man
$m\omega r^2=\text{const}$, also die Aussage, dass der Drehimpuls
erhalten ist.
Jeder Massepunkt, der sich in der Ebene mit konstanter Geschwindigkeit
und konstanter Richtung bewegt, erfüllt diese Gleichung.

Mit der Drehimpulserhaltung
\eqref{buch:zusammenhang:paralleltransport:bsp:drehimpuls}
kann jetzt 
die zweite Differentialgleichung von
\eqref{buch:zusammenhang:paralleltransport:bsp:polardgl}
zu
\begin{equation}
\ddot{r}=r\biggl(\frac{D}{r^2}\biggr)^2 = r^{-3} D^2
\label{buch:zusammenhang:paralleltransport:bsp:polardotr}
\end{equation}
vereinfacht werden.
Schreibt man $r=\sqrt{R}$ oder $r^2=R$, werden die Ableitungen von $R$
\begin{align*}
\dot{R} &= 2\dot{r} r \\
\ddot{R}
&=
2\ddot{r} r + 2\dot{r}^2
=
\\
\dddot{R}
&=
2\dddot{r} r + 2\ddot{r}\dot{r} + 4\ddot{r}\dot{r}
=
2\dddot{r}r + 6\ddot{r}\dot{r}
\intertext{oder nach Einsetzen von 
\eqref{buch:zusammenhang:paralleltransport:bsp:polardotr}
und dessen Ableitung
}
&=
2(-3\dot{r}r^{-4}D^2) r + 6r^{-3}D^2\dot{r}
=
6\dot{r}D^2(-r^{-3}+r^{-3})
=
0.
\end{align*}
Es folgt, dass $R$ ein quadratisches Polynom in $t$ sein muss.
Es muss von der Form $R=at^2 + bt + c$ sein.
Da $R(t)\ge 0$ sein muss, muss $a \ge 0$ sein und die Diskriminante $\Delta$
muss 
\[
\Delta
=
b^2-4ac\le 0
\]
sein.
Für $a>0$ kann man dies durch quadratisches Ergänzen in die Form
\[
R(t)
=
a\biggl(t+\frac{b}{2a}\biggr)^2 - \frac{b^2-4ac}{4a^2}
\]
bringen.
Das Minimum wird erreicht wenn $t=-b/2a$ ist, daher kann man die
Parameter wie folgt interpretieren.
\input{chapters/100-zusammenhang/fig/fig-polargeodaete.tex}%
Schreibt man
\[
t_0=-b/2a,
\qquad
v=\sqrt{a}
\qquad\text{und}\qquad
d^2 = -\frac{b^2-4ac}{4a^2},
\]
folgt
\[
R(t)
=
(v(t-t_0))^2 + d^2
\qquad\Rightarrow\qquad
r(t)
=
\sqrt{
(v(t-t_0))^2 + d^2
}.
\]
Daraus kann man mit Hilfe des Satzes von Pythagoras ablesen, dass sich
der Punkt mit der Geschwindigkeit $v$ auf einer Geraden bewegt und
zur Zeit $t_0$ im geringsten Abstand $d$ am Nullpunkt vorbeigeht
(Abbildung~\ref{buch:zusammenhang:geodaeten:fig:polargeodaete}).
Damit lässt sich jetzt auch die Konstante $D$ bestimmen.
Aus $v=\omega d$ folgt
\[
\omega=\frac{d}{v}
\quad \Rightarrow \quad
D=\omega d^2 = \frac{d^3}{v}
\]

Aus $\dot{\varphi}=\omega = Dr^{-2} = d^3 / v r^2$
kann jetzt durch Integration auch die explizite Formel 
\[
\varphi(t)
=
\varphi_0
+
\frac{d}{v}
\int_{t_0}^t
\frac{d^2\,dt}{v^2(t-t_0)^2+d^2}
=
\varphi_0
+
\frac{d}{v}
\arctan\frac{(t-t_0)v}{d}
\]
für $\varphi(t)$ gefunden werden.

\subsubsection{Geodäten auf einer Kugeloberfläche}
In Beispiel~\ref{buch:zusammenhang:paralleltransport:bsp:kugel}
wurden die Christoffel-Symbole für Kugelkoordinaten auf einer
Kugeloberfläche berechnet.
\input{chapters/100-zusammenhang/fig/kugeldreieck.tex}%
Damit lässt sich jetzt die Differentialgleichung für eine Geodäte auf
einer Kugel finden.
Sie ist
\begin{equation}
\left.
\begin{aligned}
\ddot{x}^1 &= -\Gamma^1_{ik}\dot{x}^i\dot{x}^k
\\
\ddot{x}^2 &= -\Gamma^2_{ik}\dot{x}^i\dot{x}^k
\end{aligned}
\quad
\right\}
\qquad\Rightarrow\qquad
\left\{
\quad
\begin{aligned}
\ddot{\vartheta}
&=
\sin\vartheta\cos\vartheta \cdot \dot{\varphi}^2
\\
\ddot{\varphi}
&=
-\cot\vartheta \cdot \dot{\vartheta}\dot{\varphi}.
\end{aligned}
\right.
\label{buch:zusammenhang:geodaeten:kugeldgl}
\end{equation}
Der Äquator ist durch $\vartheta=\frac{\pi}2$ definiert.
Für diesen Wert werden die Differentialgleichungen für $\varphi$ zu
\[
\ddot{\varphi}=0,
\]
die als Lösung eine lineare Funktion
$\varphi(0)=\omega t +\varphi_0$ hat.

Ein Längenkreis ist gegeben durch $\varphi=\text{const}$ oder
$\dot{\varphi}=0$, damit wird die Differentialgleichung für $\vartheta$
zu
\[
\ddot{\vartheta} = 0,
\]
die als Lösung eine lineare Funktion der Form
$\vartheta(t) = \omega t + \vartheta_0$ hat.

Andere Geodäten schneiden den Äquator, wir können ohne Beschränkung der
Allgemeinheit annehmen, dass dies für $\varphi=0$ geschieht.
In Abbildung~\ref{buch:zusammenhang:geodaeten:fig:kugeldreieck}
ist rot der Grosskreis durch den Punkt $A$ bei $\varphi=0$ dargestellt,
der den Äquator im Winkel $\alpha$ schneidet.
Als Parameter wird die Länge $c=t$ der Hypothenuse des rechtwinkligen
Dreiecks verwendet.
Der sphärische Sinussatz liefert die Beziehung
\[
\sin a : \sin \alpha
=
\sin c : \sin\gamma
=
\sin c,
\]
wegen $\sin\gamma=\sin\frac{\pi}2=1$.
Durch einsetzen der Werte aus der
Abbildung~\ref{buch:zusammenhang:geodaeten:fig:kugeldreieck}
erhalten wir
\begin{align*}
\sin\biggl(\frac{\pi}2-\vartheta\biggr) : \sin\alpha
&=
\sin t
\\
\cos\vartheta
&=
\sin\alpha \sin t
&&\Rightarrow&
\vartheta(t) &= \arccos(\sin\alpha\sin t).
\end{align*}
Für die Berechnung von $\varphi$ kann der sphärische Kosinussatz
für rechtwinklige sphärische Dreiecke in der Form
\begin{align*}
\cos c = \cos a\cos b
\quad\Rightarrow\quad
\cos\varphi\cos\biggl(\frac{\pi}2-\vartheta(t)\biggr)
&=
\cos t
\\
\cos\varphi
&= 
\frac{
\cos t
}{
\sin\vartheta(t)
}.
\end{align*}
Daraus erhalten wir als vollständige Parametrisierung des Grosskreises
\begin{equation}
\begin{aligned}
\vartheta(t)
&=
\arccos(\sin\alpha\cos t)
\\
\varphi(t)
&=
\arccos\frac{\cos t}{\sin\vartheta(t)}.
\end{aligned}
\label{buch:zusammenhang:geodaeten:eqn:kugelgeodaeten}
\end{equation}
Durch Nachrechnen kann geprüft werden, dass diese Funktionen tatsächlich
eine Lösung der Geodätendifferentialgleichung ergeben.
Die Rechnung ist allerdings ziemlich mühsam, die Verwendung eines
Computeralgebrasystems ist empfohlen.


%
% Geodäten als kürzeste Verbindungen
%
\subsection{Geodäten als kürzeste Verbindungen
\label{buch:zusammenhang:geodaeten:subsection:kuerzeste}}
Eine Kurve zwischen zwei Punkten kann in einer Karte durch eine Funktion
\(
t\mapsto x^i(t)
\)
beschrieben werden.
Die Länge dieser Kurve zwischen den Parameterwerten $t_A$ und $t_B$
ist durch das Integral
\begin{equation}
l
=
\int_{t_A}^{t_B}
\sqrt{g_{ik}(x) \dot{x}^i \dot{x}^k }
\,dt.
\label{buch:zusammenhang:geodaeten:eqn:funktional}
\end{equation}
Die Länge der Kurve ist nicht abhängig von der Parametrisierung, es ist
daher keine Einschränkung, ein festes Parameterintervall zu verwenden.

Die Lagrange-Funktion des Funktionals
\eqref{buch:zusammenhang:geodaeten:eqn:funktional}
ist 
\[
L(x^i, \dot{x}^i)
=
\sqrt{ g_{ik}(x) \dot{x}^i \dot{x}^k }.
\]
Die Wurzel führt dazu, dass die für die Euler-Lagrange-Differentialgleichung
nötigen partiellen Ableitungen von $L$ einen hässlichen Nenner haben.
Dies kann vermieden werden, indem die Unabhängigkeit der Weglänge von der
Parametrisierung ausgenutzt wird.
Verwendet man als Parameter die Weglänge, dann ist $L(x^i,\dot{x}^i)$
konstant.
Wir schreiben daher
\[
F(x^i,\dot{x}^i)
=
\frac12 L(x^i,\dot{x}^i)^2
\]
und verlangen von der Parametrisierung, dass 
\[
\frac{d}{dt} L(x^i, \dot{x}^i) = 0
\]
ist.
Die partiellen Ableitungen für die Euler-Lagrange-Differentialgleichungen
\begin{align*}
\frac{\partial F}{\partial x^l}
&=
L(x^i,\dot{x}^i)\, \frac{\partial L}{\partial x^l} (x^i,\dot{x}^i)
\\
\frac{\partial F}{\partial \dot{x}^l}
&=
L(x^i,\dot{x}^i)\, \frac{\partial L}{\partial \dot{x}^l} (x^i,\dot{x}^i).
\end{align*}
Damit wird die Euler-Lagrange-Differentialgleichung für $F$ zu
\begin{align*}
\frac{\partial F}{\partial x^l}
-
\frac{d}{dt}\frac{\partial F}{\partial \dot{x}^l}
&=
L
\frac{\partial L}{\partial x^l}
-
\frac{d}{dt}\biggl(
L
\frac{\partial L}{\partial\dot{x}^l}
\biggr)
\\
&=
L(x^i,\dot{x}^i)
\frac{\partial L}{\partial x^l}
-
\frac{dL}{dt}
\frac{\partial L}{\partial\dot{x}^l}
-
L
\frac{d}{dt}
\frac{\partial L}{\partial\dot{x}^l}.
\intertext{Da $L$ konstant ist, fällt der mittlere Term weg und es
bleibt}
\frac{\partial F}{\partial x^l}
-
\frac{d}{dt}\frac{\partial F}{\partial \dot{x}^l}
&=
L
\cdot
\biggl(
\frac{\partial L}{\partial x^l}
-
\frac{d}{dt}
\frac{\partial L}{\partial\dot{x}^l}
\biggr).
\end{align*}
Da $L$ konstant und verschieden von $0$ ist, verschwindet die rechte
Seite nur dann, wenn auch die linke Seite verschwindet.
Statt der Lagrange-Funktion $L(x^i,\dot{x}^i)$ kann also mit gleichem
Resultat die Lagrange-Funktion $f(x^i,\dot{x}^i)$ verwendet
werden.

Die partiellen Ableitungen der Lagrange-Funktion sind
\begin{align*}
\frac{\partial F}{\partial x^l}
&=
\frac12
\frac{\partial g_{ik}}{\partial x^l}
\dot{x}^i\dot{x}^k
\\
\frac{\partial F}{\partial \dot{x}^l}
&=
\frac12
g_{ik}(\dot{x}^i\delta_i^l\dot{x}^k + \dot{x}^i\dot{x}^k\delta_k^l)
=
\frac12
\bigl(
g_{lk}\dot{x}^k
+
g_{il}\dot{x}^i
\bigr).
\intertext{Für die Euler-Lagrange-Differentialgleichung ist jetzt auch
noch die Ableitung nach $t$ zu bestimmen:}
\frac{d}{dt}
\frac{\partial F}{\partial x^l}
&=
\frac12
\biggl(
\frac{\partial g_{lk}}{\partial x^s}\dot{x}^s\dot{x}^k
+
g_{lk}\ddot{x}^k
+
\frac{\partial g_{il}}{\partial x^s}\dot{x}^s\dot{x}^i
+
g_{il}\ddot{x}^i
\biggr).
\end{align*}
Damit wird die Euler-Lagrange-Differentialgleichung
\begin{align*}
0
&=
\frac{\partial F}{\partial x^l}
-
\frac{d}{dt}\frac{\partial F}{\partial\dot{x}^l}
\\
&=
\frac12
\biggl(
\frac{\partial g_{ik}}{\partial x^l}
\dot{x}^i\dot{x}^k
-
\frac{\partial g_{lk}}{\partial x^s}\dot{x}^s\dot{x}^k
-
\frac{\partial g_{il}}{\partial x^s}\dot{x}^s\dot{x}^i
\biggr)
-
\frac12
g_{lk}\ddot{x}^k
-
\frac12
g_{lk}\ddot{x}^k
\\
&=
\frac12
\biggl(
\frac{\partial g_{ik}}{\partial x^l}
\dot{x}^i\dot{x}^k
-
\frac{\partial g_{lk}}{\partial x^i}\dot{x}^i\dot{x}^k
-
\frac{\partial g_{il}}{\partial x^k}\dot{x}^k\dot{x}^i
\biggr)
-
g_{lk}\ddot{x}^k
\\
&=
\frac12
\biggl(
\frac{\partial g_{ik}}{\partial x^l}
-
\frac{\partial g_{lk}}{\partial x^i}
-
\frac{\partial g_{il}}{\partial x^k}
\biggr)
\dot{x}^i\dot{x}^k
-
g_{lk}\ddot{x}^k
\\
&=
-\Gamma_{l,ik} \dot{x}^i\dot{x}^k
-
g_{lk} \ddot{x}^k.
\end{align*}
Multipliziert man dies mit den Einträgen $g^{jl}$ der inversen
Matrix von $g$ und summiert über $l$, ergibt sich
\[
0
=
g^{jl}\Gamma_{l,ik}\dot{x}^i\dot{x}^k+g^{jl}g_{lk}\ddot{x}^k
=
\Gamma^j_{ik}\dot{x}^i\dot{x}^k + \delta^j_k\ddot{x}^k
=
\Gamma^j_{ik}\dot{x}^i\dot{x}^k + \ddot{x}^j.
\]
Die Euler-Lagrange-Differentialgleichung für das Weglängenfunktional
ist also die Differentialgleichung für Geodäten.

%
% Die Exponentialabbildung
%
\subsection{Die Exponentialabbildung
\label{buch:zusammenhang:subsection:exponentialabbildung}}
Sei jetzt $p\in M$ ein Punkt einer $n$-dimensionalen riemannschen
Mannigfaltigkeit und $\vec{v}\in T_pM$ ein Tangentialvektor in $p$.
Die Differentialgleichung der Geodäten garantiert, dass es mindestens
in einer Umgebung von $0\subset\mathbb{R}$ eine Lösungskurve
$\gamma(t)$ gibt mit $\gamma(0)=p$ und $\dot{\gamma}(t)=\vec{v}$.

\begin{definition}[Exponentialabbildung]
\label{buch:zusammenhang:geodaeten:def:exponentialabbildung}
Die \emph{Exponentialabbildung} im Punkt $p$ einer riemannschen
\index{Exponentialabbildung}%
Mannigfaltigkeit $M$ ist diejenige Abbildung einer Umgebung $U\subset T_pM$
nach $M$ mit der Eigenschaft, dass $\exp_p(t\vec{v})$ eine Geodäte
durch $p$ mit Tangentialvektor
\[
\frac{d}{dt}
\exp_p(t\vec{v})\bigg|_{t=0}
=
\vec{v}
\]
ist.
\end{definition}

Gleichbedeutend mit der
Definition~\ref{buch:zusammenhang:geodaeten:def:exponentialabbildung}
ist die folgende Konstruktion.
Um das Bild $\exp_p(\vec{v})$ zu finden, konstruiert man die
Geodäte $\gamma(t)$ durch $p$ mit Richtungsvektor $\dot{\gamma}(0)=\vec{v}$.
Dann ist $\exp_p(\vec{v})=\gamma(1)$.
Diese Definition verlangt aber, dass die Kurve $\gamma(t)$ auch tatsächlich
für beliebige Zeit $t$ definiert ist.
Die üblichen Sätze über gewöhnliche Differentialgleichungen wie der
Satz von Picard-Lindelöf garantieren die Existenz einer Lösung nur
\index{Satz von Picard-Lindelöf}%
für eine offene Umgebung von $t=0$.

Der Satz von Picard-Lindelöf gilt für eine Differentialgleichung
in der Form $\dot{x} = f(t,x)$ mit einer Funktion
$f\colon\mathbb{R}\times\mathbb{R}^n \to \mathbb{R}^n$, die im zweiten
Argument eine Lipschitz-Bedingung erfüllt.
\index{Lipschitz-Bedingung}%
Diese garantiert, dass die Ableitung $\dot{x}$ nicht zu gross wird und
damit die Lösung $x(t)$ divergiert.
Es ist wohlbekannt, dass für stärkere Bedingungen an die die Funktion $f$ 
die Lösung für beliebige Zeit $t$ existiert.
Zum Beispiel ist die Lösung der linearen Differentialgleichung
$\dot{x}=Ax$ mit einer Matrix $A\in M_{n\times n}(\mathbb{R})$
durch die Matrixexponentialfunktion $x(t) = e^{At}x_0$ gegeben, die
für beliebige $t\in\mathbb{R}$ definiert ist.
Sogar wenn die Matrix $A$ in einem Intervall $I\subset\mathbb{R}$
auf glatte Art von $t$ abhängt lässt sich zeigen, dass sich eine
Lösung immer auf das ganze Intervall ausdehnen lässt.
Ein Beweis wird in \cite[Theorem 4.31]{buch:leerm} gegeben.

Die Differentialgleichung der Geodäten erfüllt aber eine noch 
stärkere Bedingung.
Entlang einer Lösung wird der Tangentialvektor $\dot{\gamma}(t)$ 
für alle Zeiten, für die die Lösung definiert ist, parallel transportiert.
Insbesondere bleibt seine Länge immer gleich gross, es besteht gar
nicht die Gefahr, dass $\dot{\gamma}(t)$ divergiert.
Was hingegen geschehen könnte ist, dass $\gamma(t)$ in $M$ nicht
mehr definiert werden kann.

Es gibt natürlich Fälle, wo es unvermeidlich ist, dass $\gamma(t)$
nur für ein Intervall $(a,b)$ definiert sein kann.
In einer beschränkten offenen Menge in $\mathbb{R}^n$ sind die Geodäten
Geraden.
Sie sind nur solange definiert, als $\gamma(t)$ die Menge
nicht verlässt.
Auf einer kompakten Mannigfaltigkeit ohne Rand kann dies nicht passieren,
dort dürfen wir annehmen, dass der Grenzwert $\gamma(t)$ für $t\to b$
wieder in $M$ liegt.
Das Argument zeigt also, dass eine Geodäte nicht irgendwo im inneren
einer riemannschen Mannigfaltigkeit enden kann.
Wenn der Grenzwert $\gamma(t)$ für $t\to b$ in $M$ existiert, dann
lässt sich die Geodäte auch darüber hinaus fortsetzen.

Wenn sich die Geodäte auf beliebige Parameter $t\in\mathbb{R}$
ausdehen lässt, nennen wir die Mannigfaltigkeit geodätisch vollständig.
Als formale Definition verwenden wir die folgende.

\begin{definition}[geodätisch vollständig]
\label{buch:zusammenhang:geodaeten:def:vollst}
Eine riemannsche Mannigfaltigkeit $M$ heisst \emph{geodätisch vollständig},
wenn die Exponentialabbildung $\exp_p$ für jeden Punkt $p\in M$
auf ganz $T_pM$ definiert ist.
\index{geodätisch vollständig}%
\end{definition}

In vielen Fällen ist die Voraussetzung der geodätischen Vollständigkeit
nicht nötig.
Die allgemeine Relativitätstheorie beschreibt zum Beispiel die Bahnen
von Teilchen oder Lichtstrahlen als Geodäten in einer pseudoriemannschen
Mannigfaltigkeit.
\index{Universum}%
Wir wissen nicht, ob das Universum geodätisch vollständig ist im
mathematischen Sinn der
Definition~\ref{buch:zusammenhang:geodaeten:def:vollst}.
Wir dürfen aber davon ausgehen, dass die Exponentialabbildung für alle
praktisch sinnvollen Tangentialvektoren definiert ist.

Die Definition der Exponentialabbildung verlangt, dass man entlang einer
Geodäten vom Ausgangspunkt $p$ zum Endpunkt $\exp_p(\vec{v})$ gelangen
kann.
Der Definitionsbereich der Exponentialabbildung muss also die Eigenschaft
haben, dass mit jedem Vektor $\vec{v}$ auch die Strecke zwischen dem Nullpunkt
und $\vec{v}$ im Definitionsbereich enthalten ist.
Solche Mengen heissen \emph{sternförmig}.

\begin{definition}[sternförmig]
\label{buch:zusammenhang:geodaeten:def:sternfoermig}
Eine Teilmenge $U\subset V$ eines Vektorraumes heisst \emph{sternförmig},
wenn mit jedem Punkt $u\in U$ auch alle skalierten Vektoren $tu\in U$
mit $t\in [0,1]$ in $U$ sind.
\index{sternformig@sternförmig}%
\end{definition}

%
% Normalkoordinaten
%
\subsection{Normalkoordinaten}
Die Exponentialabbildung kann dazu verwendet werden, in der
Umgebung eines Punktes der Mannigfaltigkeit ein Koordinatensystem
zu definieren, in dem die Christoffel-Symbole besonders einfache
Form annehmen.

\begin{definition}[Normale Umgebung]
Eine \emph{normale Umgebung} eines Punktes $p$ in einer 
riemannschen Mannigfaltigkeit ist eine offene Menge $U\subset M$,
die diffeomorphes Bild einer sternförmigen Menge in $T_pM$ ist.
\end{definition}

In einer normalen Umgebung kann man jetzt Koordinaten einführen,
indem man in $T_pM$ eine Basis wählt.
Bilden die Vektoren $b_1,\dots,b_n$ eine Basis, dann gibt es reelle
Zahlen $v^1,\dots,v^n$ mit 
\[
\exp_p(v^1b_1+\dots+v^nb_n)=u.
\]
Diese Abbildung ist die Zusammensetzung der 
Abbildung
\[
B
\colon
\mathbb{R}^n \to T_pM
:
(v^1,\dots,v^n)
\mapsto
v^1b_1+\dots+v^nb_n
\]
mit der Exponentialabbildung.
Sie ist definiert, solange $B\xi$ im Definitionsbereich von $\exp_p$
enthalten ist.
Die Umkehrung $B^{-1}\circ\exp_p^{-1}$ ist eine Karte auf $U$.

Wählt man für $b_1,\dots,b_n$ eine orthonormale Basis von $T_pM$,
dann entsteht eine Karte mit sogenannten \emph{Normalkoordinaten}
zentriert im Punkt $p$,
die sich durch besondere Eigenschaften auszeichnen.
Zunächst sind nach Konstruktion die zur Basis gehörigen Tangentialvektoren
orthonormiert, so dass der metrische Tensor durch $g_{ik}=\delta_{ik}$
gegeben ist.

\begin{satz}[Eigenschaften von Normalkoordinaten]
\label{zusammenhang:geodaeten:satz:normalkoordinaten}
Sei $M$ eine $n$-dimensionale riemannsche Mannigfaltigkeit und $(U,x^i)$
ein Normalkoordinatensystem im Punkt $p\in M$.
Dann gilt
\begin{enumerate}
\item Die Koordinaten von $p$  sind $(0,\dots,0)$.
\item Die Komponenten des metrischen Tensors in $p$ sind $g_{ik}=\delta_{ik}$.
\item Für jeden Tangentialvektor $v=v^i\partial_i$ ist die Geodäte mit
Richtung $v$ in Normalkoordinaten gegeben durch $(tv^1,\dots,tv^n)$.
\item Die Christoffelsymbole $\Gamma^i_{kl}$ verschwinden im Punkt $p$.
\item Alle ersten partiellen Ableitungen der metrischen Koeffizienten
verschwinden im Punkt $p$.
\item Die ersten partiellen Ableitungen von $g^{ik}$ verschwinden im Punkt $p$.
\end{enumerate}
\end{satz}

\begin{proof}
Die ersten drei Eigenshaften ergeben sich unmittelbar aus der Definition
der Normalkoordinaten.
Da die Geodäten Geraden sind, verschwinden die zweiten Ableitungen
der Koordinaten eines Geodätenpunkts.
Die Geodätengleichungen reduzieren sich damit auf
\[
\Gamma^i_{kl} v^kv^l = 0
\]
im Punkt $p$.
Durch Einsetzen der Basisvektoren $b_a$ mit Koordinaten $\delta_a^i$
bekommt man daraus zunächst
\begin{equation}
0
=
\Gamma^i_{kl}v^kv^l
=
\Gamma^i_{kl}\delta_a^k\delta_a^l
=
\Gamma^i_{aa}.
\label{buch:zusammenhang:geodaeten:eqn:Giaa}
\end{equation}
Setzt man Linearkombinationen $b_a\pm b_b$ mit den Koordinaten
$\delta_a^i\pm \delta_b^i$ für $v$ ein, erhält man
\begin{align*}
0
&=
\Gamma^i_{kl} (\delta_a^k \pm \delta_b^k)(\delta_a^l \pm \delta_b^l)
\\
&=
\Gamma^i_{kl}(
    \delta_a^k\delta_a^l
\pm \delta_a^k\delta_b^l
\pm \delta_b^k\delta_a^l
  + \delta_b^k\delta_b^l
)
\\
&=
\Gamma^i_{aa}
\pm
\Gamma^i_{ab}
\pm
\Gamma^i_{ba}
+
\Gamma^i_{bb}
\\
&=
\pm
2
\Gamma^i_{ab},
\end{align*}
wobei sowohl
\eqref{buch:zusammenhang:geodaeten:eqn:Giaa}
als auch die Symmetrie der Christoffelsymbole verwendet wurden.

Die Christoffelsymbole des Levi-Cività-Zusammenhangs ergaben sich
als Lösungen des Gleichungssystems~\ref{buch:zusammenhang:kovabl:eqn:gGamma}.
Im Punkt $p$ vereinfachen Sie sich wegen $g_{ik}=\delta_{ik}$ zu
\[
\frac{\partial g_{il}}{\partial x_k}
=
g_{sl}\Gamma^s_{ik}
+
g_{is}\Gamma^s_{lk}
=
\delta_{sl}\Gamma^s_{ik}
+
\delta_{is}\Gamma^s_{lk}
=
\Gamma^l_{ik}
+
\Gamma^i_{lk}
=
0,
\]
da die Christoffelsymbole verschwinden.

Auch die Ableitungen von $g^{ik}$ verschwinden im Punkt $0$.
Dies lässt sich noch viel allgemeiner beweisen.
Seien $A(t)$ und $B(t)$ inverse Matrizen mit $A(0)=B(0)=I$, die von einem
Parameter $t$ abhängen.
Ausserdem sei $\dot{A}(0)=0$
Dann ist die Ableitung des Produktes $A(t)B(t)=I$
\begin{align*}
\frac{d}{dt}I\bigg|_{t=0}
&=
\biggl(
\frac{dA}{dt}(t) B(t)
+
A(t)
\frac{dB}{dt}(t)
\biggr)\bigg|_{t=0}
\\
&=
\dot{A}(0)
B(0)
+
A(0)
\dot{B}(0)
=
\dot{A}(0)
+
\dot{B}(0)
=
\dot{B}(0).
\end{align*}
Somit verschwindet auch die erste Ableitung der inversen Matrix von $A$.
Angewendet auf $A=g_{ik}$ und $B=g^{ik}$ und eine der Koordinaten als
Parameter folgt, dass die partiellen ersten Ableitungen von $g^{ik}$
an der Stelle $p$ ebenfalls verschwinden.
\end{proof}

Die speziell einfache Form der Christoffelsymbole in Normalkoordinaten
wird später ermöglichen, auch den riemannschen Krümmungstensor an der
Stelle zu berechnen.





%
% 3-divergenz.tex
%
% (c) 2025 Prof Dr Andreas Müller
%

%
% Kovariante Ableitung von Tensoren
%
\section{Divergenz und Erhaltungssätze
\label{buch:zusammenhang:section:divergenz}}
\kopfrechts{Divergenz und Erhaltungssätze}
In Abschnitt~\ref{buch:gauss:section:erhaltungssatz} wurde gezeigt,
wie Erhaltungssätze mit Hilfe der Divergenz formuliert werden können.
Beschreibt man den Fluss eines Feldes durch die $n-1$-dimensionale
Volumenelemente durch eine $n-1$-Form $\omega\in \Omega^{n-1}(M)$,
dann ist die äussere Ableitung $d\omega\in \Omega^n(M)$ eine $n$-Form.
Der Satz von Gauss sagt dann, dass für ein $n$-dimensionales $V$
mit dem geschlossenen, $n-1$-dimensionalen Rand $\partial V$
\[
\int_{V}d\omega
=
\int_{\partial V}\omega
\]
gilt.
Die $n$-Form $d\omega$ beschreibt die Quelldichte für das Feld im
Gebiet $V$.
Da man $d\omega$ in der Form
\[
d\omega
=
\varrho(x^1,\dots,x^n)
\,
dx^1\wedge\dots\wedge dx^n
\]
schreiben kann, betrachtet man $\varrho(x^1,\dots,x^n)$ als die
Dichte der erhaltenen Grösse.
Die Dichtefunktion $\varrho(x^1,\dots,x^n)$  ist aber abhängig
von der Wahl des Koordinatensystems, sie kann also nicht direkt 
mit einer physikalischen Grösse identifiziert werden.

Die äussere Ableitung $d\omega$ wurde mit der Divergenz des Vektorfeldes
identifiziert, dessen Komponenten die Koeffizienten der $n$ Basis-$n-1$-Formen
$dx^{i_1}\wedge\dots\wedge dx^{i_{n-1}}$, $1\le i_1<\dots<i_{n-1}\le n$, in
einem gewählten Koordinatensystem sind.
Die Kontinuitätsgleichung wurde in Abschnitt
\ref{buch:zusammenhang:erhaltungssatz:subsection:kontinuitaetsgleichung}
als der Ausdruck eines Erhaltungssatzes.
Die Komponenten des Vektorfeldes bilden den zur Dichte $\varrho$ gehörigen
erhaltenen Strom.
Auch dieses Vektorfeld hängt, genau wie die Dichtefunktion
$\varrho(x^1,\dots,x^n)$, von der Wahl des Koordinatensystems ab.

%
% Metrik und Dichte
%
\subsection{Metrik und Dichte
\label{buch:zusammenhang:divergenz:subsection:metrik}}
Die Dichtefunktion $\varrho(x^1,\dots,x^n)$ liess sich nicht 
unabhängig vom Koordinatensystem definieren, weil es keine vom
Koorinatensystem unabhängig Wahl einer Volumenform
\[
v(x^1,\dots,x^n)\,dx^1\wedge\dots\wedge dx^n
\]
gibt.
Eine Orientierung der Mannigfaltigkeit verlang zwar, dass auf
jedem Kartengebiet eine solche Form ausgezeichnet wird.
Von den Kartenwechseln wird aber nur verlangt, dass sie das Vorzeichen
der $n$-Form nicht ändern.
Es wird nichts über die Werte der $n$-Formen verlangt.

Mit einer Metrik wird es möglich, eine koordinatenunabhängige Volumenform
\[
\sqrt{\det g}
\,dx^1\wedge\dots\wedge dx^n
\]
zu definieren.
Eine beliebige $n$-Form $\omega$ kann dann in der Form
\[
\varrho(x^1,\dots,x^n)
\sqrt{\det g}
\,dx^1\wedge \dots \wedge dx^n
\]
geschrieben werden.
Die Dichtefunktion $\varrho(x^1,\dots,x^n)$ ist jetzt eine von der
Wahl des Koordinatensystems unabhängige Grösse.

Eine entsprechende ``Referenzform'' lässt sich auch für $p$-Formen mit
$p<n$ finden, die auf einer eingebetteten $p$-Mannifaltigkeit
definiert sind.
Ein $p$-dimensionale Untermannigfaltigkeit $N\subset M$ einer
riemannschen Mannigfaltigkeit $M$ mit der Metrik $g$ erbt die Metrik
von $M$.
Wegen $N\subset M$ folgt $T_xN\subset T_xM$ für jeden Punkt $x\in N$
und daher sind Tangentialvektoren von $N$ im Punkt $x$ auch
Tangentialvektoren von $M$ im selben Punkt.
Die Metrik von $M$ definiert daher auch ein Skalarprodukt von
Tangentialvektoren von $N$ und damit eine Metrik auf $N$.
Die Länge einer Kurve in $N$ wird natürlich nicht verändert.

Eine $p$-dimensionale Untermannigfaltigkeit hat daher eine $p$-dimensionale
Volumenform, die mit der gleichen Metrik definiert ist.
Auf dem $p-1$-dimensionalen Rand eines $p$-dimensionalen Gebietes definiert
die Metrik eine $p-1$-Form, die mit der $p$-Form auf dem Gebiet kompatibel
ist.

%
% Symmetrien und Erhaltungssätze
%
\subsection{Symmetrien und Erhaltungssätze
\label{buch:zusammenhang:divergenz:subsection:noether}}
Aus der Variationsrechnung \cite[Abschnitt 10.2]{buch:seminarvariation}
ist bekannt, dass zu jeder Symmetrie der Lagrange-Funktion eines
Variationsproblems ein Erhaltungssatz gehört.

XXX Divergenzform des Erhaltungssatzes

%
% Kovariante Ableitung und erhaltene Ströme
%
\subsection{Kovariante Ableitung und erhaltene Ströme
\label{buch:zusammenhang:divergenz:subsetion:kovariant}}
Die Divergenzform ermöglicht, Erhaltungssätze in Form einer 
Differentialgleichung zu formulieren.
Die Koeffizienten der Basis-$p$-Formen sind aber koordinatenabhängig,
wir können also nicht erwarten, damit einen koordinatenunabhängig
formulierten Erhaltungssatz zu konstruieren.

Die Divergenzform des Satzes von Nöther zeigt uns aber die Richtung,
in die wir gehen müssen.
Da die $n$-Form $v=\sqrt{g} dx^1\wedge \dots\wedge dx^n$
koordinatenunabhängig ist, sind Grössen, die als Dichtefunktionen
relativ zu $v$ definiert sind, Kandidaten für Grössen, für die
ein Erhaltungssatz in Divergenzform formuliert werden kann.

Etwas formaler betrachten wir jetzt eine Grösse mit Komponenten $A^i$
derart, dass $\sqrt{g}A^i$ ein Vektor ist.
Wir bezeichnen eine solche Grösse als eine Vektordichte.
In Divergenzform lautet ein Erhaltungssatz für diese Grösse
\[
\sum_{i=1}^n \frac{\partial}{\partial x^i} (\!\sqrt{g}A^i)
=
0.
\]


%
% 4-zusammenhang.tex
%
% (c) 2025 Prof Dr Andreas Müller
%

%
% Zusammenhang
%
\section{Zusammenhang
\label{buch:zusammenhang:section:zusammenhang}}
\kopfrechts{Zusammenhang}
In der bisherigen Entwicklung wurde die kovariante Ableitung vor allem
aus der Darstellung~\eqref{buch:zusammenhang:paralleltransport:eqn:kovabl}
durch Berechnung der Christoffel-Symbole und das Studium ihrer
Eigenschaften entwickelt.
Damit lassen sich weitere Abstraktionen vermeiden.
Es ist aber auch möglich, die kovariante Ableitung als ein Operator
mit geeigneten Eigenschaften axiomatisch zu beschreiben und dann zu
zeigen, dass es genau einen solchen Operator gibt.
Dies verlangt etwas abstraktere Konstruktionen, ermöglicht aber
die Verallgmeinerung auf beliebige Tensorfelder.
Diese Vorgehensweise soll in diesem Abschnitt durchgeführt werden.

\subsection{Vektorbündel}
Das Tangentialbündel $TM$ einer differenzierbaren Mannigfaltigkeit
bestand für jeden Punkt $p\in M$ aus einem Vektorraum $T_pM$, dessen
Elemente wir mit Differentialoperatoren identifiziert haben, die
auf Funktionen auf der Mannigfaltigkeit wirken.
Diese letzte Eigenschaft ist nur für Tangentialvektoren sinnvoll.
Um zum Beispiel ein Kraftfeld auf einer Mannigfaltigkeit zu beschreiben,
möchten wir ebenfalls jedem Punkt der Mannigfaltigkeit Vektoren
zuordnen können, aber diese Vektoren haben nichts mit den Funktionen
auf der Mannigfaltigkeit zu tun.
Wir müssen also das Konzept des Tangentialbündels noch etwas
erweitern, so dass in jedem Punkt Vektoren aus einem beliebigen
Vektorraum ausgewählt werden können.

Auch aus rein mathematischer Motivation ist es nötig, die Idee des
Tangentialbündels zu erweitern.
Zum Beispiel bilden die $k$-Formen in einem Punkt einen
$\binom{n}{k}$-dimensionalen, reellen Vektorraum.
Eine Verallgemeinerung des Konzepts des Tangentialbündels ist also
bereits aufgetreten, wir haben es nur nicht abstrakt definiert.
Dies soll jetzt nachgeholt werden.

Das Tangentialbündel ist nicht einfach nur eine Menge von Vektoren.
In jedem Punkt $p\in M$ gibt es den Vektorraum $T_pM$ der Tangentialvektoren.
Es gibt aber auch ein Konzept der Stetigkeit für Tangentialvektorfelder.
Ein solches ist eine Funktion $X\colon M\mapsto TM$ derart, dass 
$X(p)\in T_pM$ ist.
Ausserdem sollen Vektoren $X(p)$ und $X(q)$ in zwei Punkten $p,q\in M$,
die nahe beeinander liegen, ebenfalls nahe beeinander sein.
Weiter möchten wir mit Koordinaten arbeiten können, so wie das für $M$
selbst auch möglich ist.
Schliesslich möchten wir die Vektorraumstruktur in verschiedenen
Punkten miteinander vergleichen können.
Für das Tangentialbündel ist jeder Tangentialraum $T_pM$ ein
$n$-dimensionaler reeller Vektorraum, also $T_pM \equiv \mathbb{R}^n$.
Alle diese Forderungen führen auf die folgende Definition.

\begin{definition}[Vektorraumbündel]
Sei $M$ eine differenzierbare Mannigfaltigkeit und $V$ ein reeller
Vektorraum.
Ein \emph{differenzierbares Vektorraumbündel} über $M$ mit Faser $V$
ist eine differenzierbare Mannigfaltigkeit $E$ mit einer Abbildung
$\pi\colon E\to M$ mit folgenden Eigenschaften.
Jeder Punkt $p\in M$ hat eine offene Umgebung $U\subset M$ mit einem
Diffeomorphismus
\[
\varphi\colon \pi^{-1}(U) \to U \times V
\]
derart, dass $\operatorname{pr}_1\circ\varphi =\pi$, d.~h.
$\varphi(e)$ kann in der Form $\varphi(e) = (\pi(e), v)$ mit $v\in V$
geschrieben werden.
Eine solche Abbildung $\varphi$ heisst eine \emph{Bündelkarte}.
Für zwei Bündelkarten $\varphi_\alpha$ und $\varphi_\beta$ mit
Kartengebieten $U_\alpha$ und $U_\beta$ ist die Kartenwechselabbildung
\[
\varphi_\beta
\circ
\varphi_\alpha^{-1}
\colon
U_\alpha\cap U_\beta \times V
\to
U_\alpha\cap U_\beta \times V
\]
$E$ heisst \emph{Totalraum} des Bündels, $M$ heisst die \emph{Basis}
und $\pi$ ist die \emph{Bündelprojektion}.
Die Punkte von $E$, die auf den Punkt $p\in M$ abgebildet werden,
heisst die \emph{Faser} $E_p=\pi^{-1}(p)\subset E$ über dem Punkt $p$.
\end{definition}

\begin{beispiel}
Zu jeder Mannigfaltigkeit $M$ und jedem endlichdimensionalen
reellen Vektorraum $V$ gibt es das Bündel $M\times V$ mit
der Bündelprojektion
\[
\operatorname{pr}_1
\colon
M\times V \to M
:
(p,v) \mapsto p.
\]
Dieses Vektorraumbündel über $M$ heisst das \emph{triviale Bündel}
mit Faser $V$.
\end{beispiel}

\begin{beispiel}
\label{buch:zusammenhang:kovabl:beispiel:0d}
Die identische Abbildung $M\to M:p\mapsto p$ ist ein Vektorraumbündel
mit dem 0-dimensionalen Vektorraum $\{0\}$ als Faser.
\end{beispiel}

Wenn man Mannigfaltigkeiten miteinander vergleichen will, konstruiert
man Abbildungen zwischen diesen Mannigfaltigkeiten.
Da Vektorraumbündel aus zwei Mannigfaltigkeiten bestehen, die durch
die Bündelprojektion $\pi$ miteinander verbunden sind, müssen
Bündelabbildungen aus Abbildungen der Totalräume und der
Basismannigfaltigkeiten zusammengesetzt sein.
Ausserdem muss innerhalb der Fasern die Vektorraumstruktur zum
Ausdruck kommen.
Die folgende Definition erreicht dies.

\begin{definition}[Bündelabbildung]
Seien $\pi\colon E \to M$ und $\psi\colon F\to N$ zwei Vektorraumbündel
mit Fasern $V$ und $W$.
Eine \emph{Bündelabbildung} ist eine Abbildung $f\colon E\to F$ und
eine Abbildung $g\colon M\to N$ derart, dass
dass es eine Abbildung $g\colon M\to N$ gibt, derart dass 
$g\circ\pi=\psi\circ f$.
Ausserdem ist für jeden Punkt $p\in M$ die Einschränkung von $f$ auf
$\pi^{-1}(p)$ eine lineare Abbildung
\[
f_{|\pi^{-1}(p)}
\colon
\pi^{-1}(p)
\to
\pi^{-1}(g(p)).
\]
Sie heisst die \emph{Faserabbildung}.
\end{definition}

\begin{beispiel}
Beispiel~\ref{buch:zusammenhang:kovabl:beispiel:0d} hat gezeigt, dass
man die Mannigfaltigkeit $M$ als ein Vektorraumbündel mit Faser $\{0\}$
betrachten kann.
Die Definition einer Bündelabbildung $M\to E$ besagt dann, dass 
$f$ jedem Punkt $p$ genau ein Vektor in der Faser $E_p$ zuordnet.
\end{beispiel}

\begin{definition}
Sei $\pi\colon E\to M$ ein Vektorraumbündel.
Eine Bündelabbildung $f\colon M\to E$ hat die Eigenschaft
$\pi\circ f=\operatorname{id}_M$ und heisst ein \emph{Schnitt}
des Bündels.
Die Menge der Schnitte eines Vektorraumbündels wird mit $\Gamma(E)$
bezeichnet.
\end{definition}

In einer Bündelkarte $\varphi_\alpha:\pi^{-1}(U_\alpha) \to U_\alpha\times V$
ist ein Schnitt der Graph einer Funktion $U_\alpha\to V$.

\begin{beispiel}
Sei $\pi\colon E\to M$ ein Vektorraumbündel und 
\end{beispiel}

\subsection{Zusammenhang}
Der Begriff des Zusammenhangs soll die Eigenschaften einer Ableitung 
eines Vektors entlang einer Kurve in einer Mannigfaltigkeit beschreiben.
Wir brauchen daher ein Vektorraumbündel $\pi\colon E\to M$, welches
die Vektoren enthält, die abgeleitet werden sollen.
Ein differenzierbarer Schnitt von $E$ ist eine vektorwertige
Funktion, die Werte in der jeweils ``richtigen'' Faser annimmt.
Es müssen also Schnitte abgeleitet werden.
Die Richtung der Ableitung wird durch ein Feld von Tangentialvektoren
an die Mannigfaltigkeit gegeben.

\begin{definition}
Ein \emph{Zusammenhang} ist eine Abbildung
\index{Zusammenhang}%
\[
\nabla 
\colon
\Gamma(TM)\times \Gamma(E) \to \Gamma(E)
:
(X,Y) \mapsto \nabla_X Y
\]
mit folgenden Eigenschaften:
\begin{enumerate}
\item $\nabla_X Y$ ist $C^\infty(M)$-linear in $X$, d.~h. für 
$f_1,f_2\in C^{\infty}(M)$, $X_1,X_2\in \Gamma(TM)$ und $Y\in\Gamma(E)$ gilt
\[
\nabla_{f_1X_1+f_2X_2}Y
=
f_1\nabla_{X_1}Y + f_2\nabla_{X_2}F.
\]
\item $\nabla_X Y$ ist linear in $Y$, d.~h. für $a_1,a_2\in\mathbb{R}$
und $Y_1,Y_2\in \Gamma(E)$ gilt
\[
\nabla_X(a_1Y_1+a_2Y_2)
=
a_1\nabla_XY_1
+
a_2\nabla_XY_2
\]
für jedes $X\in\Gamma(TM)$.
\item
$\nabla$ erfüllt die Produktregel: für $f\in C^\infty(M)$ gilt
\index{Produktregel fur Zusammenhang@Produktregel für Zusammenhang}%
\[
\nabla_X (fY)
=
(X\cdot f)Y + f\nabla_X Y
\]
für $X\in\Gamma(TM)$ und $Y\in\Gamma(E)$.
\end{enumerate}
\end{definition}

Die Produktregel-Bedingung hat die interessante Konsequenz, dass der
Wert von $\nabla_XY$ in einem Punkt $p\in M$ nur vom Vektor $X(p)$ und
von den Werten von $Y$ in einer beliebig kleinen Umgebung des Punktes
$p$ in $M$ abhängt.
Wir beweisen dies nicht vollständig, sondern überlegen uns nur die
folgende Eigenschaft, mit deren Hilfe ein volständiger Beweis geführt
werden kann \cite[Proposition 4.5]{buch:leerm}.

\begin{lemma}
Sei $\nabla$ ein Zusammenhang im Vektorraumbündel $\pi\colon E\to M$,
$X\in \Gamma(TM)$, $Y\in \Gamma(E)$ und $p\in M$.
Dann hängt die kovariante Ableitung $\nabla_XY$ im Punkt $p$ nur von
den Werten von $X$ und $Y$ in einer beliebig kleinen Umgebung von $p$
ab.
Genauer: wenn $X=\tilde{X}$ und $Y=\tilde{Y}$ in einer Umgebung von
$p$ ist, dann ist $\nabla_{\tilde{X}}\tilde{Y} = \nabla_XY$ im Punkt $p$.
\end{lemma}

\begin{proof}
Wir müssen zeigen, dass $\nabla_X(Y-\tilde{Y})=0$ ist im Punkt $p$.
Die Differenz $Y-\tilde{Y}$ ist $=0$ in einer Umgebung von $p$.
Es genügt also zu zeigen, dass $\nabla_XY=0$ im Punkt $p$, wenn $Y=0$
ist in einer Umgebung $U$ des Punktes $p$.
Sei $\varphi\in C^\infty(M)$ derart, dass $\varphi(p)=1$ und 
$\varphi$ ist verschieden von $0$ nur innerhalb von $U$.
Dann ist das Produkt $\varphi Y=0$, innerhalb von $U$ weil $Y$ verschwindet
und ausserhalb von $U$ weil dort $\varphi$ verschwindet.
Wendet man $\nabla_X$ darauf an, folgt aus den Rechenregeln:
\[
0
=
\nabla X(\varphi Y)
=
(X\cdot \varphi)Y
+
\varphi\nabla_X Y
\qquad\Rightarrow\qquad
\varphi\nabla_X Y
=
-(X\cdot\varphi)Y.
\]
An der Stelle $p$ wird die linke Seite zu $\nabla_XY$ und die rechte
Seite verschwindet.
Es folgt, dass $\nabla_XY=0$ ist im Punkt $p$.

Ähnlich müssen wir jetzt auch noch zeigen, dass $\nabla_XY=0$ ist,
wenn $X=0$ ist in einer Umgebung von $p$.
Für eine Funktion $\varphi\in C^\infty(M)$ wie im ersten Teil des Beweises
folgt diesmal
\[
0
=
\nabla_{\varphi X}Y
=
\varphi\nabla_XY.
\]
Im Punkt $p$ ist $\varphi(p)=1$ und damit folgt für die rechte Seite
$\nabla_XY=0$.
\end{proof}

%
% Tensorbündel
%
\subsection{Tensorbündel}
Aus einem Vektorraumbündel $\pi\colon E\to M$ lässt sich ein neues
Vektorraumbündel konstruieren, dessen Faser beliebige Tensorprodukte
der Fasern des ursprünglichen Bündels sind.
Dazu muss nur gezeigt werden, wie lokal im Bündel Tensorprodukte
konstruiert werden sollen.
Wir bezeichnen daher den Faserraum wieder mit $V$ und betrachten
lokal das Bündel als triviales Bündel mit Hilfe einer Bündelkarte
\[
\varphi_\alpha
\colon
\pi^{-1}(U_\alpha)
\to
U_\alpha\times V 
\]
beschreiben können.

Zur Beschreibung des Tensorproduktes verwenden wir die Darstellung
mit Hilfe einer Basis.
Sei also $v_1,\dots,v_m\in V$ eine Basis des Faservektorraums.
Dann ist das $p$-fache Tensorprodukt von $V$ der Vektorraum aufgespannt
von der Basis aus Tensoren der Form
\[
v_{i_1}\otimes \cdots\otimes v_{i_p}
\]
mit $i_1,\dots,i_p\in \{1,\dots,m\}$.
Der Vektorraum des $p$-fachen Tensorproduktes ist also
\[
V^{p\otimes}
=
\underbrace{ V\otimes\cdots\otimes V }_{\displaystyle p}
=
\langle
v_{i_1}\otimes\cdots\otimes v_{i_p}
\mid
1\le
i_1,\dots,i_p 
\le m
\rangle.
\]
Durch Auswahl nur der antisymmetrischen Tensoren lässt sich auch
der Vektorraum der $k$-Vektoren aufgespannt von den Vektoren
\[
v_{i_1}\wedge\cdots\wedge v_{i_p}
\]
mit $1\le i_1 < \dots < i_p\le m$.
So entsteht der Vektorraum 
\[
V^{\wedge p}
=
\underbrace{V\wedge\dots\wedge V}_{\displaystyle p}
=
\langle
v_{i_1}\wedge\cdots\wedge v_{i_p}
\mid
1\le i_1 < \dots < i_p\le m
\rangle
\]
der $k$-Vektoren.

Diese Konstruktion kann jetzt auch für ein Vektorraumbündel
durchgeführt werden.
Dazu wählt man wie vorhin eine Basis in $V$. 
Die Abbildungen
\[
V_i
\colon
U_\alpha \to U_\alpha\times V
:
p\mapsto (p,v_i)
\]
sind $C^\infty(U_\alpha)$-linear unabhängige Schnitte des trivialen
Bündels $U_\alpha\times F$.
Zusammensetzung mit der Abbildung $\varphi_\alpha^{-1}$ macht aus den
Schnitten $V_i$ $C^\infty(M)$-linear unabhängige Schnitte
\[
E_i
\colon
U_\alpha \to \pi^{-1}(U_\alpha)
:
p \mapsto \varphi_\alpha^{-1}(p,v_i).
\]
Das $p$-fache Tensorprodukt $E^{p\otimes}=E\otimes\dots\otimes E$ von $E$ 
ist das Vektorraumbündel mit Faser $V^{p\otimes}$, welches lokal über der
offenen Menge $U_\alpha\subset M$ durch die Schnitte 
\[
E_{i_1}\otimes \dots \otimes E_{i_p}
\]
mit
$i_1,\dots,i_p\in\{1,\dots,m\}$
aufgespannt wird.

Nach dem gleichen Muster können auch Vektorraumbündel von
antisymmetrischen $k$-Vektoren konstruiert werden.
Ebenso ist es möglich, die verschiedenen Tensorpotenz oder
äusseren Potenzen mit Hilfe der direkten Summe zu einer Algebra
von Tensoren oder einer äusseren Algebra zusammenzubauen.

Zum Vektorraum $V$ ist $V^*$ der duale Vektorraum der Linearformen
auf $V$, also der linearen Abbildung von $V$ nach $\mathbb{R}$.
Zu einer Basis $\{v_i\mid i=1,\dots,m\}$ von $V$ lässt sich dann
die {\em duale Basis} aus den Linearformen $\beta_i$ konstruieren, die
\index{duale Basis}%
durch die Werte
\[
\langle \beta_i,v_k\rangle
=
\beta_i (v_k)
=
\delta_{ik}
\]
auf den Basisvektoren definiert ist.
Damit lässt sich auch das Vektorraumbünden $E^*$ der Linearformen
auf $E$ konstruieren.
Wendet man die Konstruktion des Tensorprodukts oder des Wedge-Produkts
auf das Bündel $E^*$ an, entsteht das Bündel der 
$p$-Formen.

Die Konstruktionen, die in früheren Kapiteln für Tangentialvektoren
und Differentialformen aufgebaut wurden, sind also auf 
Vektorraumbündel mit beliebigen Fasern verallgemeinerungsfähig.
Die früheren Konstruktionen sind Spezialfälle für das Tangentialbündel
$TM$.

%
% Kovariante Ableitung von Tensorfeldern
%
\subsection{Kovariante Ableitung von Tensorfeldern}
Ein Zusammenhang auf einem Vektorraumbündel $\pi\colon E\to M$ lässt
sich zu einem Zusammenhang auf dem Tensorbündel erweitern.
Da ein Zusammenhang linear ist, muss nur definiert werden, wie
der Zusammenhang auf Tensorprodukten von Vektoren wirkt.
Dies führt auf den folgenden Satz.

\begin{satz}
Sei $\pi\colon E\to M$ eine Vektorraumbündel und 
$\nabla \colon TM\times \Gamma(E)\to\Gamma(E)$
ein Zusammenhang.
Dann gibt es genau einen Zusammenhang
$\nabla \colon TM \times \Gamma(E^{\otimes p})\to\Gamma(E^{\otimes p})$
auf $E^{\otimes p}$, der
\[
\nabla_X(Y_1\otimes\dots\otimes Y_p)
=
(\nabla_XY_1) \otimes\dots\otimes Y_p
+
\dots
+
Y_1\otimes\dots\otimes (\nabla_XY_p)
\]
erfüllt.
Es gibt genau einen Zusammenhang auf $E^{\wedge p}$, der
\[
\nabla_X(Y_1\wedge\dots\wedge Y_p)
=
(\nabla_XY_1)\wedge\dots\wedge Y_p
+
\dots
+
Y_1\wedge\dots\wedge(\nabla_XY_p)
\]
erfüllt.
\end{satz}

Auch auf dem dualen Vektorraumbündel $E^*$ lässt sich ein Zusammenhang
definieren.
Dazu beachte man, dass sich aus einer Linearform $\beta\in\Gamma(E^*)$
und einem Vektorfeld $Y\in \Gamma(E)$ die Funktion
$\langle\beta,Y\rangle$ definieren lässt.
Für Funktionen stimmt die kovariante Ableitung mit der Wirkung des
Vektors $X$ überein.
Da die Abbildung $(\beta,Y)\to\langle\beta,Y\rangle$ bilinear ist,
muss die Ableitung die Form einer Produktregel annehmen, also
\[
\nabla_X\langle \beta,Y\rangle
=
X\cdot \langle \beta,Y\rangle
=
\langle \nabla_X\beta,Y\rangle
+
\langle \beta,\nabla_X Y\rangle.
\]
Aufgelöst nach $\nabla_X\beta$ wird dies zu
\begin{equation}
\langle \nabla_X\beta,Y\rangle
=
X\cdot \langle \beta,Y\rangle
-
\langle \beta,\nabla_X Y\rangle.
\label{buch:zusammenhang:kovarianteableitung:eqn:dual}
\end{equation}
Indem man für $Y$ die Schnitte einer Basis einsetzt, 
legt
\eqref{buch:zusammenhang:kovarianteableitung:eqn:dual}
die Werte der Linearform $\nabla_X\beta$ auf allen Basisvektoren
fest.

Wir verzichten für die Konstruktion der kovarianten Ableitung auf $E^*$
auf einen formellen Beweis und beschränken uns darauf, eine Schwierigkeit
der Definition \eqref{buch:zusammenhang:kovarianteableitung:eqn:dual}
zu adressieren.
Der Wert von $\nabla_X\beta$ im Punkt $p\in M$ ist festgelegt durch
die Werte auf den Vektoren $Y(p)$.
Multipliziert man $Y$ mit einer Funktion, die an der Stelle $p$ den Wert
$1$ hat, dann ändert sich der Wert auf der linken Seite nicht, aber
auf der rechten Seite treten zusätzlich Ableitungen von $f$ auf.
Sei also $f$ eine Funktion auf $M$ mit $f(p)=1$.
Setzt man $fY$ anstelle von $Y$ in die Definition
\eqref{buch:koordinaten:tangentialvektoren:def:einsteinschesummenkonvention}
ein, erhält man
\begin{align*}
\langle
\nabla_X \beta,fY
\rangle
&=
X\cdot\langle \beta,fY\rangle
-
\langle \beta,\nabla_X(fY)\rangle
\\
&=
X\cdot(f\langle \beta,Y\rangle)
-
\langle \beta,
(X\cdot f)Y
+
f\nabla_XY
\rangle
\\
&=
(X\cdot f)\langle \beta,Y\rangle
+
fX\cdot \langle \beta,Y\rangle
-
\langle \beta,
(X\cdot f)Y
\rangle
-
\langle \beta,
f\nabla_XY
\rangle
\\
&=
(X\cdot f)
\langle \beta,Y\rangle
+
f X\cdot \langle \beta,Y\rangle
-
(X\cdot f) \langle \beta, Y \rangle
-
f\langle \beta, \nabla_XY \rangle
\\
&=
f X\cdot \langle \beta,Y\rangle
-
f\langle \beta, \nabla_XY \rangle
\\
&=
\langle \nabla_X\beta,Y\rangle.
\end{align*}
Der Faktor $f$ hat also keinen Einfluss auf den Wert von
\eqref{buch:zusammenhang:kovarianteableitung:eqn:dual}.


