%
% teil3.tex -- Beispiel-File für Teil 3
%
% (c) 2020 Prof Dr Andreas Müller, Hochschule Rapperswil
%
% !TEX root = ../../buch.tex
% !TEX encoding = UTF-8
%
\section{Mathematische Herleitung der Grundgleichungen der Elastizitätstheorie}
\label{elastomechanik:section:herleitung}

\subsection{Kinematische Beziehungen}

Ausgangspunkt ist das Verschiebungsfeld $\vec{u} = [u_1, u_2, u_3]^T$. Die Verzerrung ergibt sich als symmetrischer Teil des Gradienten:

\begin{equation}
	\varepsilon_{ij} = \frac{1}{2} \left( \frac{\partial u_i}{\partial x_j} + \frac{\partial u_j}{\partial x_i} \right)
\end{equation}

Dies gilt unter der Annahme kleiner Verformungen (infinitesimaler Verzerrungstheorie), sodass Rotationen vernachlässigt werden können. Die Größe $\varepsilon_{ij}$ beschreibt sowohl Normal- ($i = j$) als auch Schubverzerrungen ($i \ne j$).

\subsection{Materialgesetz (Hooke’sches Gesetz)}

Für ein linear-elastisches, isotropes Material ergibt sich der Zusammenhang zwischen Spannung und Verzerrung durch das verallgemeinerte Hooke’sche Gesetz:

\begin{equation}
	\sigma_{ij} = \lambda \delta_{ij} \varepsilon_{kk} + 2\mu \varepsilon_{ij}
\end{equation}

Dabei sind:
\begin{itemize}
	\item $\sigma_{ij}$ die Spannungskomponenten,
	\item $\lambda, \mu$ die Lamé-Konstanten,
	\item $\delta_{ij}$ das Kronecker-Delta.
\end{itemize}

Die Lamé-Konstanten lassen sich mit den technisch verwendeten Materialkonstanten $E$ (Elastizitätsmodul) und $\nu$ (Poissonzahl) ausdrücken:

\begin{equation}
	\lambda = \frac{E \nu}{(1+\nu)(1-2\nu)}, \qquad \mu = \frac{E}{2(1+\nu)} = G
\end{equation}

\subsection{Gleichgewichtsbedingungen}

Für das mechanische Gleichgewicht im Körper gilt, dass die Summe aller Kräfte (Spannungsresultierende und Volumenkraft $f_i$) null sein muss:

\begin{equation}
	\frac{\partial \sigma_{ij}}{\partial x_j} + f_i = 0
\end{equation}

Dies sind drei partielle Differentialgleichungen (eine pro Raumrichtung), die für jedes infinitesimale Volumenelement im Körper erfüllt sein müssen.

\subsection{Zusammenfassung: Gleichungssystem der linearen Elastizität}

Das vollständige System der linearen Elastizitätstheorie besteht aus:
\begin{enumerate}
	\item \textbf{Kinematik:}
	\[
	\varepsilon_{ij} = \frac{1}{2} \left( \frac{\partial u_i}{\partial x_j} + \frac{\partial u_j}{\partial x_i} \right)
	\]
	
	\item \textbf{Materialgesetz (Isotrop, linear):}
	\[
	\sigma_{ij} = \lambda \delta_{ij} \varepsilon_{kk} + 2\mu \varepsilon_{ij}
	\]
	
	\item \textbf{Gleichgewicht:}
	\[
	\frac{\partial \sigma_{ij}}{\partial x_j} + f_i = 0
	\]
\end{enumerate}

Zusammen mit geeigneten Randbedingungen (z. B. vorgegebene Verschiebungen oder Spannungen) ergibt sich eine eindeutig lösbare Randwertaufgabe.