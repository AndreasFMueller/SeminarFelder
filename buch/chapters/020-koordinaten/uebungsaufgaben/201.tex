Zwei Koordinatensysteme $(x^1,x^2)$ und $(y^1,y^2)$ in der Ebene sind
durch die Gleichungen
\begin{equation}
\renewcommand\arraycolsep{2pt}
\begin{array}{rcrcr}
y^1 &=&  x^1 &+& 2x^2 \\
y^2 &=& 2x^1 &-& 1x^2
\end{array}
\label{buch:201:transformation}
\end{equation}
\begin{teilaufgaben}
\item
Berechnen Sie die Jacobi-Matrix dieser Koordinatentransformation.
\item
Zeichen Sie die $(y^1,y^2)$-Koordinatenlinien im
$(x^1,x^2)$-Koordinatensystem ein.
\item
Betrachten Sie die Kurve $\gamma(t) = (x^1(t),x^2(t))$ im
$(x^1,x^2)$-Koordinatensystem mit
\begin{align*}
x^1(t) &= \cos t \\
x^2(t) &= \sin t
\end{align*}
Berechnen Sie die Komponenten des Tangentialvektors im
$(x^1,x^2)$-Koordinaten\-sys\-tem.
\item
Verwenden Sie die Transformation \eqref{buch:201:transformation}
um eine Darstellung der Kurve im $(y^1,y^2)$-Koordinaten\-system
zu erhalten.
\item
Berechnen Sie die Komponenten des Tangentialvektors in einem
Punkt der Kurve im $(y^1,y^2)$-Koordinatensystem und rechnen Sie
nach, dass diese Komponenten auch durch Anwendung der Jacobi-Matrix
auf die in Teilaufgabe c) gefundenen Komponenten berechnet werden
können.
\end{teilaufgaben}

\begin{loesung}
\begin{teilaufgaben}
\item
Die Jacobi-Matrix ist
\[
J
=
\bgroup
\renewcommand\arraystretch{1.4}
\begin{pmatrix}
\frac{\partial y^1}{\partial x^1}& \frac{\partial y^1}{\partial x^2} \\
\frac{\partial y^2}{\partial x^1}& \frac{\partial y^2}{\partial x^2}
\end{pmatrix}
\egroup
=
\begin{pmatrix}
1 &  2 \\
2 & -1 
\end{pmatrix}.
\]
\item
Die Koordinatenlinien mit konstanter $y^1$- bzw.~$y^2$-Koordinate
erfüllen die Gleichungen
\begin{align*}
 x^1 +          2 x^2 &= \operatorname{const} \\
\text{bzw.}\qquad
2x^1 - \phantom{1}x^2 &= \operatorname{const},
\end{align*}
dies sind Koordinatengleichungen von Geraden.
Die Normalen dieser Geraden sind
\[
\vec{n}_1 = \begin{pmatrix} 1 \\  2 \end{pmatrix}
\qquad\text{und}\qquad
\vec{n}_2 = \begin{pmatrix} 2 \\ -1 \end{pmatrix}.
\]
\begin{figure}
\centering
\begin{tikzpicture}[thick,>=latex,scale=1]
\begin{scope}
\clip (-5,-4) rectangle (5,4);
\pgfmathparse{1/sqrt(5)}
\xdef\l{\pgfmathresult}
\foreach \n in {-21,...,21}{
	\draw[color=darkred]
		($\n*\l*\l*(1,2)-5*(2,-1)$) -- ($\n*\l*\l*(1,2)+5*(2,-1)$);
}
\foreach \n in {-21,...,21}{
	\draw[color=blue]
		($\n*\l*\l*(1,-2)-5*(2,1)$) -- ($\n*\l*\l*(1,-2)+5*(2,1)$);
}
\end{scope}
\draw[->] (-5.1,0) -- (5.5,0) coordinate[label={$x^1$}];
\draw[->] (0,-4.1) -- (0,4.3) coordinate[label={right:$x^2$}];
\foreach \x in {1,...,5}{
	\draw (\x,-0.05) -- (\x,0.05);
	\node at (\x,-0.05) [below] {$\x$};
	\draw (-\x,-0.05) -- (-\x,0.05);
	\node at (-\x,-0.05) [below] {$-\x$};
}
\foreach \y in {1,...,4}{
	\draw (-0.05,\y) -- (0.05,\y);
	\node at (-0.05,\y) [left] {$\y$};
	\draw (-0.05,-\y) -- (0.05,-\y);
	\node at (-0.05,-\y) [left] {$-\y$};
}
\end{tikzpicture}
\caption{Koordinatenlinien $y^1\in\mathbb{Z}$ in rot und 
$y^2\in\mathbb{Z}$ in blau
\label{buch:201:fig}}
\end{figure}
Da Division durch die Länge der Normalenvektoren die Hessesche
Normalform ergibt, kann man schliessen, dass die Geraden mit
konstantem $y^1$ in einem Abstand am Nullpunkt vorbeigehen, 
der ein Vielfaches von $1/|\vec{n}_1|=1/\sqrt{5}$ ist.
Analog haben die Geraden mit konstantem $y_2$ einen Abstand,
der ein Vielfaches von $1/|\vec{n}_2|=1/\sqrt{5}$.
Die Geraden sind in Abbildung~\ref{buch:201:fig} dargestellt.
\item
Die Ableitung der Kurvenkoordinaten ist
\[
\dot{x}^1(t) = -\sin t
\qquad\text{und}\qquad
\dot{x}^2(t) = \cos t.
\]
\item
In den $(y^1,y^2)$-Koordinaten hat die Kurve die Koordinaten
\[
(y^1(t), y^2(t))
=
(4\cos t+3\sin t, 5\cos t + 4\sin t).
\]
\item
Der Tangentialvektor im $(x^1,x^2)$-Koordinatensystem hat die
Komponenten
\[
(\dot{y}^1(t),\dot{y}^2(t))
=
(-4\sin t+3\cos t, -5\sin t + 4\cos t).
\]
Anwendung der $J$-Matrix auf den Tangentialvektor von Aufgabe c) ergibt
\[
\begin{pmatrix}
4 & 3 \\
5 & 4
\end{pmatrix}
\begin{pmatrix}
-\sin t\\
 \cos t
\end{pmatrix}
=
\begin{pmatrix}
-4\sin t + 3\cos t\\
-5\sin t + 4\cos t
\end{pmatrix},
\]
was mit dem Resultat von Teilaufgabe b) übereinstimmt.
\qedhere
\end{teilaufgaben}
\end{loesung}

