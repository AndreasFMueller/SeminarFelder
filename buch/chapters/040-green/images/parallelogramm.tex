%
% parallelogramm.tex -- 2-Vektor-Eigenschaften des Flächeninhalts
%
% (c) 2025 Prof Dr Andreas Müller, OST Ostschweizer Fachhochschule
%
\documentclass[tikz]{standalone}
\usepackage{amsmath}
\usepackage{times}
\usepackage{txfonts}
\usepackage{pgfplots}
\usepackage{csvsimple}
\usetikzlibrary{arrows,intersections,math,calc}
\definecolor{darkred}{rgb}{0.8,0,0}
\begin{document}
\def\skala{1}
\begin{tikzpicture}[>=latex,thick,scale=\skala]

\coordinate (A) at (2,1.5);
\coordinate (B) at (-1,2);

\begin{scope}[xshift=-4.3cm]
\fill[color=gray!40,opacity=0.5]
	(0,0) -- ++(2,1.5) -- ++(-1,2) -- ++(-2,-1.5) -- cycle;
\node at ($0.5*(2,1.5)+0.5*(-1,2)$) {$F$};
\fill[color=darkred!20,opacity=0.5]
	(0,0) -- (2,1.5) -- ($(2,1.5)+(-1,2)+0.4*(2,1.5)$) -- ++(-2,-1.5)
	-- cycle;
\draw[->] (0,0) -- (2,1.5);
\draw[->] (0,0) -- (-1,2);
\draw[->] (0,0) -- ($(-1,2)+0.4*(2,1.5)$);
\node at (2,1.5) [right] {$\vec{a}$};
\node at (-1,2) [left] {$\vec{b}$};
\fill (0,0) circle[radius=0.05];
\node at (0,0) [below] {$O$};
\node at ($(-1,2)+0.4*(2,1.5)+(0.2,0)$) [above left] {$\vec{b}+s\vec{a}$};
\end{scope}

\begin{scope}
\fill[color=darkred!20] (0,0) -- ++(A) -- ++(B) -- ++($-1*(A)$) -- cycle;
\draw[->] (0,0) -- ++(A);
\draw[->] (0,0) -- ++(B);
\node at (2,1.5) [right] {$\vec{a}$};
\node at (-1,2) [left] {$\vec{b}$};
\node[color=darkred] at ($0.5*(2,1.5)+0.5*(-1,2)$) {$F$};
\fill (0,0) circle[radius=0.05];
\node at (0,0) [below] {$O$};
\end{scope}

\begin{scope}[xshift=4.3cm]
\fill[color=gray!40,opacity=0.5]
	(0,0) -- ++(2,1.5) -- ++(-1,2) -- ++(-2,-1.5) -- cycle;
\node at ($0.5*(2,1.5)+0.5*(-1,2)$) {$F$};
\draw[->] (0,0) -- (2,1.5);
\node at (2,1.5) [right] {$\vec{a}$};
\node at (-1,2) [left] {$\vec{b}$};
\fill[color=darkred!20,opacity=0.5]
	(0,0) -- ($(2,1.5)-0.5*(-1,2)$) -- ++(-1,2) -- (-1,2) -- cycle;
\draw[->] (0,0) -- (-1,2);
\draw[->] (0,0) -- ($(2,1.5)-0.5*(-1,2)$);
\pgfmathparse{atan(0.5/2.5)}
\xdef\w{\pgfmathresult}
\node at ($0.5*(2.5,0.5)+(0,0.1)$) [below,rotate={\w}] {$\vec{a}+s\vec{b}$};
\fill (0,0) circle[radius=0.05];
\node at (0,0) [below] {$O$};
\end{scope}

\end{tikzpicture}
\end{document}

