%
% stationaer.tex -- Diskretistion 5x5 für stationäre Lösung der Wärmeleitung
%
% (c) 2021 Prof Dr Andreas Müller, OST Ostschweizer Fachhochschule
%
\documentclass[tikz]{standalone}
\usepackage{amsmath}
\usepackage{times}
\usepackage{txfonts}
\usepackage{pgfplots}
\usepackage{csvsimple}
\usetikzlibrary{arrows,intersections,math}
\definecolor{darkred}{rgb}{0.8,0,0}
\begin{document}
\def\skala{1}
\def\h{1}
\def\punkt#1#2{ ({(#2)*\h},{(4-(#1))*\h}) }
\def\knoten#1#2#3{
	\node at \punkt{#1}{#2} {$#3$};
}
\begin{tikzpicture}[>=latex,thick,scale=\skala]

\fill[color=blue!20] \punkt{0.5}{1.5} rectangle \punkt{1.5}{4.5};
\fill[color=blue!20] \punkt{4.5}{1.5} rectangle \punkt{5.5}{4.5};
\fill[color=blue!20] \punkt{1.5}{0.5} rectangle \punkt{4.5}{1.5};
\fill[color=blue!20] \punkt{1.5}{4.5} rectangle \punkt{4.5}{5.5};
\fill[color=darkred!20] \punkt{1.5}{1.5} rectangle \punkt{4.5}{4.5};
\foreach \x in {0.5,...,5.5}{
	\draw[color=white,line width=1.4pt] \punkt{0.5}{\x} -- \punkt{5.5}{\x};
	\draw[color=white,line width=1.4pt] \punkt{\x}{0.5} -- \punkt{\x}{5.5};
}

\knoten{1}{2}{80^\circ}
\knoten{1}{3}{80^\circ}
\knoten{1}{4}{80^\circ}

\knoten{2}{1}{100^\circ}
\knoten{2}{2}{T_{2,2}}
\knoten{2}{3}{T_{2,3}}
\knoten{2}{4}{T_{2,4}}
\knoten{2}{5}{0^\circ}

\knoten{3}{1}{100^\circ}
\knoten{3}{2}{T_{3,2}}
\knoten{3}{3}{T_{3,3}}
\knoten{3}{4}{T_{3,4}}
\knoten{3}{5}{0^\circ}

\knoten{4}{1}{100^\circ}
\knoten{4}{2}{T_{4,2}}
\knoten{4}{3}{T_{4,3}}
\knoten{4}{4}{T_{4,4}}
\knoten{4}{5}{0^\circ}

\knoten{5}{2}{20^\circ}
\knoten{5}{3}{20^\circ}
\knoten{5}{4}{20^\circ}

\end{tikzpicture}
\end{document}

