%
% teil2.tex -- Beispiel-File für teil2 
%
% (c) 2020 Prof Dr Andreas Müller, Hochschule Rapperswil
%
% !TEX root = ../../buch.tex
% !TEX encoding = UTF-8
%
\section{Oszilierende Muster
\label{reaktdiff:section:teil2}}
\kopfrechts{Teil 2}
Zu schluss dieses Papers werden Systeme beleuchtet welche Oszilieren. Als erstes wird die Lotka-Volterra Reaktion behandelt.
Anschliessend wird die Belousov-Zhabotinsky Reaktion kurz betrachtet das sie sehr Faszinierend ist.

\subsection{Lotka-Volterra Reaktion 
\label{reaktdiff:subsection:bonorum}}

Die Lotka-Volterra Reaktion besteht ebenfalls wie die Turing-Muster aus einem Reaktionsdiffusionsmodell mit 2 Gleichungen.
Der grosse Unterschied zu den Turing-Muster ist, dass Lotka-Volterra Reaktion bereits ohne Diffusion instabil ist.
Die Lotka-Volterra Reaktion (oder Gleichung) wird ebenfalls benutzt um das klassische Räuber-Beute-Modell zu beschreiben.

\subsubsection{Mathematik hinter der Lotka-Volterra Reaktion}

Wieder werden \(u(x,t)\) und \(v(x,t)\) verwendet um die Konzentration an einem bestimmten Ort zu einem bestimmten Zeitpunkt zu beschreiben.
Die Reaktionsterme
\begin{equation}
     f(u,v) = \alpha u -  \beta v, \quad g(u,v)= \delta uv - \gamma
\end{equation}
bestehen neben den Konsentrastionen \(u,v\) aus den Konstanten \(\alpha,\beta,\delta \)und \(\gamma\).

Das dazugehörige Reaktionsdiffusionssystem
\begin{align}
    \frac{\partial u}{\partial t} &= D_u \Delta u + \alpha u - \beta u v, \\+
    \frac{\partial v}{\partial t} &= D_v \Delta v + \delta u v - \gamma v.
    \label{reaktdiff:equation:lvsys}
\end{align}
besteht wieder aus \(D_u,D_v > 0\).
Das System hat Nustellen bei  \(u = 0,v = 0\) (trivial) und bei \(u = \frac{\gamma}{\delta}, v = \frac{\alpha}{\beta}\).
Um die Stabilität an diesen Punkte zu untersuchen wird wie auch zuvor im Kapitel das System Linearisiert und an der intressanten Nullstelle untersucht.
Die Jacobi-Matrix des Systems lautet
\begin{equation}
        J(u,v) =
        \begin{pmatrix}
        \frac{\partial f}{\partial u} & \frac{\partial f}{\partial v} \\
        \frac{\partial g}{\partial u} & \frac{\partial g}{\partial v}
        \end{pmatrix}
        =
        \begin{pmatrix}
        \alpha - \beta v & -\beta u \\
        \delta v & \delta u - \gamma
        \end{pmatrix}.
\end{equation}
An der Stelle \(u = \frac{\gamma}{\delta}, v = \frac{\alpha}{\beta}\) ergibt das
\begin{equation}
         J(\frac{\gamma}{\delta},\frac{\alpha}{\beta}) =
        \begin{pmatrix}
        0 & -\beta \cdot\frac{\gamma}{\delta} \\
        \delta \cdot \frac{\alpha}{\beta} & 0
        \end{pmatrix}. 
\end{equation}
Somit erhält man
\begin{equation}
    \lambda^2 + \sqrt{\frac{\delta \alpha \cdot \beta \gamma}{\delta \beta}}
     = 
     \lambda^2 + \alpha \gamma = 0 
     \rightarrow
     \lambda = \pm j \sqrt{a\gamma}
\end{equation}
als Eigenewerte.
Die Eigenewerte sind rein Imaginär
Das Bedeutet das eine kleine Störung im System keine Dämpfung erfährt.
Es entsteht eine oszillierendes Muster.

\subsubsection{Simulation der LV-Reaktion}

Für die Simulation werden die Werte \(\alpha = 1, \beta = 0.1, \gamma = 1.5, \delta = 0.75\) verwendet.
Somit befindet sich das Gleichgewicht bei
\begin{equation}
    u = \frac{\gamma}{\delta} = \frac{1.5}{0.75} = 2, 
    v = \frac{\alpha}{\beta} = \frac{1}{0.1} = 10.
\end{equation}
Die Eigenwerte haben den Betrag \(\sqrt{\alpha\gamma} = \sqrt{1 \cdot 1.5} \approx 1.225\).


\subsection{Belousov-Zhabotinsky Reaktion
\label{reaktdiff:subsection:bonorum}}
Wenn man über die BZ-Reaktion recherchiert kommt man schnell mit dem Begriff Oregonator in Kontakt.
Der Oregonator ist ein Modell zur Beschreibung von oszillierenden chemischen Reaktionen.
In userem


