%
% rnd.tex -- Rand eines zweidimensionalen simplizialen Komplexes
%
% (c) 2021 Prof Dr Andreas Müller, OST Ostschweizer Fachhochschule
%
\documentclass[tikz]{standalone}
\usepackage{amsmath}
\usepackage{times}
\usepackage{txfonts}
\usepackage{pgfplots}
\usepackage{csvsimple}
\usetikzlibrary{arrows,intersections,math,calc}
\begin{document}
\definecolor{darkred}{rgb}{0.8,0,0}
\def\skala{1}
\def\h{1}
\pgfmathparse{sqrt(3)/2}
\xdef\v{\pgfmathresult}
\def\punkt#1#2#3{
	\fill[color=white] #1 circle[radius=0.17];
	\draw[color=#2] #1 circle[radius=0.17];
	\node[color=#2] at ($#1+(0,-0.015)$) {$\scriptstyle #3\mathstrut$};
}
\begin{tikzpicture}[>=latex,thick,scale=\skala,
declare function = {
	f(\T) = 0.2 * (rand() - 0.5);
}]

%
% randpoints.tex -- generated by randpoints.m - do not modify!
%
% (c) 2025 Prof Dr Andreas Müller
%
\coordinate (A0) at (-4.2185,-0.3062);
\coordinate (A1) at (-2.2718,-0.1179);
\coordinate (A2) at (-0.2798,-0.0180);
\coordinate (A3) at (1.9843,-0.0920);
\coordinate (A4) at (3.9656,-0.0290);
\coordinate (B0) at (-3.1368,1.3648);
\coordinate (B1) at (-1.2934,1.5920);
\coordinate (B2) at (0.8378,1.6547);
\coordinate (B3) at (2.8043,1.3843);
\coordinate (B4) at (-2.2281,3.0812);
\coordinate (B5) at (-0.0900,3.1121);
\coordinate (B6) at (1.6708,3.1865);
\coordinate (C0) at (-3.3009,-2.0097);
\coordinate (C1) at (-1.3160,-2.1235);
\coordinate (C2) at (0.8439,-2.0684);
\coordinate (C3) at (2.8779,-1.9896);
\coordinate (C4) at (-2.1859,-3.5015);
\coordinate (C5) at (-0.3166,-3.7509);
\coordinate (C6) at (1.6902,-3.7706);


\fill[color=gray!20] (A0) -- (C0) -- (C4) -- (C5) -- (C6) -- (C3) -- (A4) -- (B3) -- (B6) -- (B5) -- (B4) -- (B0) -- cycle;
\fill[color=white] (B1) -- (A2) -- (C2) -- (A3) -- (B2) -- cycle;

\draw (A0) -- (A1);
\draw[color=darkred,line width=1.4pt] (A0) -- (B0);
\draw[color=darkred,line width=1.4pt] (A0) -- (C0);

\draw (A1) -- (A2);
\draw (A1) -- (B1);
\draw (A1) -- (C1);

%\draw (A2) -- (A3);
%\draw (A2) -- (B2);
\draw[color=darkred,line width=1.4pt] (A2) -- (C2);

\draw (A3) -- (A4);
\draw (A3) -- (B3);
\draw (A3) -- (C3);

\draw (B0) -- (B1);
\draw (B0) -- (A1);
\draw[color=darkred,line width=1.4pt] (B0) -- (B4);

\draw[color=darkred,line width=1.4pt] (B1) -- (B2);
\draw[color=darkred,line width=1.4pt] (B1) -- (A2);
\draw (B1) -- (B5);

\draw (B2) -- (B3);
\draw[color=darkred,line width=1.4pt] (B2) -- (A3);
\draw (B2) -- (B6);

\draw[color=darkred,line width=1.4pt] (B3) -- (A4);

\draw[color=darkred,line width=1.4pt] (B4) -- (B5);
\draw (B4) -- (B1);

\draw[color=darkred,line width=1.4pt] (B5) -- (B6);
\draw (B5) -- (B2);

\draw[color=darkred,line width=1.4pt] (B6) -- (B3);

\draw (C0) -- (C1);
\draw (C0) -- (A1);
\draw[color=darkred,line width=1.4pt] (C0) -- (C4);

\draw (C1) -- (C2);
\draw (C1) -- (A2);
\draw (C1) -- (C5);

\draw (C2) -- (C3);
\draw[color=darkred,line width=1.4pt] (C2) -- (A3);
\draw (C2) -- (C6);

\draw[color=darkred,line width=1.4pt] (C3) -- (A4);

\draw[color=darkred,line width=1.4pt] (C4) -- (C5);
\draw (C4) -- (C1);

\draw[color=darkred,line width=1.4pt] (C5) -- (C6);
\draw (C5) -- (C2);

\draw[color=darkred,line width=1.4pt] (C6) -- (C3);

\punkt{(A0)}{darkred}{7}
\punkt{(A1)}{black}{8}
\punkt{(A2)}{darkred}{9}
\punkt{(A3)}{darkred}{10}
\punkt{(A4)}{darkred}{11}

\punkt{(B0)}{darkred}{3}
\punkt{(B1)}{darkred}{4}
\punkt{(B2)}{darkred}{5}
\punkt{(B3)}{darkred}{6}
\punkt{(B4)}{darkred}{0}
\punkt{(B5)}{darkred}{1}
\punkt{(B6)}{darkred}{2}

\punkt{(C0)}{darkred}{12}
\punkt{(C1)}{black}{13}
\punkt{(C2)}{darkred}{14}
\punkt{(C3)}{darkred}{15}
\punkt{(C4)}{darkred}{16}
\punkt{(C5)}{darkred}{17}
\punkt{(C6)}{darkred}{18}

%\fill[color=darkred] (A0) circle[radius=0.07];
%\fill (A1) circle[radius=0.07];
%\fill (A2) circle[radius=0.07];
%\fill (A3) circle[radius=0.07];
%\fill[color=darkred] (A4) circle[radius=0.07];
%
%\fill[color=darkred] (B0) circle[radius=0.07];
%\fill (B1) circle[radius=0.07];
%\fill (B2) circle[radius=0.07];
%\fill[color=darkred] (B3) circle[radius=0.07];
%\fill[color=darkred] (B4) circle[radius=0.07];
%\fill[color=darkred] (B5) circle[radius=0.07];
%\fill[color=darkred] (B6) circle[radius=0.07];
%
%\fill[color=darkred] (C0) circle[radius=0.07];
%\fill (C1) circle[radius=0.07];
%\fill (C2) circle[radius=0.07];
%\fill[color=darkred] (C3) circle[radius=0.07];
%\fill[color=darkred] (C4) circle[radius=0.07];
%\fill[color=darkred] (C5) circle[radius=0.07];
%\fill[color=darkred] (C6) circle[radius=0.07];

\end{tikzpicture}
\end{document}

