%
% main.tex -- Paper zum Thema <particles>
%
% (c) 2020 Autor, OST Ostschweizer Fachhochschule
%
% !TEX root = ../../buch.tex
% !TEX encoding = UTF-8
%
\chapter{Turning Waves into Particles\label{chapter:particles}}
\kopflinks{Turning Waves into Particles}
\begin{refsection}
\chapterauthor{Flurin Brechbühler, Laurin Heitzer}

Felder werden oft als linear angenommen. 
So nimmt man in der Physik oft an, dass eine Verdoppelung der Spannung in einem Spannungsfeld mit einer Verdoppelung der Deformation einhergeht.
Die Beziehung zwischen der Spannung $\sigma$ und der Deformation $\varepsilon$ kann dann einfach durch das Hookesche Gesetz~\cite{todo} als
\[
    \sigma = C \varepsilon \label{particles:gleichung:hooke}
\]
beschrieben werden. % TODO: Cite
Dabei ist der Elastizitätstensor $C$ unabhängig von der Deformation.
Ohne Annahme dieser Unabhängigkeit wäre das Lösen solcher Probleme deutlich schwieriger.

Im Elektromagnetismus werden ähnliche Vereinfachungen gemacht. 
Die elektrische Flussdichte $D$ wird dabei ebenfalls durch den Tensor $\varepsilon$ als
\[
    D = \varepsilon E \label{particles:eq:D}
\]
in ein lineares Verhältnis mit der elektrischen Feldstärke $E$ gebracht.
Dasselbe gilt bei dem magnetischen Fluss $H$ und der magnetischen Feldstärke $B$, die durch
\[
    H = \mu B \label{particles:eq:H}
\]
mit dem Tensor $\mu$ in ein lineares Verhältnis gebracht werden.

Dass der in~\ref{particles:gleichung:hooke} beschriebene, lineare Zusammenhang für hohe Deformationen nicht mehr stimmt, ist intuitiv.
Ein Bauteil, das immer weiter deformiert wird, wird irgendwann seine Widerstandskraft verlieren und brechen oder permanent verformt werden.
Abbildung~\ref{particles:abb:deformationskurve} zeigt als Beispiel die Spannungs-Dehnungs-Kennlinie einer gängigen Aluminiumlegierung. % TODO: Abb.
Es ist dabei klar zu erkennen, dass die Abhängigkeit der Spannung von der Deformation besonders bei hohen Deformationen keineswegs linear ist.
Ist die Beanspruchung des Materials jedoch klein, so kann zur Berechnung das Hookesche Gesetz~\ref{particles:gleichung:hooke} verwendet werden.
Dies erleichtert die Simulation des Spannungsfelds im Bauteil erheblich.

Bei den Elektromagnetischen Feldern soll eine solche Nichtlinearität auch auftreten.
Die Schwinger-Grenze, welche von \textbf{wer auch immer} in~\cite{todo} postuliert wird, markiert dabei die Feldstärken, ab denen die in~\ref{particles:eq:D} und~\ref{particles:eq:H} gezeigten Beziehungen nichtlinear werden. % TODO: Cite, evtl. umschreiben

Dieses Paper soll ein Phänomen, welches bei der Simulation eines solchen nichtlinearen Feldes entsteht, aufzeigen.

% TODO: Unser Paper verwendet numerische Simulation um "einen blick über den Horlizont der linearen Felder zu werfen"

\input{papers/particles/1_simulation.tex} % TODO: Wie funktioniert die Simulation?
\input{papers/particles/2_lineares_feld.tex} % TODO: Vielleicht: Funktioniert die Superposition?
\input{papers/particles/3_nichtlineares_feld.tex} % TODO: Was passiert bei nichtlinearitäten
\input{papers/particles/4_fazit.tex} % TODO: Fazit: Was könnten wir hier beobachtet haben?

\printbibliography[heading=subbibliography]
\end{refsection}
