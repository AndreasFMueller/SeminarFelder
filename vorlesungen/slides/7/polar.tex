%
% polar.tex -- polarkoordinaten
%
% (c) 2021 Prof Dr Andreas Müller, OST Ostschweizer Fachhochschule
%
\bgroup
\begin{frame}[t]
\setlength{\abovedisplayskip}{5pt}
\setlength{\belowdisplayskip}{5pt}
\frametitle{Polarkoordinaten $(r,\varphi)$}
\vspace{-20pt}
\begin{columns}[t,onlytextwidth]
\begin{column}{0.58\textwidth}
\begin{enumerate}
\item<2-> Metrik: 
$\displaystyle
g_{ik}
=
\begin{pmatrix}
1&0\\0&r^2
\end{pmatrix}
$
\item<3-> Inverse: $\displaystyle
g^{ik}
=
\begin{pmatrix}
1&0\\0&r^{-2}
\end{pmatrix}
$
\item<4-> Ableitungen: 
$\displaystyle
\frac{\partial g_{22}}{\partial x^1} = 2r
$
\item<5-> Christoffel-Symbole erster Art:
\[
\begin{aligned}
\Gamma_{1,11} &= 0,
&
\Gamma_{1,12} &= \Gamma_{1,21} = 0,
&
\Gamma_{1,22} &= -r
\\
\Gamma_{2,11} &= 0,
&
\Gamma_{2,12} &= \Gamma_{2,21} = r,
&
\Gamma_{2,22} &= 0.
\end{aligned}
\]
\item<6-> Christoffel-Symbole zweiter Art:
\[
\begin{aligned}
\Gamma^1_{11} &= 0,
&
\Gamma^1_{21} &= \Gamma^1_{12} = 0,
&
\Gamma^1_{22} &= -r,
\\
\Gamma^2_{11} &= 0,
&
\Gamma^2_{21} &= \Gamma^2_{21} = \frac1r,
&
\Gamma^2_{22} &= 0.
\end{aligned}
\]
\end{enumerate}
\end{column}
\begin{column}{0.38\textwidth}
\uncover<7->{%
\begin{block}{Geodäten-Gleichung}
\begin{align*}
\uncover<8->{
\ddot{r}
&=
r \dot{\varphi}^2
}
\\
\uncover<9->{
\ddot{\varphi}
&=
-\frac{2}{r}\dot{\varphi}\dot{r}
}
\end{align*}
\uncover<10->{Lösungen sind Geraden in der Ebene}
\end{block}}
\end{column}
\end{columns}
\end{frame}
\egroup
