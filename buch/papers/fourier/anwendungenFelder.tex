%
% anwendungenFelder.tex -- 3.	Anwendungen auf Felder, PDEs in ODE umwandeln (Grundlage, um Feldgleichungen als eine Familie von Oszillator-Gleichungen zu interpretieren)
%
% (c) 2020 Prof Dr Andreas Müller, Hochschule Rapperswil
%
% !TEX root = ../../buch.tex
% !TEX encoding = UTF-8
%

%    Anwendungen auf Felder, PDEs in ODE umwandeln (Grundlage, um Feldgleichungen als eine Familie von Oszillator-Gleichungen zu interpretieren) 


\section{Anwendung auf ein Feld\label{fourier:section:AnwendungAufFeld}}
\kopfrechts{Anwendung auf ein Feld}

Eine gewöhnliche Differentialgleichung enthält Ableitungen nur nach einer unabhängigen Variable, während eine partielle Differentialgleichung Ableitungen nach mehreren unabhängigen Variablen enthält.  
In diesem Abschnitt überführen wir eine partielle Differentialgleichung mittels Fourier-Reihe in eine gewöhnliche Differentialgleichung für die Zeitvariable.
Als Beispiel betrachten wir die eindimensionale Wellengleichung  
\index{Wellengleichung}
\begin{equation}\label{eq:wellengleichung}
	\frac{\partial^2 u(x,t)}{\partial t^2} = c^2 \frac{\partial^2 u(x,t)}{\partial x^2}
\end{equation}  
mit Ausbreitungsgeschwindigkeit $c$. Dieses Modell beschreibt unter anderem elektromagnetische Felder.  
\index{elektromagnetisches Feld}%
Unter der Annahme, dass $u(x, t)$ eine periodisch Funktion von $x$ ist, führen wir eine Fourier-Reihenentwicklung durch. 
Neu besteht 
\begin{equation}
	u(x,t) = \frac{a_0(t)}{2} + \sum_{n=1}^{\infty} \left( a_n(t) \cos(n \omega x) + b_n(t) \sin(n \omega x) \right)
\end{equation}
aus einer unendlichen Anzahl von Schwingungen.
Alle drei Arten von Fourier-Koeffizienten erhält man durch die Integralrechnungen im Abschnitt~\ref{fourier:section:GrundlagenFourierAnalyse}. 
Da das Ziel ist, $x$ zu eliminieren, integriert man über eine Periode nach $x$.
Somit entstehen Fourier-Koeffizienten, die von $t$ abhängig sind. 

Nun wird $u(x,t)$ in seiner Summenform in die Wellengleichung eingesetzt. 
Dazu muss $u(x,t)$ je zwei Mal nach $t$ und $x$ abgeleitet werden:

\begin{equation}
	\begin{aligned}
		\frac{\partial^2 u(x,t)}{\partial t^2}
		&= \frac{1}{2}\ddot{a}_0(t) + \sum_{n=1}^{\infty}\Bigl(\ddot{a}_n(t)\cos(n\omega x)+ \ddot{b}_n(t)\sin(n\omega x))
		\\
		\frac{\partial^2 u(x,t)}{\partial x^2}
		&= \sum_{n=1}^{\infty}\left(-n^2\omega^2\,a_n(t)\cos(n\omega x)-n^2\omega^2\,b_n(t)\sin(n\omega x)\right).
	\end{aligned}
\end{equation}

Von nun an wird als Ableitungsoperator $\frac{d}{dt}$ beziehungsweise äquivalent die Punktnotation, zum Beispiel $\dot{x}(t)$, verwendet, da die Gleichung nur noch von $t$ abhängig ist. Die Resultate können nun in die Wellengleichung eingesetzt werden. 
Formal entspricht die Gleichung
\begin{multline}
	\frac{1}{2}\ddot{a}_0(t)
	+ \sum_{n=1}^{\infty}\Bigl(
	\ddot{a}_n(t)\cos(n\omega x)
	+ \ddot{b}_n(t)\sin(n\omega x)
	\Bigr)
	= \\[-0.8ex]
	c^2 \sum_{n=1}^{\infty}\Bigl(
	-a_n(t)\,n^2\omega^2\cos(n\omega x)
	-b_n(t)\,n^2\omega^2\sin(n\omega x)
	\Bigr)
\end{multline}
der Wellengleichung, wobei sie hier in der Fourier-Reihendarstellung mit zeitabhängigen Koeffizienten formuliert ist.
Diese Gleichung kann nun mithilfe eines Koeffizientenvergleichs gelöst werden.
\index{Koeffizientenvergleich}%
So entstehen folgende drei Gleichungen:

\begin{equation}
	\begin{alignedat}{2}
		&\text{konstanter Term:} 
		&\quad \frac{1}{2} \frac{d^2 a_0(t)}{dt^2} &= 0 \\[0.8em]
		&\text{Koeffizient von }\cos(n\omega x): 
		&\quad \frac{d^2 a_n(t)}{dt^2} &= c^2 \cdot (-a_n(t))\,n^2\omega^2 \\[0.8em]
		&\text{Koeffizient von }\sin(n\omega x): 
		&\quad \frac{d^2 b_n(t)}{dt^2} &= c^2 \cdot (-b_n(t))\,n^2\omega^2
	\end{alignedat}
\end{equation}

Die zweite Ableitung des Mittelwerts $a_0(t)$ ist 0, somit ist $a_0(t)$ eine lineare Funktion, eine Konstante oder 0. Also ist $a_0(t)$ von der Form $a_0(t)=C_1 t + C_2$.
Die Integrationskonstanten $C_1$ und $C_2$ werden durch die Anfangsbedingungen des Systems festgelegt.
%Die Gleichungen mit $a_n(t)$ und $b_n(t)$ kann man wiederum mit einem Koefizientenvergleich lösen, diesmal jedoch über eine unendliche Summe! 
Da $n$ bis unendlich läuft, resultiert eine unendliche Anzahl von Gleichungen, jeweils eine pro harmonischer.
%Da sie alle dieselbe Form haben, bleibt man mit der Schreibweise bei $n$.
%Zudem sind die $\cos(n\omega x)$ und $\sin(n\omega x)$ zu kürzen. 
Es entstehen die folgenden zwei Arten von gewöhnlichen Differentialgleichungen:

\begin{equation}\label{eq:ODE_Wellengleichung}
	\frac{d^2}{dt^2} a_n(t) + a_n(t) n^2 \omega^2 c^2 = 0
	  \quad   \text{und} \quad  \frac{d^2}{dt^2} b_n(t) + b_n(t) n^2 \omega^2 c^2 = 0.
\end{equation}

Die Struktur der Differentialgleichungen aus~\eqref{eq:ODE_Wellengleichung} entsprechen formal der Bewegungsgleichung eines ungedämpften Federpendels: 
\begin{equation}\label{eq:loesungFederpendel}
	\frac{d^2}{dt^2}x(t) + \omega^2 x(t) = 0.
\end{equation}

Die allgemeine Lösung von~\eqref{eq:loesungFederpendel} lautet  
\begin{equation}
	x(t) = A \cos(\omega t) + B \sin(\omega t).
	%\quad\text{oder}\quad
	%x(t) = C \cos(\omega t - \varphi).
\end{equation}
Dieses Ergebnis beschreibt die Frequenz und Amplitude, mit der eine Masse an einer Feder schwingt.  
Da beide Gleichungen identische Struktur aufweisen, ergibt sich für~\eqref{eq:ODE_Wellengleichung} 
\begin{equation}
	a_n(t) = A_n \cos(n \omega c t) + B_n \sin(n \omega c t)
	\quad\text{und}\quad
	b_n(t) = A_n \cos(n \omega c t) + B_n \sin(n \omega c t).
\end{equation}
Die Resultate sind als eine unendliche Anzahl von Federpendeln zu intepretieren.
Die Kreisfrequenz ist $n \omega c$. 
Ein $a_0\neq0$ fügt lediglich einen konstanten oder linear wachsenden Versatz hinzu, Frequenz und Form der Schwingungen bleiben davon jedoch unbeeinflusst.
Beim Federpendel ist es die Masse, die hin- und herschwingt.
Welches physikalische Objekt übernimmt diese Rolle im elektromagnetischen Feld? 
Auf diese Frage wird später eingegangen.

In diesem Kapitel wurde die Thematik vereinfacht dargestellt. 
Anstelle eines dreidimensionalen Feldes $u(x,y,z,t)$ wurde lediglich eine eindimensionale Funktion $u(x,t)$ betrachtet. Die Fourier-Reihe lässt sich auch auf mehrdimensionale Felder anwenden, der damit verbundene mathematische Aufwand ist jedoch höher. 








