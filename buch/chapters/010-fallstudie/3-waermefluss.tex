%
% 3-waermefluss.tex
%
% (c) 2024 Prof Dr Andreas Müller
%

%
% Wärmefluss
%
\section{Wärmefluss
\label{buch:fallstudie:waermefluss}}
\kopfrechts{Wärmefluss}
An einer Stelle im Körper beschreibt der Gradient
\[
\operatorname{grad} T(x)
=
\nabla T
=
\renewcommand{\arraystretch}{1.9}
\begin{pmatrix}
\displaystyle \frac{\partial T}{\partial x_1}\\
\displaystyle \frac{\partial T}{\partial x_2}\\
\displaystyle \frac{\partial T}{\partial x_3}
\end{pmatrix}
\]
die Richtung der grössten Temperaturzunahme.
Ein Flächenstück senkrecht auf die Richtung des Gradienten
trennt zwei Teile des Körpers mit grösstmöglichem Temperaturunterschied,
in diese Richtung wird daher die Energie bevorzugt transportiert werden.
Die Energie fliesst durch das Flächenstück.

Steht das Flächenstück senkrecht auf dem Gradienten, kann keine
Wärme durch das Flächenstück fliessen.
Etwas allgemeiner ist der Energiefluss durch ein Flächenstück mit der
Normalen $\vec{n}$ proportional zum Skalarprodukt $\vec{n}\cdot\nabla T$
und zum Flächeninhalt des Flächenstücks.
In einer Ebene als Definitionsbereich des Temperaturfeldes muss
das Flächenstück durch eine Strecke ersetzt werden und der Flächeninhalt
durch die Länge der Strecke.

Um den Zusammenhang vollständig auszudrücken, wird ein Konstrukt 
benötigt, welches die Flächenmessung eines Randsegmentes und das
Skalarprodukt miteinander vereinigt und koordinatenunabhängig
ausdrückt.

\begin{aufgabe}
Man konstruiere eine koordinatenunabhängige Funktion, welche aus einem
Vektor den damit beschriebenen Fluss durch ein Randstück berechnet.
\end{aufgabe}

Die Lösung dieser Aufgabe wird auf den Begriff der Differentialformen
und des äusseren Produktes führen, welches den Gradienten, das Skalarprodukt
und die Konstruktion des Flächenelementes verallgemeinert.
Das äussere Produkt von Differentialformen wird ausserdem das
Vektorprodukt, welches nur in drei Dimensionen definiert ist, auf beliebige
Dimensionen verallgemeinern.

Betrachtet man einen von einer geschlossenen Fläche $S$ berandetes
Volumen in einem Körper, dann bewirken Temperaturunterschiede zwischen
Teilen des Körpers auf beiden Seiten der Fläche $S$, dass Wärme durch
die Fläche fliessen.

Unterscheiden sich zwei Punkte um den Vektor $\vec{v}$, dann ist der
Temperaturunterschied in erster Näherung gegeben durch das Skalarprodukt
$\vec{v}\cdot \nabla T$.
Sofern der Gradient parallel zur Fläche $S$ ist, kann keine Energie
durch die Fläche fliessen und die Energie im Inneren von $S$
ist erhalten.

Ist $\vec{n}$ in jedem Punkt der Fläche ein Vektor, der senkrecht
auf der Fläche steht, dann ist $\vec{n}\cdot \nabla T$ proportional
zum Wärmefluss.
Um den Energieverlust oder -gewinn durch die Fläche $S$ zu
berechnen, muss ein Integralbegriff konstruiert werden, der
nur vom Gradienten $\nabla T$ und der Oberfläche $S$ abhängt.
Die Parametrisierung der Oberfläche darf dabei genausowenig einen
Einfluss haben wie die Wahl des Koordinatensystems, mit dem der
Gradientvektor berechnet wird.

Ein weiteres Beispiel ist aus der Mechanik bekannt.
Das Gravitationsfeld $\smash{\vec{F}}$ als Vektorfeld kann als
Gradient $\smash{\vec{F}}=\nabla\varphi$ des Gravitationspotenials
$\varphi$ gefunden werden.
Die Arbeit, die zwischen den Punkten $A$ und $B$ geleistet
wird, wird durch das Integral
\[
W
=
\int_A^B \vec{F}\cdot d\vec{s}
=
\int_A^B \nabla \varphi \cdot d\vec{s}
\]
unabhängig vom gewählten Weg berechnet.
Auch in diesem Fall ist eine zentrale physikalische Grösse durch
ein Integral von Funktionen und Ableitungen gegeben, welches über
eine in diesem Fall eindimnensionale Untermannigfaltigkeit 
erstreckt wird.
Sowohl die Ableitungen wie auch die Parametrisierung des Weges
sind koordinatenabhängig.

\begin{aufgabe}
Konstruiere eine koordinatenunabhängige Theorie der Integration
von Funktionen und Ableitungen auf beliebigen Untermannigfaltigkeiten
des Raumes und Sätze, die Bilanzbeziehungen wie die Energieerhaltung
beim Fluss Wärmeenergiefluss durch eine Oberfläche zu formulieren
erlauben.
\end{aufgabe}

