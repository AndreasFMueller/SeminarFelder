%
% 1-hodge.tex -- Der Hodge-Operator
%
% (c) 2025 Prof Dr Andreas Müller
%
\section{Der Hodge-Operator
\label{buch:hodge:section:hodge}}
\kopfrechts{Der Hodge-Operator}
Der Hodge-Operator bildet $p$-Formen umkehrbar auf $n-p$-Formen ab.
Im folgenden wird er zunächst ad hoc mit Hilfe der Basis-$p$-Formen
konstruiert.

%
% Basisformen
%
\subsection{Basisformen}
Der Raum der $p$-Formen hat als Basis die $p$-Formen der Form
\[
\alpha = dx^{i_1}\wedge\dots\wedge dx^{i_p}.
\]
Die Indizes $i_1,\dots,i_p$ müssen alle verschieden sein, weil
$dx^i\wedge dx^i=0$ ist.
Zu jeder Auswahl von $p$ Indizes $i_1,\dots,i_p$ aus der Mengen
$[n]=\{1,\dots,n\}$ der Zahlen von 1 bis $n$ gibt es genau eine Basisform.
Die Dimension des Raums der $p$-Formen ist daher die Anzahl der möglichen
Auswahlen von $p$ Zahlen aus den $n$ Zahlen $[n]$.

Die Auswahl von $p$ Zahlen aus $[n]$ lässt aber genau $n-p$ Basisformen
zurück.
Aus den Indizes $n-p$ Zahlen aus $]n]$, die in $i_1,\dots,i_p$ nicht
vorkommen, lässt sich eine Basis $n-p$-Form bilden.
Diese Zuordnung einer Basis-$p$-Form $dx^{i_1}\wedge\dots\wedge dx^{i_p}$
ist offensichtlich umkehrbar.
Es bleibt noch die Freiheit, das Vorzeichen festzulegen.

\begin{definition}[Hodge-Operator]
\label{buch:hodge:hodge:definition:hodge}
Der \emph{Hodge-Operator} ist der lineare Operator, der der
\index{Hodge-Operator}%
Basis-$p$-Form
$dx^{i_1}\wedge\dots\wedge dx^{i_p}$ mit $i_1<i_2<\dots<i_p$
die $n-p$-Form
\[
\ast(dx^{i_1}\wedge\dots\wedge dx^{i_p})
=
s\,dx^{j_1}\wedge\dots\wedge dx^{j_{n-p}}
\]
mit $j_1<\dots <j_{n-p}$, $s\in\{\pm1\}$ zuordnet, für die
$\{i_1,\dots,i_p\}\cup\{j_1,\dots,j_{n-p}\}=[n]$ gilt.
Das Vorzeichen $s$ muss so gewählt werden, dass
\[
(dx^{i_1}\wedge\dots\wedge dx^{i_p})
\wedge
(s\,dx^{j_1}\wedge\dots\wedge dx^{j_{n-p}})
=
dx^1\wedge dx^2\wedge\dots\wedge dx^n
\]
gilt.
\index{*-Operator@$*$-Operator}%
\end{definition}

Ganz allgemein lässt sich daraus bereits beantworten, wie $0$-Formen und
$n$-Formen durch den Hodge-Operator abgebildet werden.
Die einzige 0-Form ist $\alpha=1$.
Es muss eine $n$-Form $\beta$ gefunden werden, so dass
$\alpha\wedge\beta=dx^1\wedge\dots\wedge dx^n$ ist.
Da es nur die eine $n$-Form $dx^1\wedge\dots\wedge dx^n$ gibt, muss
nur noch das Vorzeichen gefunden werden.
Da aber
\[
\alpha\wedge
s\,dx^1\wedge\dots\wedge dx^n
=
s\,dx^1\wedge\dots\wedge dx^n
=
dx^1\wedge\dots\wedge dx^n
\]
sein soll, muss $s=1$ gewählt werden.
Somit ist $\ast 1=dx^1\wedge\dots\wedge dx^n$.
Auf die gleiche Art kann man folgern, dass
$\ast(dx^1\wedge\dots\wedge dx^n)=1$.

\begin{beispiel}
\label{buch:hodge:hodge:beispiel:r2}
Hodge-Operator in $\mathbb{R}^2$.
Die Basis-1-Formen sind $dx^1$ und $dx^2$ und die Form höchsten Grades ist
die 2-Form $dx^1\wedge dx^2$.
Aus der Definition lässt sich jetzt sofort die Wirkung des Hodge-Operators
ableiten.
Für 0- und 2-Formen wurde dies im Anschluss an die
Definition~\ref{buch:hodge:hodge:definition:hodge} durchgeführt.

Die 1-Formen $dx^1$ und $dx^2$ werden aufeinander abgebildet, es müssen
aber noch die Vorzeichen ermittelt werden.
Wenn $\*dx^1=s\,dx^2$ ist, dann soll
$dx^1\wedge \*dx^1=dx^1\wedge s\,dx^2=dx^1\wedge dx^2$ sein.
Somit muss $s=1$ gewählt werden und es folgt $\*dx^1=dx^2$.
Wenn $\*dx^2=s\,dx^1$ ist, dann soll
$dx^2\wedge(\*dx^2)=dx^2\wedge(s\,dx^1)=dx^1\wedge dx^2$ sein.
Dies kann erreicht werden, dann man im mittleren Ausdruck der
Gleichungskette $dx^1$ und $dx^2$ vertauscht und dafür $s=-1$ setzt.
Es folgt $\*dx^2=-dx^1$.
\end{beispiel}

\begin{beispiel}
\label{buch:hodge:hodge:beispiel:r3}
Hodge-Operator in $\mathbb{R}^3$.
Im dreidimensionalen Fall ist nur die Abbildung von 1-Formen auf
2-Formen und zurück zu untersuchen.
Nach der gleichen Methode wie im
Beispiel~\ref{buch:hodge:hodge:beispiel:r2}
sind nur noch die Vorzeichen zu bestimmen.
Die Rechnung ergibt die Zuordnung
\begin{align*}
\ast dx^1 &= \phantom{-}dx^2\wedge dx^3 \\
\ast dx^2 &=          - dx^1\wedge dx^3 \\
\ast dx^3 &= \phantom{-}dx^1\wedge dx^2 \\
\ast(dx^1\wedge dx^2) &= \phantom{-}dx^3 \\
\ast(dx^1\wedge dx^3) &=          - dx^2 \\
\ast(dx^2\wedge dx^3) &= \phantom{-}dx^1
\qedhere
\end{align*}
\end{beispiel}

Da die Indexmengen $\{i_1,\dots,i_p\}$ und
$\{j_1,\dots,j_{n-p}\}$ in $[n]$ komplementär sein müssen,
muss
\(
\ast{\ast}(dx^{i_1}\wedge\dots\wedge dx^{i_p})
=
\pm dx^{i_1}\wedge\dots\wedge dx^{i_p}
\)
sein.
Aus den Beispielen kann man ablesen, dass in den Fällen $n=2$ und $n=3$
der iterierte Hodge-Operator  $\ast\ast$ durch
\[
\left.
\begin{aligned}
\ast{\ast dx^1} &= \phantom{-}{\ast dx^2} = -dx^1\\
\ast{\ast dx^2} &=          - {\ast dx^1} = -dx^2
\end{aligned}
\quad
\right\}
\qquad
\text{bzw.}
\qquad
\left\{
\quad
\begin{aligned}
\ast{\ast dx^1} &= \phantom{-}{\ast(dx^2\wedge dx^3)} = dx^1 \\
\ast{\ast dx^2} &=          - {\ast(dx^1\wedge dx^3)} = dx^2 \\
\ast{\ast dx^3} &= \phantom{-}{\ast(dx^1\wedge dx^2)} = dx^3 
\end{aligned}
\right.
\]
gegeben ist.
Diese Überlegungen können verallgemeinert werden und ergeben den
folgenden Satz.

\begin{satz}
Für jede $p$-Form auf einer $n$-dimensionalen Mannigfaltigkeit gilt
\[
\ast{\ast \omega}
=
(-1)^{p(n-p)}\omega.
\]
\end{satz}

\begin{proof}
Sei $\omega = dx^{i_1}\wedge\dots\wedge dx^{i_p}$ und 
$\ast\omega = s\,dx^{j_1}\wedge\dots\wedge dx^{j_{n-p}}$ mit
$s\in\{\pm 1\}$.
Das Vorzeichen $s$ kommt von den Vertauschungen der Basis-1-Formen,
mit denen man die Standardreihenfolge $dx^1\wedge \dots\wedge dx^n$
erreicht.
Es erfüllt
\begin{equation}
(dx^{i_1}\wedge\dots\wedge dx^{i_p})
\wedge
s(dx^{j_1}\wedge\dots\wedge dx^{j_{n-p}})
=
dx^1\wedge\dots\wedge dx^n.
\label{buch:hodge:hodge:vorz1}
\end{equation}

Jetzt soll $\ast{\ast\,\omega}$ berechnet werden.
Nach Definition gibt es einen Vorzeichenfaktor $t\in\{\pm1\}$ mit
\[
\ast(dx^{j_1}\wedge\dots\wedge dx^{j_{n-p}})
=
t\,dx^{i_1}\wedge\dots\wedge dx^{i_p}.
\]
Er muss so gewählt werden, dass
\begin{align}
(dx^{j_1}\wedge\dots\wedge dx^{j_{n-p}})
\wedge
t(dx^{i_1}\wedge\dots\wedge dx^{i_p})
&=
dx^1\wedge\dots\wedge dx^n
\label{buch:hodge:hodge:vorz2}
\end{align}
gilt.
Für den iterierten Hodge-Operator findet man dann
\begin{align}
\ast{\ast\, \omega}
=
\ast(s\,dx^{j_1}\wedge\dots\wedge dx^{j_{n-p}})
&=
st\, dx^{i_1}\wedge\dots\wedge dx^{i_p}
\notag
\end{align}
Der iterierte Hodge-Operator ist also das Vorzeichen $st\in\{\pm 1\}$.

Durch $p$ Vertauschungen mit den 1-Formen $dx^{j_p}$,
$dx^{j_{p-1}},\dots,dx^{j_1}$ auf der linken Seite von
\eqref{buch:hodge:hodge:vorz2}
kann man $dx^{i_1}$ an den Anfang
der $n$-Form bringen.
Durch Widerholung für die 1-Formen $dx^{i_2}$ bis $dx^{i_p}$ kann
man mit insgesamt $p(n-p)$ Vertauschungen alle $dx^{i_k}$, $k=1,\dots,p$
nach vorne bringen und erhält so
\begin{align}
(-1)^{p(n-p)}
t(dx^{i_1}\wedge\dots\wedge dx^{i_p})
\wedge
(dx^{j_1}\wedge\dots\wedge dx^{j_p})
&=
dx^1\wedge\dots\wedge dx^n.
\end{align}
Durch Vergleich mit 
\eqref{buch:hodge:hodge:vorz1}
kann man ablesen, dass $(-1)^{p(n-p)}t=s$ gelten muss.
Daraus kann man
$st=(-1)^{p(n-p)}$ ablesen, was den Satz beweist.
\end{proof}

Wenn $n$ ungerade ist, dann ist immer einer der beiden Faktoren $p$
oder $n-p$ gerade und damit ist $(-1)^{p(n-p)}=1$.
Für ungerades $n$ ist der Hodge-Operator also eine Involution.
\index{Involution}%
Für gerades $n$ sind entweder beide Zahlen $p$ und $n-p$ ungerade
oder keine von beiden.
Für $p$ ungerade ist auch $n-p$ und damit $p(n-p)$ ungerade
und entsprechend ist $\ast{\ast\omega}=-\omega$ für $p$-Formen.
Der iterierte Hodge-Operator ist also eine Antiinvolution auf
$p$-Formen ungeraden Grades.

