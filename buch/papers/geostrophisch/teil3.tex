%
% teil3.tex -- Beispiel-File für Teil 3
%
% (c) 2020 Prof Dr Andreas Müller, Hochschule Rapperswil
%
% !TEX root = ../../buch.tex
% !TEX encoding = UTF-8
%
\section{Geostrophische Näherung / heutiger Ansatz
\label{geostrophisch:section:geoNäherung}}
\kopfrechts{Teil 3}
Der Heutige Ansatz... 

\begin{equation}
\boldsymbol{
f\, *\vec{k} \times \vec{v}_g 
+
\frac{1}{\rho} \nabla p
=
0
}
\label{geostrophisch:equation3}
\end{equation}
Die Gleichung besagt, der geostrophische Wind entsteht genau dann, wenn sich die Druckgradientenkraft\eqref{geostrophisch:equation2} und die Corioliskraft\eqref{geostrophisch:equation1} ausgleichen.
Heisst \eqref{geostrophisch:equation1} Corioliskraft ist gleich \eqref{geostrophisch:equation2} Gradientenkraft.




