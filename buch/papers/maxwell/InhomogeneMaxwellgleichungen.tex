%
% InhomogeneMaxwellgleichungen.tex -- Beispiel-File für das Paper
%
% (c) 2020 Prof Dr Andreas Müller, Hochschule Rapperswil
%
% !TEX root = ../../buch.tex
% !TEX encoding = UTF-8
%
\section{Inhomogene Maxwellgleichungen
	\label{maxwell:section:InhomogeneMaxwellgleichungen}}
\kopfrechts{Inhomogene Maxwellgleichungen}
Den Faraday-Tensor haben wir definiert als
\begin{equation*}
	F_{\mu \nu} = \begin{pmatrix}
		0 & -E_1/c & -E_2/c & -E_3/c \\ E_1/c & 0 & B_3 & -B_2 \\ E_2/c & -B_3 & 0 & B_1 \\ E_3/c & B_2 & -B_1 & 0 
	\end{pmatrix}.
\end{equation*}
Wir haben dabei festgestellt, dass die äussere Ableitung von $F$ verschwindet, da $F$ eine geschlossene 2-Form ist. Dadurch haben wir die homogenen Maxwell-Gleichungen erhalten. 
%Um nun die inhomogenen Maxwell-Gleichungen bestimmen zu können, müssen wir den Hodge-Operator auf $F$ anwenden.

Um nun die inhomogenen Maxwell-Gleichungen bestimmen zu können, benötigen wir eine Art Verallgemeinerung der Divergenz, die das Quellenverhalten des Faraday-Tensors misst. Hierfür eignet sich das Kodifferential
\begin{equation*}
	\delta = \ast d {\ast}.
\end{equation*}
%Diese Operation erlaubt es, quellenartige Beiträge geometrisch zu erfassen.
Aus den klassischen inhomogenen Maxwell-Gleichungen
\begin{align}
	\nabla \cdot \vec{E} &= \frac{\rho}{\varepsilon_0}
	\label{maxwell:section:inhomogeneMaxwellgleichungen:gaussE},\\
	\nabla \times \vec{B} - \varepsilon_0 \mu_0 \frac{\partial \vec{E}}{\partial t}&= \mu_0 \vec{J}
	\label{maxwell:section:inhomogeneMaxwellgleichungen:ampereMaxwell}
\end{align}
sehen wir, welche Operationen mit den $\vec{E}$ und $\vec{B}$ gemacht werden müssen.
Da im Faraday-Tensor die elektrische Feldstärke $\vec{E}$ und magnetischen Flussdichte $\vec{B}$ enthalten sind, wenden wir das Kodifferential auf $F$ an und untersuchen, ob die obigen Operationen in der Sprache der Differentialformen herauskommen.

Damit wir wissen, welche geometrische Form die Operation 
\begin{equation*}
	\delta F = \ast d{\ast} F
\end{equation*}
haben wird, überlegen wir uns, was der Hodge-Operator und die äussere Ableitung mit dem Grad machen.

Der Hodge-Operator bildet eine $k$-Form auf eine $(n-k)$-Form ab.
Da wir uns im vierdimensionalen Minkowsky-Raum befinden wird durch Anwendung des Hodge-Operators auf die 2-Form $F$ wieder eine 2-Form.
Anwenden der äusseren Ableitung erhöht den Grad um eins und ergibt somit eine 3-Form.
Durch erneutes Anwenden des Hodge-Operators entsteht schlussendlich eine 1-Form.

Aus den Gleichungen \eqref{maxwell:section:inhomogeneMaxwellgleichungen:gaussE} und \eqref{maxwell:section:inhomogeneMaxwellgleichungen:ampereMaxwell} sehen wir, dass auf der rechten Seite Terme in Form von Ladungs- und Stromdichten auftreten.
Es liegt nahe, auch $\rho$ und $\vec{J}$ zu einem einzigen geometrischen Objekt zusammenzufassen, weil beides Quellenterme sind und das Ziel ist, beide inhomogenen Maxwell-Gleichungen in eine Gleichung unterzubringen.

Folglich muss das geometrische Objekt, welches die Ladungs- und Stromdichte enthält, eine 1-Form sein.
Diese werden wir zukünftig Viererstromdichte $J$ nennen.
Sobald wir das $\ast d {\ast} F$ berechnet haben, definieren wir $J$ so, dass die Gleichung aufgeht.

\subsection{Berechnung des Kodifferentials von $F$}

Wir erinnern uns, dass $F$ ausgeschrieben mit Wedge-Produkten definiert ist als
\begin{align*}
	F = 
	- &\frac{E_{1}}{c} \, dx^0 \wedge dx^1 - \frac{E_{2}}{c} \, dx^0 \wedge dx^2 - \frac{E_{3}}{c} \, dx^0 \wedge dx^3 \\
	+ & \, B_3 \, \, dx^1 \wedge dx^2 - B_2 \, \, dx^1 \wedge dx^3 + \, B_1 \, \, dx^2 \wedge dx^3.
\end{align*}

\subsubsection{Berechnung von $\ast F$}
Zuerst wenden wir den Hodge-Operator auf $F$ an und erhalten
\begin{align*}
	\ast F =
	& - \frac{E_{1}}{c} \, {\ast}(dx^0 \wedge dx^1) - \frac{E_{2}}{c} \, {\ast}(dx^0 \wedge dx^2) - \frac{E_{3}}{c} \, {\ast}(dx^0 \wedge dx^3) \\
	& + \, B_3 \, \, {\ast}(dx^1 \wedge dx^2) - \, B_2 \, \, {\ast}(dx^1 \wedge dx^3) + B_1 \, {\ast}(dx^2 \wedge dx^3).
\end{align*}
Die Hodge-Duale können aus der Tabelle \ref{maxwell:section:teil1:Hodge-Tabelle} entnommen werden und wir erhalten
\begin{align*}
	\ast F =
	& \, \frac{E_{1}}{c} \, dx^2 \wedge dx^3 - \frac{E_{2}}{c} \, dx^1 \wedge dx^3 + \frac{E_{3}}{c} \, dx^1 \wedge dx^2 \\
	 + & \, B_3 \, \, dx^0 \wedge dx^3 + \, B_2 \, \, dx^0 \wedge dx^2 + \, B_1 \, \, dx^0 \wedge dx^1.
\end{align*}
Die erhaltene 2-Form kann nun wieder als Matrix
\begin{equation}
	\ast F = \begin{pmatrix}
		0 & B_1 & B_2 & B_3 \\ -B_1 & 0 & E_3/c & -E_2/c \\ -B_2 & -E_3/c & 0 & E_1/c \\ -B_3 & E_2/c & -E_1/c & 0 
	\end{pmatrix}
	%	\label{maxwell:section:teil1:metrik}
\end{equation}
geschrieben werden.
Wir sehen, dass durch den Hodge-Operator die $E$ und $B$ ihre Plätze tauschen.
\subsubsection{Berechnung von $d {\ast} F$}
Nun wenden wir die äussere Ableitung an und erhalten
\begin{align*}
	d{\ast} F
	= &\phantom{+} d \, \Bigg(
	\frac{E_{1}}{c} \, dx^2 \wedge dx^3
	- \frac{E_{2}}{c} \, dx^1 \wedge dx^3
	+ \frac{E_{3}}{c} \, dx^1 \wedge dx^2 
	\\
	&+ B_3 \, dx^0 \wedge dx^3
	+ B_2 \, dx^0 \wedge dx^2
	+ B_1 \, dx^0 \wedge dx^1
	\Bigg) \\[1ex]
	= \phantom{+} &\frac{1}{c} \frac{\partial E_1}{\partial x^0} \, dx^0 \wedge dx^2 \wedge dx^3
	+ \frac{1}{c} \frac{\partial E_1}{\partial x^1} \, dx^1 \wedge dx^2 \wedge dx^3
	- \frac{1}{c} \frac{\partial E_2}{\partial x^0} \, dx^0 \wedge dx^1 \wedge dx^3 
	\\
	- &\frac{1}{c} \frac{\partial E_2}{\partial x^2} \, dx^2 \wedge dx^1 \wedge dx^3
	+ \frac{1}{c} \frac{\partial E_3}{\partial x^0} \, dx^0 \wedge dx^1 \wedge dx^2
	+ \frac{1}{c} \frac{\partial E_3}{\partial x^3} \, dx^3 \wedge dx^1 \wedge dx^2 
	\\
	+&\phantom{\frac{1}{c}} \frac{\partial B_3}{\partial x^1} \, dx^1 \wedge dx^0 \wedge dx^3
	+ \phantom{\frac{1}{c}} \frac{\partial B_3}{\partial x^2} \, dx^2 \wedge dx^0 \wedge dx^3
	+ \phantom{\frac{1}{c}} \frac{\partial B_2}{\partial x^1} \, dx^1 \wedge dx^0 \wedge dx^2 \\
	+&\phantom{\frac{1}{c}} \frac{\partial B_2}{\partial x^3} \, dx^3 \wedge dx^0 \wedge dx^2
	+ \phantom{\frac{1}{c}} \frac{\partial B_1}{\partial x^2} \, dx^2 \wedge dx^0 \wedge dx^1
	+ \phantom{\frac{1}{c}} \frac{\partial B_1}{\partial x^3} \, dx^3 \wedge dx^0 \wedge dx^1 
	\\[2ex]
	= \phantom{+} &\left(
	\phantom{+} \frac{1}{c} \frac{\partial E_1}{\partial x^1}
	+ \frac{1}{c} \frac{\partial E_2}{\partial x^2}
	+ \frac{1}{c} \frac{\partial E_3}{\partial x^3}
	\right) dx^1 \wedge dx^2 \wedge dx^3 
	\\
	+ &\left(
	\phantom{+} \frac{1}{c} \frac{\partial E_1}{\partial x^0}
	+ \phantom{\frac{1}{c}} \frac{\partial B_2}{\partial x^3}
	- \phantom{\frac{1}{c}} \frac{\partial B_3}{\partial x^2}
	\right) dx^0 \wedge dx^2 \wedge dx^3 
	\\
	+ &\left(
	-\frac{1}{c} \frac{\partial E_2}{\partial x^0}
	+ \phantom{\frac{1}{c}} \frac{\partial B_1}{\partial x^3}
	- \phantom{\frac{1}{c}} \frac{\partial B_3}{\partial x^1}
	\right) dx^0 \wedge dx^1 \wedge dx^3 
	\\
	+ &\left(
	\phantom{+} \frac{1}{c} \frac{\partial E_3}{\partial x^0}
	+ \phantom{\frac{1}{c}} \frac{\partial B_1}{\partial x^2}
	- \phantom{\frac{1}{c}} \frac{\partial B_2}{\partial x^1}
	\right) dx^0 \wedge dx^1 \wedge dx^2.
\end{align*}
\subsubsection{Berechnung von $\ast d {\ast} F$}
Wenden wir nun noch einmal den Hodge-Operator auf die erhaltene 3-Form an, ergibt sich gemäss Tabelle ~\ref{maxwell:section:teil1:Hodge-Tabelle}
\begin{align*}
	\ast d {\ast} F 
	= 
	\phantom{+} &\left(\phantom{+} \frac{1}{c}\frac{\partial E_1}{\partial x^1} + \frac{1}{c}\frac{\partial E_2}{\partial x^2} + \frac{1}{c}\frac{\partial E_3}{\partial x^3} \right) (-dx^0) +
	\left(\frac{1}{c}\frac{\partial E_1}{\partial x^0} + \frac{\partial B_2}{\partial x^3} - \frac{\partial B_3}{\partial x^2} \right) (-dx^1)
	\\
	+ &\left( -\frac{1}{c}\frac{\partial E_2}{\partial x^0} + \phantom{\frac{1}{c}} \frac{\partial B_1}{\partial x^3} - \phantom{\frac{1}{c}} \frac{\partial B_3}{\partial x^1} \right) \quad dx^2 \: \, +
	\left( \frac{1}{c}\frac{\partial E_3}{\partial x^0} + \frac{\partial B_1}{\partial x^2} - \frac{\partial B_2}{\partial x^1} \right) (-dx^3).\\[2ex]
	=
	\phantom{+} &\left( -\frac{1}{c}\frac{\partial E_1}{\partial x^1} -\frac{1}{c}\frac{\partial E_2}{\partial x^2} - \frac{1}{c}\frac{\partial E_3}{\partial x^3} \right) dx^0 +
	\left(-\frac{1}{c}\frac{\partial E_1}{\partial x^0} - \frac{\partial B_2}{\partial x^3} + \frac{\partial B_3}{\partial x^2} \right) dx^1
	\\
	+ &\left( -\frac{1}{c}\frac{\partial E_2}{\partial x^0} + \phantom{\frac{1}{c}} \frac{\partial B_1}{\partial x^3} - \phantom{\frac{1}{c}} \frac{\partial B_3}{\partial x^1} \right) dx^2 +
	\left( -\frac{1}{c}\frac{\partial E_3}{\partial x^0} - \frac{\partial B_1}{\partial x^2} + \frac{\partial B_2}{\partial x^1} \right) dx^3.
\end{align*}
%Damit wir nun wieder auf etwas physikalisch Sinnvolles kommen, ersetzen wir alle $dx^0$ mit $cdt$, $dx^1$ mit $dx$, %$dx^2$ mit $dy$ und $dx^3$ mit $dz$ und erhalten
Damit wir nun wieder auf etwas physikalisch leichter Verständliches kommen, ersetzen wir
\begin{equation}
	dx^0 = c\,dt, \quad dx^1 = dx, \quad dx^2 = dy, \quad dx^3 = dz,
\end{equation}
und erhalten
\begin{align*}
	\ast d {\ast} F 
	= 
	\phantom{+} &\left( -\frac{1}{c}\frac{\partial E_1}{\partial x} -\frac{1}{c}\frac{\partial E_2}{\partial y} - \frac{1}{c}\frac{\partial E_3}{\partial z} \right) c \, dt +
	\left(-\frac{1}{c}\frac{\partial E_1}{\partial ct} - \frac{\partial B_2}{\partial z} + \frac{\partial B_3}{\partial y} \right) dx
	\\
	+ &\left( -\frac{1}{c}\frac{\partial E_2}{\partial ct} + \phantom{\frac{1}{c}} \frac{\partial B_1}{\partial z} - \phantom{\frac{1}{c}} \frac{\partial B_3}{\partial x} \right) \phantom{c \,}dy +
	\left( -\frac{1}{c}\frac{\partial E_3}{\partial ct} - \frac{\partial B_1}{\partial y} + \frac{\partial B_2}{\partial x} \right) dz
	\\[2ex]
	=
	\phantom{+} &\left( -\phantom{\frac{1}{c^2}}\frac{\partial E_1}{\partial x} -\frac{\partial E_2}{\partial y} - \frac{\partial E_3}{\partial z} \right) dt +
	\left(-\frac{1}{c^2}\frac{\partial E_1}{\partial t} - \frac{\partial B_2}{\partial z} + \frac{\partial B_3}{\partial y} \right) dx
	%\label{maxwell:section:inhomogeneMaxwellgleichungen:linkeSeite}
	\\
	+ &\left( -\frac{1}{c^2}\frac{\partial E_2}{\partial t} + \frac{\partial B_1}{\partial z} - \frac{\partial B_3}{\partial x} \right) dy +
	\left( -\frac{1}{c^2}\frac{\partial E_3}{\partial t} - \frac{\partial B_1}{\partial y} + \frac{\partial B_2}{\partial x} \right) dz.
\end{align*}
Mit Hilfe der Lichtgeschwindigkeit im Vakuum 
\begin{equation*}
	c = \frac{1}{\sqrt{\varepsilon_0 \mu_0}}
\end{equation*}
können wir die Gleichung noch etwas umformulieren und erhalten
\begin{align}
	\ast d {\ast} F
	=
	\phantom{+} &\left( -\phantom{\varepsilon_0\mu_0}\frac{\partial E_1}{\partial x} -\frac{\partial E_2}{\partial y} - \frac{\partial E_3}{\partial z} \right) dt +
	\left(-\varepsilon_0\mu_0\frac{\partial E_1}{\partial t} - \frac{\partial B_2}{\partial z} + \frac{\partial B_3}{\partial y} \right) dx
	\label{maxwell:section:inhomogeneMaxwellgleichungen:linkeSeite}
	\\
	+ &\left( -\varepsilon_0\mu_0\frac{\partial E_2}{\partial t} + \frac{\partial B_1}{\partial z} - \frac{\partial B_3}{\partial x} \right) dy +
	\left( -\varepsilon_0\mu_0\frac{\partial E_3}{\partial t} - \frac{\partial B_1}{\partial y} + \frac{\partial B_2}{\partial x} \right) dz. \notag
\end{align}

Dieser Ausdruck soll nun also alle Teile der einen Seite der inhomogenen Maxwell-Gleichungen enthalten.
Wir erkennen die Divergenz der elektrischen Feldstärke, die Rotation der magnetischen Flussdichte und die zeitliche Ableitung der elektrischen Feldstärke in den jeweiligen Komponenten.
\subsection{Bestimmung der Viererstromdichte $J$}

Unser Ziel ist es nun, eine 1-Form zu finden, welche die Ladungs- und Stromdichte enthält und dem Ausdruck \eqref{maxwell:section:inhomogeneMaxwellgleichungen:linkeSeite} gleichgesetzt werden kann.
Wenn uns das gelingt, haben wir eine kovariante Formulierung der inhomogenen Maxwell-Gleichungen gefunden.

Der erste Term von \eqref{maxwell:section:inhomogeneMaxwellgleichungen:linkeSeite} muss, wegen dem gaussschen Gesetz der Elektrostatik gemäss \eqref{maxwell:section:inhomogeneMaxwellgleichungen:gaussE}, 
\begin{equation*}
	-\frac{\rho}{\varepsilon_0} \,dt
\end{equation*}
entsprechen.

Die anderen drei Terme mit $dx,dy,dz$ deuten stark auf die linke Seite der Ampère-Maxwell-Gleichung gemäss \eqref{maxwell:section:inhomogeneMaxwellgleichungen:ampereMaxwell} hin.
Beachten wir, dass 
\begin{equation*}
	\nabla \times \vec{B} = 
	\begin{pmatrix}
		\frac{\partial B_3}{\partial y} - \frac{\partial B_2}{\partial z}\\[1ex]
		\frac{\partial B_1}{\partial z} - \frac{\partial B_3}{\partial x}\\[1ex]
		\frac{\partial B_2}{\partial x} - \frac{\partial B_1}{\partial y}
	\end{pmatrix}
\end{equation*}
gilt, erkennen wir nun deutlich, dass gemäss \eqref{maxwell:section:inhomogeneMaxwellgleichungen:ampereMaxwell},
\begin{align*}
	&\left(-\varepsilon_0\mu_0\frac{\partial E_1}{\partial t} - \frac{\partial B_2}{\partial z} + \frac{\partial B_3}{\partial y} \right) dx = \mu_0 J_1 \, dx\\
	&\left( -\varepsilon_0\mu_0\frac{\partial E_2}{\partial t} + \frac{\partial B_1}{\partial z} - \frac{\partial B_3}{\partial x} \right) dy = \mu_0 J_2 \, dy\\
	&\left( -\varepsilon_0\mu_0\frac{\partial E_3}{\partial t} - \frac{\partial B_1}{\partial y} + \frac{\partial B_2}{\partial x} \right) dz = \mu_0 J_3 \, dz
\end{align*}
gelten muss.

Nun haben wir also herausgefunden dass
\begin{equation}
	\ast d {\ast} F = -\frac{\rho}{\varepsilon_0}\,dt + \mu_0 (J_1 \,dx +  J_2 \,dy +  J_3 \,dz)
	\label{maxwell:section:inhomogemeMaxwellgleichungen:hdhF}
\end{equation}
sein muss.

Da alle räumlichen Komponenten auf der rechten Seite von \eqref{maxwell:section:inhomogemeMaxwellgleichungen:hdhF} proportional zu $\mu_0$ sind und der erste Term die Ladungsdichte über $\varepsilon_0$ enthält, bietet es sich an, $\mu_0$ auszuklammern und als Vorfaktor zu verwenden.
Somit ergibt sich
\begin{align*}
	\ast d {\ast} F
	& = \mu_0 \left( -\frac{\rho}{\varepsilon_0 \mu_0}\,dt + J_1\, dx + J_2\, dy + J_3\, dz \right)\\[2ex]
	& = \mu_0 \left( -c^2 \rho \,dt + J_1 \,dx + J_2 \,dy + J_3 \,dz \right).
\end{align*}
Den Klammerausdruck definieren wir nun als Viererstromdichte
\begin{equation}
	J = -c^2 \rho\, dt + J_1 \,dx + J_2 \,dy + J_3 \,dz.
\end{equation}
In koordinatenfreier Sprache mit $dx^0 = c\,dt, dx^1 = dx,dx^2=dy$ und $dx^3 = dz$ entspricht dies
\begin{equation}
	J = -c\,\rho\,dx^0 + J_1 \,dx^1 + J_2 \,dx^2 + J_3 \,dx^3.
\end{equation}

Wir haben nun also eine sehr kompakte Form gefunden, die inhomogenen Maxwell-Gleichungen zu formulieren.
Die gesamte Information der beiden Gleichungen steckt in
\begin{equation}
	\ast d {\ast} F = \mu_0 J.
\end{equation}
Zusätzlich ist die Formulierung kovariant und behält somit seine Form bei Koordinatentransformationen.
\begin{comment}
Damit wir nun den erhaltenen Ausdruck auch interpretieren können definieren wir die Viererstromdichte als
\begin{equation}
	J = -c\rho dx^0 + J_1 dx^1 + J_2 dx^2 +J_3 dx^3.
\end{equation}
Schreiben wir auch diese 1-Form in physikalisch sinnvollen Einheiten erhalten wir 
\begin{equation}
	J = -c^2\rho dt + J_1 dx + J_2 dy + J_3 dz.
\end{equation}
Multiplizieren wir zu $J$ noch ein $\mu_0$ erhalten wir den Ausdruck
\begin{equation}
	\mu_0 J = -\frac{\rho}{\varepsilon_0}dt + \mu_0 (J_1 dx +  J_2 dy +  J_3 dz),
\end{equation}
weil im Vakuum $c = \frac{1}{\sqrt{\varepsilon_0 \mu_0}}$ gilt. Setzen wir nun die beiden Teile der Gleichung zusammen ergibt sich
\begin{equation}
	\ast d \ast F = \mu_0 J.
\end{equation}
Jetzt können wir die einzelnen Teile vergleichen und wissen bereits, dass es sich um vier unabhängige Gleichungen handeln muss, nämlich eine für jede Basis-1-Form.

Betrachten wir nur die Koeffizienten der Zeitkomponente $dt$, resultiert die Gleichung
\begin{equation}
	\frac{\partial E_1}{\partial x} +\frac{\partial E_2}{\partial y} + \frac{\partial E_3}{\partial z} = \frac{\rho}{\varepsilon_0},
\end{equation}
(Frage: hier die $dt$ auf beiden Seiten der Gleichung explizit hinschreiben damit klar ist, dass es sich um eine 1-Form handelt oder ist das egal?)

welche als das Gausssche Gesetz der Elektrostatik bekannt ist.
%Weil es sich um Differentialformen handelt, wird hier nicht der Nabla-Operator verwendet für die Divergenz der %elektrischen Feldstärke.
diese Gleichung lässt sich, sofern man eine kartesische Raumstruktur wählt, in die klassische Vektorform
\begin{equation}
	\nabla \cdot \vec{E} = \frac{\rho}{\varepsilon_0}
\end{equation}
umschreiben.

Betrachten wir nun die restlichen Teile der Gleichung $\ast d \ast F = \mu_0 J,$ ergeben sich durch Vergleich der Koeffizienten von $dx,dy,dz$ die drei Gleichungen
\begin{equation}
	-\frac{1}{c^2}\frac{\partial E_1}{\partial t} - \frac{\partial B_2}{\partial z} + \frac{\partial B_3}{\partial y} = \mu_0 J_1
\end{equation}
\begin{equation}
	-\frac{1}{c^2}\frac{\partial E_2}{\partial t} + \frac{\partial B_1}{\partial z} - \frac{\partial B_3}{\partial x} = \mu_0 J_2\\
\end{equation}
\begin{equation}
	 -\frac{1}{c^2}\frac{\partial E_3}{\partial t} - \frac{\partial B_1}{\partial y} + \frac{\partial B_2}{\partial x} = \mu_0 J_3.
\end{equation}
(auch hier Frage: sollen wir die $dx,dy,dz$ hinschreiben?)

Schreiben wir erneut $c$ mit der Permittivität und Permeabilität des Vakuums erhalten wir die Gleichungen
\begin{equation}
	 - \frac{\partial B_2}{\partial z} + \frac{\partial B_3}{\partial y} = \mu_0 J_1 + \varepsilon_0\mu_0\frac{\partial E_1}{\partial t}
\end{equation}
\begin{equation}
 	\frac{\partial B_1}{\partial z} - \frac{\partial B_3}{\partial x} = \mu_0 J_2	+ \varepsilon_0\mu_0\frac{\partial E_2}{\partial t}
\end{equation}
\begin{equation}
	 - \frac{\partial B_1}{\partial y} + \frac{\partial B_2}{\partial x} = \mu_0 J_3 + \varepsilon_0\mu_0\frac{\partial E_3}{\partial t}.
\end{equation}

Diese drei Gleichungen lassen sich, sofern man wiederum eine kartesische Raumstruktur wählt, in die klassische Vektorform 
\begin{equation}
	\nabla \times \vec{B} = \mu_0 \vec{J} + \varepsilon_0 \mu_0 \frac{\partial \vec{E}}{\partial t}
\end{equation}
zusammenfassen.
Diese Gleichung in vektorieller Schreibweise ist bekannt unter dem Namen Ampère-Maxwell-Gleichung.
Sie beschreibt, dass eine elektrische Stromdichte und eine zeitliche Änderung der elektrischen Feldstärke zu einer Rotation der magnetischen Flussdichte führt.

%subsection?
Wir haben nun also gesehen, dass wir nur durch Anwenden der äusseren Ableitung und dem Hodge-Operator auf $F$ die beiden inhomogenen Maxwell-Gleichungen erhalten. Zusätzlich ist die Formulierung kovariant und behält somit seine Form bei Koordinatentransformationen.
\end{comment}