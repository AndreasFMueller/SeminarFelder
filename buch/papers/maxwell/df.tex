\section{Homogene Maxwell-Gleichungen}
\kopfrechts{Homogene Maxwell-Gleichungen}
Die homogenen Maxwellgleichungen bekommen wir durch die äussere Ableitung des Faraday-Tensors 
\[
F_{\mu\nu}
= 
\begin{pmatrix}
	0 & -E_1/c & -E_2/c & -E_3/c \\
	E_1/c &  0 &  B_3 & -B_2 \\
	E_2/c & -B_3 &  0 &  B_1 \\
	E_3/c &  B_2 & -B_1 &  0 \\
\end{pmatrix} ,
\]
den wir in Kapitel \ref{maxwell:faraday} definiert haben.
Da $F$ eine geschlossene 2-Form ist, gilt
\[
dF = 0 .
\]
Für die Berechnung der äusseren Ableitung muss der Tensor zuerst als Differentialform geschrieben werden.
Es resultiert die Zweiform
\[
F
= \frac{1}{2} F_{\mu\nu} \, dx^\mu \wedge dx^\nu,
\]
man beachte dabei die einsteinsche Summationskonvention \ref{buch:koordinaten:tangentialvektoren:def:einsteinschesummenkonvention}.
Da der Feldstärketensor antisymmetrisch $F_{\mu\nu} = -F_{\nu\mu}$ ist, benötigt man den Faktor $\frac{1}{2}$, um die doppelte Zählung der Komponenten zu vermeiden:
\begin{align*}
	F
	= \,
	&\frac{1}{2} F_{\mu\nu} \, dx^\mu \wedge dx^\nu
	\\
	= \,
	&\frac{1}{2} F_{00} \, dx^0 \wedge dx^0 + \frac{1}{2} F_{01} \, dx^0 \wedge dx^1 + \frac{1}{2} F_{02} \, dx^0 \wedge dx^2 + \frac{1}{2} F_{03} \, dx^0 \wedge dx^3
	\\
	+ \, &\frac{1}{2} F_{10} \, dx^1 \wedge dx^0 + \frac{1}{2} F_{11} \, dx^1 \wedge dx^1 + \frac{1}{2} F_{12} \, dx^1 \wedge dx^2 + \frac{1}{2} F_{13} \, dx^1 \wedge dx^3
	\\
	+ \, &\frac{1}{2} F_{20} \, dx^2 \wedge dx^0 + \frac{1}{2} F_{21} \, dx^2 \wedge dx^1 + \frac{1}{2} F_{22} \, dx^2 \wedge dx^2 + \frac{1}{2} F_{23} \, dx^2 \wedge dx^3
	\\
	+ \, &\frac{1}{2} F_{30} \, dx^3 \wedge dx^0 + \frac{1}{2} F_{31} \, dx^3 \wedge dx^1 + \frac{1}{2} F_{32} \, dx^3 \wedge dx^2 + \frac{1}{2} F_{33} \, dx^3 \wedge dx^3 .
\end{align*} 
%-------------------------------------------ab da mit dx^îjk arbeiten !!-------------------------------------------------------------------
Mit der Definition \ref{maxwell:hodge:kurzschreibweise} gilt $dx^{ik} = 0$ wenn $i=k$ und $dx^{ik} = -dx^{ki}$ wenn $i \neq k $ . Somit folgt nach aufsteigenden Indizes sortiert
\begin{align*}
	F
	=
	&\phantom{+ \frac{1}{2} F_{00} \, dx^{00}} + \frac{1}{2} F_{01} \, dx^{01} + \frac{1}{2} F_{02} \, dx^{02} + \frac{1}{2} F_{03} \, dx^{03}
	\\
	&- \frac{1}{2} F_{10} \, dx^{01} \phantom{+ \frac{1}{2} F_{11} \, dx^{11}} + \frac{1}{2} F_{12} \, dx^{12} + \frac{1}{2} F_{13} \, dx^{13}
	\\
	&- \frac{1}{2} F_{20} \, dx^{02} - \frac{1}{2} F_{21} \, dx^{12} \phantom{+ \frac{1}{2} F_{22} \, dx^{22}} + \frac{1}{2} F_{23} \, dx^{23}
	\\
	&- \frac{1}{2} F_{30} \, dx^{03} - \frac{1}{2} F_{31} \, dx^{13} - \frac{1}{2} F_{32} \, dx^{23} \phantom{+ \frac{1}{2} F_{33} \, dx^{33}}
\end{align*}
und wegen $F_{\mu\nu} = -F_{\nu\mu}$
\begin{align*}
	F
	=
	&\phantom{+ \frac{1}{2} F_{00} \, dx^{00}} + \frac{1}{2} F_{01} \, dx^{01} + \frac{1}{2} F_{02} \, dx^{02} + \frac{1}{2} F_{03} \, dx^{03}
	\\
	&+ \frac{1}{2} F_{01} \, dx^{01} \phantom{+ \frac{1}{2} F_{11} \, dx^{11}} + \frac{1}{2} F_{12} \, dx^{12} + \frac{1}{2} F_{13} \, dx^{13}
	\\
	&+ \frac{1}{2} F_{02} \, dx^{02} + \frac{1}{2} F_{12} \, dx^{12} \phantom{+ \frac{1}{2} F_{22} \, dx^{22}} + \frac{1}{2} F_{23} \, dx^{23}
	\\
	&+ \frac{1}{2} F_{03} \, dx^{03} + \frac{1}{2} F_{13} \, dx^{13} + \frac{1}{2} F_{21} \, dx^{23} \phantom{+ \frac{1}{2} F_{33} \, dx^{33}}
\end{align*}
folgt anschliessend
\begin{align*}
	F
	=
	\phantom{+} F_{01} \, dx^{01} + F_{02} \, dx^{02} &+ F_{03} \, dx^{03}
	\\
	\phantom{+ F_{11} \, dx^{11}} + F_{12} \, dx^{12} &+ F_{13} \, dx^{13}
	\\
	\phantom{+ F_{12} \, dx^{12}} \phantom{+ F_{22} \, dx^{22}} &+ F_{23} \, dx^{23} \, .
\end{align*}
Zum besseren Verständnis setzen wir nun konkret die Komponenten von $F_{\mu\nu}$ ein und erhalten
\begin{align*}
	F
	=
	- \frac{1}{c} E_1 \, dx^{01} - \frac{1}{c} E_2 \, dx^{02} &- \frac{1}{c} E_3 \, dx^{03}
	\\
	\phantom{+ \frac{1}{c} F_{11} \, dx^{11}} + \phantom{\frac{1}{c}} B_3 \, dx^{12} &- \phantom{\frac{1}{c}} B_2 \, dx^{13}
	\\
	\phantom{+ \frac{1}{c} F_{12} \, dx^{12}} \phantom{+ \frac{1}{c} F_{22} \, dx^{22}} &+ \phantom{\frac{1}{c}} B_1 \, dx^{23} \, ,
\end{align*}
was schön den Komponenten oberhalb der Diagonalen von $F_{\mu\nu}$ entspricht.
Nun wenden wir die äussere Ableitung auf die 2-Form $F$ an, man beachte dabei Definition \ref{maxwell:hodge:kurzschreibweise}
 \begin{align*}	
 	dF =
 	& -  \frac{1}{c} \frac{\d E_1}{\d x^0} \, dx^{001} - \frac{1}{c} \frac{\d E_1}{\d x^1} \, dx^{101}
 	- \frac{1}{c} \frac{\d E_1}{\d x^2} \, dx^{201} - \frac{1}{c} \frac{\d E_1}{\d x^3} \, dx^{301}
 	\\
 	&- \frac{1}{c} \frac{\d E_2}{\d x^0} \, dx^{002} - \frac{1}{c} \frac{\d E_2}{\d x^1} \, dx^{102}
 	 - \frac{1}{c} \frac{\d E_2}{\d x^2} \, dx^{202} - \frac{1}{c} \frac{\d E_2}{\d x^3} \, dx^{302}
 	\\
 	& - \frac{1}{c} \frac{\d E_3}{\d x^0} \, dx^{003} - \frac{1}{c} \frac{\d E_3}{\d x^1} \, dx^{103}
 	- \frac{1}{c} \frac{\d E_3}{\d x^2} \, dx^{203} - \frac{1}{c} \frac{\d E_3}{\d x^3} \, dx^{303}
 	\\
 	&+ \phantom{\frac{1}{c}}\frac{\d B_3}{\d x^0} \, dx^{012} + \phantom{\frac{1}{c}}\frac{\d B_3}{\d x^1} \, dx^{112}
 	+ \phantom{\frac{1}{c}}\frac{\d B_3}{\d x^2} \, dx^{212} + \phantom{\frac{1}{c}}\frac{\d B_3}{\d x^3} \, dx^{312}
 	\\
 	&- \phantom{\frac{1}{c}}\frac{\d B_2}{\d x^0} \, dx^{013} - \phantom{\frac{1}{c}}\frac{\d B_2}{\d x^1} \, dx^{113}
 	- \phantom{\frac{1}{c}}\frac{\d B_2}{\d x^2} \, dx^{213} - \phantom{\frac{1}{c}}\frac{\d B_2}{\d x^3} \, dx^{313}
 	\\
 	&+ \phantom{\frac{1}{c}}\frac{\d B_1}{\d x^0} \, dx^{023} + \phantom{\frac{1}{c}}\frac{\d B_1}{\d x^1} \, dx^{123}
 	+ \phantom{\frac{1}{c}}\frac{\d B_1}{\d x^2} \, dx^{223} + \phantom{\frac{1}{c}}\frac{\d B_1}{\d x^3} \, dx^{323} .
 	\\
	\intertext{Bei doppelten Indizes der Wedgeprodukte sind diese $=0$:}
	dF
 	=
 	&- \frac{1}{c} \frac{\d E_1}{\d x^2} \, dx^{201} - \frac{1}{c} \frac{\d E_1}{\d x^3} \, dx^{301}
 	 - \frac{1}{c} \frac{\d E_2}{\d x^1} \, dx^{102} - \frac{1}{c} \frac{\d E_2}{\d x^3} \, dx^{302}
 	\\
 	&- \frac{1}{c} \frac{\d E_3}{\d x^1} \, dx^{103} - \frac{1}{c} \frac{\d E_3}{\d x^2} \, dx^{203}
 	 + \phantom{\frac{1}{c}} \frac{\d B_3}{\d x^0} \, dx^{012} + \phantom{\frac{1}{c}} \frac{\d B_3}{\d x^3} \, dx^{312}
 	\\
 	&- \phantom{\frac{1}{c}} \frac{\d B_2}{\d x^0} \, dx^{013} - \phantom{\frac{1}{c}} \frac{\d B_2}{\d x^2} \, dx^{213}
 	 + \phantom{\frac{1}{c}} \frac{\d B_1}{\d x^0} \, dx^{023} + \phantom{\frac{1}{c}} \frac{\d B_1}{\d x^1} \, dx^{123} .  
	\\
	\intertext{Weiter sortieren wir nach aufsteigenden Indizes}
	dF =
	&- \frac{1}{c} \frac{\d E_1}{\d x^2} \, dx^{012} - \frac{1}{c} \frac{\d E_1}{\d x^3} \, dx^{013}
	 + \frac{1}{c} \frac{\d E_2}{\d x^1} \, dx^{012} - \frac{1}{c} \frac{\d E_2}{\d x^3} \, dx^{023}
	\\
	&+ \frac{1}{c} \frac{\d E_3}{\d x^1} \, dx^{013} + \frac{1}{c} \frac{\d E_3}{\d x^2} \, dx^{023}
	 + \phantom{\frac{1}{c}} \frac{\d B_3}{\d x^0} \, dx^{012} + \phantom{\frac{1}{c}} \frac{\d B_3}{\d x^3} \, dx^{123}
	\\
	&- \phantom{\frac{1}{c}} \frac{\d B_2}{\d x^0} \, dx^{013} + \phantom{\frac{1}{c}} \frac{\d B_2}{\d x^2} \, dx^{123}
	 + \phantom{\frac{1}{c}} \frac{\d B_1}{\d x^0} \, dx^{023} + \phantom{\frac{1}{c}} \frac{\d B_1}{\d x^1} \, dx^{123}
\end{align*}
und fassen anschliessend alles zusammen
\begin{align*}
	dF =
	&\left( - \frac{1}{c} \frac{\d E_1}{\d x^2} + \frac{1}{c} \frac{\d E_2}{\d x^1} + \phantom{\frac{1}{c}} \frac{\d B_3}{\d x^0} \right) dx^{012}
	\\
	+&\left( - \frac{1}{c} \frac{\d E_1}{\d x^3} + \frac{1}{c} \frac{\d E_3}{\d x^1} - \phantom{\frac{1}{c}} \frac{\d B_2}{\d x^0} \right) dx^{013}
	\\
	+&\left( - \frac{1}{c} \frac{\d E_2}{\d x^3} + \frac{1}{c} \frac{\d E_3}{\d x^2} + \phantom{\frac{1}{c}} \frac{\d B_1}{\d x^0} \right) dx^{023}
	\\
	+&\left( \phantom{+} \phantom{\frac{1}{c}} \frac{\d B_3}{\d x^3} + \phantom{\frac{1}{c}} \frac{\d B_2}{\d x^2} + \phantom{\frac{1}{c}} \frac{\d B_1}{\d x^1} \right) dx^{123} \, .
	\\
	\intertext{Wechseln wir nun zu den Raum-Zeit Koordinaten gemäss \eqref{maxwell:koordinaten}, erhalten wir}
	dF =
	&\left( - \frac{1}{c} \frac{\d E_1}{\d y} + \frac{1}{c} \frac{\d E_2}{\d x} + \frac{1}{c} \frac{\d B_3}{\d t} \right) c \, dt \wedge dx \wedge dy
	\\
	+&\left( - \frac{1}{c} \frac{\d E_1}{\d z} + \frac{1}{c} \frac{\d E_3}{\d x} - \frac{1}{c} \frac{\d B_2}{\d t} \right) c \, dt \wedge dx \wedge dz
	\\
	+&\left( - \frac{1}{c} \frac{\d E_2}{\d z} + \frac{1}{c} \frac{\d E_3}{\d y} + \frac{1}{c} \frac{\d B_1}{\d t} \right) c\, dt \wedge dy \wedge dz
	\\
	+&\left( \phantom{+} \phantom{\frac{1}{c}} \frac{\d B_3}{\d z} + \phantom{\frac{1}{c}} \frac{\d B_2}{\d y} + \phantom{\frac{1}{c}} \frac{\d B_1}{\d x} \right) \phantom{c} dx \wedge dy \wedge dz
	\\
	\intertext{und der Faktor $c$ kann überall gekürzt werden:}
	dF =
	&\left( -  \frac{\d E_1}{\d y} +  \frac{\d E_2}{\d x} +  \frac{\d B_3}{\d t} \right) dt \wedge dx \wedge dy
	\\
	+&\left( -  \frac{\d E_1}{\d z} +  \frac{\d E_3}{\d x} -  \frac{\d B_2}{\d t} \right)dt \wedge dx \wedge dz
	\\
	+&\left( -  \frac{\d E_2}{\d z} +  \frac{\d E_3}{\d y} +  \frac{\d B_1}{\d t} \right)dt \wedge dy \wedge dz
	\\
	+&\left( \phantom{+}  \frac{\d B_3}{\d z} +  \frac{\d B_2}{\d y} +  \frac{\d B_1}{\d x} \right) dx \wedge dy \wedge dz \, .
	\\
	\intertext{Ändern wir nun in der zweiten Zeile die Reihenfolge der Wedgeprodukte, können wir folgendes erkennen:}
	dF =
	&\Bigg( \underbrace{ - \frac{\d E_x}{\d y} + \frac{\d E_y}{\d x}}_{(\nabla \times E)_z} + \frac{\d B_z}{\d t} \Bigg) \, dt \wedge dx \wedge dy
	\\
	+&\Bigg( \underbrace{ \phantom{+} \frac{\d E_x}{\d z} - \frac{\d E_z}{\d x}}_{(\nabla \times E)_y} + \frac{\d B_y}{\d t} \Bigg) \, dt \wedge dz \wedge dx
	\\
	+&\Bigg( \underbrace{ - \frac{\d E_y}{\d z} + \frac{\d E_z}{\d y}}_{(\nabla \times E)_x} + \frac{\d B_x}{\d t} \Bigg) \, dt \wedge dy \wedge dz
	\\
	+&\Bigg( \underbrace{ \phantom{+} \frac{\d B_z}{\d z} + \frac{\d B_y}{\d y} + \frac{\d B_x}{\d x}}_{\nabla \cdot \vec{B}} \Bigg) \, dx \wedge dy \wedge dz  \, \,
	\\
	\phantom{=} =&0 \, .
\end{align*}
Nun haben wir vier unabhängige Gleichungen erhalten, welche allgemein kovariant sind und in der Sprache der Vektoranalysis in zwei Gleichungen zusammengefasst werden können. 
In diesem Fall entsprechen sie genau den klassischen homogenen Maxwell-Gleichungen
\[
\nabla \times \vec{E} + \frac{\d \vec{B}}{\d t} = 0
\]
und
\[
\nabla \cdot \vec{B} = 0 \, .
\]
