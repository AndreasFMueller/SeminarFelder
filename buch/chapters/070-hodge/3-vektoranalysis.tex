%
% 3-vektoranalysis.tex -- Operatoren der Vektoranalysis
%
% (c) 2025 Prof Dr Andreas Müller
%
\section{Die Operatoren der Vektoranalysis
\label{buch:hodge:section:vektoranlysis}}
\kopfrechts{Die Operatoren der Vektoranalysis}
Die klassische Vektoranalysis definiert dagegen mehr oder weniger
ad hoc eine Reihe von Operatoren auf Vektorfeldern, meistens nur
in kartesischen Koordinaten.
Feldgleichungen für Strömungsfelder oder für das elektromagnetische
Feld können dann mit diesen Operatoren formuliert werden, aber auch
diese nur in kartesischen Koordinaten.
Es entsteht die etwas unbefriedigende Situation, dass man nicht
sicher sein kann, dass die konstruierten Gleichungen Aussagen
darstellen, die nicht Artefakte der Wahl des Koordinatensystems
sind.

Die äussere Ableitung von $p$-Formen wurde als allgemeine Theorie
der offensichtlich koordinatenunabhängig definierbaren
Differentialoperatoren aufgebaut.
Wenn es gelingt, die klassischen Operatoren der Vektoranalysis
durch Operatoren auszudrücken, die koordinatenunabhängig formuliert
werden können, dann weiss man auch, dass die Gleichungen, die man
aus diesen Operatoren konstruiert, unabhängig sind von der Wahl
eines Koordinatensystems.
Gleichzeitig ermöglicht der allgemeine Formalismus, diese Operatoren
für beliebige Koordinatensysteme darzustellen.

Die klassiche Vektoranalysis arbeitet nur in Dimension 3, wir gehen
daher in den nachfolgenden Ausführungen normalerweise von einem
dreidimensionalen Raum aus.
Einige der Konstruktionen funktionieren allerdings auch in $n$.

%
% Gradient
%
\subsection{Gradient}
Die Richtungsableitung einer Funktion $f\colon M\to\mathbb{R}$ in 
\index{Richtungsableitung}%
Richtung des Tangentialvektors $X$ im Punkt $p\in M$ ist gegeben
durch das Differential
\[
X\cdot f(p) = \langle df, X\rangle.
\]
In einer Karte ist
\[
\langle df,X\rangle
=
\frac{\partial f}{\partial x^k}(p)
\cdot X^k
\]
Da dies wie ein Skalarprodukt aussieht, werden in der klassischen
Vektoranalysis die Koordinaten des Differentials mit den Komponenten
eines Vektors identifiziert.
Wir verwenden daher in den nachfolgenden jeweils die Identifikation
\[
V\colon
a_i\,dx^i \mapsto \begin{pmatrix} a_1\\a_2\\a_3\end{pmatrix}.
\]
In dieser Identifikation ist
\[
V(df)
=
V\biggl(
\frac{\partial f}{\partial x^1}\,dx^1
+
\frac{\partial f}{\partial x^2}\,dx^2
+
\frac{\partial f}{\partial x^3}\,dx^3
\biggr)
=
\renewcommand{\arraystretch}{1.8}
\begin{pmatrix}
\displaystyle\frac{\partial f}{\partial x^1}\\
\displaystyle\frac{\partial f}{\partial x^2}\\
\displaystyle\frac{\partial f}{\partial x^3}
\end{pmatrix}
\]
Die Identifikation von Vektoren mit 1-Formen ist nicht nur in 
3 Dimensionen möglich sondern auch für Formen auf $\mathbb{R}^n$.
Der {\em Gradient} als Differentialoperator ist also auch auf $\mathbb{R}^n$
\index{Gradient}%
\index{grad}%
definiert.

%
% Rotation
%
\subsection{Rotation}
Wir betrachten jetzt die äussere Ableitung der 1-Form
$\omega=f_i(x)\,dx^i$.
Sie ist definiert durch
\[
d\omega
=
\frac{\partial f_i}{\partial x^k}\,dx^k\wedge dx^i
\in
\Omega^3(\mathbb{R}^3).
\]
In drei Dimensionen ausgeschrieben ist dies
\begin{align*}
d\omega
&=\phantom{+}
\frac{\partial f_1}{\partial x^1}\,dx^1\wedge dx^1
+
\frac{\partial f_1}{\partial x^2}\,dx^2\wedge dx^1
+
\frac{\partial f_1}{\partial x^3}\,dx^3\wedge dx^1
\\
&
\phantom{=}+
\frac{\partial f_2}{\partial x^1}\,dx^1\wedge dx^2
+
\frac{\partial f_2}{\partial x^2}\,dx^2\wedge dx^2
+
\frac{\partial f_2}{\partial x^3}\,dx^3\wedge dx^2
\\
&
\phantom{=}+
\frac{\partial f_3}{\partial x^1}\,dx^1\wedge dx^3
+
\frac{\partial f_3}{\partial x^2}\,dx^2\wedge dx^3
+
\frac{\partial f_3}{\partial x^3}\,dx^3\wedge dx^3.
\intertext{Die 2-Formen $dx^i\wedge dx^i$ verschwinden alle und die
die 2-Formen mit verschiedenen Faktoren sind zum Teil noch nicht in
der Standardreihenfolge aufsteigender Indizes, wodurch zusätzlich
Vorzeichen auftreten:}
&=-
\frac{\partial f_1}{\partial x^2}\,dx^1\wedge dx^2
-
\frac{\partial f_1}{\partial x^3}\,dx^1\wedge dx^3
\\
&
\phantom{=}+
\frac{\partial f_2}{\partial x^1}\,dx^1\wedge dx^2
-
\frac{\partial f_2}{\partial x^3}\,dx^2\wedge dx^3
\\
&
\phantom{=}+
\frac{\partial f_3}{\partial x^1}\,dx^1\wedge dx^3
+
\frac{\partial f_3}{\partial x^2}\,dx^2\wedge dx^3.
\intertext{Durch Zusammenfassen gleicher Basis-2-Formen erhält man}
&=
\biggl(
\frac{\partial f_3}{\partial x^2}
-
\frac{\partial f_2}{\partial x^3}
\biggr)\,dx^2\wedge dx^3
+
\biggl(
\frac{\partial f_3}{\partial x^1}
-
\frac{\partial f_1}{\partial x^3}
\biggr)\,dx^1\wedge dx^3
+
\biggl(
\frac{\partial f_2}{\partial x^1}
-
\frac{\partial f_1}{\partial x^2}
\biggr)\,dx^1\wedge dx^2.
\end{align*}

$d\omega$ ist eine 2-Form ist, die nicht direkt mit einem Vektor identifiziert
werden kann.
Durch Anwendung des Hodge-Operators kann die 2-Form $d\omega$ in eine 1-Form
verwendelt werden, ohne dass dabei Information verloren geht.
Der Hodge-Operator auf $\mathbb{R}^3$ wurde in
Beispiel~\ref{buch:hodge:hodge:beispiel:r3} berechnet.
Angewendet auf $d\omega$ ergibt sich
\begin{align*}
\ast d\omega
&=
\biggl(
\frac{\partial f_3}{\partial x^2}
-
\frac{\partial f_2}{\partial x^3}
\biggr)\, dx^1
+
\biggl(
\frac{\partial f_1}{\partial x^3}
-
\frac{\partial f_3}{\partial x^1}
\biggr)\,dx^3
+
\biggl(
\frac{\partial f_2}{\partial x^1}
-
\frac{\partial f_1}{\partial x^2}
\biggr)\,dx^3.
\end{align*}
Man beachte den Vorzeichenwechsel beim mittleren Teil.
Diese 1-Form wird durch die Abbildung $V$ mit dem Vektor
\[
V(\ast d\omega)
=
\bgroup
\renewcommand\arraystretch{2.0}
\begin{pmatrix}
\displaystyle
\frac{\partial f_3}{\partial x^2}
-
\frac{\partial f_2}{\partial x^3}
\\
\displaystyle
\frac{\partial f_1}{\partial x^3}
-
\frac{\partial f_3}{\partial x^1}
\\
\displaystyle
\frac{\partial f_2}{\partial x^1}
-
\frac{\partial f_1}{\partial x^2}
\end{pmatrix}
\egroup
=
\operatorname{rot}
\begin{pmatrix}
f_1\\
f_2\\
f_3
\end{pmatrix}
=
\operatorname{rot}\vec{f}
\]
identifiziert.
Dies ist die {\em Rotation} des Vektorfeldes $\vec{f}$.
\index{Rotation}%
\index{rot}%
In der englischsprachigen Literatur wird die Rotation auch als
$\operatorname{curl}\vec{f}$ bezeichnet.
\index{curl}%
Sie kann formal auch als das Vektorprodukt mit dem Nabla-Operator
\[
\nabla\times\vec{f}
=
\operatorname{rot}\vec{f}
\]
geschrieben werden.

%
% Divergenz
%
\subsection{Divergenz
\label{buch:hodge:skalarprodukt:subsection:divergenz}}
Die Rotation wurde durch äussere Ableitung einer 1-Form erhalten.
Es war nötig, die Ableitung durch Anwendung des Hodge-Operators in eine
1-Form zu verwandeln, die dann wieder mit einem Vektor identifiziert
werden konnte.

Um die Ableitung einer 2-Form mit einer Vektoranalysis-Operation
zu identifizieren, muss erst eine 2-Form gewonnen werden.
Ein Vektor kann mit einer 1-Form identifiziert werden, die durch
den Hodge-Operator in eine 2-Form umgewandelt werden kann.
Wir beginnen daher mit dem Vektor $\vec{f}=V(f_i\,dx^i)=V(\omega)$
und wenden den Hodge-Operator darauf an:
\begin{align*}
\ast(f_i\,dx^i)
&=
f_1\, dx^2\wedge dx^3
-
f_2\, dx^1\wedge dx^3
+
f_3\, dx^1\wedge dx^2
\end{align*}
Bei der Berechnung der äusseren Ableitung kommen zu den Basis-2-Formen
weitere Basis-1-Formen hinzu, jedoch gibt es in jedem Term immer nur eine
einzige Basis-1-Form, die das Wedge-Produkt nicht zu 0 macht.
Daher ist die äusser Ableitung
\begin{align*}
d{\ast\omega}
&=
\frac{\partial f_1}{\partial x^1}dx^1\wedge dx^2\wedge dx^3
-
\frac{\partial f_2}{\partial x^2}dx^2\wedge dx^1\wedge dx^3
+
\frac{\partial f_3}{\partial x^3}dx^3\wedge dx^1\wedge dx^2
\intertext{Die 3-Formen auf der rechten Seite müssen in die
Standardreihenfolge gebracht werden und nehmen bei einer ungeraden
Anzahl Vertauschungen ein negatives Vorzeichen auf:}
&=
\biggl(
\frac{\partial f_1}{\partial x_1}
+
\frac{\partial f_2}{\partial x_2}
+
\frac{\partial f_3}{\partial x_3}
\biggr)
\,
dx^1\wedge dx^2\wedge dx^3.
\intertext{Dies ist eine 3-Form, in der klassischen Vektoranalysis gibt
es aber nur Funktionen und Vektoren.
Daher wird jetzt nochmals der Hodge-Operator angewendet, um die 3-Form
in eine Funktion zu verwandeln.
So entsteht die Gleichung}
\ast{d{\ast\omega}}
&=
\frac{\partial f_1}{\partial x_1}
+
\frac{\partial f_2}{\partial x_2}
+
\frac{\partial f_3}{\partial x_3}.
\end{align*}
Diese Funktion ist auch als die {\em Divergenz}
\index{Divergenz}%
\index{div}%
\begin{equation}
\operatorname{div}\vec{f}
=
\sum_{i=1}^n \frac{\partial f_i}{\partial x^i}
\label{buch:hodge:hodge:divergenz:eqn:divdef}
\end{equation}
bekannt.
Formal kann sie auch als Skalarprodukt
\[
\operatorname{div}\vec{f}
=
\nabla\cdot\vec{f}
\]
mit dem Nabla-Operator geschrieben werden.

Die Divergenz eines Vektorfeldes ist also die Funktion, die durch
Anwendung des Operators $*{d}*$ auf die zugehörige 1-Form entsteht.
Dieser Operator funktioniert aber auch auf 1-Formen in $n$ Dimensionen.
Der Hodge-Operator macht aus einer 1-Form auf $\mathbb{R}^n$ eine
$n-1$-Form.
Die äussere Ableitung davon ist eine $n$-Form, die der Hodge-Operator
wieder zu einer Funktion macht.
Auch die Formel \eqref{buch:hodge:hodge:divergenz:eqn:divdef} gilt
für bliebige $n$.
%
% table-operatoren.tex
%
% (c) 2025 Prof Dr Andreas Müller
%
\begin{table}
\centering
\begin{tabular}{|>{$}c<{$}|>{$}c<{$}|>{$}c<{$}|}
\hline
\text{Vektoranalsis} & \text{$p$-Formen} & \text{Nabla}
\\
\hline
\operatorname{grad} &  d  & \nabla 
\\
\operatorname{rot}  & *d  & \nabla\times\mathstrut
\\
\operatorname{div}  & *d* & \nabla\cdot\mathstrut
\\
\hline
\end{tabular}
\caption{Korrespondenz wischen Operatoren der klassischen Vektoranalysis,
den Kombinationen von Hodge-Operator und äusserer Ableitung und der
Schreibweise der Operatoren der Vektoranalysis mit dem Nabla-Operator.
\label{buch:hodge:hodge:table:operatoren}}
\end{table}
%

%
% Operatorrelationen
%
\subsection{Operatorrelationen
\label{buch:vektoranalysis:subsetion:operatorrelationen}}
Die äussere Ableitung ist nilpotent vom Grad 2, d.~h.~die iterierte
äussere Ableitung $d^2=0$ verschwindet.
Zusammen mit den Identifikationen mit Vektorfeldern ergeben sich
daraus Kombinationen von Operatoren der Vektoranalysis, die
0 ergeben.

%
% Rotation eines Gradienten
%
\subsubsection{Rotation eines Gradienten}
Der Gradient einer Funktion $f$ ist
\[
\operatorname{rot}\operatorname{grad}f
=
*{d}df
=
*d^2f
=
0
\]
\index{rot grad}%
verschwindet, weil der äussere Differentialoperator nilpotent mit
Ordnung $2$ ist.

%
% Divergenz einer Rotation
%
\subsubsection{Divergenz einer Rotation}
Die Divergenz der Rotation eines Vektorfeldes ist
\[
\operatorname{div}\operatorname{rot} \vec{f}
=
*{d}{*}{*}d V^{-1}(\vec{f})
=
*d^2 V^{-1}(\vec{f})
=
0,
\]
\index{div rot}%
wobei beim zweiten Gleichheitszeichen die Involutionseigenschaft
des Hodge-Operators und beim dritten die Nilpotenz der äusseren Ableitung
verwendet wird.

%
% Rotation einer Rotation
%
\subsubsection{Rotation einer Rotation\label{buch:vektoranalysis:subsection:rotrot}}
Die Verkettungen $\operatorname{rot}\operatorname{grad}$ und
$\operatorname{div}\operatorname{rot}$ war einfach auszurechnen,
weil in der Darstellung mit Hilfe der äusseren Ableitung und des
Hodge-Operators jeweils zwei äussere Ableitungen verkettet werden,
die zusammen die Nullabbildung ergeben.
Die Verketten $\operatorname{rot}\operatorname{rot}$ wird in der
Darstellung durch $p$-Formen zu
\index{rot rot}%
\[
\operatorname{rot}\operatorname{rot}
=
*{d}{*}d,
\]
die im Allgemeinen nicht verschwindet.

In der Tat können wir sie in Vektorform sofort ausrechnen.
Anwendung der früher hergeleiteten Koordinaten-Formeln liefern
\bgroup
\renewcommand{\arraystretch}{1.9}
\begin{align}
\operatorname{rot}\operatorname{rot}\vec{f}
&=
\begin{pmatrix}
\displaystyle
\frac{\partial}{\partial x^1}\\
\displaystyle
\frac{\partial}{\partial x^2}\\
\displaystyle
\frac{\partial}{\partial x^3}
\end{pmatrix}
\begin{pmatrix}
\displaystyle
\frac{\partial f_3}{\partial x^2} - \frac{\partial f_2}{\partial x^3}\\
\displaystyle
\frac{\partial f_1}{\partial x^3} - \frac{\partial f_3}{\partial x^1}\\
\displaystyle
\frac{\partial f_2}{\partial x^1} - \frac{\partial f_1}{\partial x^2}
\end{pmatrix}
\notag
\\
&=
\begin{pmatrix}
\displaystyle
\frac{\partial}{\partial x^2}
\biggl(
\frac{\partial f_2}{\partial x^1} - \frac{\partial f_1}{\partial x^2}
\biggr)
-
\frac{\partial}{\partial x^3}
\biggl(
\frac{\partial f_1}{\partial x^3} - \frac{\partial f_3}{\partial x^1}
\biggr)
\\
\displaystyle
\frac{\partial}{\partial x^3}
\biggl(
\frac{\partial f_3}{\partial x^2} - \frac{\partial f_2}{\partial x^3}
\biggr)
-
\frac{\partial}{\partial x^1}
\biggl(
\frac{\partial f_2}{\partial x^1} - \frac{\partial f_1}{\partial x^2}
\biggr)
\\
\displaystyle
\frac{\partial}{\partial x^1}
\biggl(
\frac{\partial f_1}{\partial x^3} - \frac{\partial f_3}{\partial x^1}
\biggr)
-
\frac{\partial}{\partial x^2}
\biggl(
\frac{\partial f_3}{\partial x^2} - \frac{\partial f_2}{\partial x^3}
\biggr)
\end{pmatrix}
\notag
\\
&=
\begin{pmatrix}
\displaystyle
\frac{\partial^2 f_2}{\partial x^1\,\partial x^2}
-
\frac{\partial}{\partial x^2}
\frac{\partial^2 f_1}{\partial (x^2)^2}
-
\frac{\partial^2 f_1}{\partial (x^3)^2}
+
\frac{\partial^2 f_3}{\partial x^1\,\partial x^3}
\\
\displaystyle
\frac{\partial^2 f_3}{\partial x^2\,\partial x^3}
-
\frac{\partial^2 f_2}{\partial (x^3)^2}
-
\frac{\partial^2 f_2}{\partial (x^1)^2}
+
\frac{\partial^2 f_1}{\partial x^2\,\partial x^1}
\\
\displaystyle
\frac{\partial^2 f_1}{\partial x^3\,\partial x^1}
-
\frac{\partial^2 f_3}{\partial (x^1)^2}
-
\frac{\partial^2 f_3}{\partial (x^2)^2}
+
\frac{\partial^2 f_2}{\partial x^3\,\partial x^2}
\end{pmatrix}.
\notag
\intertext{Nach Vertauschung der partiellen Ableitung finden wir auf
der ersten Zeile zwei Ableitungen nach $x^1$ zwei Termen, die Teil
der Divergenz von $\vec{f}$ sind.
Die verbleibenden Terme sind zweite Ableitungen, die einen Teil
des klassischen Laplace-Operators ergeben.
Um dieses Resultat mit bekannten Operatoren der Vektoranalysis
schreiben zu können, müssen Terme mit entgegengesetzten Vorzeichen
ergänzt werden.
Die neuen Term sind {\color{darkred}rot} eingefügt:}
&=
\begin{pmatrix}
\displaystyle
\frac{\partial}{\partial x^1} \biggl(
{\color{darkred}\frac{\partial f_1}{\partial x^1}
+}
\frac{\partial f_2}{\partial x^2}
+
\frac{\partial f_3}{\partial x^3}
\biggr)
-
\biggl(
{\color{darkred}\frac{\partial^2 f_1}{\partial (x^1)^2}
+}
\frac{\partial^2 f_1}{\partial (x^2)^2}
+
\frac{\partial^2 f_1}{\partial (x^3)^2}
\biggr)
\\
\displaystyle
\frac{\partial}{\partial x^2} \biggl(
\frac{\partial f_1}{\partial x^1}
{\color{darkred}
+
}
\frac{\partial f_2}{\partial x^2}
+
\frac{\partial f_3}{\partial x^3}
\biggr)
-
\biggl(
\frac{\partial^2 f_2}{\partial (x^1)^2}
{\color{darkred}
+
\frac{\partial^2 f_2}{\partial (x^2)^2}
}
+
\frac{\partial^2 f_2}{\partial (x^3)^2}
\biggr)
\\
\displaystyle
\frac{\partial}{\partial x^3} \biggl(
\frac{\partial f_1}{\partial x^1}
+
\frac{\partial f_2}{\partial x^2}
{\color{darkred}
+
\frac{\partial f_3}{\partial x^3}
}
\biggr)
-
\biggl(
\frac{\partial^2 f_3}{\partial (x^1)^2}
+
\frac{\partial^2 f_3}{\partial (x^2)^2}
{\color{darkred}
+
\frac{\partial^2 f_3}{\partial (x^3)^2}
}
\biggr)
\end{pmatrix}.
\notag
\intertext{Die Klammerausdrücke können jetzt mit den bekannten
Operatoren der Vektoranalysis geschrieben werden und ergeben}
\operatorname{rot}\operatorname{rot}\vec{f}
&=
\begin{pmatrix}
\displaystyle
\frac{\partial}{\partial x^1} \operatorname{div}\vec{f}
-
\Delta f_1
\\
\displaystyle
\frac{\partial}{\partial x^2} \operatorname{div}\vec{f}
-
\Delta f_2
\\
\displaystyle
\frac{\partial}{\partial x^3} \operatorname{div}\vec{f}
-
\Delta f_3
\end{pmatrix}
=
\operatorname{grad}\operatorname{div}\vec{f} - \Delta \vec{f}
=
\nabla(\nabla\cdot\vec{f}) - (\nabla\cdot\nabla)\vec{f}.
\label{buch:vektoranalysis:eqn:rotrot}
\end{align}
\egroup
Darin ist $\Delta$ der klassische Laplace-Operator
\index{Laplace-Operator}%
\[
\Delta
=
\nabla\cdot\nabla
=
\frac{\partial^2}{\partial (x^1)^2}
+
\frac{\partial^2}{\partial (x^2)^2}
+
\frac{\partial^2}{\partial (x^3)^2}.
\]
Die Identität~\eqref{buch:vektoranalysis:eqn:rotrot} wird in
Abschnitt~\ref{buch:hodge:section:kodifferential} als Motivation
für die Definition des Kodifferentials dienen.
In Abschnitt~\ref{buch:hodge:laplace:subsection:rotrot} wird sie
schliesslich verallgemeinert und es wird gezeigt, dass sie nichts
anderes ist als die Definition des Hodge-Laplace-Operators.

%
% Poincaré-Lemma in Vektoranalysis-Schreibweise
%
\subsubsection{Poincaré-Lemma in Vektoranalysis-Schreibweise}
Das Poincaré-Lemma besagt, dass geschlossene Differentialformen auf
$\mathbb{R}^n$ exakt ist.
Eine geschlossene 0-Form ist eine Funktion $f$, deren Differential
$df=0$ verschwindet.
In einer Karte verschwinden alle partiellen Ableitungen, die Funktion
ist konstant.

Für $n=3$ gibt es zwei interessante Fälle, nämlich geschlossene 1-Formen
und geschlossene 2-Formen.
Wir übersetzen die Aussagen des Poincaré-Lemmas für diese beiden
Fälle in die Schreibweise der Vektoranalsis.

Eine geschlossene 1-Form $\omega$ entspricht einem Vektorfeld.
Geschlossen bedeutet, dass die Rotation dieses Vektorfeldes verschwindet.
Nach dem Poincaré-Lemma gibt es eine 0-Form $f$ derart dass
\index{Poincare-Lemma@Poincaré-Lemma}%
$df=\omega$ ist.
Die Identifikation von 1-Formen mit Vektorfeldern besagt also, dass
ein Vektorfeld, dessen Rotation verschwindet, ein Gradientfeld
einer Funktion $f$ ist.

Eine geschlossene 2-Form $\omega$ entspricht einer 1-Form $\ast\omega$,
auf die mit Hilfe des Hodge-Operators angewendet worden ist.
$d\omega=0$ bedeutet, dass die Divergenz des zu $\ast\omega$
gehörigen Vektorfeldes verschwindet.
Da $d\omega=0$ ist, gibt es nach dem Poincaré-Lemma eine 1-Form $\alpha$,
deren äussere Ableitung $d\alpha=\omega$ ist.
Folglich ist 
\[
d\alpha = \omega
\qquad\Rightarrow\qquad
\ast d\alpha = \ast\omega.
\]
Der 1-Form $\beta$ entspricht ein Vektorfeld, und $\ast d\beta$ entspricht
der Rotation dieses Vektorfelds.
Es folgt, dass ein Vektorfeld, dessen Divergenz verschwindet, als
Rotation eines anderen Vektorfeldes geschrieben werden kann.

\begin{satz}[Poincaré-Lemma der Vektoranalsis]
Wenn die Rotation eines Vektorfeldes $\vec{v}$ auf $\mathbb{R}^3$
verschwindet, dann gibt es eine Funktion $f$ mit
$\vec{v}=\operatorname{grad}f$.
Wenn die Divergenz eines Vektorfeldes $\vec{v}$ verschwindet,
dann gibt es ein Vektorveld $\vec{a}$, dessen Rotation
$\operatorname{rot}\vec{a}=\vec{v}$ das Vektorfeld ist.
\end{satz}

