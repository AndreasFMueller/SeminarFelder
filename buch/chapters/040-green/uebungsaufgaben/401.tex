Berechnen Sie das Wegintegral der 1-Form
\[
\omega
=
y
\,dx
+
x^2
\,dy
\]
entlang eines Parabelbogens $\gamma_a$, der durch die Punkte
$(-1,0)$, $(1,0)$ und $(0,a)$ geht.
\begin{teilaufgaben}
\item Berechnen Sie die 2-Form $d\omega$.
\item Berechnen Sie das Integral der Zweiform über die Fläche zwischen
den Kurven $\gamma_{-1}$ und $\gamma_1$.
\item Verifizieren Sie den Satz von Green für die Situation in Teilaufgabe b).
\item Ist die 1-Form ein Potentialfeld?
\end{teilaufgaben}

\begin{figure}[h]
\centering
\begin{tikzpicture}[>=latex,thick]
\xdef\a{3}
\fill[color=darkred!20]
	plot[domain=-3:3,samples=100]
		({\x},{-\a*(3+\x)*(3-\x)/9})
	--
	plot[domain=-3:3,samples=100]
		({-\x},{\a*(3+\x)*(3-\x)/9})
	-- cycle;
\draw[color=darkred,line width=1.2pt]
	plot[domain=-4:4,samples=100]
		({\x},{\a*(3+\x)*(3-\x)/9});
\node at (1,1.5) {$\Omega$};
\node at (-2,1.66) [left] {$\gamma_1$};
\draw[->] (-4.1,0) -- (4.4,0) coordinate[label={$x$}];
\draw[->] (0,-3.1) -- (0,3.6) coordinate[label={right:$y$}];
\draw (-3,-0.1) -- (-3,0.1);
\draw (3,-0.1) -- (3,0.1);
\draw (-0.1,3) -- (0.1,3);
\node at (0,3) [above left] {$1$};
\node at (-3,0) [below right] {$-1$};
\node at (3,0) [below left] {$1$};
\end{tikzpicture}
\caption{Weg für das Wegintegral in Aufgabe~\ref{401}
\label{buch:401:fig}}
\end{figure}

\begin{hinweis}
Zur Parametrisierung der Randkurven kann die Funktion
\[
\gamma_a
\colon
[-1,1] \to \mathbb{R}^2
:
t
\mapsto
(t, a(1-t)(1+t))
\]
verwendet werden.
Für die Berechnung das Kurvenintegrals muss über $\gamma_1$ und über
$\gamma_{-1}$ (in umgekehrter Richtung) integriert werden.
\end{hinweis}

\begin{loesung}
Der Parabelbogen wird durch die Funktion
\[
\gamma_a
\colon
[-1,1]\to\mathbb{R}^2
:
t\mapsto (t,a(1-x)(1+t))
\]
parametrisiert.
Für das Wegintegral wird der Tangentialvektor benötigt, der
\begin{equation}
\frac{d\gamma_a}{dt}(t)
=
(1,-2at)
\label{buch:401:ableitung}
\end{equation}
ist.
Das Wegintegral wird damit zu
\begin{align*}
\int_{\gamma_a} \omega
&=
\int_{\gamma_a} y\,dx + x^2\,dy
=
\int_{-1}^1
y(t)\dot{x}(t) + x(t)^2 \dot{y}(t)
\,dt
\intertext{Aus
\eqref{buch:401:ableitung}
folgt $\dot{x}(t)=1$ und $\dot{y}(t) = -2at$, womit das Integral
zu
}
&=
\int_{-1}^1 a(1-t)(1+t) + t^2 (-2at) \,dt
\\
&=
a\int_{-1}^1 1-t^2-2t^3\,dt
\\
&=
a
\biggl[
t-\frac{t^3}{3}-\frac{t^4}{2}
\biggr]_{-1}^1
\\
&=
a\biggl(
2-\frac23
\biggr)
=
\frac{4a}{3}
\end{align*}
wird.
\begin{teilaufgaben}
\item
Die äussere Ableitung von $\omega$ ist
\begin{align*}
d\omega
&=
\biggl(
\frac{\partial}{\partial x}x^2
-
\frac{\partial}{\partial y}y
\biggr)\,dx\wedge dy
\\
&=
(2x-1)
\,dx\wedge dy.
\end{align*}
\item
Das Flächenintegral ist
\begin{align*}
\int_{\Omega}
d\omega
&=
\int_{-1}^1
\int_{\gamma_{-1}(x)}^{\gamma_1(x)}
2x-1
\,dy
\,dx
\\
&=
\int_{-1}^1
\biggl[(2x-1)y\biggr]_{-1+x^2}^{1-x^2}
\,dx
\\
&=
\int_{-1}^1
(2x-1)
\bigl(
1-x^2-(-1+x^2)
\bigr)
\,dx
\\
&=
2
\int_{-1}^1
(2x-1)
(1-x^2)
\,dx
=
-\frac{8}{3}.
\end{align*}
\item
Das oben berechnete Wegintegral durchläuft die Kurve von links nach
rechts.
Für den Beitrag des oberen Randes zum Integral über den Rand von
$\Omega$ muss daher das Vorzeichen gekehrt werden.
Damit wir das Wegintegral
\begin{align*}
\int_{\partial\Omega} \omega
&=
\int_{\gamma_{-1}}\omega - \int_{\gamma_{1}}\omega
\\
&=
-\frac{4}{3}-\frac{4}{3}
=
-\frac{8}3,
\end{align*}
was mit dem Resultat von Teilaufgabe b) übereinstimmt.
\item
Wäre $\omega$ ein Potentialfeld, dann müsste das Wegintegral vom Parameter
$a$ unabhängig sein, was es aber ganz offensichtlich nicht ist.
Somit kann $\omega$ kein Potentialfeld sein.
\qedhere
\end{teilaufgaben}
\end{loesung}

