%
% 3_weiteres.tex -- Diskussion & weitere Anwendungsmöglichkeiten
%
% (c) 2025 Roman Cvijanovic & Nicola Dall'Acqua, Hochschule Rapperswil
%
% !TEX root = ../../buch.tex
% !TEX encoding = UTF-8
%

\section{Diskussion}\label{neuronal:section:diskussion}
\kopfrechts{Diskussion}

Der grosse Vorteil dieser Methode liegt in ihrer breiten Anwendbarkeit.
Sie kann grundsätzlich auf beliebige Feldgleichungen angewendet werden, unabhängig von deren Komplexität.
Im Gegensatz dazu sind klassische Verfahren wie die Finite-Differenzen- oder Finite-Elemente-Methode oft auf bestimmte Gleichungstypen beschränkt und stossen insbesondere bei stark nichtlinearen oder hochdimensionalen Problemen an ihre Grenzen.

Neuronale Netzwerke bieten durch das \emph{Universal Approximation Theorem} die theoretische Grundlage, um beliebige stetige Funktionen mit beliebiger Genauigkeit zu approximieren \cite{neuronal:universal_approximation_theorem}.
Dies befähigt sie theoretisch dazu, Lösungen für jede beliebige Feldgleichung zu approximieren.

In der Praxis zeigen neuronale Netzwerke vielversprechende Ergebnisse beim Lösen von Feldgleichungen \cite{neuronal:pinns}.
Jedoch befindet sich die Methode noch in einem frühen Entwicklungsstadium.
Es gibt viele offene Fragen, insbesondere hinsichtlich der Effizienz und Genauigkeit im Vergleich zu klassischen Verfahren.
Wie die Schwierigkeiten bei den Übergängen bei der Wellengleichung zeigen, kann die Methode anfällig für Fehler sein.
Zusätzliche Forschungsergebnisse dürften bezüglich Fehleranfälligkeit und offenen Fragen weitere Erkenntnisse liefern.
Eine abschließende Bewertung der Methode ist daher zum jetzigen Zeitpunkt noch nicht möglich.
