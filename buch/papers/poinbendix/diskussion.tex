\section{Diskussion} \label{poinbendix:section:diskussion}
\kopfrechts{Diskussion}
Die vorgängigen Abschnitte zum Satz von Poincaré-Bendixson haben gezeigt, dass Bahnkurven von zweidimensionalen dynamischen Systemen nur eingeschränkte Lösungen annehmen können.
Dies ist ein starker Kontrast zu drei- oder noch höher dimensionalen Systemen.
In diesen kann chaotisches Verhalten auftreten.
Dies bedeutet, dass kleine Änderungen in den Anfangsbedingungen zu komplett unterschiedlichen Bahnkurven führen können.
Dadurch werden Bahnkurven möglich, welche weder periodisch sind noch auf eine Nullstelle treffen.

Diese Einschränkung von Lösungen in zwei Dimensionen ermöglicht es, stabile Systeme zu entwerfen, wo kleine Ungenauigkeiten (z.B. in einem Fertigungsprozess oder externe Störungen) keinen starken Einfluss haben.
\index{stabil}%
Man verlässt zwar den Orbit, aber es wird entweder sofort ein ähnlicher Orbit gefunden oder das System konvergiert auf den vorgesehenen Orbit.
\index{Orbit}%

