\section{$dA = F$
	\label{maxwell:section:dA}}
\kopfrechts{dA}
Den Faraday-Tensor haben wir definiert als
\begin{equation}
	F = \begin{pmatrix}
		0 & -E_1/c & -E_2/c & -E_3/c \\ E_1/c & 0 & B_3 & -B_2 \\ E_2/c & -B_3 & 0 & B_1 \\ E_3/c & B_2 & -B_1 & 0 
	\end{pmatrix}.
	\label{maxwell:section:dA:Faraday-Tensor}
\end{equation}
Wir haben gesehen, dass die äussere Ableitung von $F$ verschwindet.
Aufgrund des Pointcaré-Lemmas wissen wir, dass es eine 1-Form geben muss, deren äussere Ableitung $F$ ergibt.
Eine solche Differentialform nennt man auch Potentialform.
Dieser Name lässt die Frage offen, ob man $F$ somit mit einer 1-Form, welche das magnetischen Vektorpotential und das elektrischen Potential beinhaltet, ausdrücken kann.
Zuerst ist unser Ziel, die elektrische Feldstärke und magnetische Flussdichte rein mit den beiden Potentialen auszudrücken und dann einen Übergang in die Differentialformen zu schaffen. 

\subsection{$B$ und $E$ aus Maxwell-Gleichungen herleiten}
Aus dem gaussschen Gesetz für Magnetfelder
\begin{equation*}
	\nabla \cdot \vec{B} = 0
\end{equation*}
und der Tatsache, dass jedes divergenzfreie, stetig differenzierbare Vektorfeld als Rotation eines anderen Vektorfelds lokal dargestellt werden kann,
folgt
\begin{equation}
	\label{maxwell:dA;rotA}
	\vec{B} = \nabla \times \vec{A}.
\end{equation}
Diese Beziehung steht in engem Zusammenhang mit dem Prinzip, dass die zweifache äussere Ableitung auf Differentialformen verschwindet. Hier beziehen wir uns aber auf Vektoren.

Das Induktionsgesetz lautet
\begin{equation*}
	\nabla \times \vec{E} = - \frac{\partial \vec{B}}{\partial t}.
\end{equation*}
Setzt man für $\vec{B}$ die obige Definition ein, folgt
\begin{equation*}
	\nabla \times \vec{E} = - \frac{\partial}{\partial t}(\nabla \times \vec{A}).
\end{equation*}
Die Rotation eines Vektorfelds ist linear und daraus folgt
\begin{equation*}
	\nabla \times \biggl( \vec{E} + \frac{\partial \vec{A}}{\partial t}\biggr) = 0.
\end{equation*}
Ist ein Vektorfeld rotationsfrei, so kann es als Gradient eines Skalarfelds geschrieben werden.
Daraus folgt
\begin{equation}
	\label{maxwell:dA:defE}
	\vec{E} + \frac{\partial \vec{A}}{\partial t} = -\nabla \varphi \quad \Rightarrow \quad \vec{E} = -\nabla \varphi -\frac{\partial \vec{A}}{\partial t}.
\end{equation}
Das Minuszeichen vor dem Gradienten des elektrischen Potentials $\varphi$ ergibt sich daraus, dass das elektrische Feld immer in Richtung des niedrigeren Potentials zeigt.

Jetzt haben wir zwei Formeln gefunden, mit welchen sich die elektrische Feldstärke und die magnetische Flussdichte aus dem elektrischen Potential und dem magnetischen Vektorpotential herleiten lassen.
\subsection{Potentialform von $F$ finden}
Aus den beiden hergeleiteten Gleichungen
\begin{align*}
	\vec{B} &= \nabla \times \vec{A},\\
	\vec{E} &= -\nabla \varphi -\frac{\partial \vec{A}}{\partial t},
\end{align*}
können wir bereits erkennen, dass die gesuchte 1-Form die drei Komponenten des magetischen Vektorpotentials und das elektrische Potential enthalten muss.
Zusätzlich müssen wir auf die Vorzeichen achten und den Faktor $c$, der in der Definition \eqref{maxwell:section:dA:Faraday-Tensor} von $F$ vorkommt, unterbringen.
Weil im Faraday-Tensor nur die $E$-Teile das $c$ beinhalten, bringen wir dieses beim elektrischen Potential $\varphi$ unter.
Aufgrund der $(-+++)$-Metrik wählen wir die Zeitkomponente negativ und die drei Raumkomponenten positiv.
Wir verwenden also die 1-Form 
\begin{equation}
	A = -\frac{\varphi}{c}\,dx^0 + A_1 \,dx^1 + A_2 \,dx^2 + A_3 \,dx^3.
\end{equation}
Um nun zu prüfen, was die äussere Ableitung von $A$ ergibt, rechnen wir
\begin{align*}
	dA = \phantom{+}
	 \frac{1}{c} &\frac{\partial (-\varphi)}{\partial x^1} \,dx^1 \wedge dx^0
	+ \frac{1}{c}\frac{\partial (-\varphi)}{\partial x^2} \,dx^2 \wedge dx^0
	+ \frac{1}{c}\frac{\partial (-\varphi)}{\partial x^3} \,dx^3 \wedge dx^0
	\\
	+ \phantom{\frac{1}{c}} &\frac{\partial A_1}{\partial x^0} \,dx^0 \wedge dx^1
	+ \frac{\partial A_1}{\partial x^2} \,dx^2 \wedge dx^1
	+ \frac{\partial A_1}{\partial x^3} \,dx^3 \wedge dx^1
	\\
	+ \phantom{\frac{1}{c}} &\frac{\partial A_2}{\partial x^0} \,dx^0 \wedge dx^2
	+ \frac{\partial A_2}{\partial x^1} \,dx^1 \wedge dx^2
	+ \frac{\partial A_2}{\partial x^3} \,dx^3 \wedge dx^2
	\\
	+ \phantom{\frac{1}{c}} &\frac{\partial A_3}{\partial x^0} \,dx^0 \wedge dx^3
	+ \frac{\partial A_3}{\partial x^1} \,dx^1 \wedge dx^3
	+ \frac{\partial A_3}{\partial x^2} \,dx^2 \wedge dx^3
	\\[2ex] 
	=
	\phantom{+} & \left(\frac{1}{c}\frac{\partial (-\varphi)}{\partial x^1}-\frac{\partial A_1}{\partial x^0}\right) dx^1 \wedge dx^0 
	+ \left(\frac{1}{c}\frac{\partial (-\varphi)}{\partial x^2}-\frac{\partial A_2}{\partial x^0}\right) dx^2 \wedge dx^0
	\\
	+ &\left(\frac{1}{c}\frac{\partial (-\varphi)}{\partial x^3}-\frac{\partial A_3}{\partial x^0}\right) dx^3 \wedge dx^0
	+ \left(\frac{\partial A_1}{\partial x^2}-\frac{\partial A_2}{\partial x^1}\right) dx^2 \wedge dx^1
	\\
	+ & \left(\frac{\partial A_1}{\partial x^3}-\frac{\partial A_3}{\partial x^1}\right) dx^3 \wedge dx^1 
	+ \left(\frac{\partial A_2}{\partial x^3}-\frac{\partial A_3}{\partial x^2}\right) dx^3 \wedge dx^2.				
\end{align*}
Wir wissen, dass $dx^0 = c\,dt$ entspricht und nutzen dies hier aus, um die Klammerausdrücke besser interpretieren zu können.
In der Ableitung nach der Koordinate $x^0$ verbirgt sich nämlich noch der Faktor $c$ im Nenner.
Um zu zeigen dass dies so funktioniert, wechseln wir kurz in die anderen Koordinaten und erhalten
\begin{align*}
	dA = \phantom{\frac{1}{c}}
	& \left(\frac{1}{c}\frac{\partial (-\varphi)}{\partial x}-\frac{1}{c}\frac{\partial A_1}{\partial t}\right) dx \wedge c\,dt 
	+ \left(\frac{1}{c}\frac{\partial (-\varphi)}{\partial y}-\frac{1}{c}\frac{\partial A_2}{\partial t}\right) dy \wedge c\,dt 
	\\
	+ &\left(\frac{1}{c}\frac{\partial (-\varphi)}{\partial z}-\frac{1}{c}\frac{\partial A_3}{\partial t}\right) dz \wedge c\,dt
	+ \left(\frac{\partial A_1}{\partial y}-\frac{\partial A_2}{\partial x}\right) dy \wedge dx 
	\\
	+ &\left(\frac{\partial A_1}{\partial z}-\frac{\partial A_3}{\partial x}\right) dz \wedge dx 
	+ \left(\frac{\partial A_2}{\partial z}-\frac{\partial A_3}{\partial y}\right) dz \wedge dy.
\end{align*}
\begin{comment}
Diesen Teil oben rausgenommen da hier die c verschwinden (vermutlich irrelevant:)
& = \left(\frac{\partial (-\phi)}{\partial x}-\frac{\partial A_1}{\partial t}\right) dx \wedge dt +
\left(\frac{\partial (-\phi)}{\partial y}-\frac{\partial A_2}{\partial t}\right) dy \wedge dt +
\left(\frac{\partial (-\phi)}{\partial z}-\frac{\partial A_3}{\partial t}\right) dz \wedge dt\\
& + \left(\frac{\partial A_1}{\partial y}-\frac{\partial A_2}{\partial x}\right) dy \wedge dx +
\left(\frac{\partial A_1}{\partial z}-\frac{\partial A_3}{\partial x}\right) dz \wedge dx +
\left(\frac{\partial A_2}{\partial z}-\frac{\partial A_3}{\partial y}\right) dz \wedge dy.\\
\end{comment}
Setzen wir hier die obigen Formeln für $E$ und $B$ ein erhalten wir
\begin{align*}
	dA = \phantom{+}
	\frac{E_1}{c} \,dx &\wedge c\,dt +
	\frac{E_2}{c} \,dy \wedge c\,dt +
	\frac{E_3}{c} \,dz \wedge c\,dt
	\\
	\, - \, B_3 \, dy &\wedge \phantom{c}dx +
	\, B_2 \,  \, dz \wedge \phantom{c} dx -
	\, B_1 \, \, dz \wedge \phantom{c} dy.
\end{align*}
Wechseln wir nun wieder in die ursprünglichen Koordinaten ergibt sich
\begin{align*}
	dA = \phantom{+}
	& \frac{E_1}{c} \,dx^1 \wedge dx^0 +
	\frac{E_2}{c} \,dx^2 \wedge dx^0 +
	\frac{E_3}{c} \,dx^3 \wedge dx^0
	\\
	- \, &B_3 \, dx^2 \wedge dx^1 +
	\, B_2 \, \, dx^3 \wedge dx^1 -
	\, B_1 \, \, dx^3 \wedge dx^2
	\\[2ex] 
	= \quad 
	&\begin{pmatrix}
		0 & -E_1/c & -E_2/c & -E_3/c \\ E_1/c & 0 & B_3 & -B_2 \\ E_2/c & -B_3 & 0 & B_1 \\ E_3/c & B_2 & -B_1 & 0 
	\end{pmatrix},
\end{align*}
was exakt dem Faraday-Tensor entspricht.
Wir haben nun also eine 1-Form gefunden, deren äussere Ableitung den Faraday-Tensor ergibt.
Diese 1-Form definieren wir als Viererpotential
\begin{equation}
	A = -\frac{\varphi}{c}\,dx^0 + A_1 \,dx^1 + A_2 \,dx^2 + A_3 \,dx^3.
\end{equation}
Verwenden wir wiederum die Definitionen $dx^0 = c\,dt, dx^1 = dx, dx^2 = dy$ und $dx^3 = dz$,
ergibt sich das Viererpotential als
\begin{equation}
	A = -\varphi\,dt + A_1 \,dx + A_2 \,dy + A_3 \,dz.
\end{equation}