%
% einleitung.tex -- Beispiel-File für die Einleitung
%
% (c) 2020 Prof Dr Andreas Müller, Hochschule Rapperswil
%
% !TEX root = ../../buch.tex
% !TEX encoding = UTF-8
%
\section{Einleitung zu Open Foam\label{openfoam:section:Einleitung}}
\kopfrechts{Einleitung Zu Open Foam}

\subsection{Ziel und Inhalt unserer Arbeit}
OpenFOAM ist eine Open Source Software zur Berechnung von Strömungsfeldern. 
\index{OpenFOAM}%
Das Ziel dieses Papers ist es, dem Leser einen Überblick zu verschaffen, was OpenFOAM ist, die Mathematik hinter Fluid-Dynamik-Simulationen und anhand eines 
\index{Fluid-Dynamik-Simulation}%
Anwendungsbeispiels zu zeigen, wie eine einfache Simulation selbst durchgeführt werden kann.


\subsection{Was ist eigentlich OpenFOAM?\label{openfoam:section:WasIstOpenFoam}}
OpenFOAM ist, wie der Name sagt, eine Open Source Software die unter der GNU
\index{GNU}%
General Public License kostenlos zugänglich ist und dadurch auch für kommerzielle Zwecke 
\index{General Public License (GPL)}%
verwendet werden kann. 

Bei OpenFOAM handelt es sich um eine \emph{CFD} Software, dies steht für \emph{C}omputational \emph{F}luid \emph{D}ynamics, also numerische Strömungsdynamik.
\index{CFD}%
OpenFOAM kann dementsprechend verwendet werden, um näherungsweise Lösungen für 
strömungsmechanische Probleme zu finden.
\index{strömungsmechanisch}%
Es können zum Beispiel Widerstandsbeiwerte von Autos oder die Wirbelschleppen einer Tragfläche simuliert werden, zum Thema Wirbelschleppen gibt es noch einen Abschnitt im Kapitel~\ref{chapter:wirbelringe} über Wirbelringe.
\index{Widerstandsbeiwert}%
\index{Wirbelschleppe}%
Dabei liegen für OpenFOAM im Vergleich zu anderer CFD-Software die Stärken nicht bei Nutzerfreundlichkeit oder Geschwindigkeit, sondern bei der Anpassbarkeit, zum einen da es sich um offene Software handelt, andererseits aber auch dadurch, dass OpenFOAM entwickelt wurde mit dem Ziel, die Erweiterbarkeit durch den Nutzer zu ermöglichen.
\index{Nutzerfreundlichkeit}%
\index{Anpassbarkeit}%
Dies macht es weniger praktikabel, die Software als Simulator zum Lösen von Standardproblemen anzuwenden, sondern für das Entwickeln von Solvern für neue Problemstellungen, für die noch keine herkömmliche Solver existieren.

OpenFOAM wird seit 2004 hauptsächlich von OpenCFD Ltd entwickelt, es wird jedoch auch aktiv durch die Community weiterentwickelt.
\index{OpeNCFD Ltd}%
Die getesteten Releases werden alle sechs Monate herausgegeben \cite{openfoam:greenshieldsweller2022}.

\subsection{Was macht OpenFOAM?}
Zum Simulieren strömungsmechanischer Probleme muss man die partiellen Differenzialgleichungen (v. a. Erhaltungssätze), die das Verhalten des Mediums beschreiben, ausgehend von Anfangs- und Randbedingungen lösen.
\index{Erhaltungssatz}%
\index{partielle Differenzialgleichung}%
Dabei beschreiben die Differenzialgleichungen Felder für die Simulationsgrößen wie Geschwindigkeit, Druck, Temperatur, Dichte und je nach Anwendung noch viele mehr. OpenFOAM nutzt die Manipulation von solchen Feldern, um eine numerische Lösung zu finden. Daher kommt auch das Akronym FOAM, es steht für Field Operation And Manipulation.
\index{Geschwindigkeit}%
\index{Druck}%
\index{Temperatur}%
\index{Dichte}%
\index{Field Operation And Manipulation (FOAM)}%

In dieser Arbeit wird zuerst erläutert was für mathematische Modelle benötigt werden, um das Verhalten einer Flüssigkeit zu beschreiben.
\index{mathematisches Modell}%
\index{Euler-Gleichung}%
\index{Navier-Stokes-Gleichung}%
Dafür werden die Euler-Gleichungen, die Navier-Stokes-Gleichungen und einige Randbedingungen vorgestellt.
Danach wird gezeigt wie OpenFOAM diese Modelle verwendet um eine CFD-Simulation durchzuführen.
Zuletzt wird ein kleines Anwendungsbeispiel für OpenFOAM gezeigt.
