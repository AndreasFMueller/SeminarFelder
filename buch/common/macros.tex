%
% macros.tex -- some common macro definitions
%
% (c) 2021 Prof Dr Andreas Müller, OST Ostschweizer Fachhochschule
%
\hypersetup{
    linktoc=all,
    linkcolor=blue
}
\newcounter{beispiel}
\newenvironment{beispiele}{
\bgroup\smallskip\parindent0pt\bf Beispiele\egroup

\begin{list}{\arabic{beispiel}.}
  {\usecounter{beispiel}
  \setlength{\labelsep}{5mm}
  \setlength{\rightmargin}{0pt}
}}{\end{list}}
\newcounter{uebungsaufgabezaehler}
% environment fuer uebungsaufgaben
\newenvironment{uebungsaufgaben}{%
}{%
\vfill\pagebreak}
\newenvironment{teilaufgaben}{
\begin{enumerate}
\renewcommand{\theenumi}{\alph{enumi})}
%\renewcommand{\labelenumi}{\alph{enumi})}
\renewcommand{\labelenumi}{\theenumi}
}{\end{enumerate}}
% Aufgabe
\newcounter{problemcounter}[chapter]
\def\aufgabenpath{chapters/uebungsaufgaben/}
\def\ainput#1{\input\aufgabenpath/#1}
\def\verbatimainput#1{\expandafter\verbatiminput{\aufgabenpath/#1}}
\def\aufgabetoplevel#1{%
\expandafter\def\expandafter\inputpath{#1}%
\let\aufgabepath=\inputpath
}
\def\includeagraphics[#1]#2{\expandafter\includegraphics[#1]{\aufgabepath#2}}
% \aufgabe
\renewcommand\theproblemcounter{\thechapter.\arabic{problemcounter}}
\newcommand{\uebungsaufgabe}[1]{%
\refstepcounter{problemcounter}%
\label{#1}%
\bigskip{\parindent0pt\strut}\hbox{\bf\theproblemcounter. }%
\expandafter\def\csname aufgabenpath\endcsname{\inputpath/}%
\expandafter\input{\aufgabenpath/#1.tex}
}
% linsys
\newcolumntype{\linsysR}{>{$}r<{$}}
\newcolumntype{\linsysL}{>{$}l<{$}}
\newcolumntype{\linsysC}{>{$}c<{$}}
\newenvironment{linsys}[1]{%
\begin{tabular}{*{#1}{\linsysR@{\;}\linsysC}@{\;}\linsysR}}%
{\end{tabular}}

% Loesung
\def\swallow#1{
%nothing
}
\NewEnviron{loesung}[1][Lösung]{%
\begin{proof}[#1]%
\renewcommand{\qedsymbol}{$\bigcirc$}
\BODY
\end{proof}
}
\NewEnviron{bewertung}{%
\begin{proof}[Bewertung]%
\renewcommand{\qedsymbol}{}
\BODY
\end{proof}
}
\NewEnviron{diskussion}{%
\begin{proof}[Diskussion]%
\renewcommand{\qedsymbol}{}
\BODY
\end{proof}
}
\NewEnviron{hinweis}{%
\begin{proof}[Hinweis]%
\renewcommand{\qedsymbol}{}
\BODY
\end{proof}
}
\def\keineloesungen{%
\RenewEnviron{loesung}{\relax}
\RenewEnviron{bewertung}{\relax}
\RenewEnviron{diskussion}{\relax}
}
\newenvironment{beispiel}{%
\refstepcounter{satz}
\begin{proof}[Beispiel \arabic{chapter}.\arabic{satz}]%
\renewcommand{\qedsymbol}{$\bigcirc$}
}{\end{proof}}

\allowdisplaybreaks

\lhead{Inhaltsverzeichnis}
\rhead{}
\tableofcontents
\newtheorem{satz}{Satz}[chapter]
\newtheorem{axiom}[satz]{Axiom}
\newtheorem{hilfssatz}[satz]{Hilfssatz}
\newtheorem{korollar}[satz]{Korollar}
\newtheorem{lemma}[satz]{Lemma}
\newtheorem{definition}[satz]{Definition}
\newtheorem{annahme}[satz]{Annahme}
\newtheorem{frage}[satz]{Frage}
\newtheorem{problem}[satz]{Problem}
\newtheorem{aufgabe}[satz]{Aufgabe}
\newtheorem{prinzip}[satz]{Prinzip}
\newtheorem*{problem*}{Problem}
\newtheorem{forderung}{Forderung}[chapter]
\newtheorem{konsequenz}[satz]{Konsequenz}
\newtheorem{algorithmus}[satz]{Algorithmus}

% English variants
\newtheorem{theorem}[satz]{Theorem}

\renewcommand{\floatpagefraction}{0.7}

\definecolor{darkgreen}{rgb}{0,0.6,0}
\definecolor{darkred}{rgb}{0.8,0,0}
\definecolor{orange}{rgb}{1,0.6,0}
\definecolor{gelb}{rgb}{1,0.8,0}
%
% Kopfzeilen
%
\fancyfoot{}
\fancyhead{}
%\def\kopfrechts#1{
%\edef\theshortsection{\arabic{section}}
%\rhead{\theshortsection. #1}}
%\def\kopflinks#1{\lhead{\thechapter. #1}
%\rhead{}}
\def\kopfrechts#1{
\global\def\kopfinhalt{#1}
\xdef\theshortsection{\thechapter.\arabic{section}}
\fancyhead[LO]{\theshortsection. #1}}
%
\def\kopflinks#1{
\global\def\kapitelinhalt{#1}
\fancyhead[RE]{\thechapter. #1}
\thispagestyle{empty}
\fancyhead[LO]{}}
% Kapitelautor
\def\chapterauthor#1{{\large #1}\bigskip\bigskip}
%
%
% Uebungsaufgaben
%
\def\uebungsabschnitt{%
\begingroup

\bigskip

\noindent
\vbox to0.3cm{\hbox to\textwidth{\normalfont\large\bfseries Übungsaufgaben}\vfill}
%\keineloesungen%
\nopagebreak%
\fancyhead[RE]{\thechapter. \kapitelinhalt}%
\fancyhead[LE]{\thepage}%
\fancyhead[LO]{Übungsaufgaben}%
\fancyhead[RO]{\thepage}%
\addcontentsline{toc}{section}{Übungsaufgaben}%
}
\def\enduebungsabschnitt{
\endgroup
\bigskip
}
%
\fancyhead[LE]{\thepage}
\fancyhead[RO]{\thepage}
% Transposition
\def\transpose#1{{{#1}^t}}
% Real- und Imaginaerteil
\def\Re{\operatorname{Re}}
\def\Im{\operatorname{Im}}
\def\tr{\operatorname{tr}}
\def\sign{\operatorname{sign}}
\def\grad{\operatorname{grad}}
\def\supp{\operatorname{supp}}%
