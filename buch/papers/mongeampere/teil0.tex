%
% einleitung.tex -- Beispiel-File für die Einleitung
%
% (c) 2020 Prof Dr Andreas Müller, Hochschule Rapperswil
%
% !TEX root = ../../buch.tex
% !TEX encoding = UTF-8
%
\section{Teil 0\label{mongeampere:section:teil0}}
\kopfrechts{Teil 0}
Die Monge-Ampèresche Gleichung
\begin{equation}
  \det D^2 u = f(x, u, \nabla u)
  \label{mongeampere:eq:mongeampere}
\end{equation}
wobei $D$ die Hessche Matrix
\begin{equation}
  D^2 u =
  \begin{pmatrix}
    \pdv[2]{u}{x_1} & \dots &\pdv{u}{x_1}{x_n} \\
  \vdots & \ddots & \vdots \\
  \pdv{u}{x_n}{x_1} & \dots &\pdv[2]{u}{x_n} \\
  \end{pmatrix}
  \label{mongeampere:eq:hess}
\end{equation}
ist.

