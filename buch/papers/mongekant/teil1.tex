%
% teil1.tex -- Beispiel-File für das Paper
%
% (c) 2020 Prof Dr Andreas Müller, Hochschule Rapperswil
%
% !TEX root = ../../buch.tex
% !TEX encoding = UTF-8
%
\section{Monge-Formulierung%
\label{mongekant:section:teil1}}
\kopfrechts{Monge-Formulierung}

Nachdem die Grundbegriffe des Transports eingeführt wurden,
formulieren wir das von Gaspard Monge (1781) aufgestellte Optimierungsproblem.
\index{Monge, Gaspard}%
Dazu können wir \eqref{mongekant:eq:monge_transport_cost} verwenden,
um die günstigsten Transportkosten zu berechnen:
\index{Transportkosten}%
\begin{align}
C_M(\mu, \nu)
&:=
\inf_{T}
\int_X c\,\bigl(x, T(x)\bigr)\, d\mu(x)
,\quad
\text{wobei } T_{\#}\mu=\nu
.
\label{mongekant:eq:monge_problem}
\end{align}
Wir verwenden das \emph{Infimum} und nicht das \emph{Minimum},
\index{Infimum}%
\index{Minimum}%
weil es nicht immer eine Abbildung $T$ gibt,
die das Infimum annimmt.
Ein konkretes Gegenbeispiel sehen wir in
Abschnitt~\ref{mongekant:subsection:monge_inexistence}.

\subsection{Spezialfälle mit Existenzgarantie}
Zwei besonders wichtige Fälle,
in denen das Infimum tatsächlich ein Minimum ist,
sind:
\begin{itemize}
\item Der diskreten Fall, wenn
\begin{align*}
\mu
&=
\frac{1}{n} \sum_{i=1}^n \delta_{x_i}
\\
\nu
&=
\frac{1}{n} \sum_{j=1}^n \delta_{y_j}
,
\end{align*}
wobei $\delta$ die Dirac-Funktion bezeichnet.
\item Der absolut stetigen Fall, wenn
\begin{align}
%\left.
\begin{aligned}
d\mu
&=
f(x)\, dx
\\
d\nu
&=
g(y)\, dy
.
\end{aligned}
%\right\}
%\quad\Rightarrow\quad
%f(x) = g(T(x))\,T'(x)
\label{mongekant:eq:absolute_densities}
\end{align}
Diesen Zusammenhang erhält man,
wenn man \eqref{mongekant:eq:absolute_continuity} ableitet und
Gleichheit für alle möglichen Teilmengen $A$ und $B$ fordert.
\end{itemize}
Die Beweise für die Existenz eines optimalen Transports $T$ in beiden Szenarien
benötigen tiefere Resultate aus Topologie und Masstheorie und
\index{Topologie}%
\index{Masstheorie}%
liegen ausserhalb des Umfangs dieses Seminars.

\subsection{Zusammenhang mit der Monge-Ampère-Gleichung}
Der interessierte Leser fragt sich vielleicht,
ob die Monge-Ampère-Gleichung~\eqref{mongeampere:eq:mongeampere}
aus Kapitel~\ref{chapter:mongeampere}
mit dem Monge-Problem zusammenhängt.
Tatsächlich gibt es einen Zusammenhang,
den wir nun kurz erläutern wollen.

Betrachten wir den absolut stetigen Fall aus \eqref{mongekant:eq:absolute_densities}.
Wir wissen,
dass die Massenerhaltung gilt,
also muss
\begin{align*}
\int_{A} f(x) \, dx
&=
\int_{T(A)} g(y) \, dy
\end{align*}
für alle messbaren $A$ gelten.
Setzt man $y = T(x)$,
folgt
\begin{align*}
\int_{A} f(x) \, dx
&=
\int_{A} g \bigl(T(x)\bigr) \det \bigl(DT(x)\bigr) \, dx
.
\end{align*}
Wird nun $T = \nabla \varphi$ gewählt,
wobei $\varphi$ eine konvexe Funktion ist,
die zweimal stetig differenzierbar ist,
ergibt sich
\begin{align*}
\int_{A} f(x) \, dx
&=
\int_{A} g\bigl(\nabla \varphi(x)\bigr) \det\bigl(D^2 \varphi(x)\bigr) \, dx
.
\end{align*}
Da die Gleichheit für alle messbaren $A$ gelten muss,
kann man die Integranden gleichsetzen und erhält
\begin{align*}
f(x)
&=
g\bigl(\nabla \varphi(x)\bigr) \det\bigl(D^2 \varphi(x)\bigr)
.
\end{align*}
Löst man nach $\det(D^2 \varphi(x))$ auf,
folgt die Monge-Ampère-Gleichung
\index{Monge-Ampere-Gleichung@Monge-Ampère-Gleichung}%
\begin{align*}
\det\bigl(D^2 \varphi(x)\bigr)
&=
\frac{f(x)}{g\bigl(\nabla \varphi(x)\bigr)}
.
\end{align*}
In \cite{mongekant:brenier} wird eindrücklich gezeigt,
dass die Abbildung $T = \nabla \varphi$
die eindeutige Lösung des Monge-Problems \eqref{mongekant:eq:monge_problem} ist,
wenn $c(x,y) = \left\lvert x - y\right\rvert^2$ ist und die obigen Bedingungen gelten.

\subsection{Fallbeispiel: Nichtexistenz einer Lösung%
\label{mongekant:subsection:monge_inexistence}}
Betrachten wir den Fall,
in dem die Quellverteilung $\mu$ eine Punktmasse an Stelle $x_1$ ist,
\begin{align*}
\mu
&=
\delta_{x_1}
\end{align*}
und die Zielverteilung $\nu$ aus zwei unterschiedlichen Punkten besteht
\begin{align*}
\nu
&=
\frac{1}{2} \delta_{y_1} + \frac{1}{2} \delta_{y_2}
,\quad
\text{wobei } y_1 \neq y_2
.
\end{align*}
Dann gilt $\nu(\{y_1\}) = \frac{1}{2}$,
aber für
\begin{align*}
T_{\#}\mu\left(\left\{y_1\right\}\right)
=
\begin{cases}
0,
&\text{wenn }
T^{-1}(y_1)
\neq
x_1
\\
1,
&\text{wenn }
T^{-1}(y_1)
=
x_1
.
\end{cases}
\end{align*}
Somit kann kein Bildmass $T_{\#}\mu$ die Gleichung $T_{\#}\mu=\nu$ erfüllen ---
das Monge-Problem \eqref{mongekant:eq:monge_problem}
besitzt in diesem Beispiel keine Lösung.


