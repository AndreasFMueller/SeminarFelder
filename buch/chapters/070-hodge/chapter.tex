%
% chapter.tex -- Hodge-Operator, Vektoroperatoren, Codifferential und Laplace
%
% (c) 2025 Prof Dr Andreas Müller
%
\chapter{Hodge-Operator, Vektoranalysis und Laplace-Operator
\label{chapter:hodge}}
\kopflinks{Hodge-Operator, Vektoranalysis und Laplace-Operator}
In der Theorie der $p$-Formen wurde klar, dass der Vektorraum der
$p$-Formen $\binom{n}{p}$-dimensional ist.
Diese Eigenschaft folgt daraus, dass die Basisvektoren
$dx^{i_1}\wedge\dots\wedge dx^{i_p}$ durch die Auswahl von $p$ Elementen
aus den $n$ möglichen Basis-1-Formen $dx^1,\dots,dx^n$ gebildet
werden.
Der Raum der $n-p$-Formen hat wegen $\binom{n}{n-p}=\binom{n}{p}$
die gleiche Dimension.
Dies ist nicht nur ein Zufall.
Es nämlich eine natürlich invertierbare lineare Abbildung, die $p$-Formen
in $n-p$-Formen umwandelt, den Hodge-Operator, der in
Abschnitt~\ref{buch:hodge:section:hodge} definiert wird.
Die klassischen Operatoren der Vektoranalysis lassen sich als
Kombinationen des Hodge-Operators mit der äusseren Ableitung identifizieren,
was in Abschnitt~\ref{buch:hodge:section:vektoranalysis} durchgeführt
wird.

%
% 1-hodge.tex -- Der Hodge-Operator
%
% (c) 2025 Prof Dr Andreas Müller
%
\section{Der Hodge-Operator
\label{buch:hodge:section:hodge}}
\kopfrechts{Der Hodge-Operator}
Der Hodge-Operator bildet $p$-Formen umkehrbar auf $n-p$-Formen ab.
Im folgenden wird er zunächst ad hoc mit Hilfe der Basis-$p$-Formen
konstruiert.

%
% Basisformen
%
\subsection{Basisformen}
Der Raum der $p$-Formen hat als Basis die $p$-Formen der Form
\[
\alpha = dx^{i_1}\wedge\dots\wedge dx^{i_p}.
\]
Die Indizes $i_1,\dots,i_p$ müssen alle verschieden sein, weil
$dx^i\wedge dx^i=0$ ist.
Zu jeder Auswahl von $p$ Indizes $i_1,\dots,i_p$ aus der Mengen
$[n]=\{1,\dots,n\}$ der Zahlen von 1 bis $n$ gibt es genau eine Basisform.
Die Dimension des Raums der $p$-Formen ist daher die Anzahl der möglichen
Auswahlen von $p$ Zahlen aus den $n$ Zahlen $[n]$.

Die Auswahl von $p$ Zahlen aus $[n]$ lässt aber genau $n-p$ Basisformen
zurück.
Aus den Indizes $n-p$ Zahlen aus $]n]$, die in $i_1,\dots,i_p$ nicht
vorkommen, lässt sich eine Basis $n-p$-Form bilden.
Diese Zuordnung einer Basis-$p$-Form $dx^{i_1}\wedge\dots\wedge dx^{i_p}$
ist offensichtlich umkehrbar.
Es bleibt noch die Freiheit, das Vorzeichen festzulegen.

\begin{definition}[Hodge-Operator]
\label{buch:hodge:hodge:definition:hodge}
Der \emph{Hodge-Operator} ist der lineare Operator, der der
\index{Hodge-Operator}%
Basis-$p$-Form
$dx^{i_1}\wedge\dots\wedge dx^{i_p}$ mit $i_1<i_2<\dots<i_p$
die $n-p$-Form
\[
\ast(dx^{i_1}\wedge\dots\wedge dx^{i_p})
=
s\,dx^{j_1}\wedge\dots\wedge dx^{j_{n-p}}
\]
mit $j_1<\dots <j_{n-p}$, $s\in\{\pm1\}$ zuordnet, für die
$\{i_1,\dots,i_p\}\cup\{j_1,\dots,j_{n-p}\}=[n]$ gilt.
Das Vorzeichen $s$ muss so gewählt werden, dass
\[
(dx^{i_1}\wedge\dots\wedge dx^{i_p})
\wedge
(s\,dx^{j_1}\wedge\dots\wedge dx^{j_{n-p}})
=
dx^1\wedge dx^2\wedge\dots\wedge dx^n
\]
gilt.
\index{*-Operator@$*$-Operator}%
\end{definition}

Ganz allgemein lässt sich daraus bereits beantworten, wie $0$-Formen und
$n$-Formen durch den Hodge-Operator abgebildet werden.
Die einzige 0-Form ist $\alpha=1$.
Es muss eine $n$-Form $\beta$ gefunden werden, so dass
$\alpha\wedge\beta=dx^1\wedge\dots\wedge dx^n$ ist.
Da es nur die eine $n$-Form $dx^1\wedge\dots\wedge dx^n$ gibt, muss
nur noch das Vorzeichen gefunden werden.
Da aber
\[
\alpha\wedge
s\,dx^1\wedge\dots\wedge dx^n
=
s\,dx^1\wedge\dots\wedge dx^n
=
dx^1\wedge\dots\wedge dx^n
\]
sein soll, muss $s=1$ gewählt werden.
Somit ist $\ast 1=dx^1\wedge\dots\wedge dx^n$.
Auf die gleiche Art kann man folgern, dass
$\ast(dx^1\wedge\dots\wedge dx^n)=1$.

\begin{beispiel}
\label{buch:hodge:hodge:beispiel:r2}
Hodge-Operator in $\mathbb{R}^2$.
Die Basis-1-Formen sind $dx^1$ und $dx^2$ und die Form höchsten Grades ist
die 2-Form $dx^1\wedge dx^2$.
Aus der Definition lässt sich jetzt sofort die Wirkung des Hodge-Operators
ableiten.
Für 0- und 2-Formen wurde dies im Anschluss an die
Definition~\ref{buch:hodge:hodge:definition:hodge} durchgeführt.

Die 1-Formen $dx^1$ und $dx^2$ werden aufeinander abgebildet, es müssen
aber noch die Vorzeichen ermittelt werden.
Wenn $\*dx^1=s\,dx^2$ ist, dann soll
$dx^1\wedge \*dx^1=dx^1\wedge s\,dx^2=dx^1\wedge dx^2$ sein.
Somit muss $s=1$ gewählt werden und es folgt $\*dx^1=dx^2$.
Wenn $\*dx^2=s\,dx^1$ ist, dann soll
$dx^2\wedge(\*dx^2)=dx^2\wedge(s\,dx^1)=dx^1\wedge dx^2$ sein.
Dies kann erreicht werden, dann man im mittleren Ausdruck der
Gleichungskette $dx^1$ und $dx^2$ vertauscht und dafür $s=-1$ setzt.
Es folgt $\*dx^2=-dx^1$.
\end{beispiel}

\begin{beispiel}
\label{buch:hodge:hodge:beispiel:r3}
Hodge-Operator in $\mathbb{R}^3$.
Im dreidimensionalen Fall ist nur die Abbildung von 1-Formen auf
2-Formen und zurück zu untersuchen.
Nach der gleichen Methode wie im
Beispiel~\ref{buch:hodge:hodge:beispiel:r2}
sind nur noch die Vorzeichen zu bestimmen.
Die Rechnung ergibt die Zuordnung
\begin{align*}
\ast dx^1 &= \phantom{-}dx^2\wedge dx^3 \\
\ast dx^2 &=          - dx^1\wedge dx^3 \\
\ast dx^3 &= \phantom{-}dx^1\wedge dx^2 \\
\ast(dx^1\wedge dx^2) &= \phantom{-}dx^3 \\
\ast(dx^1\wedge dx^3) &=          - dx^2 \\
\ast(dx^2\wedge dx^3) &= \phantom{-}dx^1
\qedhere
\end{align*}
\end{beispiel}

Da die Indexmengen $\{i_1,\dots,i_p\}$ und
$\{j_1,\dots,j_{n-p}\}$ in $[n]$ komplementär sein müssen,
muss
\(
\ast{\ast}(dx^{i_1}\wedge\dots\wedge dx^{i_p})
=
\pm dx^{i_1}\wedge\dots\wedge dx^{i_p}
\)
sein.
Aus den Beispielen kann man ablesen, dass in den Fällen $n=2$ und $n=3$
der iterierte Hodge-Operator  $\ast\ast$ durch
\[
\left.
\begin{aligned}
\ast{\ast dx^1} &= \phantom{-}{\ast dx^2} = -dx^1\\
\ast{\ast dx^2} &=          - {\ast dx^1} = -dx^2
\end{aligned}
\quad
\right\}
\qquad
\text{bzw.}
\qquad
\left\{
\quad
\begin{aligned}
\ast{\ast dx^1} &= \phantom{-}{\ast(dx^2\wedge dx^3)} = dx^1 \\
\ast{\ast dx^2} &=          - {\ast(dx^1\wedge dx^3)} = dx^2 \\
\ast{\ast dx^3} &= \phantom{-}{\ast(dx^1\wedge dx^2)} = dx^3 
\end{aligned}
\right.
\]
gegeben ist.
Diese Überlegungen können verallgemeinert werden und ergeben den
folgenden Satz.

\begin{satz}
Für jede $p$-Form auf einer $n$-dimensionalen Mannigfaltigkeit gilt
\[
\ast{\ast \omega}
=
(-1)^{p(n-p)}\omega.
\]
\end{satz}

\begin{proof}
Sei $\omega = dx^{i_1}\wedge\dots\wedge dx^{i_p}$ und 
$\ast\omega = s\,dx^{j_1}\wedge\dots\wedge dx^{j_{n-p}}$ mit
$s\in\{\pm 1\}$.
Das Vorzeichen $s$ kommt von den Vertauschungen der Basis-1-Formen,
mit denen man die Standardreihenfolge $dx^1\wedge \dots\wedge dx^n$
erreicht.
Es erfüllt
\begin{equation}
(dx^{i_1}\wedge\dots\wedge dx^{i_p})
\wedge
s(dx^{j_1}\wedge\dots\wedge dx^{j_{n-p}})
=
dx^1\wedge\dots\wedge dx^n.
\label{buch:hodge:hodge:vorz1}
\end{equation}

Jetzt soll $\ast{\ast\,\omega}$ berechnet werden.
Nach Definition gibt es einen Vorzeichenfaktor $t\in\{\pm1\}$ mit
\[
\ast(dx^{j_1}\wedge\dots\wedge dx^{j_{n-p}})
=
t\,dx^{i_1}\wedge\dots\wedge dx^{i_p}.
\]
Er muss so gewählt werden, dass
\begin{align}
(dx^{j_1}\wedge\dots\wedge dx^{j_{n-p}})
\wedge
t(dx^{i_1}\wedge\dots\wedge dx^{i_p})
&=
dx^1\wedge\dots\wedge dx^n
\label{buch:hodge:hodge:vorz2}
\end{align}
gilt.
Für den iterierten Hodge-Operator findet man dann
\begin{align}
\ast{\ast\, \omega}
=
\ast(s\,dx^{j_1}\wedge\dots\wedge dx^{j_{n-p}})
&=
st\, dx^{i_1}\wedge\dots\wedge dx^{i_p}
\notag
\end{align}
Der iterierte Hodge-Operator ist also das Vorzeichen $st\in\{\pm 1\}$.

Durch $p$ Vertauschungen mit den 1-Formen $dx^{j_p}$,
$dx^{j_{p-1}},\dots,dx^{j_1}$ auf der linken Seite von
\eqref{buch:hodge:hodge:vorz2}
kann man $dx^{i_1}$ an den Anfang
der $n$-Form bringen.
Durch Widerholung für die 1-Formen $dx^{i_2}$ bis $dx^{i_p}$ kann
man mit insgesamt $p(n-p)$ Vertauschungen alle $dx^{i_k}$, $k=1,\dots,p$
nach vorne bringen und erhält so
\begin{align}
(-1)^{p(n-p)}
t(dx^{i_1}\wedge\dots\wedge dx^{i_p})
\wedge
(dx^{j_1}\wedge\dots\wedge dx^{j_p})
&=
dx^1\wedge\dots\wedge dx^n.
\end{align}
Durch Vergleich mit 
\eqref{buch:hodge:hodge:vorz1}
kann man ablesen, dass $(-1)^{p(n-p)}t=s$ gelten muss.
Daraus kann man
$st=(-1)^{p(n-p)}$ ablesen, was den Satz beweist.
\end{proof}

Wenn $n$ ungerade ist, dann ist immer einer der beiden Faktoren $p$
oder $n-p$ gerade und damit ist $(-1)^{p(n-p)}=1$.
Für ungerades $n$ ist der Hodge-Operator also eine Involution.
\index{Involution}%
Für gerades $n$ sind entweder beide Zahlen $p$ und $n-p$ ungerade
oder keine von beiden.
Für $p$ ungerade ist auch $n-p$ und damit $p(n-p)$ ungerade
und entsprechend ist $\ast{\ast\omega}=-\omega$ für $p$-Formen.
Der iterierte Hodge-Operator ist also eine Antiinvolution auf
$p$-Formen ungeraden Grades.


%
% 2-koordinatenfrei.tex -- Koordinatenfreie Definition
%
% (c) 2025 Prof Dr Andreas Müller
%
\section{Koordinatenfreie Definition
\label{buch:hodge:section:koordinatenfrei}}
\kopfrechts{Koordinatenfreie Definition}
Der Hodge-Operators ist bis jetzt explizit mit Hilfe eines
Koordinatensystems definiert worden.
Es ist in keiner Weise offensichtlich, dass die Konstruktion auf
den Basis-$p$-Formen zum richtigen Koordinatentransformtionsverhalten führen
könnte.
In diesem Abschnitt wird daher eine weitere Definition gegeben,
die offensichtlich koordinatenunabhängig ist.
Sie zeigt auch, dass zur Definition des Hodge-Operators als
zusätzliche Eigenschaft der Mannigfaltigkeit ein Skalarprodukt
benötigt wird, die in den bisherigen Entwicklungen keine Rolle
gespielt hat.
Dies ist keine wesentliche Einschränkung der Nützlichkeit der
Definition, denn in allen praktischen Anwendungsfällen ist ein
Skalarprodukt automatisch gegeben oder kann ohne weitere Konsequenzen
verfügbar gemacht werden.

%
% Hodge-Operator und Skalarprodukt
%
\subsection{Hodge-Operator und Skalarprodukt
\label{buch:hodge:koordinatenfrei:subsection:skalarprodukt}}
Der Hodge-Operator ist eine lineare Abbildung, die 
$p$-Formen auf $(n-p)$-Formen abbildet.
Die Definition auf Basis-$p$-Formen in einem Koordinatensystem
hat verlangt, dass zu einer $p$-Form $\omega=dx^{i_1}\wedge\dots\wedge dx^{i_p}$
eine Basis-$(n-p)$-Form $\ast\omega$ so gefunden werden muss, so dass
\begin{equation}
\omega\wedge {\ast\omega}
=
dx^1\wedge\dots\wedge dx^n
\label{buch:hodge:skalarprodukt:eqn:def}
\end{equation}
ergibt.
Auf der rechten Seite von \eqref{buch:hodge:skalarprodukt:eqn:def}
wird die vom gewählten Koordinatensystem abhängige $n$-Form
$dx^1\wedge\dots\wedge dx^n$ verwendet.
Da der Raum der $n$-Formen eindimensional ist, wäre jedes Vielfache
dieser $n$-Form genauso als Basis für die Definition zulässig.
Eine Koordinatentransformation multipliziert die $n$-Form mit der
Funktionaldeterminante der Transformation.
Eine koordinatenunabhängige Definition des Hodge-Operators braucht
daher als Basis die Wahl einer $n$-Form.
Wir nehmen daher im Folgenden an, dass eine $n$-Form $\nu$ gegeben
ist.
Da das Integral einer $n$-Form das $n$-dimensionale Volumen misst,
nennen wir sie auch die {\em Volumenform} auf der Mannigfaltigkeit.

Auf der linken Seite von \eqref{buch:hodge:skalarprodukt:eqn:def}
treten in der Definition nur Basis-$p$-Formen auf.
Lässt man eine beliebige $p$-Formen $\alpha$ und $\beta$ zu, dann
entsteht die $n$-Form $\alpha\wedge{\ast\beta}$, die natürlich ein
Vielfaches der Volumenform $\nu$ sein muss.
Es gibt also eine Zahl $s(\alpha,\beta)$ derart, dass
\begin{equation}
\alpha \wedge {\ast\beta} = s(\alpha,\beta)\,\nu
\label{buch:hodge:skalarprodukt:eqn:salphabeta}
\end{equation}
gilt.
Das Wedge-Produkt auf der linken Seite erfüllt das Distributivgesetz,
die linke Seite ist also linear in $\alpha$.
Der Hodge-Operator wurde als linearer Operator definiert, so dass 
die linke Seite auch linear ist in $\beta$.
Die Funktion $s(\alpha,\beta)$ muss daher bilinear sein.

Für Basis-$p$-Formen $\alpha=dx^{i_1}\wedge\dots\wedge dx^{i_p}$
und $\beta=dx^{j_1}\wedge\dots\wedge dx^{j_p}$ ergibt die
Definition~\eqref{buch:hodge:skalarprodukt:eqn:salphabeta}
\[
s(\alpha,\beta)
=
s(
dx^{i_1}\wedge\dots\wedge dx^{i_p},
dx^{j_1}\wedge\dots\wedge dx^{j_p}
)
=
\begin{cases}
1&\qquad\text{falls }i_1=j_1,\dots,i_p=j_p\\
0&\qquad\text{sonst.}
\end{cases}
\]
Dies bedeutet, dass die Funktion $s(\alpha,\beta)$ ein Skalarprodukt
ist, in dem die Basis-$p$-Formen orthonormiert sind.
Wir schreiben die Funktion daher im folgenden als
$s(\alpha,\beta)=\langle\alpha,\beta\rangle$, die Eigenschaft
\eqref{buch:hodge:skalarprodukt:eqn:salphabeta}
wird daher zu
\begin{equation}
\alpha\wedge {\ast\beta}
=
\langle\alpha,\beta\rangle\,\nu.
\label{buch:hodge:skalarprodukt:eqn:skalar}
\end{equation}
\index{Skalarprodukt von $p$-Formen}%
In dieser Form kommt das Koordinatensystem nicht mehr vor.
Die Definition~\ref{buch:hodge:hodge:definition:hodge}
war so einfach, weil wir von einem kartesischen Koordinatensystem
ausgegangen sind, in dem es plausibel ist, dass die Basis-1-Formen
orthonormiert sind.

Es soll jetzt gezeigt werden, dass
\eqref{buch:hodge:skalarprodukt:eqn:skalar}
den Hodge-Operator eindeutig definiert.
Es muss also gezeigt werden, dass zu einer gegebenen $p$-Form
$\beta$ genau eine $(n-p)$-Form ${\color{darkred}\gamma}$ gibt, so dass
\begin{equation}
\alpha\wedge {\color{darkred}\gamma}
=
\langle\alpha,\beta\rangle\,\nu
\label{buch:hodge:skalarprodukt:def:gamma}
\end{equation}
für alle $p$-Formen $\alpha$.
Für ein Skalarprodukt verschwindet die rechte
Seite von \eqref{buch:hodge:skalarprodukt:def:gamma}
nicht identisch, es ist aber weniger klar, dass dies auch für die
linke Seite gilt.

\begin{satz}
\label{buch:hodge:skalarprodukt:satz:nullteilerfrei}
Das Wedge-Produkt
\[
\Omega^p \times \Omega^{n-p}
\to
\Omega^n
:
(\alpha,{\color{darkred}\gamma})
\mapsto
\alpha\wedge{\color{darkred}\gamma}
\]
ist nicht entartet, d.~h. wenn $\alpha\wedge{\color{darkred}\gamma}=0$
ist, dann ist einer der Faktoren 0.
\end{satz}

\begin{proof}
Die Behauptung des Satzes ist gleichbedeutend mit der Aussage, dass 
sich zu jeder $p$-Form $\alpha\ne 0$ eine $(n-p)$-Form ${\color{darkred}\gamma}$
finden lässt, so dass $\alpha\wedge{\color{darkred}\gamma}\ne  0$ ist.
Wir zeigen dies mit Hilfe einer Basis.

Ist $\lambda^1,\dots,\lambda^n$ eine Basis von 1-Formen, dann bilden
die
\[
\lambda^{i_1}\wedge\dots\wedge\lambda^{i_p},\quad i_1<\dots<i_p
\]
eine Basis der $p$-Formen und die
\[
\lambda^{j_1}\wedge\dots\wedge\lambda^{j_{n-p}},\quad j_1<\dots<j_{n-p}
\]
eine Basis der $(n-p)$-Formen.
Eine $p$-Form $\alpha$ ist eine Linearkombination
\[
\alpha
=
\sum_{i_1<\dots<i_p}
a_{i_1\dots i_p} \lambda^{i_1}\wedge\dots\wedge \lambda^{i_p}.
\]
Sei $i_1,\dots,i_p$ eine Wahl von Indizes so, dass
der $a_{i_1\dots i_p}\ne 0$ ist, und seien die aufsteigend geordneten
Indizes $j_1,\dots,j_{n-p}$ so gewählt, dass
\[
\{i_1,\dots,i_p\}\cup\{j_1,\dots,j_{n-p}\} = [n].
\]
Dann ist
\[
\alpha\wedge(\lambda^{j_1}\wedge\dots\wedge\lambda^{j_{n-p}})
=
\pm
a_{i_1\dots i_p} \lambda^1\wedge\dots\wedge\lambda^n
\ne
0.
\]
${\color{darkred}\gamma}=\lambda^{j_1}\wedge\dots\wedge\lambda^{j_{n-p}}$
erfüllt also die Forderungen.
\end{proof}

Aus Satz~\ref{buch:hodge:skalarprodukt:satz:nullteilerfrei} kann jetzt
abgeleitet werden, dass sich immer eine $(n-p)$-Form ${\color{darkred}\gamma}$
finden lässt, für die $\alpha\wedge{\color{darkred}\gamma}=
\langle\alpha,\beta\rangle\,\nu$ für alle $\alpha$ gilt.
Wir zeigen dies, indem wir ausnutzen, dass der Vektorraum der
$(n-p)$-Formen endlichdimensional ist und sich damit das Problem
auf eine Aufgabe über lineare Gleichungssysteme reduzieren lässt.
Zur Konstruktion des Gleichungssystems definieren zunächst die
Abbildung $a_\alpha$ wie folgt.

\begin{definition}
\label{buch:hodge:koordinatenfrei:def:hodgekoordinatenfrei}
Sei $a_\alpha$ die lineare Abbildung
\[
a_\alpha
\colon
\Omega^{n-p}
\to
\mathbb{R}
:
{\color{darkred}\gamma}
\mapsto
a_\alpha({\color{darkred}\gamma})\,\nu
\]
für die
\[
\alpha\wedge{\color{darkred}\gamma}
=
a_\alpha({\color{darkred}\gamma})\,\nu
\]
ist.
\end{definition}

Die Funktion $a_\alpha({\color{darkred}\gamma})$ ist eine lineare
Funktion von ${\color{darkred}\gamma}$.
Wir möchten zeigen, dass sich damit ein bijektive Abbildung zwischen
endlichdimensionalen Vektorräumen konstruieren lässt.

\begin{satz}
\label{buch:hodge:skalarprodukt:satz:endlich}
Ist $\omega_i$, $i=1,\dots,N$ eine Basis von $p$-Formen, dann ist die
lineare Abbildung
\[
f
\colon
\Omega^{n-p}
\to
\mathbb{R}^N
:
{\color{darkred}\gamma}
\mapsto
\begin{pmatrix}
a_i({\color{darkred}\gamma})\\
\vdots\\
a_N({\color{darkred}\gamma})
\end{pmatrix}
\]
eine Bijektion.
\end{satz}

\begin{proof}
Gäbe es ein ${\color{darkred}\gamma}$ mit $f({\color{darkred}\gamma})=0$,
dann ist $\omega_i\wedge{\color{darkred}\gamma}=0$ für jedes $i$.
Da die $\omega_i$ eine Basis bilden, folgt
$\omega\wedge{\color{darkred}\gamma}=0$ für alle $p$-Formen $\omega$.
Dies widerspricht aber dem
Satz~\ref{buch:hodge:skalarprodukt:satz:nullteilerfrei}.
Somit kann es keine solches ${\color{darkred}\gamma}$ geben.
Da der Kern von $f$ nur aus der Nullform besteht, ist $f$ injektiv.
Da die beiden Vektorräume die gleiche Dimension haben, ist $f$ bijektiv.
\end{proof}

Um zu zeigen, dass mit \eqref{buch:hodge:skalarprodukt:eqn:skalar}
tatsächlich der Hodge-Operator definiert werden kann, müssen wir
nachweisen, dass das Skalarprodukt auf $p$-Formen den Hodge-Operator
vollständig festlegt.

\begin{satz}
\label{buch:hodge:satz:eindeutigkeit}
Ist $\langle\alpha,\beta\rangle$ eine nicht entartete Bilinearform
auf $p$-Formen, und $\nu$ eine $n$-Form, dann gibt es eine lineare
Abbildung $\beta\mapsto \ast\beta$ von $p$-Formen in $n-p$-Formen
derart, dass
\begin{equation}
\alpha\wedge {\ast\beta}
=
\langle\alpha,\beta\rangle\,\nu.
\label{buch:hodge:skalarprodukt:satz:eqn}
\end{equation}
Der Operator $\ast$ ist durch
\eqref{buch:hodge:skalarprodukt:satz:eqn}
eindeutig bestimmt.
\end{satz}

\begin{proof}
In einer Basis $\omega_i$ der $p$-Formen wie in
Satz~\ref{buch:hodge:skalarprodukt:satz:endlich}
ist ${\color{darkred}\gamma}$ durch 
\[
f({\color{darkred}\gamma})
=
\begin{pmatrix}
\langle \omega_1,\beta\rangle\\
\vdots\\
\langle \omega_N,\beta\rangle
\end{pmatrix}
\]
definiert.
Da $f$ bijektiv ist, ist ${\color{darkred}\gamma}$ eindeutig
bestimmt und kann durch Lösung eines linearen Gleichungssystems
gefunden werden.
\end{proof}

%
% Skalarprodukt auf Formen
%
\subsection{Skalarprodukt auf Formen}
Der Definitionsansatz~\eqref{buch:hodge:skalarprodukt:def:gamma}
für den Hodge-Operator verlangt, dass sich für beliebige $p$-Formen
ein Skalarprodukt definieren lässt.
Auf einer beliebigen Mannigfaltigkeit kann man nicht einmal für
Tangentialvektoren von der Existenz eines Skalarprodukts ausgehen.
Für die nachfolgende Konstruktion setzt daher voraus, dass auf
der Mannigfaltigkeit eine Metrik gegeben ist, die durch den 
metrischen Tensor $g_{ik}$ beschrieben wird.

%
% Volumenform
%
\subsubsection{Volumenform}
Für die Definition wird die Definition des Volumens benötigt.
Zu einem $n$-Vektor $X_1\wedge\dots\wedge X_n$ muss diese $n$-Form
das Volumen des von diesen Vektoren aufgespannten infinitesimalen
Parallelepipeds bestimmen.
Das Volumen wird durch die Gram-Determinante
(siehe \cite[Abschnitt 8.4]{buch:linalg}).
\begin{align*}
\operatorname{vol}(X_1,\dots,X_n)^2
&=
\operatorname{Gram}(X_1,\dots,X_n)
\\
&=
\left|
\begin{matrix}
\langle X_1,X_1\rangle
	&\langle X_1,X_2\rangle
	&\dots
	&\langle X_1,X_n\rangle
\\
\langle X_2,X_1\rangle
	&\langle X_2,X_2\rangle
	&\dots
	&\langle X_2,X_n\rangle
\\[-2pt]
\vdots
	&\vdots
	&\ddots
	&\vdots
\\
\langle X_n,X_1\rangle
	&\langle X_n,X_2\rangle
	&\dots
	&\langle X_n,X_n\rangle
\end{matrix}
\right|
\intertext{der Vektoren $X_1,\dots,X_n$ gegeben.
Wählt man für die Vektoren die Standardbasisvektoren
$X_k=\partial/\partial x^k$, sind die Skalarprodukte nach
Definition des metrischen Tensors durch die Matrixelement
$g_{ik}$ gegeben.
Somit ist das Volumen}
&=
\left|
\begin{matrix}
g_{11} & g_{12} & \dots  & g_{1n} \\
g_{21} & g_{22} & \dots  & g_{2n} \\
\vdots & \vdots & \ddots & \vdots \\
g_{n1} & g_{n2} & \dots  & g_{nn}
\end{matrix}
\right|
=
\det(g_{ik}).
\end{align*}
Die Determinante der Matrix des metrischen Tensors $g_{ik}$ wird auch
auch mit $g=\det(g_{ik})$ bezeichnet.
Die Volumenform ist daher
\[
\nu
=
\sqrt{g\mathstrut}\, dx^1\wedge\dots\wedge dx^n.
\]

\begin{beispiel}
In Polarkoordinaten $(r,\varphi)$ ist die Metrik durch die
\index{Polarkoordinaten}%
Matrix
\[
(g_{ik})
=
\begin{pmatrix}
1 & 0 \\
0 & r^2
\end{pmatrix}
\]
gegeben.
Die Determinante ist
\[
g
=
\det(g_{ik})
=
r^2
\qquad
\Rightarrow
\qquad
\nu
=
\sqrt{g\mathstrut}\, dr\wedge d\varphi
=
r\,dr\wedge d\varphi.
\]
Dies ist das bekannte Flächenelement in Polarkoordinaten.
\index{Flachenelement in Polarkoordinaten@Flächenelement in Polarkoordinaten}%
\end{beispiel}

\begin{beispiel}
In Kugelkoordinaten $(r,\vartheta,\varphi)$ ist die Metrik durch die
\index{Kugelkoordinaten}%
Matrix
\[
(g_{ik})
=
\begin{pmatrix}
1 &  0  & 0 \\
0 & r^2 & 0 \\
0 &  0  & r^2 \sin^2\varphi
\end{pmatrix}
\qquad
\text{mit der Determinanten}
\qquad
g = \det(g_{ik}) = r^4\sin^2\varphi
\]
gegeben.
Tatsächlich ist das Volumenelement 
\[
\nu 
=
r^2 \sin\varphi\,dr\wedge d\vartheta\wedge d\varphi
\]
in Kugelkoordinaten.
\index{Volumenelement in Kugelkoordinaten}%
\end{beispiel}

%
% Ableitungen von $g^{ik}$
%
\subsubsection{Ableitungen von $g^{ik}$}
Für spätere Anwendungen studieren wir hier noch ein paar Identitäten
für die Ableitungen von $g$ und $^{ik}$.

Die Matrix mit Einträgen $g^{ik}$ ist invers zur Matrix mit
Einträgen $g_{ik}$.
Die Produktmatrix ist daher die Einheitsmatrix, in Komponenten gilt
\begin{equation*}
g^{ik}g_{kl}
=
\delta^i_l.
\end{equation*}
Die Ableitung nach der Koordinaten $x^m$ ist nach der Produktregel
\begin{equation*}
\frac{\partial g^{ik}}{\partial x^m} g_{kl}
+
g^{ik} \frac{\partial g_{kl}}{\partial x^m}
=
0
\qquad\Rightarrow\qquad
\frac{\partial g^{ik}}{\partial x^m} g_{kl}
=
-
g^{ik} \frac{\partial g_{kl}}{\partial x^m}.
\end{equation*}
Die Ableitung von $g^{ik}$ kann daraus bestimmt werden, indem auf
beiden Seiten mit der Inversen von $g_{kl}$ multipliziert wird.
Multiplikation mit der $g^{l\!j}$ auf beiden Seiten ergibt
\begin{align*}
\frac{\partial g^{ik}}{\partial x^m} g_{kl}  g^{lj}
&=
-
g^{ik} \frac{\partial g_{kl}}{\partial x^m} g^{lm}
\intertext{und wegen $g_{kl}g^{l\!j}=\delta^j_k$ folgt}
\frac{\partial g^{i\!j}}{\partial x^m}
&=
-
g^{ik}
g^{l\!j}
\frac{\partial g_{kl}}{\partial x^m}.
\end{align*}
Die Ableitung der kontravarianten metrischen Koeffizienten entstehen
also durch Hochziehen der Indizes der Ableitung der kovarianten
metrischen Koeffizienten und einen Vorzeichenwechsel.
Der Vorzeichenwechsel erinnert daran, dass Ableiten und Hochziehen
der Herunterziehen eines Index nicht vertauschen müssen.

%
% Ableitungen von $g$
%
\subsubsection{Ableitungen von $g$}
Die Determinanten $g$ ist eine algebraische Funktion allein der 
metrischen Koeffizienten $g_{ik}$.
Die Ableitung der Determinante $g$ nach einer Koordinate kann daher
mit der Kettenregel als
\begin{equation*}
\frac{\partial g}{\partial x^m}
=
\sum_{i,k=1}^n
\frac{\partial g}{\partial g_{ik}}
\frac{\partial g_{ik}}{\partial x^m}
\end{equation*}
geschrieben werden.
Die partiellen Ableitungen von $g$ nach den $g_{ik}$ lassen sich
mit dem laplaceschen Entwicklungssatz berechnen.
Wir bezeichnen mit $G_{ik}$ die Minormatrix von $g_{ik}$, in der
die Zeile $i$ und die Spalte $k$ weggelassen sind.
Dann ist 
\begin{align*}
g
&=
\sum_{l=1}^n (-1)^{l+k} g_{lk} \det G_{lk}
&&\text{die Entwicklung nach Zeile $i$ und}
\\
&=
\sum_{l=1}^n (-1)^{i+l} g_{il} \det G_{il}
&&\text{die Entwicklung nach Spalte $k$.}
\end{align*}
In der Minormatrix $G_{ik}$ kommen die $g_{il}$ und $g_{lk}$ für
$l=1,\dots,n$, nicht vor.
Insbesondere kommt $g_{ik}$ nur in einem einzigen Term vor und es
folgt
\[
\frac{\partial g}{\partial g_{ik}}
=
(-1)^{i+k}
\det G_{ik}.
\]
Andererseits lassen sich auch die Einträge $g^{ik}$ der inversen
Matrix als
\[
g^{ik} = \frac{1}{g} (-1)^{i+k} \det G_{ik}
\]
schreiben.
Es folgt, dass
\[
\frac{\partial g}{\partial g_{ik}}
=
g\, g^{ik}
\]
ist.
Die Ableitung nach $x^m$ ist daher
\[
\frac{\partial g}{\partial x^m}
=
\sum_{i,k=1}^n g\,g^{ik} \frac{\partial g_{ik}}{\partial x^m}.
\]
Diese Eigenschaft lässt sich auch als das Differential
\begin{equation}
dg
=
g\,g^{ik}\,dg_{ik}
\label{buch:hodge:koorinatenfrei:eqn:dg}
\end{equation}
schreiben.

Da $g^{ik}g_{kl}=\delta^i_l$ gilt, ist
\[
g^{ik}g_{ki} = \delta^i_i = n
\]
konstant.
Das Differential ist nach der Kettenregel
\[
g_{ki}\,dg^{ik}
+
g^{ik}\,dg_{ik}
=
0
\qquad\Rightarrow\qquad
g_{ki}\,dg^{ik}
=
-
g^{ik}\,dg_{ik}.
\]
Das Differential $dg$ kann daher ausser in der Form
\eqref{buch:hodge:koorinatenfrei:eqn:dg}
auch in der Form
\begin{equation}
dg
=
g\,g^{ik}\,dg_{ik}
=
-
g\,g_{ik}\,dg^{ik}
\label{buch:hodge:koorinatenfrei:eqn:dg2}
\end{equation}
geschrieben werden.

%
% Ableitung einer Determinanten
%
\subsubsection{Ableitung der Determinanten}
Die explizite Berechnung der Ableitung der Determinangen $g$ in
Komponenten kann auch in für eine beliebige matrixwertige Funktion
\[
A
\colon
\mathbb{R}\to M_{n\times n}(\mathbb{R})
:
t\mapsto A(t)
\]
durchgeführt werden.
Dazu berechnet man
\begin{align*}
\frac{d}{dt}
\det A(t)
&=
\lim_{h\to 0}
\frac{\det A(t+h) - \det A(t)}{h}
\\
&=
\det A(t)
\lim_{h\to 0}
\frac{\det A(t)^{-1}\bigl(\det A(t+h) - \det A(t)\bigr)}{h}
\\
&=
\det A(t)
\lim_{h\to 0}
\frac{\det(A(t)^{-1} A(t+h)) - 1)}{h}.
\intertext{Der Grenzwert auf der rechten Seite ist die Ableitung 
der Matrixfunktion $B(h)=A(t)^{-1}A(t+h)$ an der Stelle $h=0$
mit der Eigenschaft $B(0)=I$.}
&=
\det A(t)
\frac{d}{dh} \det(A(t)^{-1} A(t+h))\bigg|_{h=0}
\intertext{Eine solche Ableitung ist die Spur, es folgt daher, dass die
Ableitung von $\det A(t)$}
&=
\det A(t) \operatorname{tr} \frac{d}{dh}A(t)^{-1}A(t+h)\bigg|_{h=0}
\\
&=
\det A(t) \operatorname{tr}\biggl( A(t)\frac{dA(t)}{dt}\biggr).
\intertext{In Komponenten ausgeschrieben ist dies}
\frac{d}{dt}\det A(t)
&=
\det A(t)
\sum_{i,k=1}^n
a_{ik}(t) a_{ki}'(t),
\end{align*}
was für eine symmetrische Matrix mit den für $g$ gefundenen Formeln
übereinstimmt.


%
% Skalarprodukt der 1-Formen
%
\subsubsection{Skalarprodukt der 1-Formen}
Der metrische Tensor beschreibt das Skalarprodukt von Tangentialvektoren,
insbesondere ist
\[
\biggl\langle
\frac{\partial}{\partial x^i},\frac{\partial}{\partial x^k}
\biggr\rangle
=
g_{ik}.
\]
Benötigt wird jetzt das Skalarprodukt von 1-Formen.
Die Basisformen $dx^k$ sind dual zu den Basisvektoren, es gilt
\[
\biggl\langle dx^i,\frac{\partial}{\partial x^k}\biggr\rangle
=
\delta_{ik}.
\]
Das Skalarprodukt zweier beliebigen 1-Formen $a_i\,dx^i$ und
$b_k\,dx^k$ ist eine bilineare Funktion der Koeffizienten
$a_i$ und $b_i$, muss also durch einen kontravarianten 
Tensor $g^{ik}$ vermittelt werden.
Die Einträge $g^{ik}$ bilden eine Matrix, die zur Matrix $(g_{ik})$
invers ist.

%
% Skalarprodukt der p-Formen
%
\subsubsection{Skalarprodukt der $p$-Formen}
Um ein Skalarprodukt von $p$-Formen zu definieren, müssen in einem
Koordinatensystem die Skalarprodukt beliebiger Basis-$p$-Formen
\[
g^{i_1\dots i_pk_1\dots k_p}
=
\langle
dx^{i_1}\wedge\dots\wedge dx^{i_p}
,
dx^{k_1}\wedge\dots\wedge dx^{k_p}
\rangle
\]
bestimmt werden.
Falls zwei Indizes gleich sind, verschwindet das Skalarprodukt.
Vertauscht man zwei verschiedene Indizes in einer Basis-$p$-Form,
kehrt das Vorzeichen.
Eine bilineare Funktion mit diesen Eigenschaften ist ein Vielfaches
einer Bilinearform, die nach dem Muster der Gram-Determinante 
konstruiert wird.
Es muss
\[
g^{i_1\dots i_pk_1\dots k_p}
=
\left|
\begin{matrix}
\langle dx^{i_1}, dx^{k_1} \rangle
	&\langle dx^{i_1}, dx^{k_2} \rangle
	&\dots
	&\langle dx^{i_1}, dx^{k_p} \rangle
\\
\langle dx^{i_2}, dx^{k_1} \rangle
	&\langle dx^{i_2}, dx^{k_2} \rangle
	&\dots
	&\langle dx^{i_2}, dx^{k_p} \rangle
\\[-2pt]
\vdots
	&\vdots
	&\ddots
	&\vdots
\\
\langle dx^{i_p}, dx^{k_1} \rangle
	&\langle dx^{i_p}, dx^{k_2} \rangle
	&\dots
	&\langle dx^{i_p}, dx^{k_p} \rangle
\end{matrix}
\right|
\]
gesetzt werden.

%
% Definition des Hodge-Operators
%
\subsection{Definition des Hodge-Operators}
In den vorangegangenen Abschnitten wurde das Skalarprodukt auf
beliebigen $p$-Formen definiert.
Damit wird es jetzt möglich, den Hodge-Operator zu definieren.

\begin{definition}[Hodge-Operator]
\label{buch:hodge:koordinatenfrei:def:hodge-operator}
Der Hodge-Operator auf einer differenzierbaren Mannigfaltigkeit
mit einer Metrik ist eine lineare Abbildung, die $p$-Formen in
$(n-p)$-Formen abbildet.
Der Hodge-Operator $\ast\omega$ einer $p$-Form $\omega\in\Omega^p$
ist durch die Bedingung
\begin{equation}
\alpha \wedge (\ast\omega) = \langle \alpha,\omega\rangle \nu
\label{buch:hodge:koordinatenfrei:eqn:definition}
\end{equation}
für alle $p$-Formen $\alpha\in\Omega^p$ definiert.
\end{definition}

Die Existenz der $(n-p)$-Form $\ast\omega$ wird durch den
Satz~\ref{buch:hodge:skalarprodukt:satz:endlich} sichergestellt.
Satz~\ref{buch:hodge:satz:eindeutigkeit} garantiert, dass die
Bedingung~\eqref{buch:hodge:koordinatenfrei:eqn:definition}
die $(n-p)$-Form $\ast\omega$ eindeutig definiert ist.
Beide Sätze sind Koordinatenunabhängige Aussagen, sie verwenden
ein Koordinatensystem höchstens für den Beweis.
Damit legt die
Definition~\ref{buch:hodge:koordinatenfrei:def:hodge-operator}
den Hodge-Operator koordinatenfrei fest.

%
% Berechnung des Hodge-Operators
%
\subsection{Berechnung des Hodge-Operators}
Um den Hodge-Operator in einem beliebigen Koordinatensystem zu
bestimmen, ist wie folgt vorzugehen.
\begin{enumerate}
\item
Bestimmung der Metrik, die für Volumenform und Skalarprodukt
von $p$-Formen verwendet werden sollen.
\item
Bestimmung der Volumenform zu dieser Basis.
\item
Bestimmung des Skalarproduktes für 1-Formen.
\item
Erweiterung des Skalarproduktes auf $p$-Formen mit $p>1$.
\item
Berechnung des Hodge-Operators für Basis-$p$-Formen.
\end{enumerate}

%
% Hodge-Operator in Polarkoordinaten
%
\subsubsection{Hodge-Operator in Polarkoordinaten}
\index{Hodge-Operator!in Polarkoordinaten}%
Wir betrachten Polarkoordinaten $(r,\varphi)$ auf der Ebene $\mathbb{R}^2$.
Die Umrechnung von Polarkoordinaten in kartesische Koordinaten ist durch
\begin{align*}
x^1 &= r\cos\varphi \\
x^2 &= r\sin\varphi
\end{align*}
gegeben.
Die Differentiale sind
\begin{align*}
dx^1 &= \cos\varphi\,dr - r\sin\varphi\,d\varphi \\
dx^2 &= \sin\varphi\,dr + r\cos\varphi\,d\varphi.
\end{align*}
Das Gleichungssystem kann man auch nach den Differentialen $dr$
und $d\varphi$ auflösen und erhalt
\begin{align*}
dr       &= \cos\varphi\, dx^1 + \sin\varphi\, dx^2 \\
d\varphi &= -\frac1r \sin\varphi\,dx^1 + \frac1r\cos\varphi\,dx^2.
\end{align*}
Zur Berechnung des Hodge-Operators gehen wir nach dem Rezept des 
vorangegangenen Abschnitts vor.
\begin{enumerate}
\item
{\bf Metrik:}
In kartesischen Koordinaten ist die Metrik in kartesichen Koordinaten
ist $g=(dx^1)^2+(dx^2)^2$, was in Polarkoordinaten zu
\begin{align*}
g
&=
(\cos\varphi\,dr - r\sin\varphi\,d\varphi)^2
+
(\sin\varphi\,dr + r\cos\varphi\,d\varphi)^2
\\
&=
\cos^2\varphi\,(dr)^2
-
2r\sin\varphi\cos\varphi\,dr\,d\varphi
+
r^2\sin^2\varphi\,(d\varphi)^2
\\
&+
\sin^2\varphi\,(dr)^2
+
2r\cos\varphi\sin\varphi\,dr\,d\varphi
+
r^2\cos^2\varphi\,(d\varphi)^2
\\
&=
(dr)^2 + r^2(d\varphi)^2.
\end{align*}
Der metrische Tensor in Matrixform ist
\[
g_{ik}
=
\begin{pmatrix}
1&0\\
0&r^2
\end{pmatrix}.
\]
\item {\bf Volumenform:}
Die Volumenform in Polarkoordinaten ist
\begin{align*}
\nu
=
dx^1\wedge dx^2
&=
(\cos\varphi\,dr - r\sin\varphi\,d\varphi)
\wedge
(\sin\varphi\,dr + r\cos\varphi\,d\varphi)
\\
&=
r \cos^2\varphi\,dr \wedge d\varphi
+
r \sin^2\varphi\,dr \wedge d\varphi
\\
&= r \,dr\wedge d\varphi.
\end{align*}
\item{\bf Skalarprodukt von $1$-Formen:}
Für das Skalarprodukt der 1-Formen muss die inverse Matrix
\[
g^{ik}
=
\begin{pmatrix}
1&0\\
0&\frac{1}{r^2}
\end{pmatrix}
\]
verwendet werden.
Die Determinante des metrischen Tensors ist $\det g = r^2$.
Die Basisformen $dr$ und $d\varphi$ sind orthogonal und haben die
Skalarprodukte
\begin{align*}
\langle dr,dr\rangle
&=
1
&
\langle d\varphi,d\varphi\rangle
&=
\frac{1}{r^2}.
\end{align*}
\item{\bf Skalarprodukt von $2$-Formen:}
Das Skalarprodukt von $dr\wedge d\varphi$ mit sich selbst wird durch die
Gram-Determinante
\[
\langle dr\wedge d\varphi,dr\wedge d\varphi\rangle
=
\left|
\begin{matrix}
\langle dr,dr\rangle         & \langle dr,d\varphi\rangle \\
\langle d\varphi, dr \rangle & \langle d\varphi,d\varphi\rangle
\end{matrix}
\right|
=
\left|
\begin{matrix}
1&0\\
0&\frac{1}{r^2}
\end{matrix}
\right|
=
\frac{1}{r^2}
\]
gegeben.
\item{\bf Hodge-Operator:}
Für 1-Formen ergibt die Definition
\[
\renewcommand{\arraycolsep}{1.5pt}
\begin{array}{rcrcrclcrclclcrcl}
1\,\nu
&=&
\langle dr,dr\rangle\,\nu
&=&
dr &\wedge& {\ast\,dr}
&=&
dr &\wedge& a(r,\varphi)\,d\varphi
&=&
\phantom{-}a(r,\varphi)
\frac1r
\, \nu
&\;\Rightarrow\;&
\ast\,dr
&=&
r\,d\varphi
\\
\frac{1}{r^2}\nu
&=&
\langle d\varphi,d\varphi\rangle\,\nu
&=&
d\varphi &\wedge& {\ast\,d\varphi}
&=&
d\varphi &\wedge& b(r,\varphi)\,dr
&=&
-b(r,\varphi)\frac{1}{r}\,\nu
&\;\Rightarrow\;&
\ast\,d\varphi
&=&
-\frac1r\, dr.
\end{array}
\]
Zur Kontrolle berecchnen wir den iterierten Hodge-Operator:
\[
\renewcommand{\arraycolsep}{1.5pt}
\begin{array}{rclclcl}
\ast{\ast\,dr}
&=&
\ast(r\,d\varphi)
&=&
r(-\frac1r\,dr)
&=&
-dr
\\
\ast{\ast\,d\varphi}
&=&
\ast(-\frac1{r}\,dr)
&=&
-\frac1r(r\, d\varphi)
&=&
-d\varphi.
\end{array}
\]
Der Hodge-Operator der 2-Form $dr\wedge d\varphi$ ist eine Funktion
$\ast(dr\wedge d\varphi)=c(r,\varphi)$, gemäss Definition muss gelten
\begin{align*}
dr\wedge d\varphi\wedge c(r,\varphi)
&=
c(r,\varphi)\,dr\wedge d\varphi
\\
&=
\langle dr\wedge d\varphi,dr\wedge d\varphi\rangle \nu
\\
&=
\frac{1}{r^2} r\,dr\wedge d\varphi
&&\Rightarrow&
c(r,\varphi)
&=
\ast(dr\wedge d\varphi)
=
\frac1r.
\end{align*}
Schliesslich ist $\ast 1$ die Zweiform $d(r,\varphi)\,dr\wedge d\varphi$,
nach Definition folgt
\begin{align*}
1\wedge{\ast 1}
&=
1\wedge d(r,\varphi)\,dr\wedge d\varphi
\\
&=
\langle 1,1\rangle \nu
=
r\,dr\wedge d\varphi
&&\Rightarrow&
d(r,\varphi)
&=
r
&&\Rightarrow&
\ast 1
&=
r\,dr\wedge d\varphi.
\end{align*}
Auch für diese Kombination berechnen wir den iterierten Hodge-Operator
\[
\renewcommand{\arraycolsep}{1.5pt}
\begin{array}{rclclclcl}
\ast{\ast 1}
&=&
\ast (r\,dr\wedge d\varphi)
&=&
r\,{\ast(dr\wedge d\varphi)}
&=&
r\cdot\frac1r
&=&
1
\\
\ast{\ast(dr\wedge d\varphi)}
&=&
\ast(\frac1r)
&=&
\frac1r\cdot({\ast\, 1})
&=&
\frac1r(r\,dr\wedge d\varphi)
&=&
dr\wedge d\varphi.
\end{array}
\]
In beiden Fällen ist der iterierte Hodge-Operator die identische
Abbildung.
\end{enumerate}

In kartesischen Koordinaten haben wir den Hodge-Operator schon früher
berechnet.
Durch Umrechnung der Formeln für Polarkoordinaten in kartesische Koordinaten
müssten sich die Formeln von Beispiel~\ref{buch:hodge:hodge:beispiel:r2} ergeben.
\begin{align*}
\ast dx^1
&=
\cos\varphi (\ast dr) - r\sin\varphi(\ast d\varphi)
\\
&=
\cos\varphi (r\,d\varphi) - r\sin\varphi(-{\textstyle\frac1r}\,dr)
\\
&=
r\cos\varphi d\varphi + \sin\varphi\,dr
\\
&=
r\cos\varphi(-{\textstyle\frac1r}\sin\varphi\,dx^1 + {\textstyle\frac1r}\cos\varphi\, dx^2)
+
\sin\varphi(\cos\varphi\,dx^1+\sin\varphi\,dx^2)
\\
&=
dx^2,
\\
\ast dx^2
&=
\sin\varphi\,(\ast dr) + r\cos\varphi\,(\ast d\varphi)
\\
&=
r\sin\varphi\,d\varphi + r\cos\varphi (-{\textstyle\frac1r})\,dr
\\
&=
r\sin\varphi (-{\textstyle\frac1r}\sin\varphi\,dx^1+{\textstyle\frac1r}\cos\varphi\,dx^2)
+
r\cos\varphi(-{\textstyle\frac1r})(\cos\varphi\,dx^1+\sin\varphi\,dx^2)
\\
&=
-dx^1,
\end{align*}
in Übereinstimmung mit den Resultaten von
Beispiel~\ref{buch:hodge:hodge:beispiel:r2}.

%
% Hodge-Operator in Kugelkoordinaten
%
\subsubsection{Hodge-Operator in Kugelkoordinaten}
\index{Hodge-Operator!in Kugelkoordinaten}%
Kugelkoordinaten $(r,\vartheta,\varphi)$ sind gegeben durch die
Umrechnungsformeln
\begin{equation}
\begin{aligned}
dx^1 &= r \sin\vartheta \cos\varphi \\
dx^2 &= r \sin\vartheta \sin\varphi \\
dx^3 &= r \cos\vartheta
\end{aligned}
\label{buch:hodge:skalarprodukt:eqn:kugelkoordinaten}
\end{equation}
mit den Differentialen
\begin{align*}
dx^1
&=
\sin\vartheta \cos\varphi \,dr
+
r \cos\vartheta \cos\varphi \,d\vartheta
-
r \sin\vartheta \sin\varphi \,d\varphi
\\
dx^2
&=
\sin\vartheta \sin\varphi \,dr
+
r \cos\vartheta \sin\varphi \,d\vartheta
+
r \sin\vartheta \cos\varphi \,d\varphi
\\
dx^3
&=
\cos\vartheta\,dr
-
r \sin\vartheta\,d\vartheta.
\end{align*}
Wie im vorangegangenen Abschnitt soll der Hodge-Operator in
Kugelkoordinaten berechnet werden.
\begin{enumerate}
\item {\bf Metrik:}
Wie bei den Polarkoordinaten kann die Metrik direkt aus 
\eqref{buch:hodge:skalarprodukt:eqn:kugelkoordinaten}
gewinnen.
\begin{align*}
g
&=
(dx^1)^2 + (dx^2)^2 + (dx^3)^2
\\
&=\phantom{+}
(
  \cos\varphi \sin\vartheta \,dr
+
r \cos\varphi \cos\vartheta \,d\vartheta
-
r \sin\varphi \sin\vartheta \,d\varphi
)^2
\\
&\phantom{=}\mathstrut+
(
  \sin\varphi \sin\vartheta \,dr
+
r \cos\varphi \sin\vartheta \,d\varphi
+
r \sin\varphi \cos\vartheta \,d\vartheta
)^2
\\
&\phantom{=}\mathstrut+
(
  \cos\vartheta \,dr
-
r \sin\vartheta \,d\vartheta
)^2
\\
&=\mathstrut\phantom{+}
\cos^2\varphi \sin^2\vartheta \, (dr)^2
+
r^2 \cos^2\varphi \cos^2\vartheta \, (d\vartheta)^2
+
r^2 \sin^2\varphi \sin^2\vartheta \, (d\varphi)^2
\\
&\phantom{=}\mathstrut-
2r \cos\varphi \sin\varphi \sin^2\vartheta \,dr\,d\varphi
+
2r \cos^2\varphi \sin\vartheta \cos\vartheta \,dr\,d\vartheta
\\
&\phantom{=}\mathstrut-
2r^2 \sin\varphi \cos\varphi \sin\vartheta \cos\vartheta \,d\varphi\,d\vartheta
\\
&\phantom{=}\mathstrut+
\sin^2\varphi\sin^2\vartheta \,(dr)^2
+
r^2\sin^2\varphi\cos^2\vartheta \,(d\vartheta)^2
+
r^2\cos^2\varphi\sin^2\vartheta \,(d\varphi)^2
\\
&\phantom{=}\mathstrut+
2r\sin\varphi\cos\varphi\sin^2\vartheta \,dr\,d\varphi
+
2r\sin^2\varphi\sin\vartheta\cos\vartheta \,dr\,d\vartheta
\\
&\phantom{=}\mathstrut+
2r^2\cos\varphi\sin\varphi\sin\vartheta\cos\vartheta \,d\varphi\,d\vartheta
\\
&\phantom{=}\mathstrut+
   \cos^2\vartheta \,(dr)^2
+
r^2 \sin^2\vartheta \,(d\vartheta)^2
-
2r \cos\vartheta \sin\vartheta \,dr\,d\vartheta
\\
&=
(dr)^2
+
r^2\, (d\vartheta)^2
+
r^2\sin^2\vartheta\,(d\varphi)^2
\end{align*}
Die Matrix des metrischen Tensors ist
\[
g
=
\begin{pmatrix}
1 &  0  &         0           \\
0 & r^2 &         0           \\
0 &  0  & r^2 \sin^2\vartheta
\end{pmatrix}
\]
mit der Determinante $\det g=r^4\sin^2\vartheta$ und der inversen Matrix
\[
g^{-1}
=
\begin{pmatrix}
1 & 0 & 0 \\
0 & \frac1{r^2} & 0 \\
0 & 0 & \frac{1}{r^2\sin^2\vartheta}
\end{pmatrix}.
\]
\item {\bf Volumenform:}
Die Volumenform ist
\begin{align*}
\nu
&=
dx^1\wedge dx^2 \wedge dx^3
\\
&=
\bigl(
r^2 \cos^2\vartheta \sin\vartheta (\sin^2\varphi+\cos^2\varphi)
+
r^2 \sin^3\vartheta (\cos^2\varphi +\sin^2\varphi)
\bigr)\,
dr\wedge d\vartheta \wedge d\varphi
\\
&=
r^2 \sin\vartheta
(
\cos^2\vartheta
+
\sin^2\vartheta
)
\,dr\wedge d\vartheta \wedge d\varphi
\\
&=
r^2 \sin\vartheta\, dr\wedge d\vartheta\wedge d\varphi.
\end{align*}
\item {\bf Skalarprodukt von 1-Formen:}
Für die Metrik der 1-Formen muss die inverse Matrix
des metrischen Tensors verwendet werden.
Es folgen die Skalarprodukte
\begin{align*}
\langle dr,dr\rangle &= 1
&
\langle d\vartheta,d\vartheta\rangle &= \frac{1}{r^2}
&
\langle d\varphi,d\varphi\rangle &= \frac{1}{r^2\sin^2\vartheta}
\end{align*}
von Basis-1-Formen.
Alle anderen Skalarprodukt von Basis-1-Formen verschwinden.

\item {\bf Skalarprodukt von $p$-Formen:}
Für 2-Formen muss die Gram-Determinante verwendet werden:
\begin{align*}
\langle dr\wedge d\vartheta , dr\wedge d\vartheta \rangle
&=
\left|\begin{matrix}
\langle dr, dr \rangle        & \langle dr,d\vartheta \rangle \\
\langle d\vartheta,dr \rangle & \langle d\vartheta, d\vartheta \rangle
\end{matrix}\right|
=
\langle dr, dr \rangle \langle d\vartheta, d\vartheta \rangle
=
\frac1{r^2}
\\
\langle dr\wedge d\varphi , dr\wedge d\varphi \rangle
&=
\left|\begin{matrix}
\langle dr, dr \rangle      & \langle dr,d\varphi \rangle \\
\langle d\varphi,dr \rangle & \langle d\varphi, d\varphi \rangle
\end{matrix}\right|
=
\langle dr, dr \rangle \langle d\varphi, d\varphi \rangle
=
\frac1{r^2\sin^2\vartheta}
\\
\langle d\vartheta\wedge d\varphi , d\vartheta\wedge d\varphi \rangle
&=
\left|\begin{matrix}
\langle d\vartheta, d\vartheta \rangle
	& \langle d\vartheta,d\varphi \rangle \\
\langle d\varphi,d\vartheta \rangle
	& \langle d\varphi, d\varphi \rangle
\end{matrix}\right|
=
\langle d\vartheta, d\vartheta \rangle \langle d\varphi, d\varphi \rangle
=
\frac1{r^4\sin^2\vartheta}
\end{align*}
Alle anderen Skalarprodukte von 2-Formen verschwinden.

Es gibt nur eine 3-Form, das Skalarprodukt ist wieder durch die 
Gram-Determinante
\begin{align*}
\langle
dr\wedge d\vartheta \wedge d\varphi,
dr\wedge d\vartheta \wedge d\varphi
\rangle
&=
\left|
\begin{matrix}
\langle dr, dr \rangle
&\langle dr, d\vartheta \rangle
&\langle dr, d\varphi \rangle
\\
\langle d\vartheta, dr \rangle
&\langle d\vartheta, d\vartheta \rangle
&\langle d\vartheta, d\varphi \rangle
\\
\langle d\varphi, dr \rangle
&\langle d\varphi, d\vartheta \rangle
&\langle d\varphi, d\varphi \rangle
\end{matrix}
\right|
\\
&=
\left|\begin{matrix}
1 & 0 & 0 \\
0 & \frac{1}{r^2} & 0 \\
0 & 0 & \frac{1}{r^2\sin^2\vartheta}
\end{matrix}\right|
=
\frac{1}{r^4\sin^2\vartheta}
\end{align*}
gegeben.
\item {\bf Hodge-Operator:}
Der Hodge-Operator angewendet auf die 1-Form $dv$ für
$v\in\{r,\vartheta,\varphi\}$  ergibt ein Vielfaches der Basis-2-Form,
die $dv$ nicht enthält.
Wir schreiben den Faktor $a_v$, also
z.~B.~$*dr = a_r\,d\vartheta\wedge d\varphi$.
Aus der Bedingung $\omega\wedge {\ast\omega}=\langle\omega,\omega\rangle\,\nu$
entstehen die Gleichungen
\begin{align*}
a_r\,
dr\wedge(d\vartheta\wedge d\varphi)
&=
\langle dr, dr\rangle
r^2\sin\vartheta\, dr\wedge d\vartheta\wedge d\varphi
&&\Rightarrow&
a_r &= r^2 \sin\vartheta
\\
a_\vartheta\,
d\vartheta \wedge(dr \wedge d\varphi)
&=
\langle d\vartheta,d\vartheta\rangle
r^2\sin\vartheta\,dr\wedge d\vartheta\wedge d\varphi
&&\Rightarrow&
a_\vartheta &= -\sin\vartheta
\\
a_\varphi\,
d\varphi \wedge(dr \wedge d\vartheta)
&=
\langle d\varphi,d\varphi\rangle
r^2\sin\vartheta\,dr\wedge d\vartheta\wedge d\varphi
&&\Rightarrow&
a_\varphi &= \frac{1}{\sin\vartheta}.
\end{align*}
Die resultierenden Werte des Hodge-Operators auf 1-Formen sind in
Tabelle~\ref{buch:hodge:skalarprodukt:table:kugelhodge} zusammengestellt.

In der gleichen Art ist der Wert des Hodge-Operators auf einer
Basis-2-Form $\alpha$ ein Vielfaches von $dv$, wenn $v$ in $\alpha$
fehlt.
Wir schreiben den Faktor wieder als $b_v$ und stellen die Gleichungen
$\alpha\wedge {\ast\alpha} = \langle \alpha,\alpha\rangle\,\nu$ auf.
Es ergeben sich 
\begin{align*}
b_r(d\vartheta\wedge d\varphi)\wedge dr
&=
\langle d\vartheta\wedge d\varphi,d\vartheta\wedge d\varphi\rangle
r^2\sin\vartheta\, dr\wedge d\vartheta\wedge d\varphi
&&\Rightarrow&
b_r &= \frac{1}{r^2\sin \vartheta}
\\
b_\vartheta(dr\wedge d\varphi)\wedge d\vartheta
&=
\langle dr \wedge d\varphi,dr \wedge d\varphi\rangle
r^2\sin\vartheta\, dr\wedge d\vartheta\wedge d\varphi
&&\Rightarrow&
b_\vartheta &= -\frac{1}{\sin \vartheta}
\\
b_\varphi(dr\wedge d\vartheta)\wedge d\varphi
&=
\langle dr \wedge d\vartheta,dr \wedge d\vartheta\rangle
r^2\sin\vartheta\, dr\wedge d\vartheta\wedge d\varphi
&&\Rightarrow&
b_\varphi &= \sin \vartheta.
\end{align*}

Es bleibt noch den Hodge-Operator für die Basis-3-Form
$\omega=dr\wedge d\vartheta\wedge d\varphi$ zu bestimmen.
$\ast\omega$ ist eine Zahl $c$, für die
\begin{align*}
\omega\wedge c
&=
c
\,dr\wedge d\vartheta\wedge d\varphi
\\
&=
\langle
dr\wedge d\vartheta\wedge d\varphi,
dr\wedge d\vartheta\wedge d\varphi
\rangle
r^2\sin\vartheta
\,
dr\wedge d\vartheta\wedge d\varphi
\\
&=
\frac{1}{r^2\sin\vartheta}
dr\wedge d\vartheta\wedge d\varphi
\qquad\Rightarrow\qquad
c=\frac{1}{r^2\sin\vartheta}
\end{align*}
gelten muss.
Somit ist
\[
\ast(dr \wedge d\vartheta \wedge d\varphi)
=
\frac{1}{r^2\sin\vartheta}
\qquad\text{und}\qquad
{\ast}1
=
r^2\sin\vartheta\, dr \wedge d\vartheta \wedge d\varphi.
\]
Wie im Falle der Polarkoordinaten kann man nachprüfen, dass der
iterierte Hodge-operator $\ast{\ast}$ die identische Abbildung ist.
\end{enumerate}
%
% table-kugelhodge.tex
%
% (c) 2025 Prof Dr Andreas Müller
%
\begin{table}
\centering
\renewcommand\arraystretch{1.3}
\begin{tabular}{|>{$}c<{$}|>{$}c<{$}||>{$}c<{$}|>{$}c<{$}|}
\hline
\text{1-Form $\omega$} & \ast\omega & \text{2-Form $\alpha$}    & \ast\alpha\\
\hline
dr
& r^2\sin\vartheta\, d\vartheta\wedge d\varphi
& d\vartheta\wedge d\varphi
& \displaystyle \frac{1}{r^2\sin\vartheta}\,dr
\raisebox{9pt}{\mathstrut}
\\[7pt]
d\vartheta
& -\sin\vartheta\, dr\wedge d\varphi
& dr\wedge d\varphi
& \displaystyle -\frac{1}{\sin\vartheta}\,d\vartheta
\\[7pt]
d\varphi
& \displaystyle \frac{1}{\sin\vartheta}\, dr\wedge d\vartheta
& dr\wedge d\vartheta
& \sin\vartheta\, d\varphi
\\[7pt]
\hline
\end{tabular}
\caption{Tabelle der Wirkung des Hodge-Operators auf 1-Formen und 2-Formen
in Kugelkoordinaten.
\label{buch:hodge:skalarprodukt:table:kugelhodge}}
\end{table}



%
% 3-vektoranalysis.tex -- Operatoren der Vektoranalysis
%
% (c) 2025 Prof Dr Andreas Müller
%
\section{Die Operatoren der Vektoranalysis
\label{buch:hodge:section:vektoranlysis}}
\kopfrechts{Die Operatoren der Vektoranalysis}
Die klassische Vektoranalysis definiert dagegen mehr oder weniger
ad hoc eine Reihe von Operatoren auf Vektorfeldern, meistens nur
in kartesischen Koordinaten.
Feldgleichungen für Strömungsfelder oder für das elektromagnetische
Feld können dann mit diesen Operatoren formuliert werden, aber auch
diese nur in kartesischen Koordinaten.
Es entsteht die etwas unbefriedigende Situation, dass man nicht
sicher sein kann, dass die konstruierten Gleichungen Aussagen
darstellen, die nicht Artefakte der Wahl des Koordinatensystems
sind.

Die äussere Ableitung von $p$-Formen wurde als allgemeine Theorie
der offensichtlich koordinatenunabhängig definierbaren
Differentialoperatoren aufgebaut.
Wenn es gelingt, die klassischen Operatoren der Vektoranalysis
durch Operatoren auszudrücken, die koordinatenunabhängig formuliert
werden können, dann weiss man auch, dass die Gleichungen, die man
aus diesen Operatoren konstruiert, unabhängig sind von der Wahl
eines Koordinatensystems.
Gleichzeitig ermöglicht der allgemeine Formalismus, diese Operatoren
für beliebige Koordinatensysteme darzustellen.

Die klassiche Vektoranalysis arbeitet nur in Dimension 3, wir gehen
daher in den nachfolgenden Ausführungen normalerweise von einem
dreidimensionalen Raum aus.
Einige der Konstruktionen funktionieren allerdings auch in $n$.

%
% Gradient
%
\subsection{Gradient}
Die Richtungsableitung einer Funktion $f\colon M\to\mathbb{R}$ in 
\index{Richtungsableitung}%
Richtung des Tangentialvektors $X$ im Punkt $p\in M$ ist gegeben
durch das Differential
\[
X\cdot f(p) = \langle df, X\rangle.
\]
In einer Karte ist
\[
\langle df,X\rangle
=
\frac{\partial f}{\partial x^k}(p)
\cdot X^k
\]
Da dies wie ein Skalarprodukt aussieht, werden in der klassischen
Vektoranalysis die Koordinaten des Differentials mit den Komponenten
eines Vektors identifiziert.
Wir verwenden daher in den nachfolgenden jeweils die Identifikation
\[
V\colon
a_i\,dx^i \mapsto \begin{pmatrix} a_1\\a_2\\a_3\end{pmatrix}.
\]
In dieser Identifikation ist
\[
V(df)
=
V\biggl(
\frac{\partial f}{\partial x^1}\,dx^1
+
\frac{\partial f}{\partial x^2}\,dx^2
+
\frac{\partial f}{\partial x^3}\,dx^3
\biggr)
=
\renewcommand{\arraystretch}{1.8}
\begin{pmatrix}
\displaystyle\frac{\partial f}{\partial x^1}\\
\displaystyle\frac{\partial f}{\partial x^2}\\
\displaystyle\frac{\partial f}{\partial x^3}
\end{pmatrix}
\]
Die Identifikation von Vektoren mit 1-Formen ist nicht nur in 
3 Dimensionen möglich sondern auch für Formen auf $\mathbb{R}^n$.
Der {\em Gradient} als Differentialoperator ist also auch auf $\mathbb{R}^n$
\index{Gradient}%
\index{grad}%
definiert.

%
% Rotation
%
\subsection{Rotation}
Wir betrachten jetzt die äussere Ableitung der 1-Form
$\omega=f_i(x)\,dx^i$.
Sie ist definiert durch
\[
d\omega
=
\frac{\partial f_i}{\partial x^k}\,dx^k\wedge dx^i
\in
\Omega^3(\mathbb{R}^3).
\]
In drei Dimensionen ausgeschrieben ist dies
\begin{align*}
d\omega
&=\phantom{+}
\frac{\partial f_1}{\partial x^1}\,dx^1\wedge dx^1
+
\frac{\partial f_1}{\partial x^2}\,dx^2\wedge dx^1
+
\frac{\partial f_1}{\partial x^3}\,dx^3\wedge dx^1
\\
&
\phantom{=}+
\frac{\partial f_2}{\partial x^1}\,dx^1\wedge dx^2
+
\frac{\partial f_2}{\partial x^2}\,dx^2\wedge dx^2
+
\frac{\partial f_2}{\partial x^3}\,dx^3\wedge dx^2
\\
&
\phantom{=}+
\frac{\partial f_3}{\partial x^1}\,dx^1\wedge dx^3
+
\frac{\partial f_3}{\partial x^2}\,dx^2\wedge dx^3
+
\frac{\partial f_3}{\partial x^3}\,dx^3\wedge dx^3.
\intertext{Die 2-Formen $dx^i\wedge dx^i$ verschwinden alle und die
die 2-Formen mit verschiedenen Faktoren sind zum Teil noch nicht in
der Standardreihenfolge aufsteigender Indizes, wodurch zusätzlich
Vorzeichen auftreten:}
&=-
\frac{\partial f_1}{\partial x^2}\,dx^1\wedge dx^2
-
\frac{\partial f_1}{\partial x^3}\,dx^1\wedge dx^3
\\
&
\phantom{=}+
\frac{\partial f_2}{\partial x^1}\,dx^1\wedge dx^2
-
\frac{\partial f_2}{\partial x^3}\,dx^2\wedge dx^3
\\
&
\phantom{=}+
\frac{\partial f_3}{\partial x^1}\,dx^1\wedge dx^3
+
\frac{\partial f_3}{\partial x^2}\,dx^2\wedge dx^3.
\intertext{Durch Zusammenfassen gleicher Basis-2-Formen erhält man}
&=
\biggl(
\frac{\partial f_3}{\partial x^2}
-
\frac{\partial f_2}{\partial x^3}
\biggr)\,dx^2\wedge dx^3
+
\biggl(
\frac{\partial f_3}{\partial x^1}
-
\frac{\partial f_1}{\partial x^3}
\biggr)\,dx^1\wedge dx^3
+
\biggl(
\frac{\partial f_2}{\partial x^1}
-
\frac{\partial f_1}{\partial x^2}
\biggr)\,dx^1\wedge dx^2.
\end{align*}

$d\omega$ ist eine 2-Form ist, die nicht direkt mit einem Vektor identifiziert
werden kann.
Durch Anwendung des Hodge-Operators kann die 2-Form $d\omega$ in eine 1-Form
verwendelt werden, ohne dass dabei Information verloren geht.
Der Hodge-Operator auf $\mathbb{R}^3$ wurde in
Beispiel~\ref{buch:hodge:hodge:beispiel:r3} berechnet.
Angewendet auf $d\omega$ ergibt sich
\begin{align*}
\ast d\omega
&=
\biggl(
\frac{\partial f_3}{\partial x^2}
-
\frac{\partial f_2}{\partial x^3}
\biggr)\, dx^1
+
\biggl(
\frac{\partial f_1}{\partial x^3}
-
\frac{\partial f_3}{\partial x^1}
\biggr)\,dx^3
+
\biggl(
\frac{\partial f_2}{\partial x^1}
-
\frac{\partial f_1}{\partial x^2}
\biggr)\,dx^3.
\end{align*}
Man beachte den Vorzeichenwechsel beim mittleren Teil.
Diese 1-Form wird durch die Abbildung $V$ mit dem Vektor
\[
V(\ast d\omega)
=
\bgroup
\renewcommand\arraystretch{2.0}
\begin{pmatrix}
\displaystyle
\frac{\partial f_3}{\partial x^2}
-
\frac{\partial f_2}{\partial x^3}
\\
\displaystyle
\frac{\partial f_1}{\partial x^3}
-
\frac{\partial f_3}{\partial x^1}
\\
\displaystyle
\frac{\partial f_2}{\partial x^1}
-
\frac{\partial f_1}{\partial x^2}
\end{pmatrix}
\egroup
=
\operatorname{rot}
\begin{pmatrix}
f_1\\
f_2\\
f_3
\end{pmatrix}
=
\operatorname{rot}\vec{f}
\]
identifiziert.
Dies ist die {\em Rotation} des Vektorfeldes $\vec{f}$.
\index{Rotation}%
\index{rot}%
In der englischsprachigen Literatur wird die Rotation auch als
$\operatorname{curl}\vec{f}$ bezeichnet.
\index{curl}%
Sie kann formal auch als das Vektorprodukt mit dem Nabla-Operator
\[
\nabla\times\vec{f}
=
\operatorname{rot}\vec{f}
\]
geschrieben werden.

%
% Divergenz
%
\subsection{Divergenz
\label{buch:hodge:skalarprodukt:subsection:divergenz}}
Die Rotation wurde durch äussere Ableitung einer 1-Form erhalten.
Es war nötig, die Ableitung durch Anwendung des Hodge-Operators in eine
1-Form zu verwandeln, die dann wieder mit einem Vektor identifiziert
werden konnte.

Um die Ableitung einer 2-Form mit einer Vektoranalysis-Operation
zu identifizieren, muss erst eine 2-Form gewonnen werden.
Ein Vektor kann mit einer 1-Form identifiziert werden, die durch
den Hodge-Operator in eine 2-Form umgewandelt werden kann.
Wir beginnen daher mit dem Vektor $\vec{f}=V(f_i\,dx^i)=V(\omega)$
und wenden den Hodge-Operator darauf an:
\begin{align*}
\ast(f_i\,dx^i)
&=
f_1\, dx^2\wedge dx^3
-
f_2\, dx^1\wedge dx^3
+
f_3\, dx^1\wedge dx^2
\end{align*}
Bei der Berechnung der äusseren Ableitung kommen zu den Basis-2-Formen
weitere Basis-1-Formen hinzu, jedoch gibt es in jedem Term immer nur eine
einzige Basis-1-Form, die das Wedge-Produkt nicht zu 0 macht.
Daher ist die äusser Ableitung
\begin{align*}
d{\ast\omega}
&=
\frac{\partial f_1}{\partial x^1}dx^1\wedge dx^2\wedge dx^3
-
\frac{\partial f_2}{\partial x^2}dx^2\wedge dx^1\wedge dx^3
+
\frac{\partial f_3}{\partial x^3}dx^3\wedge dx^1\wedge dx^2
\intertext{Die 3-Formen auf der rechten Seite müssen in die
Standardreihenfolge gebracht werden und nehmen bei einer ungeraden
Anzahl Vertauschungen ein negatives Vorzeichen auf:}
&=
\biggl(
\frac{\partial f_1}{\partial x_1}
+
\frac{\partial f_2}{\partial x_2}
+
\frac{\partial f_3}{\partial x_3}
\biggr)
\,
dx^1\wedge dx^2\wedge dx^3.
\intertext{Dies ist eine 3-Form, in der klassischen Vektoranalysis gibt
es aber nur Funktionen und Vektoren.
Daher wird jetzt nochmals der Hodge-Operator angewendet, um die 3-Form
in eine Funktion zu verwandeln.
So entsteht die Gleichung}
\ast{d{\ast\omega}}
&=
\frac{\partial f_1}{\partial x_1}
+
\frac{\partial f_2}{\partial x_2}
+
\frac{\partial f_3}{\partial x_3}.
\end{align*}
Diese Funktion ist auch als die {\em Divergenz}
\index{Divergenz}%
\index{div}%
\begin{equation}
\operatorname{div}\vec{f}
=
\sum_{i=1}^n \frac{\partial f_i}{\partial x^i}
\label{buch:hodge:hodge:divergenz:eqn:divdef}
\end{equation}
bekannt.
Formal kann sie auch als Skalarprodukt
\[
\operatorname{div}\vec{f}
=
\nabla\cdot\vec{f}
\]
mit dem Nabla-Operator geschrieben werden.

Die Divergenz eines Vektorfeldes ist also die Funktion, die durch
Anwendung des Operators $*{d}*$ auf die zugehörige 1-Form entsteht.
Dieser Operator funktioniert aber auch auf 1-Formen in $n$ Dimensionen.
Der Hodge-Operator macht aus einer 1-Form auf $\mathbb{R}^n$ eine
$n-1$-Form.
Die äussere Ableitung davon ist eine $n$-Form, die der Hodge-Operator
wieder zu einer Funktion macht.
Auch die Formel \eqref{buch:hodge:hodge:divergenz:eqn:divdef} gilt
für bliebige $n$.
%
% table-operatoren.tex
%
% (c) 2025 Prof Dr Andreas Müller
%
\begin{table}
\centering
\begin{tabular}{|>{$}c<{$}|>{$}c<{$}|>{$}c<{$}|}
\hline
\text{Vektoranalsis} & \text{$p$-Formen} & \text{Nabla}
\\
\hline
\operatorname{grad} &  d  & \nabla 
\\
\operatorname{rot}  & *d  & \nabla\times\mathstrut
\\
\operatorname{div}  & *d* & \nabla\cdot\mathstrut
\\
\hline
\end{tabular}
\caption{Korrespondenz wischen Operatoren der klassischen Vektoranalysis,
den Kombinationen von Hodge-Operator und äusserer Ableitung und der
Schreibweise der Operatoren der Vektoranalysis mit dem Nabla-Operator.
\label{buch:hodge:hodge:table:operatoren}}
\end{table}
%

%
% Operatorrelationen
%
\subsection{Operatorrelationen}
Die äussere Ableitung ist nilpotent vom Grad 2, d.~h.~die iterierte
äussere Ableitung $d^2=0$ verschwindet.
Zusammen mit den Identifikationen mit Vektorfeldern ergeben sich
daraus Kombinationen von Operatoren der Vektoranalysis, die
0 ergeben.

%
% Rotation eines Gradienten
%
\subsubsection{Rotation eines Gradienten}
Der Gradient einer Funktion $f$ ist
\[
\operatorname{rot}\operatorname{grad}f
=
*{d}df
=
*d^2f
=
0
\]
\index{rot grad}%
verschwindet, weil der äussere Differentialoperator nilpotent mit
Ordnung $2$ ist.

%
% Divergenz einer Rotation
%
\subsubsection{Divergenz einer Rotation}
Die Divergenz der Rotation eines Vektorfeldes ist
\[
\operatorname{div}\operatorname{rot} \vec{f}
=
*{d}{*}{*}d V^{-1}(\vec{f})
=
*d^2 V^{-1}(\vec{f})
=
0,
\]
\index{div rot}%
wobei beim zweiten Gleichheitszeichen die Involutionseigenschaft
des Hodge-Operators und beim dritten die Nilpotenz der äusseren Ableitung
verwendet wird.

%
% Rotation einer Rotation
%
\subsubsection{Rotation einer Rotation}
Die Verkettungen $\operatorname{rot}\operatorname{grad}$ und
$\operatorname{div}\operatorname{rot}$ war einfach auszurechnen,
weil in der Darstellung mit Hilfe der äusseren Ableitung und des
Hodge-Operators jeweils zwei äussere Ableitungen verkettet werden,
die zusammen die Nullabbildung ergeben.
Die Verketten $\operatorname{rot}\operatorname{rot}$ wird in der
Darstellung durch $p$-Formen zu
\index{rot rot}%
\[
\operatorname{rot}\operatorname{rot}
=
*{d}{*}d,
\]
die im Allgemeinen nicht verschwindet.

In der Tat können wir sie in Vektorform sofort ausrechnen.
Anwendung der früher hergeleiteten Koordinaten-Formeln liefern
\bgroup
\renewcommand{\arraystretch}{1.9}
\begin{align}
\operatorname{rot}\operatorname{rot}\vec{f}
&=
\begin{pmatrix}
\displaystyle
\frac{\partial}{\partial x^1}\\
\displaystyle
\frac{\partial}{\partial x^2}\\
\displaystyle
\frac{\partial}{\partial x^3}
\end{pmatrix}
\begin{pmatrix}
\displaystyle
\frac{\partial f_3}{\partial x^2} - \frac{\partial f_2}{\partial x^3}\\
\displaystyle
\frac{\partial f_1}{\partial x^3} - \frac{\partial f_3}{\partial x^1}\\
\displaystyle
\frac{\partial f_2}{\partial x^1} - \frac{\partial f_1}{\partial x^2}
\end{pmatrix}
\notag
\\
&=
\begin{pmatrix}
\displaystyle
\frac{\partial}{\partial x^2}
\biggl(
\frac{\partial f_2}{\partial x^1} - \frac{\partial f_1}{\partial x^2}
\biggr)
-
\frac{\partial}{\partial x^3}
\biggl(
\frac{\partial f_1}{\partial x^3} - \frac{\partial f_3}{\partial x^1}
\biggr)
\\
\displaystyle
\frac{\partial}{\partial x^3}
\biggl(
\frac{\partial f_3}{\partial x^2} - \frac{\partial f_2}{\partial x^3}
\biggr)
-
\frac{\partial}{\partial x^1}
\biggl(
\frac{\partial f_2}{\partial x^1} - \frac{\partial f_1}{\partial x^2}
\biggr)
\\
\displaystyle
\frac{\partial}{\partial x^1}
\biggl(
\frac{\partial f_1}{\partial x^3} - \frac{\partial f_3}{\partial x^1}
\biggr)
-
\frac{\partial}{\partial x^2}
\biggl(
\frac{\partial f_3}{\partial x^2} - \frac{\partial f_2}{\partial x^3}
\biggr)
\end{pmatrix}
\notag
\\
&=
\begin{pmatrix}
\displaystyle
\frac{\partial^2 f_2}{\partial x^1\,\partial x^2}
-
\frac{\partial}{\partial x^2}
\frac{\partial^2 f_1}{\partial (x^2)^2}
-
\frac{\partial^2 f_1}{\partial (x^3)^2}
+
\frac{\partial^2 f_3}{\partial x^1\,\partial x^3}
\\
\displaystyle
\frac{\partial^2 f_3}{\partial x^2\,\partial x^3}
-
\frac{\partial^2 f_2}{\partial (x^3)^2}
-
\frac{\partial^2 f_2}{\partial (x^1)^2}
+
\frac{\partial^2 f_1}{\partial x^2\,\partial x^1}
\\
\displaystyle
\frac{\partial^2 f_1}{\partial x^3\,\partial x^1}
-
\frac{\partial^2 f_3}{\partial (x^1)^2}
-
\frac{\partial^2 f_3}{\partial (x^2)^2}
+
\frac{\partial^2 f_2}{\partial x^3\,\partial x^2}
\end{pmatrix}.
\notag
\intertext{Nach Vertauschung der partiellen Ableitung finden wir auf
der ersten Zeile zwei Ableitungen nach $x^1$ zwei Termen, die Teil
der Divergenz von $\vec{f}$ sind.
Die verbleibenden Terme sind zweite Ableitungen, die einen Teil
des klassischen Laplace-Operators ergeben.
Um dieses Resultat mit bekannten Operatoren der Vektoranalysis
schreiben zu können, müssen Terme mit entgegengesetzten Vorzeichen
ergänzt werden.
Die neuen Term sind {\color{darkred}rot} eingefügt:}
&=
\begin{pmatrix}
\displaystyle
\frac{\partial}{\partial x^1} \biggl(
{\color{darkred}\frac{\partial f_1}{\partial x^1}
+}
\frac{\partial f_2}{\partial x^2}
+
\frac{\partial f_3}{\partial x^3}
\biggr)
-
\biggl(
{\color{darkred}\frac{\partial^2 f_1}{\partial (x^1)^2}
+}
\frac{\partial^2 f_1}{\partial (x^2)^2}
+
\frac{\partial^2 f_1}{\partial (x^3)^2}
\biggr)
\\
\displaystyle
\frac{\partial}{\partial x^2} \biggl(
\frac{\partial f_1}{\partial x^1}
{\color{darkred}
+
}
\frac{\partial f_2}{\partial x^2}
+
\frac{\partial f_3}{\partial x^3}
\biggr)
-
\biggl(
\frac{\partial^2 f_2}{\partial (x^1)^2}
{\color{darkred}
+
\frac{\partial^2 f_2}{\partial (x^2)^2}
}
+
\frac{\partial^2 f_2}{\partial (x^3)^2}
\biggr)
\\
\displaystyle
\frac{\partial}{\partial x^3} \biggl(
\frac{\partial f_1}{\partial x^1}
+
\frac{\partial f_2}{\partial x^2}
{\color{darkred}
+
\frac{\partial f_3}{\partial x^3}
}
\biggr)
-
\biggl(
\frac{\partial^2 f_3}{\partial (x^1)^2}
+
\frac{\partial^2 f_3}{\partial (x^2)^2}
{\color{darkred}
+
\frac{\partial^2 f_3}{\partial (x^3)^2}
}
\biggr)
\end{pmatrix}.
\notag
\intertext{Die Klammerausdrücke können jetzt mit den bekannten
Operatoren der Vektoranalysis geschrieben werden und ergeben}
\operatorname{rot}\operatorname{rot}\vec{f}
&=
\begin{pmatrix}
\displaystyle
\frac{\partial}{\partial x^1} \operatorname{div}\vec{f}
-
\Delta f_1
\\
\displaystyle
\frac{\partial}{\partial x^2} \operatorname{div}\vec{f}
-
\Delta f_2
\\
\displaystyle
\frac{\partial}{\partial x^3} \operatorname{div}\vec{f}
-
\Delta f_3
\end{pmatrix}
=
\operatorname{grad}\operatorname{div}\vec{f} - \Delta \vec{f}
=
\nabla(\nabla\cdot\vec{f}) - (\nabla\cdot\nabla)\vec{f}.
\label{buch:vektoranalysis:eqn:rotrot}
\end{align}
\egroup
Darin ist $\Delta$ der klassische Laplace-Operator
\index{Laplace-Operator}%
\[
\Delta
=
\nabla\cdot\nabla
=
\frac{\partial^2}{\partial (x^1)^2}
+
\frac{\partial^2}{\partial (x^2)^2}
+
\frac{\partial^2}{\partial (x^3)^2}
\]
Die Identität~\eqref{buch:vektoranalysis:eqn:rotrot} wird in
Abschnitt~\ref{buch:hodge:section:kodifferential} als Motivation
für die Definition des Kodifferentials dienen.
In Abschnitt~\ref{buch:hodge:laplace:subsection:rotrot} wird sie
schliesslich verallgemeinert und es wird gezeigt, dass sie nichts
anderes ist als die Definition des Hodge-Laplace-Operators.

%
% Poincaré-Lemma in Vektoranalysis-Schreibweise
%
\subsubsection{Poincaré-Lemma in Vektoranalysis-Schreibweise}
Das Poincaré-Lemma besagt, dass geschlossene Differentialformen auf
$\mathbb{R}^n$ exakt ist.
Eine geschlossene 0-Form ist eine Funktion $f$, deren Differential
$df=0$ verschwindet.
In einer Karte verschwinden alle partiellen Ableitungen, die Funktion
ist konstant.

Für $n=3$ gibt es zwei interessante Fälle, nämlich geschlossene 1-Formen
und geschlossene 2-Formen.
Wir übersetzen die Aussagen des Poincaré-Lemmas für diese beiden
Fälle in die Schreibweise der Vektoranalsis.

Eine geschlossene 1-Form $\omega$ entspricht einem Vektorfeld.
Geschlossen bedeutet, dass die Rotation dieses Vektorfeldes verschwindet.
Nach dem Poincaré-Lemma gibt es eine 0-Form $f$ derart dass
\index{Poincare-Lemma@Poincaré-Lemma}%
$df=\omega$ ist.
Die Identifikation von 1-Formen mit Vektorfeldern besagt also, dass
ein Vektorfeld, dessen Rotation verschwindet, ein Gradientfeld
einer Funktion $f$ ist.

Eine geschlossene 2-Form $\omega$ entspricht einer 1-Form $\ast\omega$,
auf die mit Hilfe des Hodge-Operators angewendet worden ist.
$d\omega=0$ bedeutet, dass die Divergenz des zu $\ast\omega$
gehörigen Vektorfeldes verschwindet.
Da $d\omega=0$ ist, gibt es nach dem Poincaré-Lemma eine 1-Form $\alpha$,
deren äussere Ableitung $d\alpha=\omega$ ist.
Folglich ist 
\[
d\alpha = \omega
\qquad\Rightarrow\qquad
\ast d\alpha = \ast\omega.
\]
Der 1-Form $\beta$ entspricht ein Vektorfeld, und $\ast d\beta$ entspricht
der Rotation dieses Vektorfelds.
Es folgt, dass ein Vektorfeld, dessen Divergenz verschwindet, als
Rotation eines anderen Vektorfeldes geschrieben werden kann.

\begin{satz}[Poincaré-Lemma der Vektoranalsis]
Wenn die Rotation eines Vektorfeldes $\vec{v}$ auf $\mathbb{R}^3$
verschwindet, dann gibt es eine Funktion $f$ mit
$\vec{v}=\operatorname{grad}f$.
Wenn die Divergenz eines Vektorfeldes $\vec{v}$ verschwindet,
dann gibt es ein Vektorveld $\vec{a}$, dessen Rotation
$\operatorname{rot}\vec{a}=\vec{v}$ das Vektorfeld ist.
\end{satz}


%
% 4-kodifferential.tex -- Kodifferential
%
% (c) 2025 Prof Dr Andreas Müller
%
\section{Kodifferential
\label{buch:hodge:section:kodifferential}}
\kopfrechts{Kodifferential}
Der $d$-Operator erhöht den Grad einer Differentialform um $1$.
Aus der Zusammensetzung mit dem Hodge-Operator entsteht ein neuer
Operator, das Kodifferential $\delta$, der den Grad erniedrigt.
Zusammen mit $d$ entstehen damit weitere Möglichkeiten,
koordinatenunabhängige Operatoren zu definieren.

%
% Definition
%
\subsection{Definition}
Bei der Berechnung der Rotation einer Rotation trat die Kombination
${*}d{*}d$ von Differentialoperatoren auf $p$-Formen und dem
Hodge-Operator auf.
Die äussere Ableitung erhöht den Grad, der Teil ${*}d{*}$
verwandelt eine $p$-Form in eine Differentialform vom Grad
\[
n-(\underbrace{(\underbrace{n-p}_{\displaystyle *})+1}_{\displaystyle d})
=
p-1,
\]
der Grad wird also erniedrigt.
Der Operator ${*}d{*}$ ist also ein Differentialoperator, der den
Grad in umgekehrter Richtung im Vergleich zur äusseren Ableitung
verändert.
Daher verdient er eine eigene Definition.

\begin{definition}[Kodifferential]
\label{buch:hodge:kodifferential:def:delta}
Das {\em Kodifferential} ist der lineare Operator
\index{Kodifferential}%
\index{delta@$\delta$}%
\[
\delta
=
(-1)^{p(n-p)+p}
{\ast}d{\ast}
\colon
\Omega^p(M)\to\Omega^{p-1}
:
\omega \mapsto (-1)^{p(n-p)+p}{\ast}d{\ast}\omega
\]
vom Grad $-1$.
\end{definition}

Der Vorzeichenfaktor setzt sich aus zwei Beiträgen zusammen.
Der Faktor $(-1)^{p(n-p)}$ zusammen mit dem Hodge-Operator ${\ast}$
ist der inverse Hodge-Operator: $\ast^{-1} = (-1)^{p(n-p)}{\ast}$.
Der verbleibende Faktor $(-1)^p$ kehrt das Vorzeichen für ungerade
$p$-Formen und ist die zweckmässigere Wahl, wie im Abschnitt über
den Laplace-Operator weiter unten klar werden wird.
Das Kodifferential kann daher auch als
\[
\delta
=
(-1)^p
{\ast}^{-1}d{\ast}
\]
geschrieben werden.

%
% Vektoranalsis und Kodifferential
%
\subsection{Vektoranalysis und Kodifferential}
Kombinationen des Hodge-Operators mit dem Differential reproduzieren
gemäss der Tabelle~\ref{buch:hodge:hodge:table:operatoren} die 
klassischen Operatoren der Vektoranalysis.
Wir erwarten daher, dass diese Operatoren sich noch etwas leichter
durch das Kodifferential ausdrücken lassen.

Die Divergenz eines Vektorfeldes in $\mathbb{R}^3$ ist auf der
zugehörigen 1-Form gegeben durch den Operator ${\ast}d{\ast}$,
der sich von $\delta$ nur um den Faktor $(-1)^{1(3-1)+1}=-1$
unterscheidet.

Am Ende von Abschnitt~\ref{buch:hodge:skalarprodukt:subsection:divergenz}
wurde darauf hingewiesen, dass die Divergenz auch für Vektorfelder
auf $\mathbb{R}^n$ durch ${\ast}d{\ast}$ berechnet wird.
In diesem Fall ist der Unterschied ein Vorzeichenfaktor der Form
$(-1)^{1\cdot(n-1)+1}=(-1)^n$.

%
% Poincaré-Lemma für das Kodifferential
%
\subsection{Poincaré-Lemma für das Kodifferential}
Auch für das Kodifferential gilt ein Poincaré-Lemma.

\begin{satz}[Poincaré-Lemma für das Kodifferential]
Wenn $\omega \in \Omega^p(\mathbb{R}^n)$ eine $p$-Form ist mit
$\delta\omega=0$, dann gibt es eine $p+1$-Form
$\alpha\in\Omega^{p+1}(\mathbb{R}^n)$ mit $\delta\alpha=\omega$.
\end{satz}

\begin{proof}
Nach Voraussetzung ist $\delta \omega = {\ast}d{\ast}\omega = 0$.
Da der Hodge-Operator invertierbar ist, muss auch $d{\ast}\omega=0$
sein.
Die $(n-p)$-Form $\ast\omega$ ist geschlossen, nach dem Poincaré-Lemma
gibt es eine $(n-p-1)$-Form $\beta$ mit $d\beta=\ast\omega$.
Durch erneute Anwendung des Hodge-Operators wird daraus
\[
{\ast}d{\ast} {\ast}\beta = (-1)^{p(n-p)}\omega.
\]
Setzen wir $\alpha=(-1)^{p(n-p)}\beta$, dann folgt
\[
\delta \alpha = \omega,
\]
womit das Poincaré-Lemma für das Kodifferential bewiesen ist.
\end{proof}


%
% 5-laplace.tex -- Laplace-Operator
%
% (c) 2025 Prof Dr Andreas Müller
%
\section{Der Laplace-Operator}
\kopfrechts{Der Laplace-Operator}
Das Kodifferential $\delta$ reduziert die Ordnung einer Differentialform,
während $d$ die Ordnung erhöht.
Die Zusammensetzung dieser beiden Operatoren ist ein Differentialoperator
zweiter Ordnung, die die Ordnung einer Differentialform nicht ändert.
Um diese Operatoren besser zu verstehen, berechnen wir $\delta d$ und
$d\delta$ in einigen Beispielen.

\begin{beispiel}
Die Operatoren $\delta d$ und $d\delta$ auf $\Omega^*(\mathbb{R}^2)$.
Eine 0-Form ist eine Funktion $f$, $\delta f={\ast}d{\ast}f =0$, weil
$*f$ eine $n$-Form ist, aber $d$ macht alle $n$-Formen zu 0.
Daher ist auch $d\delta=0$ auf 0-Formen.
Umgekehrt ist $\delta d=0$ auf $n$-Formen.

Auf der Funktion $f$ ist 
\begin{align*}
\delta d f
&=
{\ast}d{\ast}
\biggl(
\frac{\partial f}{\partial x^1}dx^1
+
\frac{\partial f}{\partial x^2}dx^2
\biggr)
\\
&=
{\ast}d
\biggl(
\frac{\partial f}{\partial x^1}dx^2
-
\frac{\partial f}{\partial x^2}dx^1
\biggr)
\\
&=
{\ast}\biggl(
\frac{\partial^2 f}{\partial (x^1)^2}dx^1\wedge dx^2
-
\frac{\partial^2 f}{\partial (x^2)^2}dx^2\wedge dx^1
\biggr)
=
\ast \Delta f dx^1\wedge dx^2
=
\Delta f
\intertext{Umgekehrt ist auf einer $n$-Form $\omega = f\,dx^1\wedge dx^2$}
d\delta \omega
&=
d{\ast}d{\ast}(f\,dx^1\wedge dx^2)
\\
&= d{\ast}df
\\
&=
d{\ast}
\biggl(
\frac{\partial f}{\partial x^1}dx^1
+
\frac{\partial f}{\partial x^2}dx^2
\biggr)
\\
&=
d
\biggl(
\frac{\partial f}{\partial x^1}dx^2
-
\frac{\partial f}{\partial x^2}dx^1
\biggr)
\\
&=
\frac{\partial^2 f}{\partial (x^1)^2} dx^1\wedge dx^2
-
\frac{\partial^2 f}{\partial (x^2)^2} dx^2\wedge dx^1
\\
&=
\Delta f\, dx^1\wedge dx^2
.
\end{align*}
Der Operator $d\delta$ ist also im Wesentlichen die Wirkung
des Laplace-Operators auf dem Koeffizienten $f$.

Schliesslich können wir die Operatoren $d\delta$ und $\delta d$ auf
der 1-Form $\omega = f_1\,dx^1+f_2\,dx^2$ berechnen:
\begin{align*}
d\delta\omega
&=
d{\ast}d{\ast}\omega
\\
&=
d{\ast}d\bigl(f_1\,dx_2 - f_2\,dx_1)
\\
&=
d{\ast}\biggl(
\frac{\partial f_1}{\partial x^1}\,dx^1\wedge dx^2
-
\frac{\partial f_2}{\partial x^2}\,dx^2\wedge dx^1
\biggr)
\\
&=
d\biggl(
\frac{\partial f_1}{\partial x^1}
+
\frac{\partial f_2}{\partial x^2}
\biggr)
\\
&=
\frac{\partial^2 f_1}{\partial (x^1)^2}\,dx^1
+
\frac{\partial^2 f_1}{\partial x^2\,\partial x^1}\,dx^2
+
\frac{\partial^2 f_2}{\partial x^1\,\partial x^2}\, dx^1
+
\frac{\partial^2 f_2}{\partial (x^2)^2}\, dx^2
\\
&=
\biggl(
\frac{\partial^2 f_1}{\partial (x^1)^2}
+
\frac{\partial^2 f_2}{\partial x^1\,\partial x^2}
\biggr)
\, dx_1
+
\biggl(
\frac{\partial^2 f_1}{\partial x^2\,\partial x^1}
+
\frac{\partial^2 f_2}{\partial (x^2)^2}
\biggr)
\, dx_2
\\
\delta d\omega
&=
{\ast}d{\ast}
\biggl(
\frac{\partial f_1}{\partial x^2}dx^2\wedge dx^1
+
\frac{\partial f_2}{\partial x^2}dx^2\wedge dx^2
\biggr)
\\
&=
{\ast}d{\ast}
\biggl(
\frac{\partial f_1}{\partial x^2}
-
\frac{\partial f_2}{\partial x^1}
\biggr) dx^1\wedge dx^2
\\
&=
{\ast}d
\biggl(
\frac{\partial f_1}{\partial x^2}
-
\frac{\partial f_2}{\partial x^1}
\biggr)
\\
&=
{\ast}
\biggl(
\frac{\partial^2 f_1}{\partial x^1\,\partial x^2}
-
\frac{\partial^2 f_2}{\partial (x^1)^2}
\biggr)
dx^1
+
{\ast}
\biggl(
\frac{\partial^2 f_1}{\partial (x^2)^2}
-
\frac{\partial^2 f_2}{\partial x^1\,\partial x^2}
\biggr) 
dx^2
\\
&=
\biggl(
\frac{\partial^2 f_1}{\partial x^1\,\partial x^2}
-
\frac{\partial^2 f_2}{\partial (x^1)^2}
\biggr)
dx^2
+
\biggl(
\frac{\partial^2 f_1}{\partial x^1\,\partial x^2}
-
\frac{\partial^2 f_2}{\partial (x^2)^2}
\biggr) 
dx^1
\intertext{Daraus kann man ableiten, dass}
(d\delta+\delta d)\omega
&=
\Delta f_1\,dx^1
+
\Delta F_2\,dx^2
\end{align*}
In allen Fällen ist also $d\delta + \delta d$ im wesentlichen
der Laplace-Operator.
\end{beispiel}

Das Beispiel zeigt, dass der Laplace-Operator koordinatenunabhängig
definiert werden kann.
Man kann aber sogar nachrechnen, dass für eine beliebige Funktion
auf $\mathbb{R}^n$ der Operator $\delta d$ der Laplace-Operator
ist:
\begin{align*}
\delta d f
&=
{\ast}d{\ast}\sum_{k=1}^n \frac{\partial f}{\partial x^k}\,dx^k
\\
&=
{\ast}d\sum_{k=1} \frac{\partial f}{\partial x^k}
(-1)^k
dx^1\wedge\dots\wedge \widehat{dx^k}\wedge\dots\wedge dx^n
\\
&=
{\ast}
\sum_{k=1}^n
(-1)^k
\frac{\partial^2 f}{\partial (x^k)^2}
dx^k\wedge dx^1\wedge\dots\wedge\widehat{dx^k}\wedge\dots\wedge dx^n
\\
&=
{\ast}
\biggl(
\sum_{k=1}^n
\frac{\partial^2 f}{\partial (x^k)^2}
\biggr)
dx^1\wedge\dots\wedge dx^n
\\
&=
\sum_{k=1}^n \frac{\partial^2 f}{\partial (x^k)^2}
=
\Delta f,
\end{align*}
wobei der Hut $\widehat{dx^k}$ bedeutet, dass diese 1-Form im
Produkt weggelassen werden muss.

\begin{definition}
Sei $M$ eine Mannigfaltigkeit mit einer Metrik, dann ist
der {\em Hodge-Laplace-Operator} ist der Operator
\index{Hodge-Laplace-Operator}%
\[
\Delta = d\delta + \delta d.
\]
Der Hodge-Laplace-Operator ist auf ganz $\Omega^*$ definiert.
\end{definition}

\begin{beispiel}
Wir berechnen den Hodge-Laplace-Operator auf $\mathbb{R}^2$ in
Polarkoordinaten.
Dazu verwenden wir die früher berechnete Darstellung des Hodge-Operators
in Polarkoordinaten.
\begin{align*}
\Delta f
&=
d\delta f
+
\delta d f
=
\delta d f
\\
&=
(-1)^{1\cdot 1+1}
{\ast}d{\ast}d f
=
{\ast}d{\ast}d f
\\
&=
{\ast}d{\ast}
\biggl(
\frac{\partial f}{\partial r}\,dr
+
\frac{\partial f}{\partial \varphi}\,d\varphi
\biggr)
\\
&=
{\ast}d\biggl(
\frac{\partial f}{\partial r}\,r\,d\varphi
-
\frac{\partial f}{\partial \varphi}\,\frac{1}{r}\,dr
\biggr)
\\
&=
{\ast}
\biggl(
\biggl(
\frac{\partial }{\partial r} r \frac{\partial f}{\partial r}
+
\frac{1}{r}
\frac{\partial^2 f}{\partial\varphi^2}
\biggr)
\,dr\wedge d\varphi
\biggr)
\\
&=
\biggl(
\frac{1}{r}
\frac{\partial}{\partial r} 
r
\frac{\partial}{\partial r}
+
\frac{1}{r^2}
\frac{\partial^2}{\partial \varphi^2}
\biggr)
f.
\end{align*}
Dies ist tatsächlich die wohlbekannte Darstellung des
Laplace-Operators in Polarkoordinaten
\end{beispiel}

\begin{beispiel}
Der Hodge-Laplace-Operator auf $\mathbb{R}^3$ in
Kugelkoordinaten.
Die gleiche Art von Rechnung wie für den Hodge-Laplace-Operator in
Polarikoordinaten, aber unter Verwendung der Resultate von
Beispiel~\ref{buch:hodge:hodge:beispiel:r3}, ergibt 
\begin{align*}
\Delta f
&=
(d\delta + \delta d) f
=
\delta d f
\\
&=
(-1)^{1\cdot 2+2}
{\ast}d{\ast}df
\\
&=
{\ast}d{\ast}\biggl(
\frac{\partial f}{\partial r}\,dr
+
\frac{\partial f}{\partial\vartheta}\,d\vartheta
+
\frac{\partial f}{\partial\varphi}\,d\varphi
\biggr)
\\
&=
{\ast}d\biggl(
\frac{\partial f}{\partial r} r^2\sin\vartheta \, d\vartheta\wedge d\varphi
-
\frac{\partial f}{\partial\vartheta}\,\sin\vartheta\,dr\wedge d\varphi
+
\frac{\partial f}{\partial\varphi}\,\frac{1}{\sin\vartheta}\,dr\wedge d\vartheta
\biggr)
\\
&=
{\ast}
\biggl(
\frac{\partial}{\partial r}r^2\frac{\partial f}{\partial r}
\sin\vartheta
+
\frac{\partial}{\partial \vartheta}
\sin\vartheta
\frac{\partial f}{\partial \vartheta}
+
\frac{1}{\sin\vartheta}
\frac{\partial^2 f}{\partial\varphi^2}
\biggr)
\,dr\wedge d\vartheta\wedge d\varphi
\\
&=
\frac{1}{r^2}
\frac{\partial}{\partial r}r^2\frac{\partial f}{\partial r}
+
\frac{1}{r^2\sin\vartheta}
\frac{\partial}{\partial \vartheta}
\sin\vartheta
\frac{\partial f}{\partial \vartheta}
+
\frac{1}{r^2\sin^2\vartheta}
\frac{\partial^2 f}{\partial \varphi^2}.
\end{align*}
Dies ist die wohlbekannte Formel für den Laplace-Operator in 
Kugelkoordinaten.
\end{beispiel}

\begin{satz}
Für den Hodge-Laplace-Operator gilt $\Delta = (d+\delta)^2$.
\end{satz}

\begin{proof}
Durch direkte Rechnung finden wir
\[
(d+\delta)^2
=
dd+d\delta+\delta d+\delta\delta
=
d\delta +\delta d.
\qedhere
\]
\end{proof}

An dieser Stelle wird klar, warum das Vorzeichen $(-1)^p$ in der
Definition des Kodifferentials zweckmässig war.




\uebungsabschnitt

\aufgabetoplevel{chapters/070-hodge/uebungsaufgaben}
\begin{uebungsaufgaben}
\uebungsaufgabe{701}
\end{uebungsaufgaben}
\enduebungsabschnitt
