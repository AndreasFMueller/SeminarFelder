%
% teil2.tex -- Beispiel-File für teil2 
%
% (c) 2020 Prof Dr Andreas Müller, Hochschule Rapperswil
%
% !TEX root = ../../buch.tex
% !TEX encoding = UTF-8
%
\section{Der Laplace-Operator
\label{diffortho:section:laplace}}
\kopfrechts{Der Laplace-Operator}
Mit  der Darstellung der partiellen Ableitung nach den kartesischen
Koordinaten $y^k$ durch die Koordinaten kann jetzt auch durch die
partiellen Ableitungen nach den neuen Koordinaten $x^i$ ausgedrückt
werden.
Der Laplace-Operator in kartesischen Koordinaten ist
\index{Laplace-Operator}%
\[
\Delta
=
\frac{\partial^2}{\partial (y^1)^2}
+
\dots
+
\frac{\partial^2}{\partial (y^n)^2}
=
\sum_{k=1}^n
\frac{\partial^2}{\partial (y^k)^2}
=
\sum_{k=1}^n
\frac{\partial}{\partial y^k}
\frac{\partial}{\partial y^k}.
\]
Mit \eqref{diffortho:eqn:1ordnung} kann man die Ableitungen nach
den $y^k$ durch Ableitungen nach den $x^i$ ausdrücken:
\begin{align}
\Delta
&=
\sum_{k=1}^n
\biggl(
\sum_{i=1}^n
{\tilde J}^i\mathstrut_k
\frac{\partial}{\partial x^i}
\biggr)
\biggl(
\sum_{j=1}^n
{\tilde J}^j\mathstrut_k
\frac{\partial}{\partial x^j}
\biggr)
\notag
\\
&=
\sum_{k,i,j=1}^n
{\tilde J}^i\mathstrut_k
\frac{\partial}{\partial x^i}
{\tilde J}^j\mathstrut_k
\frac{\partial}{\partial x^j}.
\notag
\intertext{Der erste partielle Differentialoperator wirkt auf alles,
was rechts von ihm steht.
Für die Ableitung des Produktes von $\tilde{J}^j\mathstrut_k$ mit
dem zweiten Differentialoperator ist daher die Produktregel zu verwenden.
Es entstehen zwei Terme, im ersten Term wird $\tilde{J}^j\mathstrut_k$
nach $x^i$ abgeleitet und es bleibt ein Differentialoperator erster
Ordnung stehen.
Im zweiten Term bleiben alle zweiten Ableitungen erhalten:
}
&=
\sum_{k,i,j=1}^n
{\tilde J}^i\mathstrut_k
\biggl(
\frac{\partial \tilde{J}^j\mathstrut_k}{\partial x^i}
\frac{\partial}{\partial x^j}
+
{\tilde J}^j\mathstrut_k
\frac{\partial^2}{\partial x^i\,\partial x^j}
\biggr)
\notag
\\
&=
\sum_{k,i,j=1}^n
{\tilde J}^i\mathstrut_k
\frac{\partial \tilde{J}^j\mathstrut_k}{\partial x^i}
\frac{\partial}{\partial x^j}
+
\sum_{k,i,j=1}^n
{\tilde J}^i\mathstrut_k
{\tilde J}^j\mathstrut_k
\frac{\partial^2}{\partial x^i\,\partial x^j}
\label{diffortho:eqn:laplaceterme}
\end{align}
Die Koeffizienten in der zweiten Summe von
\eqref{diffortho:eqn:laplaceterme}
sind die die Skalarprodukte
der Spalten von $\tilde{J}$, die wie die Spalten von $J$ orthogonal
sind.
Aus
\eqref{diffortho:eqn:Jij}
folgt, dass die Koeffizienten der zweiten Ableitungen die quadrierten
Kehrwerte der metrischen Faktoren $h_i(x)$ sind.

Der zweite Term von
\eqref{diffortho:eqn:laplaceterme}
ist
\begin{equation}
\sum_{i=1}^n \frac{1}{h_i(x)^2}\frac{\partial^2}{\partial (x^i)^2}.
\label{diffortho:eqn:2abl}
\end{equation}
Wir können daraus bereits schliessen, dass im Laplace-Operator
ausgedrückt in $x^i$-Koordinaten keine gemischten zweiten Ableitungen
vorkommen.

Die Symbolmatrix $A$ eines Differentialoperators besteht aus den
Koeffizienten der zweiten Ableitungen.
Die Gleichung \eqref{diffortho:eqn:2abl} enthält alle Operatoren zweiter
Ordnung im Laplace-Operator, die Symbolmatrix hat daher die Form
\begin{equation}
A
=
\tilde{J}\tilde{J}^t
=
\begin{pmatrix}
h_1(x)^{-2} & \dots  & 0          \\
\vdots      & \ddots & \vdots     \\[-2pt]
0           & \dots  & h_n(x)^{-2}
\end{pmatrix}.
\label{diffortho:eqn:symbolmatrix}
\end{equation}
Ein Operator heisst {\em elliptisch}, wenn alle Eigenwerte der
Symbolmatrix das gleiche Vorzeichen haben.
Aus der Diagonalform \eqref{diffortho:eqn:symbolmatrix} der Symbolmatrix
kann man die Eigenwerte sofort ablesen, sie sind
\[
\lambda_i
=
\frac{1}{h_i(x)^2}
>
0.
\]
Der Laplace-Operator ist also wie es sein muss elliptisch.

%In der erste Summe von
%\eqref{diffortho:eqn:laplaceterme}
%wird nur der zweite Faktor abgeleitet.
%Durch Anwendung der Produktregel kann ihn auch als
%\begin{align*}
%\sum_{k,i,j=1}^n
%{\tilde J}^i\mathstrut_k
%\frac{\partial \tilde{J}^j\mathstrut_k}{\partial x^i}
%\frac{\partial}{\partial x^j}
%&=
%\sum_{k,i,j=1}^n
%\tilde{J}^i\mathstrut_k
%\frac{\partial(\tilde{J}^i\mathstrut_k\tilde{J}^j\mathstrut_k)}{\partial x^i}
%\frac{\partial}{\partial x^j}
%-
%\sum_{k,i,j=1}^n
%\frac{\partial \tilde{J}^i\mathstrut_k}{\partial x^i}
%\tilde{J}^j\mathstrut_k
%\frac{\partial}{\partial x^j}
%\end{align*}
%geschrieben werden.

%\begin{beispiel}
%\end{beispiel}


