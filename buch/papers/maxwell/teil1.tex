%
% teil1.tex -- Beispiel-File für das Paper
%
% (c) 2020 Prof Dr Andreas Müller, Hochschule Rapperswil
%
% !TEX root = ../../buch.tex
% !TEX encoding = UTF-8
%
\section{Metrik und Hodge-Theorie 
\label{maxwell:section:teil1}}
\kopfrechts{Metrik und Hodge-Theorie}

\subsection{Minkowski Metrik}
In der speziellen Relativitätstheorie (SRT) wird die Minkowski-Metrik verwendet.
\index{spezielle Relativitätstheorie}%
\index{Relativitätstheorie, spezielle}%
\index{Minkowski-Metrik}%
Da es in der SRT keine Krümmung und Gravitation gibt, sind alle Elemente ausserhalb der Diagonale des metrischen Tensors null und somit ist die Raum-Zeit flach.
Zwei Signaturen sind üblich.
Einerseits gibt es die $({-}{+}{+}{+})$-Signatur, bei welcher die Zeitkomponente negativ und die Raumkomponenten positiv gezählt werden.
Andererseits gibt es die $({+}{-}{-}{-})$-Signatur, bei welcher die Zeitkomponente positiv und die Raumkomponenten negativ gezählt werden.
Beide Signaturen sind gleichwertig, solange man sich auf eine Metrik festlegt und diese konsequent beibehält.
Im Folgenden werden wir uns an die $({-}{+}{+}{+})$-Signatur halten.
Daher definieren wir den metrischen Tensor als
\begin{equation}
	g^{ik} = \begin{pmatrix}
		-1 & 0 & 0 & 0 \\ 0 & 1 & 0 & 0 \\ 0 & 0 & 1 & 0 \\ 0 & 0 & 0 & 1 
	\end{pmatrix}.
	\label{maxwell:section:teil1:metrik}
\end{equation}
Der Ausdruck für ein Linienelement in dieser Metrik ist definiert als
\begin{equation*}
	dl^2 = -(dx^0)^2 +(dx^1)^2+(dx^2)^2+(dx^3)^2.
\end{equation*}
Damit wir Raum und Zeit in dieser Metrik gleichartig behandeln können, wählen wir beim Übergang in physikalische Einheiten 
\begin{equation}
	\label{maxwell:koordinaten}
	x^0 = ct,\quad x^1 = x,\quad x^2 = y, \quad x^3 = z .
\end{equation}
Dabei entspricht $c$ der Lichtgeschwindigkeit und ein Linienelement ist somit definiert als
\begin{equation*}
	dl^2 = -c^2dt^2 +dx^2+dy^2+dz^2.
\end{equation*}
Eine Konsequenz dieser Signatur ist, dass zeitartige Abstände $dl^2 < 0$ und raumartige Abstände $dl^2 > 0$ sind.

\subsection{Hodge-Duale der Basis-$k$-Formen}
Für den Teil der inhomogenen Maxwell-Gleichungen benötigen wir die Hodge-Duale von 1-Formen, 2-Formen und 3-Formen im vierdimensionalen Minkowski-Raum.
Um dabei die korrekten Vorzeichen zu erhalten, muss die Hodge-Dualität mit Hilfe des metrischen Tensors $g^{ik}$  der Minkowski-Metrik verwendet werden.

Wir verwenden die Definition des Hodge-Operators
\begin{equation*}
	\alpha \wedge \ast \beta = \langle \alpha, \beta \rangle \operatorname{vol}(M),
\end{equation*}
wobei $\langle \cdot , \cdot \rangle$ das durch die Metrik $g^{ik}$ induzierte Skalarprodukt ist.
Im Folgenden führen wir alle Berechnungen der Hodge-Duale von 1-, 2- und 3-Formen durch.
Wir verwenden $g^{ik}$ gemäss \eqref{maxwell:section:teil1:metrik} und $\operatorname{vol}(M) = dx^0 \wedge dx^1 \wedge dx^2 \wedge dx^3$.
\begin{definition}
\label{maxwell:hodge:kurzschreibweise}
Um die Notation kompakter und übersichtlicher zu gestalten, führen wir die Schreibweise
\begin{align*}
	dx^{i\!j} &:= dx^i \wedge dx^j, 
	%\label{maxwell:hodge:zwei}
	\\
	dx^{i\!jk} &:= dx^i \wedge dx^j \wedge dx^k, 
	%\label{maxwell:hodge:drei}
	\\
	dx^{i\!jkl} & := dx^i \wedge dx^j \wedge dx^k \wedge dx^l
	\notag
\end{align*}
für Wedge-Produkte von Basisformen ein.
\end{definition}
Mit dieser Vorbereitung können wir nun die konkreten Hodge-Duale der Basisformen berechnen.
\subsubsection{Hodge-Duale von 1-Formen}
\begin{align*}
	\ast dx^0 
	&= s \, dx^{123} \\
	dx^0 \wedge \ast dx^0 
	&= dx^0 \wedge s \, dx^{123} = s \, dx^{0123} \\
	&= \langle dx^0, dx^0 \rangle \operatorname{vol}(M) = g^{00} \, dx^{0123} = -dx^{0123} \\
	\Rightarrow s &= -1 \Rightarrow \boxed{\ast dx^0 = - dx^{123}}
	\\[1em]
	\ast dx^1 
	&= s \, dx^{023} \\
	dx^1 \wedge \ast dx^1 
	&= dx^1 \wedge s \, dx^{023} = -s \, dx^{0123} \\
	&= \langle dx^1, dx^1 \rangle \operatorname{vol}(M) = g^{11} \, dx^{0123} = dx^{0123} \\
	\Rightarrow s &= -1 \Rightarrow \boxed{\ast dx^1 = - dx^{023}}
	\\[1em]
	\ast dx^2 
	&= s \, dx^{013} \\
	dx^2 \wedge \ast dx^2 
	&= dx^2 \wedge s \, dx^{013} = s \, dx^{0123} \\
	&= \langle dx^2, dx^2 \rangle \operatorname{vol}(M) = g^{22} \, dx^{0123} = dx^{0123} \\
	\Rightarrow s &= +1 \Rightarrow \boxed{\ast dx^2 = dx^{013}}
	\\[1em]
	\ast dx^3 
	&= s \, dx^{012} \\
	dx^3 \wedge \ast dx^3 
	&= dx^3 \wedge s \, dx^{012} = -s \, dx^{0123} \\
	&= \langle dx^3, dx^3 \rangle \operatorname{vol}(M) = g^{33} \, dx^{0123} = dx^{0123} \\
	\Rightarrow s &= -1 \Rightarrow \boxed{\ast dx^3 = - dx^{012}}
\end{align*}

\subsubsection{Hodge-Duale von 2-Formen}
\begin{align*}
	\ast dx^{01} &= s \, dx^{23} \\
	dx^{01} \wedge \ast dx^{01} &= s \, dx^{0123} \\
	&= \langle dx^{01}, dx^{01} \rangle \, \operatorname{vol}(M) 
	= g^{00} g^{11} \operatorname{vol}(M) = -dx^{0123} \\
	\Rightarrow s &= -1 \Rightarrow \boxed{\ast dx^{01} = - dx^{23}}
	\\[1em]
	\ast dx^{12} &= s \, dx^{03} \\
	dx^{12} \wedge \ast dx^{12} &= s \, dx^{0123} \\
	&= \langle dx^{12}, dx^{12} \rangle \, \operatorname{vol}(M) 
	= g^{11} g^{22} \operatorname{vol}(M) = dx^{0123} \\
	\Rightarrow s &= +1 \Rightarrow \boxed{\ast dx^{12} = dx^{03}}
	\\[1em]
	\ast dx^{23} &= s \, dx^{01} \\
	dx^{23} \wedge \ast dx^{23} &= s \, dx^{0123} \\
	&= \langle dx^{23}, dx^{23} \rangle \, \operatorname{vol}(M) 
	= g^{22} g^{33} \operatorname{vol}(M) = dx^{0123} \\
	\Rightarrow s &= +1 \Rightarrow \boxed{\ast dx^{23} = dx^{01}}
	\\[1em]
	\ast dx^{02} &= s \, dx^{13} \\
	dx^{02} \wedge \ast dx^{02} &= -s \, dx^{0123} \\
	&= \langle dx^{02}, dx^{02} \rangle \, \operatorname{vol}(M) 
	= g^{00} g^{22} \operatorname{vol}(M) = -dx^{0123} \\
	\Rightarrow s &= +1 \Rightarrow \boxed{\ast dx^{02} = dx^{13}}
	\\[1em]
	\ast dx^{03} &= s \, dx^{12} \\
	dx^{03} \wedge \ast dx^{03} &= s \, dx^{0123} \\
	&= \langle dx^{03}, dx^{03} \rangle \, \operatorname{vol}(M) 
	= g^{00} g^{33} \operatorname{vol}(M) = -dx^{0123} \\
	\Rightarrow s &= -1 \Rightarrow \boxed{\ast dx^{03} = - dx^{12}}
	\\[1em]
	\ast dx^{13} &= s \, dx^{02} \\
	dx^{13} \wedge \ast dx^{13} &= -s \, dx^{0123} \\
	&= \langle dx^{13}, dx^{13} \rangle \, \operatorname{vol}(M) 
	= g^{11} g^{33} \operatorname{vol}(M) = dx^{0123} \\
	\Rightarrow s &= -1 \Rightarrow \boxed{\ast dx^{13} = - dx^{02}}
\end{align*}

\subsubsection{Hodge-Duale von 3-Formen}
\begin{align*}
	\ast dx^{012} &= s \, dx^3 \\
	dx^{012} \wedge \ast dx^{012} &= s \, dx^{0123} \\
	&= \langle dx^{012}, dx^{012} \rangle \, \operatorname{vol}(M) 
	= g^{00} g^{11} g^{22} \operatorname{vol}(M) = -dx^{0123} \\
	\Rightarrow s &= -1 \Rightarrow \boxed{\ast dx^{012} = - dx^3}
	\\[1em]
	\ast dx^{013} &= s \, dx^2 \\
	dx^{013} \wedge \ast dx^{013} &= -s \, dx^{0123} \\
	&= \langle dx^{013}, dx^{013} \rangle \, \operatorname{vol}(M) 
	= g^{00} g^{11} g^{33} \operatorname{vol}(M) = -dx^{0123} \\
	\Rightarrow s &= 1 \Rightarrow \boxed{\ast dx^{013} = dx^2}
	\\[1em]
	\ast dx^{023} &= s \, dx^1 \\
	dx^{023} \wedge \ast dx^{023} &= s \, dx^{0123} \\
	&= \langle dx^{023}, dx^{023} \rangle \, \operatorname{vol}(M) 
	= g^{00} g^{22} g^{33} \operatorname{vol}(M) = -dx^{0123} \\
	\Rightarrow s &= -1 \Rightarrow \boxed{\ast dx^{023} = - dx^1}
	\\[1em]
	\ast dx^{123} &= s \, dx^0 \\
	dx^{123} \wedge \ast dx^{123} &= -s \, dx^{0123} \\
	&= \langle dx^{123}, dx^{123} \rangle \, \operatorname{vol}(M)
	= g^{11} g^{22} g^{33} \operatorname{vol}(M) = dx^{0123} \\
	\Rightarrow s &= -1 \Rightarrow \boxed{\ast dx^{123} = - dx^0}
\end{align*}
In der Tabelle \ref{maxwell:section:teil1:Hodge-Tabelle} sind alle berechneten Hodge-Duale noch einmal zusammengefasst.
\begin{table}
	\centering
	%\caption{Hodge-Duale}
	%\label{maxwell:section:teil1:Hodge-Tabelle}
	\begin{tabularx}{\textwidth}{ 
			| >{\centering\arraybackslash}X 
			| >{\centering\arraybackslash}X 
			| >{\centering\arraybackslash}X | }
		\hline
		\textbf{1-Form} & \textbf{2-Form} & \textbf{3-Form} \\
		\hline
		\( \rule{0pt}{1.5em} {\ast} dx^0 = -dx^1 \wedge dx^2 \wedge dx^3 \) \newline
	\( {\ast} dx^1 = -dx^0 \wedge dx^2 \wedge dx^3 \) \newline
	\( {\ast} dx^2 = \phantom{-} dx^0 \wedge dx^1 \wedge dx^3 \) \newline
	\( {\ast} dx^3 = -dx^0 \wedge dx^1 \wedge dx^2 \, \) 
	&
	\( \rule{0pt}{1.5em} {\ast} (dx^0 \wedge dx^1) = -dx^2 \wedge dx^3 \) \newline
	\( {\ast} (dx^1 \wedge dx^2) = \phantom{-} dx^0 \wedge dx^3 \) \newline
	\( {\ast} (dx^2 \wedge dx^3) = \phantom{-} dx^0 \wedge dx^1 \) \newline
	\( {\ast} (dx^0 \wedge dx^2) = \phantom{-} dx^1 \wedge dx^3 \) \newline
	\( {\ast} (dx^0 \wedge dx^3) = -dx^1 \wedge dx^2 \) \newline
	\( {\ast} (dx^1 \wedge dx^3) = -dx^0 \wedge dx^2 \)
	&
	\( \rule{0pt}{1.5em} {\ast} (dx^0 \wedge dx^1 \wedge dx^2) = -dx^3 \) \newline
	\( {\ast} (dx^0 \wedge dx^1 \wedge dx^3) = \phantom{-} dx^2 \) \newline
	\( {\ast} (dx^0 \wedge dx^2 \wedge dx^3) = -dx^1 \) \newline
	\( {\ast} (dx^1 \wedge dx^2 \wedge dx^3) = -dx^0 \)
		\\
		\hline
	\end{tabularx}
	\caption{Tabelle aller Hodge-Duale von 1-, 2-, und 3-Formen mit $({-}{+}{+}{+})$-Signatur}
	\label{maxwell:section:teil1:Hodge-Tabelle}
\end{table}







