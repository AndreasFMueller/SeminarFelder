Wir betrachten die Koordinatentransformation
\[
\renewcommand{\arraycolsep}{2pt}
\left.
\begin{array}{rcrcr}
y^1 &=&  x^1 &+& 2x^2 \\
y^2 &=& 3x^1 &+& 5x^2
\end{array}
\right\}
\qquad\text{mit der inversen Transformation}\qquad
\left\{
\begin{array}{rcrcr}
x^1 &=& -5y^1 &+& 2y^2\phantom{.} \\
x^2 &=&  3y^1 &-&  y^2.
\end{array}
\right.
\]
\begin{teilaufgaben}
\item
Drücken Sie die Ableitungen nach den $x^i$ durch die Ableitungen nach
$y^k$ aus.
\item
Drücken Sie den Laplace-Operator in den Koordinaten $x^i$, also 
\[
\delta =
\frac{\partial^2}{\partial (x^1)^2}
+
\frac{\partial^2}{\partial (x^2)^2}
\]
in $y^k$-Koordinaten aus.
\item
Drücken sie die Differentiale $dx^i$ durch durch $dy^k$ aus.
\item
Die euklidische Metrik in den $x_i$-Koordinaten ist gegeben durch
den Tensor
\[
g
=
dx^1\otimes dx^1
+
dx^2\otimes dx^2
=
\sum_{i,k=1}^2
\delta_{ik}dx^i\otimes dx^k
\]
zweiter Stufe.
Drücken Sie ihn durch die Differentiale $dy^1$ und $dy^2$ aus.
\item
Vergleichen Sie die Koeffizienten der zweiten Ableitungen des
Laplace-Operators in den $y^k$-Koordinaten mit den metrischen
Koeffizienten von Teilafugabe d).
\end{teilaufgaben}

\begin{loesung}
Wir schreiben die Transformationen in Matrixform als
\[
\begin{pmatrix}
y^1\\y^2
\end{pmatrix}
=
\underbrace{
\begin{pmatrix}
1&2\\ 3&5
\end{pmatrix}
}_{\displaystyle = A}
\begin{pmatrix}
x^1\\x^2
\end{pmatrix}
\qquad\text{und}\qquad
\begin{pmatrix}
x^1\\x^2
\end{pmatrix}
=
\underbrace{
\begin{pmatrix*}[r]
-5&2\\ 3&-1
\end{pmatrix*}
}_{\displaystyle = A}
\begin{pmatrix}
y^1\\y^2
\end{pmatrix}.
\]
\begin{teilaufgaben}
\item Die Ableitung einer Funktion $f$ nach $x^i$ ist mit der
Kettenregel
\begin{align*}
\frac{\partial f}{\partial x^i}
&=
\frac{\partial f}{\partial y^1}\frac{\partial y^1}{\partial x^i}
+
\frac{\partial f}{\partial y^2}\frac{\partial y^2}{\partial x^i}
=
\sum_{k=1}^2
\frac{\partial y^k}{\partial x^i}
\frac{\partial f}{\partial y^k}.
\end{align*}
In Matrixschreibweise ist dies
\begin{equation}
\renewcommand{\arraystretch}{1.8}
\begin{pmatrix}
\displaystyle \frac{\partial f}{\partial x^1}\\
\displaystyle \frac{\partial f}{\partial x^2}
\end{pmatrix}
=
A
\begin{pmatrix}
\displaystyle \frac{\partial f}{\partial y^1}\\
\displaystyle \frac{\partial f}{\partial y^2}
\end{pmatrix}
\qquad\Rightarrow\qquad
\begin{pmatrix}
\displaystyle \frac{\partial }{\partial x^1}\\
\displaystyle \frac{\partial }{\partial x^2}
\end{pmatrix}
=
A
\begin{pmatrix}
\displaystyle \frac{\partial }{\partial y^1}\\
\displaystyle \frac{\partial }{\partial y^2}
\end{pmatrix}.
\label{701:eqn:gradoperator}
\end{equation}
\item
Der Laplace-Operator entsteht als das Skalarprodukt
\[
\Delta
=
\renewcommand{\arraystretch}{1.8}
\begin{pmatrix}
\displaystyle \frac{\partial }{\partial x^1}\\
\displaystyle \frac{\partial }{\partial x^2}
\end{pmatrix}^t
\begin{pmatrix}
\displaystyle \frac{\partial }{\partial x^1}\\
\displaystyle \frac{\partial }{\partial x^2}
\end{pmatrix}.
\]
Unter Verwendung von \eqref{701:eqn:gradoperator} erhalten wir
\begin{align*}
\Delta
&=
\left( A
\bgroup
\renewcommand{\arraystretch}{1.8}
\begin{pmatrix}
\displaystyle \frac{\partial }{\partial y^1}\\
\displaystyle \frac{\partial }{\partial y^2}
\end{pmatrix}
\egroup
\right)^t
A
\bgroup
\renewcommand{\arraystretch}{1.8}
\begin{pmatrix}
\displaystyle \frac{\partial }{\partial y^1}\\
\displaystyle \frac{\partial }{\partial y^2}
\end{pmatrix}
\egroup
=
\bgroup
\renewcommand{\arraystretch}{1.8}
\begin{pmatrix}
\displaystyle \frac{\partial }{\partial y^1}\\
\displaystyle \frac{\partial }{\partial y^2}
\end{pmatrix}^t
\egroup
A^tA
\bgroup
\renewcommand{\arraystretch}{1.8}
\begin{pmatrix}
\displaystyle \frac{\partial }{\partial y^1}\\
\displaystyle \frac{\partial }{\partial y^2}
\end{pmatrix}
\egroup
=
\bgroup
\renewcommand{\arraystretch}{1.8}
\begin{pmatrix}
\displaystyle \frac{\partial }{\partial y^1}\\
\displaystyle \frac{\partial }{\partial y^2}
\end{pmatrix}^t
\egroup
\begin{pmatrix}
10&17\\
17&29
\end{pmatrix}
\bgroup
\renewcommand{\arraystretch}{1.8}
\begin{pmatrix}
\displaystyle \frac{\partial }{\partial y^1}\\
\displaystyle \frac{\partial }{\partial y^2}
\end{pmatrix}
\egroup
\\
&=
10\frac{\partial^2}{\partial (x^1)^2}
+
34\frac{\partial^2}{\partial x^1\,\partial x^2}
+
29\frac{\partial^2}{\partial (x^2)^2}.
\end{align*}
\item
Aus den Formeln für die Variablentransformation folgt
\begin{equation}
x^i = \sum_{k=1}^2 a^i\mathstrut_k y^k
\qquad\Rightarrow\qquad
dx^i
=
\sum_{k=1}^n a^i\mathstrut_k dy^k
\quad
\Leftrightarrow
\quad
\begin{pmatrix}dx^1\\dx^2\end{pmatrix}
=
A
\begin{pmatrix}dy^1\\dy^2\end{pmatrix}
\label{701:eqn:differentialformentransformation}
\end{equation}
Darin sind die Koeffizienten $a^i\mathstrut_k$ die Einträge der Matrix $A$.
\item
Der metrische Tensor entsteht als ``Skalarprodukt'' 
\[
g
=
\sum_{i=1}^2\, dx^i\otimes dx^i.
\]
Mithilfe der Transformationsformeln
\eqref{701:eqn:differentialformentransformation}
wird daraus
\begin{align*}
g
&=
\sum_{i=1}^n
\biggl(\sum_{k=1} a^i\mathstrut_k\,dx^k\biggr)
\otimes
\biggl(\sum_{k=1} a^i\mathstrut_l\,dx^l\biggr)
=
\sum_{k,l=1}^2
\biggl(
\sum_{i=1}^2
a^i\mathstrut_k
a^i\mathstrut_l
\biggr)
\,
dx^k\otimes dx^l.
\end{align*}
Die Koeffizienten in der grossen Klammer sind die Einträge der Matrix
\begin{align*}
A^tA
&=
\begin{pmatrix}
10&17\\
17&29
\end{pmatrix}.
\end{align*}
\item
Die Koeffizienten der partiellen Ableitungen des Laplace-Operators
in den $y^k$-Koor\-dinaten sind also genau die metrischen Koeffizienten.
\qedhere
\end{teilaufgaben}
\end{loesung}
