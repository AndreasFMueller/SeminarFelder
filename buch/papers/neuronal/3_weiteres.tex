%
% 3_weiteres.tex -- Diskussion & weitere Anwendungsmöglichkeiten
%
% (c) 2025 Roman Cvijanovic & Nicola Dall'Acqua, Hochschule Rapperswil
%
% !TEX root = ../../buch.tex
% !TEX encoding = UTF-8
%

\section{Diskussion}\label{neuronal:section:diskussion}
\kopfrechts{Diskussion}

Der grosse Vorteil dieser Methode liegt in ihrer breiten Anwendbarkeit.
Sie kann grundsätzlich auf beliebige Feldgleichungen angewendet werden, unabhängig von deren Komplexität.
Im Gegensatz dazu sind klassische Verfahren wie die Finite-Differenzen- oder Finite-Elemente-Methode oft auf bestimmte Gleichungstypen beschränkt und stossen insbesondere bei stark nichtlinearen oder hochdimensionalen Problemen an ihre Grenzen.
Neuronale Netzwerke bieten durch das \emph{Universal Approximation Theorem} die theoretische Grundlage, um beliebige stetige Funktionen mit beliebiger Genauigkeit zu approximieren \cite{neuronal:universal_approximation_theorem}.
Dies befähigt sie theoretisch dazu, Lösungen für jede beliebige Feldgleichung zu approximieren.

Was die vorangegangenen Beispiele jedoch klar zeigen, ist dass diese Methode bei der Burgers-Gleichung deutlich besser funktioniert als bei der Wellengleichung.
Der Grund dafür liegt in den Gleichungen selbst: 
Die Wellengleichung ist eine hyperbolische, die Burgers-Gleichung eine parabolische partielle Differentialgleichung.
Neuronale Netzwerke haben beim Lösen von hyperbolischen Gleichungen Schwierigkeiten \cite{neuronal:hyperbolisch_1}, \cite{neuronal:hyperbolisch_2}, \cite{neuronal:hyperbolisch_3}.
Zu beachten ist, dass diese Schwierigkeiten nicht auf einen Mangel an Ausdruckskraft des neuronalen Netzwerks zurückzuführen sind.
Vielmehr liegt der Grund im Lösen des vorgestellten Optimierungsproblems \ref{neuronal:subsection:lösen_optimierungsproblem} \cite{neuronal:hyperbolisch_4}.
Dieses wird für bestimmte Gleichungstypen, wie hyperbolische Feldgleichungen, durch deren charakteristische Eigenschaften -- zum Beispiel kontinuierliche Wellenausbreitung -- erheblich erschwert.
Dadurch kann das Finden einer guten Lösung sehr schwer werden.
Somit ist die Existenz einer guten Lösung zwar durch das \emph{Universal Approximation Theorem} garantiert, nicht aber das Finden davon.

In der Praxis zeigen neuronale Netzwerke vielversprechende Ergebnisse beim Lösen von Feldgleichungen \cite{neuronal:pinns}.
Jedoch befindet sich die Methode noch in einem frühen Entwicklungsstadium.
Es gibt viele offene Fragen, insbesondere hinsichtlich der Effizienz und Genauigkeit im Vergleich zu klassischen Verfahren.
Zusätzliche Forschungsergebnisse dürften hierzu weitere Erkenntnisse liefern.
Eine abschließende Bewertung der Methode ist daher zum jetzigen Zeitpunkt noch nicht möglich.
