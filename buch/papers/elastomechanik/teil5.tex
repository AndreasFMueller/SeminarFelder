%
% teil5.tex -- Beispiel-File für Teil 5
%
% (c) 2020 Prof Dr Andreas Müller, Hochschule Rapperswil
%
% !TEX root = ../../buch.tex
% !TEX encoding = UTF-8
%
\section{Fazit
\label{elastomechanik:section:teil5}}
\index{Fazit}
Die Elastomechanik mag auf den ersten Blick wie eine trockene
Ansammlung von Indizes, Tensoren und Gleichungen wirken, doch bei
näherer Betrachtung offenbart sich eine elegante Theorie, die sowohl
mathematische Schönheit als auch praktische Relevanz vereint.
In dieser Arbeit wurde gezeigt, wie sich aus den Grundprinzipien
der Mechanik, nämlich den kinematischen Beziehungen, den Materialgesetzen
und den Gleichgewichtsbedingungen, die vollständige Beschreibung
des Verformungsverhaltens elastischer Körper ableiten lässt.
Besonders die Herleitung der Navier-Cauchy-Gleichungen zeigt
eindrucksvoll, dass auch in der technischen Mechanik manchmal alles
nur eine Frage der richtigen Ableitung ist.

Anhand eines einfachen Balkenmodells konnten die Auswirkungen
unterschiedlicher Materialien auf Spannungen und Durchbiegungen
veranschaulicht werden.
Dabei wurde deutlich, dass die Verformung von Stahl deutlich geringer
ist als die von Holz.
Dieser Unterschied entsteht, weil Holz ein geringeres Elastizitätsmodul
besitzt.

Anhand des Oedometerversuchs kann man die Steifigkeit eines Bodens
bestimmen, was wiederum im Bauwesen wichtig ist, um Setzungen im
Baugrund zu berechnen.

Zusammenfassend lässt sich sagen, dass die Elastizitätstheorie mehr
ist als reine Formelakrobatik.
Sie ist ein essenzielles Werkzeug des Ingenieurwesens, das hilft,
Strukturen sicher zu gestalten.
Und manchmal, wenn man tief genug in den Spannungstensor eintaucht,
erkennt man, dass selbst in einer symmetrischen $3 \times 3$ Matrix
mehr Drama steckt als in manchen Bauprojekten.


