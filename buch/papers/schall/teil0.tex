%
% einleitung.tex -- Beispiel-File für die Einleitung
%
% (c) 2020 Prof Dr Andreas Müller, Hochschule Rapperswil
%
% !TEX root = ../../buch.tex
% !TEX encoding = UTF-8
%
\section{Einleitung\label{schall:section:teil0}}
\kopfrechts{Teil 0}

Schall breitet sich in einem Medium als Druckschwankung aus, die sich mit endlicher Geschwindigkeit fortpflanzt.
Im einfachsten physikalischen Modell vernachlässigen wir Effekte wie Reflexionen, Hindernisse, starke Druck- oder Materialänderungen und betrachten die Ausbreitung in einer Dimension entlang einer Achse $x \in \mathbb{R}$.
Die charakteristische Größe ist die Schallgeschwindigkeit $c$, welche von den Eigenschaften des Mediums abhängt.
Grundsätzlich können sich Schallwellen in jedem Medium ausbreiten.
Deshalb unterscheiden wir Gase, Flüssigkeiten und Festkörper.

Bei Gasen sind, im Vergleich zu Flüssigkeiten und Festkörpern, die einzelnen Moleküle frei bewegbar und nicht direkt verbunden mit benachbarten Molekülen.
Durch den vorhandenen Zwischenraum sind Gase kompressibel und temperaturabhängig.
Da die einzelnen Moleküle nicht direkt miteinander verbunden sind, ist die Schallgeschwindigkeit in Gasen am niedrigsten.
Für ein ideales Gas gilt
\begin{equation}
    c_{g} = \sqrt{\kappa \, \frac{p}{\rho}},
\end{equation}
wobei $\kappa$ der Adiabatenexponent, $p$ der Druck und $\rho$ die Dichte des Mediums ist.
Setzt man $p = \rho R T$, welches die ideale Gasgleichung mit der spezifischen Gaskonstante $R = \SI{8.314e2}{\frac{J}{mol \cdot K}}$ und der Temperatur $T$ beschreibt, erhält man die Darstellung
\begin{equation}
    c_{g} = \sqrt{\kappa R T}.
\end{equation}
Für Luft bei Normalbedingungen, welches mit $T \approx 20^\circ \mathrm{C}$ und $p \approx 1\,\mathrm{bar}$ gegeben ist, erhält man so näherungsweise
\begin{equation}
    c_{g,Luft} \approx 343 \,\frac{\mathrm{m}}{\mathrm{s}}.
\end{equation}

Flüssigkeiten sind im Vergleich zu Gasen inkompressibel, da die Moleküle durch intermolekulare Kräfte stark aneinander gebunden sind.
Die Schallgeschwindigkeit ist deshalb deutlich höher als in Gasen.
Für Flüssigkeiten gilt näherungsweise
\begin{equation}
    c_{l} = \sqrt{\frac{K}{\rho}},
\end{equation}
wobei $K$ das Kompressionsmodul und $\rho$ die Dichte der Flüssigkeit ist.
Bei Gasen, sowie bei Flüssigkeiten, sind nur longitudiale Wellenausbreitungen möglich.
Für Wasser bei $T \approx 20^\circ \mathrm{C}$ gilt näherungsweise
\begin{equation}
    c_{l,Wasser} \approx 1480 \,\frac{\mathrm{m}}{\mathrm{s}}.
\end{equation}


Festkörper sind im Vergleich zu Gasen und Flüssigkeiten noch weniger kompressibel, da die Moleküle durch starke Bindungen aneinander gebunden sind.
Die Schallgeschwindigkeit ist deshalb am höchsten.
In Festkörpern sind sowohl longitudinale als auch transversale Wellenausbreitungen möglich.
Die Schallgeschwindigkeit in Festkörpern hängt von der Richtung der Ausbreitung ab.
Longitudinale Wellen breiten sich in Festkörpern mit der Geschwindigkeit
\begin{equation}
    c_{s,long} = \sqrt{\frac{E(1-\nu)}{\rho(1+\nu)(1-2\nu)}}
\end{equation}
aus, wobei $E$ der Elastizitätsmodul, $\nu$ die Querkontraktionszahl und $\rho$ die Dichte des Festkörpers ist.
Transversale Wellen breiten sich mit der Geschwindigkeit
\begin{equation}
    c_{l,trans} = \sqrt{\frac{E}{2\rho(1+\nu)}} = \sqrt{\frac{G}{\rho}}
\end{equation}
aus, wobei $G$ das Schubmodul des Festkörpers ist.
Generell sind longitudinale Wellen schneller als transversale Wellen.
Für Stahl gilt die Schallgeschwindigkeit näherungsweise
\begin{equation}
    c \approx 5000 \,\frac{\mathrm{m}}{\mathrm{s}}.
\end{equation}

Damit können wir Schall im einfachsten Ansatz als Welle verstehen, die sich gleichmäßig mit einer vom Medium bestimmten Geschwindigkeit ausbreitet.
Woher diese Schallgeschwindigkeit kommt und wie sie sich bei komplexeren Modellen ändert, wird in den folgenden Abschnitten untersucht.