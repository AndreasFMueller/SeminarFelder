\section{Trivektoren und $k$-Vektoren}
\label{geoalgebra:section:trivectors-n-vectors}
In zwei Dimensionen sind wir limitiert auf Bivektoren. Was passiert jedoch, wenn wir einen Bivektor
im dreidimensionalen Raum wieder mit dem Wedgeprodukt mit einem dritten Vektor verknüpfen?

{
\begin{equation} 
\begin{aligned}
\a \wedge \b \wedge \c &= (a_1 \eone + a_2 \etwo + a_3 \ethree) \wedge (b_1 \eone + b_2 \etwo + b_3 \ethree) \wedge (c_1 \eone + c_2 \etwo + c_3 \ethree) \\
&= ((a_1 b_2 - a_2 b_1) \eone \wedge \etwo +(a_2 b_3 - a_3 b_2) \etwo \wedge \ethree +(a_3 b_1 - a_1 b_3) \ethree \wedge \eone) \\
&\quad\wedge (c_1 \eone + c_2 \etwo + c_3 \ethree) \\
&= (a_1 b_2 - a_2 b_1) c_1 \eone \wedge \etwo \wedge \eone \cdots
\end{aligned}
\label{geoalgebra:eq:trivector}
\end{equation}
Es fällt auf, dass viele Terme mit Wedgeprodukten zwischen gleichen Standardbasisvektoren übrig bleiben. Zum Beispiel ist der erste Term in \eqref{geoalgebra:eq:trivector}
\begin{equation}
(a_1 b_2 - a_2 b_1) c_1 \eone \wedge \etwo \wedge \eone.
\end{equation}
Durch Umformen mit den bekannten Axiomen erhalten wir
\begin{equation}
=-(a_1 b_2 - a_2 b_1) c_1 \etwo \wedge \underbrace{\eone \wedge \eone}_{0} = 0
\end{equation}
Es fallen alle Kombination bis auf $\eone \wedge \etwo \wedge \ethree$ (und Permutationen) weg.
Durch Umformungen mit \eqref{geoalgebra:eq:antikommutativ} und
\eqref{geoalgebra:lemma:null} erhalten wir dann
\begin{equation}
  \begin{aligned}
  \a \wedge \b \wedge \c &= (a_1 b_2 c_3 - a_1 b_3 c_2 + a_2 b_1 c_3 + a_2 b_3 c_1 - a_3 b_2 c_1 + a_3 b_1 c_2) \eone \wedge \etwo \wedge \ethree \\
  &= \lambda \eone \wedge \etwo \wedge \ethree
  \end{aligned}
\end{equation}
wobei der Koeffizient $\lambda$ wieder als Determinante
\begin{equation}
\begin{vmatrix} a_1 & a_2 & a_3 \\ b_1 & b_2 & b_3 \\ c_1 & c_2 & c_3 \end{vmatrix} \eone \wedge \etwo \wedge \ethree
\end{equation}
geschrieben werden kann.

Ein Trivektor ist ein \emph{orientiertes Volumen}, welches von den drei Vektoren als \emph{Parallelpiped} aufgespannt wird.

Die Anzahl Komponenten $K$ eines $k$-Vektors in $N$ Dimensionen ergibt sich analog zu \eqref{geoalgebra:eq:components-bivectors} durch
\begin{equation}
  K = \binom{N}{k}
\end{equation}
}
