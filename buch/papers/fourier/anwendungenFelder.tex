%
% anwendungenFelder.tex -- 3.	Anwendungen auf Felder, PDEs in ODE umwandeln (Grundlage, um Feldgleichungen als eine Familie von Oszillator-Gleichungen zu interpretieren)
%
% (c) 2020 Prof Dr Andreas Müller, Hochschule Rapperswil
%
% !TEX root = ../../buch.tex
% !TEX encoding = UTF-8
%

%    Anwendungen auf Felder, PDEs in ODE umwandeln (Grundlage, um Feldgleichungen als eine Familie von Oszillator-Gleichungen zu interpretieren) 


\section{Anwendung auf Feld\label{fourier:section:AnwendungAufFeld}}
\kopfrechts{Anwendung auf Feld}

Eine gewöhnliche Differentialgleichung enthält Ableitungen nur nach einer unabhängigen Variable, während eine partielle Differentialgleichung Ableitungen nach mehreren unabhängigen Variablen enthält.  
In diesem Abschnitt überführen wir eine partielle Differentialgleichung mittels Fourierreihe in eine gewöhnliche Differentialgleichung für die Zeitvariable.
Als Beispiel betrachten wir die eindimensionale Wellengleichung  
\begin{equation}\label{eq:wellengleichung}
	\frac{\partial^2 u(x,t)}{\partial t^2} = c^2 \frac{\partial^2 u(x,t)}{\partial x^2}
\end{equation}  
mit Ausbreitungsgeschwindigkeit $c$. Dieses Modell beschreibt unter anderem elektromagnetische Felder.  
Unter der Annahme, dass $u(x, t)$ eine periodisch Funktion ist, führen wir eine Fourierreihen-Entwicklung durch. 
Neu besteht $u(x, t)$ 
\begin{equation}
	u(x,t) = \frac{a_0(t)}{2} + \sum_{n=1}^{\infty} \left( a_n(t) \cos(n \omega x) + b_n(t) \sin(n \omega x) \right)
\end{equation}
aus einer unendlichen Anzahl von Schwingungen.
Alle drei Fourier Koeffizienten erhält man durch die Integral-Rechnungen im Abschnitt \ref{fourier:section:GrundlagenFourierAnalyse}. 
Da das Ziel ist, $x$ zu eliminieren, integriert man über eine Periode nach $x$.
Somit entstehen Fourier-Koeffizienten, die von $t$ abhängig sind. 
Nun wird $u(x,t)$ in seiner Summenform in die Wellengleichung eingesetzt. 
Dazu muss $u(x,t)$ je zwei Mal nach $t$ und $x$ abgeleitet werden:

\begin{equation}
	\begin{aligned}
		\frac{\partial^2 u(x,t)}{\partial t^2}
		&= \frac{1}{2}\frac{\mathrm d^2 a_0(t)}{\mathrm d t^2}
		+ \sum_{n=1}^{\infty}\bigl(\ddot a_n(t)\cos(n\omega x)+\ddot b_n(t)\sin(n\omega x)\bigr)\\
		\frac{\partial^2 u(x,t)}{\partial x^2}
		&= \sum_{n=1}^{\infty}\bigl(-n^2\omega^2\,a_n(t)\cos(n\omega x)-n^2\omega^2\,b_n(t)\sin(n\omega x)\bigr)
	\end{aligned}
\end{equation}

Von nun an wird als Ableitungsoperator $\frac{d}{dt}$ verwendet, da die Gleichung nur noch von $t$ abhängig ist.
Die Resultate können jetzt in die Wellengleichung eingesetzt werden. 

%\begin{equation}
%	 \frac{1}{2} \frac{d^2 a_0(t)}{d t^2} + \sum_{n=1}^{\infty} \left( \frac{d^2}{dt^2} a_n(t) \cos(n \omega x) + \frac{d^2}{dt^2} b_n(t) \sin(n \omega x) \right) = c^2  \sum_{n=1}^{\infty} \left( -a_n(t) n^2 \omega^2 \cos(n \omega t) - b_n(t) n^2 \omega^2 \sin(n \omega t) \right) 
%\end{equation}


\begin{multline}
	\frac{1}{2}\frac{d^2 a_0(t)}{dt^2}
	+ \sum_{n=1}^{\infty}\Bigl(
	\frac{d^2 a_n(t)}{dt^2}\cos(n\omega x)
	+ \frac{d^2 b_n(t)}{dt^2}\sin(n\omega x)
	\Bigr)
	= \\[-0.8ex]
	c^2 \sum_{n=1}^{\infty}\Bigl(
	-a_n(t)\,n^2\omega^2\cos(n\omega x)
	-b_n(t)\,n^2\omega^2\sin(n\omega x)
	\Bigr)
\end{multline}


Diese Gleichung kann nun mithilfe eines Koeffizientenvergleichs gelöst werden.
So entstehen folgende drei Gleichungen:

\begin{equation}
	\begin{aligned}
		\frac{1}{2} \frac{d^2 a_0(t)}{d t^2} &= 0 \\
		\sum_{n=1}^{\infty}\frac{d^2 a_n(t)}{dt^2}\cos(n\omega x) &= c^2 \sum_{n=1}^{\infty}-a_n(t)\,n^2\omega^2\cos(n\omega x)\\
		\sum_{n=1}^{\infty}\frac{d^2 b_n(t)}{dt^2}\sin(n\omega x) &= c^2 \sum_{n=1}^{\infty}-b_n(t)\,n^2\omega^2\sin(n\omega x)
	\end{aligned}
\end{equation}


Die zweite Ableitung des Mittelwerts $a_0(t)$ ist 0, somit ist $a_0(t)$ eine lineare Funktion, eine Konstante oder 0. Also ist $a_0(t)$ von der Form $a_0(t)=C_1 t + C_2$.
Die Integrationskonstanten $C_1$ und $C_2$ werden durch die Anfangsbedingungen des Systems festgelegt.
Die Gleichungen mit $a_n(t)$ und $b_n(t)$ kann man wiederum mit einem Koefizientenvergleich lösen, diesmal jedoch über eine unendliche Summe! Man endet mit einer unendlichen Anzahl von Gleichungen für jede Frequenz. Da sie alle dieselbe Form haben, bleibt man mit der Schreibweise bei $n$.
Zudem sind die $\cos(n\omega x)$ und $\sin(n\omega x)$ zu kürzen. 
Somit entstehen die folgenden zwei gewöhnlichen Differentialgleichungen:

\begin{equation}\label{eq:ODE_Wellengleichung}
	\frac{d^2}{dt^2} a_n(t) + a_n(t) n^2 \omega^2 c^2 = 0
	  \quad   \text{und} \quad  \frac{d^2}{dt^2} b_n(t) + b_n(t) n^2 \omega^2 c^2 = 0
\end{equation}

Die Struktur der Wellengleichungen aus \eqref{eq:ODE_Wellengleichung} entsprechen formal der Bewegungsgleichung eines ungedämpften Federpendels  
\begin{equation}\label{eq:loesungFederpendel}
	\frac{d^2 x(t)}{d t^2} + \omega^2 x(t) = 0
\end{equation}
Die allgemeine Lösung von \eqref{eq:loesungFederpendel} lautet  
\begin{equation}
	x(t) = A \cos(\omega t) + B \sin(\omega t)
	\quad\text{oder}\quad
	x(t) = C \cos(\omega t - \varphi).
\end{equation}
Dieses Ergebnis beschreibt die Frequenz und Amplitude, mit der eine Masse an einer Feder schwingt.  
Da beide Gleichungen identische Struktur aufweisen, ergibt sich für \eqref{eq:ODE_Wellengleichung} 
\begin{equation}
	a_n(t) = A \cos(n \omega c t) + B \sin(n \omega c t)
	\quad\text{oder}\quad
	b_n(t) = A \cos(n \omega c t) + B \sin(n \omega c t).
\end{equation}
Die Resultate sind als eine unendliche Anzahl von Felderpendeln zu intepretieren, da $n$ für jede natürliche Zahl steht. 
Die Kreisfrequenz ist $n \omega c$. 
Ein $a_0\neq0$ fügt lediglich einen konstanten oder linear wachsenden Versatz hinzu, Frequenz und Form der Schwingungen bleiben davon jedoch unbeeinflusst.
Dies macht mathematisch Sinn, physikalisch müsste man sich jedoch fragen, was genau hin und her pendelt und ob eine unendliche Frequenz möglich ist?



In diesem Kapitel wurde die Thematik vereinfacht dargestellt. 
Anstelle eines dreidimensionalen Feldes $u(x,y,z,t)$ wurde lediglich eine eindimensionale Funktion $u(x,t)$ betrachtet. Die Fourier-Reihe lässt sich auch auf mehrdimensionale Felder anwenden, der damit verbundene mathematische Aufwand ist jedoch höher. 








