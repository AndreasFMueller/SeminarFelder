%
% kubisch.tex -- Kubische Approximationsfunktionen
%
% (c) 2021 Prof Dr Andreas Müller, OST Ostschweizer Fachhochschule
%
\documentclass[tikz]{standalone}
\usepackage{amsmath}
\usepackage{times}
\usepackage{txfonts}
\usepackage{pgfplots}
\usepackage{csvsimple}
\usetikzlibrary{arrows,intersections,math}
\definecolor{darkred}{rgb}{0.8,0,0}
\begin{document}
\def\skala{1}
\begin{tikzpicture}[>=latex,thick,scale=\skala,
declare function = {
	Hl(\X) = (1-2*\X)*(1+\X)*(1+\X);
	Hr(\X) = (1+2*\X)*(1-\X)*(1-\X);
	H1r(\X) = \X*(1-\X)*(1-\X);
	H1l(\X) = \X*(1+\X)*(1+\X);
}]

\begin{scope}[xshift=-3.1cm]
\draw[->] (-2.8,0) -- (3.1,0) coordinate[label={above left:$x$}];
\draw[->] (0,-0.5) -- (0,2.3) coordinate[label={right:$H(x)$}];
\fill (-2,0) circle[radius=0.05];
\fill (0,0) circle[radius=0.05];
\fill (2,0) circle[radius=0.05];
\draw[color=darkred,line width=1.2pt]
	(-2.9,0) -- (-2,0)
	-- plot[domain=-1:0,smooth] ({2*\x},{2*Hl(\x)})
	-- plot[domain=0:1,smooth] ({2*\x},{2*Hr(\x)})
	-- (2.9,0);
\end{scope}

\begin{scope}[xshift=3.1cm]
\draw[->] (-2.8,0) -- (3.1,0) coordinate[label={above left:$x$}];
\draw[->] (0,-0.5) -- (0,2.3) coordinate[label={right:$H^1(x)$}];
\fill (-2,0) circle[radius=0.05];
\fill (0,0) circle[radius=0.05];
\fill (2,0) circle[radius=0.05];
\draw[color=darkred,line width=1.2pt]
	(-2.9,0) -- (-2,0)
	-- plot[domain=-1:0,smooth] ({2*\x},{2*H1l(\x)})
	-- plot[domain=0:1,smooth] ({2*\x},{2*H1r(\x)})
	-- (2.9,0);
\end{scope}

\end{tikzpicture}
\end{document}

