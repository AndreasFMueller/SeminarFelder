%
% einleitung.tex -- Beispiel-File für die Einleitung
%
% (c) 2020 Prof Dr Andreas Müller, Hochschule Rapperswil
%
% !TEX root = ../../buch.tex
% !TEX encoding = UTF-8
%
\section{Einleitung Zu Open Foam\label{openfoam:section:Einleitung}}
\kopfrechts{Einleitung Zu Open Foam}
\subsection{Ziel und Inhalt unserer Arbeit}

\subsection{Was ist eigentlich OpenFoam?\label{openfoam:section:WasIstOpenFoam}}
Open Foam ist, wie der Name sagt, eine Open Source Software die unter der GNU
General Public License Kostenlos zugänglich ist und daurch auch für Komerzielle zwecke 
verwendet werden kann. 
Bei Open Foam handelt es sich um eine \emph{CFD} Software, 
dies steht für \emph{C}omputational \emph{F}luid \emph{D}ynamics also Numerische Strömungsdynamik.
Open Foam kann dementsprechend verwendet werden um näherungsweise lösungen für 
Strömungsmechanische Probleme zu finden.
Es können zum Beispiel Wiedrstandsbeiwerte von Autos oder die Wirbelschleppen einer Tragfläche simuliert werden.
Open Foam wird seit 2004 hauptsächlich von OpenCFD Ltd entwickelt, es wird jedoch auch aktiv durch die Community weiterentwickelt.
Die getesteten releases werden alle 6 Monate herausgegeben.~\cite{openfoam:greenshieldsweller2022}
%TODO: add OpenFoam reference
\subsection{Was macht Open Foam?}
Wie in~\ref{openfoam:section:WasIstOpenFoam} erwänt ist Open Foam eine Sofwtare für das lösen von Strömungsdynamischen Problemen.
Open foam kann also die Bewegung und Kräfte die durch Flüssigkeiten ausgehen berechnen, 
dies ist heute aber auch noch mit Wärmetransfer also Thermodynamik, Chemische Prozesse wie Verbrennung
und die Wechselwirkung von Festkörpern und Flüssigkeiten zu erweitern. 
All diese Prozesse kann eine CFD Software mittels Numerischen Methoden bestimmen. 
Zur anwendung kommen hierbei, unter anderem, die Finite Element Methode, die Finite Volumen Methode und die Finite Differenzen Methode.
Dies sind alles Verfahren zum lösen von Differenzialgleichungen.
Solche Verfahren werden dazu genutz um die vom Nutzer gegebenen Parameter, wie Funktionen die das Verhalten von Druck und Volumen einer Flüssigkeit,
die Region in der sich die Flüssigkeit befindet und Initial und Randbedingungen beschreiben, zu verrechnen und dadurch eine Lösung generiert, 
mit der das Verhalten dieser Flüssigkeit bestimmt werden kann.
~\cite{openfoam:greenshieldsweller2022}