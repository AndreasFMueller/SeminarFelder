%
% 1d.tex -- eindimensionale Diskretisierung
%
% (c) 2025 Prof Dr Andreas Müller, OST Ostschweizer Fachhochschule
%
\documentclass[tikz]{standalone}
\usepackage{amsmath}
\usepackage{times}
\usepackage{txfonts}
\usepackage{pgfplots}
\usepackage{csvsimple}
\usetikzlibrary{arrows,intersections,math}
\definecolor{darkred}{rgb}{0.8,0,0}
\begin{document}
\def\skala{1}
\def\punkt#1{
	\fill[color=white] #1 circle[radius=0.07];
	\draw[color=darkred,line width=1.2pt] #1 circle[radius=0.07];
}
\begin{tikzpicture}[>=latex,thick,scale=\skala,
declare function = {
	f(\X) = 0.7*(2*sin(50*\X) - 0.1*(\X-4) * (\X-4) + 0.6*\X + 2);
}]

\draw[->] (-0.1,0) -- (11.4,0) coordinate[label={above left:$x\mathstrut$}];
\draw[->] (0,-0.1) -- (0,5.4) coordinate[label={right:$y$}];

\node at (0,0) [below] {$a\mathstrut$};
\node at (0.5,0) [above] {$h\mathstrut$};
\node at (1,0) [below] {$x_1\mathstrut$};
\node at (1.5,0) [above] {$h\mathstrut$};
\node at (2,0) [below] {$x_2\mathstrut$};

\node at (4,0) [below] {$x_{k-2}\mathstrut$};
\node at (5,0) [below] {$x_{k-1}\mathstrut$};
\node at (6,0) [below] {$x_k\mathstrut$};
\node at (7,0) [below] {$x_{k+1}\mathstrut$};
\node at (8,0) [below] {$x_{k+2}\mathstrut$};
\node at (9,0) [below] {$x_{k+3}\mathstrut$};

\node at (4.5,0) [above] {$h\mathstrut$};
\node at (5.5,0) [above] {$h\mathstrut$};
\node at (6.5,0) [above] {$h\mathstrut$};
\node at (7.5,0) [above] {$h\mathstrut$};

\node at (10,0) [below] {$x_{n-1}\mathstrut$};
\node at (10.5,0) [above] {$h\mathstrut$};
\node at (11,0) [below] {$b\mathstrut$};

\draw[color=darkred,line width=1.2pt]
	plot[domain=-0.1:11.1,smooth] ({\x},{f(\x)});

\draw[color=blue,line width=1.4pt] (5,{f(5)}) -- (6,{f(6)}) -- (7,{f(7)});

\node[color=blue] at (6.35,{0.5*f(6)+0.5*f(7)}) [below right] {$y'_k$};
\node[color=blue] at (5.1,{0.8*f(5)+0.2*f(6)}) [below right] {$y'_{k-1}$};

\foreach \x in {0,1,2,4,5,6,7,8,10,11}{
	\draw[color=red,line width=0.2pt] (\x,0) -- (\x,{f(\x)});
	\punkt{(\x,{f(\x)})}
	\draw (\x,-0.05) -- (\x,0.05);
}

\node[color=darkred] at (1,{f(1)}) [above left] {$y_1$};
\node[color=darkred] at (2,{f(2)}) [above left] {$y_2$};
\node[color=darkred] at (4,{f(4)}) [above right] {$y_{k-2}$};
\node[color=darkred] at (5,{f(5)}) [above] {$y_{k-1}$};
\node[color=darkred] at (6,{f(6)}) [above] {$y_k$};
\node[color=darkred] at (7,{f(7)}) [above left] {$y_{k+1}$};
\node[color=darkred] at (8,{f(8)}) [above left] {$y_{k+2}$};
\node[color=darkred] at (10,{f(10)}) [above right] {$y_{n-1}$};
\node[color=darkred] at (11,{f(11)}) [above right] {$y_{n}$};

\end{tikzpicture}
\end{document}

