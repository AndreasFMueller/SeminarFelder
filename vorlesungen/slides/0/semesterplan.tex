%
% semesterplan.tex -- Semesterplan
%
% (c) 2021 Prof Dr Andreas Müller, OST Ostschweizer Fachhochschule
%
\bgroup
\definecolor{darkgreen}{rgb}{0,0.6,0}
\definecolor{darkred}{rgb}{0.8,0,0}
\def\w#1#2{#1{\tiny/#2}}
\begin{frame}[t]
\setlength{\abovedisplayskip}{5pt}
\setlength{\belowdisplayskip}{5pt}
%\frametitle{Semesterplan}
\vspace*{-3pt}
\begin{center}
\hspace*{-4pt}
\begin{tabular}{|r|r|l|}
\hline
\rowcolor{black}
\color{white}\bf Wo&
\color{white}\bf Datum&
\color{white}\bf Inhalt\raisebox{3pt}{\strut}
\\[1pt]
\hline
\rowcolor{darkgreen!10}
\color<1>{darkred} \w{1}{\phantom{0}8}&
\color<1>{darkred} 17.~2.&
\color<1>{darkred}Koordinaten \raisebox{2pt}{\strut}
\\
\rowcolor{darkgreen!10}
\color<2>{darkred} \w{2}{\phantom{0}9}&
\color<2>{darkred} 24.~2.&
\color<2>{darkred}Tangentialvektoren, 1-Formen, Wegintegral
\\
\rowcolor{darkgreen!10}
\color<3>{darkred} \w{3}{10}&
\color<3>{darkred}  3.~3.&
\color<3>{darkred}Wegabhängigkeit, 2-Vektoren und 2-Formen, Satz von Green
\\
\rowcolor{darkgreen!10}
\color<4>{darkred} \w{4}{11}&
\color<4>{darkred} 10.~3.&
\color<4>{darkred}Integration von 2-Formen, Satz von Stokes, äussere Ableitung
\\
\rowcolor{darkgreen!10}
\color<5>{darkred} \w{5}{12}&
\color<5>{darkred} 17.~3.&
\color<5>{darkred}$p$-Formen, Divergenz, Satz von Gauss
\\
\rowcolor{darkgreen!10}
\color<6>{darkred} \w{6}{13}&
\color<6>{darkred} 24.~3.&
\color<6>{darkred}Hodge-Operator, Laplace-Operator, Feldgleichungen der Physik
\\
\rowcolor{darkgreen!10}
\color<7>{darkred} \w{7}{14}&
\color<7>{darkred} 31.~3.&
\color<7>{darkred}Paralleltransport, Zusammenhang, Geodäten
\\
\rowcolor{gray!40}
\color{white}\w{  }{15}&
\color{white}  7.~4.&
\color{white}keine Vorlesung, Frühjahrsferien
\\
\rowcolor{darkgreen!10}
\color<8>{darkred}\w{8}{16}&
\color<8>{darkred} 14.~4.&
\color<8>{darkred}Krümmung
\\
\rowcolor{gray!40}
\color{white}\w{  }{17}&
\color{white} 21.~4.&
\color{white}keine Vorlesung, Ostermontag
\\
\rowcolor{blue!10}
\color<9>{darkred}\w{9}{18}&
\color<9>{darkred} 28.~4.&
\color<9>{darkred}Vorträge 1: Geometrische Algebra, Geostrophische Näherung
\\
\rowcolor{blue!10}
\color<10>{darkred}\w{10}{19}&
\color<10>{darkred}  5.~5.&
\color<10>{darkred}Vorträge 2: Überschall, RANS
\\
\rowcolor{blue!10}
\color<11>{darkred}\w{11}{20}&
\color<11>{darkred} 12.~5.&
\color<11>{darkred}Vorträge 3: Wirbelringe, Elastizitätstheorie
\\
\rowcolor{blue!10}
\color<12>{darkred}\w{12}{21}&
\color<12>{darkred} 19.~5.&
\color<12>{darkred}Vorträge 4: Nervenfasern, Waves to particles
\\
\rowcolor{blue!10}
\color<13>{darkred}\w{13}{22}&
\color<13>{darkred} 26.~5.&
\color<13>{darkred}Vorträge 5: OpenFOAM, Parallelisierung, neuronale Lösung
\\
\rowcolor{blue!10}
\color<14>{darkred}\w{14}{23}&
\color<14>{darkred}  2.~6.&
\color<14>{darkred}Vorträge 6: Vierdimensional, Fourier. Abschlusssitzung
\\[1pt]
\hline
\end{tabular}
\end{center}
\end{frame}
\egroup
