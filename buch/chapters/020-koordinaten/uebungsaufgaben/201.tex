Zwei Koordinatensysteme $(x^1,x^2)$ und $(y^1,y^2)$ in der Ebene sind
durch die Gleichungen
\begin{equation}
\renewcommand\arraycolsep{2pt}
\begin{array}{rcrcr}
y^1 &=&  x^1 &+& 2x^2 \\
y^2 &=& 2x^1 &-&  x^2
\end{array}
\label{buch:201:transformation}
\end{equation}
\begin{teilaufgaben}
\item
Berechnen Sie die Jacobi-Matrix dieser Koordinatentransformation.
\item
Zeichnen Sie die $(y^1,y^2)$-Koordinatenlinien im
$(x^1,x^2)$-Koordinatensystem ein.
\item
Betrachten Sie die Kurve $\gamma(t) = (x^1(t),x^2(t))$ im
$(x^1,x^2)$-Koordinatensystem mit
\begin{align*}
x^1(t) &= \cos t \\
x^2(t) &= \sin t
\end{align*}
Berechnen Sie die Komponenten des Tangentialvektors im
$(x^1,x^2)$-Koordinaten\-sys\-tem.
\item
Verwenden Sie die Transformation \eqref{buch:201:transformation}
um eine Darstellung der Kurve im $(y^1,y^2)$-Koordinaten\-system
zu erhalten.
\item
Berechnen Sie die Komponenten des Tangentialvektors in einem
Punkt der Kurve im $(y^1,y^2)$-Koordinatensystem und rechnen Sie
nach, dass diese Komponenten auch durch Anwendung der Jacobi-Matrix
auf die in Teilaufgabe c) gefundenen Komponenten berechnet werden
können.
\end{teilaufgaben}

\begin{loesung}
\begin{teilaufgaben}
\item
Die Jacobi-Matrix ist
\[
J
=
\bgroup
\renewcommand\arraystretch{1.4}
\begin{pmatrix}
\frac{\partial y^1}{\partial x^1}& \frac{\partial y^1}{\partial x^2} \\
\frac{\partial y^2}{\partial x^1}& \frac{\partial y^2}{\partial x^2}
\end{pmatrix}
\egroup
=
\begin{pmatrix*}[r]
1 &  2 \\
2 & -1 
\end{pmatrix*}.
\]
\item
Die Koordinatenlinien mit konstanter $y^1$- bzw.~$y^2$-Koordinate
erfüllen die Gleichungen
\begin{align*}
 x^1 +          2 x^2 &= \operatorname{const} \\
\text{bzw.}\qquad
2x^1 - \phantom{1}x^2 &= \operatorname{const},
\end{align*}
dies sind Koordinatengleichungen von Geraden.
Die Normalen dieser Geraden sind
\[
\vec{n}_1 = \begin{pmatrix} 1 \\  2 \end{pmatrix}
\qquad\text{und}\qquad
\vec{n}_2 = \begin{pmatrix*}[r] 2 \\ -1 \end{pmatrix*}.
\]
\begin{figure}
\centering
\begin{tikzpicture}[thick,>=latex,scale=1]
\begin{scope}
\clip (-5,-4) rectangle (5,4);
\pgfmathparse{1/sqrt(5)}
\xdef\l{\pgfmathresult}
\foreach \n in {-21,...,21}{
	\draw[color=darkred]
		($\n*\l*\l*(1,2)-5*(2,-1)$) -- ($\n*\l*\l*(1,2)+5*(2,-1)$);
}
\foreach \n in {-21,...,21}{
	\draw[color=blue]
		($\n*\l*\l*(1,-2)-5*(2,1)$) -- ($\n*\l*\l*(1,-2)+5*(2,1)$);
}
\end{scope}
\draw[->] (-5.1,0) -- (5.5,0) coordinate[label={$x^1$}];
\draw[->] (0,-4.1) -- (0,4.3) coordinate[label={right:$x^2$}];
\foreach \x in {1,...,5}{
	\draw (\x,-0.05) -- (\x,0.05);
	\node at (\x,-0.05) [below] {$\x$};
	\draw (-\x,-0.05) -- (-\x,0.05);
	\node at (-\x,-0.05) [below] {$-\x$};
}
\foreach \y in {1,...,4}{
	\draw (-0.05,\y) -- (0.05,\y);
	\node at (-0.05,\y) [left] {$\y$};
	\draw (-0.05,-\y) -- (0.05,-\y);
	\node at (-0.05,-\y) [left] {$-\y$};
}
\end{tikzpicture}
\caption{Koordinatenlinien $y^1\in\mathbb{Z}$ in rot und 
$y^2\in\mathbb{Z}$ in blau
\label{buch:201:fig}}
\end{figure}%
Da Division durch die Länge der Normalenvektoren die Hessesche
Normalform ergibt, kann man schliessen, dass die Geraden mit
konstantem $y^1$ in einem Abstand am Nullpunkt vorbeigehen, 
der ein Vielfaches von $1/|\vec{n}_1|=1/\sqrt{5}$ ist.
Analog haben die Geraden mit konstantem $y_2$ einen Abstand,
der ein Vielfaches von $1/|\vec{n}_2|=1/\sqrt{5}$ ist.
Die Geraden sind in Abbildung~\ref{buch:201:fig} dargestellt.
\item
Die Ableitung der Kurvenkoordinaten ist
\begin{equation}
\dot{x}^1(t) = -\sin t
\qquad\text{und}\qquad
\dot{x}^2(t) = \cos t.
\label{201:eqn:x}
\end{equation}
\item
In den $(y^1,y^2)$-Koordinaten hat die Kurve die Koordinaten
\begin{equation}
(y^1(t), y^2(t))
=
(\cos t+2\sin t, 2\cos t - \sin t).
\label{201:eqn:y}
\end{equation}
\item
Der Tangentialvektor in $(x^1,x^2)$-Koordinaten wird durch
Ableiten von \eqref{201:eqn:x} von Teilaufgabe c) nach $t$
als
\[
\begin{pmatrix*}[r]
\dot{x}^1(t)\\
\dot{x}^2(t)
\end{pmatrix*}
=
\begin{pmatrix*}[r]
-\sin t\\
 \cos t
\end{pmatrix*}
\]
gefunden.
Anwendung der $J$-Matrix auf diesen Tangentialvektor ergibt
\[
\begin{pmatrix*}[r]
1 &  2 \\
2 & -1
\end{pmatrix*}
\begin{pmatrix*}[r]
-\sin t\\
 \cos t
\end{pmatrix*}
=
\left(
\renewcommand{\arraycolsep}{1.1pt}
\begin{array}{rcr}
 -\sin t &+& 2\cos t\\
-2\sin t &-&  \cos t
\end{array}
\right).
\]
Andererseits findet man durch Ableiten von \eqref{201:eqn:y} in Teilaufgabe d)
nach $t$ für
den Tangentialvektor im $(x^1,x^2)$-Koordinatensystem die
Komponenten
\[
\begin{pmatrix*}[r]
\dot{y}^1(t)\\
\dot{y}^2(t)
\end{pmatrix*}
=
\left(
\renewcommand{\arraycolsep}{1.1pt}
\begin{array}{rcr}
 -\sin t &+& 2\cos t\\
-2\sin t &-& \cos t
\end{array}
\right).
\]
Die beiden Resultate stimmen übrerein.
Die Ableitung nach $t$ wird also tatsächlich mit der Matrix $J$
von $x$-Koordinaten in $y$-Koordinaten umgerechnet.
\qedhere
\end{teilaufgaben}
\end{loesung}

