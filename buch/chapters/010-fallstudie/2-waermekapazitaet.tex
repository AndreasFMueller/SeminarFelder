%
% 2-waermekapazitaet.tex
%
% (c) 2024 Prof Dr Andreas Müller
%

%
% Wärmekapazität
%
\section{Wärmekapazität
\label{buch:fallstudie:waermekapazitaet}}
\kopfrechts{Wärmekapazität}
Wie schon angedeutet ist die Temperatur $T$ ein Mass für die im Körper
vorhandene kinetische Energie $U$.
Für kleine Temperaturdifferenzen kann man zunächst annehmen, dass
es einen linearen Zusammenhang zwischen Energie und Temperatur gibt.
Die Steigung 
\[
C
=
\frac{\partial U}{\partial T}
\]
heisst die {\em Wärmekapazität} des Körpers.

Je mehr Material zur Verfügung steht, desto grösser ist auch die Energie,
die bei einem Körper gespeichert werden kann.
Die Lokalität suggeriert, dass sich die Gesamtenergie eines Körpers 
additiv aus den Energiemengenn der einzelnen Teile zusammensetzt.
Teilt man durch die Masse eines solchen kleinen Teils, erhält man die
spezifische Wärmekapazität $c$.
Sie ist eine Eigenschaft des Materials, aus dem das Teil besteht,
und hängt nicht mehr von der Grösse des Körpers ab.
Es ist aber möglich, dass $c$ von weiteren Grössen wie dem Druck oder der
chemischen Zusammensetzung abhängt, die nicht homogen sein muss.
Insbesondere kann $c$ wieder eine Funktion der Raumkoordinaten sein.

Multipliziert man $c(x)$ mit der Dichte $\varrho(x)$ des Materials,
entsteht eine Grösse, mit der sich aus der Temperatur sofort die
Dichte der gespeicherten Energie berechnen lässt.
In einem gewählten Koordinatensystem mit den Koordinaten $(x_1,\dots,x_n)$
wird daher bekannt sein, wie gross das Volumen eines kleinen
Quaders mit Kantenlängen $\Delta x_1$, $\Delta x_2$ und $\Delta x_3$ 
bestimmt werden kann.
Schreiben wir $v(x) \Delta x_1\,\Delta x_2\,\Delta x_3$ für dieses 
Volumen, wird der Energieinhalt durch das Volumenintegral
\[
U
=
\int_V T(x) c(x)\varrho(x) v(x)\,dx_1\,dx_2\,dx_3
\]
gegeben.
Während die Faktoren $T(x) c(x) \varrho(x)$ nicht von der Wahl der
Koordinaten abhängt, hängt die Funktion $v(x)$ vom Koordinatensystem
ab.

\begin{aufgabe}
Die Theorie der Integration ist so zu formulieren, dass automatisch
sichergestellt ist, dass ein Koordinatenwechsel die mit dem Integralbegriff
formulierten Gesetzmässigkeiten nicht verändert.
\end{aufgabe}

