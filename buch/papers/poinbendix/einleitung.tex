\section{Einleitung} \label{poinbendix:section:einleitung}

Der Satz von Poincaré-Bendixson beschreibt mögliche Bahnkurven von zweidimensionalen dynamischen Systemen.
Unter einem dynamischen System versteht man die Bewegung durch ein Vektorfeld entlang der Vektoren.
Beschrieben wird dies durch ein Differentialgleichungssystem in den Koordinaten $x$ und $y$.

Unabhängig von der Natur des zweidimensionalen dynamischen Systems schränkt der Satz von Poincaré-Bendixson die möglichen Bahnkurven auf drei Fälle ein.
Somit können Bahnkurven in zwei Dimensionen kein chaotisches oder unberechenbares Verhalten zeigen.

Im ersten Abschnitt \ref{poinbendix:section:nullklinen} wird ein intuitives Verständnis über das verhalten solcher Bahnkurven mithilfe der sogenannten Nullklinen aufgebaut.
Dies hilft uns den eigentlichen Satz in \ref{poinbendix:section:poinbendix} besser zu verstehen.
Neben dem Satz von Poincaré-Bendixson finden sich in diesem Abschnitt auch Beispiele für alle drei Fälle.
