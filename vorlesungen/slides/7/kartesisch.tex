%
% template.tex -- slide template
%
% (c) 2021 Prof Dr Andreas Müller, OST Ostschweizer Fachhochschule
%
\bgroup
\begin{frame}[t]
\setlength{\abovedisplayskip}{5pt}
\setlength{\belowdisplayskip}{5pt}
\frametitle{Kartesische Koordinaten}
\vspace{-10pt}
\begin{columns}[t,onlytextwidth]
\begin{column}{0.48\textwidth}
\begin{enumerate}
\item<2-> Metrik: $\displaystyle
g_{ik} = \delta_{ik}
$
\item<3-> Inverse: $\displaystyle
g^{ik} = \delta^{ik}
$
\item<4-> Ableitungen: $\displaystyle
\frac{\partial g_{ik}}{\partial x^l} = 0
$
\item<5-> Christoffelsymbole erster Art
\[
\Gamma_{l,ik} = 0
\]
\item<6-> Christoffelsymbole zweiter Art
\[
\Gamma^j_{ik} = 0
\]
\end{enumerate}
\end{column}
\begin{column}{0.48\textwidth}
\uncover<7->{%
\begin{block}{Geodätengleichung}
\[
\ddot{x}^k = 0
\]
\uncover<8->{Lösungen sind lineare Funktionen
\[
x^k(t) = v^kt + x^k(0)
\]
der Zeit}\uncover<9->{, also Geraden}
\end{block}}
\end{column}
\end{columns}
\end{frame}
\egroup
