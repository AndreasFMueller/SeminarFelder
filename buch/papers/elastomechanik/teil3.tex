%
% teil3.tex -- Beispiel-File für Teil 3
%
% (c) 2020 Prof Dr Andreas Müller, Hochschule Rapperswil
%
% !TEX root = ../../buch.tex
% !TEX encoding = UTF-8
%
\section{Mathematische Herleitung der Feldgleichungen der linearen Elastizitätstheorie}
\label{elastomechanik:section:herleitung2}
\subsection{Einsteinsche Summationskonvention}
In der Tensorrechnung, wie sie in der Elastizitätstheorie verwendet wird, gilt die sogenannte Einsteinsche Summationskonvention. 
Diese Konvention besagt, dass über zweifach auftretende Indizes in einem Produkt automatisch summiert wird, ein explizites Summenzeichen ist nicht mehr notwendig.

Dies bedeutet beispielsweise:
	\begin{equation}
		a_i b_i = 
		\sum_i a_i b_i
	\end{equation}
Die Konvention wird insbesondere bei der Notation von Spannung-Dehnungs-Beziehungen verwendet. Ein zentrales Beispiel in der linearen Elastizität ist:
	\begin{equation}
		\sigma_{ij} = 
		C_{ijkl} \, \varepsilon_{kl}	
	\end{equation}
Hier wird über die Indizes $k$ und $l$ summiert (von 1 bis 3). 
Diese Kurzschreibweise spart Schreibaufwand und erhöht die Übersichtlichkeit komplexer tensorwertiger Gleichungen.

Die Einsteinsche Konvention ist insbesondere in der physikalischen Mechanik, Thermodynamik und Kontinuumsmechanik weit verbreitet.

Im Rahmen dieser Arbeit wird die Einsteinsche Summationskonvention verwendet. 
Das bedeutet: Wenn in einer Formel ein Index zweimal vorkommt (z.~B. $a_ib_i$), ist automatisch über diesen Index zu summieren, ohne dass ein explizites Summenzeichen angegeben wird.
\subsection{Kinematische Beziehungen und Verzerrungstensor}
Wir beschreiben die Verformung eines Punktes durch seinen Verschiebungsvektor $u_i(x)$ mit $i$ = 1,2,3 ($x$, $y$, $z$).
$u_1$ entspricht der Verschiebung in $x$-Richtung, $u_2$ beschreibt die Verschiebung in $y$-Richtung und $u_3$ beschreibt die Verschiebung in $z$-Richtung.
Ausgehend vom Verschiebungsfeld $u_i(x)$ definieren wir den Verzerrungstensor $\varepsilon_{ij}$ als symmetrisierten Gradienten des Verschiebungsfeldes:
	\begin{equation}
		\varepsilon_{ij} = 
		\frac{1}{2} \left( \frac{\partial u_i}{\partial x_j} + \frac{\partial u_j}{\partial x_i} \right)
	\end{equation}
Diese Definition gewährleistet die Symmetrie des Verzerrungstensors ($\varepsilon_{ij}$ = $\varepsilon_{ji}$) und erfasst sowohl Dehnungen ($i$ = $j$) als auch Scherungen ($i$ $\neq$ $j$).
Als tensorielle Schreibweise, sieht der Formel folgendermassen aus:
	\begin{equation}
		\boldsymbol{\varepsilon} = 
		\frac{1}{2} \left( \nabla \vec{u} + (\nabla \vec{u})^T \right)
	\end{equation}

\subsection{Konstitutives Gesetz für isotrope Materialien}
Das Hooke'sche Gesetz in tensorieller Form verknüpft Spannungen $\sigma_{ij}$ mit Verzerrungen $\varepsilon_{kl}$ zu:
	\begin{equation}
		\sigma_{ij} = 
	\lambda \delta_{ij} \varepsilon_{kk} + 2\mu \varepsilon_{ij}
	\end{equation}
Hierbei sind $\lambda$ und $\mu$ (Lamé-Konstanten) materialabhängige Parameter. Für den Zusammenhang mit technischen Konstanten gilt:
	\begin{equation}
		\mu = 
		G = 
		\frac{E}{2(1+\nu)}, \quad \lambda = 
		\frac{E \nu}{(1+\nu)(1-2\nu)}
	\end{equation}	
Als Deviatorische Darstellung ergibt das:
	\begin{equation}
		s_{ij} =
		2\mu e_{ij}
	\end{equation}
mit
	\begin{equation}
		e_{ij} = 
		\varepsilon_{ij} - \frac{1}{3} \varepsilon_{kk} \delta_{ij}
	\end{equation}

\subsection{Gleichgewichtsbedingungen}
Aus der Impulsbilanz für ein infinitesimales Volumenelement folgt:
	\begin{equation}
		\frac{\partial \sigma_{ij}}{\partial x_j} + f_i =
		0
\end{equation}
wobei $f_i$ Volumenkräfte darstellen. 
Diese Gleichung beschreibt das lokale Kräftegleichgewicht in allen Raumrichtungen.

\subsection{Navier-Cauchy-Gleichungen (Feldgleichungen)}
Durch Einsetzen des Materialgesetzes in die Gleichgewichtsbedingungen erhalten wir die fundamentalen Feldgleichungen:
	\begin{equation}
		\mu \nabla^2 u_i + (\lambda + \mu) \frac{\partial}{\partial x_i} (\nabla \cdot \vec{u}) + f_i =
		0
	\end{equation}
In Vektorenform ergibt das:
	\begin{equation}
		\mu \nabla^2 \vec{u} + (\lambda + \mu) \nabla (\nabla \cdot \vec{u}) + \vec{f} = 
		0
	\end{equation}

\subsection{Herleitung der Navier-Cauchy-Gleichungen}
Jetzt setzen wir die kinematischen, konstitutiven und Gleichgewichtsbeziehungen zusammen, um die vollständigen Feldgleichungen der linearen Elastizitätstheorie herzuleiten.
Wir ersetzen $\sigma_{ij}$ in die Gleichgewichtsbedingung:
	\begin{equation}
		\frac{\partial}{\partial x_j}\left( \lambda \delta_{ij} \varepsilon_{kk} + 2\mu \varepsilon_{ij} \right) + f_i = 
		0
	\end{equation}
Einzeln ausrechnen:
	\begin{equation}
		\lambda \frac{\partial \varepsilon_{kk}}{\partial x_i} + 2\mu \frac{\partial \varepsilon_{ij}}{\partial x_j} + f_i = 
		0
	\end{equation}
Jetzt drücken wir die Verzerrungen durch die Verschiebungen aus:
	\begin{equation}
		\varepsilon_{kk} = 
		\frac{\partial u_1}{\partial x_1} + \frac{\partial u_2}{\partial x_2} + \frac{\partial u_3}{\partial x_3} = 
		\nabla \cdot \vec{u}
	\end{equation}
	\begin{equation}
		\varepsilon_{ij} = 
		\frac{1}{2}\left( \frac{\partial u_i}{\partial x_j} + \frac{\partial 	u_j}{\partial x_i} \right)
	\end{equation}
Zwischenschritt: Ableitung von $\varepsilon_{ij}$ einsetzen:
	\begin{equation}
		\frac{\partial \varepsilon_{ij}}{\partial x_j} = 
		\frac{1}{2}\left( \frac{\partial^2 u_i}{\partial x_j \partial x_j} + \frac{\partial^2 u_j}{\partial x_i \partial x_j} \right)
	\end{equation}
und wegen Vertauschbarkeit der Ableitungen:
	\begin{equation}
		\frac{\partial^2 u_j}{\partial x_i \partial x_j} = 
		\frac{\partial}{\partial x_i}\left( \frac{\partial u_j}{\partial x_j} \right) = 	\frac{\partial}{\partial x_i}(\nabla \cdot \vec{u})
	\end{equation}
Somit:
	\begin{equation}
		2\mu \frac{\partial \varepsilon_{ij}}{\partial x_j} = 
		\mu \nabla^2 u_i + \mu 	\frac{\partial}{\partial x_i}(\nabla \cdot \vec{u})
	\end{equation}
Zusammen ergibt sich schliesslich:
	\begin{equation}
		\mu \nabla^2 u_i + (\lambda+\mu) \frac{\partial}{\partial x_i}(\nabla \cdot \vec{u}) + f_i = 
		0
	\end{equation}
In kompakter Vektorschreibweise lautet die Gleichung:
	\begin{equation}
		\mu \nabla^2 \vec{u} + (\lambda+\mu) \nabla (\nabla \cdot \vec{u}) + \vec{f} = 
		0
	\end{equation}
Dies sind die Navier-Cauchy-Gleichungen.
	
\subsection{Variationsformulierung der Elastizität}
Zusätzlich kann das Problem über ein Energieprinzip formuliert werden:

\textbf{Potentielle Energie:}
	\begin{equation}
		\Pi = 
		\int_V \left( \frac{1}{2} \sigma_{ij} \varepsilon_{ij} - f_i u_i \right) \, dV
	\end{equation}
	
\textbf{Interpretation:} Das System minimiert seine potentielle Energie im Gleichgewicht.

\textbf{Bemerkung:} Durch Variation $\delta \Pi = 0$ erhalten wir wieder die Navier-Cauchy-Gleichungen (Euler-Lagrange-Prinzip).

\bigskip

\textbf{Mathematische Herleitung:}
Das Prinzip der minimalen potentiellen Energie liefert einen alternativen Zugang:
	\begin{equation}
		\Pi = 
		\int_V \left( \frac{1}{2} \sigma_{ij} \varepsilon_{ij} - f_i u_i \right) \, dV - \int_{\partial V} 	t_i u_i \, dS
	\end{equation}
\textbf{Stationaritätbedingung:}
	\begin{equation}
		\delta \Pi = 
		0 \quad \Rightarrow \quad \text{Euler-Lagrange-Gleichungen entsprechen den Feldgleichungen}
	\end{equation}
\textbf{Schrittweise Variation:}
\begin{enumerate}
	\item Variation der Verzerrungsenergie:
	\begin{equation}
		\delta U = 
		\int_V \sigma_{ij} \, \delta \varepsilon_{ij} \, dV
	\end{equation}
	
	\item Variation der äusseren Arbeit:
	\begin{equation}
		\delta W = 
		\int_V f_i \, \delta u_i \, dV + \int_{\partial V} t_i \, \delta u_i \, dS
	\end{equation}
	
	\item Anwendung des Gauss'schen Satzes führt auf die schwache Form der Gleichgewichtsbedingungen.
\end{enumerate}

\subsection{Randbedingungen und Eindeutigkeit}
Für eine eindeutige Lösung benötigen wir Randwerte:
	\begin{enumerate}
		\item[\textbf{Dirichlet-Randbedingung}] (Verschiebungen vorgegeben)
			\begin{equation}
				u_i = 
				\bar{u}_i \quad \text{auf} \quad \partial V_u
			\end{equation}
		\item[\textbf{Naumann-Randbedingung}] (Spannungen oder Kräfte vorgegeben)
			\begin{equation}
				\sigma_{ij} n_j = 
				\bar{t}_i \quad \text{auf} \quad \partial V_t
			\end{equation}		
	\end{enumerate}
Der Satz von Kirchhoff garantiert die Eindeutigkeit der Lösung bei Vorgabe konsistenter Randbedingungen.

\subsection{Tensorielle Ableitungsschritte}
Ableitung der Navier-Cauchy-Gleichungen:
	\begin{equation}
		\frac{\partial \sigma_{ij}}{\partial x_j} = 
		\frac{\partial}{\partial x_j} \left( \lambda \delta_{ij} \varepsilon_{kk} + 2\mu \varepsilon_{ij} \right) = 
		\lambda \frac{\partial \varepsilon_{kk}}{\partial x_i} + 2\mu \frac{\partial \varepsilon_{ij}}{\partial x_j}	
		= \mu \nabla^2 u_i + (\lambda + \mu) \frac{\partial}{\partial x_i} (\nabla \cdot \vec{u})
	\end{equation}

\subsection{Energieerhaltung und Dissipation}
Für hyperelastische Materialien gilt die Energiebilanz:
	\begin{equation}
		\frac{d}{dt} \int_V \left( \frac{1}{2} \rho \dot{u}_i \dot{u}_i + \frac{1}{2} \sigma_{ij} \varepsilon_{ij} \right) \, dV = 
		\int_V f_i \dot{u}_i \, dV + \int_{\partial V} t_i \dot{u}_i \, dS
	\end{equation}
Diese Gleichung zeigt die Erhaltung mechanischer Energie bei Abwesenheit dissipativer Effekte.

\subsection{Anwendungsbeispiel: Ebener Verzerrungszustand}
Für \(\varepsilon_{33} = \gamma_{13} = \gamma_{23} = 0\) vereinfachen sich die Gleichungen zu:
	\begin{align}
		\sigma_{11} &=	\frac{E (1-\nu)}{(1+\nu)(1-2\nu)} \varepsilon_{11} + \frac{E \nu}{(1+\nu)(1-2\nu)} \varepsilon_{22}
		\sigma_{22} &=	\frac{E \nu}{(1+\nu)(1-2\nu)} \varepsilon_{11} + \frac{E (1-\nu)}{(1+\nu)(1-2\nu)} \varepsilon_{22}
		\sigma_{12} &=	\frac{E}{2(1+\nu)} \gamma_{12}
	\end{align}
