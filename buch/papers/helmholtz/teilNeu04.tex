%
% teilNeu04.tex -- 
%
% (c) 2020 Prof Dr Andreas Müller, Hochschule Rapperswil
%
% !TEX root = ../../buch.tex
% !TEX encoding = UTF-8
%
\section{Akustische Grundlagen und Feldtheorie
\label{helmholtz:section:akustische_Grundlagen}}
\kopfrechts{Akustische Grundlagen und Feldtheorie}

Um die Anwendung der Helmholtz-Zerlegung in der Akustik zu verstehen, bedarf es grundlegender Kenntnisse der physikalischen Akustik.
Hier werden die zentralen physikalischen Grössen Schalldruck, Schallschnelle und deren Zusammenhänge erläutert.
In vielen akustischen Aufgaben wird mit der komplexen Schreibweise gearbeitet, weshalb die Formeln vorzugsweise im Frequenzbereich dargestellt werden.

\subsubsection{Die komplexe Schreibweise in der Akustik}
\index{komplexe Schreibweise}%
In der physikalischen Akustik wird häufig mit komplexen Grössen gearbeitet, bei denen der Imaginärteil eine konkrete physikalische Bedeutung hat. Diese mathematische Darstellung ist nicht nur ein Rechentrick, sondern erlaubt eine präzise Beschreibung von Phasenbeziehungen zwischen akustischen Grössen:

\begin{itemize}
\item Der Realteil einer komplexen akustischen Grösse entspricht der tatsächlichen, messbaren physikalischen Grösse zu einem bestimmten Zeitpunkt.

\item Der Imaginärteil repräsentiert die um 90° phasenverschobene Komponente und beschreibt somit zeitliche Verzögerungen im harmonischen Schwingungsverhalten.
\end{itemize}
Diese komplexe Darstellung hat besondere Bedeutung für die Energiebetrachtung in akustischen Feldern worauf später noch genauer beschrieben wird.

Die physikalische Interpretation der komplexen Grössen wird besonders deutlich bei der Betrachtung der komplexen Schallintensität, wo der Realteil die aktive und der Imaginärteil die reaktive Energiekomponente beschreibt \cite{helmholtz:paper}.



\subsection{Grundbegriffe der physikalischen Akustik
\label{helmholtz:subsection:Grundbegriffe_Akustik}}

\subsubsection{Schalldruck $p$}
\index{Schalldruck}% 
Der Schalldruck ist eine der fundamentalen Grössen in der Akustik und beschreibt lokale Druckschwankungen im Medium:
 
\begin{itemize}
\item $P  (\boldsymbol{r})$ bezeichnet die Amplitude des Schalldrucks am Ort $\boldsymbol{r}$ und ist eine reelle Grösse.
\index{Amplitude}%
\item $\omega$ ist die Kreisfrequenz.
\index{Kreisfrequenz}%
\item $\phi  (\boldsymbol{r})$ beschreibt die Phase am Ort $\boldsymbol{r}$ und ist ebenfalls eine reelle Grösse.
\end{itemize}
 
Die zeitabhängige Darstellung des Schalldrucks lautet:
\begin{equation*}
p(\boldsymbol{r},t)
=
P(\boldsymbol{r})  e^{j( \omega t - \phi(\boldsymbol{r}))}\, [\si{\pascal}].
\end{equation*}
 
Im Frequenzbereich, auch Phasor-Darstellung genannt, bei einer festen
Frequenz $\omega$ vereinfacht sich dies zu:
\begin{equation*}
p(\boldsymbol{r}) = P(\boldsymbol{r})  e^{-j \phi (\boldsymbol{r})}.
\label{helmholtz:PhasorSchalldruck}
\end{equation*}
 
\subsubsection{Schallschnelle $\boldsymbol{u}$}
\index{Schallschnelle}% 
Die Schallschnelle $\boldsymbol{u}$ auch Teilchengeschwindigkeit
\index{Teilchengeschwindigkeit}%
genannt, wird aus dem Druckgradienten $\nabla p$ abgeleitet.
\index{Druckgradient}%
Für die linearisierte Form im Frequenzbereich $\boldsymbol{u} =
\frac{j}{\rho_0 \omega} \nabla p$ bei harmonischen Feldern
$\boldsymbol{u}(\boldsymbol{r},t)$ ergibt sich:
\begin{equation*}
\boldsymbol{u}(\boldsymbol{r},t)
=
\frac{1}{\omega \rho} \nabla p(\boldsymbol{r},t)
=
\frac{1}{\omega \rho}
\bigl(
\nabla P(\boldsymbol{r}) 
-
jP(\boldsymbol{r})
\nabla\phi(\boldsymbol{r})
\bigr)
e^{j(\omega t -\phi(\boldsymbol{r}))}
\; [\si{\metre / \second}].
\end{equation*}
Dabei sind:
\begin{itemize}
\item
$\nabla \phi (\boldsymbol{r})$ der Gradient der Phase zeigt in
Richtung der steilsten Phasenänderung, senkrecht zur Wellenfront.
\item
$\nabla P (\boldsymbol{r})$ der Gradient der Druckamplitude zeigt
in Richtung der schnellsten Amplitudenänderung.
\item
$\rho$ die Dichte des Mediums.
\end{itemize}
Als Phasor bei einer fixen Frequenz $\omega$ ergibt sich die
Schallschnelle als
\begin{equation*}
\boldsymbol{u}(\boldsymbol{r})
=
\frac{1}{\omega \rho}  \bigl(
\nabla P(\boldsymbol{r})  - j P(\boldsymbol{r}) \nabla\phi(\boldsymbol{r})
\bigr)
e^{-j\phi (\boldsymbol{r})}.
%\label{helmholtz:PhasorSchallschnelle}
\end{equation*}
 
%\subsubsection{Wellengleichung und Helmholtz-Gleichung}
 
%In der linearen Akustik sind Schalldruck und Schallschnelle durch folgende Grundgleichungen verbunden:
 
%\begin{itemize}
%\item Die Euler-Gleichung %(Impulserhaltung):
%\begin{equation}
%\rho_0 \frac{\partial \boldsymbol{u}}{\partial t} = -\nabla p
%\end{equation}
 
%\item Die Kontinuitätsgleichung %(Massenerhaltung):
%\begin{equation}
%\frac{\partial \rho}{\partial t} + \rho_0 \nabla \cdot \boldsymbol{u} = 0
%\end{equation}
%\end{itemize}
 
%\noindent Durch Kombination dieser Gleichungen erhält man die Wellengleichung für den Schalldruck:
%\begin{equation}
%\nabla^2 p - \frac{1}{c^2}\frac{\partial^2 p}{\partial t^2} = 0
%\end{equation}
 
%\noindent Für zeitharmonische Anregung mit $e^{j\omega t}$ führt dies zur Helmholtz-Gleichung:
%\begin{equation}
%\nabla^2 p + k^2 p = 0
%\end{equation}
 
%\noindent wobei $k = \omega/c$ die Wellenzahl ist und $c$ die Schallgeschwindigkeit im Medium.

\subsection{Energiebetrachtungen in akustischen Feldern
\label{helmholtz:subsection:Energiebetrachtung}}
\index{akustische Energie}%
Die akustische Energie in einem Schallfeld setzt sich aus kinetischer
und potentieller Energie zusammen.
Die zeitlich gemittelte Energiedichte kann durch die Gleichung
\index{gemittelte Energiedichte}%
\begin{equation*}
\langle w \rangle
=
\frac{1}{4}\left(\frac{|p|^2}{\rho_0 c^2} + \rho_0 |\boldsymbol{u}|^2 \right)
\end{equation*}
ausgedrückt werden,
wobei der erste Term die potentielle Energiedichte durch die
Kompression des Mediums und der zweite Term die kinetische
Energiedichte durch die Bewegung der Teilchen darstellt.

Der Transport dieser Energie wird durch die Schallintensität
beschrieben.
Für harmonische Schallfelder betrachten wir die zeitlich
gemittelte aktive Intensität
\begin{equation*}
\boldsymbol{I}(\boldsymbol{r})
=
\frac{1}{T}\int_0^T \boldsymbol{I}_i(\boldsymbol{r},t)\,dt
=
\frac{1}{2}\Re (p(\boldsymbol{r}) \boldsymbol{u}^*(\boldsymbol{r})),
\end{equation*}
wobei $\boldsymbol{I}_i(\boldsymbol{r},t)$ die instantane
Schallintensität und $T$ die Periodendauer der Schwingung ist.
Der Stern bezeichnet die komplexe Konjugation.
Diese aktive Intensität beschreibt den Netto-Energiefluss und ist
für die Schallausbreitung von zentraler Bedeutung.

\subsubsection{Zeitabhängige und zeitgemittelte Grössen}

In der Akustik unterscheiden wir zwischen momentanen, zeitabhängigen Grössen und zeitgemittelten Grössen:

\begin{itemize}
\item \emph{Momentane Grössen} wie die instantane Intensität
$\boldsymbol{I}(\boldsymbol{r},t)$ beschreiben den augenblicklichen
Energietransport
an einem bestimmten Ort und zu einem bestimmten Zeitpunkt.
\index{momentan}%

\item \emph{Zeitgemittelte Grössen} wie die aktive Intensität
$\boldsymbol{I}(\boldsymbol{r})$ beschreiben den durchschnittlichen
Energietransport
über eine Periode der harmonischen Schwingung und sind besonders
wichtig für die Analyse der effektiven Energieübertragung in
akustischen Feldern.
\index{zeitgemittelt}%
\end{itemize}

Diese Unterscheidung ist von grosser Bedeutung, da in vielen praktischen Anwendungen nicht die instantanen, sondern die zeitlich gemittelten Grössen von Interesse sind.




