%Hast du gut gemacht
% einleitung.tex -- Beispiel-File für die Einleitung
%
% (c) 2020 Prof Dr Andreas Müller, Hochschule Rapperswil
%
% !TEX root = ../../buch.tex
% !TEX encoding = UTF-8
%
\section{Gebietsunterteilung}
\label{parallelisierung:sec:dd}

\subsection{Motivation}

Wie im Kapitel der numerischen Methoden gezeigt, führt eine Verfeinerung der Raum- und Zeitauflösung zwar zu genaueren Ergebnissen der Simulation, jedoch auch zu einem erheblich höheren Rechenaufwand. 
Mit zunehmender Raumgitter-Feinheit steigt die Anzahl der Unbekannten quadratisch (in 2D) bzw. kubisch (in 3D).  
Um diese steigende Komplexität effizient bewältigen zu können, bietet sich die Methode der \emph{Gebietsunterteilung} an.

Feldgleichungen wie die Wärmeleitungsgleichung haben wie bereits diskutiert eine lokale Struktur: 
der Wert in einem Gitterpunkt hängt nur von den direkten Nachbarn (räumlich und zeitlich) ab.  
Diese Eigenschaft ermöglicht es, ein großes Problem in mehrere kleinere Teilprobleme zu zerlegen, sie separat zu berechnen und anschließend wieder konsistent zusammenzufügen.



\subsection{Mathematische Zerlegung}

Sei $\Omega \subset \mathbb{R}^2$ das gesamte Gebiet, diskretisiert durch ein kartesisches Gitter.  
Wir zerlegen $\Omega$ in $P$ disjunkte Teilgebiete $\Omega_p$:
\begin{equation}
	\Omega = \bigcup_{p=1}^P \Omega_p,
	\qquad 
	\Omega_p \cap \Omega_q = \emptyset \quad \text{für } p \neq q.
\end{equation}

Das bedeutet:
\begin{itemize}
	\item Das Gesamtgebiet setzt sich vollständig aus den Teilgebieten zusammen.
	\item Jedes Gitterelement gehört genau zu einem Teilgebiet.
	\item Die Teilgebiete überlappen sich nicht, können sich aber an den Rändern berühren.
\end{itemize}

Anschaulich entspricht dies einem Puzzle: 
jedes Teilgebiet ist ein Puzzlestück, das nahtlos mit den Nachbarstücken zusammenpassen muss.

\subsection{Beispiel: 2D-Wärmeleitungsgleichung}

Betrachten wir erneut die zweidimensionale Wärmeleitungsgleichung
\begin{equation}
	\frac{\partial T}{\partial t} = 
	\alpha \left(
	\frac{\partial^2 T}{\partial x^2} + \frac{\partial^2 T}{\partial y^2}
	\right).
\end{equation}

Mit der Finite-Differenzen-Methode (FDM) ergibt sich wie in Kapitel \ref{parallelisierung:sec:numerische-methoden} gezeigt für ein quadratisches Gitter mit $\Delta x=\Delta y=\Delta l$ und $\lambda=\alpha\,\Delta t/(\Delta l)^2$ die explizite Update-Formel:
\begin{equation}
	T_{i,j}^{n+1}
	=
	T_{i,j}^{n}
	+
	\lambda \left(
	T_{i+1,j}^{n}+T_{i-1,j}^{n}+T_{i,j+1}^{n}+T_{i,j-1}^{n}-4\,T_{i,j}^{n}
	\right).
	\label{eq:update-dd}
\end{equation}

Für innere Punkte eines Teilgebiets $\Omega_p$ kann \eqref{eq:update-dd} direkt berechnet werden.  
An den Schnittkanten hingegen werden Werte aus benachbarten Teilgebieten benötigt.

\subsection{Kopplungsbedingungen an der Schnittkante}

Damit die Zerlegung konsistent bleibt, müssen an den Schnittstellen (also an den Rändern) zwischen Teilgebieten bestimmte Kopplungsbedingungen erfüllt sein:

\subsubsection*{Kontinuität der Lösung}
Die Temperatur muss an der Schnittkante stetig sein:
\begin{equation}
	T_{\Omega_p}(x,y,t) = T_{\Omega_q}(x,y,t)
	\qquad \forall (x,y) \in \partial \Omega_p \cap \partial \Omega_q.
\end{equation}

\subsubsection*{Kontinuität des Flusses}
Auch der Wärmefluss muss stetig sein.  
Nach Fourier gilt:
\begin{equation}
	- \kappa \, \nabla T_{\Omega_p} \cdot n
	=
	- \kappa \, \nabla T_{\Omega_q} \cdot n
	\qquad \forall (x,y) \in \partial \Omega_p \cap \partial \Omega_q.
\end{equation}
Physikalisch heißt das: Es darf keine künstliche Wärmequelle oder -senke an der Schnittkante entstehen.

\subsection{Diskrete Umsetzung mit Ghost-Zellen}

In der FDM-Diskretisierung wird dies durch \emph{Ghost-Zellen} realisiert:
\begin{itemize}
	\item Jedes Teilgebiet speichert an seinen Rändern eine zusätzliche Schicht von Gitterpunkten.
	\item Diese Ghost-Zellen enthalten die Werte aus den benachbarten Teilgebieten.
	\item Nach jedem Zeitschritt werden die Ghost-Zellen synchronisiert.
\end{itemize}

Dadurch kann jedes Teilgebiet seine Update-Formel lokal anwenden, ohne direkten Zugriff auf fremde Datenstrukturen zu benötigen.

\subsection{Praktische Aspekte}

\subsubsection {Lastverteilung und Kommunikation.}
Die Kosten pro Zeitschritt sind proportional zur Anzahl der Gitterpunkte im Teilgebiet 
($\mathcal{O}(|\Omega_p|)$).  
Die Kommunikationskosten entstehen durch den Austausch von Ghost-Zellen und hängen von der Länge der Schnittkante ab.  
Für eine effiziente Parallelisierung ist es daher vorteilhaft, Subdomains mit möglichst kleinem 
Umfang/Flächen-Verhältnis zu wählen.  
In 2D bedeutet dies, dass quadratische Teilgebiete günstiger sind als lange, schmale Streifen; 
im 3D-Fall spricht man analog vom Oberflächen/Volumen-Verhältnis.  

\subsubsection {Stabilität.}
Die Stabilitätskriterien der expliziten Finite-Differenzen-Methode gelten auch bei der Gebietsunterteilung unverändert.  
Insbesondere muss in 2D für ein quadratisches Gitter stets
\[
\lambda = \frac{\alpha \Delta t}{(\Delta l)^2} \leq \tfrac{1}{4}
\]
gelten.  
Diese Bedingung ist lokal in jedem Teilgebiet genauso einzuhalten wie global.  



\subsubsection{Stationärer Grenzfall und Matrixformulierung}

Für $n \to \infty$ strebt das explizite Zeitschrittverfahren \eqref{eq:update-dd} gegen ein stationäres Temperaturfeld. 
Dann verschwindet die zeitliche Ableitung, und die Wärmeleitungsgleichung reduziert sich im allgemeinen auf die Laplace-Gleichung


\begin{equation}
	\nabla^2 T
	= 
	0.
\end{equation}

Dieser speziealfall läst sich besonders einfach zu 
\begin{equation}
	\mathbf{A} \vec{u}
	=
	\vec{b}
\end{equation}
numerisch approximieren.
Um dies nachzuvollziehen betrachten wir ein kleines Beispiel.

\subsubsection*{Beispiel: $4\times 4$-Gitter mit Dirichlet-Randbedingungen}

Wir betrachten ein quadratisches Gitter mit $4\times 4$ Punkten.  
Die äußeren Werte (Rand) seien fest vorgegeben (z.\,B. $T=0$), sodass nur die \emph{inneren} Punkte Unbekannte sind.  
Das Gitter sieht dann schematisch so aus:

\[
\begin{array}{cccc}
	R & R & R & R \\
	R & I & I & R \\
	R & I & I & R \\
	R & R & R & R \\
\end{array}
\]

- $R =$ Randpunkt (bekannt)  
- $I =$ innerer Punkt (unbekannt)

Somit gibt es $2\times 2 = 4$ innere Unbekannte: 
\[
T_{2,2}, \quad T_{2,3}, \quad T_{3,2}, \quad T_{3,3}.
\]


Für einen inneren Punkt $(i,j)$ lautet die Diskretisierung:
\[
-4T_{i,j} + T_{i+1,j} + T_{i-1,j} + T_{i,j+1} + T_{i,j-1} = 0.
\]

Damit ergeben sich im Beispiel die folgenden Gleichungen:

\begin{align*}
	-4T_{2,2} + T_{3,2} + T_{2,3} &= 0, \\
	-4T_{2,3} + T_{3,3} + T_{2,2} &= 0, \\
	-4T_{3,2} + T_{2,2} + T_{3,3} &= 0, \\
	-4T_{3,3} + T_{2,3} + T_{3,2} &= 0.
\end{align*}

\subsubsection*{Matrixformulierung}

Wir ordnen die Unbekannten in einem Vektor
\[
\mathbf{u} =
\begin{bmatrix}
	T_{2,2} \\ T_{2,3} \\ T_{3,2} \\ T_{3,3}
\end{bmatrix}.
\]

Dann können die obigen Gleichungen in Matrixform geschrieben werden als:
\[
\underbrace{\begin{bmatrix}
		-4 &  1 &  1 &  0 \\
		1 & -4 &  0 &  1 \\
		1 &  0 & -4 &  1 \\
		0 &  1 &  1 & -4
\end{bmatrix}}_{A}
\begin{bmatrix}
	T_{2,2} \\ T_{2,3} \\ T_{3,2} \\ T_{3,3}
\end{bmatrix}
=
\underbrace{\begin{bmatrix}
		0 \\ 0 \\ 0 \\ 0
\end{bmatrix}}_{\mathbf{b}}.
\]

\subsubsection*{Interpretation}
\begin{itemize}
	\item Die Matrix $A$ enthält die typische Struktur der \emph{5-Punkt-Stern-Approximation}: 
	\item Auf der Diagonalen '$-4$' für den jeweiligen Punkt,
	\item '$-1$' an den Positionen der direkten Nachbarn.  
	\item Der rechte Seitenvektor $\vec{b}$ enthält die Beiträge der Randwerte.  
	In diesem Beispiel mit $T=0$ am Rand ist $\vec{b}=\vec{0}$.  
	Wären die Randwerte ungleich null, würden sie hier erscheinen. 
\end{itemize}
Damit sehen wir, wie aus der Laplace-Gleichung in diskreter Form ein lineares Gleichungssystem
\[
\mathbf{A} \vec{u}
=
\vec{b}
\]
entsteht.

\subsubsection*{Lösung des stationären Systems}

Im Beispiel oben gilt am gesamten Rand $T=0$, daher ist die rechte Seite $\vec{b}=\vec{0}$.
Das stationäre Gleichungssystem
\[
\mathbf{A}\,\vec{u}=\vec{b}
\]
hat wegen der streng diagonaldominanten, symmetrisch positiv definiten Matrix $A$ die eindeutige Lösung
\[
\vec{u}=\vec{0},
\]
d.\,h. alle vier inneren Temperaturen sind im stationären Zustand gleich null:
\[
T_{2,2}=T_{2,3}=T_{3,2}=T_{3,3}=0.
\]
Physikalisch klar: der Rand hält das System bei $0^\circ$C; durch reine Diffusion kann sich im Gleichgewicht kein anderes Niveau im Inneren halten.


\subsubsection{Bedeutung für Simulationen}

Die obige Herleitung zeigt, dass der stationäre Grenzfall auf ein lineares Gleichungssystem
\(
A \mathbf{u} = \mathbf{b}
\)
führt.  

\begin{itemize}
	\item Im \emph{zeitabhängigen Fall} erfolgt die Berechnung schrittweise durch       das Update-Schema 
	\eqref{eq:update-dd}. Die Kopplung der Teilgebiete geschieht dabei explizit durch den 
	Austausch von Ghost-Zellen an den Schnittkanten.  
	
	\item Im \emph{stationären Fall} löst man direkt das Gleichungssystem. 
	Durch die Gebietsunterteilung ergibt sich dabei eine Blockstruktur: 
	die innere Dynamik jedes Teilgebiets kann lokal berechnet werden, 
	während die globale Kopplung nur über die Werte an den Interfaces berücksichtigt werden muss.  
	Effiziente Verfahren (z.\,B. Schur-Komplement oder iterative Domänendecomposition-Methoden) 
	nutzen diese Struktur gezielt aus, um den Kommunikationsaufwand zu reduzieren.
\end{itemize}


\medskip
\noindent
Damit zeigt sich: unabhängig davon, ob man eine zeitabhängige oder stationäre Simulation betrachtet, 
bleibt die Grundidee gleich – die Wechselwirkung zwischen Teilgebieten beschränkt sich auf die Schnittkanten. 
Diese Lokalität ist die Grundlage für effiziente Parallelisierung.


