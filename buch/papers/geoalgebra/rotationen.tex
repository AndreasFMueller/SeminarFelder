\section{Rotationen}
\renewcommand{\u}{\hat{\mathbf{u}}}
\subsection{Rotationen sind zwei aufeinanderfolgende Spiegelungen}
Eine Rotation kann durch zwei aufeinanderfolgende Spiegelungen ausgedrückt werden (siehe
\autoref{geoalgebra:fig:rotation-as-two-reflections}).
Die Rotation wird am Drehpunkt $O$ durchgeführt, dem Schnittpunkt der beiden Vektoren $\u, \v$.
\begin{figure}
  \begin{center}
\begin{tikzpicture}[>=latex, scale=2]
\fill[green!30] (-0.5, 0.5) -- (-0.5, 2) -- (-0.1, 0.9) -- cycle;
\draw[thick, ->] (0, 0) -- (0.5, 2) node[left]{$\v$};
\draw[thick, ->] (0, 0) -- (2, 0.5) node[above]{$\u$};
\fill[green!30] (0.68, 0.21) -- (1.38, 1.53) -- (0.51, 0.75) -- cycle;
\fill[green!30] (0.69, 0.13) -- (1.94, -0.70) -- (0.80, -0.42) -- cycle;
\draw[thick, blue] (-0.5, 0.5) -- (0.68, 0.21);
\draw[thick, blue] (-0.5, 2) -- (1.38, 1.53);
\draw[thick, blue] (-0.1, 0.9) -- (0.51, 0.75);

\draw[thick, red] (0.68, 0.21) -- (0.69, 0.13);
\draw[thick, red] (1.38, 1.53) -- (1.94, -0.70);
\draw[thick, red] (0.51, 0.75) -- (0.80, -0.42);
\end{tikzpicture}

  \end{center}
  \caption{Eine Rotation kann durch zwei aufeinanderfolgende Spiegelungen abgebildet werden}
\label{geoalgebra:fig:rotation-as-two-reflections}
\end{figure}

Dasselbe funktioniert auch in höheren Dimensionen. In drei Dimensionen hätten wir es mit einem
Rotationsvektor zu tun. Dieser Rotationsvektor ist orthogonal zu $\u$ und $\v$, zum Beispiel $\u \times \v$,
wobei die Länge irrelevant ist. Relevant ist ausserdem noch eine fixe Position des Vektors, in diesem Fall
$O$.

\subsection{Rotation eines Vektors}
Man betrachte den Vektor $\mathbf{x}$, der zunächst an $\v$ und danach an $\u$ mit Zwischenwinkel $\theta$ gespiegelt wird (siehe \autoref{geoalgebra:fig:rotation}).
Die Spiegelung $\mathbf{x}'$ von $\mathbf{x}$ an $\v$ ergibt sich durch
\begin{equation}
\mathbf{x}' = \v \mathbf{x} \v.
\end{equation}
Analog dazu kann der neu entstandene Vektor $\mathbf{x}'$ an $\u$ gespiegelt werden
\begin{equation}
\mathbf{x}'' = \u \mathbf{x}' \u.
\end{equation}
Die ganze Rotation wird somit durch
\begin{equation}
\mathbf{x}'' = \u \v \mathbf{x} \v \u
  \label{geoalgebra:eq:rotation}
\end{equation}
berechnet. Der Rotationswinkel beträgt dabei genau $2\theta$.
\begin{figure}
  \begin{center}
\begin{tikzpicture}[>=latex, scale=2]
\draw[thick, darkgreen, ->] (0, 0) -- (-0.5, 2) node[left]{$\mathbf{x}$};
\draw[thick, ->] (0, 0) -- (0.5, 2) node[left]{$\v$};
\draw[thick, ->] (0, 0) -- (2, 0.5) node[above]{$\u$};;
\draw[thick, darkgreen!30, ->] (0, 0) -- (1.38, 1.53) node[left]{$\mathbf{x}'$};
\draw[thick, darkgreen!30, ->] (0, 0) -- (1.94, -0.70) node[below]{$\mathbf{x}''$};
\draw[thick, blue] (-0.5, 2) -- (1.38, 1.53);
\draw[thick, darkred] (1.38, 1.53) -- (1.94, -0.70);
\end{tikzpicture}


  \end{center}
  \caption{Rotation eines Vektors}
\label{geoalgebra:fig:rotation}
\end{figure}
Verglichen mit der Drehmatrix in zwei Dimensionen
\begin{equation}
  R_{2\theta} = \begin{pmatrix}
    \cos{2\theta} & -\sin{2\theta} \\
    \sin{2\theta} & \cos{2\theta}
  \end{pmatrix}
\end{equation}
ist dieses Resultat sehr elegant und dazu gültig in \emph{beliebigen Dimensionen}!

\subsection{Rotoren}
Bei den bisherigen Rotationen ergibt sich das Problem, dass zwei Vektoren
in bestimmten Richtungen nötig sind, die die Rotation bestimmen. Möchten wir
jedoch um einen beliebigen Vektor $\theta$ rotieren, wird ein weiteres Konzept
benötigt: \emph{Rotoren}.

Ein Rotor $R$ ist definiert als
\begin{equation}
  R = \u \v,
\end{equation}
wobei $\u$ und $\v$ Einheitsvektoren sind.
Das Gegenstück zu $R$ ist
\begin{equation}
  \tilde{R} = \v \u.
\end{equation}
Der Ausdruck \eqref{geoalgebra:eq:rotation} kann mithilfe des Rotors $R = \u \v$ somit als
\begin{equation}
  \mathbf{x}'' = R \mathbf{x} \tilde{R}
\end{equation}
dargestellt werden.
Es gilt also
\begin{align}
  R &= \u \v = \u \cdot \v - \v \wedge \u \\
  \tilde{R} &= \v \u = \u \cdot \v + \v \wedge \u.
\end{align}
Das Skalarprodukt von zwei Einheitsvektoren $\u$ und $\v$ ist äquivalent zum Kosinus des Winkels dazwischen:
\begin{equation}
  \u \cdot \v = \cos{\theta}.
\end{equation}

Das Wedgeprodukt zwischen zwei Vektoren bildet wie bereits bekannt den Bivektor dazwischen, d.h. die orientierte Fläche.
Da $\u$ und $\v$ Einheitsvektoren sind, beträgt die Fläche dazwischen genau $\sin{\theta}$. Für das Wedgeprodukt gilt daher
\begin{equation}
  \v \wedge \u = \sin{\theta} \cdot \hat{B},
\end{equation}
wobei $\hat{B}$ ein Einheitsbivektor ist. In zwei Dimensionen bietet sich hier natürlich $\eot$ an.
Daher sind $R$ und $\tilde{R}$:
\begin{align}
  R = \cos{\theta} - \sin{\theta} \cdot \eot \\
  R = \cos{\theta} + \sin{\theta} \cdot \eot.
\end{align}
Dies erzeugt jedoch wie gesehen die Rotation um den doppelten Winkel $2 \theta$. Wir ersetzen $\alpha = 2\theta$ und erhalten die Rotation um
den tatsächlichen Vektor $\alpha$
\begin{equation}
  \mathbf{x}'' = (\cos{\frac{\alpha}{2}} - \sin{\frac{\alpha}{2}} \eot) \mathbf{x} (\cos{\frac{\alpha}{2}} + \sin{\frac{\alpha}{2}} \eot).
\end{equation}


\begin{beispiel}
Angenommen,
\begin{align}
  \mathbf{x} &= \eone + \etwo \\
  \alpha &= \frac{\pi}{2}.
\end{align}
Somit gilt
\begin{equation}
  \newcommand{\fraction}{\frac{\sqrt{2}}{2}}
  \begin{aligned}
    \mathbf{x}'' &= \left( \cos{\frac{\alpha}{2}} - \sin{\frac{\alpha}{2}} \eot \right) (\eone + \etwo) \left(\cos{\frac{\alpha}{2}} + \sin{\frac{\alpha}{2}} \eot\right) \\
    &= \left( \fraction - \fraction \eot \right) (\eone + \etwo) \left( \fraction + \fraction \eot \right) \\
    &= \left( \fraction \eone + \fraction \etwo + \fraction \etwo - \fraction \eone \right) \left( \fraction + \fraction \eot \right) \\
    &= \sqrt{2} \etwo \left( \fraction + \fraction \eot \right) \\
    &= -\eone + \etwo,
  \end{aligned}
\end{equation}
was genau der Rotation entspricht!
\end{beispiel}
Diese Rotationen funktionieren in beliebigen Dimensionen, in drei Dimensionen und höher muss jedoch darauf geachtet werden,
dass man die Ebene für die Drehbasis korrekt wählt. Dies drückt sich durch den gewählten Einheitsbivektor $\hat{B}$ aus.

\begin{beispiel}
Angenommen, wir möchten den Vektor
  \begin{equation}
    \mathbf{x} = \begin{pmatrix} 2 \\ 3 \\ 4\end{pmatrix} = 2 \eone + 3 \etwo + 4 \ethree
  \end{equation}
an der Ebene $\eot$
  \begin{equation}
    \eot = \eone \wedge \etwo
  \end{equation}
um den Drehwinkel
\begin{equation}
  \alpha = \frac{\pi}{3}
\end{equation}
drehen.

Somit ist
  \begin{equation}
    \newcommand{\fraca}{\frac{\sqrt{3}}{2}}
    \newcommand{\fracb}{\frac{1}{2}}
    \begin{aligned}
    \mathbf{x}'' &= \left( \cos{\frac{\alpha}{2}} - \sin{\frac{\alpha}{2}} \eot \right) (2 \eone + 3 \etwo + 4 \ethree) \left(\cos{\frac{\alpha}{2}} + \sin{\frac{\alpha}{2}} \eot\right) \\
      &= \left(\fraca - \fracb \eot \right) (2\eone + 3 \etwo + 4 \ethree) \left(\fraca + \fracb \eot \right) \\
      &= \left(\sqrt{3} \eone + \frac{3\sqrt{3}}{2} \etwo + 2\sqrt{3}\ethree + \etwo - \frac{3}{2}\eone - 2 \mathbf{e}_{123}\right) \left(\fraca + \fracb \eot\right) \\
      &= \left( \left( \sqrt{3} - \frac{3}{2} \right) \eone + \left(\frac{3\sqrt{3}}{2} + 1\right)\etwo + 2 \sqrt{3} \ethree - 2 \mathbf{e}_{123} \right) \left(\fraca + \fracb \eot\right) \\
      &= \left(1 - \frac{3 \sqrt{3}}{2}\right) \eone + \left(\frac{3}{2} + \sqrt{3}\right) \etwo + 4 \ethree \\
      &= \begin{pmatrix} 1 - \frac{3 \sqrt{3}}{2} \\ \frac{3}{2} + \sqrt{3} \\ 4 \end{pmatrix},
    \end{aligned}
  \end{equation}
  was der Rotation um die $\eot$-Achse entspricht:
  \begin{equation}
    \begin{pmatrix}
      \cos{\alpha} & -\sin{\alpha} & 0 \\
      \sin{\alpha} & \cos{\alpha} & 0 \\
      0 & 0 & 1
    \end{pmatrix} \begin{pmatrix} 2 \\ 3 \\ 4 \end{pmatrix}.
  \end{equation}
\end{beispiel}
