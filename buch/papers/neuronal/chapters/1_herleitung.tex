%
% 1_herleitung.tex -- Herleitung der Methode
%
% (c) 2025 Roman Cvijanovic & Nicola Dall'Acqua, Hochschule Rapperswil
%
% !TEX root = ../../buch.tex
% !TEX encoding = UTF-8
%

\section{Herleitung der Methode\label{neuronal:section:herleitung}}
\kopfrechts{Herleitung der Methode}

Im Folgenden wird die Methode zum Lösen von Feldgleichungen mittels eines neuronalen Netzes theoretisch hergeleitet.
Dies wird anhand des Beispiels der Wellengleichung gemacht
\begin{equation}
    \frac{\partial^2 u}{\partial t^2} = c^2 \left( \frac{\partial^2 u}{\partial x^2} + \frac{\partial^2 u}{\partial y^2} \right)
    \label{neuronal:wellengleichung}
\end{equation}

Wobei \( u(x, y, t) \) die z-Koordinate am Punkt \( (x, y) \) zum Zeitpunkt \( t \) darstellt.
Anders ausgedrückt ist \( u \) eine gewellte Oberfläche. Zudem ist \( c \in \mathbb{R} \) eine Konstante und stellt die Verbreitungsgeschwindigkeit der Welle dar.

Weiter sei das neuronale Netzwerk gegeben als
\begin{equation}
    \hat{u}(x, y, t; \theta)
    \label{neuronal:nn}
\end{equation}
Das Netz hängt von den gleichen Variablen ab wie \( u \).
Zusätzlich besitzt es einen Vektor \( \theta \in \mathbb{R}^n \) der n \emph{trainierbaren Parameter}.
Das Ziel des Trainings eines neuronalen Netzes ist es, diese Parameter so zu wählen, dass das Netz die gesuchte Funktion (hier \( u \)) möglichst gut approximiert.


\subsection{Formulierung als Optimierungsproblem}\label{neuronal:subsection:optimierungsproblem}
Durch Subtrahieren der rechten Seite von der Wellengleichung \eqref{neuronal:wellengleichung} erhält man
\begin{equation}
    \frac{\partial^2 u}{\partial t^2} - c^2 \left( \frac{\partial^2 u}{\partial x^2} + \frac{\partial^2 u}{\partial y^2} \right) = 0
\end{equation}
Substituiert man nun das neuronale Netz \eqref{neuronal:nn} für \( u \) und quadriert anschliessend, lässt sich das Training des Netzes als Optimierungsproblem formulieren:\newline

Wähle \( \theta \) so, dass
\begin{equation}
    \left(\frac{\partial^2 \hat{u}}{\partial t^2} - c^2 \left( \frac{\partial^2 \hat{u}}{\partial x^2} + \frac{\partial^2 \hat{u}}{\partial y^2} \right)\right)^2
    \label{neuronal:optimierung}
\end{equation}
minimal wird.\newline

Durch das Quadrieren wird erreicht, dass der Term für alle \( x, y, t \) immer \( \geq 0 \) ist.
Somit sind Minima des Terms in der Nähe von 0 und nicht beliebig klein.
Sobald eine Lösung \( \theta \) zu diesem Problem gefunden wurde, ist
\begin{equation}
    \left(\frac{\partial^2 \hat{u}}{\partial t^2} - c^2 \left( \frac{\partial^2 \hat{u}}{\partial x^2} + \frac{\partial^2 \hat{u}}{\partial y^2} \right)\right)^2 \approx 0
    \iff
    \frac{\partial^2 \hat{u}}{\partial t^2} - c^2 \left( \frac{\partial^2 \hat{u}}{\partial x^2} + \frac{\partial^2 \hat{u}}{\partial y^2} \right) \approx 0
    \iff
    \frac{\partial^2 \hat{u}}{\partial t^2} \approx c^2 \left( \frac{\partial^2 \hat{u}}{\partial x^2} + \frac{\partial^2 \hat{u}}{\partial y^2} \right)
\end{equation}
und die Wellengleichung wird somit approximativ durch das neuronale Netz gelöst.


\subsection{Training des neuronalen Netzes}\label{neuronal:subsection:training_nn}

