%
% teil1.tex -- Beispiel-File für das Paper
%
% (c) 2020 Prof Dr Andreas Müller, Hochschule Rapperswil
%
% !TEX root = ../../buch.tex
% !TEX encoding = UTF-8
%
\section{Komplexe Zahlen
\label{geoalgebra:section:komplexe-zahlen}}
\kopfrechts{Problemstellung}
Um die geometrische Algebra zu verstehen, hilft es, wenn wir nochmals kurz die komplexen Zahlen anschauen, da sie ähnliche Axiome verwenden.
\index{komplexe Zahl}%
Die imaginäre Einheit $i$ ist definiert durch
\begin{equation*}
  i^2 = -1.
\end{equation*}
Wir definieren zwei komplexe Zahlen
\begin{align*}
  z_1 = a_1 + i a_2 \\
  z_2 = b_1 + i b_2 
\end{align*}
mit $a_i, b_i \in \mathbb{R}$
und bilden das Produkt $z_1 \cdot{} z_2$. Es ergibt sich
\begin{align*}
  z_1 z_2 &= (a_1 + i a_2) (b_1 + i b_2) \\
  &= a_1 b_1 + i a_1 b_2 + i a_2 b_1 + i^2 a_2 b_2 \\
  &= (a_1 b_1 - a_2 b_2) + i (a_1 b_2 + a_2 b_1).
\end{align*}
Wie wir hier sehen, bleiben die gemeinsamen Terme $(a_1 b_1 - a_2 b_2)$ als reelle Zahlen übrig, während die \emph{gemischten} Terme $(a_1 b_2 + a_2 b_1)$ weiterhin mit $i$
multipliziert werden, d.~h. dieser Teil ist imaginär. Wir werden etwas Ähnliches bei der geometrischen Algebra beobachten, wo die Mischterme übrig bleiben.




