%
% einleitung.tex -- Beispiel-File für die Einleitung
%
% (c) 2020 Prof Dr Andreas Müller, Hochschule Rapperswil
%
% !TEX root = ../../buch.tex
% !TEX encoding = UTF-8
%
\section{Die Differentialgleichung\label{mongeampere:section:teil0}}
\kopfrechts{Teil 0}
Die Monge-Ampèresche Gleichung ist
\begin{equation}
  \det D^2 u = f(x, u, \nabla u),
  \label{mongeampere:eq:mongeampere}
\end{equation}
wobei $D$ die Hessche Matrix
\begin{equation}
  D^2 u =
  \begin{pmatrix}
    \pdv[2]{u}{x_1} & \dots &\pdv{u}{x_1}{x_n} \\
  \vdots & \ddots & \vdots \\
  \pdv{u}{x_n}{x_1} & \dots &\pdv[2]{u}{x_n} \\
  \end{pmatrix}
  \label{mongeampere:eq:hess}
\end{equation}
ist.
Wenn man die Gleichung ansieht, erkennt man, dass es sich bei ihr um eine partielle
nicht-lineare Gleichung handelt.
Das Lösen einer solchen Differentialgelichung ist im generellen ein anspruchvolles Problem.
Anhand von Bedinugnen für $u$ und $f$ kann man \eqref{mongeampere:eq:mongeampere} 
in einen bestimmten Typ Differnetialgleichung bringen, was unterschiedliche Lösungsansätze ermöglicht.
Setzt man voraus, dass $u$ eine glatte konvexe Function ist, und $f$ positiv, dann ist \eqref{mongeampere:eq:mongeampere}
eine elliptische partielle Differnetialgleichung.
Gemäss \cite{mongeampere:figalli2018mongeampereequation} gibt eine solche Form Hoffnung auf die existenz einer regulären Lösung.
Ein anderes beispiel ist in \cite{mongeampere:figalli2022prescribednegativegausscurvature} zu finden, wo die 
Gleichung linearisiert wird um lokale Lösungen zu finden.

\subsection{Minkowski Problem}
Eine Anwendung, welche vom Deutsche Mathematiker Hermann Minkwoski entdekt wurde, ist das Lösen des
Prescribed Gaussian Curvature Problem mit der Monge-Ampere Gleichung.
Diese Anwendung befasst damit eine konvexe kompakte Fläche zu finden, dessen Gaussche Krümmung 
an jedem Punkt gegeben ist.

