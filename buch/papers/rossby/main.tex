%
% main.tex -- Paper zum Thema <rossby>
%
% (c) 2020 Autor, OST Ostschweizer Fachhochschule
%
% !TEX root = ../../buch.tex
% !TEX encoding = UTF-8
%
\chapter{Thema\label{chapter:rossby}}
\kopflinks{Thema}
\begin{refsection}
\chapterauthor{Michael Schmid}

Ein paar Hinweise für die korrekte Formatierung des Textes
\begin{itemize}
\item
Absätze werden gebildet, indem man eine Leerzeile einfügt.
Die Verwendung von \verb+\\+ ist nur in Tabellen und Arrays gestattet.
\item
Die explizite Platzierung von Bildern ist nicht erlaubt, entsprechende
Optionen werden gelöscht. 
Verwenden Sie Labels und Verweise, um auf Bilder hinzuweisen.
\item
Beginnen Sie jeden Satz auf einer neuen Zeile. 
Damit ermöglichen Sie dem Versionsverwaltungssysteme, Änderungen
in verschiedenen Sätzen von verschiedenen Autoren ohne Konflikt 
anzuwenden.
\item 
Bilden Sie auch für Formeln kurze Zeilen, einerseits der besseren
Übersicht wegen, aber auch um GIT die Arbeit zu erleichtern.
\end{itemize}


\printbibliography[heading=subbibliography]
\end{refsection}
