%
% 4-kodifferential.tex -- Kodifferential
%
% (c) 2025 Prof Dr Andreas Müller
%
\section{Kodifferential
\label{buch:hodge:section:kodifferential}}
\kopfrechts{Kodifferential}
Der $d$-Operator erhöht den Grad einer Differentialform um $1$.
Aus der Zusammensetzung mit dem Hodge-Operator entsteht ein neuer
Operator, das Kodifferential $\delta$, der den Grad erniedrigt.
Zusammen mit $d$ entstehen damit weitere Möglichkeiten,
koordinatenunabhängige Operatoren zu definieren.

%
% Definition
%
\subsection{Definition}
Bei der Berechnung der Rotation einer Rotation trat die Kombination
${*}d{*}d$ von Differentialoperatoren auf $p$-Formen und dem
Hodge-Operator auf.
Die äussere Ableitung erhöht den Grad, der Teil ${*}d{*}$
verwandelt eine $p$-Form in eine Differentialform vom Grad
\[
n-(\underbrace{(\underbrace{n-p}_{*})+1}_{d})
=
p-1,
\]
der Grad wird also erniedrigt.
Der Operator ${*}d{*}$ ist also ein Differentialoperator, der den
Grad in umgekehrter Richtung im Vergleich zur äusseren Ableitung
verändert.
Daher verdient er eine eigene Definition.

\begin{definition}[Kodifferential]
\label{buch:hodge:kodifferential:def:delta}
Das {\em Kodifferential} ist der lineare Operator
\index{Kodifferential}%
\index{delta@$\delta$}%
\[
\delta
=
(-1)^{p(n-p)+p}
{\ast}d{\ast}
\colon
\Omega^p(M)\to\Omega^{p-1}
:
\omega \mapsto (-1)^{p(n-p)+p}{\ast}d{\ast}\omega
\]
vom Grad $-1$.
\end{definition}

Der Vorzeichenfaktor setzt sich aus zwei Beiträgen zusammen.
Der Faktor $(-1)^{p(n-p)}$ zusammen mit dem Hodge-Operator ${\ast}$
ist der inverse Hodge-Operator: $\ast^{-1} = (-1)^{p(n-p)}{\ast}$.
Der verbleibende Faktor $(-1)^p$ kehrt das Vorzeichen für ungerade
$p$-Formen und ist die zweckmässigere Wahl, wie im Abschnitt über
den Laplace-Operator weiter unten klar werden wird.
Das Kodifferential kann daher auch als
\[
\delta
=
(-1)^p
{\ast}^{-1}d{\ast}
\]
geschrieben werden.

%
% Vektoranalsis und Kodifferential
%
\subsection{Vektoranalysis und Kodifferential}
Kombinationen des Hodge-Operators mit dem Differential reproduzieren
gemäss der Tabelle~\ref{buch:hodge:hodge:table:operatoren} die 
klassischen Operatoren der Vektoranalysis.
Wir erwarten daher, dass diese Operatoren sich noch etwas leichter
durch das Kodifferential ausdrücken lassen.

Die Divergenz eines Vektorfeldes in $\mathbb{R}^3$ ist auf der
zugehörigen 1-Form gegeben durch den Operator ${\ast}d{\ast}$,
der sich von $\delta$ nur um den Faktor $(-1)^{1(3-1)+1}=-1$
unterscheidet.

Am Ende von Abschnitt~\ref{buch:hodge:skalarprodukt:subsection:divergenz}
wurde darauf hingewiesen, dass die Divergenz auch für Vektorfelder
auf $\mathbb{R}^n$ durch ${\ast}d{\ast}$ berechnet wird.
In diesem Fall ist der Unterschied ein Vorzeichenfaktor der Form
$(-1)^{1\cdot(n-1)+1}=(-1)^n$.

%
% Poincaré-Lemma für das Kodifferential
%
\subsection{Poincaré-Lemma für das Kodifferential}
Auch für das Kodifferential gilt ein Poincaré-Lemma.

\begin{satz}[Poincaré-Lemma für das Kodifferential]
Wenn $\omega \in \Omega^p(\mathbb{R}^n)$ eine $p$-Form ist mit
$\delta\omega=0$, dann gibt es eine $p+1$-Form
$\alpha\in\Omega^{p+1}(\mathbb{R}^n)$ mit $\delta\alpha=\omega$.
\end{satz}

\begin{proof}
Nach Voraussetzung ist $\delta \omega = {\ast}d{\ast}\omega = 0$.
Da der Hodge-Operator invertierbar ist, muss auch $d{\ast}\omega=0$
sein.
Die $(n-p)$-Form $\ast\omega$ ist geschlossen, nach dem Poincaré-Lemma
gibt es eine $(n-p-1)$-Form $\beta$ mit $d\beta=\ast\omega$.
Durch erneute Anwendung des Hodge-Operators wird daraus
\[
{\ast}d{\ast} {\ast}\beta = (-1)^{p(n-p)}\omega.
\]
Setzen wir $\alpha=(-1)^{p(n-p)}\beta$, dann folgt
\[
\delta \alpha = \omega,
\]
womit ist das Poincaré-Lemma für das Kodifferential bewiesen.
\end{proof}

