%
% p-Formen
%
\section{$p$-Formen
\label{buch:pformen:section:pformen}}
\kopfrechts{$p$-Formen}
Im ersten Schritt müssen die Begriffe der 2-Vektoren und der 
Differentialformen auf Vektoren und Formen beliebigen Grades
verallgemeinert werden.

%
% p-Vektoren
%
\subsection{$p$-Vektoren}
In früheren Kapiteln würden 2-Vektoren, $(n-1)$-Vektoren und $n$-Vektoren
eingeführt, sobald sie für die Erklärung der Sätze von Green,
Stokes und Gauss nötig waren.
Die Konstruktion lässt sich aber auf den Begriff eines $p$-Vektors
für beliebiges $p$ auf einer $n$-dimensionalen Mannigfaltigkeit
verallgemeinern.

\begin{definition}
Ein $p$-Vektor ist ein vollständig antisymmetrischer Tensor
$p$-ter Stufe vom Typ $(0,p)$.
\end{definition}

Aus einer Basis $X_1,\dots,X_n$ von $TM$ entsteht eine Basis des
Vektorraums der $p$-Vektoren besteht daher mit den
total antisymmetrischen Tensoren
\[
X_{i_1}\wedge \dots\wedge X_{i_p}
=
\sum_{k_1,\dots,k_p=1}^{n}
\varepsilon_{k_1\dots k_p}
X_{i_{k_1}}\otimes\dots\otimes X_{i_{k_p}}
\]
Darin ist $\varepsilon_{k_1\dots k_p}$ ist das Levi-Cività-Symbol
$p$-ter Stufe.
Der Vektorraum der $p$-Vektoren wird auch als 
$\bigwedge^p TM$ geschrieben.

%
% Antisymmetrische Multilinearformen
%
\subsection{Antisymmetrische Multilinearformen auf $TM$}
Dual zu den $p$-Vektoren sind die Linearformen auf einer $n$-dimensionalen
Mannigfaltigkeit sind die $p$-Formen.

\begin{definition}
Eine $p$-Form auf der $n$-dimensionalen Mannigfaltigkeit ist eine vollständig
antisymmetrische Multilinearform vom Grad $p$.
Die $p$-Formen $dx^{i_1}\wedge\dots\wedge dx^{i_p}$ sind dual zu den 
$p$-Vektoren $X_{i_1}\wedge\dots\wedge X_{i_p}$:
\[
\langle
dx^{i_1}\wedge\dots\wedge dx^{i_p},
X_{k_1}\wedge\dots\wedge X_{k_p}
\rangle
=
\delta^{i_1}_{k_1}
\dots
\delta^{i_p}_{k_p}.
\]
Die Menge der $p$-Formen auf einer Mannigfaltigkeit wird mit $\Omega^p(M)$
bezeichnet.
\index{OmegapM@$\Omega^p(M)$}%
\end{definition}

Die Menge $\Omega^0(M)$ der $0$-Formen besteht nur aus beliebig oft stetig
differenzierbaren Funktionen auf der Mannigfaltigkeit, die mit
$C^\infty(M)$ bezeichnet wird.
\index{Cinf@$C^\infty(M)$}%
Die $p$-Formen im Vektorraum der $p$-Formen können mit Funktionen aus
$C^\infty(M)$ multipliziert werden, eine Eigenschaft, die man einen
$C^\infty(M)$-Modul nennt.

\begin{satz}
Die $p$-Formen auf einer Mannigfaltigkeit bilden einen $C^\infty(M)$-Modul.
In einer Karte kann eine $p$-Form $\omega\in \Omega^p(M)$ als
\begin{align*}
\omega
&=
\sum_{i_1<i_2<\dots< i_p}
f_{i_1 i_2\dots i_p}(x)
\, dx^{i_1}\wedge dx^{i_2}\wedge\dots\wedge dx^{i_p}
\\
&=
\frac{1}{p!}
\sum_{i_1,i_2,\dots,i_p}
f_{i_1 2_2\dots i_p}(x)
\, dx^{i_1}\wedge dx^{i_2}\wedge\dots\wedge dx^{i_p}
\end{align*}
geschrieben werden, wobei $f_{i_1 i_2\dots i_p}$  antisymmetrisch ist in jedem
Indexpaar.
\end{satz}

%
% Das \wedge-Produkt von $p$-Formen
%
\subsection{Das $\wedge$-Produkt}
Das Wedge-Produkt von Differentialformen ist eine bilineare Abbildung
\[
\Omega^p(M)\times \Omega^q(M) \to \Omega^{p+q}(M)
:
(\alpha,\beta) \mapsto \alpha\wedge\beta.
\]
Dies bedeutet, dass es ausreicht, das Wedge-Produkt auf einer Basis
zu definieren.

Seien jetzt also
\begin{align*}
\alpha &= dx^{i_1}\wedge\dots\wedge dx^{i_p} \\
\beta  &= dx^{j_{p+1}}\wedge\dots\wedge dx^{j_{p+q}}
\end{align*}
zwei Basis-$p$- bzw.~-$q$-Formen.
Es muss das Produkt
\begin{equation}
dx^{i_1}\wedge\dots\wedge dx^{i_p}
\wedge
dx^{i_{p+1}}\wedge\dots\wedge dx^{i_{p+q}}
\label{buch:pformen:pformen:eqn:basisprodukt}
\end{equation}
durch Basis-$(p+q)$-Formen ausgedrückt werden.
Falls $(i_1,\dots,i_p)$ und $(i_{p+1},\dots,i_{p+q})$ einen Index gemeinsam
haben, dann verschwindet das Produkt.
Das Wedge-Produkt~\eqref{buch:pformen:pformen:eqn:basisprodukt}
ist also nur dann von $0$ verschieden, wenn alle Indizes in
$(i_1,\dots,i_p,i_{p+1},\dots,i_{p+q})$ verschieden sind.
Das Produkt~\eqref{buch:pformen:pformen:eqn:basisprodukt} ist aber
normalerweise keine Basis-$(p+q)$-Form, da in den Basisformen die
Indizes aufsteigend sind.
Die 1-Formen in \eqref{buch:pformen:pformen:eqn:basisprodukt} müssen
also so vertauscht werden, dass sich eine aufsteigende Reihenfolge
$(k_1,\dots,k_{p+q})$
der Indizes ergibt.
Bei jeder Vertauschung ändert das Vorzeichen.
Das Produkt \eqref{buch:pformen:pformen:eqn:basisprodukt} unterscheidet
sich also nur durch das Vorzeichen von 
\(
dx^{k_1}\wedge\dots\wedge dx^{k_{p+q}}
\).

Zur Bestimmung des Vorzeichens des Produktes $\alpha\wedge\beta$ 
sei $\sigma\in S_{p+q}$ die Permutation\footnote{Die Theorie der
Permutationen, der Gruppen $S_n$ und des Vorzeichens einer Permutation
wird ausführlich erklärt in \cite[Abschnitt~4.3.4]{buch:linalg}.}
der Zahlen $1,\dots,p+q$,
die $(i_1,\dots,i_p,i_{p+1},\dots,i_{p+q})$ in die geordnete
Reihenfolge
\[
(i_{\sigma(1)},\dots,i_{\sigma(p)},i_{\sigma(p+1)},\dots,i_{\sigma(p+q)})
=
(k_1,\dots,k_{p+q})
\]
überführt.
Die Anzahl $t(\sigma)$ der Vertauschungen, mit denen sich die
Permutation $\sigma$ schreiben lässt, ist nicht eindeutig bestimmt,
aber sie ist entweder gerade oder ungerade.
Das gesuchte Vorzeichen ist daher das sogenannte Signum
\[
\operatorname{sgn}(\sigma) = (-1)^{t(\sigma)}
\]
oder Vorzeichen der Permutation.

%
% Die Summenformel
%
\subsubsection{Die Summenformel}
Bei der konkreten Berechnung eines Wedge-Produktes ist es jeweils
einfach, die nötigen Vertauschungen vorzunehmen und damit das
richtige Vorzeichen zu finden.
Für die algebraische Rechnung ist dies jedoch etwas komplizierter,
da die Algebra keine gute Notation für die richtige Reihenfolge der
Indizes anbietet.

Um eine einfachere Formel zu finden, schreiben wir erst die Basis
auf eine neue Art.
Statt als Basisvektoren die Produkte von 1-Formen in einer vorgegebenen
Reihenfolge zu nehmen, lassen wir alle Reihenfolgen der Indizes zu.
Die so entstehenden Produkte unterscheiden sich nur durch ein
Vorzeichen.
Jedes der Produkte
\[
dx^{i_1}\wedge\dots\wedge dx^{i_p}
=
\operatorname{sgn}(\sigma)
\,
dx^{i_{\sigma(1)}} \wedge\dots\wedge dx^{i_{\sigma(p)}}
\]
für eine Permutation $\sigma\in S_p$ ist also gleichermassen geeignet
als Basisvektor.
Es gibt $p!$ solche Permutationen.
Damit wir die verschiedenen Basisvektoren gleichberechtigt verwenden
können, nehmen wir deren Mittelwert
\[
dx^{i_1}\wedge\dots\wedge dx^{i_p}
=
\frac{1}{p!}
\sum_{\sigma\in S_p} dx^{i_{\sigma(1)}} \wedge\dots\wedge dx^{i_{\sigma(p)}}
\]
Diese Summe ändert nicht, wenn die Indizes $i_1,\dots,i_p$ mit einer
geraden Permutation permutiert werden.

Eine beliebige $p$-Form ist in der Basis der
$dx^{i_1}\wedge\dots\wedge dx^{i_p}$  mit $i_1<\dots<i_p$
durch die Koeffizienten $f_{i_1\dots i_p}$ gegeben.
Die Forderung der aufsteigend sortierten Indizes in
\[
\omega
=
\sum_{i_1<\dots<i_p}
f_{i_1\dots i_p}\,
dx^{i_1}\wedge\dots\wedge dx^{i_p}
\]
ist etwas schwerfällig.
Die Beschreibung wird einfacher, wenn man verlangt, dass die
Koeffizienten vollständig antisymmetrisch sind, also
\[
f_{i_1\dots i_p}
=
\operatorname{sgn}\,
f_{i_{\sigma(1)}\dots i_{\sigma(p)}}
\]
für jede Permutation $\sigma\in S_p$.
Damit fällt jetzt die Foderung nach aufsteigenden Indizes weg und man
kann
\begin{equation}
\omega
=
\frac{1}{p!}
\sum_{i_1,\dots,i_p=1}^n
f_{i_1\dots i_p}
\,
dx^{i_1}\wedge\dots\wedge dx^{i_p}
\label{buch:pformen:pformen:eqn:summenformel}
\end{equation}
schreiben.

%
% Kommutativität
%
\subsubsection{Kommutativität}
Das Wedge-Produkt ist nicht kommutativ.
Für 1-Formen gilt $dx^i\wedge dx^k=-dx^k\wedge dx^i$.
Es ist aber auch nicht antikommutativ, denn 
\[
(dx^2\wedge dx^3)\wedge dx^1
=
-dx^2\wedge dx^1\wedge dx^3
=
dx^1\wedge dx^2\wedge dx^3,
\]
in diesem Fall ändert das Vorzeichen bei der Vertauschung der
Faktoren $dx^1$ und $dx^2\wedge dx^3$ nicht.
Der Grund ist, dass {\em zwei} Vertauschungen nötig sind, um
die Faktoren in die Standardreihenfolge zu bringen.
Wir beweisen die folgende Kommutationsformel.

\begin{satz}
Für eine $p$-Form $\alpha\in\Omega^p(M)$ und eine $q$-Form
$\beta\in\Omega^q(M)$ gilt
\begin{equation}
\alpha\wedge\beta
=
(-1)^{pq}\,\beta\wedge\alpha.
\label{buch:pformen:pformen:eqn:kommutativ}
\end{equation}
\end{satz}

\begin{proof}
Wir müssen zeigen, dass die
Formel~\eqref{buch:pformen:pformen:eqn:kommutativ}
Für die Basisdifferentialformen gilt.
Wir verwenden die Farben {\color{darkred}rot} und {\color{blue}blau},
um die Vertauschungen der 1-Form-Faktoren deutlich zu machen.
Wir schreiben also
\begin{align*}
{\color{darkred}\alpha}
&=
{\color{darkred}dx^{i_1}}
\wedge \dots \wedge
{\color{darkred}dx^{i_p}}
\\
\text{und}\qquad
{\color{blue}\beta}
&=
{\color{blue}dx^{k_1}}
\wedge \dots \wedge
{\color{blue}dx^{k_p}}
\end{align*}
und berechnen die Produkte $\alpha\wedge\beta$ und $\beta\wedge\alpha$.
Um das Produkt $\beta\wedge\alpha$ in die Reihenfolge der 1-Form-Faktoren
von $\alpha\wedge\beta$ zu bringen, sind jeweils Vertauschungen mit
Nachbarfaktoren notwendig:
\begin{align*}
{\color{darkred}\alpha}\wedge{\color{blue}\beta}
&=
{\color{darkred}dx^{i_1}}
\wedge\dots\wedge
{\color{darkred}dx^{i_p}}
\wedge
{\color{blue}dx^{k_1}}
\wedge\dots\wedge
{\color{blue}dx^{k_q}}
\\
&=
{\color{darkred}dx^{i_1}}
\wedge\dots\wedge
{\color{blue}dx^{k_1}}
\wedge
{\color{darkred}dx^{i_p}}
\wedge
{\color{blue}dx^{k_2}}
\wedge\dots\wedge
{\color{blue}dx^{k_q}}
\\
&=
(-1)^p
\,
{\color{blue}dx^{k_1}}
\wedge
{\color{darkred}dx^{i_1}}
\wedge\dots\wedge
{\color{darkred}dx^{i_p}}
\wedge
{\color{blue}dx^{k_2}}
\wedge\dots\wedge
{\color{blue} dx^{k_q}}
\\
&=
(-1)^2q
\,
{\color{blue}dx^{k_1}}
\wedge
{\color{blue}dx^{k_2}}
\wedge
{\color{darkred}dx^{i_1}}
\wedge\dots\wedge
{\color{darkred}dx^{i_p}}
\wedge
{\color{blue}dx^{k_3}}
\wedge\dots\wedge
{\color{blue} dx^{k_q}}
\\
&=
(-1)^{pq}
\,
{\color{blue}dx^{k_1}}
\wedge\dots\wedge
{\color{blue}dx^{k_q}}
\wedge
{\color{darkred}dx^{i_1}}
\wedge\dots\wedge
{\color{darkred}dx^{i_p}}
\\
&=
(-1)^{pq}
\,
{\color{blue}\beta}\wedge{\color{darkred}\alpha}.
\end{align*}
Damit ist die Formel bewiesen.
\end{proof}

%
% Die graduierte Algebra der p-Formen
%
\subsection{Die graduierte Algebra der $p$-Formen}
Die Räume $\Omega^p(M)$ haben für verschiedene $0$ keine gemeinsamen 
nichttrivialen Formen.
Die Nullform, die auf jedem beliebigen $p$-Vektor verschwindet, kann man
aber durchaus als ein gemeinsames Element betrachten.
Rein formal kann man daher auch Linearkombinationen von $p$-Formen für
verschiedenes $p$ bilden.
So entsteht die Menge
\[
\Omega^*(M)
=
\bigoplus_{p=0}^n \Omega^p (M)
\]
der Differentialformen auf der Mannigfaltigkeit.
$\Omega^*(M)$ ist ein reeller Vektorraum: Differentialformen können mit
Zahlen multipliziert und addiert werden.

Die $0$-Formen sind die Funktionen in $C^\infty(M)$ und können mit
glatten Funktionen in $\Omega^0(M)=C^\infty(M)$ multipliziert werden.
Die glatten Funktionen bilden keinen Körper, da man durch Funktionen mit
Nullstellen nicht dividieren kann.
Die glatten Funktionen bilden aber einen sogenannten Ring. 
\index{Ring}%
Es sind alle Rechenregeln definiert, die auch in einem Körper gelten,
ausser die Existenz eines multiplikativen Inversen.

Die Differentialformen können mit Funktionen in $C^\infty(M)$
multipliziert werden.
Dadurch wird $\Omega^*(M)$ zu einem sogenannten {\em Modul} über
dem Ring $C^\infty(M)$.
Wenn $f\in C^\infty(M)$ eine glatte Funktion und $\omega\in\Omega^p(M)$
eine $p$-Form ist, dann ist $f\omega\in\Omega^p(M)$ ebenfalls eine
$p$-Form.
Die Multiplikation mit einer glatten Funktion ändert den Grad einer
$p$-Form nicht.
Jeder einzelne Summand $\Omega^p(M)\subset\Omega^*(M)$ ist also für
sich genommen ebenfalls ein $C^\infty(M)$-Modul.
Man nennt $\Omega^*(M)$ daher einen {\em graduierten} $C^\infty(M)$-Modul.
\index{graduiert}%
\index{Modul, graduiert}%

Das Wedge-Produkt der Differentialformen verknüpft eine $p$-Form
$\alpha\in\Omega^p(M)$ mit einer $q$-Form $\beta\in\Omega^q(M)$
zu einer $p+q$-Form $\alpha\wedge\beta\in\Omega^{p+q}(M)$.
Die Menge $\Omega^*(M)$ hat also nicht nur die Struktur eines
Moduls über dem Ring $C^\infty(M)$ sondern auch die Struktur
einer Algebra über dem Ring $C^\infty(M)$.
Das Produkt respektiert aber auch die Graduierung von $\Omega^*(M)$,
daher heisst $\Omega^*(M)$ auch eine graduierte $C^\infty(M)$-Algebra.

