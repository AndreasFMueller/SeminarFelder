%
% uebersicht.tex -- Uebersicht ueber die Seminar-Arbeiten
%
% (c) 2022 Prof Dr Andreas Mueller, OST Ostschweizer Fachhochschule
%
\chapter*{Übersicht}
\fancyhead[RE]{}
\fancyhead[LO]{Übersicht}
\label{buch:uebersicht}
Im zweiten Teil kommen die Teilnehmer des Seminars selbst zu Wort.
Die im ersten Teil dargelegten mathematischen Methoden und
grundlegenden Modelle werden dabei verfeinert, verallgemeinert
und auch numerisch überprüft.

Die Kulmination der Erforschung der elektrischen und magnetischen
Phänomene war die Erkenntnis, dass sie nicht einfach Fernwirkungen
zwischen entfernen Ladungen oder Strömen sind, wie man sich zu
dieser Zeit die Wirkung der Gravitation vorstellte.
Vielmehr werden elektrische Kräfte und Magnetismus durch ein
physikalische Entitäten vermittelt, die den ganzen Raum zwischen den
Ladungen und Strömen ausfüllt, das elektrische und das magnetische
Feld.
Aus den anschaulichen Vorstellungen von Michael Faraday wurden in den
\index{Faraday, Michael}%
Händen von James Clerk Maxwell eine moderne mathematische Feldtheorie,
\index{Maxwell, James Clerk}%
mit der sich alle Phänomene auf einheitliche Art erklären liessen.
Die Masxwell-Gleichungen zeigen aber auch, dass elektrisches und
magnetisches Feld untrennbar miteinander verbunden sind.
Die vierdimensionaonal Darstellung, die \emph{Pascal Widmer}
\index{Widmer, Pascal}%
\index{Pascal Widmer}%
und \emph{Mike Peng}
\index{Mike Peng}%
\index{Peng, Mike}%
im Kapitel~\ref{chapter:maxwell} im Formulismus der $k$-Formen auf
der vierdimensionalen Raumzeit entwickeln, zeigt auf spektakuläre
Weise, dass die komplizierten Maxwell-Gleichungen eigentlich die
natürlichsten Feldgleichungen für ein kovariantes Tensorfeld zweiter
Stufe sind.

Die meisten Feldgleichungen der Physik sind von zweiter Ordnung, da
sie auf das newtonsche Gesetz $F=ma$ zurückgehen, die zweiten
Ableitungen stecken in der Beschleunigung $a$.
Die Elastomechanik dagegen hat Feldgleichungen vierter Ordnung,
\index{Elastomechanik}%
was sowohl deren Formulierung wie auch die Lösung anspruchsvoller
macht.
\emph{Sofia Aaltonen}
\index{Sofia Aaltonen}%
\index{Aaltonen, Sofia}%
und
\emph{Rafael Barbosa}
\index{Rafael Barbosa}%
\index{Barbosa, Rafael}%
zeigen in Kapitel~\ref{chapter:elastomechanik}, wie man auf diese
Gleichungen kommen kann.

Der Formalismus der $k$-Vektoren, insbesondere auch die Vereinheitlichung
der skalaren und vektoriellen Produkte in eine einzige Algebra der
$k$-Vektoren, ermöglicht nicht nur vereinfachte Feldbeschreibung,
er erlaubt auch geometrische Anwendungen.
\emph{Damien Flury} zeigt in Kapitel~\ref{chapter:geoalgebra}, wie
\index{Damien Flury}%
\index{Flury, Damien}%
sich geometrische Transformationen in drei Dimensionen mit dieser 
\index{geometrische Algebra}%
geometrischen Algebra prägnant beschreiben lassen.

Der Fourier-Theorie kommt bei der Lösung von Feldgleichungen eine
besondere Bedeutung zu.
\index{Fourier-Theorie}%
Zunächst ermöglichst sie die Behandlung partieller Differentialgleichungen
mit den spektralen Methoden.
\index{spektrale Methoden}%
\emph{Martina Knobel}
\index{Martina Knobel}%
\index{Knobel, Martina}%
und \emph{Gian Kraus}
\index{Gian Kraus}%
\index{Kraus, Gian}%
zeigen in Kapitel~\ref{chapter:fourier} aber auch, wie die moderne
Physik den Fourier-Koeffizienten eine neue Bedeutung gibt.
Durch Fourier-Transformation wird das elektromagnetische Feld
zerlegt in eine Menge von harmonischen Oszillatoren, einem für jede
Wellenzahl.
\index{Wellenzahl}%
\index{harmonischer Oszillator}%
Die Quantenfeldtheorie findet die quantisierten Lösungen dieser
\index{Quantenfeldtheorie}%
harmonischen Oszillatoren und identifiziert die diskreten Zustände
dieser Oszillatoren mit den Photonen, die das Feld ausmachen.
\index{Photon}%

Die Strömungsgleichungen nach Euler oder Navier-Stokes
sind oft zu kompliziert für eine direkte Lösung.
Es sind daher vereinfachende Annahmen nötig, um die numerische
Lösung zu ermöglichen
Lewis Fry Richardson hat versucht, Wettervorhersagen durch Lösung
\index{Richardson, Lewis Fry}%
\index{Wettervorhersage}%
der Strömungsgleichungen der Atmosphäre zu finden.
Der Artikel~\ref{chapter:geostrophisch} über die geostrophische
Näherung von 
\index{geostrophise Näherung}%
\emph{Philip Brun}
\index{Philip Brun}%
\index{Brun, Philip}%
und
\emph{Loris Trüb}
\index{Loris Trüb}%
\index{Trüb, Loris}%
zeigt, welche Vereinfachungen dazu notwendig waren.

Vektorfelder sind oft sehr kompliziert und schwer verständlich.
Die Helmholtz-Zer\-le\-gung, die
\index{Helmholtz-Zerlegung}%
\emph{Joël Rechsteiner} 
\index{Joël Rechsteiner}%
\index{Rechsteiner, Joël}%
in Kapitel~\ref{chapter:helmholtz} vorstellt, ermöglicht ein solches
Feld in physikalisch leichter zu interpretierende Komponenten zu
zerlegen. 
Er demonstriert diese Art der Interpretation anhand des Schallfeldes.

\emph{Alain Keller}
\index{Alain Keller}%
\index{keller, Alain}%
zeigt in Kapitel~\ref{chapter:mongeampere}, wie die Monge-Ampère-Gleichung
\index{Monge-Ampère-Gleichung}%
das schwierige geometrische Problem lösen kann, eine Fläche mit vorgegebener
gaussscher Krümmung zu finden.
Die ursprüngliche Fragestellung, die Monge auf die Monge-Ampère-Gleichung
geführt hat, war jedoch ein Transportproblem.
\index{Transportproblem}%
\emph{Patrik Müller}
\index{Patrik Müller}%
\index{Müller, Patrik}%
zeigt in Kapitel~\ref{chapter:mongekant} die Geschichte dieses Problems
bis zur Lösung nach Kantorowitch nach.
\index{Kantorowitch}%

Feldgleichungen können auch mit neuronalen Netzwerken gelöst werden.
\index{neuronales Netz}%
\emph{Roman Cvijanovic}
\index{Roman Cvijanovic}%
\index{Cvijanovic, Roman}%
und
\emph{Nicola Dall'Acqua} 
\index{Nicola Dall'Acqua}%
\index{Dall'Acqua, Nicola}%
führen dies in Kapitel~\ref{chapter:neuronal} für verschieden Typen
von Feldgleichungen durch und zeigen exemplarisch, dass nicht jeder
Gleichungstyp gleichermassen zugänglich ist für eine Lösung mit dieser
Methode. 
Konkret zeigen sie, dass parabolische partielle Differentialgleichungen
gut gelöst werden können, dass hyperbolische partielle Differentialgleichungen
aber sehr viel schlechter für diesen Ansatz geeignet sind.

Nervenzellen transportieren Information durch Ausbreitung eines
\index{Nervenzelle}%
elektrischen Potentials entlang des Axons.
\index{Axon}%
Das Modell von Hodgkin und Huxley beschreibt diesen Vorgang.
\index{Hodgin-Huxley-Modell}%
\emph{Dino Ramcilovic}
\index{Dino Ramcilovic}%
\index{Ramcilovic, Dino}%
und
\emph{Tobias Zuber}
\index{Tobias Zuber}%
\index{Zuber, Tobias}%
zeigen in Kapitel~\ref{chapter:nerven}, wie sich dieses Modell
zum FitzHugh-Nagumo-Modell vereinfachen lässt, welches mit der
\index{FitzHugh-Nagumo-Modell}%
Methode der Nullklinen diskutiert und verstanden werden kann.
\index{Nullkline}%

Moderne Computer ermöglichen Strömungssimulationen mit dem Computer
\index{Strömungssimulation}%
durchzuführen.
\emph{Jero Barahona}
\index{Jero Barahona}
\index{Barahona, Jero}%
und
\emph{Yanick Diggelmann}
\index{Yanick Diggelmann}%
\index{Diggelmann, Yanick}%
untersuchen in Kapitel~\ref{chapter:openfoam} die Open-Source-Software
OpenFOAM und beschreiben, was alles bereitgestellt werden muss, damit
\index{OpenFOAM}%
eine erfolgreiche Simulation möglich wird.
Als Beispiel haben Sie die Windströmung um die Gebäude der OST in
\index{OST}%
\index{Windströmung}%
Rapperswil berechnet.
\index{Rapperswil}%
Eine Visualisierung ihrer Resultate schmückt auch den Umschlag dieses
Buches.

Die Quantenmechanik sagt, dass jedem Teilchen auch Welleneigenschaften
\index{Quantenmechanik}%
zukommen und umgekehrt.
Die Kopenhagen-Interpretation der Quantenmechanik interpretiert die
Welle als eine Wahrscheinlichkeitsdichte.
Doch
\emph{Flurin Brechbühler}
\index{Flurin Brechbühler}%
\index{Brechbühler, Flurin}%
und
\emph{Laurin Heitzer}
\index{Laurin Heitzer}%
\index{Heitzer, Laurin}
lassen sich von einem YouTube-Video insprieren und versuchen mit einer
numerischen Simulation in Kapitel~\ref{chapter:particles} zu zeigen,
dass man Teilchen vielleicht auch
als Lösungen einer nichtlinearen Wellengleichung betrachten kann,
in deren Lösungen sich ``Klumpen'' von Energie bilden können.

Eine zentrale Eigenschaft physikalischer Felder ist ihre Lokalität.
\index{Lokalität}%
Als Konsequenz davon wird es möglich, numerische Lösungen auf eine
grosse Zahl von Computer zu parallelisieren.
\index{parallelisieren}%
\emph{Andrin Kälin}
\index{Andrin Kälin}%
\index{Kälin, Andrin}%
und
\emph{Andrin Rütsche}
\index{Andrin Rütsche}%
\index{Rütsche, Andrin}%
zeigen in ihrer Arbeit (Kapitel~\ref{chapter:parallelisierung})
zunächst an einfach nachvollziehbaren Excel-Spreadsheets, wie dies
\index{Excel}%
\index{Spreadsheet}%
funktioniert, finden dann aber auch einschränkende Bedingungen,
die für zuverlässige Lösungen gestellt werden müssen.

Gewöhnliche Differentialgleichungen können als Vektorfelder interpretiert
werden.
Lösungen der Differentialgleichungen sind Feldlinien des Vektorfeldes.
In drei oder mehr Dimensionen können diese Lösungen sehr kompliziert
werden.
In zwei Dimensionen sieht die Situation allerdings ganz anders aus.
Der Satz von Poincaré-Bendixson, den 
\index{Poincaré-Bendixson, Satz von}%
\emph{Raphael Unterer}
\index{Raphael Unterer}%
\index{Unterer, Raphael}%
in Kapitel~\ref{chapter:poinbendix} verständlich macht, besagt, dass
die Bahnen langfristig vollständig durch die kritischen Punkte
\index{kritischer Punkt}%
und die Bahnen zwischen dazwischen bestimmt sind.

Alan Turing
\index{Turing, Alan}%
entdeckte, dass Reaktionsdiffusionsgleichungen unter geeigneten
Bedingungen die Entstehung von Mustern in Ingenieurskonstruktionen,
chemischen Prozessen und sogar in der Biologie erklären können.
\emph{Lukas Schöpf}
\index{Lukas Schöpf}%
\index{Schöpf, Lukas}%
leitet diese Bedingungen her und illustriert sie an einigen interessanten
Beispielen.

Die Strömungsdifferentialgleichungen nach Navier-Stokes sind nur
\index{Navier-Stokes-Gleichungen}%
mit sehr grossem Rechenaufwand in der Lage, gute Lösungen zu liefern.
Die Schwierigkeit steckt in den kleinskaligen und hochfrequenten 
Phänomenen der Turbulenz.
Die Methode des Reynolds-Averaging, die von
\emph{Damian Birchler}
\index{Damian Birchler}%
\index{Birchler, Damian}%
und
\emph{Sebastian Eggli}
\index{Sebastian Eggli}%
\index{Eggli, Sebastian}%
in Kapitel~\ref{chapter:reynolds} erklärt wird, konsolidiert die Effekte
der Turbulenz in einen zusätzlichen Term der ansonsten unveränderten
Navier-Stokes-Gleichungen.
Turbulenz-Modelle können diesen Term abschätzen und damit Lösungen mit
\index{Turbulenz-Modell}%
sehr viel geringerem Rechenaufwand ermöglichen.

Die grosse Wettervariabilität in den gemässigten Breiten hat ihre
\index{Wetter}%
Ursache in wellenförmigen Strukturen, die sich dort in der Atmosphäre
ausbilden.
\emph{Michael Schmid}
\index{Michael Schmid}%
\index{Schmid, Michael}%
beschreibt im Kapitel~\ref{chapter:rossby}, wie diese sogenannten
Rossby-Wellen als Konsequenz der Feldgleichung für die potenzielle
Vortizität verstanden werden können und illustriert sie mit
\index{Vortizität}%
eindrücklichen Datensätzen von extremen Wetterereignissen.

%
% Schallfeld, Robin Eberle
%

Im 20.~Jahrhundert gelang es erstmals, schneller als der Schall zu
fliegen.
Dazu war das Verständnis des Überschallströmungsfeldes nötig, zu dem
\index{Überschallströmung}%
Jakob Ackeret von der ETH wesentliche Beiträge geleistet hat.
\index{Ackeret, Jakob}%
\emph{Emir Arslan}
\index{Emir Arslan}%
\index{Arslan, Emir}%
und
\emph{Shaarujan Kamslanathan}
\index{Shaarujan Kamslanathan}%
\index{Kamslanathan, Shaarujan}%
erklären die Näherung mit Hilfe einer Potentialströmung, mit der
die Berechnung des Strömungswiderstandes möglich wird.
Insbesondere illustrieren ihre Simulationen eindrücklich die
charakteristischen Unterschiede der Strömungsfelder im Unterschall-
und Überschallbereich.

Ringförmige Wirbel sind erstaunlich stabil und können sich über
\index{Wirbelring}%
grosse Distanzen bewegen.
\emph{Nino Briker}
\index{Nino Briker}%
\index{Briker, Nino}%
und
\emph{Fabian Steiner}
\index{Fabian Steiner}%
\index{Steiner, Fabian}%
gehen in Kapitel~\ref{chapter:wirbelringe} auf die Helmholtzschen
\index{Wirbelsatz}%
\index{Helmholtz}%
Wirbelsätze ein, die dieses und andere Wirbelphänomene verständlich
machen.
Sie gehen auch auf die Bedeutung für die Luftfahrt ein.
\index{Luftfahrt}%



