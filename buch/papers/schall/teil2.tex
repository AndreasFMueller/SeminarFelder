%
% teil2.tex -- Beispiel-File für teil2
%
% (c) 2020 Prof Dr Andreas Müller, Hochschule Rapperswil
%
% !TEX root = ../../buch.tex
% !TEX encoding = UTF-8
%
\section{Teilweise Reflexion, Transmission und Absorption an einer Grenzfläche
\label{schall:section:teil2}}
\kopfrechts{Teil 2}

In diesem Abschnitt untersuchen wir eine ebene, zeit\-harmonische Schallwelle,
die aus Medium~1 auf eine planare Grenzfläche zu Medium~2 trifft.
Ein Teil der Welle wird reflektiert, ein Teil wird transmittiert.
Im einfachsten, sprich im verlustlosen Fall, genügen die Druck- und
Geschwindigkeitsfelder in beiden Medien der linearen Akustik;
bei realen Medien in \emph{Medium~2} modellieren wir Absorption durch
einen komplexen Wellenzahl- oder Impedanzterm.
\blindtext\todo[inline]{Grafik: Grenzfläche mit einfallender, reflektierter und transmittierter Welle}

\subsection{Randbedingungen und Definitionen (Normal\-einfall)}
Wir betrachten eine ebene Welle bei Normal\-einfall.
Die Druckanteile lauten
\[
    p_i(x,t)=\Re( P_i\,e^{\mathrm{i}(\omega t-k_1 x)}),\quad
    p_r(x,t)=\Re( P_r\,e^{\mathrm{i}(\omega t+k_1 x)}),\quad
    p_t(x,t)=\Re( P_t\,e^{\mathrm{i}(\omega t-k_2 x)}),
\]
wobei die Indexschreibweise $i,r,t$ für \emph{incident}, \emph{reflected} und \emph{transmitted} steht.
Die charakteristische akustische Impedanz sei für Fluide $Z_n=\rho_n c_n$, $n\in\{1,2\}$.
An der Grenzfläche $x=0$ fordern wir Kontinuität von Druck und Normalgeschwindigkeit:
\[
    p_i+p_r=p_t,\qquad \frac{p_i-p_r}{Z_1}=\frac{p_t}{Z_2}.
\]
Daraus folgen die \emph{amplitudenbezogenen} Reflexions- und Transmissionskoeffizienten
\begin{equation}
    R_p=\frac{P_r}{P_i}=\frac{Z_2-Z_1}{Z_2+Z_1},
    \qquad
    T_p=\frac{P_t}{P_i}=\frac{2Z_2}{Z_2+Z_1}.
    \label{eq:RpTp}
\end{equation}
Für Intensitäten, sprich für den Leistungsfluss, ergibt sich bei reellen Impedanzen \cite{schall:kinsler, schall:allenRT}
\begin{equation}
    R_I=\left|R_p\right|^2=\left(\frac{Z_2-Z_1}{Z_2+Z_1}\right)^2,\qquad
    T_I=\frac{4Z_1Z_2}{(Z_2+Z_1)^2},
    \qquad
    R_I+T_I=1.
    \label{eq:RI_TI}
\end{equation}
\blindtext\todo[inline]{Umfomuleren}

\subsection{Absorption im zweiten Medium}
Reale Medien weisen Dämpfung auf, sprich es kommt zu einer
verlustbehafteten Transmission.
Dies lässt sich mit einem \emph{komplexen Wellenzahl}
\(
    k_2 = \beta_2 - \mathrm{i}\alpha
\)
oder äquivalent mit einer \emph{komplexen Impedanz} modellieren.
Für die \emph{Amplitude} des transmittierten Drucks gilt entlang $x>0$
\[
    |P_t(x)| = |P_t(0)|\,e^{-\alpha x},
\]
woraus für die \emph{Intensität} die Exponentialabnahme
\begin{equation}
    I_t(x)=I_i\,T_I\,e^{-2\alpha x}
    \label{eq:intensity_decay}
\end{equation}
bei ebenen Wellen folgt.
Für viele Medien wird die frequenzabhängige Dämpfung durch ein Gesetz
$\alpha(\omega)=\alpha_0\,\omega^\eta$ beschrieben
(Materialklassen: $\eta\in[0,2]$) \cite{schall:allenAbsorption}.

Damit definiert man die \emph{absorptionsbezogene Energiebilanz} über einer Strecke $L$ in Medium~2:
\begin{equation}
    1 \;=\; R_I \;+\; T_I\,e^{-2\alpha L} \;+\; A(L),
    \qquad
    A(L)=1-R_I - T_I e^{-2\alpha L},
    \label{eq:absorption_balance}
\end{equation}
wobei $A(L)$ der bis zur Tiefe $L$ absorbierte Intensitätsanteil ist.
Im Spezialfall einer \emph{dünnen, rein absorbierenden Schicht},
sprich ein Dämpfungs\-modell ohne Mehrfachreflexionen, geht
$A \approx 1-R_I-T_I$ für $L\to0$ über.
Experimentell wird die \emph{Absorptionskoeffizient}
$\alpha_\mathrm{surf}$ häufig über Impedanz\- bzw.\ Stehwellen\-rohre
bestimmt; für Normal\-einfall gilt (aus dem gemessenen Stehwellenverhältnis SWR)
\begin{equation}
    \alpha_\mathrm{surf} \;=\; 1 - R_I \;=\; 1 - \left(\frac{\mathrm{SWR}-1}{\mathrm{SWR}+1}\right)^2,
    \label{eq:alpha_SWR}
\end{equation}
wobei
\[
    \mathrm{SWR} = \frac{p_\mathrm{max}}{p_\mathrm{min}} = \frac{1+|R_p|}{1-|R_p|}
\]
das Stehwellenverhältnis ist.

\paragraph{Bemerkung zu komplexen Impedanzen.}
Sind Materialverluste in $Z_2$ als $Z_2(\omega)\in\mathbb{C}$ erfasst,
bleiben \eqref{eq:RpTp} formal gültig; für Intensitäten ist dann
\[
    R_I=\left|R_p\right|^2,\qquad
    T_I=\frac{4\,\Re\{Z_1\}\,\Re\{Z_2\}}{|Z_1+Z_2|^2},
\]
wobei $Z_1$ für Fluide typischerweise reell ist.
In vielen Anwendungen ist die Modellierung über $k_2=\beta_2-\mathrm{i}\alpha$
mittels \eqref{eq:intensity_decay} praktischer \cite{schall:kinsler,schall:allenRT}.

\subsection{Schräg\-einfall}
Für Schräg\-einfall gilt die akustische Snell'sche Brechung
\(
    \sin\theta_1/c_1=\sin\theta_2/c_2
\),
und die \emph{normale} Impedanzkomponente skaliert zu
$Z_{n} = Z/\cos\theta$ (Fluid--Fluid).
Entsprechend ändern sich $R_p$, $T_p$ und damit $R_I$, $T_I$;
jenseits des kritischen Winkels tritt ein evaneszentes Feld in
Medium~2 auf, dessen Amplitude exponentiell abnimmt.
Ein evaneszentes Feld ist eine scheinbare, sich nicht ausbreitende
im zweiten Medium, die nicht als laufende Welle in die Tiefe propagiert,
sondern exponentiell zur Grenzfläche hin abklingt \cite{schall:wikiSnell}.
\blindtext\todo[inline]{Herleitung}