\section{Faraday-Tensor}
\label{maxwell:faraday}
Wie im Kapitel \ref{maxwell:motivation} gezeigt wurde, benötigen wir ein Objekt, welches $E$ und $B$ beinhaltet und durch Koordinatentransformation seine Information behält.
Aufgrund dieser Voraussetzungen bietet sich ein Tensor an.
Da das elektromagnetische Feld auf Flächen wirkt, ist es mathematisch sinnvoll, ein Tensor zweiter Stufe zu verwenden.
Ein Tensor zweiter Stufe in vier Dimensionen hat 16 Komponenten. Aufgrund das $E$ und $B$ mit jeweils drei Komponenten vollständig beschrieben werden können, muss der Tensor antisymmetrisch sein, weil nur sechs Komponenten unabhängig sind.
Wir definieren deshalb
\[
F_{\mu\nu}
= 
\begin{pmatrix}
	0 & -E_1/c & -E_2/c & -E_3/c \\
	E_1/c &  0 &  B_3 & -B_2 \\
	E_2/c & -B_3 &  0 &  B_1 \\
	E_3/c &  B_2 & -B_1 &  0 \\
\end{pmatrix}.
\]  
Dabei ist $E$ von Zeit und Raum abhängig und $B$ nur vom Raum, siehe Gleichungen \refeq{maxwell:dA;rotA} und \refeq{maxwell:dA:defE}.
Deshalb ist nur $E$ in den Koordinaten $dx^0$ ersichtlich.
Da $dx^0 = c \, dt$ ist, benötigt man bei den Komponenten von $E$ den Faktor $\frac{1}{c}$.
Wie sich später zeigen wird, ist $F$ eine geschlossene 2-Form, siehe Kapitel................


%Für unsere Berechnungen benötigen wir ein mathematisches Objekt...
%Das Ziel ist es, e und bfeld zu vereinheitlichen
%es soll algemein kovariant sein, deshalb tensor
%4d
%lorentztransformatio erwähnen


%zweiform
%allgemein kovariant
%e und b
