%
% 1koordinaten.tex -- Koordinaten für ein eindimensionales Definitionsgebiet
%
% (c) 2021 Prof Dr Andreas Müller, OST Ostschweizer Fachhochschule
%
\documentclass[tikz]{standalone}
\usepackage{amsmath}
\usepackage{times}
\usepackage{txfonts}
\usepackage{pgfplots}
\usepackage{csvsimple}
\usetikzlibrary{arrows,intersections,math}
\begin{document}
\def\skala{1}
\begin{tikzpicture}[>=latex,thick,scale=\skala,
declare function={
	x(\t) = (\t/70);
	f(\t) = sin(\t)-(\t/140);
}]

\begin{scope}[yshift=4cm]
\draw[] plot[domain=-140:140] ({x(\x)},{f(\x)});
\draw[line width=1.4pt] plot[domain=-80:80] ({x(\x)},{f(\x)});
\fill (0,0) circle[radius=0.08];
\fill ({x(-80)},{f(-80)}) circle[radius=0.08];
\node at ({x(-80)},{f(-80)}) [above] {$A$};
\fill ({x(80)},{f(80)}) circle[radius=0.08];
\node at ({x(80)},{f(80)}) [above] {$B$};
\fill ({x(80)},{f(80)}) circle[radius=0.08];
\node at (0,0) [above] {$P$};
\node at (2,0.1) {$M$};
\draw[->,shorten <= 1cm,shorten >= 1cm] (0,0) -- ++(-3,-4);
\node at (-1.5,-2) [above left] {$\varphi$};
\draw[->,shorten <= 1cm,shorten >= 1cm] (0,0) -- ++(3,-4);
\node at (1.5,-2) [above right] {$\psi$};
\node at (0,1.3) {$\displaystyle I=\int_{AB}\alpha$};
\end{scope}

\begin{scope}[xshift=-3cm]
\draw[->] (-2.3,0) -- (2.5,0) coordinate[label=$x^1\mathstrut$];
\draw[line width=1.4pt] (-1,0) -- (2,0);
\fill (0,0) circle[radius=0.08];
\fill (-1,0) circle[radius=0.08];
\node at (-1,0) [above] {$a\mathstrut$};
\fill (2,0) circle[radius=0.08];
\node at (2,0) [above] {$b\mathstrut$};
\fill (0,0) circle[radius=0.08];
\node at (0,0) [below] {$x^1(P)\mathstrut$};
\node at (-0.5,-1.3) {$\displaystyle I=\int_{a\mathstrut}^{b\mathstrut} f(x^1)\,dx^1$};
\end{scope}

\begin{scope}[xshift=3cm]
\draw[->] (-2.3,0) -- (2.5,0) coordinate[label=$y^1\mathstrut$];
\draw[line width=1.4pt] (-2,0) -- (1,0);
\fill (0,0) circle[radius=0.08];
\fill (-2,0) circle[radius=0.08];
\node at (-2,0) [above] {$a'\mathstrut$};
\fill (1,0) circle[radius=0.08];
\node at (1,0) [above] {$b'\mathstrut$};
\node at (0,0) [below] {$y^1(P)\mathstrut$};
\node at (0.8,-1.3) {$\displaystyle I=\int_{a'\mathstrut}^{b'\mathstrut} g(y^1)\,dy^1$};
\end{scope}

\xdef\r{5}
\pgfmathparse{asin(2/\r)}
\xdef\w{\pgfmathresult}
\pgfmathparse{0.6+\r*(1-cos(\w))}
\xdef\y{\pgfmathresult}

\draw[->] (-2,0.6) arc({90+\w}:{90-\w}:5);
\node at (0,\y) [above] {$\psi\circ\varphi^{-1}$};
\draw[->] (2,-0.3) arc({\w-90}:{-90-\w}:5);
\node at (0,{-\y+0.3}) [below] {$\varphi\circ\psi^{-1}$};

\end{tikzpicture}
\end{document}

