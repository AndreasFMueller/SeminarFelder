%
% chapter.tex -- Kapitel 3: 1-Formen und Kurvenintegrale
%
% (c) 2024 Prof Dr Andreas Müller
%
\chapter{Differentialformen und Kurvenintegrale
\label{chapter:kurvenintegral}}
\kopflinks{Differentialformen und Kurvenintegrale}
Ein Flug zum Mars muss Energie aufwenden, um gegen die Schwerkraft der
Sonne von der Erdbahn aus zu der ungefährt 1.5-mal so grossen Marsbahn
Entfernung von der Sonne zu gewinnen.
In der Praxis wird dies dadurch erreicht, dass man dem Raumfahrzeug
mithilfe von Raketentriebwerken genügend kinetische Energie erteilt.
Es folgt dann einer elliptischen Bahn, die ihren sonnenfernsten Punkt
bei der Marsbahn hat.
Während des Fluges wird laufend kinetische Energie des Raumschiffs in
potenzielle Energie umgewandelt werden.
Mit jedem kleinen Teilstück der Flugbahn verliert das Raumschiff 
kinetische Energie, die proportional ist zur Kraftkomponente parallel
zur Flugbahn. 
Das potentielle Energiepaket das während des Teilstücks gewonnen wird,
ist linear vom Tangentialvektor der Bahn abhängig.
Die gesamte potentielle Energie, die zwischen Erde und Marsbahn
gewonnen wird, ist die Summe dieser Teilstücke.
Mathematisch wird es durch eine Art Integral einer Funktion dargestellt,
welches sowohl linear von der Bahntangente wie auch vom Pfad des
Raumschiffs abhängt.
Diese Integralkonstruktion muss aber so erfolgen, dass Sie nicht von der
Wahl von Koordinatensystemen und Parametrisierungen abhängt, was in
diesem Kapitel durchgeführt werden soll.
Sie muss ausserdem so funktionieren, dass sie sich über mehrer
Koordinatensysteme hinweg zusammensetzen lässt, wie dies bei einer
Bahn auf einer Mannifaltigkeit unvermeidlich wird.

%
% 1-Formen
%
\section{1-Formen}

%
% Kurvenintegral einer 1-Form
%
\section{Kurvenintegral einer 1-Form}

%
% Differential einer Funktion
%
\section{Differential einer Funktion}

%
% Differenzierbare Zerlegungen der Einheit
%
\section{Differenzierbare Zerlegung der Einheit}
Ein Wegintegral kann sich entlang eines Pfades erstrecken, der
mehrere Kartengebiete einer Mannigfaltigkeit durchauert.
Das Kurvenintegral ist mithilfe eines Koordinatensystems definiert,
kann also immer nur innerhalb eines Kartengebietes berechnet werden.
Es muss also ein Technik gefunden werden, mit der Teilintegrale in
einzelnen Kartengebieten zu einem Integral über die ganze Kurve
zusammengesetzt werden.
Die Konstruktion muss von der Wahl der Koordinatensysteme enthlange
des Pfades unabhängig sein.
Dies wird erreicht mit einer differenzierbaren Zerlegung der Einheit
und dank der Tatsache, dass das Kurvenintegral linear in der 1-Form
ist.

\subsection{Glatte Funktionen mit Träger in einem Interval}

\subsection{Überdeckung mit offenen Intervallen}

\subsection{Differenzierbare Zerlegung der Einheit}

\subsection{Zerlegung eines Kurvenintegrals}


